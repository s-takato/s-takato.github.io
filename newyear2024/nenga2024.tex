\documentclass[10pt,dvipdfmx]{ujarticle}
\special{papersize=148mm,100mm}
\usepackage{hyperref} 
\usepackage{graphicx,color}
\usepackage{amsmath,amssymb}
\usepackage{qrcode}
\usepackage{pict2e}
\usepackage{ketpic2e}
\usepackage{ketlayer2e}
\usepackage{lmodern}

\newcommand*{\myfont}[4]{{%
  #1\fontseries{#2}\fontshape{#3}\selectfont
  #4
}}
%\definecolor{slidecolora}{cmyk}{0.98,0.13,0,0.43}

\setmargin{10}{10}{10}{10}

\pagestyle{empty}

\begin{document}

\begin{layer}{150}{0}
\putnotese{0}{-5}{%
\includegraphics[bb=0.00 0.00 457.00 490.00,width=70mm]{fig/kan-en.pdf}
}ß
\putnotese{68}{5}{%
\begin{minipage}{55mm}
 25年ほど前に数式処理システムMapleで和算解法パッケージを作っていました.最近,久しぶりに動かしたところ,ちゃんと動いたので少し嬉しくなりました.現在,これをフリーのMaximaに移植して
\ketcindy と連携する作業を進めています.図はそれを使って作成した環円問題のHTMLで,
QRコードのWebサイトに行くと,実際に回転させることができます.
そろそろ自分のしてきたことをまとめたいと
思っています.\\
 皆様のご健康をお祈りいたします.
\end{minipage}
}


\putnotese{0}{63}{\scalebox{0.85}{\qrcode{https://s-takato.github.io/offline/2024newyearJ.html}}}



\putnotee{50}{77}{〒\!292-0041 木更津市清見台東3-29-25 髙遠節夫}

\end{layer}

%%

%\vspace{2mm}
%
%\noindent
%\begin{minipage}{76mm}
%%\hfill〒\!292-0041 木更津市清見台東3-29-25 \\
%\hfill 髙遠節夫\vspace{-1mm}\\
%\end{minipage}



\end{document}

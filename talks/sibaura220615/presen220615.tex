%%% Title presen220615
\documentclass[landscape,10pt]{jarticle}
\special{papersize=\the\paperwidth,\the\paperheight}
\usepackage{ketpic,ketlayer}
\usepackage{ketslide}
\usepackage{amsmath,amssymb}
\usepackage{bm,enumerate}
\usepackage[dvipdfmx]{graphicx}
\usepackage{color}
\definecolor{slidecolora}{cmyk}{0.98,0.13,0,0.43}
\definecolor{slidecolorb}{cmyk}{0.2,0,0,0}
\definecolor{slidecolorc}{cmyk}{0.2,0,0,0}
\definecolor{slidecolord}{cmyk}{0.2,0,0,0}
\definecolor{slidecolore}{cmyk}{0,0,0,0.5}
\definecolor{slidecolorf}{cmyk}{0,0,0,0.5}
\definecolor{slidecolori}{cmyk}{0.98,0.13,0,0.43}
\def\setthin#1{\def\thin{#1}}
\setthin{0}
\newcommand{\slidepage}[1][s]{%
\setcounter{ketpicctra}{18}%
\if#1m \setcounter{ketpicctra}{1}\fi
\hypersetup{linkcolor=black}%

\begin{layer}{118}{0}
\putnotee{122}{-\theketpicctra.05}{\small\thepage/\pageref{pageend}}
\end{layer}\hypersetup{linkcolor=blue}

}
\usepackage{pict2e}
\usepackage{ketlayermorewith2e}
\usepackage[dvipdfmx,colorlinks=true,linkcolor=blue,filecolor=blue]{hyperref}

\setmargin{25}{145}{15}{100}

\ketslideinit

\pagestyle{empty}

\begin{document}

\begin{layer}{120}{0}
\putnotese{0}{0}{{\Large\bf
\color[cmyk]{1,1,0,0}

\begin{layer}{120}{0}
{\Huge \putnotes{60}{20}{三角関数の性質}}
\putnotes{60}{70}{2022.05.16}
\end{layer}

}
}
\end{layer}

\def\mainslidetitley{22}
\def\ketcletter{slidecolora}
\def\ketcbox{slidecolorb}
\def\ketdbox{slidecolorc}
\def\ketcframe{slidecolord}
\def\ketcshadow{slidecolore}
\def\ketdshadow{slidecolorf}
\def\slidetitlex{6}
\def\slidetitlesize{1.3}
\def\mketcletter{slidecolori}
\def\mketcbox{yellow}
\def\mketdbox{yellow}
\def\mketcframe{yellow}
\def\mslidetitlex{62}
\def\mslidetitlesize{2}

\color{black}
\Large\bf\boldmath
\addtocounter{page}{-1}

%%%%%%%%%%%%

%%%%%%%%%%%%%%%%%%%%

\mainslide{ KeTCindyを使うまで}


\slidepage[m]
%%%%%%%%%%%%%

%%%%%%%%%%%%%%%%%%%%

\newslide{インストール}

\vspace*{18mm}

\slidepage
\begin{itemize}
\item
\href{https://s-takato.github.io/ketcindyorg/indexj.html}{ketcindy home}で検索\vspace{-2mm}
\item
以下をインストール\vspace{-2mm}
\begin{enumerate}[(1)]
\item
\Ltab{30mm}{Cinderella}DGSの1つ \href{https://www.youtube.com/watch?v=eAdhjGW438I}{YouTube}\vspace{-2mm}
\item
\Ltab{30mm}{KeTTeX}TeXLiveのコンパクト版\href{https://www.youtube.com/watch?v=_6zPk-3sAP8}{YouTube}\vspace{-2mm}
\item
\Ltab{30mm}{R}\href{https://cran.r-project.org}{統計ソフト(必須)}\vspace{-2mm}
\item
\Ltab{30mm}{Sumatra}\href{https://cran.r-project.org}{PDFビューア(Windowsのみ)}\vspace{-2mm}
\item
\Ltab{30mm}{KeTCindy}\href{https://www.youtube.com/watch?v=hVRXyeH31SM}{YouTube}\vspace{-2mm}
\item
\Ltab{30mm}{TeXWorks}\href{https://www.youtube.com/watch?v=H71sd1FvpIQ}{YouTube}\vspace{-2mm}
\end{enumerate}
\end{itemize}
%%%%%%%%%%%%%

%%%%%%%%%%%%%%%%%%%%

\newslide{\ketcindy による\TeX 図の作成}

\vspace*{18mm}

\slidepage
\begin{itemize}
\item
ユーザホーム/ketcindy+日付 のcdyファイル利用\\
 ・ketcindyreference.pdf, samples, template\vspace{-2mm}
\item
ファイルを選び,作業フォルダにコピー,名称変更\vspace{-2mm}
\item
CindyScriptのDraw/Figuresを開く\vspace{-2mm}
\item
KetinitとWindispgの間にコマンド,ギアマーク\vspace{-2mm}
\item
Cindy画面で「要素を動かす」を確認\vspace{-2mm}
\item
Figureボタンを押すと,pdfが表示される\\
 ・Cinderellaを終了するときは「ファイル保存」
\end{itemize}
%%%%%%%%%%%%%

%%%%%%%%%%%%%%%%%%%%

\newslide{figフォルダのファイル群}

\vspace*{18mm}

\slidepage
\begin{itemize}
\item
Cindyファイルと同じ階層にfigフォルダができる\vspace{-2mm}
\item
ファイルの説明\vspace{-1mm}\\
{\large
\Ltab{50mm}{ex1.r}描画コマンド生成(R)\vspace{-0.5mm}\\
\Ltab{50mm}{ex1.tex}\TeX 描画コマンドのファイル)\vspace{-0.5mm}\\
\Ltab{50mm}{ex1main.aux}\TeX 参照用 \vspace{-0.5mm}\\
\Ltab{50mm}{ex1main.log}\TeX 作業記録 \vspace{-0.5mm}\\
\Ltab{50mm}{ex1main.pdf}出来上がりのPDF \vspace{-0.5mm}\\
\Ltab{50mm}{ex1main.tex}\TeX 親ファイル(確認用))\vspace{-0.5mm}\\
\Ltab{50mm}{kc.command(bat)}バッチファイル
}
\end{itemize}
%%%%%%%%%%%%

%%%%%%%%%%%%%%%%%%%%

\mainslide{ \TeX 文書の作成}


\slidepage[m]
%%%%%%%%%%%%%

%%%%%%%%%%%%%%%%%%%%

\newslide{\TeX エディタの利用}

\vspace*{18mm}

\slidepage
\begin{itemize}
\item
\Ltab{50mm}{TeXWorksの設定}\href{https://www.youtube.com/watch?v=H71sd1FvpIQ}{YouTube}\vspace{-2mm}
\item
 Macの場合は,TeXShopも使いやすい\vspace{-2mm}
\item
ex1main.texをダブルクリックで開く\vspace{-2mm}
{\normalsize
\begin{verbatim}
\documentclass{jarticle}
\usepackage[dvipdfmx]{pict2e}
\usepackage{ketpic2e}
\usepackage{ketlayer2e}
\usepackage{ketlayermorewith2e}
\usepackage{amsmath,amssymb}
\usepackage{graphicx}
\usepackage{color}
\setmargin{20}{20}{20}{20}
\end{verbatim}
}
\end{itemize}
%%%%%%%%%%%%%

%%%%%%%%%%%%%%%%%%%%

\newslide{\TeX エディタの利用(続)}

\vspace*{18mm}

\slidepage
\begin{itemize}
\item
[]
{\normalsize
\begin{verbatim}
\begin{document}
\verb|ex1| by \ketcindy
\vspace{5mm}
%%% /githubio.git/talks/sibaura220615/fig/ex1.tex 
%%% Generator=ex1.cdy 
{\unitlength=1cm%
\begin{picture}%
(9.94,9.69)(-4.94,-4.69)%
\linethickness{0.008in}%%
\polyline(-4.00000,1.00000)(-3.85995,0.97042)(-3.72279,0.94193)(-3.58854,0.91452)%
(-3.45719,0.88821)(-3.32873,0.86299)(-3.20318,0.83885)(-3.08053,0.81581)(-2.96078,0.79385)%
(-2.84393,0.77298)(-2.72998,0.75321)(-2.61893,0.73452)(-2.51079,0.71692)(-2.40554,0.70041)%
(-2.30319,0.68499)(-2.20374,0.67066)(-2.10720,0.65742)(-2.01355,0.64527)(-1.92281,0.63421)%
(-1.83496,0.62424)(-1.75002,0.61536)(-1.66798,0.60756)(-1.58884,0.60086)(-1.51259,0.59524)%
(-1.43925,0.59072)(-1.36881,0.58728)(-1.30127,0.58494)(-1.23663,0.58368)(-1.17489,0.58351)%
(-1.11605,0.58444)(-1.06012,0.58645)(-0.97871,0.59118)(-0.89794,0.59766)(-0.81782,0.60583)%
(-0.73832,0.61563)(-0.65945,0.62699)(-0.58119,0.63986)(-0.50355,0.65418)(-0.42652,0.66987)%
(-0.35008,0.68689)(-0.27424,0.70516)(-0.19898,0.72464)(-0.12430,0.74525)(-0.05020,0.76694)%
(0.02333,0.78964)(0.09631,0.81329)(0.16872,0.83783)(0.24059,0.86320)(0.31192,0.88934)%
(0.38271,0.91619)(0.45297,0.94368)(0.52270,0.97175)(0.59192,1.00035)(0.66063,1.02941)%
(0.72883,1.05887)(0.79653,1.08866)(0.86374,1.11874)(0.93046,1.14902)(0.99671,1.17946)%
(1.06248,1.21000)(1.12778,1.24056)(1.19864,1.27410)(1.26919,1.30806)(1.33940,1.34245)%
(1.40925,1.37728)(1.47872,1.41258)(1.54778,1.44838)(1.61641,1.48467)(1.68459,1.52150)%
(1.75230,1.55887)(1.81950,1.59680)(1.88618,1.63532)(1.95232,1.67444)(2.01788,1.71419)%
(2.08285,1.75457)(2.14720,1.79562)(2.21092,1.83735)(2.27396,1.87977)(2.33632,1.92292)%
(2.39797,1.96680)(2.45889,2.01144)(2.51904,2.05685)(2.57842,2.10306)(2.63699,2.15009)%
(2.69473,2.19794)(2.75162,2.24666)(2.80763,2.29624)(2.86274,2.34671)(2.91694,2.39810)%
(2.97018,2.45041)(3.02246,2.50368)(3.05703,2.54034)(3.09171,2.57868)(3.12649,2.61869)%
(3.16138,2.66037)(3.19637,2.70372)(3.23147,2.74874)(3.26667,2.79543)(3.30198,2.84380)%
(3.33739,2.89384)(3.37290,2.94554)(3.40852,2.99892)(3.44424,3.05397)(3.48007,3.11069)%
(3.51600,3.16909)(3.55204,3.22915)(3.58818,3.29089)(3.62443,3.35430)(3.66078,3.41937)%
(3.69723,3.48612)(3.73379,3.55455)(3.77046,3.62464)(3.80722,3.69640)(3.84410,3.76984)%
(3.88107,3.84494)(3.91815,3.92172)(3.95534,4.00017)(3.99263,4.08029)(4.03002,4.16209)%
(4.06752,4.24555)(4.10513,4.33068)%
%
{%
\color[cmyk]{0,1,1,0}%
\polyline(-4.94000,-2.17559)(-4.74120,-2.06462)(-4.54240,-1.95365)(-4.34360,-1.84268)%
(-4.14480,-1.73171)(-3.94600,-1.62074)(-3.74720,-1.50978)(-3.54840,-1.39881)(-3.34960,-1.28784)%
(-3.15080,-1.17687)(-2.95200,-1.06590)(-2.75320,-0.95493)(-2.55440,-0.84396)(-2.35560,-0.73300)%
(-2.15680,-0.62203)(-1.95800,-0.51106)(-1.75920,-0.40009)(-1.56040,-0.28912)(-1.36160,-0.17815)%
(-1.16280,-0.06718)(-0.96400,0.04378)(-0.76520,0.15475)(-0.56640,0.26572)(-0.36760,0.37669)%
(-0.16880,0.48766)(0.03000,0.59863)(0.22880,0.70960)(0.42760,0.82056)(0.62640,0.93153)%
(0.82520,1.04250)(1.02400,1.15347)(1.22280,1.26444)(1.42160,1.37541)(1.62040,1.48638)%
(1.81920,1.59734)(2.01800,1.70831)(2.21680,1.81928)(2.41560,1.93025)(2.61440,2.04122)%
(2.81320,2.15219)(3.01200,2.26316)(3.21080,2.37412)(3.40960,2.48509)(3.60840,2.59606)%
(3.80720,2.70703)(4.00600,2.81800)(4.20480,2.92897)(4.40360,3.03994)(4.60240,3.15090)%
(4.80120,3.26187)(5.00000,3.37284)%
%
}%
\polyline(-4.93969,0.00000)(5.00000,0.00000)%
%
\polyline(0.00000,-4.69000)(0.00000,5.00000)%
%
\settowidth{\Width}{$x$}\setlength{\Width}{0\Width}%
\settoheight{\Height}{$x$}\settodepth{\Depth}{$x$}\setlength{\Height}{-0.5\Height}\setlength{\Depth}{0.5\Depth}\addtolength{\Height}{\Depth}%
\put(5.0500000,0.0000000){\hspace*{\Width}\raisebox{\Height}{$x$}}%
%
\settowidth{\Width}{$y$}\setlength{\Width}{-0.5\Width}%
\settoheight{\Height}{$y$}\settodepth{\Depth}{$y$}\setlength{\Height}{\Depth}%
\put(0.0000000,5.0500000){\hspace*{\Width}\raisebox{\Height}{$y$}}%
%
\settowidth{\Width}{O}\setlength{\Width}{-1\Width}%
\settoheight{\Height}{O}\settodepth{\Depth}{O}\setlength{\Height}{-\Height}%
\put(-0.0500000,-0.0500000){\hspace*{\Width}\raisebox{\Height}{O}}%
%
\end{picture}}%
\end{document}
\end{verbatim}
}
\item
\verb|\begin{document}|と\verb|\end{document}|の間を編集
\end{itemize}
%%%%%%%%%%%%%

%%%%%%%%%%%%%%%%%%%%

\newslide{layer環境の利用(流れ)}

\vspace*{18mm}

\slidepage
\begin{layer}{120}{70}
\end{layer}
\begin{itemize}
\item
グリッドを描く\\
 \verb|\begin{layer}{120}{70}|\\
 ...\\
 \verb|\end{layer}|
\item
図を配置\\
 \verb|\begin{layer}{120}{70}|\\
 \verb|\putnotese{60}{15}{%%% /githubio.git/talks/sibaura220615/fig/ex1.tex 
%%% Generator=ex1.cdy 
{\unitlength=1cm%
\begin{picture}%
(9.94,9.69)(-4.94,-4.69)%
\linethickness{0.008in}%%
\polyline(-4.00000,1.00000)(-3.85995,0.97042)(-3.72279,0.94193)(-3.58854,0.91452)%
(-3.45719,0.88821)(-3.32873,0.86299)(-3.20318,0.83885)(-3.08053,0.81581)(-2.96078,0.79385)%
(-2.84393,0.77298)(-2.72998,0.75321)(-2.61893,0.73452)(-2.51079,0.71692)(-2.40554,0.70041)%
(-2.30319,0.68499)(-2.20374,0.67066)(-2.10720,0.65742)(-2.01355,0.64527)(-1.92281,0.63421)%
(-1.83496,0.62424)(-1.75002,0.61536)(-1.66798,0.60756)(-1.58884,0.60086)(-1.51259,0.59524)%
(-1.43925,0.59072)(-1.36881,0.58728)(-1.30127,0.58494)(-1.23663,0.58368)(-1.17489,0.58351)%
(-1.11605,0.58444)(-1.06012,0.58645)(-0.97871,0.59118)(-0.89794,0.59766)(-0.81782,0.60583)%
(-0.73832,0.61563)(-0.65945,0.62699)(-0.58119,0.63986)(-0.50355,0.65418)(-0.42652,0.66987)%
(-0.35008,0.68689)(-0.27424,0.70516)(-0.19898,0.72464)(-0.12430,0.74525)(-0.05020,0.76694)%
(0.02333,0.78964)(0.09631,0.81329)(0.16872,0.83783)(0.24059,0.86320)(0.31192,0.88934)%
(0.38271,0.91619)(0.45297,0.94368)(0.52270,0.97175)(0.59192,1.00035)(0.66063,1.02941)%
(0.72883,1.05887)(0.79653,1.08866)(0.86374,1.11874)(0.93046,1.14902)(0.99671,1.17946)%
(1.06248,1.21000)(1.12778,1.24056)(1.19864,1.27410)(1.26919,1.30806)(1.33940,1.34245)%
(1.40925,1.37728)(1.47872,1.41258)(1.54778,1.44838)(1.61641,1.48467)(1.68459,1.52150)%
(1.75230,1.55887)(1.81950,1.59680)(1.88618,1.63532)(1.95232,1.67444)(2.01788,1.71419)%
(2.08285,1.75457)(2.14720,1.79562)(2.21092,1.83735)(2.27396,1.87977)(2.33632,1.92292)%
(2.39797,1.96680)(2.45889,2.01144)(2.51904,2.05685)(2.57842,2.10306)(2.63699,2.15009)%
(2.69473,2.19794)(2.75162,2.24666)(2.80763,2.29624)(2.86274,2.34671)(2.91694,2.39810)%
(2.97018,2.45041)(3.02246,2.50368)(3.05703,2.54034)(3.09171,2.57868)(3.12649,2.61869)%
(3.16138,2.66037)(3.19637,2.70372)(3.23147,2.74874)(3.26667,2.79543)(3.30198,2.84380)%
(3.33739,2.89384)(3.37290,2.94554)(3.40852,2.99892)(3.44424,3.05397)(3.48007,3.11069)%
(3.51600,3.16909)(3.55204,3.22915)(3.58818,3.29089)(3.62443,3.35430)(3.66078,3.41937)%
(3.69723,3.48612)(3.73379,3.55455)(3.77046,3.62464)(3.80722,3.69640)(3.84410,3.76984)%
(3.88107,3.84494)(3.91815,3.92172)(3.95534,4.00017)(3.99263,4.08029)(4.03002,4.16209)%
(4.06752,4.24555)(4.10513,4.33068)%
%
{%
\color[cmyk]{0,1,1,0}%
\polyline(-4.94000,-2.17559)(-4.74120,-2.06462)(-4.54240,-1.95365)(-4.34360,-1.84268)%
(-4.14480,-1.73171)(-3.94600,-1.62074)(-3.74720,-1.50978)(-3.54840,-1.39881)(-3.34960,-1.28784)%
(-3.15080,-1.17687)(-2.95200,-1.06590)(-2.75320,-0.95493)(-2.55440,-0.84396)(-2.35560,-0.73300)%
(-2.15680,-0.62203)(-1.95800,-0.51106)(-1.75920,-0.40009)(-1.56040,-0.28912)(-1.36160,-0.17815)%
(-1.16280,-0.06718)(-0.96400,0.04378)(-0.76520,0.15475)(-0.56640,0.26572)(-0.36760,0.37669)%
(-0.16880,0.48766)(0.03000,0.59863)(0.22880,0.70960)(0.42760,0.82056)(0.62640,0.93153)%
(0.82520,1.04250)(1.02400,1.15347)(1.22280,1.26444)(1.42160,1.37541)(1.62040,1.48638)%
(1.81920,1.59734)(2.01800,1.70831)(2.21680,1.81928)(2.41560,1.93025)(2.61440,2.04122)%
(2.81320,2.15219)(3.01200,2.26316)(3.21080,2.37412)(3.40960,2.48509)(3.60840,2.59606)%
(3.80720,2.70703)(4.00600,2.81800)(4.20480,2.92897)(4.40360,3.03994)(4.60240,3.15090)%
(4.80120,3.26187)(5.00000,3.37284)%
%
}%
\polyline(-4.93969,0.00000)(5.00000,0.00000)%
%
\polyline(0.00000,-4.69000)(0.00000,5.00000)%
%
\settowidth{\Width}{$x$}\setlength{\Width}{0\Width}%
\settoheight{\Height}{$x$}\settodepth{\Depth}{$x$}\setlength{\Height}{-0.5\Height}\setlength{\Depth}{0.5\Depth}\addtolength{\Height}{\Depth}%
\put(5.0500000,0.0000000){\hspace*{\Width}\raisebox{\Height}{$x$}}%
%
\settowidth{\Width}{$y$}\setlength{\Width}{-0.5\Width}%
\settoheight{\Height}{$y$}\settodepth{\Depth}{$y$}\setlength{\Height}{\Depth}%
\put(0.0000000,5.0500000){\hspace*{\Width}\raisebox{\Height}{$y$}}%
%
\settowidth{\Width}{O}\setlength{\Width}{-1\Width}%
\settoheight{\Height}{O}\settodepth{\Depth}{O}\setlength{\Height}{-\Height}%
\put(-0.0500000,-0.0500000){\hspace*{\Width}\raisebox{\Height}{O}}%
%
\end{picture}}%}|\\
 \verb|\end{layer}{140}{50}|
\end{itemize}
%%%%%%%%%%%%%

%%%%%%%%%%%%%%%%%%%%

\newslide{layer環境の利用(実際)}

\vspace*{18mm}

\slidepage
\begin{layer}{120}{70}
\putnotese{60}{15}{\scalebox{0.5}{%%% /githubio.git/talks/sibaura220615/fig/ex1.tex 
%%% Generator=ex1.cdy 
{\unitlength=1cm%
\begin{picture}%
(9.94,9.69)(-4.94,-4.69)%
\linethickness{0.008in}%%
\polyline(-4.00000,1.00000)(-3.85995,0.97042)(-3.72279,0.94193)(-3.58854,0.91452)%
(-3.45719,0.88821)(-3.32873,0.86299)(-3.20318,0.83885)(-3.08053,0.81581)(-2.96078,0.79385)%
(-2.84393,0.77298)(-2.72998,0.75321)(-2.61893,0.73452)(-2.51079,0.71692)(-2.40554,0.70041)%
(-2.30319,0.68499)(-2.20374,0.67066)(-2.10720,0.65742)(-2.01355,0.64527)(-1.92281,0.63421)%
(-1.83496,0.62424)(-1.75002,0.61536)(-1.66798,0.60756)(-1.58884,0.60086)(-1.51259,0.59524)%
(-1.43925,0.59072)(-1.36881,0.58728)(-1.30127,0.58494)(-1.23663,0.58368)(-1.17489,0.58351)%
(-1.11605,0.58444)(-1.06012,0.58645)(-0.97871,0.59118)(-0.89794,0.59766)(-0.81782,0.60583)%
(-0.73832,0.61563)(-0.65945,0.62699)(-0.58119,0.63986)(-0.50355,0.65418)(-0.42652,0.66987)%
(-0.35008,0.68689)(-0.27424,0.70516)(-0.19898,0.72464)(-0.12430,0.74525)(-0.05020,0.76694)%
(0.02333,0.78964)(0.09631,0.81329)(0.16872,0.83783)(0.24059,0.86320)(0.31192,0.88934)%
(0.38271,0.91619)(0.45297,0.94368)(0.52270,0.97175)(0.59192,1.00035)(0.66063,1.02941)%
(0.72883,1.05887)(0.79653,1.08866)(0.86374,1.11874)(0.93046,1.14902)(0.99671,1.17946)%
(1.06248,1.21000)(1.12778,1.24056)(1.19864,1.27410)(1.26919,1.30806)(1.33940,1.34245)%
(1.40925,1.37728)(1.47872,1.41258)(1.54778,1.44838)(1.61641,1.48467)(1.68459,1.52150)%
(1.75230,1.55887)(1.81950,1.59680)(1.88618,1.63532)(1.95232,1.67444)(2.01788,1.71419)%
(2.08285,1.75457)(2.14720,1.79562)(2.21092,1.83735)(2.27396,1.87977)(2.33632,1.92292)%
(2.39797,1.96680)(2.45889,2.01144)(2.51904,2.05685)(2.57842,2.10306)(2.63699,2.15009)%
(2.69473,2.19794)(2.75162,2.24666)(2.80763,2.29624)(2.86274,2.34671)(2.91694,2.39810)%
(2.97018,2.45041)(3.02246,2.50368)(3.05703,2.54034)(3.09171,2.57868)(3.12649,2.61869)%
(3.16138,2.66037)(3.19637,2.70372)(3.23147,2.74874)(3.26667,2.79543)(3.30198,2.84380)%
(3.33739,2.89384)(3.37290,2.94554)(3.40852,2.99892)(3.44424,3.05397)(3.48007,3.11069)%
(3.51600,3.16909)(3.55204,3.22915)(3.58818,3.29089)(3.62443,3.35430)(3.66078,3.41937)%
(3.69723,3.48612)(3.73379,3.55455)(3.77046,3.62464)(3.80722,3.69640)(3.84410,3.76984)%
(3.88107,3.84494)(3.91815,3.92172)(3.95534,4.00017)(3.99263,4.08029)(4.03002,4.16209)%
(4.06752,4.24555)(4.10513,4.33068)%
%
{%
\color[cmyk]{0,1,1,0}%
\polyline(-4.94000,-2.17559)(-4.74120,-2.06462)(-4.54240,-1.95365)(-4.34360,-1.84268)%
(-4.14480,-1.73171)(-3.94600,-1.62074)(-3.74720,-1.50978)(-3.54840,-1.39881)(-3.34960,-1.28784)%
(-3.15080,-1.17687)(-2.95200,-1.06590)(-2.75320,-0.95493)(-2.55440,-0.84396)(-2.35560,-0.73300)%
(-2.15680,-0.62203)(-1.95800,-0.51106)(-1.75920,-0.40009)(-1.56040,-0.28912)(-1.36160,-0.17815)%
(-1.16280,-0.06718)(-0.96400,0.04378)(-0.76520,0.15475)(-0.56640,0.26572)(-0.36760,0.37669)%
(-0.16880,0.48766)(0.03000,0.59863)(0.22880,0.70960)(0.42760,0.82056)(0.62640,0.93153)%
(0.82520,1.04250)(1.02400,1.15347)(1.22280,1.26444)(1.42160,1.37541)(1.62040,1.48638)%
(1.81920,1.59734)(2.01800,1.70831)(2.21680,1.81928)(2.41560,1.93025)(2.61440,2.04122)%
(2.81320,2.15219)(3.01200,2.26316)(3.21080,2.37412)(3.40960,2.48509)(3.60840,2.59606)%
(3.80720,2.70703)(4.00600,2.81800)(4.20480,2.92897)(4.40360,3.03994)(4.60240,3.15090)%
(4.80120,3.26187)(5.00000,3.37284)%
%
}%
\polyline(-4.93969,0.00000)(5.00000,0.00000)%
%
\polyline(0.00000,-4.69000)(0.00000,5.00000)%
%
\settowidth{\Width}{$x$}\setlength{\Width}{0\Width}%
\settoheight{\Height}{$x$}\settodepth{\Depth}{$x$}\setlength{\Height}{-0.5\Height}\setlength{\Depth}{0.5\Depth}\addtolength{\Height}{\Depth}%
\put(5.0500000,0.0000000){\hspace*{\Width}\raisebox{\Height}{$x$}}%
%
\settowidth{\Width}{$y$}\setlength{\Width}{-0.5\Width}%
\settoheight{\Height}{$y$}\settodepth{\Depth}{$y$}\setlength{\Height}{\Depth}%
\put(0.0000000,5.0500000){\hspace*{\Width}\raisebox{\Height}{$y$}}%
%
\settowidth{\Width}{O}\setlength{\Width}{-1\Width}%
\settoheight{\Height}{O}\settodepth{\Depth}{O}\setlength{\Height}{-\Height}%
\put(-0.0500000,-0.0500000){\hspace*{\Width}\raisebox{\Height}{O}}%
%
\end{picture}}%}}
\end{layer}
\begin{itemize}
\item
図が(60,15)の位置から右下に配置される\\
 ・方向は 8方向(nsew)+中心(c)
\item
適当な位置に移動する
\item
よければ\\
 ・第2引数を0とする\\
 ・グリッドが消える\\
 ・文字位置は変わらない
\end{itemize}
%%%%%%%%%%%%%

%%%%%%%%%%%%%%%%%%%%

\newslide{layer環境の利用(出来上がり)}

\vspace*{18mm}

\slidepage
\begin{layer}{120}{0}
\putnotese{60}{15}{\scalebox{0.5}{%%% /githubio.git/talks/sibaura220615/fig/ex1.tex 
%%% Generator=ex1.cdy 
{\unitlength=1cm%
\begin{picture}%
(9.94,9.69)(-4.94,-4.69)%
\linethickness{0.008in}%%
\polyline(-4.00000,1.00000)(-3.85995,0.97042)(-3.72279,0.94193)(-3.58854,0.91452)%
(-3.45719,0.88821)(-3.32873,0.86299)(-3.20318,0.83885)(-3.08053,0.81581)(-2.96078,0.79385)%
(-2.84393,0.77298)(-2.72998,0.75321)(-2.61893,0.73452)(-2.51079,0.71692)(-2.40554,0.70041)%
(-2.30319,0.68499)(-2.20374,0.67066)(-2.10720,0.65742)(-2.01355,0.64527)(-1.92281,0.63421)%
(-1.83496,0.62424)(-1.75002,0.61536)(-1.66798,0.60756)(-1.58884,0.60086)(-1.51259,0.59524)%
(-1.43925,0.59072)(-1.36881,0.58728)(-1.30127,0.58494)(-1.23663,0.58368)(-1.17489,0.58351)%
(-1.11605,0.58444)(-1.06012,0.58645)(-0.97871,0.59118)(-0.89794,0.59766)(-0.81782,0.60583)%
(-0.73832,0.61563)(-0.65945,0.62699)(-0.58119,0.63986)(-0.50355,0.65418)(-0.42652,0.66987)%
(-0.35008,0.68689)(-0.27424,0.70516)(-0.19898,0.72464)(-0.12430,0.74525)(-0.05020,0.76694)%
(0.02333,0.78964)(0.09631,0.81329)(0.16872,0.83783)(0.24059,0.86320)(0.31192,0.88934)%
(0.38271,0.91619)(0.45297,0.94368)(0.52270,0.97175)(0.59192,1.00035)(0.66063,1.02941)%
(0.72883,1.05887)(0.79653,1.08866)(0.86374,1.11874)(0.93046,1.14902)(0.99671,1.17946)%
(1.06248,1.21000)(1.12778,1.24056)(1.19864,1.27410)(1.26919,1.30806)(1.33940,1.34245)%
(1.40925,1.37728)(1.47872,1.41258)(1.54778,1.44838)(1.61641,1.48467)(1.68459,1.52150)%
(1.75230,1.55887)(1.81950,1.59680)(1.88618,1.63532)(1.95232,1.67444)(2.01788,1.71419)%
(2.08285,1.75457)(2.14720,1.79562)(2.21092,1.83735)(2.27396,1.87977)(2.33632,1.92292)%
(2.39797,1.96680)(2.45889,2.01144)(2.51904,2.05685)(2.57842,2.10306)(2.63699,2.15009)%
(2.69473,2.19794)(2.75162,2.24666)(2.80763,2.29624)(2.86274,2.34671)(2.91694,2.39810)%
(2.97018,2.45041)(3.02246,2.50368)(3.05703,2.54034)(3.09171,2.57868)(3.12649,2.61869)%
(3.16138,2.66037)(3.19637,2.70372)(3.23147,2.74874)(3.26667,2.79543)(3.30198,2.84380)%
(3.33739,2.89384)(3.37290,2.94554)(3.40852,2.99892)(3.44424,3.05397)(3.48007,3.11069)%
(3.51600,3.16909)(3.55204,3.22915)(3.58818,3.29089)(3.62443,3.35430)(3.66078,3.41937)%
(3.69723,3.48612)(3.73379,3.55455)(3.77046,3.62464)(3.80722,3.69640)(3.84410,3.76984)%
(3.88107,3.84494)(3.91815,3.92172)(3.95534,4.00017)(3.99263,4.08029)(4.03002,4.16209)%
(4.06752,4.24555)(4.10513,4.33068)%
%
{%
\color[cmyk]{0,1,1,0}%
\polyline(-4.94000,-2.17559)(-4.74120,-2.06462)(-4.54240,-1.95365)(-4.34360,-1.84268)%
(-4.14480,-1.73171)(-3.94600,-1.62074)(-3.74720,-1.50978)(-3.54840,-1.39881)(-3.34960,-1.28784)%
(-3.15080,-1.17687)(-2.95200,-1.06590)(-2.75320,-0.95493)(-2.55440,-0.84396)(-2.35560,-0.73300)%
(-2.15680,-0.62203)(-1.95800,-0.51106)(-1.75920,-0.40009)(-1.56040,-0.28912)(-1.36160,-0.17815)%
(-1.16280,-0.06718)(-0.96400,0.04378)(-0.76520,0.15475)(-0.56640,0.26572)(-0.36760,0.37669)%
(-0.16880,0.48766)(0.03000,0.59863)(0.22880,0.70960)(0.42760,0.82056)(0.62640,0.93153)%
(0.82520,1.04250)(1.02400,1.15347)(1.22280,1.26444)(1.42160,1.37541)(1.62040,1.48638)%
(1.81920,1.59734)(2.01800,1.70831)(2.21680,1.81928)(2.41560,1.93025)(2.61440,2.04122)%
(2.81320,2.15219)(3.01200,2.26316)(3.21080,2.37412)(3.40960,2.48509)(3.60840,2.59606)%
(3.80720,2.70703)(4.00600,2.81800)(4.20480,2.92897)(4.40360,3.03994)(4.60240,3.15090)%
(4.80120,3.26187)(5.00000,3.37284)%
%
}%
\polyline(-4.93969,0.00000)(5.00000,0.00000)%
%
\polyline(0.00000,-4.69000)(0.00000,5.00000)%
%
\settowidth{\Width}{$x$}\setlength{\Width}{0\Width}%
\settoheight{\Height}{$x$}\settodepth{\Depth}{$x$}\setlength{\Height}{-0.5\Height}\setlength{\Depth}{0.5\Depth}\addtolength{\Height}{\Depth}%
\put(5.0500000,0.0000000){\hspace*{\Width}\raisebox{\Height}{$x$}}%
%
\settowidth{\Width}{$y$}\setlength{\Width}{-0.5\Width}%
\settoheight{\Height}{$y$}\settodepth{\Depth}{$y$}\setlength{\Height}{\Depth}%
\put(0.0000000,5.0500000){\hspace*{\Width}\raisebox{\Height}{$y$}}%
%
\settowidth{\Width}{O}\setlength{\Width}{-1\Width}%
\settoheight{\Height}{O}\settodepth{\Depth}{O}\setlength{\Height}{-\Height}%
\put(-0.0500000,-0.0500000){\hspace*{\Width}\raisebox{\Height}{O}}%
%
\end{picture}}%}}
\end{layer}
\begin{itemize}
\item
図が(60,15)の位置から右下に配置される\\
 ・方向は 8方向(nsew)+中心(c)
\item
適当な位置に移動する
\item
よければ\\
 ・第2引数を0とする\\
 ・グリッドが消える\\
 ・文字位置は変わらない
\end{itemize}
%%%%%%%%%%%%

%%%%%%%%%%%%%%%%%%%%

\mainslide{ \ketcindy JS}


\slidepage[m]
%%%%%%%%%%%%%

%%%%%%%%%%%%%%%%%%%%

\newslide{\ketcindy JSの開発}

\vspace*{18mm}

\slidepage
\begin{itemize}
\item
2014年ミュンヘン工科大(TUM)のグループが\\
\href{https://cindyjs.org}{CindyJS}を開発 \href{https://cindyjs.org/gallery/main}{ギャラリー}
\item
Cinderellaとほぼ互換なWeb作成システム
\item
2016年KeTCindyに\ketcindy JSとして組み込み
\item
\href{https://s-takato.github.io/ketcindysample/}{ketcindy sample}に多くのサンプルがある
\end{itemize}
%%%%%%%%%%%%%

%%%%%%%%%%%%%%%%%%%%

\newslide{簡易数式入力システムKeTMath}

\vspace*{18mm}

\slidepage
\label{pageend}\mbox{}

\end{document}

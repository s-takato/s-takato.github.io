%%% Title presen22102
\documentclass[landscape,10pt]{ujarticle}
\special{papersize=\the\paperwidth,\the\paperheight}
\usepackage{ketpic,ketlayer}
\usepackage{ketslide}
\usepackage{amsmath,amssymb}
\usepackage{bm,enumerate}
\usepackage[dvipdfmx]{graphicx}
\usepackage{color}
\definecolor{slidecolora}{cmyk}{0.98,0.13,0,0.43}
\definecolor{slidecolorb}{cmyk}{0.2,0,0,0}
\definecolor{slidecolorc}{cmyk}{0.2,0,0,0}
\definecolor{slidecolord}{cmyk}{0.2,0,0,0}
\definecolor{slidecolore}{cmyk}{0,0,0,0.5}
\definecolor{slidecolorf}{cmyk}{0,0,0,0.5}
\definecolor{slidecolori}{cmyk}{0.98,0.13,0,0.43}
\def\setthin#1{\def\thin{#1}}
\setthin{0}
\newcommand{\slidepage}[1][s]{%
\setcounter{ketpicctra}{18}%
\if#1m \setcounter{ketpicctra}{1}\fi
\hypersetup{linkcolor=black}%

\begin{layer}{118}{0}
\putnotee{122}{-\theketpicctra.05}{\small\thepage/\pageref{pageend}}
\end{layer}\hypersetup{linkcolor=blue}

}
\usepackage{emath}
\usepackage{emathEy}
\usepackage{emathMw}
\usepackage{pict2e}
\usepackage{ketlayermorewith2e}
\usepackage[dvipdfmx,colorlinks=true,linkcolor=blue,filecolor=blue]{hyperref}
\newcommand{\hiduke}{0418}
\newcommand{\hako}[2][1]{\fbox{\raisebox{#1mm}{\mbox{}}\raisebox{-#1mm}{\mbox{}}\,\phantom{#2}\,}}
\newcommand{\hakoa}[2][1]{\fbox{\raisebox{#1mm}{\mbox{}}\raisebox{-#1mm}{\mbox{}}\,#2\,}}
\newcommand{\hakom}[2][1]{\hako[#1]{$#2$}}
\newcommand{\hakoma}[2][1]{\hakoa[#1]{$#2$}}
\def\rad{\;\mathrm{rad}}
\def\deg#1{#1^{\circ}}
\newcommand{\sbunsuu}[2]{\scalebox{0.6}{$\bunsuu{#1}{#2}$}}
\def\pow{$\hspace{-1.5mm}^\hspace{-1mm}$}
\def\dlim{\displaystyle\lim}
\newcommand{\brd}[2][1]{\scalebox{#1}{\color{red}\fbox{\color{black}$#2$}}}
\newcommand\down[1][0.5zw]{\vspace{#1}\\}

\setmargin{25}{145}{15}{100}

\ketslideinit

\pagestyle{empty}

\begin{document}

\begin{layer}{120}{0}
\putnotese{0}{0}{{\Large\bf
\color[cmyk]{1,1,0,0}

\begin{layer}{120}{0}
{\Huge \putnotes{60}{20}{指数・対数関数}}
\putnotes{60}{70}{2022.05.23}
\end{layer}

}
}
\end{layer}

\def\mainslidetitley{22}
\def\ketcletter{slidecolora}
\def\ketcbox{slidecolorb}
\def\ketdbox{slidecolorc}
\def\ketcframe{slidecolord}
\def\ketcshadow{slidecolore}
\def\ketdshadow{slidecolorf}
\def\slidetitlex{6}
\def\slidetitlesize{1.3}
\def\mketcletter{slidecolori}
\def\mketcbox{yellow}
\def\mketdbox{yellow}
\def\mketcframe{yellow}
\def\mslidetitlex{62}
\def\mslidetitlesize{2}

\color{black}
\Large\bf\boldmath
\addtocounter{page}{-1}

\def\MARU{}
\renewcommand{\MARU}[1]{{\ooalign{\hfil$#1$\/\hfil\crcr\raise.167ex\hbox{\mathhexbox20D}}}}
\renewcommand{\slidepage}[1][s]{%
\setcounter{ketpicctra}{18}%
\if#1m \setcounter{ketpicctra}{1}\fi
\hypersetup{linkcolor=black}%
\begin{layer}{118}{0}
\putnotee{115}{-\theketpicctra.05}{\small\hiduke-\thepage/\pageref{pageend}}
\end{layer}\hypersetup{linkcolor=blue}
}
\newcounter{ban}
\setcounter{ban}{1}
\newcommand{\monban}[1][\hiduke]{%
#1-\theban\ %
\addtocounter{ban}{1}%
}
\newcommand{\monbannoadd}[1][\hiduke]{%
#1-\theban\ %
}
\newcommand{\addban}{%
\addtocounter{ban}{1}%%210614
}
\newcounter{edawidth}
\newcounter{edactr}
\newcommand{\seteda}[1]{
\setcounter{edawidth}{#1}
\setcounter{edactr}{1}
}
\newcommand{\eda}[2][\theedawidth ]{%
\noindent\Ltab{#1 mm}{[\theedactr]\ #2}%
\addtocounter{edactr}{1}%
}
%%%%%%%%%%%%

%%%%%%%%%%%%%%%%%%%%

\mainslide{関数}


\slidepage[m]
%%%%%%%%%%%%

%%%%%%%%%%%%%%%%%%%%

\newslide{関数}

\vspace*{18mm}

\slidepage
\begin{itemize}
\item
変数$x$の値を与えると変数$y$の値が求まる\\
\hspace*{1zw}例)$y=2x+1,\ y=x^2+2x+1$
\item
これを変数$x$の関数という
\item
変数$x$の関数であることを$f(x)$などで表す\\
 例1)$f(x)=2x+1$(1次関数)\\
 例2)$g(x)=x^2+2x+1$(2次関数)
\end{itemize}

\newslide{関数記号}

\vspace*{18mm}

\slidepage
\seteda{55}
\begin{itemize}
\item
関数$f(x)$の$x$に定数$a$を代入した値を$f(a)$で表す
\item
例)$f(x)=x^2+x-1$のとき $f(2)=2^2+2-1=5$
\item
課題\monban $f(x)=3x+1$のとき,次を求めよ.\\
\eda{$f(0)$}\eda{$f(2)$}\\
\eda{$f(-3)$}\eda{$f(a-1)$\ \ \ ($a$は定数)}\\
\hfill TextP80
\end{itemize}

\newslide{関数のグラフ}

\vspace*{18mm}

\slidepage
\down
関数$y=f(x)$
\begin{itemize}
\item
$x$を変えるとき,点$\bigl(x,\ f(x)\bigr)$も変わる.
\item
[]例) 1次関数$y=2x+1$\vspace{1mm}\\
\input{fig/table1aans}
\item
この点の集まりを,その関数の{\color{red}グラフ}という.
\end{itemize}

\newslide{1次関数のグラフ}

\vspace*{18mm}

\slidepage
\down
例)$y=2x+1$

\begin{layer}{120}{0}
\putnotese{70}{-3}{\scalebox{0.6}{%%% /polytech.git/n101/fig/table1b.tex 
%%% Generator=presen0601.cdy 
{\unitlength=1cm%
\begin{picture}%
(9.6,1.2)(0,0)%
\special{pn 8}%
%
\Large\bf\boldmath%
\small%
{%
\color{red}%
\special{pa     0  -472}\special{pa     0    -0}%
\special{fp}%
}%
{%
\color{red}%
\special{pa   315  -472}\special{pa   315    -0}%
\special{fp}%
}%
{%
\color{red}%
\special{pa   630  -472}\special{pa   630    -0}%
\special{fp}%
}%
{%
\color{red}%
\special{pa   945  -472}\special{pa   945    -0}%
\special{fp}%
}%
{%
\color{red}%
\special{pa  1260  -472}\special{pa  1260    -0}%
\special{fp}%
}%
{%
\color{red}%
\special{pa  1575  -472}\special{pa  1575    -0}%
\special{fp}%
}%
{%
\color{red}%
\special{pa  1890  -472}\special{pa  1890    -0}%
\special{fp}%
}%
{%
\color{red}%
\special{pa  2205  -472}\special{pa  2205    -0}%
\special{fp}%
}%
{%
\color{red}%
\special{pa  2520  -472}\special{pa  2520    -0}%
\special{fp}%
}%
{%
\color{red}%
\special{pa  2835  -472}\special{pa  2835    -0}%
\special{fp}%
}%
{%
\color{red}%
\special{pa  3150  -472}\special{pa  3150    -0}%
\special{fp}%
}%
{%
\color{red}%
\special{pa  3465  -472}\special{pa  3465    -0}%
\special{fp}%
}%
{%
\color{red}%
\special{pa  3780  -472}\special{pa  3780    -0}%
\special{fp}%
}%
{%
\color{red}%
\special{pa     0  -472}\special{pa  3780  -472}%
\special{fp}%
}%
{%
\color{red}%
\special{pa     0  -236}\special{pa  3780  -236}%
\special{fp}%
}%
{%
\color{red}%
\special{pa     0    -0}\special{pa  3780    -0}%
\special{fp}%
}%
{%
\color{red}%
\settowidth{\Width}{$x$}\setlength{\Width}{-0.5\Width}%
\settoheight{\Height}{$x$}\settodepth{\Depth}{$x$}\setlength{\Height}{-0.5\Height}\setlength{\Depth}{0.5\Depth}\addtolength{\Height}{\Depth}%
\put(0.4000000,0.9000000){\hspace*{\Width}\raisebox{\Height}{$x$}}%
%
}%
{%
\color{red}%
\settowidth{\Width}{$-5$}\setlength{\Width}{-0.5\Width}%
\settoheight{\Height}{$-5$}\settodepth{\Depth}{$-5$}\setlength{\Height}{-0.5\Height}\setlength{\Depth}{0.5\Depth}\addtolength{\Height}{\Depth}%
\put(1.2000000,0.9000000){\hspace*{\Width}\raisebox{\Height}{$-5$}}%
%
}%
{%
\color{red}%
\settowidth{\Width}{$-4$}\setlength{\Width}{-0.5\Width}%
\settoheight{\Height}{$-4$}\settodepth{\Depth}{$-4$}\setlength{\Height}{-0.5\Height}\setlength{\Depth}{0.5\Depth}\addtolength{\Height}{\Depth}%
\put(2.0000000,0.9000000){\hspace*{\Width}\raisebox{\Height}{$-4$}}%
%
}%
{%
\color{red}%
\settowidth{\Width}{$-3$}\setlength{\Width}{-0.5\Width}%
\settoheight{\Height}{$-3$}\settodepth{\Depth}{$-3$}\setlength{\Height}{-0.5\Height}\setlength{\Depth}{0.5\Depth}\addtolength{\Height}{\Depth}%
\put(2.8000000,0.9000000){\hspace*{\Width}\raisebox{\Height}{$-3$}}%
%
}%
{%
\color{red}%
\settowidth{\Width}{$-2$}\setlength{\Width}{-0.5\Width}%
\settoheight{\Height}{$-2$}\settodepth{\Depth}{$-2$}\setlength{\Height}{-0.5\Height}\setlength{\Depth}{0.5\Depth}\addtolength{\Height}{\Depth}%
\put(3.6000000,0.9000000){\hspace*{\Width}\raisebox{\Height}{$-2$}}%
%
}%
{%
\color{red}%
\settowidth{\Width}{$-1$}\setlength{\Width}{-0.5\Width}%
\settoheight{\Height}{$-1$}\settodepth{\Depth}{$-1$}\setlength{\Height}{-0.5\Height}\setlength{\Depth}{0.5\Depth}\addtolength{\Height}{\Depth}%
\put(4.4000000,0.9000000){\hspace*{\Width}\raisebox{\Height}{$-1$}}%
%
}%
{%
\color{red}%
\settowidth{\Width}{$0$}\setlength{\Width}{-0.5\Width}%
\settoheight{\Height}{$0$}\settodepth{\Depth}{$0$}\setlength{\Height}{-0.5\Height}\setlength{\Depth}{0.5\Depth}\addtolength{\Height}{\Depth}%
\put(5.2000000,0.9000000){\hspace*{\Width}\raisebox{\Height}{$0$}}%
%
}%
{%
\color{red}%
\settowidth{\Width}{$1$}\setlength{\Width}{-0.5\Width}%
\settoheight{\Height}{$1$}\settodepth{\Depth}{$1$}\setlength{\Height}{-0.5\Height}\setlength{\Depth}{0.5\Depth}\addtolength{\Height}{\Depth}%
\put(6.0000000,0.9000000){\hspace*{\Width}\raisebox{\Height}{$1$}}%
%
}%
{%
\color{red}%
\settowidth{\Width}{$2$}\setlength{\Width}{-0.5\Width}%
\settoheight{\Height}{$2$}\settodepth{\Depth}{$2$}\setlength{\Height}{-0.5\Height}\setlength{\Depth}{0.5\Depth}\addtolength{\Height}{\Depth}%
\put(6.8000000,0.9000000){\hspace*{\Width}\raisebox{\Height}{$2$}}%
%
}%
{%
\color{red}%
\settowidth{\Width}{$3$}\setlength{\Width}{-0.5\Width}%
\settoheight{\Height}{$3$}\settodepth{\Depth}{$3$}\setlength{\Height}{-0.5\Height}\setlength{\Depth}{0.5\Depth}\addtolength{\Height}{\Depth}%
\put(7.6000000,0.9000000){\hspace*{\Width}\raisebox{\Height}{$3$}}%
%
}%
{%
\color{red}%
\settowidth{\Width}{$4$}\setlength{\Width}{-0.5\Width}%
\settoheight{\Height}{$4$}\settodepth{\Depth}{$4$}\setlength{\Height}{-0.5\Height}\setlength{\Depth}{0.5\Depth}\addtolength{\Height}{\Depth}%
\put(8.4000000,0.9000000){\hspace*{\Width}\raisebox{\Height}{$4$}}%
%
}%
{%
\color{red}%
\settowidth{\Width}{$5$}\setlength{\Width}{-0.5\Width}%
\settoheight{\Height}{$5$}\settodepth{\Depth}{$5$}\setlength{\Height}{-0.5\Height}\setlength{\Depth}{0.5\Depth}\addtolength{\Height}{\Depth}%
\put(9.2000000,0.9000000){\hspace*{\Width}\raisebox{\Height}{$5$}}%
%
}%
{%
\color{red}%
\settowidth{\Width}{$y$}\setlength{\Width}{-0.5\Width}%
\settoheight{\Height}{$y$}\settodepth{\Depth}{$y$}\setlength{\Height}{-0.5\Height}\setlength{\Depth}{0.5\Depth}\addtolength{\Height}{\Depth}%
\put(0.4000000,0.3000000){\hspace*{\Width}\raisebox{\Height}{$y$}}%
%
}%
{%
\color{red}%
\settowidth{\Width}{$-9$}\setlength{\Width}{-0.5\Width}%
\settoheight{\Height}{$-9$}\settodepth{\Depth}{$-9$}\setlength{\Height}{-0.5\Height}\setlength{\Depth}{0.5\Depth}\addtolength{\Height}{\Depth}%
\put(1.2000000,0.3000000){\hspace*{\Width}\raisebox{\Height}{$-9$}}%
%
}%
{%
\color{red}%
\settowidth{\Width}{$-7$}\setlength{\Width}{-0.5\Width}%
\settoheight{\Height}{$-7$}\settodepth{\Depth}{$-7$}\setlength{\Height}{-0.5\Height}\setlength{\Depth}{0.5\Depth}\addtolength{\Height}{\Depth}%
\put(2.0000000,0.3000000){\hspace*{\Width}\raisebox{\Height}{$-7$}}%
%
}%
{%
\color{red}%
\settowidth{\Width}{$-5$}\setlength{\Width}{-0.5\Width}%
\settoheight{\Height}{$-5$}\settodepth{\Depth}{$-5$}\setlength{\Height}{-0.5\Height}\setlength{\Depth}{0.5\Depth}\addtolength{\Height}{\Depth}%
\put(2.8000000,0.3000000){\hspace*{\Width}\raisebox{\Height}{$-5$}}%
%
}%
{%
\color{red}%
\settowidth{\Width}{$-3$}\setlength{\Width}{-0.5\Width}%
\settoheight{\Height}{$-3$}\settodepth{\Depth}{$-3$}\setlength{\Height}{-0.5\Height}\setlength{\Depth}{0.5\Depth}\addtolength{\Height}{\Depth}%
\put(3.6000000,0.3000000){\hspace*{\Width}\raisebox{\Height}{$-3$}}%
%
}%
{%
\color{red}%
\settowidth{\Width}{$-1$}\setlength{\Width}{-0.5\Width}%
\settoheight{\Height}{$-1$}\settodepth{\Depth}{$-1$}\setlength{\Height}{-0.5\Height}\setlength{\Depth}{0.5\Depth}\addtolength{\Height}{\Depth}%
\put(4.4000000,0.3000000){\hspace*{\Width}\raisebox{\Height}{$-1$}}%
%
}%
{%
\color{red}%
\settowidth{\Width}{$1$}\setlength{\Width}{-0.5\Width}%
\settoheight{\Height}{$1$}\settodepth{\Depth}{$1$}\setlength{\Height}{-0.5\Height}\setlength{\Depth}{0.5\Depth}\addtolength{\Height}{\Depth}%
\put(5.2000000,0.3000000){\hspace*{\Width}\raisebox{\Height}{$1$}}%
%
}%
{%
\color{red}%
\settowidth{\Width}{$3$}\setlength{\Width}{-0.5\Width}%
\settoheight{\Height}{$3$}\settodepth{\Depth}{$3$}\setlength{\Height}{-0.5\Height}\setlength{\Depth}{0.5\Depth}\addtolength{\Height}{\Depth}%
\put(6.0000000,0.3000000){\hspace*{\Width}\raisebox{\Height}{$3$}}%
%
}%
{%
\color{red}%
\settowidth{\Width}{$5$}\setlength{\Width}{-0.5\Width}%
\settoheight{\Height}{$5$}\settodepth{\Depth}{$5$}\setlength{\Height}{-0.5\Height}\setlength{\Depth}{0.5\Depth}\addtolength{\Height}{\Depth}%
\put(6.8000000,0.3000000){\hspace*{\Width}\raisebox{\Height}{$5$}}%
%
}%
{%
\color{red}%
\settowidth{\Width}{$7$}\setlength{\Width}{-0.5\Width}%
\settoheight{\Height}{$7$}\settodepth{\Depth}{$7$}\setlength{\Height}{-0.5\Height}\setlength{\Depth}{0.5\Depth}\addtolength{\Height}{\Depth}%
\put(7.6000000,0.3000000){\hspace*{\Width}\raisebox{\Height}{$7$}}%
%
}%
{%
\color{red}%
\settowidth{\Width}{$9$}\setlength{\Width}{-0.5\Width}%
\settoheight{\Height}{$9$}\settodepth{\Depth}{$9$}\setlength{\Height}{-0.5\Height}\setlength{\Depth}{0.5\Depth}\addtolength{\Height}{\Depth}%
\put(8.4000000,0.3000000){\hspace*{\Width}\raisebox{\Height}{$9$}}%
%
}%
{%
\color{red}%
\settowidth{\Width}{$11$}\setlength{\Width}{-0.5\Width}%
\settoheight{\Height}{$11$}\settodepth{\Depth}{$11$}\setlength{\Height}{-0.5\Height}\setlength{\Depth}{0.5\Depth}\addtolength{\Height}{\Depth}%
\put(9.2000000,0.3000000){\hspace*{\Width}\raisebox{\Height}{$11$}}%
%
}%
\end{picture}}%}}
\putnotes{60}{6}{\scalebox{0.5}{\input{fig/graphpaper3.tex}}}
\putnotese{100}{30}{\color{blue}傾き\ \ $2$}
\putnotese{100}{36}{\color{blue}$y$切片\ $1$}
\end{layer}


\newslide{1次関数のグラフ}

\vspace*{18mm}

%%repeat=3
\slidepage
\seteda{90}
\begin{enumerate}[(1)]
\item
[課題]\monban 「2.関数のグラフ」を用いて,次の1次関数のグラフをかけ.
また,傾き$a$と$y$切片$b$を求めよ.\\
\eda{$y=3x+3$} \hfill TextP81\\
\eda{$y=10-2x$} \hfill TextP81\\
\eda{$y=2x+2$} \hfill TextP81\\
\eda{$y=\bunsuu{1}{2}x+1$}
\end{enumerate}
%%%%%%%%%%%%

%%%%%%%%%%%%%%%%%%%%

\newslide{2次関数のグラフ(基本形)}

\vspace*{18mm}

\slidepage
\down
「2.関数のグラフ」で$y=x^2,\ y=-x^2$をかこう.

\begin{layer}{120}{0}
\putnotese{55}{5}{\scalebox{0.7}{\input{fig/parabola1.tex}}}
\arrowlineseg{80}{30}{20}{-30}
\putnotee{97}{40}{頂点}
\arrowlineseg{80}{20}{20}{0}
\putnotee{100}{20}{軸}
\putnotese{55}{5}{\scalebox{0.7}{%%% /polytech22.git/102-0418/presen/fig/parabola2.tex 
%%% Generator=presen22102.cdy 
{\unitlength=7mm%
\begin{picture}%
(10,10)(-5,-5)%
\linethickness{0.008in}%%
\Large\bf\boldmath%
\small%
{%
\color[cmyk]{0,1,1,0}%
\polyline(-2.23478,-5.00000)(-2.20000,-4.84000)(-2.00000,-4.00000)(-1.80000,-3.24000)%
(-1.60000,-2.56000)(-1.40000,-1.96000)(-1.20000,-1.44000)(-1.00000,-1.00000)(-0.80000,-0.64000)%
(-0.60000,-0.36000)(-0.40000,-0.16000)(-0.20000,-0.04000)(0.00000,0.00000)(0.20000,-0.04000)%
(0.40000,-0.16000)(0.60000,-0.36000)(0.80000,-0.64000)(1.00000,-1.00000)(1.20000,-1.44000)%
(1.40000,-1.96000)(1.60000,-2.56000)(1.80000,-3.24000)(2.00000,-4.00000)(2.20000,-4.84000)%
(2.23478,-5.00000)%
%
}%
\polyline(-4.00000,0.07143)(-4.00000,-0.07143)%
%
\settowidth{\Width}{$-4$}\setlength{\Width}{-0.5\Width}%
\settoheight{\Height}{$-4$}\settodepth{\Depth}{$-4$}\setlength{\Height}{-\Height}%
\put(-4.0000000,-0.1428571){\hspace*{\Width}\raisebox{\Height}{$-4$}}%
%
\polyline(-3.00000,0.07143)(-3.00000,-0.07143)%
%
\settowidth{\Width}{$-3$}\setlength{\Width}{-0.5\Width}%
\settoheight{\Height}{$-3$}\settodepth{\Depth}{$-3$}\setlength{\Height}{-\Height}%
\put(-3.0000000,-0.1428571){\hspace*{\Width}\raisebox{\Height}{$-3$}}%
%
\polyline(-2.00000,0.07143)(-2.00000,-0.07143)%
%
\settowidth{\Width}{$-2$}\setlength{\Width}{-0.5\Width}%
\settoheight{\Height}{$-2$}\settodepth{\Depth}{$-2$}\setlength{\Height}{-\Height}%
\put(-2.0000000,-0.1428571){\hspace*{\Width}\raisebox{\Height}{$-2$}}%
%
\polyline(-1.00000,0.07143)(-1.00000,-0.07143)%
%
\settowidth{\Width}{$-1$}\setlength{\Width}{-0.5\Width}%
\settoheight{\Height}{$-1$}\settodepth{\Depth}{$-1$}\setlength{\Height}{-\Height}%
\put(-1.0000000,-0.1428571){\hspace*{\Width}\raisebox{\Height}{$-1$}}%
%
\polyline(1.00000,0.07143)(1.00000,-0.07143)%
%
\settowidth{\Width}{$1$}\setlength{\Width}{-0.5\Width}%
\settoheight{\Height}{$1$}\settodepth{\Depth}{$1$}\setlength{\Height}{-\Height}%
\put(1.0000000,-0.1428571){\hspace*{\Width}\raisebox{\Height}{$1$}}%
%
\polyline(2.00000,0.07143)(2.00000,-0.07143)%
%
\settowidth{\Width}{$2$}\setlength{\Width}{-0.5\Width}%
\settoheight{\Height}{$2$}\settodepth{\Depth}{$2$}\setlength{\Height}{-\Height}%
\put(2.0000000,-0.1428571){\hspace*{\Width}\raisebox{\Height}{$2$}}%
%
\polyline(3.00000,0.07143)(3.00000,-0.07143)%
%
\settowidth{\Width}{$3$}\setlength{\Width}{-0.5\Width}%
\settoheight{\Height}{$3$}\settodepth{\Depth}{$3$}\setlength{\Height}{-\Height}%
\put(3.0000000,-0.1428571){\hspace*{\Width}\raisebox{\Height}{$3$}}%
%
\polyline(4.00000,0.07143)(4.00000,-0.07143)%
%
\settowidth{\Width}{$4$}\setlength{\Width}{-0.5\Width}%
\settoheight{\Height}{$4$}\settodepth{\Depth}{$4$}\setlength{\Height}{-\Height}%
\put(4.0000000,-0.1428571){\hspace*{\Width}\raisebox{\Height}{$4$}}%
%
\polyline(0.07143,-4.00000)(-0.07143,-4.00000)%
%
\settowidth{\Width}{$-4$}\setlength{\Width}{-1\Width}%
\settoheight{\Height}{$-4$}\settodepth{\Depth}{$-4$}\setlength{\Height}{-0.5\Height}\setlength{\Depth}{0.5\Depth}\addtolength{\Height}{\Depth}%
\put(-0.1428571,-4.0000000){\hspace*{\Width}\raisebox{\Height}{$-4$}}%
%
\polyline(0.07143,-3.00000)(-0.07143,-3.00000)%
%
\settowidth{\Width}{$-3$}\setlength{\Width}{-1\Width}%
\settoheight{\Height}{$-3$}\settodepth{\Depth}{$-3$}\setlength{\Height}{-0.5\Height}\setlength{\Depth}{0.5\Depth}\addtolength{\Height}{\Depth}%
\put(-0.1428571,-3.0000000){\hspace*{\Width}\raisebox{\Height}{$-3$}}%
%
\polyline(0.07143,-2.00000)(-0.07143,-2.00000)%
%
\settowidth{\Width}{$-2$}\setlength{\Width}{-1\Width}%
\settoheight{\Height}{$-2$}\settodepth{\Depth}{$-2$}\setlength{\Height}{-0.5\Height}\setlength{\Depth}{0.5\Depth}\addtolength{\Height}{\Depth}%
\put(-0.1428571,-2.0000000){\hspace*{\Width}\raisebox{\Height}{$-2$}}%
%
\polyline(0.07143,-1.00000)(-0.07143,-1.00000)%
%
\settowidth{\Width}{$-1$}\setlength{\Width}{-1\Width}%
\settoheight{\Height}{$-1$}\settodepth{\Depth}{$-1$}\setlength{\Height}{-0.5\Height}\setlength{\Depth}{0.5\Depth}\addtolength{\Height}{\Depth}%
\put(-0.1428571,-1.0000000){\hspace*{\Width}\raisebox{\Height}{$-1$}}%
%
\polyline(0.07143,1.00000)(-0.07143,1.00000)%
%
\settowidth{\Width}{$1$}\setlength{\Width}{-1\Width}%
\settoheight{\Height}{$1$}\settodepth{\Depth}{$1$}\setlength{\Height}{-0.5\Height}\setlength{\Depth}{0.5\Depth}\addtolength{\Height}{\Depth}%
\put(-0.1428571,1.0000000){\hspace*{\Width}\raisebox{\Height}{$1$}}%
%
\polyline(0.07143,2.00000)(-0.07143,2.00000)%
%
\settowidth{\Width}{$2$}\setlength{\Width}{-1\Width}%
\settoheight{\Height}{$2$}\settodepth{\Depth}{$2$}\setlength{\Height}{-0.5\Height}\setlength{\Depth}{0.5\Depth}\addtolength{\Height}{\Depth}%
\put(-0.1428571,2.0000000){\hspace*{\Width}\raisebox{\Height}{$2$}}%
%
\polyline(0.07143,3.00000)(-0.07143,3.00000)%
%
\settowidth{\Width}{$3$}\setlength{\Width}{-1\Width}%
\settoheight{\Height}{$3$}\settodepth{\Depth}{$3$}\setlength{\Height}{-0.5\Height}\setlength{\Depth}{0.5\Depth}\addtolength{\Height}{\Depth}%
\put(-0.1428571,3.0000000){\hspace*{\Width}\raisebox{\Height}{$3$}}%
%
\polyline(0.07143,4.00000)(-0.07143,4.00000)%
%
\settowidth{\Width}{$4$}\setlength{\Width}{-1\Width}%
\settoheight{\Height}{$4$}\settodepth{\Depth}{$4$}\setlength{\Height}{-0.5\Height}\setlength{\Depth}{0.5\Depth}\addtolength{\Height}{\Depth}%
\put(-0.1428571,4.0000000){\hspace*{\Width}\raisebox{\Height}{$4$}}%
%
\polyline(-5.00000,0.00000)(5.00000,0.00000)%
%
\polyline(0.00000,-5.00000)(0.00000,5.00000)%
%
\settowidth{\Width}{$x$}\setlength{\Width}{0\Width}%
\settoheight{\Height}{$x$}\settodepth{\Depth}{$x$}\setlength{\Height}{-0.5\Height}\setlength{\Depth}{0.5\Depth}\addtolength{\Height}{\Depth}%
\put(5.0714286,0.0000000){\hspace*{\Width}\raisebox{\Height}{$x$}}%
%
\settowidth{\Width}{$y$}\setlength{\Width}{-0.5\Width}%
\settoheight{\Height}{$y$}\settodepth{\Depth}{$y$}\setlength{\Height}{\Depth}%
\put(0.0000000,5.0714286){\hspace*{\Width}\raisebox{\Height}{$y$}}%
%
\settowidth{\Width}{O}\setlength{\Width}{-1\Width}%
\settoheight{\Height}{O}\settodepth{\Depth}{O}\setlength{\Height}{-\Height}%
\put(-0.0714286,-0.0714286){\hspace*{\Width}\raisebox{\Height}{O}}%
%
\end{picture}}%}}
\end{layer}

\begin{itemize}
\item
$y=x^2$
\item
[] 軸は$x=0$($y$軸)
\item
[] 頂点は$(0,\ 0)$
\item
[] 下に凸
\item
$y=-x^2$
\item
[] 上に凸
\end{itemize}

\newslide{2次関数のグラフ2}

\vspace*{18mm}

\slidepage

\begin{layer}{120}{0}
\putnotese{75}{18}{\scalebox{0.8}{%%% /polytech22.git/102-0418/presen/fig/idou4.tex 
%%% Generator=presen22102.cdy 
{\unitlength=1cm%
\begin{picture}%
(5.5,5.5)(-2,-0.5)%
\linethickness{0.008in}%%
\Large\bf\boldmath%
\small%
\linethickness{0.012in}%%
\polyline(-2.00000,4.00000)(-1.89000,3.57210)(-1.78000,3.16840)(-1.67000,2.78890)%
(-1.56000,2.43360)(-1.45000,2.10250)(-1.34000,1.79560)(-1.23000,1.51290)(-1.12000,1.25440)%
(-1.01000,1.02010)(-0.90000,0.81000)(-0.79000,0.62410)(-0.68000,0.46240)(-0.57000,0.32490)%
(-0.46000,0.21160)(-0.35000,0.12250)(-0.24000,0.05760)(-0.13000,0.01690)(-0.02000,0.00040)%
(0.09000,0.00810)(0.20000,0.04000)(0.31000,0.09610)(0.42000,0.17640)(0.53000,0.28090)%
(0.64000,0.40960)(0.75000,0.56250)(0.86000,0.73960)(0.97000,0.94090)(1.08000,1.16640)%
(1.19000,1.41610)(1.30000,1.69000)(1.41000,1.98810)(1.52000,2.31040)(1.63000,2.65690)%
(1.74000,3.02760)(1.85000,3.42250)(1.96000,3.84160)(2.07000,4.28490)(2.18000,4.75240)%
(2.23539,5.00000)%
%
\linethickness{0.008in}%%
{%
\color[cmyk]{0,1,1,0}%
\linethickness{0.012in}%%
\polyline(-0.73118,5.00000)(-0.68000,4.82240)(-0.57000,4.46490)(-0.46000,4.13160)%
(-0.35000,3.82250)(-0.24000,3.53760)(-0.13000,3.27690)(-0.02000,3.04040)(0.09000,2.82810)%
(0.20000,2.64000)(0.31000,2.47610)(0.42000,2.33640)(0.53000,2.22090)(0.64000,2.12960)%
(0.75000,2.06250)(0.86000,2.01960)(0.97000,2.00090)(1.08000,2.00640)(1.19000,2.03610)%
(1.30000,2.09000)(1.41000,2.16810)(1.52000,2.27040)(1.63000,2.39690)(1.74000,2.54760)%
(1.85000,2.72250)(1.96000,2.92160)(2.07000,3.14490)(2.18000,3.39240)(2.29000,3.66410)%
(2.40000,3.96000)(2.51000,4.28010)(2.62000,4.62440)(2.73000,4.99290)(2.73199,5.00000)%
%
\linethickness{0.008in}%%
}%
\polyline(-2.00000,0.00000)(3.50000,0.00000)%
%
\polyline(0.00000,-0.50000)(0.00000,5.00000)%
%
\settowidth{\Width}{$x$}\setlength{\Width}{0\Width}%
\settoheight{\Height}{$x$}\settodepth{\Depth}{$x$}\setlength{\Height}{-0.5\Height}\setlength{\Depth}{0.5\Depth}\addtolength{\Height}{\Depth}%
\put(3.5500000,0.0000000){\hspace*{\Width}\raisebox{\Height}{$x$}}%
%
\settowidth{\Width}{$y$}\setlength{\Width}{-0.5\Width}%
\settoheight{\Height}{$y$}\settodepth{\Depth}{$y$}\setlength{\Height}{\Depth}%
\put(0.0000000,5.0500000){\hspace*{\Width}\raisebox{\Height}{$y$}}%
%
\settowidth{\Width}{O}\setlength{\Width}{-1\Width}%
\settoheight{\Height}{O}\settodepth{\Depth}{O}\setlength{\Height}{-\Height}%
\put(-0.0500000,-0.0500000){\hspace*{\Width}\raisebox{\Height}{O}}%
%
\end{picture}}%}}
\end{layer}

カッコ内の定数を変えたときのグラフをかこう.
\begin{enumerate}[(1)]
\item
$y=ax^2$(定数$a$)\vspace{-2mm}
\item
[] {\color{blue}開き(増え方)が変わる}\vspace{-2mm}
\item
$y=ax^2+c$(定数$c$)\vspace{-2mm}
\item
[] {\color{blue}縦方向に$c$だけ平行移動}\vspace{-2mm}
\item
$y=a(x-b)^2$(定数$b$)\vspace{-2mm}
\item
[] {\color{blue}横方向に$b$だけ平行移動}\vspace{-2mm}
\item
$y=a(x-b)^2+c$(定数$b,\ c$)\vspace{-2mm}
\item
[] {\color{blue}頂点の座標は $(b,\ c)$}
\end{enumerate}

\newslide{課題 2次関数のグラフ}

\vspace*{18mm}

%%repeat=2
\slidepage
\seteda{90}
\begin{enumerate}[(1)]
\item
[課題]\monban 「2.関数のグラフ」を用いて,次の2次関数のグラフをかけ.
また,$y=x^2$のグラフをどのように移動(変形)したかを答えよ.\\
\eda{$y=2x^2$}\\
\eda{$y=x^2+1$}\\
\eda{$y=(x-3)^2$}\\
\eda{$y=(x+1)^2$}\\
%%\eda{$y=-(x-2)^2$}
\end{enumerate}
%%%%%%%%%%%%

%%%%%%%%%%%%%%%%%%%%

\newslide{2次関数のグラフ3}

\vspace*{18mm}

\slidepage

\begin{layer}{120}{0}
\putnoten{90}{0}{{\color{red}$(x^2+2bx+b^2)+d$}}
\putnoten{90}{5}{{\color{red}\rotatebox[origin=c]{90}{$=$}}}
\putnotese{75}{20}{\scalebox{0.8}{%%% /polytech22.git/102-0418/presen/fig/idou6.tex 
%%% Generator=presen22102.cdy 
{\unitlength=1cm%
\begin{picture}%
(5.5,6)(-2,-0.5)%
\linethickness{0.008in}%%
\Large\bf\boldmath%
\small%
{%
\color[cmyk]{0,1,1,0}%
\linethickness{0.012in}%%
\polyline(-0.34510,-0.50000)(-0.24000,-0.01760)(-0.13000,0.46310)(-0.02000,0.91960)%
(0.09000,1.35190)(0.20000,1.76000)(0.31000,2.14390)(0.42000,2.50360)(0.53000,2.83910)%
(0.64000,3.15040)(0.75000,3.43750)(0.86000,3.70040)(0.97000,3.93910)(1.08000,4.15360)%
(1.19000,4.34390)(1.30000,4.51000)(1.41000,4.65190)(1.52000,4.76960)(1.63000,4.86310)%
(1.74000,4.93240)(1.85000,4.97750)(1.96000,4.99840)(2.07000,4.99510)(2.18000,4.96760)%
(2.29000,4.91590)(2.40000,4.84000)(2.51000,4.73990)(2.62000,4.61560)(2.73000,4.46710)%
(2.84000,4.29440)(2.95000,4.09750)(3.06000,3.87640)(3.17000,3.63110)(3.28000,3.36160)%
(3.39000,3.06790)(3.50000,2.75000)%
%
\linethickness{0.008in}%%
}%
\polyline(2.00000,0.00000)(2.00000,0.09859)\polyline(2.00000,0.19718)(2.00000,0.29577)%
\polyline(2.00000,0.39437)(2.00000,0.49296)\polyline(2.00000,0.59155)(2.00000,0.69014)%
\polyline(2.00000,0.78873)(2.00000,0.88732)\polyline(2.00000,0.98592)(2.00000,1.08451)%
\polyline(2.00000,1.18310)(2.00000,1.28169)\polyline(2.00000,1.38028)(2.00000,1.47887)%
\polyline(2.00000,1.57746)(2.00000,1.67606)\polyline(2.00000,1.77465)(2.00000,1.87324)%
\polyline(2.00000,1.97183)(2.00000,2.07042)\polyline(2.00000,2.16901)(2.00000,2.26761)%
\polyline(2.00000,2.36620)(2.00000,2.46479)\polyline(2.00000,2.56338)(2.00000,2.66197)%
\polyline(2.00000,2.76056)(2.00000,2.85915)\polyline(2.00000,2.95775)(2.00000,3.05634)%
\polyline(2.00000,3.15493)(2.00000,3.25352)\polyline(2.00000,3.35211)(2.00000,3.45070)%
\polyline(2.00000,3.54930)(2.00000,3.64789)\polyline(2.00000,3.74648)(2.00000,3.84507)%
\polyline(2.00000,3.94366)(2.00000,4.04225)\polyline(2.00000,4.14085)(2.00000,4.23944)%
\polyline(2.00000,4.33803)(2.00000,4.43662)\polyline(2.00000,4.53521)(2.00000,4.63380)%
\polyline(2.00000,4.73239)(2.00000,4.83099)\polyline(2.00000,4.92958)(2.00000,5.00000)(1.97183,5.00000)%
\polyline(1.87324,5.00000)(1.77465,5.00000)\polyline(1.67606,5.00000)(1.57746,5.00000)%
\polyline(1.47887,5.00000)(1.38028,5.00000)\polyline(1.28169,5.00000)(1.18310,5.00000)%
\polyline(1.08451,5.00000)(0.98592,5.00000)\polyline(0.88732,5.00000)(0.78873,5.00000)%
\polyline(0.69014,5.00000)(0.59155,5.00000)\polyline(0.49296,5.00000)(0.39437,5.00000)%
\polyline(0.29577,5.00000)(0.19718,5.00000)\polyline(0.09859,5.00000)(0.00000,5.00000)%
%
%
\polyline(2.00000,0.05000)(2.00000,-0.05000)%
%
\settowidth{\Width}{$2$}\setlength{\Width}{-0.5\Width}%
\settoheight{\Height}{$2$}\settodepth{\Depth}{$2$}\setlength{\Height}{-\Height}%
\put(2.0000000,-0.1000000){\hspace*{\Width}\raisebox{\Height}{$2$}}%
%
\polyline(0.05000,5.00000)(-0.05000,5.00000)%
%
\settowidth{\Width}{$5$}\setlength{\Width}{-1\Width}%
\settoheight{\Height}{$5$}\settodepth{\Depth}{$5$}\setlength{\Height}{-0.5\Height}\setlength{\Depth}{0.5\Depth}\addtolength{\Height}{\Depth}%
\put(-0.1000000,5.0000000){\hspace*{\Width}\raisebox{\Height}{$5$}}%
%
\polyline(-2.00000,0.00000)(3.50000,0.00000)%
%
\polyline(0.00000,-0.50000)(0.00000,5.50000)%
%
\settowidth{\Width}{$x$}\setlength{\Width}{0\Width}%
\settoheight{\Height}{$x$}\settodepth{\Depth}{$x$}\setlength{\Height}{-0.5\Height}\setlength{\Depth}{0.5\Depth}\addtolength{\Height}{\Depth}%
\put(3.5500000,0.0000000){\hspace*{\Width}\raisebox{\Height}{$x$}}%
%
\settowidth{\Width}{$y$}\setlength{\Width}{-0.5\Width}%
\settoheight{\Height}{$y$}\settodepth{\Depth}{$y$}\setlength{\Height}{\Depth}%
\put(0.0000000,5.5500000){\hspace*{\Width}\raisebox{\Height}{$y$}}%
%
\settowidth{\Width}{O}\setlength{\Width}{-1\Width}%
\settoheight{\Height}{O}\settodepth{\Depth}{O}\setlength{\Height}{-\Height}%
\put(-0.0500000,-0.0500000){\hspace*{\Width}\raisebox{\Height}{O}}%
%
\end{picture}}%}}
\end{layer}

\begin{itemize}
\item
$y=x^2+2bx+c$\hfill{\color{blue}$\Longrightarrow (x+b)^2+d$の形に変形}
\item
[(例)] $y=x^2-2x+3$
\\ $\phantom{y}=(x^2-2x+1)-1+3$
\\$\phantom{y}=(x-1)^2+2$
\item
[(例)] $y=-x^2-4x+1$
\\ $\phantom{y}=-(x^2+4x)+1$
\\ $\phantom{y}=-\bigl((x+2)^2-4\bigr)+1$
\\$\phantom{y}=-(x+2)^2+5$
\end{itemize}

\newslide{課題(2次関数のグラフ)}

\vspace*{18mm}

%%repeat=9
\slidepage
\seteda{60}
\begin{itemize}
\item
[課題]\monban $a(x+b)^2+c$の形に変形せよ.\\
\eda{$y=x^2+4x-5$}\\
\eda{$y=x^2-2x-1$}\\
\eda{$y=-x^2-4x+1$}\\
\eda{$y=x^2+x+1$}
\end{itemize}
%%%%%%%%%%%%

%%%%%%%%%%%%%%%%%%%%

\mainslide{2次方程式}


\slidepage[m]
%%%%%%%%%%%%

%%%%%%%%%%%%%%%%%%%%

\newslide{2次式の因数分解}

\vspace*{18mm}

\slidepage
\begin{enumerate}[(1)]
\item
$x^2-a^2=(x+a)(x-a)$\\
\hspace*{2zw}$x^2-9$
$=x^2-3^2$
$=(x+3)(x-3)$
\item
$x^2+2ax+a^2=(x+a)^2$\\
\hspace*{2zw}$x^2+4x+4$
$=x^2+2\cdot 2+2^2$
$=(x+2)^2$
\item
$x^2+(a+b)x+ab=(x+a)(x+b)$\\
\hspace*{2zw}$x^2+5x+6$
$=(x+2)(x+3)$\\
\hspace*{2zw}$x^2-6x+8$
$=(x-2)(x-4)$
\end{enumerate}

\newslide{2次方程式(因数分解)}

\vspace*{18mm}

\slidepage
\begin{itemize}
\item
「$AB=0$ならば$A=0$または$B=0$」を用いる.\vspace{-2mm}
\item
[(例) ]$x^2-9=0$\\
\hspace*{1zw}$\Longleftrightarrow\ (x+3)(x-3)=0$\\
\hspace*{1zw}$\Longleftrightarrow\ x=-3\ \mbox{(または)}\ x= 3$\\
\hspace*{1zw}$\Longleftrightarrow\ x=\pm 3\ \ \mbox{\color{blue} と書く}$\vspace{-2mm}
\item
[課題]\monban 次の方程式を解け.\seteda{60}\\
\eda{$x^2-49=0$}\eda{$x^2-2x+1=0$}\\
\eda{$x^2-7x+12=0$}\eda{$x^2-x-20=0$}
\end{itemize}

\newslide{平方根}

\vspace*{18mm}

\slidepage
\begin{itemize}
\item
2乗して$4$になる数($x^2=4$となる$x$)
\item
[]\hspace*{2zw}$\Longrightarrow\ 2,\ -2$の2つがある.
\item
このうち,正の方の$2$を$\sqrt{4}$とかく
\item
正の数$a$について,2乗して$a$になる数のうち正の方をを$\sqrt{a}$とかく
\item
[]\hspace*{4zw}$(\sqrt{a})^2=a,\ (-\sqrt{a})^2=a$
\end{itemize}

\newslide{平方根の性質}

\vspace*{18mm}

\slidepage
\begin{itemize}
\item
$a>0$のとき,$\sqrt{a^2}=a$\\
\hspace*{2zw}2乗して$4^2(=16)$になるのは$4$と$-4$\\
\hspace*{2zw}正の方をとって,$\sqrt{4^2}=4$\vspace{-2mm}
\item
[]$a<0$のとき,$\sqrt{a^2}=-a$
\\\hspace*{2zw}2乗して$(-4)^2$になるのも$4$と$-4$
\\\hspace*{2zw}正の方をとって,$\sqrt{(-4)^2}=4$\vspace{-2mm}
\item
{\color{red}$\sqrt{a^2}=|a|$}\vspace{-2mm}
\item
$b>0$のとき,$\sqrt{a^2b}=|a|\sqrt{b}$
\end{itemize}

\newslide{課題 平方根}

\vspace*{18mm}

%%repeat=6
\slidepage
\begin{itemize}
\item
[課題]\monban 次の数を根号を用いないで表せ\seteda{55}\hfill TextP17\vspace{2mm}\\
\eda{$-\sqrt{64}$}\eda{$\sqrt{\bunsuu{4}{9}}$}\\
\eda{$\bigl(-\sqrt{11}\bigr)^2$}\eda{$-\bigl(-\sqrt{3}\bigr)^2$}\\
\item
[課題]\monban 次を計算せよ($\sqrt{\phantom{5}}$の中を簡単にせよ)\seteda{55}\\
\eda{$-\sqrt{12}$}\eda{$\sqrt{18}$}\\
\eda{$\sqrt{27}-\sqrt{3}$}\eda{$\sqrt{100}\sqrt{8}$}
\end{itemize}
%%%%%%%%%%%%

%%%%%%%%%%%%%%%%%%%%

\newslide{2次方程式(平方完成)}

\vspace*{18mm}

\slidepage

\begin{layer}{120}{0}
\putnotee{40}{10}{\normalsize\color{blue}$(x+a)^2=x^2+2ax+a^2$}
\end{layer}

\begin{itemize}
\item
平方完成\\
\hspace*{1zw}$x^2+6x+2=$
$(x^2+6x+9)-9+2=$
$(x+3)^2-7$\vspace{-2mm}
\item
2次方程式$x^2+6x+2=0$
\\\hspace*{5zw}$(x+3)^2-7=0$
\\\hspace*{5zw}$(x+3)^2=7$
\\\hspace*{5zw}$x+3=\sqrt{7},\ -\sqrt{7}$
\\\Ltab{5.5zw}{合わせて}$x+3=\pm \sqrt{7}$
\\\hspace*{5zw}$x=-3\pm \sqrt{7}$
\end{itemize}

\newslide{解の公式1}

\vspace*{18mm}

\slidepage
\begin{itemize}
\item
$x^2+2ax+b=0$
\\$(x+a)^2-a^2+b=0$
\\$(x+a)^2=a^2-b$
\\$x+a=\pm \sqrt{a^2-b}$
\vspace{2mm}\\よって \fbox{\color{red}$x=-a\pm \sqrt{a^2-b}$}
\item
[課題]\monban 次の2次方程式を解け.\seteda{57}\\
\eda{$x^2+4x+2=0$}\eda{$x^2+2x-2=0$}\\
\eda{$x^2-6x+1=0$}\eda{$x^2-8x+2=0$}
\end{itemize}

\newslide{解の公式}

\vspace*{18mm}

\slidepage
\begin{itemize}
\item
2次方程式$ax^2+bx+c=0$の解は\vspace{1mm}\\
\hspace*{3zw}\fbox{$x=\bunsuu{-b\pm\sqrt{b^2-4ac}}{2a}$}\vspace{-2mm}
\item
[]例)$2x^2-5x+1=0$\\
\hspace*{3zw}$x=\bunsuu{5\pm \sqrt{5^2-4\cdot 2\cdot1}}{2\cdot 2}$
$=\bunsuu{5\pm \sqrt{17}}{4}$\vspace{-2mm}
\item
[課題] \monban $ax^2+bx+c=0$より $x^2+\dfrac{b}{a}x+\dfrac{c}{a}=0$\\
 これを用いて上の公式を導け
\end{itemize}
\label{pageend}\mbox{}

\end{document}

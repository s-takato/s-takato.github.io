\documentclass[10pt,dvipdfmx]{jarticle}
\usepackage{amsmath,amssymb}
\usepackage{graphicx}
\usepackage{color}
\usepackage[dvipdfmx]{pict2e}
\usepackage{ketpic2e}
\usepackage{ketlayer2e}
\usepackage{ketlayermorewith2e}
\setmargin{15}{15}{15}{15}
\pagestyle{headings}
\begin{document}
\begin{center}
\verb|question0613e-07_08_09_10_11_12.txt|\\
\end{center}
$Q061307$\\
$a>0,b>0\text{とする}$\\
$[1]\;\text{指数法則}a^x\;a^y=\text{(    )}$\\
$[2]\;\text{指数法則}(a^x)^y=\text{(    )}$\\
$[3]\;\text{指数法則}(ab)^x=\text{(    )}$\\
$[4]\;a^{-3}\text{を分数で表せ}$\\
\text{Sheet} 
$[1]\;=\;\;::2$ 
$[2]\;=\;\;::2$ 
$[3]\;=\;\;::2$ 
$[4]\;=\;\;::2$ 
\\
\text{Ans}\\
$[1]\;a^{x+y}$\\
$[2]\;a^{xy}$\\
$[3]\;a^xb^x$\\
$[4]\;\dfrac{1}{a^3}$\\
$Q052308$\\
$\text{計算せよ}$\\
$[1]\;32^{\frac{1}{5}}::16^{\frac{3}{4}}$\\
$[2]\;\sqrt[4]{64}::\sqrt[3]{81}$\\
\text{Sheet} 
$[1]\;=\;\;::4$ 
$[2]\;=\;\;::4$ 
\\
\text{Ans}\\
$[1]\;2::8$\\
$[2]\;2\sqrt{2}::3\sqrt[3]{3}$\\
$Q061309$\\
$\text{方程式を解け(}x\text{を求めよ)}$\\
$[1]\;4^x=2::8^x=2$\\
$[2]\;3^{x+1}=\sqrt{3}::2^{x-1}=\sqrt{2}$\\
\text{Sheet} 
$[1]\;=\;\;::4$ 
$[2]\;=\;\;::4$ 
\\
\text{Ans}\\
$[1]\;x=\dfrac{1}{2}::x=\dfrac{1}{3}$\\
$[2]\;x=-\dfrac{1}{2}::x=\dfrac{3}{2}$\\
$Q061310$\\
$\text{問いに答えよ}$\\
$[1]\;y=\log_{a} x\;{\Longleftrightarrow}\;\text{(     )}$\\
$[2]\;\log_{2} 2\text{を求めよ}::\log_{3} 1\text{を求めよ}$\\
$[3]\;\log_{2} \dfrac{1}{2}\text{を求めよ}::\log_{3} \sqrt{3}\text{を求めよ}$\\
\text{Sheet} 
$[1]\;=\;\;::2::-1$ 
$[2]\;=\;\;::4$ 
$[3]\;=\;\;::4$ 
\\
\text{Ans}\\
$[1]\;a^y=x$\\
$[2]\;1::0$\\
$[3]\;-1::\dfrac{1}{2}$\\
$Q061311$\\
$a_n=\dfrac{n+1}{n},\;b_n=n^2\text{とする}$\\
$[1]\;a_1,a_2,a_3,a_4\text{を求めよ}::b_1,b_2,b_3,b_4\text{を求めよ.}$\\
$[2]\;\displaystyle\sum_{k=1}^{4} a_k\text{を求めよ}::\displaystyle\sum_{k=1}^{4} b_k\text{を求めよ.}$\\
\text{Sheet} 
$[1]\;=\;\;::4$ 
$[2]\;=\;\;::4$ 
\\
\text{Ans}\\
$[1]\;\dfrac{2}{1},\dfrac{3}{2},\dfrac{4}{3},\dfrac{5}{4}::1,4,9,16$\\
$[2]\;\displaystyle\sum_{k=1}^{4} (\dfrac{k+1}{k})=\dfrac{187}{60}::\displaystyle\sum_{k=1}^{4} k^2=30$\\
$Q061312$\\
$\text{問いに答えよ}$\\
$[1]\;(2+i)(1+2i)\text{を計算せよ}::(3+i)(1+3i)\text{を計算せよ}$\\
$[2]\;z=2i\text{の偏角をラジアンで求めよ}::z=-3\text{の偏角をラジアンで求めよ}$\\
\text{Sheet} 
$[1]\;=\;\;::4$ 
$[2]\;=\;\;::4$ 
\\
\text{Ans}\\
$[1]\;5i::10i$\\
$[2]\;\dfrac{{\pi}}{2}::{\pi}$\\
\newpage
\begin{center}
\verb|question0613e-01_02_03_04_05_06.txt|\\
\end{center}
$Q061301$\\
$a(x+b)^2+c\text{の形に変形せよ}$\\
$[1]\;y=x^2+2x+2::y=x^2-4x+6$\\
$[2]\;y=x^2-2x-1::y=x^2-2x-3$\\
\text{Sheet} 
$[1]\;y=\;\;::4$ 
$[2]\;y=\;\;::4$ 
\\
\text{Ans}\\
$[1]\;(x+1)^2+1::(x-2)^2+2$\\
$[2]\;(x-1)^2-2::(x-1)^2-4$\\
$Q061302$\\
$\text{次を}\sqrt{2},\sqrt{3}\text{だけの式で表せ.}$\\
$[1]\;\sqrt{8}::\sqrt{27}$\\
$[2]\;\sqrt{24}::\sqrt{18}$\\
\text{Sheet} 
$[1]\;=\;::4$ 
$[2]\;=\;::4$ 
\\
\text{Ans}\\
$[1]\;2\sqrt{2}::3\sqrt{3}$\\
$[2]\;2\sqrt{2}\sqrt{3}::3\sqrt{2}$\\
$Q061303$\\
${\theta}\text{は第}2\text{象限の角とする(}90^{\circ}<{\theta}<180^{\circ}\text{)}$\\
$[1]\;90^{\circ}<{\theta}<180^{\circ}\text{はラジアンでどのような不等式になるか}$\\
$[2]\;\sin {\theta} =\dfrac{1}{2}\text{のとき,}\cos {\theta} \text{を求めよ}::\sin {\theta} =\dfrac{1}{3}\text{のとき,}\cos {\theta} \text{を求めよ}$\\
\text{Sheet} 
$[1]\;\;::4::-1$ 
$[2]\;=\;::4$ 
\\
\text{Ans}\\
$[1]\;\dfrac{{\pi}}{2}<{\theta}<{\pi}$\\
$[2]\;-\dfrac{\sqrt{3}}{2}::-\dfrac{2\sqrt{2}}{3}$\\
$Q061304$\\
$\text{次に答えよ.}$\\
$[1]\;y=3\sin x \text{の振幅}::y=2\cos x \text{の振幅}$\\
$[2]\;y=2\sin 2x \text{の周期}::y=\sin \dfrac{x}{3} \text{の周期}$\\
\text{Sheet} 
$[1]\;=\;::4$ 
$[2]\;=\;::4$ 
\\
\text{Ans}\\
$[1]\;3::2$\\
$[2]\;{\pi}::6{\pi}$\\
$Q061305$\\
$y=\sin x \text{からの位相のずれを答えよ(左}+\text{,右}-\text{とせよ)}$\\
$[1]\;y=\sin (x-1) ::y=\sin (x-2) $\\
$[2]\;y=\sin (x+2) ::y=\sin (x+1) $\\
\text{Sheet} 
$[1]\;=\;::4::-1$ 
$[2]\;=\;::4::-1$ 
\\
\text{Ans}\\
$[1]\;-1::-2$\\
$[2]\;2::1$\\
$Q061306$\\
$\text{( )に当てはまる数式を答えよ}$\\
$[1]\;\text{加法定理}\;\sin (a+x) =\text{(      )}$\\
$[2]\;\cos a =\dfrac{4}{5},\;\sin a =\dfrac{3}{5}\;\text{を満たす角}\;a\text{をとると}//\text{  }\sin (a+x) =\text{(      )}$\\
$[3]\;3\cos x +4\sin x =\text{(      )}$\\
\text{Sheet} 
$[1]\;=\;::2::-1$ 
$[2]\;=\;::4::-1$ 
$[3]\;=\;::4::-1$ 
\\
\text{Ans}\\
$[1]\;\sin a \cos x +\cos a \sin x $\\
$[2]\;\dfrac{1}{5}(3\cos x +4\sin x )$\\
$[3]\;5\sin (x+a) $\\
\newpage
\end{document}

%%% タイトル presen209
\documentclass[landscape,10pt]{jarticle}
\usepackage{pict2e}
\special{papersize=\the\paperwidth,\the\paperheight}
\usepackage{ketpic,ketlayer}
\usepackage{ketslide}
\usepackage{amsmath,amssymb}
\usepackage{bm,enumerate}
\usepackage[dvipdfmx]{graphicx}
\usepackage{color}
\definecolor{slidecolora}{cmyk}{0.98,0.13,0,0.43}
\definecolor{slidecolorb}{cmyk}{0.2,0,0,0}
\definecolor{slidecolorc}{cmyk}{0.2,0,0,0}
\definecolor{slidecolord}{cmyk}{0.2,0,0,0}
\definecolor{slidecolore}{cmyk}{0,0,0,0.5}
\definecolor{slidecolorf}{cmyk}{0,0,0,0.5}
\definecolor{slidecolori}{cmyk}{0.98,0.13,0,0.43}
\def\setthin#1{\def\thin{#1}}
\setthin{0}
\newcommand{\slidepage}[1][s]{%
\setcounter{ketpicctra}{18}%
\if#1m \setcounter{ketpicctra}{1}\fi
\hypersetup{linkcolor=black}%

\begin{layer}{118}{0}
\putnotee{122}{-\theketpicctra.05}{\small\thepage/\pageref{pageend}}
\end{layer}\hypersetup{linkcolor=blue}

}
\usepackage{emath}
\usepackage[dvipdfmx,colorlinks=true,linkcolor=blue,filecolor=blue]{hyperref}
\newcommand{\hiduke}{0930}
\newcommand{\hako}[2][1]{\fbox{\raisebox{#1mm}{\mbox{}}\raisebox{-#1mm}{\mbox{}}\,\phantom{#2}\,}}
\newcommand{\hakoa}[2][1]{\fbox{\raisebox{#1mm}{\mbox{}}\raisebox{-#1mm}{\mbox{}}\,#2\,}}
\newcommand{\hakom}[2][1]{\hako[#1]{$#2$}}
\newcommand{\hakoma}[2][1]{\hakoa[#1]{$#2$}}
\def\rad{\;\mathrm{rad}}
\def\deg#1{#1^{\circ}}
\newcommand{\sbunsuu}[2]{\scalebox{0.6}{$\bunsuu{#1}{#2}$}}
\def\pow{$\hspace{-1.5mm}^\hspace{-1mm}$}
\def\dlim{\displaystyle\lim}
\newcommand\down[1][0.5zw]{\vspace{#1}\\}
\newcommand{\sfrac}[3][0.65]{\scalebox{#1}{$\frac{#2}{#3}$}}
\newcommand{\phn}[1]{\phantom{#1}}
\newcommand{\dsum}{\displaystyle\sum}
\def\pow{$\hspace{-1.5mm}^\hspace{-1mm}$}
\def\dlim{\displaystyle\lim}
\def\dint{\displaystyle\int}
\def\dsum{\displaystyle\sum}
\newcommand{\brd}[2][1]{\scalebox{#1}{\color{red}\fbox{\color{black}$#2$}}}
\newcommand{\scb}[2][0.6]{\scalebox{#1}{#2}}
\newcommand{\dpar}[2]{\bunsuu{\partial #1}{\partial #2}}

\setmargin{25}{145}{15}{100}

\ketslideinit

\pagestyle{empty}

\begin{document}

\begin{layer}{120}{0}
\putnotese{0}{0}{{\Large\bf
\color[cmyk]{1,1,0,0}

\begin{layer}{120}{0}
{\Huge \putnotes{65}{20}{2変数関数}}
\putnotes{65}{70}{2022.09.30}
\end{layer}

}
}
\end{layer}

\def\mainslidetitley{22}
\def\ketcletter{slidecolora}
\def\ketcbox{slidecolorb}
\def\ketdbox{slidecolorc}
\def\ketcframe{slidecolord}
\def\ketcshadow{slidecolore}
\def\ketdshadow{slidecolorf}
\def\slidetitlex{6}
\def\slidetitlesize{1.3}
\def\mketcletter{slidecolori}
\def\mketcbox{yellow}
\def\mketdbox{yellow}
\def\mketcframe{yellow}
\def\mslidetitlex{62}
\def\mslidetitlesize{2}

\color{black}
\large\bf\boldmath
\addtocounter{page}{-1}

\def\MARU{}
\renewcommand{\MARU}[1]{{\ooalign{\hfil$#1$\/\hfil\crcr\raise.167ex\hbox{\mathhexbox20D}}}}
\renewcommand{\slidepage}[1][s]{%
\setcounter{ketpicctra}{18}%
\if#1m \setcounter{ketpicctra}{1}\fi
\hypersetup{linkcolor=black}%
\begin{layer}{118}{0}
\putnotee{115}{-\theketpicctra.05}{\small\hiduke-\thepage/\pageref{pageend}}
\end{layer}\hypersetup{linkcolor=blue}
}
\newcounter{ban}
\setcounter{ban}{1}
\newcommand{\monban}[1][\hiduke]{%
#1-\theban\ %
\addtocounter{ban}{1}%
}
\newcommand{\monbannoadd}[1][\hiduke]{%
#1-\theban\ %
}
\newcommand{\addban}{%
\addtocounter{ban}{1}%
}
\newcounter{edawidth}
\newcounter{edactr}
\newcommand{\seteda}[1]{% 20220708 modified
\setcounter{edawidth}{#1}
\setcounter{edactr}{1}
}
\newcommand{\eda}[2][\theedawidth]{%
\Ltab{#1 mm}{[\theedactr]\ #2}%
\addtocounter{edactr}{1}%
}
%%%%%%%%%%%%%

%%%%%%%%%%%%%%%%%%%%

\mainslide{2変数関数}


\slidepage[m]
%%%%%%%%%%%%

%%%%%%%%%%%%%%%%%%%%

\newslide{1変数関数と2変数関数}

\vspace*{18mm}


\begin{layer}{120}{0}
\putnotew{96}{73}{\hyperlink{para0pg0}{\fbox{\Ctab{2.5mm}{\scalebox{1}{\scriptsize $\mathstrut||\!\lhd$}}}}}
\putnotew{101}{73}{\hyperlink{para1pg1}{\fbox{\Ctab{2.5mm}{\scalebox{1}{\scriptsize $\mathstrut|\!\lhd$}}}}}
\putnotew{108}{73}{\hyperlink{para1pg1}{\fbox{\Ctab{4.5mm}{\scalebox{1}{\scriptsize $\mathstrut\!\!\lhd\!\!$}}}}}
\putnotew{115}{73}{\hyperlink{para1pg2}{\fbox{\Ctab{4.5mm}{\scalebox{1}{\scriptsize $\mathstrut\!\rhd\!$}}}}}
\putnotew{120}{73}{\hyperlink{para1pg2}{\fbox{\Ctab{2.5mm}{\scalebox{1}{\scriptsize $\mathstrut \!\rhd\!\!|$}}}}}
\putnotew{125}{73}{\hyperlink{para2pg1}{\fbox{\Ctab{2.5mm}{\scalebox{1}{\scriptsize $\mathstrut \!\rhd\!\!||$}}}}}
\putnotee{126}{73}{\scriptsize\color{blue} 2/2}
\end{layer}

\slidepage
\begin{itemize}
\item
これまでの関数$y=f(x)$(1変数関数)\\
\hspace*{2zw}1つの値$x$を与えると,$y$の値が決まる\\
\hspace*{4zw}例)$y=x^2$
\item
2変数関数$z=f(x,\ y)$\\
\hspace*{2zw}2つの値$x,\ y$を与えると,$z$の値が決まる\\
\hspace*{4zw}例)$z=x^2+y^2$
\end{itemize}

\newslide{2変数関数のグラフ}

\vspace*{18mm}


\begin{layer}{120}{0}
\putnotew{96}{73}{\hyperlink{para1pg2}{\fbox{\Ctab{2.5mm}{\scalebox{1}{\scriptsize $\mathstrut||\!\lhd$}}}}}
\putnotew{101}{73}{\hyperlink{para2pg1}{\fbox{\Ctab{2.5mm}{\scalebox{1}{\scriptsize $\mathstrut|\!\lhd$}}}}}
\putnotew{108}{73}{\hyperlink{para2pg2}{\fbox{\Ctab{4.5mm}{\scalebox{1}{\scriptsize $\mathstrut\!\!\lhd\!\!$}}}}}
\putnotew{115}{73}{\hyperlink{para2pg3}{\fbox{\Ctab{4.5mm}{\scalebox{1}{\scriptsize $\mathstrut\!\rhd\!$}}}}}
\putnotew{120}{73}{\hyperlink{para2pg3}{\fbox{\Ctab{2.5mm}{\scalebox{1}{\scriptsize $\mathstrut \!\rhd\!\!|$}}}}}
\putnotew{125}{73}{\hyperlink{para3pg1}{\fbox{\Ctab{2.5mm}{\scalebox{1}{\scriptsize $\mathstrut \!\rhd\!\!||$}}}}}
\putnotee{126}{73}{\scriptsize\color{blue} 3/3}
\end{layer}

\slidepage

\begin{layer}{120}{0}
\putnotese{5}{25}{\scalebox{0.7}{{\unitlength=2cm%
\begin{picture}%
(4.00,4.00)(-2.00,-2.00)%
\special{pn 12}%
%
\special{pa 787 -0}\special{pa 787 -29}\special{pa 785 -58}\special{pa 783 -87}\special{pa 779 -116}%
\special{pa 774 -145}\special{pa 768 -174}\special{pa 761 -202}\special{pa 753 -231}%
\special{pa 744 -259}\special{pa 733 -287}\special{pa 722 -314}\special{pa 710 -341}%
\special{pa 696 -368}\special{pa 681 -395}\special{pa 666 -421}\special{pa 649 -446}%
\special{pa 631 -471}\special{pa 611 -496}\special{pa 591 -520}\special{pa 569 -544}%
\special{pa 552 -561}\special{pa 522 -590}\special{pa 496 -612}\special{pa 469 -633}%
\special{pa 439 -653}\special{pa 408 -673}\special{pa 386 -686}\special{pa 338 -711}%
\special{pa 314 -722}\special{pa 253 -746}\special{pa 198 -762}\special{pa 133 -776}%
\special{pa 81 -783}\special{pa 30 -787}%
\special{fp}%
\special{pa -30 -787}\special{pa -72 -784}\special{pa -124 -778}\special{pa -179 -767}%
\special{pa -238 -750}\special{pa -298 -729}\special{pa -338 -711}\special{pa -374 -693}%
\special{pa -408 -673}\special{pa -440 -653}\special{pa -469 -633}\special{pa -496 -612}%
\special{pa -522 -590}\special{pa -537 -576}\special{pa -569 -544}\special{pa -591 -520}%
\special{pa -611 -496}\special{pa -627 -477}\special{pa -649 -446}\special{pa -666 -421}%
\special{pa -681 -395}\special{pa -696 -368}\special{pa -710 -341}\special{pa -722 -314}%
\special{pa -733 -287}\special{pa -744 -259}\special{pa -753 -231}\special{pa -761 -202}%
\special{pa -768 -174}\special{pa -774 -145}\special{pa -779 -116}\special{pa -783 -87}%
\special{pa -785 -58}\special{pa -787 -29}\special{pa -787 -0}%
\special{fp}%
\special{pa -394 341}\special{pa -303 363}\special{pa -208 380}\special{pa -109 390}%
\special{pa -8 394}\special{pa 92 391}\special{pa 192 382}\special{pa 288 366}\special{pa 379 345}%
\special{pa 464 318}\special{pa 542 286}\special{pa 610 249}\special{pa 669 208}\special{pa 717 163}%
\special{pa 752 116}\special{pa 776 67}\special{pa 787 17}\special{pa 785 -34}\special{pa 769 -84}%
\special{pa 742 -132}\special{pa 702 -178}\special{pa 651 -222}\special{pa 589 -262}%
\special{pa 517 -297}\special{pa 437 -328}\special{pa 349 -353}\special{pa 256 -372}%
\special{pa 159 -386}\special{pa 59 -393}\special{pa 30 -393}%
\special{fp}%
\special{pa -30 -393}\special{pa -42 -393}\special{pa -142 -387}\special{pa -240 -375}%
\special{pa -334 -357}\special{pa -422 -332}\special{pa -504 -302}\special{pa -577 -268}%
\special{pa -641 -229}\special{pa -694 -186}\special{pa -736 -140}\special{pa -766 -92}%
\special{pa -783 -42}\special{pa -787 8}\special{pa -779 59}\special{pa -757 108}%
\special{pa -723 156}\special{pa -678 200}\special{pa -621 242}\special{pa -554 280}%
\special{pa -478 313}\special{pa -394 341}%
\special{fp}%
\special{pn 4}%
\special{pa 787 -682}\special{pa 617 -535}%
\special{fp}%
\special{pa 572 -496}\special{pa -787 682}%
\special{fp}%
\special{pa -1364 -394}\special{pa -785 -227}%
\special{fp}%
\special{pa -727 -210}\special{pa 1364 394}%
\special{fp}%
\special{pa 0 682}\special{pa 0 423}%
\special{fp}%
\special{pa 0 363}\special{pa 0 -1364}%
\special{fp}%
\special{pn 8}%
\settowidth{\Width}{$x$}\setlength{\Width}{-0.5\Width}%
\settoheight{\Height}{$x$}\settodepth{\Depth}{$x$}\setlength{\Height}{-0.5\Height}\setlength{\Depth}{0.5\Depth}\addtolength{\Height}{\Depth}%
\put(-1.0718,-0.9282){\hspace*{\Width}\raisebox{\Height}{$x$}}%
%
%
\settowidth{\Width}{$y$}\setlength{\Width}{-0.5\Width}%
\settoheight{\Height}{$y$}\settodepth{\Depth}{$y$}\setlength{\Height}{-0.5\Height}\setlength{\Depth}{0.5\Depth}\addtolength{\Height}{\Depth}%
\put(1.8233,-0.5263){\hspace*{\Width}\raisebox{\Height}{$y$}}%
%
%
\settowidth{\Width}{$z$}\setlength{\Width}{-0.5\Width}%
\settoheight{\Height}{$z$}\settodepth{\Depth}{$z$}\setlength{\Height}{-0.5\Height}\setlength{\Depth}{0.5\Depth}\addtolength{\Height}{\Depth}%
\put(0.0000,1.8271){\hspace*{\Width}\raisebox{\Height}{$z$}}%
%
%
\settowidth{\Width}{$z=\sqrt{4-x^2-y^2}$}\setlength{\Width}{0\Width}%
\settoheight{\Height}{$z=\sqrt{4-x^2-y^2}$}\settodepth{\Depth}{$z=\sqrt{4-x^2-y^2}$}\setlength{\Height}{\Depth}%
\put(0.5250,1.1250){\hspace*{\Width}\raisebox{\Height}{$z=\sqrt{4-x^2-y^2}$}}%
%
%
\end{picture}}%}}
\putnotese{75}{15}{\scalebox{0.7}{{\unitlength=1cm%
\begin{picture}%
(6.00,8.00)(-3.00,-2.00)%
\special{pn 12}%
%
\special{pa -65 19}\special{pa -47 24}\special{pa -31 26}\special{pa -16 28}\special{pa -1 28}%
\special{pa 13 28}\special{pa 28 27}\special{pa 44 24}\special{pa 60 21}\special{pa 83 13}%
\special{pa 108 3}\special{pa 135 -12}\special{pa 156 -25}\special{pa 175 -39}\special{pa 194 -54}%
\special{pa 212 -71}\special{pa 231 -88}\special{pa 248 -107}\special{pa 266 -127}%
\special{pa 283 -148}\special{pa 301 -170}\special{pa 318 -194}\special{pa 335 -218}%
\special{pa 352 -244}\special{pa 368 -270}\special{pa 385 -298}\special{pa 402 -327}%
\special{pa 419 -357}\special{pa 435 -388}\special{pa 452 -421}\special{pa 468 -454}%
\special{pa 485 -489}\special{pa 501 -525}\special{pa 518 -562}\special{pa 534 -600}%
\special{pa 551 -639}\special{pa 567 -679}\special{pa 584 -721}\special{pa 600 -763}%
\special{pa 616 -807}\special{pa 633 -852}\special{pa 649 -898}\special{pa 658 -924}%
\special{pa 665 -945}\special{pa 682 -993}\special{pa 698 -1043}\special{pa 714 -1093}%
\special{pa 730 -1145}\special{pa 747 -1198}\special{pa 763 -1252}\special{pa 779 -1307}%
\special{fp}%
\special{pa -779 -1307}\special{pa -763 -1252}\special{pa -754 -1222}\special{pa -747 -1198}%
\special{pa -730 -1145}\special{pa -714 -1093}\special{pa -698 -1043}\special{pa -682 -993}%
\special{pa -665 -945}\special{pa -649 -898}\special{pa -633 -852}\special{pa -616 -807}%
\special{pa -600 -763}\special{pa -584 -721}\special{pa -567 -679}\special{pa -551 -639}%
\special{pa -534 -600}\special{pa -518 -562}\special{pa -501 -525}\special{pa -485 -489}%
\special{pa -468 -454}\special{pa -452 -421}\special{pa -435 -388}\special{pa -419 -357}%
\special{pa -402 -327}\special{pa -385 -298}\special{pa -368 -270}\special{pa -352 -244}%
\special{pa -335 -218}\special{pa -318 -194}\special{pa -301 -170}\special{pa -283 -148}%
\special{pa -266 -127}\special{pa -248 -107}\special{pa -231 -88}\special{pa -212 -71}%
\special{pa -194 -54}\special{pa -175 -39}\special{pa -156 -25}\special{pa -145 -18}%
\special{pa -135 -12}\special{pa -114 0}\special{pa -89 11}\special{pa -65 19}%
\special{fp}%
\special{pa -394 -1023}\special{pa -303 -1000}\special{pa -208 -984}\special{pa -109 -974}%
\special{pa -8 -970}\special{pa 92 -973}\special{pa 192 -982}\special{pa 288 -997}%
\special{pa 379 -1019}\special{pa 464 -1046}\special{pa 542 -1078}\special{pa 610 -1115}%
\special{pa 669 -1156}\special{pa 717 -1201}\special{pa 752 -1248}\special{pa 776 -1297}%
\special{pa 787 -1347}\special{pa 785 -1397}\special{pa 769 -1447}\special{pa 742 -1496}%
\special{pa 702 -1542}\special{pa 651 -1586}\special{pa 589 -1625}\special{pa 517 -1661}%
\special{pa 437 -1691}\special{pa 349 -1717}\special{pa 256 -1736}\special{pa 159 -1749}%
\special{pa 59 -1756}\special{pa 30 -1757}%
\special{fp}%
\special{pa -30 -1757}\special{pa -42 -1757}\special{pa -142 -1751}\special{pa -240 -1739}%
\special{pa -334 -1720}\special{pa -422 -1696}\special{pa -504 -1666}\special{pa -577 -1632}%
\special{pa -641 -1592}\special{pa -694 -1550}\special{pa -736 -1504}\special{pa -766 -1456}%
\special{pa -783 -1406}\special{pa -787 -1355}\special{pa -779 -1305}\special{pa -757 -1256}%
\special{pa -723 -1208}\special{pa -678 -1163}\special{pa -621 -1122}\special{pa -554 -1084}%
\special{pa -478 -1051}\special{pa -394 -1023}%
\special{fp}%
\special{pn 4}%
\special{pa 591 -511}\special{pa 464 -402}%
\special{fp}%
\special{pa 384 -333}\special{pa -1 1}%
\special{fp}%
\special{pa -60 52}\special{pa -591 511}%
\special{fp}%
\special{pa -1023 -295}\special{pa -245 -71}%
\special{fp}%
\special{pa -148 -43}\special{pa 22 6}%
\special{fp}%
\special{pa 109 32}\special{pa 1023 295}%
\special{fp}%
\special{pa 0 341}\special{pa 0 58}%
\special{fp}%
\special{pa 0 -2}\special{pa 0 -940}%
\special{fp}%
\special{pa 0 -1001}\special{pa 0 -2046}%
\special{fp}%
\special{pn 8}%
\settowidth{\Width}{$x$}\setlength{\Width}{-0.5\Width}%
\settoheight{\Height}{$x$}\settodepth{\Depth}{$x$}\setlength{\Height}{-0.5\Height}\setlength{\Depth}{0.5\Depth}\addtolength{\Height}{\Depth}%
\put(-1.6436,-1.4234){\hspace*{\Width}\raisebox{\Height}{$x$}}%
%
%
\settowidth{\Width}{$y$}\setlength{\Width}{-0.5\Width}%
\settoheight{\Height}{$y$}\settodepth{\Depth}{$y$}\setlength{\Height}{-0.5\Height}\setlength{\Depth}{0.5\Depth}\addtolength{\Height}{\Depth}%
\put(2.7806,-0.8027){\hspace*{\Width}\raisebox{\Height}{$y$}}%
%
%
\settowidth{\Width}{$z$}\setlength{\Width}{-0.5\Width}%
\settoheight{\Height}{$z$}\settodepth{\Depth}{$z$}\setlength{\Height}{-0.5\Height}\setlength{\Depth}{0.5\Depth}\addtolength{\Height}{\Depth}%
\put(0.0000,5.3862){\hspace*{\Width}\raisebox{\Height}{$z$}}%
%
%
\settowidth{\Width}{$z=x^2+y^2$}\setlength{\Width}{0\Width}%
\settoheight{\Height}{$z=x^2+y^2$}\settodepth{\Depth}{$z=x^2+y^2$}\setlength{\Height}{\Depth}%
\put(0.3000,4.5500){\hspace*{\Width}\raisebox{\Height}{$z=x^2+y^2$}}%
%
%
\end{picture}}%}}
\end{layer}

\begin{itemize}
\item
1変数関数のグラフは曲線
\item
2変数関数のグラフは曲面になる
\end{itemize}

\newslide{2変数関数のグラフ(課題)}

\vspace*{18mm}

\slidepage

\begin{layer}{110}{0}
\putnotes{27}{28}{\scalebox{0.7}{%%% /Users/hama/Desktop/dntbiseki2-1107/zu2_cindy/fig/c2s11_hanenchu-x_k.tex 
%%% Generator=c2s11_hanenchu-x_k.cdy 
{\unitlength=1cm%
\begin{picture}%
(5,3.25)(-2.5,-1.25)%
\special{pn 8}%
%
\small%
\settowidth{\Width}{$x$}\setlength{\Width}{-0.5\Width}%
\settoheight{\Height}{$x$}\settodepth{\Depth}{$x$}\setlength{\Height}{-0.5\Height}\setlength{\Depth}{0.5\Depth}\addtolength{\Height}{\Depth}%
\put(-0.8900000,-0.7700000){\hspace*{\Width}\raisebox{\Height}{$x$}}%
%
\settowidth{\Width}{$y$}\setlength{\Width}{-0.5\Width}%
\settoheight{\Height}{$y$}\settodepth{\Depth}{$y$}\setlength{\Height}{-0.5\Height}\setlength{\Depth}{0.5\Depth}\addtolength{\Height}{\Depth}%
\put(1.9100000,-0.5500000){\hspace*{\Width}\raisebox{\Height}{$y$}}%
%
\settowidth{\Width}{$z$}\setlength{\Width}{-0.5\Width}%
\settoheight{\Height}{$z$}\settodepth{\Depth}{$z$}\setlength{\Height}{-0.5\Height}\setlength{\Depth}{0.5\Depth}\addtolength{\Height}{\Depth}%
\put(0.0000000,1.4900000){\hspace*{\Width}\raisebox{\Height}{$z$}}%
%
\settowidth{\Width}{O}\setlength{\Width}{0\Width}%
\settoheight{\Height}{O}\settodepth{\Depth}{O}\setlength{\Height}{-\Height}%
\put(0.0500000,-0.1500000){\hspace*{\Width}\raisebox{\Height}{O}}%
%
\special{pa  -197   170}\special{pa  -295   256}%
\special{fp}%
\special{pa   518   150}\special{pa   682   197}%
\special{fp}%
\special{pa     0  -341}\special{pa     0  -410}\special{pa     0  -511}%
\special{fp}%
\special{pn 8}%
\special{pa -576 -577}\special{pa -569 -573}\special{fp}\special{pa -542 -558}\special{pa -536 -552}\special{fp}%
\special{pa -518 -527}\special{pa -514 -520}\special{fp}\special{pa -502 -491}\special{pa -500 -483}\special{fp}%
\special{pa -492 -453}\special{pa -491 -445}\special{fp}\special{pa -487 -414}\special{pa -486 -406}\special{fp}%
\special{pa -487 -374}\special{pa -483 -368}\special{fp}\special{pa -454 -359}\special{pa -447 -356}\special{fp}%
\special{pa -417 -348}\special{pa -409 -345}\special{fp}\special{pa -379 -337}\special{pa -371 -334}\special{fp}%
\special{pa -341 -326}\special{pa -334 -324}\special{fp}\special{pa -303 -315}\special{pa -296 -313}\special{fp}%
\special{pa -266 -304}\special{pa -258 -302}\special{fp}\special{pa -228 -293}\special{pa -220 -291}\special{fp}%
\special{pa -190 -282}\special{pa -183 -280}\special{fp}\special{pa -152 -271}\special{pa -145 -269}\special{fp}%
\special{pa -115 -260}\special{pa -107 -258}\special{fp}\special{pa -77 -250}\special{pa -69 -247}\special{fp}%
\special{pa -39 -239}\special{pa -32 -236}\special{fp}\special{pa -2 -228}\special{pa 6 -226}\special{fp}%
\special{pa 36 -217}\special{pa 44 -215}\special{fp}\special{pa 74 -206}\special{pa 82 -204}\special{fp}%
\special{pa 112 -195}\special{pa 119 -193}\special{fp}\special{pa 149 -184}\special{pa 157 -182}\special{fp}%
\special{pa 187 -173}\special{pa 195 -171}\special{fp}\special{pa 225 -162}\special{pa 233 -160}\special{fp}%
\special{pa 263 -151}\special{pa 270 -149}\special{fp}\special{pa 300 -141}\special{pa 308 -138}\special{fp}%
\special{pa 338 -130}\special{pa 346 -127}\special{fp}\special{pa 376 -119}\special{pa 384 -117}\special{fp}%
\special{pa 414 -108}\special{pa 421 -106}\special{fp}\special{pa 451 -97}\special{pa 459 -95}\special{fp}%
\special{pa 489 -86}\special{pa 497 -84}\special{fp}\special{pa 527 -75}\special{pa 535 -73}\special{fp}%
\special{pa 565 -64}\special{pa 572 -62}\special{fp}\special{pa 602 -53}\special{pa 610 -51}\special{fp}%
\special{pn 8}%
{%
\color[cmyk]{1,0,0,0}%
\special{pa  -538    72}\special{pa  -537    50}\special{pa  -536    28}\special{pa  -534     5}%
\special{pa  -532   -18}\special{pa  -528   -42}\special{pa  -524   -65}\special{pa  -519   -89}%
\special{pa  -513  -113}\special{pa  -507  -137}\special{pa  -500  -161}\special{pa  -493  -184}%
\special{pa  -484  -208}\special{pa  -476  -230}\special{pa  -466  -252}\special{pa  -457  -274}%
\special{pa  -446  -295}\special{pa  -436  -315}\special{pa  -425  -334}\special{pa  -413  -353}%
\special{pa  -402  -370}\special{pa  -390  -386}\special{pa  -378  -401}\special{pa  -366  -415}%
\special{pa  -353  -428}\special{pa  -341  -439}\special{pa  -329  -449}\special{pa  -316  -458}%
\special{pa  -304  -465}\special{pa  -292  -471}\special{pa  -280  -475}\special{pa  -268  -478}%
\special{pa  -257  -480}\special{pa  -247  -479}\special{pa  -235  -478}\special{pa  -232  -477}%
\special{fp}%
}%
{%
\color[cmyk]{1,0,0,0}%
}%
{%
\color[cmyk]{1,0,0,0}%
\special{pa  -197   170}\special{pa  -196   149}\special{pa  -195   126}\special{pa  -193   104}%
\special{pa  -191    80}\special{pa  -187    57}\special{pa  -183    33}\special{pa  -178     9}%
\special{pa  -173   -15}\special{pa  -166   -39}\special{pa  -159   -62}\special{pa  -152   -86}%
\special{pa  -143  -109}\special{pa  -135  -132}\special{pa  -125  -154}\special{pa  -116  -176}%
\special{pa  -105  -197}\special{pa   -95  -217}\special{pa   -84  -236}\special{pa   -72  -254}%
\special{pa   -61  -272}\special{pa   -49  -288}\special{pa   -37  -303}\special{pa   -25  -317}%
\special{pa   -12  -330}\special{pa     0  -341}\special{pa    12  -351}\special{pa    25  -360}%
\special{pa    37  -367}\special{pa    49  -373}\special{pa    61  -377}\special{pa    72  -380}%
\special{pa    84  -381}\special{pa    94  -381}\special{pa    95  -381}\special{pa   105  -379}%
\special{pa   110  -378}%
\special{fp}%
}%
{%
\color[cmyk]{1,0,0,0}%
}%
{%
\color[cmyk]{1,0,0,0}%
\special{pa   144   269}\special{pa   144   247}\special{pa   146   225}\special{pa   148   202}%
\special{pa   150   179}\special{pa   154   155}\special{pa   158   131}\special{pa   163   108}%
\special{pa   168    84}\special{pa   175    60}\special{pa   182    36}\special{pa   189    12}%
\special{pa   197   -11}\special{pa   206   -33}\special{pa   215   -56}\special{pa   225   -77}%
\special{pa   235   -98}\special{pa   246  -118}\special{pa   257  -137}\special{pa   268  -156}%
\special{pa   280  -173}\special{pa   292  -189}\special{pa   304  -205}\special{pa   316  -218}%
\special{pa   329  -231}\special{pa   341  -243}\special{pa   353  -253}\special{pa   366  -261}%
\special{pa   378  -268}\special{pa   390  -274}\special{pa   402  -279}\special{pa   413  -281}%
\special{pa   425  -283}\special{pa   435  -282}\special{pa   446  -281}\special{pa   450  -280}%
\special{fp}%
}%
{%
\color[cmyk]{1,0,0,0}%
}%
{%
\color[cmyk]{1,0,0,0}%
\special{pa  -821  -317}\special{pa  -794  -310}\special{pa  -767  -302}\special{pa  -739  -294}%
\special{pa  -712  -286}\special{pa  -685  -278}\special{pa  -657  -270}\special{pa  -630  -262}%
\special{pa  -603  -254}\special{pa  -576  -247}\special{pa  -548  -239}\special{pa  -521  -231}%
\special{pa  -494  -223}\special{pa  -467  -215}\special{pa  -439  -207}\special{pa  -412  -199}%
\special{pa  -385  -191}\special{pa  -357  -184}\special{pa  -330  -176}\special{pa  -303  -168}%
\special{pa  -276  -160}\special{pa  -248  -152}\special{pa  -221  -144}\special{pa  -194  -136}%
\special{pa  -166  -128}\special{pa  -139  -121}\special{pa  -112  -113}\special{pa   -85  -105}%
\special{pa   -57   -97}\special{pa   -30   -89}\special{pa    -3   -81}\special{pa    24   -73}%
\special{pa    52   -65}\special{pa    79   -58}\special{pa   106   -50}\special{pa   134   -42}%
\special{pa   161   -34}\special{pa   188   -26}\special{pa   215   -18}\special{pa   243   -10}%
\special{pa   270    -2}\special{pa   297     5}\special{pa   325    13}\special{pa   352    21}%
\special{pa   379    29}\special{pa   406    37}\special{pa   434    45}\special{pa   461    53}%
\special{pa   488    61}\special{pa   515    68}\special{pa   543    76}%
\special{fp}%
}%
{%
\color[cmyk]{1,0,0,0}%
}%
{%
\color[cmyk]{1,0,0,0}%
\special{pa  -682  -538}\special{pa  -655  -530}\special{pa  -627  -522}\special{pa  -600  -514}%
\special{pa  -573  -506}\special{pa  -546  -498}\special{pa  -518  -491}\special{pa  -491  -483}%
\special{pa  -464  -475}\special{pa  -436  -467}\special{pa  -409  -459}\special{pa  -382  -451}%
\special{pa  -355  -443}\special{pa  -327  -435}\special{pa  -300  -428}\special{pa  -273  -420}%
\special{pa  -245  -412}\special{pa  -218  -404}\special{pa  -191  -396}\special{pa  -164  -388}%
\special{pa  -136  -380}\special{pa  -109  -372}\special{pa   -82  -365}\special{pa   -55  -357}%
\special{pa   -27  -349}\special{pa     0  -341}\special{pa    27  -333}\special{pa    55  -325}%
\special{pa    82  -317}\special{pa   109  -309}\special{pa   136  -302}\special{pa   164  -294}%
\special{pa   191  -286}\special{pa   218  -278}\special{pa   245  -270}\special{pa   273  -262}%
\special{pa   300  -254}\special{pa   327  -246}\special{pa   355  -239}\special{pa   382  -231}%
\special{pa   409  -223}\special{pa   436  -215}\special{pa   464  -207}\special{pa   491  -199}%
\special{pa   518  -191}\special{pa   546  -183}\special{pa   573  -176}\special{pa   600  -168}%
\special{pa   627  -160}\special{pa   655  -152}\special{pa   682  -144}%
\special{fp}%
}%
{%
\color[cmyk]{1,0,0,0}%
}%
{%
\color[cmyk]{1,0,0,0}%
\special{pa   757  -183}\special{pa   767  -181}\special{pa   794  -173}\special{pa   821  -165}%
\special{fp}%
}%
{%
\color[cmyk]{1,0,0,0}%
}%
{%
\color[cmyk]{1,0,0,0}%
\special{pn 8}%
\special{pa -235 -479}\special{pa -228 -474}\special{fp}\special{pa -201 -459}\special{pa -195 -454}\special{fp}%
\special{pa -176 -428}\special{pa -172 -421}\special{fp}\special{pa -161 -391}\special{pa -158 -383}\special{fp}%
\special{pa -151 -352}\special{pa -150 -345}\special{fp}\special{pa -146 -313}\special{pa -145 -305}\special{fp}%
\special{pa -144 -273}\special{pa -144 -265}\special{fp}\special{pn 8}%
}%
{%
\color[cmyk]{1,0,0,0}%
}%
{%
\color[cmyk]{1,0,0,0}%
\special{pn 8}%
\special{pa 106 -380}\special{pa 113 -376}\special{fp}\special{pa 140 -360}\special{pa 146 -355}\special{fp}%
\special{pa 165 -329}\special{pa 169 -322}\special{fp}\special{pa 180 -292}\special{pa 183 -285}\special{fp}%
\special{pa 190 -254}\special{pa 191 -246}\special{fp}\special{pa 195 -214}\special{pa 196 -206}\special{fp}%
\special{pa 197 -174}\special{pa 197 -166}\special{fp}\special{pn 8}%
}%
{%
\color[cmyk]{1,0,0,0}%
}%
{%
\color[cmyk]{1,0,0,0}%
\special{pn 8}%
\special{pa 447 -282}\special{pa 454 -278}\special{fp}\special{pa 481 -262}\special{pa 487 -257}\special{fp}%
\special{pa 506 -231}\special{pa 509 -224}\special{fp}\special{pa 521 -194}\special{pa 524 -187}\special{fp}%
\special{pa 531 -156}\special{pa 532 -148}\special{fp}\special{pa 536 -116}\special{pa 537 -108}\special{fp}%
\special{pa 538 -76}\special{pa 538 -68}\special{fp}\special{pn 8}%
}%
{%
\color[cmyk]{1,0,0,0}%
}%
{%
\color[cmyk]{1,0,0,0}%
\special{pn 8}%
\special{pa -547 -560}\special{pa -539 -557}\special{fp}\special{pa -508 -549}\special{pa -501 -546}\special{fp}%
\special{pa -470 -538}\special{pa -462 -535}\special{fp}\special{pa -432 -526}\special{pa -424 -524}\special{fp}%
\special{pa -394 -515}\special{pa -386 -513}\special{fp}\special{pa -355 -504}\special{pa -348 -502}\special{fp}%
\special{pa -317 -493}\special{pa -309 -491}\special{fp}\special{pa -279 -482}\special{pa -271 -480}\special{fp}%
\special{pa -241 -471}\special{pa -233 -469}\special{fp}\special{pa -202 -460}\special{pa -195 -458}\special{fp}%
\special{pa -164 -449}\special{pa -157 -447}\special{fp}\special{pa -126 -438}\special{pa -118 -436}\special{fp}%
\special{pa -88 -427}\special{pa -80 -425}\special{fp}\special{pa -50 -416}\special{pa -42 -414}\special{fp}%
\special{pa -11 -405}\special{pa -4 -403}\special{fp}\special{pa 27 -394}\special{pa 35 -392}\special{fp}%
\special{pa 65 -383}\special{pa 73 -381}\special{fp}\special{pa 103 -372}\special{pa 111 -370}\special{fp}%
\special{pa 142 -361}\special{pa 149 -359}\special{fp}\special{pa 180 -350}\special{pa 188 -348}\special{fp}%
\special{pa 218 -339}\special{pa 226 -337}\special{fp}\special{pa 256 -328}\special{pa 264 -326}\special{fp}%
\special{pa 295 -317}\special{pa 302 -315}\special{fp}\special{pa 333 -306}\special{pa 340 -304}\special{fp}%
\special{pa 371 -295}\special{pa 379 -293}\special{fp}\special{pa 409 -284}\special{pa 417 -281}\special{fp}%
\special{pa 447 -273}\special{pa 455 -270}\special{fp}\special{pa 486 -262}\special{pa 493 -259}\special{fp}%
\special{pa 524 -251}\special{pa 532 -248}\special{fp}\special{pa 562 -240}\special{pa 570 -237}\special{fp}%
\special{pa 600 -229}\special{pa 608 -226}\special{fp}\special{pa 639 -217}\special{pa 646 -215}\special{fp}%
\special{pa 677 -206}\special{pa 685 -204}\special{fp}\special{pa 715 -195}\special{pa 723 -193}\special{fp}%
\special{pa 753 -184}\special{pa 761 -182}\special{fp}\special{pn 8}%
}%
{%
\color[cmyk]{1,0,0,0}%
}%
\special{pn 8}%
\special{pa 298 -258}\special{pa 292 -253}\special{fp}\special{pa 269 -233}\special{pa 263 -228}\special{fp}%
\special{pa 240 -208}\special{pa 234 -203}\special{fp}\special{pa 211 -183}\special{pa 205 -178}\special{fp}%
\special{pa 183 -158}\special{pa 176 -153}\special{fp}\special{pa 154 -133}\special{pa 148 -128}\special{fp}%
\special{pa 125 -108}\special{pa 119 -103}\special{fp}\special{pa 96 -83}\special{pa 90 -78}\special{fp}%
\special{pa 67 -58}\special{pa 61 -53}\special{fp}\special{pa 38 -33}\special{pa 32 -27}\special{fp}%
\special{pa 9 -8}\special{pa 3 -2}\special{fp}\special{pa -20 17}\special{pa -26 23}\special{fp}%
\special{pa -49 43}\special{pa -55 48}\special{fp}\special{pa -78 68}\special{pa -84 73}\special{fp}%
\special{pa -107 93}\special{pa -113 98}\special{fp}\special{pa -136 118}\special{pa -142 123}\special{fp}%
\special{pa -165 143}\special{pa -171 148}\special{fp}\special{pa -194 168}\special{pa -200 173}\special{fp}%
\special{pn 8}%
\special{pa -686 -198}\special{pa -678 -196}\special{fp}\special{pa -648 -187}\special{pa -641 -185}\special{fp}%
\special{pa -611 -176}\special{pa -603 -174}\special{fp}\special{pa -573 -165}\special{pa -566 -163}\special{fp}%
\special{pa -536 -155}\special{pa -528 -152}\special{fp}\special{pa -498 -144}\special{pa -491 -142}\special{fp}%
\special{pa -461 -133}\special{pa -453 -131}\special{fp}\special{pa -423 -122}\special{pa -416 -120}\special{fp}%
\special{pa -386 -111}\special{pa -378 -109}\special{fp}\special{pa -348 -101}\special{pa -341 -98}\special{fp}%
\special{pa -311 -90}\special{pa -303 -87}\special{fp}\special{pa -273 -79}\special{pa -266 -77}\special{fp}%
\special{pa -236 -68}\special{pa -228 -66}\special{fp}\special{pa -198 -57}\special{pa -191 -55}\special{fp}%
\special{pa -161 -46}\special{pa -153 -44}\special{fp}\special{pa -123 -36}\special{pa -116 -33}\special{fp}%
\special{pa -86 -25}\special{pa -78 -23}\special{fp}\special{pa -48 -14}\special{pa -41 -12}\special{fp}%
\special{pa -11 -3}\special{pa -3 -1}\special{fp}\special{pa 27 8}\special{pa 34 10}\special{fp}%
\special{pa 64 19}\special{pa 72 21}\special{fp}\special{pa 102 29}\special{pa 110 32}\special{fp}%
\special{pa 139 40}\special{pa 147 42}\special{fp}\special{pa 177 51}\special{pa 185 53}\special{fp}%
\special{pa 214 62}\special{pa 222 64}\special{fp}\special{pa 252 73}\special{pa 260 75}\special{fp}%
\special{pa 289 84}\special{pa 297 86}\special{fp}\special{pa 327 94}\special{pa 335 97}\special{fp}%
\special{pa 364 105}\special{pa 372 107}\special{fp}\special{pa 402 116}\special{pa 410 118}\special{fp}%
\special{pa 439 127}\special{pa 447 129}\special{fp}\special{pa 477 138}\special{pa 485 140}\special{fp}%
\special{pa 514 148}\special{pa 522 151}\special{fp}\special{pn 8}%
\special{pa 0 174}\special{pa 0 166}\special{fp}\special{pa 0 135}\special{pa 0 127}\special{fp}%
\special{pa 0 96}\special{pa 0 88}\special{fp}\special{pa 0 56}\special{pa 0 48}\special{fp}%
\special{pa 0 17}\special{pa 0 9}\special{fp}\special{pa 0 -22}\special{pa 0 -30}\special{fp}%
\special{pa 0 -62}\special{pa 0 -70}\special{fp}\special{pa 0 -101}\special{pa 0 -109}\special{fp}%
\special{pa 0 -140}\special{pa 0 -148}\special{fp}\special{pa 0 -180}\special{pa 0 -188}\special{fp}%
\special{pa 0 -219}\special{pa 0 -227}\special{fp}\special{pa 0 -258}\special{pa 0 -266}\special{fp}%
\special{pa 0 -298}\special{pa 0 -306}\special{fp}\special{pa 0 -337}\special{pa 0 -345}\special{fp}%
\special{pn 8}%
\special{pa   791  -181}\special{pa   764  -189}\special{pa   737  -197}\special{pa   709  -205}%
\special{pa   682  -213}\special{pa   655  -221}\special{pa   627  -229}\special{pa   600  -237}%
\special{pa   573  -244}\special{pa   546  -252}\special{pa   518  -260}\special{pa   491  -268}%
\special{pa   464  -276}\special{pa   437  -284}\special{pa   409  -292}\special{pa   382  -300}%
\special{pa   355  -307}\special{pa   327  -315}\special{pa   300  -323}\special{pa   273  -331}%
\special{pa   246  -339}\special{pa   218  -347}\special{pa   191  -355}\special{pa   164  -363}%
\special{pa   136  -370}\special{pa   109  -378}\special{pa    82  -386}\special{pa    55  -394}%
\special{pa    30  -401}%
\special{fp}%
\special{pa   -30  -418}\special{pa   -54  -425}\special{pa   -82  -433}\special{pa  -109  -441}%
\special{pa  -136  -449}\special{pa  -164  -457}\special{pa  -191  -465}\special{pa  -218  -473}%
\special{pa  -245  -481}\special{pa  -273  -488}\special{pa  -300  -496}\special{pa  -327  -504}%
\special{pa  -355  -512}\special{pa  -382  -520}\special{pa  -409  -528}\special{pa  -436  -536}%
\special{pa  -464  -544}\special{pa  -491  -551}\special{pa  -518  -559}\special{pa  -545  -567}%
\special{pa  -566  -573}\special{pa  -573  -575}%
\special{fp}%
\special{pa  -879   -26}\special{pa  -851   -18}\special{pa  -824   -11}\special{pa  -797    -3}%
\special{pa  -770     5}\special{pa  -742    13}\special{pa  -715    21}\special{pa  -688    29}%
\special{pa  -661    37}\special{pa  -633    44}\special{pa  -606    52}\special{pa  -579    60}%
\special{pa  -551    68}\special{pa  -524    76}\special{pa  -497    84}\special{pa  -470    92}%
\special{pa  -442   100}\special{pa  -415   107}\special{pa  -388   115}\special{pa  -361   123}%
\special{pa  -333   131}\special{pa  -306   139}\special{pa  -279   147}\special{pa  -251   155}%
\special{pa  -224   163}\special{pa  -197   170}\special{pa  -170   178}\special{pa  -142   186}%
\special{pa  -115   194}\special{pa   -88   202}\special{pa   -60   210}\special{pa   -33   218}%
\special{pa    -6   226}\special{pa    21   233}\special{pa    49   241}\special{pa    76   249}%
\special{pa   103   257}\special{pa   130   265}\special{pa   158   273}\special{pa   185   281}%
\special{pa   212   289}\special{pa   240   296}\special{pa   267   304}\special{pa   294   312}%
\special{pa   321   320}\special{pa   349   328}\special{pa   376   336}\special{pa   403   344}%
\special{pa   431   352}\special{pa   458   359}\special{pa   485   367}\special{pa   485   346}%
\special{pa   487   323}\special{pa   489   300}\special{pa   491   277}\special{pa   495   254}%
\special{pa   499   230}\special{pa   504   206}\special{pa   509   182}\special{pa   516   158}%
\special{pa   523   134}\special{pa   530   111}\special{pa   538    88}\special{pa   547    65}%
\special{pa   556    43}\special{pa   566    21}\special{pa   576     0}\special{pa   587   -20}%
\special{pa   598   -39}\special{pa   606   -52}\special{pa   609   -57}\special{pa   621   -75}%
\special{pa   633   -91}\special{pa   645  -106}\special{pa   657  -120}\special{pa   670  -133}%
\special{pa   682  -144}\special{pa   694  -154}\special{pa   707  -163}\special{pa   719  -170}%
\special{pa   731  -176}\special{pa   743  -180}\special{pa   754  -183}\special{pa   766  -184}%
\special{pa   776  -184}\special{pa   777  -184}\special{pa   787  -182}\special{pa   798  -179}%
\special{pa   807  -175}\special{pa   817  -168}\special{pa   825  -161}\special{pa   834  -152}%
\special{pa   841  -141}\special{pa   848  -130}\special{pa   854  -117}\special{pa   860  -103}%
\special{pa   865   -87}\special{pa   869   -71}\special{pa   873   -53}\special{pa   875   -34}%
\special{pa   877   -15}\special{pa   878     5}\special{pa   879    26}\special{pa   851    18}%
\special{pa   824    11}\special{pa   797     3}\special{pa   770    -5}\special{pa   742   -13}%
\special{pa   715   -21}\special{pa   688   -29}\special{pa   661   -37}\special{pa   633   -44}%
\special{pa   606   -52}%
\special{fp}%
\special{pa  -879   -26}\special{pa  -878   -48}\special{pa  -877   -70}\special{pa  -875   -93}%
\special{pa  -873  -117}\special{pa  -869  -140}\special{pa  -865  -164}\special{pa  -860  -188}%
\special{pa  -854  -212}\special{pa  -848  -236}\special{pa  -841  -259}\special{pa  -834  -283}%
\special{pa  -825  -306}\special{pa  -817  -329}\special{pa  -807  -351}\special{pa  -798  -372}%
\special{pa  -787  -393}\special{pa  -777  -414}\special{pa  -766  -433}\special{pa  -754  -451}%
\special{pa  -743  -468}\special{pa  -731  -485}\special{pa  -719  -500}\special{pa  -707  -514}%
\special{pa  -694  -526}\special{pa  -682  -538}\special{pa  -670  -548}\special{pa  -657  -556}%
\special{pa  -645  -564}\special{pa  -633  -569}\special{pa  -621  -574}\special{pa  -609  -577}%
\special{pa  -598  -578}\special{pa  -587  -578}\special{pa  -576  -576}\special{pa  -572  -575}%
\special{fp}%
\settowidth{\Width}{$1$}\setlength{\Width}{-0.5\Width}%
\settoheight{\Height}{$1$}\settodepth{\Depth}{$1$}\setlength{\Height}{-\Height}%
\put(-0.5000000,-0.5800000){\hspace*{\Width}\raisebox{\Height}{$1$}}%
%
\settowidth{\Width}{$1$}\setlength{\Width}{-0.5\Width}%
\settoheight{\Height}{$1$}\settodepth{\Depth}{$1$}\setlength{\Height}{\Depth}%
\put(0.2000000,0.600000){\hspace*{\Width}\raisebox{\Height}{$1$}}%
%
\end{picture}}%}}
\putnotes{84}{28}{\scalebox{0.7}{%%% /Users/hama/Desktop/dntbiseki2-1107/zu2_cindy/fig/c2s11_hanenchu-y_kp1211.tex 
%%% Generator=c2s11_hanenchu-y_kp1211.cdy 
{\unitlength=1cm%
\begin{picture}%
(5,3.25)(-2.5,-1.25)%
\special{pn 8}%
%
\small%
\settowidth{\Width}{$x$}\setlength{\Width}{-0.5\Width}%
\settoheight{\Height}{$x$}\settodepth{\Depth}{$x$}\setlength{\Height}{-0.5\Height}\setlength{\Depth}{0.5\Depth}\addtolength{\Height}{\Depth}%
\put(-1.9100000,-0.5500000){\hspace*{\Width}\raisebox{\Height}{$x$}}%
%
\settowidth{\Width}{$y$}\setlength{\Width}{-0.5\Width}%
\settoheight{\Height}{$y$}\settodepth{\Depth}{$y$}\setlength{\Height}{-0.5\Height}\setlength{\Depth}{0.5\Depth}\addtolength{\Height}{\Depth}%
\put(0.8900000,-0.7700000){\hspace*{\Width}\raisebox{\Height}{$y$}}%
%
\settowidth{\Width}{$z$}\setlength{\Width}{-0.5\Width}%
\settoheight{\Height}{$z$}\settodepth{\Depth}{$z$}\setlength{\Height}{-0.5\Height}\setlength{\Depth}{0.5\Depth}\addtolength{\Height}{\Depth}%
\put(0.0000000,1.4900000){\hspace*{\Width}\raisebox{\Height}{$z$}}%
%
\settowidth{\Width}{O}\setlength{\Width}{-1\Width}%
\settoheight{\Height}{O}\settodepth{\Depth}{O}\setlength{\Height}{-\Height}%
\put(-0.0500000,-0.1500000){\hspace*{\Width}\raisebox{\Height}{O}}%
%
\special{pa  -518   150}\special{pa  -682   197}%
\special{fp}%
\special{pa   197   170}\special{pa   295   256}%
\special{fp}%
\special{pa     0  -341}\special{pa     0  -410}\special{pa     0  -511}%
\special{fp}%
\special{pn 8}%
\special{pa 576 -577}\special{pa 569 -573}\special{fp}\special{pa 542 -558}\special{pa 536 -552}\special{fp}%
\special{pa 518 -527}\special{pa 514 -520}\special{fp}\special{pa 502 -491}\special{pa 500 -483}\special{fp}%
\special{pa 492 -453}\special{pa 491 -445}\special{fp}\special{pa 487 -414}\special{pa 486 -406}\special{fp}%
\special{pa 487 -374}\special{pa 483 -368}\special{fp}\special{pa 454 -359}\special{pa 447 -356}\special{fp}%
\special{pa 417 -348}\special{pa 409 -345}\special{fp}\special{pa 379 -337}\special{pa 371 -334}\special{fp}%
\special{pa 341 -326}\special{pa 334 -324}\special{fp}\special{pa 303 -315}\special{pa 296 -313}\special{fp}%
\special{pa 266 -304}\special{pa 258 -302}\special{fp}\special{pa 228 -293}\special{pa 220 -291}\special{fp}%
\special{pa 190 -282}\special{pa 183 -280}\special{fp}\special{pa 152 -271}\special{pa 145 -269}\special{fp}%
\special{pa 115 -260}\special{pa 107 -258}\special{fp}\special{pa 77 -250}\special{pa 69 -247}\special{fp}%
\special{pa 39 -239}\special{pa 32 -236}\special{fp}\special{pa 2 -228}\special{pa -6 -226}\special{fp}%
\special{pa -36 -217}\special{pa -44 -215}\special{fp}\special{pa -74 -206}\special{pa -82 -204}\special{fp}%
\special{pa -112 -195}\special{pa -119 -193}\special{fp}\special{pa -149 -184}\special{pa -157 -182}\special{fp}%
\special{pa -187 -173}\special{pa -195 -171}\special{fp}\special{pa -225 -162}\special{pa -233 -160}\special{fp}%
\special{pa -263 -151}\special{pa -270 -149}\special{fp}\special{pa -300 -141}\special{pa -308 -138}\special{fp}%
\special{pa -338 -130}\special{pa -346 -127}\special{fp}\special{pa -376 -119}\special{pa -384 -117}\special{fp}%
\special{pa -414 -108}\special{pa -421 -106}\special{fp}\special{pa -451 -97}\special{pa -459 -95}\special{fp}%
\special{pa -489 -86}\special{pa -497 -84}\special{fp}\special{pa -527 -75}\special{pa -535 -73}\special{fp}%
\special{pa -565 -64}\special{pa -572 -62}\special{fp}\special{pa -602 -53}\special{pa -610 -51}\special{fp}%
\special{pn 8}%
{%
\color[cmyk]{1,0,0,0}%
\special{pa   538    72}\special{pa   537    50}\special{pa   536    28}\special{pa   534     5}%
\special{pa   532   -18}\special{pa   528   -42}\special{pa   524   -65}\special{pa   519   -89}%
\special{pa   513  -113}\special{pa   507  -137}\special{pa   500  -161}\special{pa   493  -184}%
\special{pa   484  -208}\special{pa   476  -230}\special{pa   466  -252}\special{pa   457  -274}%
\special{pa   446  -295}\special{pa   436  -315}\special{pa   425  -334}\special{pa   413  -353}%
\special{pa   402  -370}\special{pa   390  -386}\special{pa   378  -401}\special{pa   366  -415}%
\special{pa   353  -428}\special{pa   341  -439}\special{pa   329  -449}\special{pa   316  -458}%
\special{pa   304  -465}\special{pa   292  -471}\special{pa   280  -475}\special{pa   268  -478}%
\special{pa   257  -480}\special{pa   247  -479}\special{pa   235  -478}\special{pa   232  -477}%
\special{fp}%
}%
{%
\color[cmyk]{1,0,0,0}%
}%
{%
\color[cmyk]{1,0,0,0}%
\special{pa   197   170}\special{pa   196   149}\special{pa   195   126}\special{pa   193   104}%
\special{pa   191    80}\special{pa   187    57}\special{pa   183    33}\special{pa   178     9}%
\special{pa   173   -15}\special{pa   166   -39}\special{pa   159   -62}\special{pa   152   -86}%
\special{pa   143  -109}\special{pa   135  -132}\special{pa   125  -154}\special{pa   116  -176}%
\special{pa   105  -197}\special{pa    95  -217}\special{pa    84  -236}\special{pa    72  -254}%
\special{pa    61  -272}\special{pa    49  -288}\special{pa    37  -303}\special{pa    25  -317}%
\special{pa    12  -330}\special{pa     0  -341}\special{pa   -12  -351}\special{pa   -25  -360}%
\special{pa   -37  -367}\special{pa   -49  -373}\special{pa   -61  -377}\special{pa   -72  -380}%
\special{pa   -84  -381}\special{pa   -94  -381}\special{pa   -95  -381}\special{pa  -105  -379}%
\special{pa  -110  -378}%
\special{fp}%
}%
{%
\color[cmyk]{1,0,0,0}%
}%
{%
\color[cmyk]{1,0,0,0}%
\special{pa  -144   269}\special{pa  -144   247}\special{pa  -146   225}\special{pa  -148   202}%
\special{pa  -150   179}\special{pa  -154   155}\special{pa  -158   131}\special{pa  -163   108}%
\special{pa  -168    84}\special{pa  -175    60}\special{pa  -182    36}\special{pa  -189    12}%
\special{pa  -197   -11}\special{pa  -206   -33}\special{pa  -215   -56}\special{pa  -225   -77}%
\special{pa  -235   -98}\special{pa  -246  -118}\special{pa  -257  -137}\special{pa  -268  -156}%
\special{pa  -280  -173}\special{pa  -292  -189}\special{pa  -304  -205}\special{pa  -316  -218}%
\special{pa  -329  -231}\special{pa  -341  -243}\special{pa  -353  -253}\special{pa  -366  -261}%
\special{pa  -378  -268}\special{pa  -390  -274}\special{pa  -402  -279}\special{pa  -413  -281}%
\special{pa  -425  -283}\special{pa  -435  -282}\special{pa  -446  -281}\special{pa  -450  -280}%
\special{fp}%
}%
{%
\color[cmyk]{1,0,0,0}%
}%
{%
\color[cmyk]{1,0,0,0}%
\special{pa   821  -317}\special{pa   794  -310}\special{pa   767  -302}\special{pa   739  -294}%
\special{pa   712  -286}\special{pa   685  -278}\special{pa   657  -270}\special{pa   630  -262}%
\special{pa   603  -254}\special{pa   576  -247}\special{pa   548  -239}\special{pa   521  -231}%
\special{pa   494  -223}\special{pa   467  -215}\special{pa   439  -207}\special{pa   412  -199}%
\special{pa   385  -191}\special{pa   357  -184}\special{pa   330  -176}\special{pa   303  -168}%
\special{pa   276  -160}\special{pa   248  -152}\special{pa   221  -144}\special{pa   194  -136}%
\special{pa   166  -128}\special{pa   139  -121}\special{pa   112  -113}\special{pa    85  -105}%
\special{pa    57   -97}\special{pa    30   -89}\special{pa     3   -81}\special{pa   -24   -73}%
\special{pa   -52   -65}\special{pa   -79   -58}\special{pa  -106   -50}\special{pa  -134   -42}%
\special{pa  -161   -34}\special{pa  -188   -26}\special{pa  -215   -18}\special{pa  -243   -10}%
\special{pa  -270    -2}\special{pa  -297     5}\special{pa  -325    13}\special{pa  -352    21}%
\special{pa  -379    29}\special{pa  -406    37}\special{pa  -434    45}\special{pa  -461    53}%
\special{pa  -488    61}\special{pa  -515    68}\special{pa  -543    76}%
\special{fp}%
}%
{%
\color[cmyk]{1,0,0,0}%
}%
{%
\color[cmyk]{1,0,0,0}%
\special{pa   682  -538}\special{pa   655  -530}\special{pa   627  -522}\special{pa   600  -514}%
\special{pa   573  -506}\special{pa   546  -498}\special{pa   518  -491}\special{pa   491  -483}%
\special{pa   464  -475}\special{pa   436  -467}\special{pa   409  -459}\special{pa   382  -451}%
\special{pa   355  -443}\special{pa   327  -435}\special{pa   300  -428}\special{pa   273  -420}%
\special{pa   245  -412}\special{pa   218  -404}\special{pa   191  -396}\special{pa   164  -388}%
\special{pa   136  -380}\special{pa   109  -372}\special{pa    82  -365}\special{pa    55  -357}%
\special{pa    27  -349}\special{pa     0  -341}\special{pa   -27  -333}\special{pa   -55  -325}%
\special{pa   -82  -317}\special{pa  -109  -309}\special{pa  -136  -302}\special{pa  -164  -294}%
\special{pa  -191  -286}\special{pa  -218  -278}\special{pa  -245  -270}\special{pa  -273  -262}%
\special{pa  -300  -254}\special{pa  -327  -246}\special{pa  -355  -239}\special{pa  -382  -231}%
\special{pa  -409  -223}\special{pa  -436  -215}\special{pa  -464  -207}\special{pa  -491  -199}%
\special{pa  -518  -191}\special{pa  -546  -183}\special{pa  -573  -176}\special{pa  -600  -168}%
\special{pa  -627  -160}\special{pa  -655  -152}\special{pa  -682  -144}%
\special{fp}%
}%
{%
\color[cmyk]{1,0,0,0}%
}%
{%
\color[cmyk]{1,0,0,0}%
\special{pa  -757  -183}\special{pa  -767  -181}\special{pa  -794  -173}\special{pa  -821  -165}%
\special{fp}%
}%
{%
\color[cmyk]{1,0,0,0}%
}%
{%
\color[cmyk]{1,0,0,0}%
\special{pn 8}%
\special{pa 235 -479}\special{pa 228 -474}\special{fp}\special{pa 201 -459}\special{pa 195 -454}\special{fp}%
\special{pa 176 -428}\special{pa 172 -421}\special{fp}\special{pa 161 -391}\special{pa 158 -383}\special{fp}%
\special{pa 151 -352}\special{pa 150 -345}\special{fp}\special{pa 146 -313}\special{pa 145 -305}\special{fp}%
\special{pa 144 -273}\special{pa 144 -265}\special{fp}\special{pn 8}%
}%
{%
\color[cmyk]{1,0,0,0}%
}%
{%
\color[cmyk]{1,0,0,0}%
\special{pn 8}%
\special{pa -106 -380}\special{pa -113 -376}\special{fp}\special{pa -140 -360}\special{pa -146 -355}\special{fp}%
\special{pa -165 -329}\special{pa -169 -322}\special{fp}\special{pa -180 -292}\special{pa -183 -285}\special{fp}%
\special{pa -190 -254}\special{pa -191 -246}\special{fp}\special{pa -195 -214}\special{pa -196 -206}\special{fp}%
\special{pa -197 -174}\special{pa -197 -166}\special{fp}\special{pn 8}%
}%
{%
\color[cmyk]{1,0,0,0}%
}%
{%
\color[cmyk]{1,0,0,0}%
\special{pn 8}%
\special{pa -447 -282}\special{pa -454 -278}\special{fp}\special{pa -481 -262}\special{pa -487 -257}\special{fp}%
\special{pa -506 -231}\special{pa -509 -224}\special{fp}\special{pa -521 -194}\special{pa -524 -187}\special{fp}%
\special{pa -531 -156}\special{pa -532 -148}\special{fp}\special{pa -536 -116}\special{pa -537 -108}\special{fp}%
\special{pa -538 -76}\special{pa -538 -68}\special{fp}\special{pn 8}%
}%
{%
\color[cmyk]{1,0,0,0}%
}%
{%
\color[cmyk]{1,0,0,0}%
\special{pn 8}%
\special{pa 547 -560}\special{pa 539 -557}\special{fp}\special{pa 508 -549}\special{pa 501 -546}\special{fp}%
\special{pa 470 -538}\special{pa 462 -535}\special{fp}\special{pa 432 -526}\special{pa 424 -524}\special{fp}%
\special{pa 394 -515}\special{pa 386 -513}\special{fp}\special{pa 355 -504}\special{pa 348 -502}\special{fp}%
\special{pa 317 -493}\special{pa 309 -491}\special{fp}\special{pa 279 -482}\special{pa 271 -480}\special{fp}%
\special{pa 241 -471}\special{pa 233 -469}\special{fp}\special{pa 202 -460}\special{pa 195 -458}\special{fp}%
\special{pa 164 -449}\special{pa 157 -447}\special{fp}\special{pa 126 -438}\special{pa 118 -436}\special{fp}%
\special{pa 88 -427}\special{pa 80 -425}\special{fp}\special{pa 50 -416}\special{pa 42 -414}\special{fp}%
\special{pa 11 -405}\special{pa 4 -403}\special{fp}\special{pa -27 -394}\special{pa -35 -392}\special{fp}%
\special{pa -65 -383}\special{pa -73 -381}\special{fp}\special{pa -103 -372}\special{pa -111 -370}\special{fp}%
\special{pa -142 -361}\special{pa -149 -359}\special{fp}\special{pa -180 -350}\special{pa -188 -348}\special{fp}%
\special{pa -218 -339}\special{pa -226 -337}\special{fp}\special{pa -256 -328}\special{pa -264 -326}\special{fp}%
\special{pa -295 -317}\special{pa -302 -315}\special{fp}\special{pa -333 -306}\special{pa -340 -304}\special{fp}%
\special{pa -371 -295}\special{pa -379 -293}\special{fp}\special{pa -409 -284}\special{pa -417 -281}\special{fp}%
\special{pa -447 -273}\special{pa -455 -270}\special{fp}\special{pa -486 -262}\special{pa -493 -259}\special{fp}%
\special{pa -524 -251}\special{pa -532 -248}\special{fp}\special{pa -562 -240}\special{pa -570 -237}\special{fp}%
\special{pa -600 -229}\special{pa -608 -226}\special{fp}\special{pa -639 -217}\special{pa -646 -215}\special{fp}%
\special{pa -677 -206}\special{pa -685 -204}\special{fp}\special{pa -715 -195}\special{pa -723 -193}\special{fp}%
\special{pa -753 -184}\special{pa -761 -182}\special{fp}\special{pn 8}%
}%
{%
\color[cmyk]{1,0,0,0}%
}%
\special{pn 8}%
\special{pa 686 -198}\special{pa 678 -196}\special{fp}\special{pa 648 -187}\special{pa 641 -185}\special{fp}%
\special{pa 611 -176}\special{pa 603 -174}\special{fp}\special{pa 573 -165}\special{pa 566 -163}\special{fp}%
\special{pa 536 -155}\special{pa 528 -152}\special{fp}\special{pa 498 -144}\special{pa 491 -142}\special{fp}%
\special{pa 461 -133}\special{pa 453 -131}\special{fp}\special{pa 423 -122}\special{pa 416 -120}\special{fp}%
\special{pa 386 -111}\special{pa 378 -109}\special{fp}\special{pa 348 -101}\special{pa 341 -98}\special{fp}%
\special{pa 311 -90}\special{pa 303 -87}\special{fp}\special{pa 273 -79}\special{pa 266 -77}\special{fp}%
\special{pa 236 -68}\special{pa 228 -66}\special{fp}\special{pa 198 -57}\special{pa 191 -55}\special{fp}%
\special{pa 161 -46}\special{pa 153 -44}\special{fp}\special{pa 123 -36}\special{pa 116 -33}\special{fp}%
\special{pa 86 -25}\special{pa 78 -23}\special{fp}\special{pa 48 -14}\special{pa 41 -12}\special{fp}%
\special{pa 11 -3}\special{pa 3 -1}\special{fp}\special{pa -27 8}\special{pa -34 10}\special{fp}%
\special{pa -64 19}\special{pa -72 21}\special{fp}\special{pa -102 29}\special{pa -110 32}\special{fp}%
\special{pa -139 40}\special{pa -147 42}\special{fp}\special{pa -177 51}\special{pa -185 53}\special{fp}%
\special{pa -214 62}\special{pa -222 64}\special{fp}\special{pa -252 73}\special{pa -260 75}\special{fp}%
\special{pa -289 84}\special{pa -297 86}\special{fp}\special{pa -327 94}\special{pa -335 97}\special{fp}%
\special{pa -364 105}\special{pa -372 107}\special{fp}\special{pa -402 116}\special{pa -410 118}\special{fp}%
\special{pa -439 127}\special{pa -447 129}\special{fp}\special{pa -477 138}\special{pa -485 140}\special{fp}%
\special{pa -514 148}\special{pa -522 151}\special{fp}\special{pn 8}%
\special{pa -298 -258}\special{pa -292 -253}\special{fp}\special{pa -269 -233}\special{pa -263 -228}\special{fp}%
\special{pa -240 -208}\special{pa -234 -203}\special{fp}\special{pa -211 -183}\special{pa -205 -178}\special{fp}%
\special{pa -183 -158}\special{pa -176 -153}\special{fp}\special{pa -154 -133}\special{pa -148 -128}\special{fp}%
\special{pa -125 -108}\special{pa -119 -103}\special{fp}\special{pa -96 -83}\special{pa -90 -78}\special{fp}%
\special{pa -67 -58}\special{pa -61 -53}\special{fp}\special{pa -38 -33}\special{pa -32 -27}\special{fp}%
\special{pa -9 -8}\special{pa -3 -2}\special{fp}\special{pa 20 17}\special{pa 26 23}\special{fp}%
\special{pa 49 43}\special{pa 55 48}\special{fp}\special{pa 78 68}\special{pa 84 73}\special{fp}%
\special{pa 107 93}\special{pa 113 98}\special{fp}\special{pa 136 118}\special{pa 142 123}\special{fp}%
\special{pa 165 143}\special{pa 171 148}\special{fp}\special{pa 194 168}\special{pa 200 173}\special{fp}%
\special{pn 8}%
\special{pa 0 174}\special{pa 0 166}\special{fp}\special{pa 0 135}\special{pa 0 127}\special{fp}%
\special{pa 0 96}\special{pa 0 88}\special{fp}\special{pa 0 56}\special{pa 0 48}\special{fp}%
\special{pa 0 17}\special{pa 0 9}\special{fp}\special{pa 0 -22}\special{pa 0 -30}\special{fp}%
\special{pa 0 -62}\special{pa 0 -70}\special{fp}\special{pa 0 -101}\special{pa 0 -109}\special{fp}%
\special{pa 0 -140}\special{pa 0 -148}\special{fp}\special{pa 0 -180}\special{pa 0 -188}\special{fp}%
\special{pa 0 -219}\special{pa 0 -227}\special{fp}\special{pa 0 -258}\special{pa 0 -266}\special{fp}%
\special{pa 0 -298}\special{pa 0 -306}\special{fp}\special{pa 0 -337}\special{pa 0 -345}\special{fp}%
\special{pn 8}%
\special{pa  -791  -181}\special{pa  -764  -189}\special{pa  -737  -197}\special{pa  -709  -205}%
\special{pa  -682  -213}\special{pa  -655  -221}\special{pa  -627  -229}\special{pa  -600  -237}%
\special{pa  -573  -244}\special{pa  -546  -252}\special{pa  -518  -260}\special{pa  -491  -268}%
\special{pa  -464  -276}\special{pa  -437  -284}\special{pa  -409  -292}\special{pa  -382  -300}%
\special{pa  -355  -307}\special{pa  -327  -315}\special{pa  -300  -323}\special{pa  -273  -331}%
\special{pa  -246  -339}\special{pa  -218  -347}\special{pa  -191  -355}\special{pa  -164  -363}%
\special{pa  -136  -370}\special{pa  -109  -378}\special{pa   -82  -386}\special{pa   -55  -394}%
\special{pa   -30  -401}%
\special{fp}%
\special{pa    30  -418}\special{pa    54  -425}\special{pa    82  -433}\special{pa   109  -441}%
\special{pa   136  -449}\special{pa   164  -457}\special{pa   191  -465}\special{pa   218  -473}%
\special{pa   245  -481}\special{pa   273  -488}\special{pa   300  -496}\special{pa   327  -504}%
\special{pa   355  -512}\special{pa   382  -520}\special{pa   409  -528}\special{pa   436  -536}%
\special{pa   464  -544}\special{pa   491  -551}\special{pa   518  -559}\special{pa   545  -567}%
\special{pa   566  -573}\special{pa   573  -575}%
\special{fp}%
\special{pa   879   -26}\special{pa   851   -18}\special{pa   824   -11}\special{pa   797    -3}%
\special{pa   770     5}\special{pa   742    13}\special{pa   715    21}\special{pa   688    29}%
\special{pa   661    37}\special{pa   633    44}\special{pa   606    52}\special{pa   579    60}%
\special{pa   551    68}\special{pa   524    76}\special{pa   497    84}\special{pa   470    92}%
\special{pa   442   100}\special{pa   415   107}\special{pa   388   115}\special{pa   361   123}%
\special{pa   333   131}\special{pa   306   139}\special{pa   279   147}\special{pa   251   155}%
\special{pa   224   163}\special{pa   197   170}\special{pa   170   178}\special{pa   142   186}%
\special{pa   115   194}\special{pa    88   202}\special{pa    60   210}\special{pa    33   218}%
\special{pa     6   226}\special{pa   -21   233}\special{pa   -49   241}\special{pa   -76   249}%
\special{pa  -103   257}\special{pa  -130   265}\special{pa  -158   273}\special{pa  -185   281}%
\special{pa  -212   289}\special{pa  -240   296}\special{pa  -267   304}\special{pa  -294   312}%
\special{pa  -321   320}\special{pa  -349   328}\special{pa  -376   336}\special{pa  -403   344}%
\special{pa  -431   352}\special{pa  -458   359}\special{pa  -485   367}\special{pa  -485   346}%
\special{pa  -487   323}\special{pa  -489   300}\special{pa  -491   277}\special{pa  -495   254}%
\special{pa  -499   230}\special{pa  -504   206}\special{pa  -509   182}\special{pa  -516   158}%
\special{pa  -523   134}\special{pa  -530   111}\special{pa  -538    88}\special{pa  -547    65}%
\special{pa  -556    43}\special{pa  -566    21}\special{pa  -576     0}\special{pa  -587   -20}%
\special{pa  -598   -39}\special{pa  -606   -52}\special{pa  -609   -57}\special{pa  -621   -75}%
\special{pa  -633   -91}\special{pa  -645  -106}\special{pa  -657  -120}\special{pa  -670  -133}%
\special{pa  -682  -144}\special{pa  -694  -154}\special{pa  -707  -163}\special{pa  -719  -170}%
\special{pa  -731  -176}\special{pa  -743  -180}\special{pa  -754  -183}\special{pa  -766  -184}%
\special{pa  -776  -184}\special{pa  -777  -184}\special{pa  -787  -182}\special{pa  -798  -179}%
\special{pa  -807  -175}\special{pa  -817  -168}\special{pa  -825  -161}\special{pa  -834  -152}%
\special{pa  -841  -141}\special{pa  -848  -130}\special{pa  -854  -117}\special{pa  -860  -103}%
\special{pa  -865   -87}\special{pa  -869   -71}\special{pa  -873   -53}\special{pa  -875   -34}%
\special{pa  -877   -15}\special{pa  -878     5}\special{pa  -879    26}\special{pa  -851    18}%
\special{pa  -824    11}\special{pa  -797     3}\special{pa  -770    -5}\special{pa  -742   -13}%
\special{pa  -715   -21}\special{pa  -688   -29}\special{pa  -661   -37}\special{pa  -633   -44}%
\special{pa  -606   -52}%
\special{fp}%
\special{pa   879   -26}\special{pa   878   -48}\special{pa   877   -70}\special{pa   875   -93}%
\special{pa   873  -117}\special{pa   869  -140}\special{pa   865  -164}\special{pa   860  -188}%
\special{pa   854  -212}\special{pa   848  -236}\special{pa   841  -259}\special{pa   834  -283}%
\special{pa   825  -306}\special{pa   817  -329}\special{pa   807  -351}\special{pa   798  -372}%
\special{pa   787  -393}\special{pa   777  -414}\special{pa   766  -433}\special{pa   754  -451}%
\special{pa   743  -468}\special{pa   731  -485}\special{pa   719  -500}\special{pa   707  -514}%
\special{pa   694  -526}\special{pa   682  -538}\special{pa   670  -548}\special{pa   657  -556}%
\special{pa   645  -564}\special{pa   633  -569}\special{pa   621  -574}\special{pa   609  -577}%
\special{pa   598  -578}\special{pa   587  -578}\special{pa   576  -576}\special{pa   572  -575}%
\special{fp}%
\settowidth{\Width}{$1$}\setlength{\Width}{-0.5\Width}%
\settoheight{\Height}{$1$}\settodepth{\Depth}{$1$}\setlength{\Height}{-\Height}%
\put(0.5000000,-0.5800000){\hspace*{\Width}\raisebox{\Height}{$1$}}%
%
\settowidth{\Width}{$1$}\setlength{\Width}{-0.5\Width}%
\settoheight{\Height}{$1$}\settodepth{\Depth}{$1$}\setlength{\Height}{\Depth}%
\put(-0.1000000,0.600000){\hspace*{\Width}\raisebox{\Height}{$1$}}%
%
%
\end{picture}}%}}
\putnotes{27}{50}{\scalebox{0.7}{%%% /Users/hama/Desktop/dntbiseki2-1107/zu2_cindy/fig/c2s11_daenmen_k.tex 
%%% Generator=c2s11_daenmen_k.cdy 
{\unitlength=1cm%
\begin{picture}%
(4.75,3)(-2.25,-1)%
\special{pn 8}%
%
\small%
\settowidth{\Width}{$x$}\setlength{\Width}{-0.5\Width}%
\settoheight{\Height}{$x$}\settodepth{\Depth}{$x$}\setlength{\Height}{-0.5\Height}\setlength{\Depth}{0.5\Depth}\addtolength{\Height}{\Depth}%
\put(-0.8900000,-0.7700000){\hspace*{\Width}\raisebox{\Height}{$x$}}%
%
\settowidth{\Width}{$y$}\setlength{\Width}{-0.5\Width}%
\settoheight{\Height}{$y$}\settodepth{\Depth}{$y$}\setlength{\Height}{-0.5\Height}\setlength{\Depth}{0.5\Depth}\addtolength{\Height}{\Depth}%
\put(2.3500000,-0.6800000){\hspace*{\Width}\raisebox{\Height}{$y$}}%
%
\settowidth{\Width}{$z$}\setlength{\Width}{-0.5\Width}%
\settoheight{\Height}{$z$}\settodepth{\Depth}{$z$}\setlength{\Height}{-0.5\Height}\setlength{\Depth}{0.5\Depth}\addtolength{\Height}{\Depth}%
\put(0.0000000,1.4900000){\hspace*{\Width}\raisebox{\Height}{$z$}}%
%
\settowidth{\Width}{O}\setlength{\Width}{0\Width}%
\settoheight{\Height}{O}\settodepth{\Depth}{O}\setlength{\Height}{-\Height}%
\put(0.0500000,-0.1500000){\hspace*{\Width}\raisebox{\Height}{O}}%
%
\special{pa   394  -341}\special{pa   335  -290}%
\special{fp}%
\special{pa  -197   170}\special{pa  -295   256}%
\special{fp}%
\special{pa  -852  -246}\special{pa  -701  -202}%
\special{fp}%
\special{pa   681   196}\special{pa   852   246}%
\special{fp}%
\special{pa     0  -341}\special{pa     0  -405}\special{pa     0  -511}%
\special{fp}%
\special{pn 8}%
\special{pa 709 131}\special{pa 706 123}\special{fp}\special{pa 694 95}\special{pa 690 88}\special{fp}%
\special{pa 670 64}\special{pa 665 58}\special{fp}\special{pa 642 36}\special{pa 636 31}\special{fp}%
\special{pa 612 12}\special{pa 605 7}\special{fp}\special{pa 579 -10}\special{pa 572 -14}\special{fp}%
\special{pa 545 -30}\special{pa 539 -34}\special{fp}\special{pa 511 -49}\special{pa 504 -53}\special{fp}%
\special{pa 476 -66}\special{pa 469 -70}\special{fp}\special{pa 440 -83}\special{pa 433 -86}\special{fp}%
\special{pa 404 -98}\special{pa 397 -101}\special{fp}\special{pa 368 -113}\special{pa 360 -116}\special{fp}%
\special{pa 331 -127}\special{pa 324 -129}\special{fp}\special{pa 294 -140}\special{pa 287 -142}\special{fp}%
\special{pa 257 -152}\special{pa 249 -154}\special{fp}\special{pa 219 -164}\special{pa 212 -166}\special{fp}%
\special{pa 182 -175}\special{pa 174 -177}\special{fp}\special{pa 144 -185}\special{pa 136 -187}\special{fp}%
\special{pa 106 -195}\special{pa 98 -197}\special{fp}\special{pa 68 -203}\special{pa 60 -205}\special{fp}%
\special{pa 29 -212}\special{pa 21 -214}\special{fp}\special{pa -9 -220}\special{pa -17 -221}\special{fp}%
\special{pa -48 -227}\special{pa -56 -229}\special{fp}\special{pa -86 -234}\special{pa -94 -235}\special{fp}%
\special{pa -125 -240}\special{pa -133 -241}\special{fp}\special{pa -164 -245}\special{pa -172 -246}\special{fp}%
\special{pa -203 -249}\special{pa -211 -250}\special{fp}\special{pa -242 -254}\special{pa -250 -254}\special{fp}%
\special{pa -281 -256}\special{pa -289 -257}\special{fp}\special{pa -320 -259}\special{pa -328 -259}\special{fp}%
\special{pa -360 -260}\special{pa -368 -260}\special{fp}\special{pa -399 -260}\special{pa -407 -260}\special{fp}%
\special{pa -438 -259}\special{pa -446 -258}\special{fp}\special{pa -477 -257}\special{pa -485 -256}\special{fp}%
\special{pa -516 -253}\special{pa -524 -252}\special{fp}\special{pa -555 -246}\special{pa -563 -245}\special{fp}%
\special{pa -593 -238}\special{pa -601 -236}\special{fp}\special{pa -631 -226}\special{pa -638 -223}\special{fp}%
\special{pa -666 -209}\special{pa -672 -204}\special{fp}\special{pa -695 -182}\special{pa -699 -176}\special{fp}%
\special{pa -707 -146}\special{pa -710 -139}\special{fp}\special{pn 8}%
{%
\color[cmyk]{1,0,0,0}%
\special{pa   -75  -250}\special{pa   -42  -241}\special{pa    -8  -233}\special{pa    26  -227}%
\special{pa    60  -222}\special{pa    92  -218}\special{pa   124  -216}\special{pa   153  -215}%
\special{pa   180  -216}\special{pa   204  -219}\special{pa   225  -223}\special{pa   242  -229}%
\special{pa   256  -236}\special{pa   265  -244}\special{pa   270  -253}\special{pa   271  -264}%
\special{pa   268  -275}\special{pa   261  -286}\special{pa   249  -299}\special{pa   234  -311}%
\special{pa   214  -324}\special{pa   192  -336}\special{pa   166  -348}\special{pa   138  -359}%
\special{pa   107  -370}\special{pa    75  -380}\special{pa    42  -389}\special{pa     8  -397}%
\special{pa   -26  -403}\special{pa   -60  -408}\special{pa   -92  -412}\special{pa  -124  -414}%
\special{pa  -153  -415}\special{pa  -180  -414}\special{pa  -204  -411}\special{pa  -225  -407}%
\special{pa  -242  -401}\special{pa  -256  -394}\special{pa  -265  -386}\special{pa  -270  -377}%
\special{pa  -271  -366}\special{pa  -268  -355}\special{pa  -261  -344}\special{pa  -249  -331}%
\special{pa  -234  -319}\special{pa  -214  -307}\special{pa  -192  -294}\special{pa  -166  -282}%
\special{pa  -138  -271}\special{pa  -107  -260}\special{pa   -75  -250}\special{pa   -42  -241}%
\special{pa    -8  -233}\special{pa    26  -227}\special{pa    60  -222}\special{pa    92  -218}%
\special{pa   124  -216}\special{pa   153  -215}\special{pa   180  -216}\special{pa   204  -219}%
\special{pa   225  -223}\special{pa   242  -229}\special{pa   256  -236}\special{pa   265  -244}%
\special{pa   270  -253}\special{pa   271  -264}\special{pa   268  -275}\special{pa   261  -286}%
\special{pa   249  -299}\special{pa   234  -311}\special{pa   214  -324}\special{pa   192  -336}%
\special{pa   166  -348}\special{pa   138  -359}\special{pa   107  -370}\special{pa    75  -380}%
\special{pa    42  -389}\special{pa     8  -397}\special{pa   -26  -403}\special{pa   -60  -408}%
\special{pa   -92  -412}\special{pa  -124  -414}\special{pa  -153  -415}\special{pa  -180  -414}%
\special{pa  -204  -411}\special{pa  -225  -407}\special{pa  -242  -401}\special{pa  -256  -394}%
\special{pa  -265  -386}\special{pa  -270  -377}\special{pa  -271  -366}\special{pa  -268  -355}%
\special{pa  -261  -344}\special{pa  -249  -331}\special{pa  -234  -319}\special{pa  -214  -307}%
\special{pa  -192  -294}\special{pa  -166  -282}\special{pa  -138  -271}\special{pa  -107  -260}%
\special{pa   -75  -250}%
\special{fp}%
}%
{%
\color[cmyk]{1,0,0,0}%
}%
{%
\color[cmyk]{1,0,0,0}%
\special{pa  -353  -421}\special{pa  -371  -419}\special{pa  -377  -418}\special{pa  -395  -415}%
\special{pa  -414  -411}\special{pa  -416  -411}\special{pa  -439  -403}\special{pa  -448  -400}%
\special{pa  -472  -388}\special{pa  -490  -372}\special{pa  -500  -355}\special{pa  -502  -336}%
\special{pa  -496  -316}\special{pa  -482  -294}\special{pa  -460  -272}\special{pa  -432  -249}%
\special{pa  -396  -225}\special{pa  -354  -203}\special{pa  -307  -180}\special{pa  -255  -159}%
\special{pa  -199  -139}\special{pa  -139  -121}\special{pa   -78  -104}\special{pa   -15   -90}%
\special{pa    48   -78}\special{pa   110   -68}\special{pa   171   -62}\special{pa   229   -58}%
\special{pa   283   -57}\special{pa   333   -59}\special{pa   377   -64}\special{pa   416   -71}%
\special{pa   448   -82}\special{pa   472   -95}\special{pa   490  -110}\special{pa   500  -127}%
\special{pa   502  -146}\special{pa   496  -166}\special{pa   482  -188}\special{pa   460  -211}%
\special{pa   439  -228}\special{pa   432  -234}\special{pa   414  -245}\special{pa   404  -251}%
\special{pa   396  -257}\special{pa   396  -257}\special{pa   384  -263}\special{pa   355  -280}%
\special{fp}%
}%
{%
\color[cmyk]{1,0,0,0}%
}%
{%
\color[cmyk]{1,0,0,0}%
\special{pa  -610  -326}\special{pa  -617  -322}\special{pa  -625  -316}\special{pa  -632  -310}%
\special{pa  -638  -304}\special{pa  -640  -302}\special{pa  -647  -291}\special{pa  -653  -280}%
\special{pa  -655  -255}\special{pa  -647  -228}\special{pa  -629  -200}\special{pa  -601  -170}%
\special{pa  -564  -140}\special{pa  -517  -110}\special{pa  -463   -80}\special{pa  -401   -51}%
\special{pa  -333   -23}\special{pa  -259     3}\special{pa  -182    27}\special{pa  -101    49}%
\special{pa   -19    67}\special{pa    63    83}\special{pa   144    95}\special{pa   223   104}%
\special{pa   299   109}\special{pa   369   110}\special{pa   434   107}\special{pa   493   101}%
\special{pa   543    91}\special{pa   585    78}\special{pa   617    61}\special{pa   640    41}%
\special{pa   653    19}\special{pa   655    -6}\special{pa   647   -33}\special{pa   638   -48}%
\special{pa   630   -61}\special{pa   629   -61}\special{pa   625   -65}%
\special{fp}%
}%
{%
\color[cmyk]{1,0,0,0}%
}%
{%
\color[cmyk]{1,0,0,0}%
\special{pa     0  -341}\special{pa    21  -335}\special{pa    43  -328}\special{pa    64  -321}%
\special{pa    85  -314}\special{pa   107  -306}\special{pa   128  -298}\special{pa   149  -290}%
\special{pa   170  -281}\special{pa   190  -272}\special{pa   211  -263}\special{pa   231  -254}%
\special{pa   251  -245}\special{pa   271  -235}\special{pa   290  -225}\special{pa   310  -214}%
\special{pa   329  -204}\special{pa   347  -193}\special{pa   365  -182}\special{pa   383  -171}%
\special{pa   401  -160}\special{pa   418  -149}\special{pa   435  -137}\special{pa   451  -126}%
\special{pa   467  -114}\special{pa   482  -102}\special{pa   497   -90}\special{pa   512   -78}%
\special{pa   525   -66}\special{pa   539   -53}\special{pa   552   -41}\special{pa   564   -29}%
\special{pa   576   -16}\special{pa   587    -4}\special{pa   598     8}\special{pa   608    21}%
\special{pa   617    33}\special{pa   626    45}\special{pa   634    58}\special{pa   642    70}%
\special{pa   649    82}\special{pa   655    94}\special{pa   660   106}\special{pa   665   118}%
\special{pa   670   129}\special{pa   674   141}\special{pa   677   153}\special{pa   679   164}%
\special{pa   681   175}\special{pa   682   186}\special{pa   682   195}%
\special{fp}%
}%
{%
\color[cmyk]{1,0,0,0}%
}%
{%
\color[cmyk]{1,0,0,0}%
\special{pa     0  -341}\special{pa     6  -346}\special{pa    12  -351}\special{pa    19  -355}%
\special{pa    25  -360}\special{pa    31  -363}\special{pa    37  -367}\special{pa    43  -370}%
\special{pa    49  -373}\special{pa    55  -375}\special{pa    61  -377}\special{pa    67  -379}%
\special{pa    72  -380}\special{pa    78  -381}\special{pa    84  -381}\special{pa    89  -381}%
\special{pa    89  -381}\special{pa    93  -381}\special{pa    95  -381}\special{pa   100  -380}%
\special{pa   105  -379}\special{pa   111  -378}\special{pa   111  -378}%
\special{fp}%
}%
{%
\color[cmyk]{1,0,0,0}%
}%
{%
\color[cmyk]{1,0,0,0}%
\special{pa     0  -341}\special{pa   -21  -347}\special{pa   -43  -353}\special{pa   -64  -358}%
\special{pa   -85  -363}\special{pa  -107  -368}\special{pa  -128  -372}\special{pa  -149  -376}%
\special{pa  -170  -379}\special{pa  -190  -382}\special{pa  -211  -385}\special{pa  -231  -387}%
\special{pa  -251  -389}\special{pa  -271  -391}\special{pa  -290  -392}\special{pa  -310  -393}%
\special{pa  -329  -394}\special{pa  -347  -394}\special{pa  -365  -393}\special{pa  -383  -393}%
\special{pa  -401  -392}\special{pa  -418  -390}\special{pa  -435  -388}\special{pa  -451  -386}%
\special{pa  -467  -383}\special{pa  -482  -380}\special{pa  -497  -377}\special{pa  -512  -373}%
\special{pa  -525  -369}\special{pa  -539  -365}\special{pa  -552  -360}\special{pa  -564  -354}%
\special{pa  -576  -349}\special{pa  -581  -346}\special{pa  -587  -343}\special{pa  -590  -341}%
\special{pa  -598  -337}\special{pa  -608  -330}\special{pa  -612  -327}\special{pa  -617  -323}%
\special{pa  -617  -323}\special{pa  -625  -316}\special{pa  -626  -316}\special{pa  -634  -309}%
\special{pa  -638  -305}\special{pa  -642  -301}\special{pa  -649  -293}\special{pa  -649  -292}%
\special{fp}%
}%
{%
\color[cmyk]{1,0,0,0}%
}%
{%
\color[cmyk]{1,0,0,0}%
\special{pn 8}%
\special{pa 358 -278}\special{pa 351 -281}\special{fp}\special{pa 323 -294}\special{pa 316 -298}\special{fp}%
\special{pa 288 -310}\special{pa 281 -313}\special{fp}\special{pa 253 -324}\special{pa 245 -327}\special{fp}%
\special{pa 217 -337}\special{pa 209 -339}\special{fp}\special{pa 180 -349}\special{pa 172 -351}\special{fp}%
\special{pa 143 -360}\special{pa 136 -363}\special{fp}\special{pa 106 -370}\special{pa 99 -372}\special{fp}%
\special{pa 69 -380}\special{pa 61 -382}\special{fp}\special{pa 32 -389}\special{pa 24 -390}\special{fp}%
\special{pa -6 -396}\special{pa -14 -398}\special{fp}\special{pa -44 -404}\special{pa -51 -405}\special{fp}%
\special{pa -81 -409}\special{pa -89 -411}\special{fp}\special{pa -119 -415}\special{pa -127 -416}\special{fp}%
\special{pa -158 -419}\special{pa -166 -420}\special{fp}\special{pa -196 -422}\special{pa -204 -423}\special{fp}%
\special{pa -234 -424}\special{pa -242 -425}\special{fp}\special{pa -273 -425}\special{pa -281 -425}\special{fp}%
\special{pa -311 -424}\special{pa -319 -424}\special{fp}\special{pa -349 -421}\special{pa -357 -421}\special{fp}%
\special{pn 8}%
}%
{%
\color[cmyk]{1,0,0,0}%
}%
{%
\color[cmyk]{1,0,0,0}%
\special{pn 8}%
\special{pa 628 -63}\special{pa 623 -68}\special{fp}\special{pa 601 -91}\special{pa 595 -96}\special{fp}%
\special{pa 570 -116}\special{pa 563 -121}\special{fp}\special{pa 537 -138}\special{pa 530 -143}\special{fp}%
\special{pa 503 -159}\special{pa 496 -163}\special{fp}\special{pa 468 -178}\special{pa 461 -182}\special{fp}%
\special{pa 433 -195}\special{pa 425 -198}\special{fp}\special{pa 397 -212}\special{pa 389 -215}\special{fp}%
\special{pa 360 -227}\special{pa 353 -230}\special{fp}\special{pa 323 -241}\special{pa 316 -244}\special{fp}%
\special{pa 286 -255}\special{pa 278 -257}\special{fp}\special{pa 248 -267}\special{pa 241 -270}\special{fp}%
\special{pa 210 -279}\special{pa 203 -281}\special{fp}\special{pa 172 -290}\special{pa 165 -293}\special{fp}%
\special{pa 134 -301}\special{pa 126 -303}\special{fp}\special{pa 96 -311}\special{pa 88 -313}\special{fp}%
\special{pa 57 -320}\special{pa 49 -321}\special{fp}\special{pa 18 -328}\special{pa 11 -330}\special{fp}%
\special{pa -20 -336}\special{pa -28 -337}\special{fp}\special{pa -59 -343}\special{pa -67 -345}\special{fp}%
\special{pa -99 -349}\special{pa -106 -350}\special{fp}\special{pa -138 -355}\special{pa -146 -356}\special{fp}%
\special{pa -177 -360}\special{pa -185 -361}\special{fp}\special{pa -217 -364}\special{pa -225 -365}\special{fp}%
\special{pa -256 -367}\special{pa -264 -368}\special{fp}\special{pa -296 -370}\special{pa -304 -370}\special{fp}%
\special{pa -335 -370}\special{pa -343 -371}\special{fp}\special{pa -375 -371}\special{pa -383 -371}\special{fp}%
\special{pa -415 -369}\special{pa -423 -369}\special{fp}\special{pa -454 -366}\special{pa -462 -365}\special{fp}%
\special{pa -493 -362}\special{pa -501 -361}\special{fp}\special{pa -532 -354}\special{pa -540 -352}\special{fp}%
\special{pa -570 -343}\special{pa -578 -341}\special{fp}\special{pa -607 -327}\special{pa -614 -324}\special{fp}%
\special{pn 8}%
}%
{%
\color[cmyk]{1,0,0,0}%
}%
{%
\color[cmyk]{1,0,0,0}%
\special{pn 8}%
\special{pa 107 -380}\special{pa 114 -376}\special{fp}\special{pa 141 -360}\special{pa 147 -354}\special{fp}%
\special{pa 165 -328}\special{pa 169 -321}\special{fp}\special{pa 181 -292}\special{pa 183 -284}\special{fp}%
\special{pa 190 -253}\special{pa 191 -246}\special{fp}\special{pa 195 -214}\special{pa 196 -206}\special{fp}%
\special{pa 197 -174}\special{pa 197 -166}\special{fp}\special{pn 8}%
}%
{%
\color[cmyk]{1,0,0,0}%
}%
{%
\color[cmyk]{1,0,0,0}%
\special{pn 8}%
\special{pa -647 -295}\special{pa -651 -288}\special{fp}\special{pa -665 -266}\special{pa -669 -259}\special{fp}%
\special{pa -677 -235}\special{pa -679 -227}\special{fp}\special{pa -681 -201}\special{pa -682 -193}\special{fp}%
\special{pn 8}%
}%
{%
\color[cmyk]{1,0,0,0}%
}%
\special{pn 8}%
\special{pa 338 -293}\special{pa 332 -287}\special{fp}\special{pa 308 -267}\special{pa 302 -262}\special{fp}%
\special{pa 279 -241}\special{pa 273 -236}\special{fp}\special{pa 249 -216}\special{pa 243 -211}\special{fp}%
\special{pa 220 -190}\special{pa 214 -185}\special{fp}\special{pa 190 -165}\special{pa 184 -160}\special{fp}%
\special{pa 161 -139}\special{pa 155 -134}\special{fp}\special{pa 131 -114}\special{pa 125 -108}\special{fp}%
\special{pa 102 -88}\special{pa 96 -83}\special{fp}\special{pa 72 -62}\special{pa 66 -57}\special{fp}%
\special{pa 43 -37}\special{pa 36 -32}\special{fp}\special{pa 13 -11}\special{pa 7 -6}\special{fp}%
\special{pa -17 14}\special{pa -23 20}\special{fp}\special{pa -46 40}\special{pa -52 45}\special{fp}%
\special{pa -76 66}\special{pa -82 71}\special{fp}\special{pa -105 91}\special{pa -111 96}\special{fp}%
\special{pa -135 117}\special{pa -141 122}\special{fp}\special{pa -164 142}\special{pa -170 148}\special{fp}%
\special{pa -194 168}\special{pa -200 173}\special{fp}\special{pn 8}%
\special{pa -705 -204}\special{pa -697 -201}\special{fp}\special{pa -668 -193}\special{pa -660 -191}\special{fp}%
\special{pa -630 -182}\special{pa -623 -180}\special{fp}\special{pa -593 -171}\special{pa -585 -169}\special{fp}%
\special{pa -556 -160}\special{pa -548 -158}\special{fp}\special{pa -518 -150}\special{pa -511 -147}\special{fp}%
\special{pa -481 -139}\special{pa -473 -137}\special{fp}\special{pa -444 -128}\special{pa -436 -126}\special{fp}%
\special{pa -406 -117}\special{pa -399 -115}\special{fp}\special{pa -369 -107}\special{pa -361 -104}\special{fp}%
\special{pa -332 -96}\special{pa -324 -94}\special{fp}\special{pa -294 -85}\special{pa -287 -83}\special{fp}%
\special{pa -257 -74}\special{pa -249 -72}\special{fp}\special{pa -220 -63}\special{pa -212 -61}\special{fp}%
\special{pa -182 -53}\special{pa -175 -50}\special{fp}\special{pa -145 -42}\special{pa -137 -40}\special{fp}%
\special{pa -108 -31}\special{pa -100 -29}\special{fp}\special{pa -70 -20}\special{pa -63 -18}\special{fp}%
\special{pa -33 -9}\special{pa -25 -7}\special{fp}\special{pa 4 1}\special{pa 12 4}\special{fp}%
\special{pa 42 12}\special{pa 50 14}\special{fp}\special{pa 79 23}\special{pa 87 25}\special{fp}%
\special{pa 117 34}\special{pa 124 36}\special{fp}\special{pa 154 44}\special{pa 162 47}\special{fp}%
\special{pa 191 55}\special{pa 199 57}\special{fp}\special{pa 229 66}\special{pa 236 68}\special{fp}%
\special{pa 266 77}\special{pa 274 79}\special{fp}\special{pa 303 88}\special{pa 311 90}\special{fp}%
\special{pa 341 98}\special{pa 348 101}\special{fp}\special{pa 378 109}\special{pa 386 111}\special{fp}%
\special{pa 415 120}\special{pa 423 122}\special{fp}\special{pa 453 131}\special{pa 460 133}\special{fp}%
\special{pa 490 141}\special{pa 498 144}\special{fp}\special{pa 527 152}\special{pa 535 154}\special{fp}%
\special{pa 565 163}\special{pa 572 165}\special{fp}\special{pa 602 174}\special{pa 610 176}\special{fp}%
\special{pa 639 185}\special{pa 647 187}\special{fp}\special{pa 677 195}\special{pa 684 198}\special{fp}%
\special{pn 8}%
\special{pa 0 174}\special{pa 0 166}\special{fp}\special{pa 0 135}\special{pa 0 127}\special{fp}%
\special{pa 0 96}\special{pa 0 88}\special{fp}\special{pa 0 56}\special{pa 0 48}\special{fp}%
\special{pa 0 17}\special{pa 0 9}\special{fp}\special{pa 0 -22}\special{pa 0 -30}\special{fp}%
\special{pa 0 -62}\special{pa 0 -70}\special{fp}\special{pa 0 -101}\special{pa 0 -109}\special{fp}%
\special{pa 0 -140}\special{pa 0 -148}\special{fp}\special{pa 0 -180}\special{pa 0 -188}\special{fp}%
\special{pa 0 -219}\special{pa 0 -227}\special{fp}\special{pa 0 -258}\special{pa 0 -266}\special{fp}%
\special{pa 0 -298}\special{pa 0 -306}\special{fp}\special{pa 0 -337}\special{pa 0 -345}\special{fp}%
\special{pn 8}%
{%
\color[cmyk]{1,0,0,0}%
\special{pa  -197   170}\special{pa  -197   160}\special{pa  -196   149}\special{pa  -196   138}%
\special{pa  -195   126}\special{pa  -194   115}\special{pa  -193   104}\special{pa  -192    92}%
\special{pa  -191    80}\special{pa  -189    69}\special{pa  -187    57}\special{pa  -185    45}%
\special{pa  -183    33}\special{pa  -181    21}\special{pa  -178     9}\special{pa  -175    -3}%
\special{pa  -173   -15}\special{pa  -169   -27}\special{pa  -166   -39}\special{pa  -163   -51}%
\special{pa  -159   -62}\special{pa  -156   -74}\special{pa  -152   -86}\special{pa  -148   -98}%
\special{pa  -143  -109}\special{pa  -139  -121}\special{pa  -135  -132}\special{pa  -130  -143}%
\special{pa  -125  -154}\special{pa  -121  -165}\special{pa  -116  -176}\special{pa  -111  -186}%
\special{pa  -105  -197}\special{pa  -100  -207}\special{pa   -95  -217}\special{pa   -89  -226}%
\special{pa   -84  -236}\special{pa   -78  -245}\special{pa   -72  -254}\special{pa   -67  -263}%
\special{pa   -61  -272}\special{pa   -55  -280}\special{pa   -49  -288}\special{pa   -43  -296}%
\special{pa   -37  -303}\special{pa   -31  -310}\special{pa   -25  -317}\special{pa   -19  -323}%
\special{pa   -12  -330}\special{pa    -6  -335}\special{pa    -0  -341}%
\special{fp}%
}%
\special{pa   710   140}\special{pa   709   127}\special{pa   709   126}\special{pa   708   117}%
\special{pa   708   117}\special{pa   708   111}\special{pa   705    96}\special{pa   702    80}%
\special{pa   697    64}\special{pa   692    48}\special{pa   686    32}\special{pa   680    21}%
\special{pa   678    16}\special{pa   670    -0}\special{pa   660   -17}\special{pa   650   -33}%
\special{pa   638   -50}\special{pa   626   -66}\special{pa   612   -82}\special{pa   598   -99}%
\special{pa   582  -115}\special{pa   565  -132}\special{pa   553  -144}\special{pa   547  -148}%
\special{pa   528  -165}\special{pa   508  -181}\special{pa   487  -198}\special{pa   477  -204}%
\special{pa   464  -214}\special{pa   439  -230}\special{pa   413  -247}\special{pa   404  -252}%
\special{pa   385  -263}\special{pa   355  -280}\special{pa   336  -290}\special{pa   322  -296}%
\special{pa   286  -314}\special{pa   273  -319}\special{pa   244  -331}\special{pa   215  -343}%
\special{pa   196  -350}\special{pa   161  -362}\special{pa   131  -372}\special{pa   109  -378}%
\special{pa    60  -392}\special{pa    29  -398}%
\special{fp}%
\special{pa   -30  -410}\special{pa   -37  -412}\special{pa   -86  -419}\special{pa  -133  -424}%
\special{pa  -189  -428}\special{pa  -196  -428}\special{pa  -245  -429}\special{pa  -286  -428}%
\special{pa  -306  -427}\special{pa  -322  -425}\special{pa  -355  -422}\special{pa  -371  -419}%
\special{pa  -385  -417}\special{pa  -413  -412}\special{pa  -439  -406}\special{pa  -464  -399}%
\special{pa  -486  -392}\special{pa  -508  -384}\special{pa  -517  -381}\special{pa  -528  -376}%
\special{pa  -547  -367}\special{pa  -565  -358}\special{pa  -582  -348}\special{pa  -591  -342}%
\special{pa  -598  -338}\special{pa  -612  -327}\special{pa  -626  -316}\special{pa  -638  -305}%
\special{pa  -650  -293}\special{pa  -656  -285}\special{pa  -660  -281}\special{pa  -670  -268}%
\special{pa  -678  -255}\special{pa  -685  -242}\special{pa  -692  -228}\special{pa  -697  -215}%
\special{pa  -699  -208}\special{pa  -702  -201}\special{pa  -705  -186}\special{pa  -708  -172}%
\special{pa  -709  -158}\special{pa  -709  -157}\special{pa  -710  -143}%
\special{fp}%
\special{pa  -709  -142}\special{pa  -709  -135}\special{pa  -701  -106}\special{pa  -681   -75}%
\special{pa  -651   -43}\special{pa  -610   -10}\special{pa  -560    22}\special{pa  -501    55}%
\special{pa  -434    86}\special{pa  -360   116}\special{pa  -281   144}\special{pa  -197   170}%
\special{pa  -110   194}\special{pa   -21   214}\special{pa    68   231}\special{pa   156   244}%
\special{pa   242   254}\special{pa   323   259}\special{pa   400   260}\special{pa   470   258}%
\special{pa   533   251}\special{pa   588   240}\special{pa   633   225}\special{pa   668   207}%
\special{pa   693   186}\special{pa   707   161}\special{pa   709   140}\special{pa   709   135}%
\special{pa   707   127}%
\special{fp}%
\settowidth{\Width}{$1$}\setlength{\Width}{-0.5\Width}%
\settoheight{\Height}{$1$}\settodepth{\Depth}{$1$}\setlength{\Height}{-\Height}%
\put(-0.5000000,-0.5800000){\hspace*{\Width}\raisebox{\Height}{$1$}}%
%
\settowidth{\Width}{$2$}\setlength{\Width}{-0.5\Width}%
\settoheight{\Height}{$2$}\settodepth{\Depth}{$2$}\setlength{\Height}{-\Height}%
\put(1.7300000,-0.6500000){\hspace*{\Width}\raisebox{\Height}{$2$}}%
%

%
\end{picture}}%}}
\putnotes{84}{50}{\scalebox{0.7}{%%% /Users/y.maeda/Dropbox/Macfile/微分積分II/dntbiseki2-main/教科書2章p32p56図修正/fig/c2s11_kyokumen_km.tex 
%%% Generator=c2s11_kyokumen_km.cdy 
{\unitlength=0.42cm%
\begin{picture}%
(9,8.5)(-4.5,-3)%
\linethickness{0.008in}%%
\settowidth{\Width}{$x$}\setlength{\Width}{-0.5\Width}%
\settoheight{\Height}{$x$}\settodepth{\Depth}{$x$}\setlength{\Height}{-0.5\Height}\setlength{\Depth}{0.5\Depth}\addtolength{\Height}{\Depth}%
\put(-1.8800000,-2.7400000){\hspace*{\Width}\raisebox{\Height}{$x$}}%
%
\settowidth{\Width}{$y$}\setlength{\Width}{-0.5\Width}%
\settoheight{\Height}{$y$}\settodepth{\Depth}{$y$}\setlength{\Height}{-0.5\Height}\setlength{\Depth}{0.5\Depth}\addtolength{\Height}{\Depth}%
\put(5.1700000,-0.8900000){\hspace*{\Width}\raisebox{\Height}{$y$}}%
%
\settowidth{\Width}{$z$}\setlength{\Width}{-0.5\Width}%
\settoheight{\Height}{$z$}\settodepth{\Depth}{$z$}\setlength{\Height}{-0.5\Height}\setlength{\Depth}{0.5\Depth}\addtolength{\Height}{\Depth}%
\put(0.0000000,5.6500000){\hspace*{\Width}\raisebox{\Height}{$z$}}%
%
\polyline(-3.87138,0.58531)(-3.89970,0.60188)(-3.92455,0.64052)(-3.94589,0.69989)%
(-3.96371,0.77836)(-3.97800,0.87402)(-3.98874,0.98471)(-3.99592,1.10806)(-3.99953,1.24152)%
(-3.99958,1.38238)(-3.99606,1.52783)(-3.98897,1.67502)(-3.97832,1.82103)(-3.96413,1.96299)%
(-3.94640,2.09809)(-3.92515,2.22360)(-3.90039,2.33697)(-3.87216,2.43577)(-3.84048,2.51785)%
(-3.80536,2.58124)(-3.76686,2.62429)(-3.72499,2.64563)(-3.67980,2.64419)(-3.63132,2.61928)%
(-3.57961,2.57050)(-3.52470,2.49785)(-3.46665,2.40163)(-3.40551,2.28253)(-3.34133,2.14156)%
(-3.27416,1.98006)(-3.20408,1.79967)(-3.13114,1.60231)(-3.05540,1.39016)(-2.97694,1.16564)%
(-2.89582,0.93133)(-2.81212,0.69001)(-2.72592,0.44453)(-2.63727,0.19785)(-2.54628,-0.04706)%
(-2.45302,-0.28722)(-2.35756,-0.51970)(-2.26001,-0.74168)(-2.16043,-0.95044)(-2.05893,-1.14344)%
(-1.95560,-1.31837)(-1.85052,-1.47313)(-1.74378,-1.60589)(-1.63550,-1.71511)(-1.52575,-1.79957)%
(-1.41464,-1.85840)(-1.30227,-1.89104)%
%
\polyline(-3.32114,1.96736)(-3.32475,1.95499)(-3.35663,1.84633)\polyline(-3.38634,1.74552)(-3.42228,1.62462)%
\polyline(-3.45271,1.52402)(-3.47582,1.44792)(-3.48973,1.40345)\polyline(-3.52111,1.30315)(-3.52222,1.29961)(-3.56016,1.18322)%
\polyline(-3.59400,1.08372)(-3.60895,1.04023)(-3.63662,0.96502)\polyline(-3.67480,0.86711)(-3.68740,0.83575)(-3.72347,0.75558)(-3.72552,0.75166)%
\polyline(-3.77699,0.66013)(-3.78917,0.64116)(-3.81877,0.60719)(-3.84618,0.58864)(-3.87138,0.58530)%
%
%
{%
\color[cmyk]{1,0,0,0}%
\linethickness{0.004in}%%
\polyline(-0.55812,-0.81044)(-0.51152,-0.81601)(-0.46451,-0.81678)(-0.41712,-0.81282)%
(-0.36938,-0.80426)(-0.32135,-0.79128)(-0.27305,-0.77411)(-0.22453,-0.75303)(-0.17582,-0.72836)%
(-0.12697,-0.70046)(-0.07802,-0.66971)(-0.02900,-0.63655)(0.02004,-0.60140)(0.06907,-0.56473)%
(0.11804,-0.52702)(0.16691,-0.48874)(0.21565,-0.45037)(0.26421,-0.41237)(0.31255,-0.37519)%
(0.36064,-0.33929)(0.40843,-0.30506)(0.45588,-0.27289)(0.50297,-0.24311)(0.54964,-0.21605)%
(0.59587,-0.19195)(0.64160,-0.17104)(0.68681,-0.15347)(0.73146,-0.13936)(0.77551,-0.12878)%
(0.81892,-0.12173)(0.86167,-0.11815)(0.90371,-0.11795)(0.94501,-0.12099)(0.98554,-0.12704)%
(1.02526,-0.13588)(1.06414,-0.14719)(1.10215,-0.16067)(1.13925,-0.17593)(1.17543,-0.19258)%
(1.21064,-0.21020)(1.24487,-0.22835)(1.27807,-0.24658)(1.31023,-0.26443)(1.34131,-0.28143)%
(1.37130,-0.29714)(1.40017,-0.31111)(1.42789,-0.32291)(1.45444,-0.33214)(1.47980,-0.33843)%
(1.50395,-0.34144)(1.52687,-0.34086)%
%
\linethickness{0.008in}%%
}%
{%
\color[cmyk]{1,0,0,0}%
\linethickness{0.004in}%%
\polyline(-1.55919,0.27839)(-1.54744,0.26721)(-1.53527,0.25478)(-1.52271,0.24112)%
(-1.50974,0.22623)(-1.49638,0.21012)(-1.48263,0.19283)(-1.46848,0.17436)(-1.45395,0.15476)%
(-1.43904,0.13405)(-1.42375,0.11228)(-1.40808,0.08949)(-1.39205,0.06574)(-1.37565,0.04108)%
(-1.35889,0.01557)(-1.34177,-0.01073)(-1.32429,-0.03775)(-1.30647,-0.06541)(-1.28831,-0.09366)%
(-1.26981,-0.12240)(-1.25097,-0.15155)(-1.23181,-0.18104)(-1.21232,-0.21079)(-1.19251,-0.24069)%
(-1.17239,-0.27068)(-1.15197,-0.30065)(-1.13124,-0.33053)(-1.11021,-0.36022)(-1.08889,-0.38962)%
(-1.06728,-0.41867)(-1.04540,-0.44726)(-1.02324,-0.47530)(-1.00081,-0.50273)(-0.97811,-0.52944)%
(-0.95516,-0.55537)(-0.93196,-0.58043)(-0.90852,-0.60455)(-0.88483,-0.62765)(-0.86091,-0.64967)%
(-0.83677,-0.67054)(-0.81241,-0.69021)(-0.78783,-0.70861)(-0.76305,-0.72570)(-0.73806,-0.74142)%
(-0.71288,-0.75574)(-0.68752,-0.76862)(-0.66197,-0.78002)(-0.63625,-0.78992)(-0.61037,-0.79830)%
(-0.58432,-0.80515)(-0.55812,-0.81044)%
%
\linethickness{0.008in}%%
}%
{%
\color[cmyk]{1,0,0,0}%
\linethickness{0.004in}%%
\polyline(-0.74416,-1.08059)(-0.68074,-1.08714)(-0.61675,-1.08511)(-0.55223,-1.07464)%
(-0.48724,-1.05598)(-0.42184,-1.02950)(-0.35607,-0.99562)(-0.29000,-0.95490)(-0.22369,-0.90794)%
(-0.15718,-0.85542)(-0.09054,-0.79811)(-0.02383,-0.73681)(0.04291,-0.67235)(0.10961,-0.60563)%
(0.17622,-0.53752)(0.24268,-0.46894)(0.30893,-0.40078)(0.37491,-0.33392)(0.44058,-0.26920)%
(0.50587,-0.20743)(0.57073,-0.14936)(0.63511,-0.09569)(0.69894,-0.04703)(0.76218,-0.00392)%
(0.82476,0.03318)(0.88665,0.06392)(0.94778,0.08805)(1.00810,0.10540)(1.06756,0.11591)%
(1.12611,0.11965)(1.18370,0.11675)(1.24028,0.10749)(1.29580,0.09220)(1.35022,0.07134)%
(1.40349,0.04544)(1.45556,0.01510)(1.50639,-0.01900)(1.55594,-0.05611)(1.60416,-0.09544)%
(1.65101,-0.13616)(1.69645,-0.17742)(1.74045,-0.21834)(1.78297,-0.25804)(1.82396,-0.29565)%
(1.86340,-0.33032)(1.90125,-0.36124)(1.93748,-0.38764)(1.97206,-0.40879)(2.00496,-0.42405)%
(2.03615,-0.43283)(2.06560,-0.43464)%
%
\linethickness{0.008in}%%
}%
{%
\color[cmyk]{1,0,0,0}%
\linethickness{0.004in}%%
\polyline(-2.04174,0.61668)(-2.02562,0.59814)(-2.00900,0.57758)(-1.99191,0.55502)%
(-1.97435,0.53049)(-1.95631,0.50403)(-1.93780,0.47569)(-1.91883,0.44552)(-1.89940,0.41358)%
(-1.87952,0.37995)(-1.85919,0.34469)(-1.83841,0.30790)(-1.81720,0.26966)(-1.79555,0.23007)%
(-1.77347,0.18922)(-1.75097,0.14723)(-1.72805,0.10420)(-1.70471,0.06026)(-1.68097,0.01553)%
(-1.65683,-0.02989)(-1.63229,-0.07584)(-1.60736,-0.12221)(-1.58205,-0.16887)(-1.55636,-0.21567)%
(-1.53029,-0.26248)(-1.50387,-0.30917)(-1.47708,-0.35561)(-1.44994,-0.40164)(-1.42245,-0.44715)%
(-1.39462,-0.49199)(-1.36646,-0.53603)(-1.33798,-0.57915)(-1.30917,-0.62121)(-1.28005,-0.66209)%
(-1.25062,-0.70167)(-1.22090,-0.73983)(-1.19088,-0.77647)(-1.16058,-0.81147)(-1.13000,-0.84474)%
(-1.09916,-0.87618)(-1.06805,-0.90569)(-1.03668,-0.93321)(-1.00507,-0.95864)(-0.97321,-0.98193)%
(-0.94113,-1.00301)(-0.90881,-1.02182)(-0.87629,-1.03832)(-0.84355,-1.05248)(-0.81061,-1.06425)%
(-0.77747,-1.07363)(-0.74416,-1.08059)%
%
\linethickness{0.008in}%%
}%
{%
\color[cmyk]{1,0,0,0}%
\linethickness{0.004in}%%
\polyline(-0.93019,-1.35074)(-0.84992,-1.35773)(-0.76891,-1.35126)(-0.68722,-1.33159)%
(-0.60493,-1.29912)(-0.52211,-1.25445)(-0.43883,-1.19829)(-0.35518,-1.13154)(-0.27121,-1.05519)%
(-0.18700,-0.97038)(-0.10263,-0.87834)(-0.01817,-0.78037)(0.06630,-0.67787)(0.15072,-0.57225)%
(0.23500,-0.46498)(0.31908,-0.35751)(0.40288,-0.25129)(0.48633,-0.14774)(0.56936,-0.04822)%
(0.65188,0.04599)(0.73384,0.13369)(0.81515,0.21378)(0.89575,0.28530)(0.97557,0.34741)%
(1.05454,0.39943)(1.13258,0.44082)(1.20963,0.47122)(1.28563,0.49044)(1.36050,0.49844)%
(1.43419,0.49536)(1.50661,0.48153)(1.57773,0.45743)(1.64746,0.42367)(1.71575,0.38106)%
(1.78254,0.33051)(1.84778,0.27306)(1.91140,0.20988)(1.97334,0.14220)(2.03357,0.07135)%
(2.09201,-0.00130)(2.14863,-0.07432)(2.20337,-0.14629)(2.25618,-0.21578)(2.30702,-0.28139)%
(2.35584,-0.34175)(2.40260,-0.39559)(2.44726,-0.44167)(2.48979,-0.47889)(2.53013,-0.50626)%
(2.56827,-0.52289)(2.60415,-0.52807)%
%
\linethickness{0.008in}%%
}%
{%
\color[cmyk]{1,0,0,0}%
\linethickness{0.004in}%%
\polyline(-2.51930,1.08364)(-2.49886,1.05542)(-2.47786,1.02427)(-2.45630,0.99022)%
(-2.43420,0.95335)(-2.41155,0.91372)(-2.38836,0.87140)(-2.36463,0.82650)(-2.34038,0.77912)%
(-2.31560,0.72936)(-2.29031,0.67734)(-2.26450,0.62320)(-2.23818,0.56707)(-2.21137,0.50909)%
(-2.18406,0.44941)(-2.15626,0.38820)(-2.12798,0.32562)(-2.09922,0.26184)(-2.06999,0.19703)%
(-2.04030,0.13137)(-2.01016,0.06505)(-1.97956,-0.00174)(-1.94852,-0.06882)(-1.91704,-0.13600)%
(-1.88514,-0.20309)(-1.85281,-0.26988)(-1.82007,-0.33620)(-1.78692,-0.40184)(-1.75337,-0.46662)%
(-1.71943,-0.53036)(-1.68511,-0.59286)(-1.65040,-0.65395)(-1.61533,-0.71345)(-1.57990,-0.77118)%
(-1.54411,-0.82699)(-1.50798,-0.88070)(-1.47151,-0.93217)(-1.43471,-0.98126)(-1.39759,-1.02781)%
(-1.36016,-1.07170)(-1.32242,-1.11280)(-1.28439,-1.15101)(-1.24607,-1.18622)(-1.20747,-1.21834)%
(-1.16860,-1.24727)(-1.12946,-1.27295)(-1.09008,-1.29532)(-1.05045,-1.31432)(-1.01059,-1.32990)%
(-0.97050,-1.34205)(-0.93019,-1.35074)%
%
\linethickness{0.008in}%%
}%
{%
\color[cmyk]{1,0,0,0}%
\linethickness{0.004in}%%
\polyline(-1.11623,-1.62089)(-1.01892,-1.62776)(-0.92070,-1.61516)(-0.82165,-1.58346)%
(-0.72187,-1.53328)(-0.62145,-1.46549)(-0.52047,-1.38120)(-0.41903,-1.28172)(-0.31722,-1.16856)%
(-0.21512,-1.04342)(-0.11283,-0.90811)(-0.01044,-0.76462)(0.09196,-0.61500)(0.19427,-0.46136)%
(0.29642,-0.30588)(0.39830,-0.15073)(0.49982,0.00196)(0.60090,0.15010)(0.70144,0.29169)%
(0.80136,0.42483)(0.90056,0.54778)(0.99896,0.65895)(1.09647,0.75695)(1.19299,0.84060)%
(1.28846,0.90892)(1.38277,0.96119)(1.47586,0.99694)(1.56762,1.01593)(1.65799,1.01820)%
(1.74687,1.00404)(1.83420,0.97400)(1.91989,0.92885)(2.00387,0.86962)(2.08607,0.79754)%
(2.16640,0.71405)(2.24479,0.62076)(2.32119,0.51945)(2.39551,0.41201)(2.46770,0.30044)%
(2.53769,0.18683)(2.60541,0.07330)(2.67081,-0.03804)(2.73382,-0.14505)(2.79440,-0.24568)%
(2.85248,-0.33794)(2.90802,-0.41995)(2.96096,-0.48996)(3.01126,-0.54640)(3.05888,-0.58786)%
(3.10377,-0.61314)(3.14588,-0.62127)%
%
\linethickness{0.008in}%%
}%
{%
\color[cmyk]{1,0,0,0}%
\linethickness{0.004in}%%
\polyline(-2.99305,1.67737)(-2.96832,1.63709)(-2.94297,1.59283)(-2.91699,1.54467)%
(-2.89039,1.49270)(-2.86317,1.43704)(-2.83534,1.37781)(-2.80691,1.31514)(-2.77788,1.24919)%
(-2.74826,1.18010)(-2.71805,1.10806)(-2.68727,1.03323)(-2.65591,0.95583)(-2.62399,0.87604)%
(-2.59151,0.79408)(-2.55847,0.71016)(-2.52490,0.62452)(-2.49078,0.53738)(-2.45613,0.44898)%
(-2.42096,0.35957)(-2.38528,0.26939)(-2.34909,0.17869)(-2.31239,0.08774)(-2.27521,-0.00323)%
(-2.23754,-0.09394)(-2.19939,-0.18414)(-2.16078,-0.27359)(-2.12170,-0.36201)(-2.08217,-0.44917)%
(-2.04220,-0.53482)(-2.00180,-0.61870)(-1.96097,-0.70059)(-1.91972,-0.78025)(-1.87806,-0.85744)%
(-1.83600,-0.93197)(-1.79355,-1.00361)(-1.75072,-1.07216)(-1.70752,-1.13743)(-1.66395,-1.19924)%
(-1.62003,-1.25741)(-1.57576,-1.31180)(-1.53116,-1.36224)(-1.48623,-1.40860)(-1.44099,-1.45077)%
(-1.39544,-1.48862)(-1.34959,-1.52207)(-1.30345,-1.55104)(-1.25704,-1.57545)(-1.21036,-1.59525)%
(-1.16342,-1.61040)(-1.11623,-1.62089)%
%
\linethickness{0.008in}%%
}%
\linethickness{0.004in}%%
{%
\color[cmyk]{1,0,0,0}%
\linethickness{0.004in}%%
\polyline(-0.18604,-0.27015)(-0.11685,-0.27530)(-0.04583,-0.26839)(0.02592,-0.25229)%
(0.09726,-0.23115)(0.16707,-0.20928)(0.22615,-0.19249)%
%
\linethickness{0.004in}%%
}%
{%
\color[cmyk]{1,0,0,0}%
\put(0.36211,-0.16455){\circle*{0.018143}}\put(0.40721,-0.15760){\circle*{0.018143}}
\put(0.45191,-0.14844){\circle*{0.018143}}\put(0.49474,-0.13284){\circle*{0.018143}}
\put(0.53030,-0.10470){\circle*{0.018143}}\put(0.55387,-0.06596){\circle*{0.018143}}
\put(0.56660,-0.02230){\circle*{0.018143}}\put(0.57112,0.02306){\circle*{0.018143}}
\put(0.56835,0.06861){\circle*{0.018143}}\put(0.55849,0.11310){\circle*{0.018143}}
\put(0.54127,0.15517){\circle*{0.018143}}
\linethickness{0.004in}%%
}%
{%
\color[cmyk]{1,0,0,0}%
\linethickness{0.004in}%%
\polyline(0.45903,0.23075)(0.42332,0.24232)(0.37187,0.25006)(0.31456,0.25494)(0.25229,0.26073)%
(0.18604,0.27015)(0.11685,0.28405)(0.05366,0.29931)%
%
\linethickness{0.004in}%%
}%
{%
\color[cmyk]{1,0,0,0}%
\linethickness{0.004in}%%
\polyline(-0.05392,0.32381)(-0.09726,0.33194)(-0.16707,0.33717)(-0.23424,0.33102)%
(-0.29772,0.31208)(-0.35650,0.28117)(-0.40966,0.24116)(-0.45636,0.19637)(-0.49587,0.15157)%
(-0.52755,0.11091)(-0.55091,0.07698)(-0.56559,0.05033)(-0.57134,0.02941)(-0.56809,0.01112)%
(-0.55587,-0.00833)(-0.53489,-0.03231)(-0.50548,-0.06284)(-0.46809,-0.09993)(-0.42332,-0.14152)%
(-0.37187,-0.18381)(-0.31456,-0.22212)(-0.25229,-0.25199)(-0.18604,-0.27015)%
%
\linethickness{0.004in}%%
}%
{%
\color[cmyk]{1,0,0,0}%
\linethickness{0.004in}%%
\polyline(-0.37208,-0.54030)(-0.23371,-0.54186)(-0.09166,-0.50396)(0.05184,-0.43833)%
(0.11251,-0.40566)%
%
\linethickness{0.004in}%%
}%
{%
\color[cmyk]{1,0,0,0}%
\polyline(0.55461,-0.22282)(0.59544,-0.21489)(0.61214,-0.21491)\polyline(0.67043,-0.21500)(0.71301,-0.21506)(0.72855,-0.21732)%
\polyline(0.78624,-0.22569)(0.81933,-0.23049)(0.84384,-0.23461)\polyline(0.90132,-0.24427)(0.91273,-0.24619)(0.95945,-0.24588)%
\polyline(1.01517,-0.23438)(1.05510,-0.21516)(1.06302,-0.20365)%
%
}%
{%
\color[cmyk]{1,0,0,0}%
\polyline(1.11353,-0.10577)(1.12377,-0.06943)\polyline(1.13143,-0.04027)(1.13473,-0.00266)%
\polyline(1.13738,0.02742)(1.14069,0.06503)%
%
}%
{%
\color[cmyk]{1,0,0,0}%
\polyline(1.13878,0.17273)(1.13681,0.21574)\polyline(1.13249,0.24984)(1.12479,0.29220)%
\polyline(1.11863,0.32609)(1.11174,0.36395)(1.11011,0.36822)%
%
}%
{%
\color[cmyk]{1,0,0,0}%
\polyline(1.07159,0.46916)(1.06978,0.47391)(1.05696,0.49010)\polyline(1.04418,0.50623)(1.02820,0.52640)%
\polyline(1.01542,0.54254)(1.01095,0.54818)(0.99419,0.55610)%
%
}%
{%
\color[cmyk]{1,0,0,0}%
\polyline(0.86453,0.58505)(0.84752,0.58541)\polyline(0.83412,0.58311)(0.81739,0.58002)%
\polyline(0.80401,0.57754)(0.78728,0.57444)%
%
}%
{%
\color[cmyk]{1,0,0,0}%
\linethickness{0.004in}%%
\polyline(0.65039,0.54708)(0.62912,0.54269)(0.50458,0.53021)(0.37208,0.54030)(0.23371,0.57684)%
(0.09166,0.63522)(0.05335,0.65341)%
%
\linethickness{0.004in}%%
}%
{%
\color[cmyk]{1,0,0,0}%
\linethickness{0.004in}%%
\polyline(-0.05340,0.70403)(-0.19453,0.76468)(-0.33414,0.80224)(-0.46848,0.80288)%
(-0.59544,0.76060)(-0.71301,0.67810)(-0.81933,0.56627)(-0.91273,0.44159)(-0.99173,0.32231)%
(-1.05510,0.22404)(-1.10182,0.15619)(-1.13117,0.11982)(-1.14268,0.10767)(-1.13617,0.10617)%
(-1.11174,0.09909)(-1.06978,0.07180)(-1.01095,0.01516)(-0.93617,-0.07198)(-0.84663,-0.18225)%
(-0.74374,-0.30136)(-0.62912,-0.41143)(-0.50458,-0.49523)(-0.37208,-0.54030)%
%
\linethickness{0.004in}%%
}%
{%
\color[cmyk]{1,0,0,0}%
\polyline(1.52687,-0.34086)(1.55264,-0.33510)(1.56611,-0.32858)\polyline(1.59333,-0.31009)(1.59869,-0.30580)(1.61893,-0.28186)(1.62057,-0.27916)%
\polyline(1.63764,-0.25082)(1.65369,-0.21521)(1.65443,-0.21302)%
%
}%
{%
\color[cmyk]{1,0,0,0}%
\polyline(1.68613,-0.09719)(1.69127,-0.07113)(1.69986,-0.01283)(1.70046,-0.00717)%
\polyline(1.70755,0.06543)(1.71106,0.11586)(1.71258,0.15647)\polyline(1.71402,0.22940)(1.71424,0.25593)(1.71298,0.32057)%
%
%
}%
{%
\color[cmyk]{1,0,0,0}%
\polyline(1.70586,0.44836)(1.70398,0.47343)(1.69657,0.54453)(1.69408,0.56278)\polyline(1.68024,0.65376)(1.67578,0.67971)(1.66245,0.74228)(1.65617,0.76621)%
\polyline(1.62982,0.85435)(1.61082,0.90209)(1.58982,0.94439)(1.58104,0.95829)%
%
}%
{%
\color[cmyk]{1,0,0,0}%
\polyline(1.48982,1.04974)(1.48746,1.05118)(1.45743,1.06209)(1.42570,1.06701)(1.40433,1.06653)%
\polyline(1.33520,1.05385)(1.32066,1.04961)(1.28250,1.03484)(1.25330,1.02144)\polyline(1.19034,0.98960)(1.15919,0.97284)(1.11531,0.94879)(1.11286,0.94747)%
\polyline(1.05061,0.91426)(1.02368,0.90033)(0.97604,0.87747)(0.97141,0.87548)%
%
}%
{%
\color[cmyk]{1,0,0,0}%
\linethickness{0.004in}%%
\polyline(0.82168,0.82182)(0.77463,0.81139)(0.72187,0.80408)(0.66825,0.80131)(0.61386,0.80335)%
(0.55874,0.81034)(0.50297,0.82233)(0.44662,0.83925)(0.38974,0.86094)(0.33240,0.88711)%
(0.27468,0.91738)(0.21663,0.95128)(0.15833,0.98824)(0.09984,1.02763)(0.05257,1.06081)%
%
\linethickness{0.004in}%%
}%
{%
\color[cmyk]{1,0,0,0}%
\linethickness{0.004in}%%
\polyline(-0.05243,1.13614)(-0.07604,1.15318)(-0.13458,1.19488)(-0.19296,1.23517)%
(-0.25112,1.27324)(-0.30899,1.30831)(-0.36649,1.33964)(-0.42356,1.36655)(-0.48014,1.38841)%
(-0.53615,1.40469)(-0.59154,1.41491)(-0.64624,1.41873)(-0.70018,1.41588)(-0.75330,1.40621)%
(-0.80553,1.38970)(-0.85683,1.36641)(-0.90712,1.33653)(-0.95635,1.30036)(-1.00446,1.25829)%
(-1.05139,1.21081)(-1.09710,1.15851)(-1.14152,1.10203)(-1.18460,1.04211)(-1.22630,0.97951)%
(-1.26656,0.91504)(-1.30533,0.84953)(-1.34258,0.78383)(-1.37826,0.71876)(-1.41233,0.65513)%
(-1.44474,0.59372)(-1.47546,0.53524)(-1.50446,0.48036)(-1.53169,0.42964)(-1.55713,0.38359)%
(-1.57280,0.35638)%
%
\linethickness{0.004in}%%
}%
{%
\color[cmyk]{1,0,0,0}%
\polyline(-1.63041,0.26598)(-1.64018,0.25270)\polyline(-1.64871,0.24265)(-1.65647,0.23353)(-1.65974,0.23043)%
\polyline(-1.66930,0.22135)(-1.67060,0.22011)(-1.68278,0.21190)%
%
}%
{%
\color[cmyk]{1,0,0,0}%
\polyline(2.06560,-0.43464)(2.09733,-0.42757)(2.11671,-0.41614)\polyline(2.14923,-0.38695)(2.15365,-0.38239)(2.17818,-0.34389)(2.17941,-0.34115)%
\polyline(2.19746,-0.30102)(2.20024,-0.29484)(2.21534,-0.24903)%
%
}%
{%
\color[cmyk]{1,0,0,0}%
\polyline(2.24564,-0.11898)(2.25141,-0.08782)(2.26011,-0.02474)\polyline(2.26859,0.05107)(2.27283,0.09354)(2.27640,0.14610)%
\polyline(2.28081,0.22226)(2.28398,0.30068)(2.28424,0.31756)\polyline(2.28542,0.39384)(2.28568,0.41093)(2.28507,0.48920)%
%
%
}%
{%
\color[cmyk]{1,0,0,0}%
\polyline(2.28071,0.64986)(2.27529,0.75224)(2.27408,0.76809)\polyline(2.26685,0.86255)(2.26668,0.86484)(2.25550,0.97453)(2.25474,0.98035)%
\polyline(2.24251,1.07429)(2.24177,1.07996)(2.22551,1.17989)(2.22325,1.19113)\polyline(2.20405,1.28389)(2.18546,1.35863)(2.17324,1.39819)%
\polyline(2.14154,1.48741)(2.13553,1.50289)(2.10693,1.56028)(2.08616,1.59173)%
%
}%
{%
\color[cmyk]{1,0,0,0}%
\polyline(1.79199,1.64220)(1.74709,1.61379)(1.71274,1.58852)\polyline(1.65262,1.54105)(1.64410,1.53415)(1.58979,1.48763)(1.57979,1.47885)%
\polyline(1.52221,1.42830)(1.47584,1.38766)(1.44988,1.36552)\polyline(1.39106,1.31643)(1.35522,1.28739)(1.31549,1.25761)%
\polyline(1.25278,1.21362)(1.22847,1.19724)(1.17109,1.16370)%
%
}%
{%
\color[cmyk]{1,0,0,0}%
\linethickness{0.004in}%%
\polyline(1.01710,1.09827)(0.95891,1.08323)(0.88862,1.07356)(0.81732,1.07254)(0.74510,1.08043)%
(0.67204,1.09732)(0.59821,1.12310)(0.52371,1.15748)(0.44862,1.20001)(0.37302,1.25006)%
(0.29700,1.30683)(0.22064,1.36940)(0.14403,1.43671)(0.06726,1.50760)(0.05091,1.52318)%
%
\linethickness{0.004in}%%
}%
{%
\color[cmyk]{1,0,0,0}%
\linethickness{0.004in}%%
\polyline(-0.05064,1.62047)(-0.08642,1.65504)(-0.16315,1.72893)(-0.23971,1.80110)%
(-0.31599,1.87020)(-0.39191,1.93491)(-0.46739,1.99394)(-0.54235,2.04611)(-0.61669,2.09034)%
(-0.69033,2.12567)(-0.76319,2.15126)(-0.83519,2.16645)(-0.90625,2.17074)(-0.97628,2.16380)%
(-1.04520,2.14548)(-1.11295,2.11584)(-1.17944,2.07511)(-1.24459,2.02368)(-1.30834,1.96214)%
(-1.37061,1.89125)(-1.43133,1.81190)(-1.49043,1.72512)(-1.54784,1.63207)(-1.60351,1.53399)%
(-1.65736,1.43222)(-1.70934,1.32811)(-1.75939,1.22308)(-1.80745,1.11852)(-1.85346,1.01583)%
(-1.89738,0.91632)(-1.93915,0.82127)(-1.97874,0.73184)(-2.00865,0.66556)%
%
\linethickness{0.004in}%%
}%
{%
\color[cmyk]{1,0,0,0}%
\polyline(-2.05558,0.56494)(-2.06940,0.53678)\polyline(-2.08045,0.51425)(-2.08390,0.50722)(-2.09486,0.48640)%
\polyline(-2.10654,0.46419)(-2.11429,0.44947)(-2.12168,0.43673)%
%
}%
{%
\color[cmyk]{1,0,0,0}%
\polyline(-2.19148,0.33338)(-2.19915,0.32629)\polyline(-2.20528,0.32063)(-2.21165,0.31474)(-2.21321,0.31391)%
\polyline(-2.22058,0.30999)(-2.22980,0.30508)%
%
}%
{%
\color[cmyk]{1,0,0,0}%
\polyline(2.60415,-0.52807)(2.64179,-0.51950)(2.66265,-0.50488)\polyline(2.69772,-0.46817)(2.70826,-0.45475)(2.72959,-0.41285)%
\polyline(2.74870,-0.36539)(2.76276,-0.32585)(2.76830,-0.30445)%
%
}%
{%
\color[cmyk]{1,0,0,0}%
\polyline(2.80047,-0.15977)(2.80505,-0.13588)(2.81832,-0.04286)\polyline(2.82901,0.05115)(2.83494,0.10791)(2.83945,0.16897)%
\polyline(2.84599,0.26337)(2.85167,0.38152)\polyline(2.85438,0.47611)(2.85625,0.54967)(2.85647,0.59437)%
%
%
}%
{%
\color[cmyk]{1,0,0,0}%
\linethickness{0.004in}%%
\polyline(2.85616,0.77125)(2.85468,0.87380)(2.84916,1.03865)(2.84050,1.20253)(2.82869,1.36333)%
(2.81375,1.51899)(2.79571,1.66753)(2.77457,1.80711)(2.75037,1.93602)(2.72313,2.05276)%
(2.69288,2.15600)(2.65965,2.24468)(2.62348,2.31795)(2.58441,2.37527)(2.54248,2.41633)%
(2.49774,2.44110)(2.45024,2.44986)(2.40004,2.44311)(2.34718,2.42166)(2.29172,2.38653)%
(2.23374,2.33899)(2.17328,2.28048)(2.11042,2.21266)(2.04523,2.13731)(1.97777,2.05634)%
(1.90813,1.97173)(1.83639,1.88550)(1.76261,1.79970)(1.68688,1.71633)(1.60929,1.63735)%
(1.59997,1.62881)%
%
\linethickness{0.004in}%%
}%
{%
\color[cmyk]{1,0,0,0}%
\linethickness{0.004in}%%
\polyline(1.32774,1.42422)(1.28202,1.40008)(1.19643,1.36768)(1.10952,1.34820)(1.02137,1.34231)%
(0.93211,1.35041)(0.84181,1.37263)(0.75058,1.40882)(0.65852,1.45855)(0.56573,1.52115)%
(0.47232,1.59567)(0.37838,1.68094)(0.28403,1.77558)(0.18936,1.87800)(0.09449,1.98646)%
(0.04881,2.04062)%
%
\linethickness{0.004in}%%
}%
{%
\color[cmyk]{1,0,0,0}%
\linethickness{0.004in}%%
\polyline(-0.04830,2.15686)(-0.09548,2.21389)(-0.19035,2.32884)(-0.28502,2.44187)%
(-0.37936,2.55091)(-0.47329,2.65395)(-0.56670,2.74905)(-0.65948,2.83438)(-0.75153,2.90828)%
(-0.84275,2.96923)(-0.93304,3.01595)(-1.02230,3.04737)(-1.11043,3.06266)(-1.19733,3.06125)%
(-1.28290,3.04286)(-1.36706,3.00747)(-1.44971,2.95534)(-1.53075,2.88700)(-1.61011,2.80325)%
(-1.68768,2.70513)(-1.76339,2.59393)(-1.83715,2.47114)(-1.90887,2.33842)(-1.97849,2.19760)%
(-2.04592,2.05064)(-2.11109,1.89959)(-2.17392,1.74652)(-2.23435,1.59357)(-2.29231,1.44282)%
(-2.34774,1.29631)(-2.40057,1.15600)(-2.43227,1.07244)%
%
\linethickness{0.004in}%%
}%
{%
\color[cmyk]{1,0,0,0}%
\polyline(-2.48441,0.93677)(-2.49607,0.90664)\polyline(-2.50563,0.88262)(-2.51766,0.85264)%
\polyline(-2.52728,0.82865)(-2.53931,0.79866)%
%
}%
{%
\color[cmyk]{1,0,0,0}%
\polyline(-2.59592,0.66636)(-2.61394,0.62690)\polyline(-2.62875,0.59552)(-2.64838,0.55684)%
\polyline(-2.66457,0.52616)(-2.68657,0.48877)%
%
}%
{%
\color[cmyk]{1,0,0,0}%
\polyline(-2.77410,0.39867)(-2.77425,0.39860)\polyline(-2.77438,0.39854)(-2.77453,0.39846)%
\polyline(-2.77466,0.39840)(-2.77481,0.39833)%
%
}%
{%
\color[cmyk]{1,0,0,0}%
\polyline(3.14588,-0.62127)(3.18910,-0.60946)(3.20909,-0.59225)\polyline(3.24543,-0.54919)(3.26511,-0.51828)(3.27825,-0.48634)%
\polyline(3.29933,-0.43347)(3.31894,-0.36504)%
%
}%
{%
\color[cmyk]{1,0,0,0}%
\linethickness{0.004in}%%
\polyline(3.35186,-0.21710)(3.35243,-0.21439)(3.37429,-0.07164)(3.39248,0.08951)(3.40699,0.26715)%
(3.41780,0.45912)(3.42489,0.66300)(3.42827,0.87616)(3.42791,1.09584)(3.42384,1.31913)%
(3.41604,1.54308)(3.40454,1.76471)(3.38933,1.98109)(3.37045,2.18939)(3.34790,2.38688)%
(3.32171,2.57104)(3.29192,2.73958)(3.25854,2.89044)(3.22163,3.02190)(3.18122,3.13253)%
(3.13735,3.22127)(3.09008,3.28744)(3.03944,3.33072)(2.98551,3.35118)(2.92833,3.34927)%
(2.86797,3.32582)(2.80449,3.28201)(2.73797,3.21937)(2.66847,3.13972)(2.59607,3.04518)%
(2.52086,2.93808)(2.44290,2.82098)(2.36229,2.69657)(2.27912,2.56766)(2.19346,2.43711)%
(2.10543,2.30779)(2.01511,2.18250)(1.92259,2.06400)(1.82799,1.95485)(1.73141,1.85745)%
(1.63294,1.77398)(1.53270,1.70634)(1.43079,1.65611)(1.32733,1.62458)(1.22242,1.61266)%
(1.11619,1.62090)(1.00875,1.64946)(0.90021,1.69815)(0.79069,1.76636)(0.68031,1.85314)%
(0.56920,1.95719)(0.45746,2.07686)(0.34523,2.21022)(0.23262,2.35505)(0.11976,2.50892)%
(0.04667,2.61262)%
%
\linethickness{0.004in}%%
}%
{%
\color[cmyk]{1,0,0,0}%
\linethickness{0.004in}%%
\polyline(-0.04606,2.74586)(-0.10622,2.83315)(-0.21910,2.99788)(-0.33175,3.16050)%
(-0.44403,3.31811)(-0.55583,3.46787)(-0.66703,3.60705)(-0.77750,3.73307)(-0.88713,3.84354)%
(-0.99579,3.93633)(-1.10337,4.00958)(-1.20976,4.06172)(-1.31482,4.09154)(-1.41846,4.09820)%
(-1.52056,4.08119)(-1.62101,4.04044)(-1.71970,3.97622)(-1.81652,3.88919)(-1.91136,3.78041)%
(-2.00413,3.65125)(-2.09472,3.50342)(-2.18303,3.33896)(-2.26898,3.16014)(-2.35245,2.96946)%
(-2.43338,2.76962)(-2.51165,2.56344)(-2.58720,2.35384)(-2.65994,2.14378)(-2.72979,1.93619)%
(-2.79668,1.73396)(-2.84786,1.57836)%
%
\linethickness{0.004in}%%
}%
{%
\color[cmyk]{1,0,0,0}%
\polyline(-2.90454,1.40698)(-2.91573,1.37321)\polyline(-2.92474,1.34622)(-2.93614,1.31252)%
\polyline(-2.94525,1.28556)(-2.95665,1.25186)%
%
}%
{%
\color[cmyk]{1,0,0,0}%
\polyline(-3.00707,1.10574)(-3.03315,1.03137)(-3.04379,1.00264)\polyline(-3.07422,0.92054)(-3.08418,0.89365)(-3.11423,0.81868)%
\polyline(-3.14827,0.73803)(-3.17614,0.67574)(-3.19523,0.63922)\polyline(-3.23931,0.56370)(-3.25430,0.54091)(-3.28810,0.50566)(-3.31833,0.49162)%
%
%
}%
{%
\color[cmyk]{1,0,0,0}%
\polyline(-1.68278,0.21190)(-1.68671,0.21014)(-1.69038,0.20900)(-1.69379,0.20846)(-1.69694,0.20849)(-1.69983,0.20908)(-1.70246,0.21019)(-1.70482,0.21180)(-1.70692,0.21389)(-1.70876,0.21641)(-1.71034,0.21934)(-1.71165,0.22264)(-1.71270,0.22630)(-1.71348,0.23026)(-1.71400,0.23451)(-1.71426,0.23899)(-1.71425,0.24368)(-1.71411,0.24616)%
\polyline(-1.70072,0.29197)(-1.69966,0.29388)(-1.69675,0.29841)(-1.69359,0.30271)(-1.69016,0.30674)(-1.68647,0.31047)(-1.68253,0.31386)(-1.67832,0.31688)(-1.67385,0.31950)(-1.66913,0.32169)(-1.66415,0.32342)(-1.65891,0.32466)(-1.65366,0.32536)%
\polyline(-1.60757,0.31389)(-1.60039,0.30988)(-1.59263,0.30500)(-1.58464,0.29942)(-1.57640,0.29313)(-1.56792,0.28612)(-1.55919,0.27839)%
%
%
}%
\color[cmyk]{1,0,0,0}%
\polyline(-2.22980,0.30508)(-2.23650,0.30374)(-2.24278,0.30386)(-2.24864,0.30540)(-2.25406,0.30831)(-2.25907,0.31254)(-2.26364,0.31802)(-2.26779,0.32471)(-2.27150,0.33253)(-2.27190,0.33362)%
\polyline(-2.28254,0.37759)(-2.28362,0.38616)(-2.28475,0.39926)(-2.28545,0.41295)(-2.28571,0.42715)(-2.28563,0.43416)%
\polyline(-2.28319,0.47944)(-2.28244,0.48731)(-2.28054,0.50274)(-2.27821,0.51813)(-2.27545,0.53341)(-2.27501,0.53552)%
%
%
\polyline(-2.21957,0.66550)(-2.21683,0.66853)(-2.20896,0.67570)(-2.20067,0.68170)(-2.19197,0.68648)(-2.18285,0.68997)(-2.18150,0.69027)%
\polyline(-2.14509,0.69063)(-2.14225,0.69006)(-2.13108,0.68641)(-2.11951,0.68122)(-2.10754,0.67447)(-2.10342,0.67169)%
\polyline(-2.07436,0.64913)(-2.06924,0.64464)(-2.05569,0.63147)(-2.04174,0.61668)%
%
%
\color[cmyk]{0,0,0,1}%
{%
\color[cmyk]{1,0,0,0}%
\polyline(-2.77481,0.39833)(-2.78448,0.39780)(-2.79355,0.39997)(-2.80203,0.40476)%
(-2.80990,0.41209)(-2.81717,0.42186)(-2.82384,0.43398)(-2.82991,0.44835)(-2.83536,0.46483)%
(-2.84021,0.48332)(-2.84446,0.50368)(-2.84809,0.52577)(-2.85111,0.54946)(-2.85352,0.57459)%
(-2.85532,0.60101)(-2.85651,0.62856)(-2.85708,0.65708)(-2.85705,0.68641)(-2.85652,0.71065)%
%
}%
{%
\color[cmyk]{1,0,0,0}%
\polyline(-2.84047,0.89437)(-2.83967,0.90000)(-2.83475,0.92964)(-2.83184,0.94485)%
\polyline(-2.82343,0.98495)(-2.82308,0.98655)(-2.81634,1.01350)(-2.81036,1.03447)%
\polyline(-2.79741,1.07333)(-2.79251,1.08642)(-2.78336,1.10759)(-2.77699,1.12026)%
%
%
}%
{%
\color[cmyk]{1,0,0,0}%
\polyline(-2.69337,1.20078)(-2.68893,1.20189)(-2.67450,1.20271)(-2.65951,1.20083)(-2.64634,1.19691)%
\polyline(-2.61247,1.17926)(-2.61111,1.17842)(-2.59385,1.16525)(-2.57603,1.14918)(-2.57563,1.14877)%
\polyline(-2.54950,1.12078)(-2.53876,1.10837)(-2.51930,1.08364)%
%
}%
\color[cmyk]{1,0,0,0}%
\polyline(-3.31833,0.49162)(-3.33111,0.49215)(-3.34312,0.49700)(-3.35435,0.50609)%
(-3.36480,0.51927)(-3.37446,0.53642)(-3.38334,0.55738)(-3.39142,0.58198)(-3.39871,0.61003)%
(-3.40522,0.64135)(-3.41092,0.67573)(-3.41583,0.71294)(-3.41994,0.75275)(-3.42326,0.79493)%
(-3.42577,0.83923)(-3.42749,0.88539)(-3.42840,0.93314)(-3.42852,0.98222)(-3.42784,1.03235)%
(-3.42635,1.08325)(-3.42407,1.13464)(-3.42099,1.18624)(-3.41711,1.23777)(-3.41243,1.28893)%
(-3.40695,1.33945)(-3.40069,1.38905)(-3.39362,1.43745)(-3.38577,1.48438)(-3.37712,1.52957)%
(-3.36769,1.57277)(-3.35747,1.61373)(-3.34647,1.65219)(-3.33468,1.68793)(-3.32212,1.72073)%
(-3.30879,1.75037)(-3.29468,1.77666)(-3.27980,1.79940)(-3.26605,1.81613)%
%
\polyline(-3.16931,1.85126)(-3.15444,1.84680)(-3.13355,1.83626)(-3.11193,1.82118)%
(-3.08959,1.80153)(-3.06652,1.77731)(-3.04274,1.74853)(-3.01824,1.71520)(-2.99305,1.67737)%
%
\color[cmyk]{0,0,0,1}%
\polyline(0.00000,0.00000)(0.00000,5.19615)%
%
\polyline(1.62784,2.36380)(-1.62784,-2.36380)%
%
\polyline(-4.50000,0.77474)(-3.99816,0.68834)%
%
\polyline(-3.87031,0.66633)(-3.77908,0.65062)\polyline(-3.70609,0.63805)(-3.61486,0.62235)%
\polyline(-3.54187,0.60978)(-3.45064,0.59407)%
%
\polyline(-3.32851,0.57305)(-3.23504,0.55696)\polyline(-3.16026,0.54408)(-3.06679,0.52799)%
\polyline(-2.99201,0.51511)(-2.89854,0.49902)%
%
\polyline(-2.66958,0.45961)(-2.63435,0.45354)\polyline(-2.60617,0.44869)(-2.57093,0.44262)%
\polyline(-2.54275,0.43777)(-2.50752,0.43170)%
%
\polyline(-2.38207,0.41011)(-2.34486,0.40370)\polyline(-2.31510,0.39858)(-2.27789,0.39217)%
\polyline(-2.24813,0.38705)(-2.21092,0.38064)%
%
\polyline(-2.08080,0.35824)(-2.04397,0.35190)\polyline(-2.01451,0.34683)(-1.97767,0.34048)%
\polyline(-1.94821,0.33541)(-1.91138,0.32907)%
%
\polyline(-1.77320,0.30528)(-1.74148,0.29982)\polyline(-1.71610,0.29545)(-1.68437,0.28999)%
\polyline(-1.65899,0.28562)(-1.62727,0.28016)%
%
\polyline(-1.46685,0.25254)(0.29894,-0.05147)%
%
\polyline(0.85693,-0.14753)(0.91124,-0.15688)\polyline(0.95469,-0.16436)(1.00900,-0.17371)%
\polyline(1.05245,-0.18119)(1.10676,-0.19054)%
%
\polyline(1.30500,-0.22467)(1.40532,-0.24194)\polyline(1.48557,-0.25576)(1.58588,-0.27303)%
\polyline(1.66613,-0.28685)(1.76645,-0.30412)%
%
\polyline(1.93425,-0.33301)(2.02241,-0.34819)\polyline(2.09294,-0.36033)(2.18111,-0.37551)%
\polyline(2.25164,-0.38765)(2.33980,-0.40283)%
%
\polyline(2.50235,-0.43081)(2.58858,-0.44566)\polyline(2.65756,-0.45753)(2.74379,-0.47238)%
\polyline(2.81277,-0.48425)(2.89900,-0.49910)%
%
\polyline(3.05853,-0.52657)(3.14389,-0.54127)\polyline(3.21218,-0.55302)(3.29755,-0.56772)%
\polyline(3.36584,-0.57947)(3.45120,-0.59417)%
%
\polyline(3.60893,-0.62133)(4.50000,-0.77474)%
%
\linethickness{0.008in}%%
\polyline(-0.36065,-1.24562)(-0.23940,-1.07904)(-0.13664,-0.94046)(-0.04804,-0.82362)%
(0.02938,-0.72422)(0.09769,-0.63919)(0.15840,-0.56632)(0.21256,-0.50393)(0.26098,-0.45072)%
(0.30427,-0.40561)(0.34289,-0.36767)(0.37724,-0.33607)(0.40768,-0.31003)(0.43452,-0.28882)%
(0.45807,-0.27177)(0.47864,-0.25822)(0.49655,-0.24758)(0.51210,-0.23933)(0.52560,-0.23298)%
(0.53738,-0.22813)(0.54775,-0.22443)(0.55700,-0.22162)(0.56544,-0.21947)(0.57334,-0.21782)%
(0.58100,-0.21654)(0.58867,-0.21556)(0.59661,-0.21482)(0.60509,-0.21431)(0.61437,-0.21403)%
(0.62471,-0.21398)(0.63638,-0.21421)(0.64968,-0.21475)(0.66495,-0.21567)(0.68254,-0.21701)%
(0.70289,-0.21887)(0.72649,-0.22133)(0.75395,-0.22451)(0.78602,-0.22854)(0.82361,-0.23358)%
(0.86793,-0.23985)(0.92051,-0.24762)(0.98341,-0.25725)(1.05942,-0.26922)(1.15243,-0.28420)%
(1.26806,-0.30316)(1.41471,-0.32756)(1.60551,-0.35966)(1.86228,-0.40321)(2.22394,-0.46491)%
(2.76758,-0.55805)(3.67020,-0.71308)%
%
\polyline(-3.87138,0.58530)(-3.26399,0.48053)(-2.83117,0.40577)(-2.76428,0.39419)%
%
\polyline(-2.64379,0.37334)(-2.50808,0.34986)(-2.47155,0.34352)%
%
\polyline(-2.34928,0.32232)(-2.25847,0.30657)(-2.17034,0.29124)%
%
\polyline(-2.04477,0.26939)(-1.90037,0.24418)(-1.85843,0.23683)%
%
\polyline(-1.72764,0.21385)(-1.65898,0.20175)(-1.56671,0.18539)(-1.53468,0.17966)%
%
\polyline(-1.39104,0.15380)(-1.36475,0.14902)(-1.31581,0.14002)(-1.27383,0.13221)%
(-1.23789,0.12543)(-1.20727,0.11957)(-1.18137,0.11452)(-1.15973,0.11023)(-1.14197,0.10663)%
(-1.12779,0.10370)(-1.11695,0.10140)(-1.10927,0.09973)(-1.10463,0.09869)(-1.10410,0.09858)%
(-1.10817,0.09958)(-1.11513,0.10137)(-1.12506,0.10403)(-1.13805,0.10765)(-1.15424,0.11237)%
(-1.17382,0.11835)(-1.19702,0.12581)(-1.22413,0.13500)(-1.25551,0.14625)(-1.29160,0.15998)%
(-1.33294,0.17670)(-1.38018,0.19711)(-1.43411,0.22206)(-1.49572,0.25269)(-1.56622,0.29051)%
(-1.64712,0.33753)(-1.74034,0.39647)(-1.84832,0.47110)(-1.97424,0.56672)(-2.12225,0.69094)%
(-2.29796,0.85497)(-2.50909,1.07584)(-2.76650,1.38024)(-3.08607,1.81183)(-3.49189,2.44568)%
%
\polyline(-1.30227,-1.89104)(-1.21133,-1.89815)(-1.11970,-1.88835)(-1.02742,-1.86187)%
(-0.93455,-1.81904)(-0.84115,-1.76037)(-0.74727,-1.68648)(-0.65295,-1.59814)(-0.55827,-1.49622)%
(-0.46326,-1.38172)(-0.36798,-1.25575)(-0.27250,-1.11950)(-0.17686,-0.97425)(-0.08112,-0.82136)%
(0.01467,-0.66225)(0.11045,-0.49838)(0.20617,-0.33124)(0.30177,-0.16235)(0.39720,0.00677)%
(0.49239,0.17460)(0.58731,0.33965)(0.68189,0.50047)(0.77608,0.65563)(0.86982,0.80380)%
(0.96306,0.94370)(1.05575,1.07415)(1.14784,1.19404)(1.23927,1.30240)(1.32998,1.39835)%
(1.41994,1.48115)(1.50908,1.55018)(1.59735,1.60495)(1.68471,1.64512)(1.77110,1.67050)%
(1.85648,1.68103)(1.94079,1.67680)(2.02399,1.65804)(2.10603,1.62514)(2.18686,1.57859)%
(2.26643,1.51906)(2.34471,1.44732)(2.42164,1.36425)(2.49718,1.27088)(2.57129,1.16831)%
(2.64393,1.05773)(2.71505,0.94044)(2.78461,0.81778)(2.85258,0.69116)(2.91890,0.56203)%
(2.98356,0.43188)(3.04650,0.30220)(3.10770,0.17449)(3.16712,0.05024)(3.22471,-0.06910)%
(3.28046,-0.18208)(3.33433,-0.28736)(3.38629,-0.38361)(3.43630,-0.46962)(3.48434,-0.54423)%
(3.53038,-0.60642)(3.57440,-0.65525)(3.61637,-0.68990)(3.65627,-0.70969)(3.69407,-0.71407)%
(3.72975,-0.70261)(3.76329,-0.67503)(3.79467,-0.63121)(3.82388,-0.57116)(3.85089,-0.49504)%
(3.87570,-0.40314)(3.89828,-0.29592)(3.91862,-0.17395)(3.93672,-0.03796)(3.95256,0.11122)%
(3.96614,0.27262)(3.97744,0.44517)(3.98646,0.62770)(3.99319,0.81894)(3.99763,1.01755)%
(3.99978,1.22212)(3.99964,1.43118)(3.99720,1.64324)(3.99247,1.85675)(3.98545,2.07018)%
(3.97614,2.28197)(3.96456,2.49059)(3.95069,2.69454)(3.93457,2.89236)(3.91619,3.08263)%
(3.89556,3.26402)(3.87270,3.43526)(3.84761,3.59520)(3.82032,3.74274)(3.79084,3.87695)%
(3.75919,3.99697)(3.72537,4.10210)(3.68943,4.19176)(3.65136,4.26550)(3.61121,4.32304)%
(3.56898,4.36421)(3.52470,4.38900)(3.47840,4.39757)(3.43011,4.39017)(3.37985,4.36725)%
(3.32766,4.32937)(3.27355,4.27720)(3.21757,4.21158)(3.15974,4.13343)(3.10010,4.04381)%
(3.03868,3.94385)(2.97552,3.83479)(2.91065,3.71794)(2.84411,3.59468)(2.77594,3.46642)%
(2.70618,3.33464)(2.63487,3.20082)(2.56205,3.06647)(2.48775,2.93308)(2.41203,2.80215)%
(2.33493,2.67512)(2.25649,2.55340)(2.17675,2.43834)(2.09577,2.33123)(2.01358,2.23327)%
(1.93024,2.14557)(1.84579,2.06914)(1.76029,2.00487)(1.67377,1.95353)(1.58629,1.91578)%
(1.49791,1.89212)(1.40866,1.88294)(1.31861,1.88846)(1.22780,1.90879)(1.13629,1.94385)%
(1.04412,1.99346)(0.95136,2.05726)(0.85805,2.13478)(0.76425,2.22540)(0.67001,2.32837)%
(0.57538,2.44283)(0.48043,2.56777)(0.38520,2.70213)(0.28975,2.84470)(0.19413,2.99421)%
(0.09840,3.14933)(0.04392,3.23995)%
%
\polyline(-0.04331,3.38632)(-0.09317,3.47067)(-0.18890,3.63396)(-0.28452,3.79697)%
(-0.37999,3.95821)(-0.47523,4.11614)(-0.57020,4.26928)(-0.66484,4.41618)(-0.75911,4.55541)%
(-0.85293,4.68564)(-0.94627,4.80558)(-1.03907,4.91404)(-1.13126,5.00992)(-1.22282,5.09224)%
(-1.31366,5.16012)(-1.40376,5.21280)(-1.49305,5.24966)(-1.58149,5.27022)(-1.66901,5.27412)%
(-1.75558,5.26117)(-1.84115,5.23131)(-1.92565,5.18462)(-2.00906,5.12135)(-2.09131,5.04186)%
(-2.17236,4.94667)(-2.25216,4.83645)(-2.33068,4.71197)(-2.40786,4.57413)(-2.48365,4.42397)%
(-2.55802,4.26259)(-2.63093,4.09123)(-2.70232,3.91119)(-2.77217,3.72384)(-2.84043,3.53062)%
(-2.90705,3.33301)(-2.97202,3.13253)(-3.03527,2.93072)(-3.09679,2.72911)(-3.15652,2.52924)%
(-3.21445,2.33262)(-3.25952,2.17843)%
%
%
\end{picture}}%}}
\putnotee{8}{37}{A}
\putnotee{65}{37}{B}
\putnotee{8}{55}{C}
\putnotee{65}{54}{D}
\end{layer}

\begin{itemize}
\item
[課題]\monban 次のグラフとなる2変数関数を選べ\\
\hspace{1zw}1\ $z=\sqrt{1-y^2}$\hspace{2zw}2\ $z=\sqrt{1-x^2}$\\
\hspace{1zw}3\ $z=\bunsuu{x^2y^2}{x^2+y^2}$\hspace{2zw}4\ $z=\sqrt{1-x^2-\bunsuu{y^2}{4}}$
\end{itemize}
%%%%%%%%%%%%

%%%%%%%%%%%%%%%%%%%%


\newslide{2変数関数の微分}

\vspace*{18mm}


\begin{layer}{120}{0}
\putnotew{96}{73}{\hyperlink{para2pg1}{\fbox{\Ctab{2.5mm}{\scalebox{1}{\scriptsize $\mathstrut||\!\lhd$}}}}}
\putnotew{101}{73}{\hyperlink{para3pg1}{\fbox{\Ctab{2.5mm}{\scalebox{1}{\scriptsize $\mathstrut|\!\lhd$}}}}}
\putnotew{108}{73}{\hyperlink{para3pg7}{\fbox{\Ctab{4.5mm}{\scalebox{1}{\scriptsize $\mathstrut\!\!\lhd\!\!$}}}}}
\putnotew{115}{73}{\hyperlink{para3pg8}{\fbox{\Ctab{4.5mm}{\scalebox{1}{\scriptsize $\mathstrut\!\rhd\!$}}}}}
\putnotew{120}{73}{\hyperlink{para3pg8}{\fbox{\Ctab{2.5mm}{\scalebox{1}{\scriptsize $\mathstrut \!\rhd\!\!|$}}}}}
\putnotew{125}{73}{\hyperlink{para4pg1}{\fbox{\Ctab{2.5mm}{\scalebox{1}{\scriptsize $\mathstrut \!\rhd\!\!||$}}}}}
\putnotee{126}{73}{\scriptsize\color{blue} 8/8}
\end{layer}

\slidepage
\begin{itemize}
\item
2変数関数$z=f(x,\ y)$\\
 例えば $z=x^2+3y$
\item
$x$だけを変数と考えて($y$は定数とみて)\\
$z$を$x$で微分したものを$x$についての{\color{red}偏微分}といい\\
\hspace*{5zw}{\color{red}$\bunsuu{\partial z}{\partial x}$}\\
と書く.
\item
$y$についての偏微分$\bunsuu{\partial z}{\partial y}$も同様
\item
[注)]$z_x,\ z_y$とも書く.
\item
[注)]$z'$とは書かない.
\end{itemize}

\newslide{偏微分の計算}

\vspace*{18mm}


\begin{layer}{120}{0}
\putnotew{96}{73}{\hyperlink{para3pg8}{\fbox{\Ctab{2.5mm}{\scalebox{1}{\scriptsize $\mathstrut||\!\lhd$}}}}}
\putnotew{101}{73}{\hyperlink{para4pg1}{\fbox{\Ctab{2.5mm}{\scalebox{1}{\scriptsize $\mathstrut|\!\lhd$}}}}}
\putnotew{108}{73}{\hyperlink{para4pg6}{\fbox{\Ctab{4.5mm}{\scalebox{1}{\scriptsize $\mathstrut\!\!\lhd\!\!$}}}}}
\putnotew{115}{73}{\hyperlink{para4pg7}{\fbox{\Ctab{4.5mm}{\scalebox{1}{\scriptsize $\mathstrut\!\rhd\!$}}}}}
\putnotew{120}{73}{\hyperlink{para4pg7}{\fbox{\Ctab{2.5mm}{\scalebox{1}{\scriptsize $\mathstrut \!\rhd\!\!|$}}}}}
\putnotew{125}{73}{\hyperlink{para5pg1}{\fbox{\Ctab{2.5mm}{\scalebox{1}{\scriptsize $\mathstrut \!\rhd\!\!||$}}}}}
\putnotee{126}{73}{\scriptsize\color{blue} 7/7}
\end{layer}

\slidepage
\begin{itemize}
\item
[例)]$z=x^3+y^2+x^4y^5$
\item
$\dpar{z}{x}=\dpar{}{x}(x^3+{\color{blue}y^2}+x^4{\color{blue}y^5})$\\
$\phantom{\dpar{z}{x}}=\dpar{}{x}(x^3)+\dpar{}{x}({\color{blue}y^2})+\dpar{}{x}(x^4 \color{blue}y^5)$\\
$\phantom{\dpar{z}{x}}=3x^2+4x^3{\color{blue}y^5}$
\item
$\dpar{z}{y}=\dpar{}{x}({\color{blue}x^3}+y^2+{\color{blue}x^4}y^5)$\\
$\phantom{\dpar{z}{y}}=\dpar{}{y}({\color{blue}x^3})+\dpar{}{y}(y^2)+\dpar{}{y}({\color{blue}x^4}y^5)$\\
$\phantom{\dpar{z}{y}}=2y+5x^4y^4$
\end{itemize}

\newslide{課題(偏微分)}

\vspace*{18mm}

%%repeat=5,para
\slidepage
\seteda{60}
\begin{itemize}
\item
[課題]\monban 次の2変数関数の偏微分$z_x,z_y$を求めよ.\seteda{70}\\
\eda{$z=x^3+2y^3$のとき,$\dpar{z}{x}$}\\
\eda{$z=x^3+2y^3$のとき,$\dpar{z}{y}$}\\
\eda{$z=x^2+xy-y^2$のとき,$z_x$}\\
\eda{$z=x^2+xy-y^2$のとき,$z_y$}\\
\end{itemize}
%%%%%%%%%%%%

%%%%%%%%%%%%%%%%%%%%

\newslide{1変数関数の微分}

\vspace*{18mm}


\begin{layer}{120}{0}
\putnotew{96}{73}{\hyperlink{para4pg7}{\fbox{\Ctab{2.5mm}{\scalebox{1}{\scriptsize $\mathstrut||\!\lhd$}}}}}
\putnotew{101}{73}{\hyperlink{para5pg1}{\fbox{\Ctab{2.5mm}{\scalebox{1}{\scriptsize $\mathstrut|\!\lhd$}}}}}
\putnotew{108}{73}{\hyperlink{para5pg4}{\fbox{\Ctab{4.5mm}{\scalebox{1}{\scriptsize $\mathstrut\!\!\lhd\!\!$}}}}}
\putnotew{115}{73}{\hyperlink{para5pg5}{\fbox{\Ctab{4.5mm}{\scalebox{1}{\scriptsize $\mathstrut\!\rhd\!$}}}}}
\putnotew{120}{73}{\hyperlink{para5pg5}{\fbox{\Ctab{2.5mm}{\scalebox{1}{\scriptsize $\mathstrut \!\rhd\!\!|$}}}}}
\putnotew{125}{73}{\hyperlink{para6pg1}{\fbox{\Ctab{2.5mm}{\scalebox{1}{\scriptsize $\mathstrut \!\rhd\!\!||$}}}}}
\putnotee{126}{73}{\scriptsize\color{blue} 5/5}
\end{layer}

\slidepage

\begin{layer}{120}{0}
\putnotese{85}{0}{%%% /Users/takatoosetsuo/Dropbox/2018polytec/lecture/0910/presen/fig/bibun3.tex 
%%% Generator=presen0910.cdy 
{\unitlength=7.5mm%
\begin{picture}%
(5.5,4.5)(-0.5,-0.5)%
\special{pn 8}%
%
\Large\bf\boldmath%
\normalsize%
\special{pn 16}%
\special{pa  -148  -197}\special{pa   -87  -206}\special{pa   -21  -218}\special{pa    42  -230}%
\special{pa   101  -242}\special{pa   156  -255}\special{pa   206  -268}\special{pa   253  -281}%
\special{pa   295  -295}\special{pa   356  -317}\special{pa   416  -341}\special{pa   475  -367}%
\special{pa   535  -394}\special{pa   594  -423}\special{pa   653  -454}\special{pa   711  -486}%
\special{pa   770  -520}\special{pa   828  -555}\special{pa   886  -591}\special{pa   943  -627}%
\special{pa  1000  -665}\special{pa  1056  -704}\special{pa  1112  -745}\special{pa  1168  -787}%
\special{pa  1224  -831}\special{pa  1279  -876}\special{pa  1333  -922}\special{pa  1387  -970}%
\special{pa  1440 -1020}\special{pa  1476 -1057}\special{pa  1476 -1057}%
\special{fp}%
\special{pn 8}%
\special{pa 0 -295}\special{pa 35 -295}\special{fp}\special{pa 69 -295}\special{pa 104 -295}\special{fp}%
\special{pa 139 -295}\special{pa 174 -295}\special{fp}\special{pa 208 -295}\special{pa 243 -295}\special{fp}%
\special{pa 278 -295}\special{pa 295 -295}\special{pa 295 -278}\special{fp}\special{pa 295 -243}\special{pa 295 -208}\special{fp}%
\special{pa 295 -174}\special{pa 295 -139}\special{fp}\special{pa 295 -104}\special{pa 295 -69}\special{fp}%
\special{pa 295 -35}\special{pa 295 0}\special{fp}%
%
\special{pa 0 -591}\special{pa 38 -591}\special{fp}\special{pa 76 -591}\special{pa 114 -591}\special{fp}%
\special{pa 151 -591}\special{pa 189 -591}\special{fp}\special{pa 227 -591}\special{pa 265 -591}\special{fp}%
\special{pa 303 -591}\special{pa 341 -591}\special{fp}\special{pa 379 -591}\special{pa 416 -591}\special{fp}%
\special{pa 454 -591}\special{pa 492 -591}\special{fp}\special{pa 530 -591}\special{pa 568 -591}\special{fp}%
\special{pa 606 -591}\special{pa 644 -591}\special{fp}\special{pa 681 -591}\special{pa 719 -591}\special{fp}%
\special{pa 757 -591}\special{pa 795 -591}\special{fp}\special{pa 833 -591}\special{pa 871 -591}\special{fp}%
\special{pa 886 -568}\special{pa 886 -530}\special{fp}\special{pa 886 -492}\special{pa 886 -454}\special{fp}%
\special{pa 886 -416}\special{pa 886 -379}\special{fp}\special{pa 886 -341}\special{pa 886 -303}\special{fp}%
\special{pa 886 -265}\special{pa 886 -227}\special{fp}\special{pa 886 -189}\special{pa 886 -151}\special{fp}%
\special{pa 886 -114}\special{pa 886 -76}\special{fp}\special{pa 886 -38}\special{pa 886 0}\special{fp}%
%
%
\special{pa    20  -295}\special{pa   -20  -295}%
\special{fp}%
\settowidth{\Width}{$f(x)$}\setlength{\Width}{-1\Width}%
\settoheight{\Height}{$f(x)$}\settodepth{\Depth}{$f(x)$}\setlength{\Height}{-0.5\Height}\setlength{\Depth}{0.5\Depth}\addtolength{\Height}{\Depth}%
\put(-0.1333333,1.0000000){\hspace*{\Width}\raisebox{\Height}{$f(x)$}}%
%
\special{pa    20  -591}\special{pa   -20  -591}%
\special{fp}%
\settowidth{\Width}{$f(\xi)$}\setlength{\Width}{-1\Width}%
\settoheight{\Height}{$f(\xi)$}\settodepth{\Depth}{$f(\xi)$}\setlength{\Height}{-0.5\Height}\setlength{\Depth}{0.5\Depth}\addtolength{\Height}{\Depth}%
\put(-0.1333333,2.0000000){\hspace*{\Width}\raisebox{\Height}{$f(\xi)$}}%
%
\special{pa   295   -20}\special{pa   295    20}%
\special{fp}%
\settowidth{\Width}{$x$}\setlength{\Width}{-0.5\Width}%
\settoheight{\Height}{$x$}\settodepth{\Depth}{$x$}\setlength{\Height}{-\Height}%
\put(1.0000000,-0.1333333){\hspace*{\Width}\raisebox{\Height}{$x$}}%
%
\special{pa   886   -20}\special{pa   886    20}%
\special{fp}%
\settowidth{\Width}{$\xi$}\setlength{\Width}{-0.5\Width}%
\settoheight{\Height}{$\xi$}\settodepth{\Depth}{$\xi$}\setlength{\Height}{-\Height}%
\put(3.0000000,-0.1333333){\hspace*{\Width}\raisebox{\Height}{$\xi$}}%
%
\settowidth{\Width}{$\varDelta x$}\setlength{\Width}{-0.5\Width}%
\settoheight{\Height}{$\varDelta x$}\settodepth{\Depth}{$\varDelta x$}\setlength{\Height}{-0.5\Height}\setlength{\Depth}{0.5\Depth}\addtolength{\Height}{\Depth}%
\put(2.0000000,0.3300000){\hspace*{\Width}\raisebox{\Height}{$\varDelta x$}}%
%
\special{pa   886    -0}\special{pa   875    -5}\special{pa   863    -9}\special{pa   852   -13}%
\special{pa   841   -17}\special{pa   829   -21}\special{pa   817   -25}\special{pa   806   -28}%
\special{pa   794   -32}\special{pa   782   -35}\special{pa   771   -38}\special{pa   759   -40}%
\special{pa   747   -43}\special{pa   735   -45}\special{pa   723   -48}\special{pa   711   -50}%
\special{pa   699   -51}\special{pa   687   -53}\special{pa   675   -54}\special{pa   663   -56}%
\special{pa   651   -57}\special{pa   639   -58}\special{pa   627   -58}\special{pa   615   -59}%
\special{pa   603   -59}\special{pa   591   -59}\special{pa   578   -59}\special{pa   566   -59}%
\special{pa   554   -58}\special{pa   542   -58}\special{pa   530   -57}\special{pa   518   -56}%
\special{pa   506   -54}\special{pa   494   -53}\special{pa   482   -51}\special{pa   470   -50}%
\special{pa   458   -48}\special{pa   446   -45}\special{pa   434   -43}\special{pa   422   -40}%
\special{pa   410   -38}\special{pa   399   -35}\special{pa   387   -32}\special{pa   375   -28}%
\special{pa   364   -25}\special{pa   352   -21}\special{pa   341   -17}\special{pa   329   -13}%
\special{pa   318    -9}\special{pa   307    -5}\special{pa   295     0}%
\special{fp}%
\settowidth{\Width}{$\varDelta y$}\setlength{\Width}{-0.5\Width}%
\settoheight{\Height}{$\varDelta y$}\settodepth{\Depth}{$\varDelta y$}\setlength{\Height}{-0.5\Height}\setlength{\Depth}{0.5\Depth}\addtolength{\Height}{\Depth}%
\put(0.3700000,1.5000000){\hspace*{\Width}\raisebox{\Height}{$\varDelta y$}}%
%
\special{pa     0  -295}\special{pa     2  -301}\special{pa     4  -307}\special{pa     7  -312}%
\special{pa     9  -318}\special{pa    11  -324}\special{pa    12  -329}\special{pa    14  -335}%
\special{pa    16  -341}\special{pa    17  -347}\special{pa    19  -353}\special{pa    20  -359}%
\special{pa    21  -365}\special{pa    23  -371}\special{pa    24  -377}\special{pa    25  -383}%
\special{pa    26  -389}\special{pa    26  -395}\special{pa    27  -401}\special{pa    28  -407}%
\special{pa    28  -413}\special{pa    29  -419}\special{pa    29  -425}\special{pa    29  -431}%
\special{pa    29  -437}\special{pa    30  -443}\special{pa    29  -449}\special{pa    29  -455}%
\special{pa    29  -461}\special{pa    29  -467}\special{pa    28  -473}\special{pa    28  -479}%
\special{pa    27  -485}\special{pa    26  -491}\special{pa    26  -497}\special{pa    25  -503}%
\special{pa    24  -509}\special{pa    23  -515}\special{pa    21  -521}\special{pa    20  -527}%
\special{pa    19  -533}\special{pa    17  -539}\special{pa    16  -545}\special{pa    14  -551}%
\special{pa    12  -556}\special{pa    11  -562}\special{pa     9  -568}\special{pa     7  -574}%
\special{pa     4  -579}\special{pa     2  -585}\special{pa     0  -591}%
\special{fp}%
{%
\color[rgb]{0,0,1}%
\special{pa  -148  -141}\special{pa  1476  -708}%
\special{fp}%
}%
{%
\color[rgb]{1,0,0}%
\special{pa   295  -295}\special{pa   886  -295}\special{pa   886  -502}\special{pa   295  -295}%
\special{fp}%
}%
\settowidth{\Width}{${\color{red}dx}$}\setlength{\Width}{-0.5\Width}%
\settoheight{\Height}{${\color{red}dx}$}\settodepth{\Depth}{${\color{red}dx}$}\setlength{\Height}{-0.5\Height}\setlength{\Depth}{0.5\Depth}\addtolength{\Height}{\Depth}%
\put(2.0000000,0.6700000){\hspace*{\Width}\raisebox{\Height}{${\color{red}dx}$}}%
%
\special{pa   295  -295}\special{pa   307  -291}\special{pa   318  -286}\special{pa   329  -282}%
\special{pa   341  -278}\special{pa   352  -274}\special{pa   364  -271}\special{pa   375  -267}%
\special{pa   387  -264}\special{pa   399  -261}\special{pa   410  -258}\special{pa   422  -255}%
\special{pa   434  -252}\special{pa   446  -250}\special{pa   458  -248}\special{pa   470  -246}%
\special{pa   482  -244}\special{pa   494  -242}\special{pa   506  -241}\special{pa   518  -240}%
\special{pa   530  -239}\special{pa   542  -238}\special{pa   554  -237}\special{pa   566  -237}%
\special{pa   578  -236}\special{pa   591  -236}\special{pa   603  -236}\special{pa   615  -237}%
\special{pa   627  -237}\special{pa   639  -238}\special{pa   651  -239}\special{pa   663  -240}%
\special{pa   675  -241}\special{pa   687  -242}\special{pa   699  -244}\special{pa   711  -246}%
\special{pa   723  -248}\special{pa   735  -250}\special{pa   747  -252}\special{pa   759  -255}%
\special{pa   771  -258}\special{pa   782  -261}\special{pa   794  -264}\special{pa   806  -267}%
\special{pa   817  -271}\special{pa   829  -274}\special{pa   841  -278}\special{pa   852  -282}%
\special{pa   863  -286}\special{pa   875  -291}\special{pa   886  -295}%
\special{fp}%
\settowidth{\Width}{${\color{red}dy}$}\setlength{\Width}{-0.5\Width}%
\settoheight{\Height}{${\color{red}dy}$}\settodepth{\Depth}{${\color{red}dy}$}\setlength{\Height}{-0.5\Height}\setlength{\Depth}{0.5\Depth}\addtolength{\Height}{\Depth}%
\put(3.4100000,1.3500000){\hspace*{\Width}\raisebox{\Height}{${\color{red}dy}$}}%
%
\special{pa   886  -295}\special{pa   889  -299}\special{pa   892  -302}\special{pa   895  -306}%
\special{pa   898  -309}\special{pa   900  -313}\special{pa   903  -317}\special{pa   905  -321}%
\special{pa   907  -324}\special{pa   910  -328}\special{pa   912  -332}\special{pa   914  -337}%
\special{pa   916  -341}\special{pa   917  -345}\special{pa   919  -349}\special{pa   920  -354}%
\special{pa   921  -358}\special{pa   923  -362}\special{pa   924  -367}\special{pa   925  -371}%
\special{pa   925  -376}\special{pa   926  -380}\special{pa   926  -385}\special{pa   927  -389}%
\special{pa   927  -394}\special{pa   927  -398}\special{pa   927  -403}\special{pa   927  -407}%
\special{pa   926  -412}\special{pa   926  -417}\special{pa   925  -421}\special{pa   925  -426}%
\special{pa   924  -430}\special{pa   923  -434}\special{pa   921  -439}\special{pa   920  -443}%
\special{pa   919  -448}\special{pa   917  -452}\special{pa   916  -456}\special{pa   914  -460}%
\special{pa   912  -464}\special{pa   910  -468}\special{pa   907  -472}\special{pa   905  -476}%
\special{pa   903  -480}\special{pa   900  -484}\special{pa   898  -488}\special{pa   895  -491}%
\special{pa   892  -495}\special{pa   889  -498}\special{pa   886  -502}%
\special{fp}%
\special{pa  -148    -0}\special{pa  1476    -0}%
\special{fp}%
\special{pa     0   148}\special{pa     0 -1181}%
\special{fp}%
\settowidth{\Width}{$x$}\setlength{\Width}{0\Width}%
\settoheight{\Height}{$x$}\settodepth{\Depth}{$x$}\setlength{\Height}{-0.5\Height}\setlength{\Depth}{0.5\Depth}\addtolength{\Height}{\Depth}%
\put(5.0666667,0.0000000){\hspace*{\Width}\raisebox{\Height}{$x$}}%
%
\settowidth{\Width}{$y$}\setlength{\Width}{-0.5\Width}%
\settoheight{\Height}{$y$}\settodepth{\Depth}{$y$}\setlength{\Height}{\Depth}%
\put(0.0000000,4.0666667){\hspace*{\Width}\raisebox{\Height}{$y$}}%
%
\settowidth{\Width}{O}\setlength{\Width}{-1\Width}%
\settoheight{\Height}{O}\settodepth{\Depth}{O}\setlength{\Height}{-\Height}%
\put(-0.0666667,-0.0666667){\hspace*{\Width}\raisebox{\Height}{O}}%
%
\end{picture}}%}
\putnotee{80}{60}{\color{blue}$\varDelta x=dx$}
\end{layer}

\begin{itemize}
\item
$x$の変化量\ $\varDelta x=z-x$\vspace{-2mm}
\item
$y$の変化量\ $\varDelta y=f(\xi)-f(x)$\vspace{-2mm}
\item
$\bunsuu{dy}{dx}=\dlim_{\varDelta x\to 0}\bunsuu{\varDelta y}{\varDelta x}$\vspace{-1mm}
\item
これは図の接線の傾き
\item
赤の直角三角形の底辺と高さを$dx,\ dy$と書く
\item
$\tan\theta=\bunsuu{dy}{dx}$より\ {\color{red}$dy=\bunsuu{dy}{dx}dx$}($dx,dy$の意味付け)
\end{itemize}
\ifnum 1=0

\newslide{1変数関数の微分(まとめ)}

\vspace*{18mm}


\begin{layer}{120}{0}
\putnotew{96}{73}{\hyperlink{para5pg5}{\fbox{\Ctab{2.5mm}{\scalebox{1}{\scriptsize $\mathstrut||\!\lhd$}}}}}
\putnotew{101}{73}{\hyperlink{para6pg1}{\fbox{\Ctab{2.5mm}{\scalebox{1}{\scriptsize $\mathstrut|\!\lhd$}}}}}
\putnotew{108}{73}{\hyperlink{para6pg1}{\fbox{\Ctab{4.5mm}{\scalebox{1}{\scriptsize $\mathstrut\!\!\lhd\!\!$}}}}}
\putnotew{115}{73}{\hyperlink{para6pg2}{\fbox{\Ctab{4.5mm}{\scalebox{1}{\scriptsize $\mathstrut\!\rhd\!$}}}}}
\putnotew{120}{73}{\hyperlink{para6pg2}{\fbox{\Ctab{2.5mm}{\scalebox{1}{\scriptsize $\mathstrut \!\rhd\!\!|$}}}}}
\putnotew{125}{73}{\hyperlink{para7pg1}{\fbox{\Ctab{2.5mm}{\scalebox{1}{\scriptsize $\mathstrut \!\rhd\!\!||$}}}}}
\putnotee{126}{73}{\scriptsize\color{blue} 2/2}
\end{layer}

\slidepage

\begin{layer}{120}{0}
\putnotese{85}{9}{%%% /Users/takatoosetsuo/Dropbox/2018polytec/lecture/0910/presen/fig/bibun3.tex 
%%% Generator=presen0910.cdy 
{\unitlength=7.5mm%
\begin{picture}%
(5.5,4.5)(-0.5,-0.5)%
\special{pn 8}%
%
\Large\bf\boldmath%
\normalsize%
\special{pn 16}%
\special{pa  -148  -197}\special{pa   -87  -206}\special{pa   -21  -218}\special{pa    42  -230}%
\special{pa   101  -242}\special{pa   156  -255}\special{pa   206  -268}\special{pa   253  -281}%
\special{pa   295  -295}\special{pa   356  -317}\special{pa   416  -341}\special{pa   475  -367}%
\special{pa   535  -394}\special{pa   594  -423}\special{pa   653  -454}\special{pa   711  -486}%
\special{pa   770  -520}\special{pa   828  -555}\special{pa   886  -591}\special{pa   943  -627}%
\special{pa  1000  -665}\special{pa  1056  -704}\special{pa  1112  -745}\special{pa  1168  -787}%
\special{pa  1224  -831}\special{pa  1279  -876}\special{pa  1333  -922}\special{pa  1387  -970}%
\special{pa  1440 -1020}\special{pa  1476 -1057}\special{pa  1476 -1057}%
\special{fp}%
\special{pn 8}%
\special{pa 0 -295}\special{pa 35 -295}\special{fp}\special{pa 69 -295}\special{pa 104 -295}\special{fp}%
\special{pa 139 -295}\special{pa 174 -295}\special{fp}\special{pa 208 -295}\special{pa 243 -295}\special{fp}%
\special{pa 278 -295}\special{pa 295 -295}\special{pa 295 -278}\special{fp}\special{pa 295 -243}\special{pa 295 -208}\special{fp}%
\special{pa 295 -174}\special{pa 295 -139}\special{fp}\special{pa 295 -104}\special{pa 295 -69}\special{fp}%
\special{pa 295 -35}\special{pa 295 0}\special{fp}%
%
\special{pa 0 -591}\special{pa 38 -591}\special{fp}\special{pa 76 -591}\special{pa 114 -591}\special{fp}%
\special{pa 151 -591}\special{pa 189 -591}\special{fp}\special{pa 227 -591}\special{pa 265 -591}\special{fp}%
\special{pa 303 -591}\special{pa 341 -591}\special{fp}\special{pa 379 -591}\special{pa 416 -591}\special{fp}%
\special{pa 454 -591}\special{pa 492 -591}\special{fp}\special{pa 530 -591}\special{pa 568 -591}\special{fp}%
\special{pa 606 -591}\special{pa 644 -591}\special{fp}\special{pa 681 -591}\special{pa 719 -591}\special{fp}%
\special{pa 757 -591}\special{pa 795 -591}\special{fp}\special{pa 833 -591}\special{pa 871 -591}\special{fp}%
\special{pa 886 -568}\special{pa 886 -530}\special{fp}\special{pa 886 -492}\special{pa 886 -454}\special{fp}%
\special{pa 886 -416}\special{pa 886 -379}\special{fp}\special{pa 886 -341}\special{pa 886 -303}\special{fp}%
\special{pa 886 -265}\special{pa 886 -227}\special{fp}\special{pa 886 -189}\special{pa 886 -151}\special{fp}%
\special{pa 886 -114}\special{pa 886 -76}\special{fp}\special{pa 886 -38}\special{pa 886 0}\special{fp}%
%
%
\special{pa    20  -295}\special{pa   -20  -295}%
\special{fp}%
\settowidth{\Width}{$f(x)$}\setlength{\Width}{-1\Width}%
\settoheight{\Height}{$f(x)$}\settodepth{\Depth}{$f(x)$}\setlength{\Height}{-0.5\Height}\setlength{\Depth}{0.5\Depth}\addtolength{\Height}{\Depth}%
\put(-0.1333333,1.0000000){\hspace*{\Width}\raisebox{\Height}{$f(x)$}}%
%
\special{pa    20  -591}\special{pa   -20  -591}%
\special{fp}%
\settowidth{\Width}{$f(\xi)$}\setlength{\Width}{-1\Width}%
\settoheight{\Height}{$f(\xi)$}\settodepth{\Depth}{$f(\xi)$}\setlength{\Height}{-0.5\Height}\setlength{\Depth}{0.5\Depth}\addtolength{\Height}{\Depth}%
\put(-0.1333333,2.0000000){\hspace*{\Width}\raisebox{\Height}{$f(\xi)$}}%
%
\special{pa   295   -20}\special{pa   295    20}%
\special{fp}%
\settowidth{\Width}{$x$}\setlength{\Width}{-0.5\Width}%
\settoheight{\Height}{$x$}\settodepth{\Depth}{$x$}\setlength{\Height}{-\Height}%
\put(1.0000000,-0.1333333){\hspace*{\Width}\raisebox{\Height}{$x$}}%
%
\special{pa   886   -20}\special{pa   886    20}%
\special{fp}%
\settowidth{\Width}{$\xi$}\setlength{\Width}{-0.5\Width}%
\settoheight{\Height}{$\xi$}\settodepth{\Depth}{$\xi$}\setlength{\Height}{-\Height}%
\put(3.0000000,-0.1333333){\hspace*{\Width}\raisebox{\Height}{$\xi$}}%
%
\settowidth{\Width}{$\varDelta x$}\setlength{\Width}{-0.5\Width}%
\settoheight{\Height}{$\varDelta x$}\settodepth{\Depth}{$\varDelta x$}\setlength{\Height}{-0.5\Height}\setlength{\Depth}{0.5\Depth}\addtolength{\Height}{\Depth}%
\put(2.0000000,0.3300000){\hspace*{\Width}\raisebox{\Height}{$\varDelta x$}}%
%
\special{pa   886    -0}\special{pa   875    -5}\special{pa   863    -9}\special{pa   852   -13}%
\special{pa   841   -17}\special{pa   829   -21}\special{pa   817   -25}\special{pa   806   -28}%
\special{pa   794   -32}\special{pa   782   -35}\special{pa   771   -38}\special{pa   759   -40}%
\special{pa   747   -43}\special{pa   735   -45}\special{pa   723   -48}\special{pa   711   -50}%
\special{pa   699   -51}\special{pa   687   -53}\special{pa   675   -54}\special{pa   663   -56}%
\special{pa   651   -57}\special{pa   639   -58}\special{pa   627   -58}\special{pa   615   -59}%
\special{pa   603   -59}\special{pa   591   -59}\special{pa   578   -59}\special{pa   566   -59}%
\special{pa   554   -58}\special{pa   542   -58}\special{pa   530   -57}\special{pa   518   -56}%
\special{pa   506   -54}\special{pa   494   -53}\special{pa   482   -51}\special{pa   470   -50}%
\special{pa   458   -48}\special{pa   446   -45}\special{pa   434   -43}\special{pa   422   -40}%
\special{pa   410   -38}\special{pa   399   -35}\special{pa   387   -32}\special{pa   375   -28}%
\special{pa   364   -25}\special{pa   352   -21}\special{pa   341   -17}\special{pa   329   -13}%
\special{pa   318    -9}\special{pa   307    -5}\special{pa   295     0}%
\special{fp}%
\settowidth{\Width}{$\varDelta y$}\setlength{\Width}{-0.5\Width}%
\settoheight{\Height}{$\varDelta y$}\settodepth{\Depth}{$\varDelta y$}\setlength{\Height}{-0.5\Height}\setlength{\Depth}{0.5\Depth}\addtolength{\Height}{\Depth}%
\put(0.3700000,1.5000000){\hspace*{\Width}\raisebox{\Height}{$\varDelta y$}}%
%
\special{pa     0  -295}\special{pa     2  -301}\special{pa     4  -307}\special{pa     7  -312}%
\special{pa     9  -318}\special{pa    11  -324}\special{pa    12  -329}\special{pa    14  -335}%
\special{pa    16  -341}\special{pa    17  -347}\special{pa    19  -353}\special{pa    20  -359}%
\special{pa    21  -365}\special{pa    23  -371}\special{pa    24  -377}\special{pa    25  -383}%
\special{pa    26  -389}\special{pa    26  -395}\special{pa    27  -401}\special{pa    28  -407}%
\special{pa    28  -413}\special{pa    29  -419}\special{pa    29  -425}\special{pa    29  -431}%
\special{pa    29  -437}\special{pa    30  -443}\special{pa    29  -449}\special{pa    29  -455}%
\special{pa    29  -461}\special{pa    29  -467}\special{pa    28  -473}\special{pa    28  -479}%
\special{pa    27  -485}\special{pa    26  -491}\special{pa    26  -497}\special{pa    25  -503}%
\special{pa    24  -509}\special{pa    23  -515}\special{pa    21  -521}\special{pa    20  -527}%
\special{pa    19  -533}\special{pa    17  -539}\special{pa    16  -545}\special{pa    14  -551}%
\special{pa    12  -556}\special{pa    11  -562}\special{pa     9  -568}\special{pa     7  -574}%
\special{pa     4  -579}\special{pa     2  -585}\special{pa     0  -591}%
\special{fp}%
{%
\color[rgb]{0,0,1}%
\special{pa  -148  -141}\special{pa  1476  -708}%
\special{fp}%
}%
{%
\color[rgb]{1,0,0}%
\special{pa   295  -295}\special{pa   886  -295}\special{pa   886  -502}\special{pa   295  -295}%
\special{fp}%
}%
\settowidth{\Width}{${\color{red}dx}$}\setlength{\Width}{-0.5\Width}%
\settoheight{\Height}{${\color{red}dx}$}\settodepth{\Depth}{${\color{red}dx}$}\setlength{\Height}{-0.5\Height}\setlength{\Depth}{0.5\Depth}\addtolength{\Height}{\Depth}%
\put(2.0000000,0.6700000){\hspace*{\Width}\raisebox{\Height}{${\color{red}dx}$}}%
%
\special{pa   295  -295}\special{pa   307  -291}\special{pa   318  -286}\special{pa   329  -282}%
\special{pa   341  -278}\special{pa   352  -274}\special{pa   364  -271}\special{pa   375  -267}%
\special{pa   387  -264}\special{pa   399  -261}\special{pa   410  -258}\special{pa   422  -255}%
\special{pa   434  -252}\special{pa   446  -250}\special{pa   458  -248}\special{pa   470  -246}%
\special{pa   482  -244}\special{pa   494  -242}\special{pa   506  -241}\special{pa   518  -240}%
\special{pa   530  -239}\special{pa   542  -238}\special{pa   554  -237}\special{pa   566  -237}%
\special{pa   578  -236}\special{pa   591  -236}\special{pa   603  -236}\special{pa   615  -237}%
\special{pa   627  -237}\special{pa   639  -238}\special{pa   651  -239}\special{pa   663  -240}%
\special{pa   675  -241}\special{pa   687  -242}\special{pa   699  -244}\special{pa   711  -246}%
\special{pa   723  -248}\special{pa   735  -250}\special{pa   747  -252}\special{pa   759  -255}%
\special{pa   771  -258}\special{pa   782  -261}\special{pa   794  -264}\special{pa   806  -267}%
\special{pa   817  -271}\special{pa   829  -274}\special{pa   841  -278}\special{pa   852  -282}%
\special{pa   863  -286}\special{pa   875  -291}\special{pa   886  -295}%
\special{fp}%
\settowidth{\Width}{${\color{red}dy}$}\setlength{\Width}{-0.5\Width}%
\settoheight{\Height}{${\color{red}dy}$}\settodepth{\Depth}{${\color{red}dy}$}\setlength{\Height}{-0.5\Height}\setlength{\Depth}{0.5\Depth}\addtolength{\Height}{\Depth}%
\put(3.4100000,1.3500000){\hspace*{\Width}\raisebox{\Height}{${\color{red}dy}$}}%
%
\special{pa   886  -295}\special{pa   889  -299}\special{pa   892  -302}\special{pa   895  -306}%
\special{pa   898  -309}\special{pa   900  -313}\special{pa   903  -317}\special{pa   905  -321}%
\special{pa   907  -324}\special{pa   910  -328}\special{pa   912  -332}\special{pa   914  -337}%
\special{pa   916  -341}\special{pa   917  -345}\special{pa   919  -349}\special{pa   920  -354}%
\special{pa   921  -358}\special{pa   923  -362}\special{pa   924  -367}\special{pa   925  -371}%
\special{pa   925  -376}\special{pa   926  -380}\special{pa   926  -385}\special{pa   927  -389}%
\special{pa   927  -394}\special{pa   927  -398}\special{pa   927  -403}\special{pa   927  -407}%
\special{pa   926  -412}\special{pa   926  -417}\special{pa   925  -421}\special{pa   925  -426}%
\special{pa   924  -430}\special{pa   923  -434}\special{pa   921  -439}\special{pa   920  -443}%
\special{pa   919  -448}\special{pa   917  -452}\special{pa   916  -456}\special{pa   914  -460}%
\special{pa   912  -464}\special{pa   910  -468}\special{pa   907  -472}\special{pa   905  -476}%
\special{pa   903  -480}\special{pa   900  -484}\special{pa   898  -488}\special{pa   895  -491}%
\special{pa   892  -495}\special{pa   889  -498}\special{pa   886  -502}%
\special{fp}%
\special{pa  -148    -0}\special{pa  1476    -0}%
\special{fp}%
\special{pa     0   148}\special{pa     0 -1181}%
\special{fp}%
\settowidth{\Width}{$x$}\setlength{\Width}{0\Width}%
\settoheight{\Height}{$x$}\settodepth{\Depth}{$x$}\setlength{\Height}{-0.5\Height}\setlength{\Depth}{0.5\Depth}\addtolength{\Height}{\Depth}%
\put(5.0666667,0.0000000){\hspace*{\Width}\raisebox{\Height}{$x$}}%
%
\settowidth{\Width}{$y$}\setlength{\Width}{-0.5\Width}%
\settoheight{\Height}{$y$}\settodepth{\Depth}{$y$}\setlength{\Height}{\Depth}%
\put(0.0000000,4.0666667){\hspace*{\Width}\raisebox{\Height}{$y$}}%
%
\settowidth{\Width}{O}\setlength{\Width}{-1\Width}%
\settoheight{\Height}{O}\settodepth{\Depth}{O}\setlength{\Height}{-\Height}%
\put(-0.0666667,-0.0666667){\hspace*{\Width}\raisebox{\Height}{O}}%
%
\end{picture}}%}
\end{layer}

\begin{itemize}
\item
1変数関数$y=f(x)$
\item
$dy=\bunsuu{dy}{dx}dx$
\item
$\varDelta x$が小さいとき\\
\hspace*{2zw}$\varDelta y\fallingdotseq \bunsuu{dy}{dx}\varDelta x$
\end{itemize}

\newslide{2変数関数の場合}

\vspace*{18mm}


\begin{layer}{120}{0}
\putnotew{96}{73}{\hyperlink{para6pg2}{\fbox{\Ctab{2.5mm}{\scalebox{1}{\scriptsize $\mathstrut||\!\lhd$}}}}}
\putnotew{101}{73}{\hyperlink{para7pg1}{\fbox{\Ctab{2.5mm}{\scalebox{1}{\scriptsize $\mathstrut|\!\lhd$}}}}}
\putnotew{108}{73}{\hyperlink{para7pg2}{\fbox{\Ctab{4.5mm}{\scalebox{1}{\scriptsize $\mathstrut\!\!\lhd\!\!$}}}}}
\putnotew{115}{73}{\hyperlink{para7pg3}{\fbox{\Ctab{4.5mm}{\scalebox{1}{\scriptsize $\mathstrut\!\rhd\!$}}}}}
\putnotew{120}{73}{\hyperlink{para7pg3}{\fbox{\Ctab{2.5mm}{\scalebox{1}{\scriptsize $\mathstrut \!\rhd\!\!|$}}}}}
\putnotew{125}{73}{\hyperlink{para8pg1}{\fbox{\Ctab{2.5mm}{\scalebox{1}{\scriptsize $\mathstrut \!\rhd\!\!||$}}}}}
\putnotee{126}{73}{\scriptsize\color{blue} 3/3}
\end{layer}

\slidepage

\begin{layer}{120}{0}
\putnotes{65}{2}{%%% /Users/takatoosetsuo/Dropbox/2018polytec/lecture/0920/presen/fig/henbibun2.tex 
%%% Generator=zenbibun180920.cdy 
{\unitlength=1cm%
\begin{picture}%
(8.69,6.79)(-2.75,-1.17)%
\special{pn 8}%
%
\settowidth{\Width}{$x$}\setlength{\Width}{-0.5\Width}%
\settoheight{\Height}{$x$}\settodepth{\Depth}{$x$}\setlength{\Height}{-0.5\Height}\setlength{\Depth}{0.5\Depth}\addtolength{\Height}{\Depth}%
\put(4.2800000,-0.7800000){\hspace*{\Width}\raisebox{\Height}{$x$}}%
%
\settowidth{\Width}{$y$}\setlength{\Width}{-0.5\Width}%
\settoheight{\Height}{$y$}\settodepth{\Depth}{$y$}\setlength{\Height}{-0.5\Height}\setlength{\Depth}{0.5\Depth}\addtolength{\Height}{\Depth}%
\put(3.0500000,1.1300000){\hspace*{\Width}\raisebox{\Height}{$y$}}%
%
\settowidth{\Width}{$z$}\setlength{\Width}{-0.5\Width}%
\settoheight{\Height}{$z$}\settodepth{\Depth}{$z$}\setlength{\Height}{-0.5\Height}\setlength{\Depth}{0.5\Depth}\addtolength{\Height}{\Depth}%
\put(0.0000000,5.0200000){\hspace*{\Width}\raisebox{\Height}{$z$}}%
%
\special{pa     0    -0}\special{pa  1613   292}%
\special{fp}%
\special{pa     0    -0}\special{pa   382  -141}%
\special{fp}%
\special{pa   441  -163}\special{pa   776  -287}%
\special{fp}%
\special{pa   836  -309}\special{pa   946  -350}%
\special{fp}%
\special{pa  1005  -372}\special{pa  1129  -417}%
\special{fp}%
\special{pa     0    -0}\special{pa     0 -1901}%
\special{fp}%
\special{pa   976    84}\special{pa   976 -1274}%
\special{fp}%
\special{pa  1371   -63}\special{pa  1371 -1706}%
\special{fp}%
\special{pa   806  -165}\special{pa   806 -1114}%
\special{fp}%
\special{pa   806 -1175}\special{pa   806 -1218}%
\special{fp}%
\special{pa   806 -1279}\special{pa   806 -1573}%
\special{fp}%
\special{pa   411 -1216}\special{pa   524 -1221}\special{pa   637 -1230}\special{pa   750 -1242}%
\special{pa   863 -1256}\special{pa   976 -1274}%
\special{fp}%
\special{pa   411 -1216}\special{pa   490 -1273}\special{pa   569 -1338}\special{pa   648 -1409}%
\special{pa   727 -1488}\special{pa   806 -1573}%
\special{fp}%
\special{pa   806 -1573}\special{pa   919 -1597}\special{pa  1032 -1622}\special{pa  1145 -1648}%
\special{pa  1258 -1676}\special{pa  1371 -1706}%
\special{fp}%
\special{pa   976 -1274}\special{pa  1055 -1344}\special{pa  1134 -1422}\special{pa  1213 -1509}%
\special{pa  1292 -1604}\special{pa  1371 -1706}%
\special{fp}%
\special{pa   411   -19}\special{pa   976    84}\special{pa  1371   -63}\special{pa  1005  -129}%
\special{fp}%
\special{pa   946  -140}\special{pa   806  -165}\special{pa   411   -19}%
\special{fp}%
\special{pa   411   -19}\special{pa   411 -1216}%
\special{fp}%
\special{pa   411 -1216}\special{pa   976 -1114}\special{pa  1371 -1260}\special{pa  1132 -1303}%
\special{fp}%
\special{pa   994 -1328}\special{pa   806 -1362}\special{pa   411 -1216}%
\special{fp}%
\special{pa   976    84}\special{pa   976 -1114}%
\special{fp}%
\special{pa  1371   -63}\special{pa  1371 -1260}%
\special{fp}%
\special{pa   806  -165}\special{pa   806 -1114}%
\special{fp}%
\special{pa   806 -1175}\special{pa   806 -1218}%
\special{fp}%
\special{pa   806 -1279}\special{pa   806 -1362}%
\special{fp}%
\special{pa   976    84}\special{pa   976 -1232}%
\special{fp}%
\special{pa  1371   -63}\special{pa  1371 -1497}%
\special{fp}%
\special{pa   806  -165}\special{pa   806 -1114}%
\special{fp}%
\special{pa   806 -1175}\special{pa   806 -1218}%
\special{fp}%
\special{pa   806 -1279}\special{pa   806 -1482}%
\special{fp}%
\color[rgb]{0,0,1}%
\special{pa   411 -1216}\special{pa   976 -1232}\special{pa  1371 -1497}\special{pa  1237 -1493}%
\special{fp}%
\special{pa  1159 -1491}\special{pa   806 -1482}\special{pa   411 -1216}%
\special{fp}%
\color[rgb]{0,0,0}%
\end{picture}}%}
\end{layer}


\newslide{2変数関数の場合}

\vspace*{18mm}


\begin{layer}{120}{0}
\putnotes{65}{2}{%%% /Users/takatoosetsuo/Dropbox/2018polytec/lecture/0521/presen/fig/sinecurve/p008.tex 
%%% Generator=presen0521.cdy 
{\unitlength=12.5mm%
\begin{picture}%
(9.2,2.4)(-2.2,-1.2)%
\special{pn 8}%
%
\Large\bf\boldmath%
\small%
\special{pa     0    -0}\special{pa    -4   -62}\special{pa   -15  -122}\special{pa   -35  -181}%
\special{pa   -61  -237}\special{pa   -94  -289}\special{pa  -133  -337}\special{pa  -178  -379}%
\special{pa  -228  -416}\special{pa  -283  -445}\special{pa  -340  -468}\special{pa  -400  -483}%
\special{pa  -461  -491}\special{pa  -523  -491}\special{pa  -584  -483}\special{pa  -644  -468}%
\special{pa  -702  -445}\special{pa  -756  -416}\special{pa  -806  -379}\special{pa  -851  -337}%
\special{pa  -890  -289}\special{pa  -923  -237}\special{pa  -950  -181}\special{pa  -969  -122}%
\special{pa  -980   -62}\special{pa  -984     0}\special{pa  -980    62}\special{pa  -969   122}%
\special{pa  -950   181}\special{pa  -923   237}\special{pa  -890   289}\special{pa  -851   337}%
\special{pa  -806   379}\special{pa  -756   416}\special{pa  -702   445}\special{pa  -644   468}%
\special{pa  -584   483}\special{pa  -523   491}\special{pa  -461   491}\special{pa  -400   483}%
\special{pa  -340   468}\special{pa  -283   445}\special{pa  -228   416}\special{pa  -178   379}%
\special{pa  -133   337}\special{pa   -94   289}\special{pa   -61   237}\special{pa   -35   181}%
\special{pa   -15   122}\special{pa    -4    62}\special{pa     0     0}%
\special{fp}%
\special{pa  -492    -0}\special{pa     0    -0}%
\special{fp}%
\special{pa  -492    -0}\special{pa  -584  -483}%
\special{fp}%
\special{pa 866 -483}\special{pa 866 -446}\special{fp}\special{pa 866 -409}\special{pa 866 -372}\special{fp}%
\special{pa 866 -335}\special{pa 866 -297}\special{fp}\special{pa 866 -260}\special{pa 866 -223}\special{fp}%
\special{pa 866 -186}\special{pa 866 -149}\special{fp}\special{pa 866 -112}\special{pa 866 -74}\special{fp}%
\special{pa 866 -37}\special{pa 866 0}\special{fp}%
%
\special{pa -584 -483}\special{pa -545 -483}\special{fp}\special{pa -506 -483}\special{pa -467 -483}\special{fp}%
\special{pa -428 -483}\special{pa -388 -483}\special{fp}\special{pa -349 -483}\special{pa -310 -483}\special{fp}%
\special{pa -271 -483}\special{pa -232 -483}\special{fp}\special{pa -192 -483}\special{pa -153 -483}\special{fp}%
\special{pa -114 -483}\special{pa -75 -483}\special{fp}\special{pa -36 -483}\special{pa 4 -483}\special{fp}%
\special{pa 43 -483}\special{pa 82 -483}\special{fp}\special{pa 121 -483}\special{pa 160 -483}\special{fp}%
\special{pa 200 -483}\special{pa 239 -483}\special{fp}\special{pa 278 -483}\special{pa 317 -483}\special{fp}%
\special{pa 356 -483}\special{pa 395 -483}\special{fp}\special{pa 435 -483}\special{pa 474 -483}\special{fp}%
\special{pa 513 -483}\special{pa 552 -483}\special{fp}\special{pa 591 -483}\special{pa 631 -483}\special{fp}%
\special{pa 670 -483}\special{pa 709 -483}\special{fp}\special{pa 748 -483}\special{pa 787 -483}\special{fp}%
\special{pa 827 -483}\special{pa 866 -483}\special{fp}%
%
\settowidth{\Width}{$x$}\setlength{\Width}{-0.5\Width}%
\settoheight{\Height}{$x$}\settodepth{\Depth}{$x$}\setlength{\Height}{-0.5\Height}\setlength{\Depth}{0.5\Depth}\addtolength{\Height}{\Depth}%
\put(-0.6700000,0.4000000){\hspace*{\Width}\raisebox{\Height}{$x$}}%
%
\special{pa  -320    -0}\special{pa  -321   -22}\special{pa  -325   -43}\special{pa  -332   -63}%
\special{pa  -341   -83}\special{pa  -353  -101}\special{pa  -367  -118}\special{pa  -382  -133}%
\special{pa  -400  -145}\special{pa  -419  -156}\special{pa  -439  -164}\special{pa  -460  -169}%
\special{pa  -481  -172}\special{pa  -503  -172}\special{pa  -524  -169}%
\special{fp}%
\color[cmyk]{0,1,1,0}%
\special{pa     0    -0}\special{pa    17   -17}\special{pa    35   -35}\special{pa    52   -52}%
\special{pa    69   -69}\special{pa    87   -86}\special{pa   104  -103}\special{pa   121  -120}%
\special{pa   139  -137}\special{pa   156  -153}\special{pa   173  -170}\special{pa   190  -186}%
\special{pa   208  -202}\special{pa   225  -217}\special{pa   242  -233}\special{pa   260  -248}%
\special{pa   277  -263}\special{pa   294  -277}\special{pa   312  -291}\special{pa   329  -305}%
\special{pa   346  -318}\special{pa   364  -331}\special{pa   381  -344}\special{pa   398  -356}%
\special{pa   416  -368}\special{pa   433  -379}\special{pa   450  -390}\special{pa   468  -400}%
\special{pa   485  -410}\special{pa   502  -419}\special{pa   519  -428}\special{pa   537  -437}%
\special{pa   554  -444}\special{pa   571  -451}\special{pa   589  -458}\special{pa   606  -464}%
\special{pa   623  -470}\special{pa   641  -474}\special{pa   658  -479}\special{pa   675  -482}%
\special{pa   693  -486}\special{pa   710  -488}\special{pa   727  -490}\special{pa   745  -491}%
\special{pa   762  -492}\special{pa   779  -492}\special{pa   797  -492}\special{pa   814  -490}%
\special{pa   831  -489}\special{pa   848  -486}\special{pa   866  -483}%
\special{fp}%
\color[cmyk]{0,0,0,1}%
\special{pa   773   -20}\special{pa   773    20}%
\special{fp}%
\settowidth{\Width}{$\frac{\pi}{2}$}\setlength{\Width}{-0.5\Width}%
\settoheight{\Height}{$\frac{\pi}{2}$}\settodepth{\Depth}{$\frac{\pi}{2}$}\setlength{\Height}{\Depth}%
\put(1.5707960,0.0800000){\hspace*{\Width}\raisebox{\Height}{$\frac{\pi}{2}$}}%
%
%
\special{pa  1546   -20}\special{pa  1546    20}%
\special{fp}%
\settowidth{\Width}{$\pi$}\setlength{\Width}{-0.5\Width}%
\settoheight{\Height}{$\pi$}\settodepth{\Depth}{$\pi$}\setlength{\Height}{\Depth}%
\put(3.1415930,0.0800000){\hspace*{\Width}\raisebox{\Height}{$\pi$}}%
%
%
\special{pa  3092   -20}\special{pa  3092    20}%
\special{fp}%
\settowidth{\Width}{$2\pi$}\setlength{\Width}{-0.5\Width}%
\settoheight{\Height}{$2\pi$}\settodepth{\Depth}{$2\pi$}\setlength{\Height}{\Depth}%
\put(6.2831850,0.0800000){\hspace*{\Width}\raisebox{\Height}{$2\pi$}}%
%
%
\special{pa    20   492}\special{pa   -20   492}%
\special{fp}%
\settowidth{\Width}{$-1$}\setlength{\Width}{-1\Width}%
\settoheight{\Height}{$-1$}\settodepth{\Depth}{$-1$}\setlength{\Height}{-0.5\Height}\setlength{\Depth}{0.5\Depth}\addtolength{\Height}{\Depth}%
\put(-0.0800000,-1.0000000){\hspace*{\Width}\raisebox{\Height}{$-1$}}%
%
%
\special{pa    20  -492}\special{pa   -20  -492}%
\special{fp}%
\settowidth{\Width}{$1$}\setlength{\Width}{-1\Width}%
\settoheight{\Height}{$1$}\settodepth{\Depth}{$1$}\setlength{\Height}{-0.5\Height}\setlength{\Depth}{0.5\Depth}\addtolength{\Height}{\Depth}%
\put(-0.0800000,1.0000000){\hspace*{\Width}\raisebox{\Height}{$1$}}%
%
%
\special{pa -1083    -0}\special{pa  3445    -0}%
\special{fp}%
\special{pa     0   591}\special{pa     0  -591}%
\special{fp}%
\settowidth{\Width}{$x$}\setlength{\Width}{0\Width}%
\settoheight{\Height}{$x$}\settodepth{\Depth}{$x$}\setlength{\Height}{-0.5\Height}\setlength{\Depth}{0.5\Depth}\addtolength{\Height}{\Depth}%
\put(7.0400000,0.0000000){\hspace*{\Width}\raisebox{\Height}{$x$}}%
%
\settowidth{\Width}{$y$}\setlength{\Width}{-0.5\Width}%
\settoheight{\Height}{$y$}\settodepth{\Depth}{$y$}\setlength{\Height}{\Depth}%
\put(0.0000000,1.2400000){\hspace*{\Width}\raisebox{\Height}{$y$}}%
%
\settowidth{\Width}{O}\setlength{\Width}{0\Width}%
\settoheight{\Height}{O}\settodepth{\Depth}{O}\setlength{\Height}{-\Height}%
\put(0.0400000,-0.0400000){\hspace*{\Width}\raisebox{\Height}{O}}%
%
\end{picture}}%}
\putnotew{96}{73}{\hyperlink{para7pg3}{\fbox{\Ctab{2.5mm}{\scalebox{1}{\scriptsize $\mathstrut||\!\lhd$}}}}}
\putnotew{101}{73}{\hyperlink{para8pg1}{\fbox{\Ctab{2.5mm}{\scalebox{1}{\scriptsize $\mathstrut|\!\lhd$}}}}}
\putnotew{108}{73}{\hyperlink{para8pg7}{\fbox{\Ctab{4.5mm}{\scalebox{1}{\scriptsize $\mathstrut\!\!\lhd\!\!$}}}}}
\putnotew{115}{73}{\hyperlink{para8pg8}{\fbox{\Ctab{4.5mm}{\scalebox{1}{\scriptsize $\mathstrut\!\rhd\!$}}}}}
\putnotew{120}{73}{\hyperlink{para8pg8}{\fbox{\Ctab{2.5mm}{\scalebox{1}{\scriptsize $\mathstrut \!\rhd\!\!|$}}}}}
\putnotew{125}{73}{\hyperlink{para9pg1}{\fbox{\Ctab{2.5mm}{\scalebox{1}{\scriptsize $\mathstrut \!\rhd\!\!||$}}}}}
\putnotee{126}{73}{\scriptsize\color{blue} 8/8}
\end{layer}

\slidepage

\newslide{1変数関数の微分}

\vspace*{18mm}


\begin{layer}{120}{0}
\putnotew{96}{73}{\hyperlink{para8pg8}{\fbox{\Ctab{2.5mm}{\scalebox{1}{\scriptsize $\mathstrut||\!\lhd$}}}}}
\putnotew{101}{73}{\hyperlink{para9pg1}{\fbox{\Ctab{2.5mm}{\scalebox{1}{\scriptsize $\mathstrut|\!\lhd$}}}}}
\putnotew{108}{73}{\hyperlink{para9pg4}{\fbox{\Ctab{4.5mm}{\scalebox{1}{\scriptsize $\mathstrut\!\!\lhd\!\!$}}}}}
\putnotew{115}{73}{\hyperlink{para9pg5}{\fbox{\Ctab{4.5mm}{\scalebox{1}{\scriptsize $\mathstrut\!\rhd\!$}}}}}
\putnotew{120}{73}{\hyperlink{para9pg5}{\fbox{\Ctab{2.5mm}{\scalebox{1}{\scriptsize $\mathstrut \!\rhd\!\!|$}}}}}
\putnotew{125}{73}{\hyperlink{para10pg1}{\fbox{\Ctab{2.5mm}{\scalebox{1}{\scriptsize $\mathstrut \!\rhd\!\!||$}}}}}
\putnotee{126}{73}{\scriptsize\color{blue} 5/5}
\end{layer}

\slidepage

\begin{layer}{120}{0}
\putnotese{85}{9}{%%% /Users/takatoosetsuo/Dropbox/2018polytec/lecture/0910/presen/fig/bibun3.tex 
%%% Generator=presen0910.cdy 
{\unitlength=7.5mm%
\begin{picture}%
(5.5,4.5)(-0.5,-0.5)%
\special{pn 8}%
%
\Large\bf\boldmath%
\normalsize%
\special{pn 16}%
\special{pa  -148  -197}\special{pa   -87  -206}\special{pa   -21  -218}\special{pa    42  -230}%
\special{pa   101  -242}\special{pa   156  -255}\special{pa   206  -268}\special{pa   253  -281}%
\special{pa   295  -295}\special{pa   356  -317}\special{pa   416  -341}\special{pa   475  -367}%
\special{pa   535  -394}\special{pa   594  -423}\special{pa   653  -454}\special{pa   711  -486}%
\special{pa   770  -520}\special{pa   828  -555}\special{pa   886  -591}\special{pa   943  -627}%
\special{pa  1000  -665}\special{pa  1056  -704}\special{pa  1112  -745}\special{pa  1168  -787}%
\special{pa  1224  -831}\special{pa  1279  -876}\special{pa  1333  -922}\special{pa  1387  -970}%
\special{pa  1440 -1020}\special{pa  1476 -1057}\special{pa  1476 -1057}%
\special{fp}%
\special{pn 8}%
\special{pa 0 -295}\special{pa 35 -295}\special{fp}\special{pa 69 -295}\special{pa 104 -295}\special{fp}%
\special{pa 139 -295}\special{pa 174 -295}\special{fp}\special{pa 208 -295}\special{pa 243 -295}\special{fp}%
\special{pa 278 -295}\special{pa 295 -295}\special{pa 295 -278}\special{fp}\special{pa 295 -243}\special{pa 295 -208}\special{fp}%
\special{pa 295 -174}\special{pa 295 -139}\special{fp}\special{pa 295 -104}\special{pa 295 -69}\special{fp}%
\special{pa 295 -35}\special{pa 295 0}\special{fp}%
%
\special{pa 0 -591}\special{pa 38 -591}\special{fp}\special{pa 76 -591}\special{pa 114 -591}\special{fp}%
\special{pa 151 -591}\special{pa 189 -591}\special{fp}\special{pa 227 -591}\special{pa 265 -591}\special{fp}%
\special{pa 303 -591}\special{pa 341 -591}\special{fp}\special{pa 379 -591}\special{pa 416 -591}\special{fp}%
\special{pa 454 -591}\special{pa 492 -591}\special{fp}\special{pa 530 -591}\special{pa 568 -591}\special{fp}%
\special{pa 606 -591}\special{pa 644 -591}\special{fp}\special{pa 681 -591}\special{pa 719 -591}\special{fp}%
\special{pa 757 -591}\special{pa 795 -591}\special{fp}\special{pa 833 -591}\special{pa 871 -591}\special{fp}%
\special{pa 886 -568}\special{pa 886 -530}\special{fp}\special{pa 886 -492}\special{pa 886 -454}\special{fp}%
\special{pa 886 -416}\special{pa 886 -379}\special{fp}\special{pa 886 -341}\special{pa 886 -303}\special{fp}%
\special{pa 886 -265}\special{pa 886 -227}\special{fp}\special{pa 886 -189}\special{pa 886 -151}\special{fp}%
\special{pa 886 -114}\special{pa 886 -76}\special{fp}\special{pa 886 -38}\special{pa 886 0}\special{fp}%
%
%
\special{pa    20  -295}\special{pa   -20  -295}%
\special{fp}%
\settowidth{\Width}{$f(x)$}\setlength{\Width}{-1\Width}%
\settoheight{\Height}{$f(x)$}\settodepth{\Depth}{$f(x)$}\setlength{\Height}{-0.5\Height}\setlength{\Depth}{0.5\Depth}\addtolength{\Height}{\Depth}%
\put(-0.1333333,1.0000000){\hspace*{\Width}\raisebox{\Height}{$f(x)$}}%
%
\special{pa    20  -591}\special{pa   -20  -591}%
\special{fp}%
\settowidth{\Width}{$f(\xi)$}\setlength{\Width}{-1\Width}%
\settoheight{\Height}{$f(\xi)$}\settodepth{\Depth}{$f(\xi)$}\setlength{\Height}{-0.5\Height}\setlength{\Depth}{0.5\Depth}\addtolength{\Height}{\Depth}%
\put(-0.1333333,2.0000000){\hspace*{\Width}\raisebox{\Height}{$f(\xi)$}}%
%
\special{pa   295   -20}\special{pa   295    20}%
\special{fp}%
\settowidth{\Width}{$x$}\setlength{\Width}{-0.5\Width}%
\settoheight{\Height}{$x$}\settodepth{\Depth}{$x$}\setlength{\Height}{-\Height}%
\put(1.0000000,-0.1333333){\hspace*{\Width}\raisebox{\Height}{$x$}}%
%
\special{pa   886   -20}\special{pa   886    20}%
\special{fp}%
\settowidth{\Width}{$\xi$}\setlength{\Width}{-0.5\Width}%
\settoheight{\Height}{$\xi$}\settodepth{\Depth}{$\xi$}\setlength{\Height}{-\Height}%
\put(3.0000000,-0.1333333){\hspace*{\Width}\raisebox{\Height}{$\xi$}}%
%
\settowidth{\Width}{$\varDelta x$}\setlength{\Width}{-0.5\Width}%
\settoheight{\Height}{$\varDelta x$}\settodepth{\Depth}{$\varDelta x$}\setlength{\Height}{-0.5\Height}\setlength{\Depth}{0.5\Depth}\addtolength{\Height}{\Depth}%
\put(2.0000000,0.3300000){\hspace*{\Width}\raisebox{\Height}{$\varDelta x$}}%
%
\special{pa   886    -0}\special{pa   875    -5}\special{pa   863    -9}\special{pa   852   -13}%
\special{pa   841   -17}\special{pa   829   -21}\special{pa   817   -25}\special{pa   806   -28}%
\special{pa   794   -32}\special{pa   782   -35}\special{pa   771   -38}\special{pa   759   -40}%
\special{pa   747   -43}\special{pa   735   -45}\special{pa   723   -48}\special{pa   711   -50}%
\special{pa   699   -51}\special{pa   687   -53}\special{pa   675   -54}\special{pa   663   -56}%
\special{pa   651   -57}\special{pa   639   -58}\special{pa   627   -58}\special{pa   615   -59}%
\special{pa   603   -59}\special{pa   591   -59}\special{pa   578   -59}\special{pa   566   -59}%
\special{pa   554   -58}\special{pa   542   -58}\special{pa   530   -57}\special{pa   518   -56}%
\special{pa   506   -54}\special{pa   494   -53}\special{pa   482   -51}\special{pa   470   -50}%
\special{pa   458   -48}\special{pa   446   -45}\special{pa   434   -43}\special{pa   422   -40}%
\special{pa   410   -38}\special{pa   399   -35}\special{pa   387   -32}\special{pa   375   -28}%
\special{pa   364   -25}\special{pa   352   -21}\special{pa   341   -17}\special{pa   329   -13}%
\special{pa   318    -9}\special{pa   307    -5}\special{pa   295     0}%
\special{fp}%
\settowidth{\Width}{$\varDelta y$}\setlength{\Width}{-0.5\Width}%
\settoheight{\Height}{$\varDelta y$}\settodepth{\Depth}{$\varDelta y$}\setlength{\Height}{-0.5\Height}\setlength{\Depth}{0.5\Depth}\addtolength{\Height}{\Depth}%
\put(0.3700000,1.5000000){\hspace*{\Width}\raisebox{\Height}{$\varDelta y$}}%
%
\special{pa     0  -295}\special{pa     2  -301}\special{pa     4  -307}\special{pa     7  -312}%
\special{pa     9  -318}\special{pa    11  -324}\special{pa    12  -329}\special{pa    14  -335}%
\special{pa    16  -341}\special{pa    17  -347}\special{pa    19  -353}\special{pa    20  -359}%
\special{pa    21  -365}\special{pa    23  -371}\special{pa    24  -377}\special{pa    25  -383}%
\special{pa    26  -389}\special{pa    26  -395}\special{pa    27  -401}\special{pa    28  -407}%
\special{pa    28  -413}\special{pa    29  -419}\special{pa    29  -425}\special{pa    29  -431}%
\special{pa    29  -437}\special{pa    30  -443}\special{pa    29  -449}\special{pa    29  -455}%
\special{pa    29  -461}\special{pa    29  -467}\special{pa    28  -473}\special{pa    28  -479}%
\special{pa    27  -485}\special{pa    26  -491}\special{pa    26  -497}\special{pa    25  -503}%
\special{pa    24  -509}\special{pa    23  -515}\special{pa    21  -521}\special{pa    20  -527}%
\special{pa    19  -533}\special{pa    17  -539}\special{pa    16  -545}\special{pa    14  -551}%
\special{pa    12  -556}\special{pa    11  -562}\special{pa     9  -568}\special{pa     7  -574}%
\special{pa     4  -579}\special{pa     2  -585}\special{pa     0  -591}%
\special{fp}%
{%
\color[rgb]{0,0,1}%
\special{pa  -148  -141}\special{pa  1476  -708}%
\special{fp}%
}%
{%
\color[rgb]{1,0,0}%
\special{pa   295  -295}\special{pa   886  -295}\special{pa   886  -502}\special{pa   295  -295}%
\special{fp}%
}%
\settowidth{\Width}{${\color{red}dx}$}\setlength{\Width}{-0.5\Width}%
\settoheight{\Height}{${\color{red}dx}$}\settodepth{\Depth}{${\color{red}dx}$}\setlength{\Height}{-0.5\Height}\setlength{\Depth}{0.5\Depth}\addtolength{\Height}{\Depth}%
\put(2.0000000,0.6700000){\hspace*{\Width}\raisebox{\Height}{${\color{red}dx}$}}%
%
\special{pa   295  -295}\special{pa   307  -291}\special{pa   318  -286}\special{pa   329  -282}%
\special{pa   341  -278}\special{pa   352  -274}\special{pa   364  -271}\special{pa   375  -267}%
\special{pa   387  -264}\special{pa   399  -261}\special{pa   410  -258}\special{pa   422  -255}%
\special{pa   434  -252}\special{pa   446  -250}\special{pa   458  -248}\special{pa   470  -246}%
\special{pa   482  -244}\special{pa   494  -242}\special{pa   506  -241}\special{pa   518  -240}%
\special{pa   530  -239}\special{pa   542  -238}\special{pa   554  -237}\special{pa   566  -237}%
\special{pa   578  -236}\special{pa   591  -236}\special{pa   603  -236}\special{pa   615  -237}%
\special{pa   627  -237}\special{pa   639  -238}\special{pa   651  -239}\special{pa   663  -240}%
\special{pa   675  -241}\special{pa   687  -242}\special{pa   699  -244}\special{pa   711  -246}%
\special{pa   723  -248}\special{pa   735  -250}\special{pa   747  -252}\special{pa   759  -255}%
\special{pa   771  -258}\special{pa   782  -261}\special{pa   794  -264}\special{pa   806  -267}%
\special{pa   817  -271}\special{pa   829  -274}\special{pa   841  -278}\special{pa   852  -282}%
\special{pa   863  -286}\special{pa   875  -291}\special{pa   886  -295}%
\special{fp}%
\settowidth{\Width}{${\color{red}dy}$}\setlength{\Width}{-0.5\Width}%
\settoheight{\Height}{${\color{red}dy}$}\settodepth{\Depth}{${\color{red}dy}$}\setlength{\Height}{-0.5\Height}\setlength{\Depth}{0.5\Depth}\addtolength{\Height}{\Depth}%
\put(3.4100000,1.3500000){\hspace*{\Width}\raisebox{\Height}{${\color{red}dy}$}}%
%
\special{pa   886  -295}\special{pa   889  -299}\special{pa   892  -302}\special{pa   895  -306}%
\special{pa   898  -309}\special{pa   900  -313}\special{pa   903  -317}\special{pa   905  -321}%
\special{pa   907  -324}\special{pa   910  -328}\special{pa   912  -332}\special{pa   914  -337}%
\special{pa   916  -341}\special{pa   917  -345}\special{pa   919  -349}\special{pa   920  -354}%
\special{pa   921  -358}\special{pa   923  -362}\special{pa   924  -367}\special{pa   925  -371}%
\special{pa   925  -376}\special{pa   926  -380}\special{pa   926  -385}\special{pa   927  -389}%
\special{pa   927  -394}\special{pa   927  -398}\special{pa   927  -403}\special{pa   927  -407}%
\special{pa   926  -412}\special{pa   926  -417}\special{pa   925  -421}\special{pa   925  -426}%
\special{pa   924  -430}\special{pa   923  -434}\special{pa   921  -439}\special{pa   920  -443}%
\special{pa   919  -448}\special{pa   917  -452}\special{pa   916  -456}\special{pa   914  -460}%
\special{pa   912  -464}\special{pa   910  -468}\special{pa   907  -472}\special{pa   905  -476}%
\special{pa   903  -480}\special{pa   900  -484}\special{pa   898  -488}\special{pa   895  -491}%
\special{pa   892  -495}\special{pa   889  -498}\special{pa   886  -502}%
\special{fp}%
\special{pa  -148    -0}\special{pa  1476    -0}%
\special{fp}%
\special{pa     0   148}\special{pa     0 -1181}%
\special{fp}%
\settowidth{\Width}{$x$}\setlength{\Width}{0\Width}%
\settoheight{\Height}{$x$}\settodepth{\Depth}{$x$}\setlength{\Height}{-0.5\Height}\setlength{\Depth}{0.5\Depth}\addtolength{\Height}{\Depth}%
\put(5.0666667,0.0000000){\hspace*{\Width}\raisebox{\Height}{$x$}}%
%
\settowidth{\Width}{$y$}\setlength{\Width}{-0.5\Width}%
\settoheight{\Height}{$y$}\settodepth{\Depth}{$y$}\setlength{\Height}{\Depth}%
\put(0.0000000,4.0666667){\hspace*{\Width}\raisebox{\Height}{$y$}}%
%
\settowidth{\Width}{O}\setlength{\Width}{-1\Width}%
\settoheight{\Height}{O}\settodepth{\Depth}{O}\setlength{\Height}{-\Height}%
\put(-0.0666667,-0.0666667){\hspace*{\Width}\raisebox{\Height}{O}}%
%
\end{picture}}%}
\putnotee{80}{55}{\color{blue}$\varDelta x=dx$}
\end{layer}

\begin{itemize}
\item
$x$の変化量\ $\varDelta x=\xi-x$\vspace{-2mm}
\item
$y$の変化量\ $\varDelta y=f(\xi)-f(x)$\vspace{-2mm}
\item
$\bunsuu{dy}{dx}=\dlim_{\varDelta x\to 0}\bunsuu{\varDelta y}{\varDelta x}$\vspace{-1mm}
\item
これは図の接線の傾き
\item
赤の直角三角形の底辺と高さを$dx,\ dy$と書く
\item
$\tan\theta=\bunsuu{dy}{dx}$より\ {\color{red}$dy=\bunsuu{dy}{dx}dx$}($dx,dy$の意味付け)
\end{itemize}

\newslide{1変数関数の微分(まとめ)}

\vspace*{18mm}


\begin{layer}{120}{0}
\putnotew{96}{73}{\hyperlink{para9pg5}{\fbox{\Ctab{2.5mm}{\scalebox{1}{\scriptsize $\mathstrut||\!\lhd$}}}}}
\putnotew{101}{73}{\hyperlink{para10pg1}{\fbox{\Ctab{2.5mm}{\scalebox{1}{\scriptsize $\mathstrut|\!\lhd$}}}}}
\putnotew{108}{73}{\hyperlink{para10pg1}{\fbox{\Ctab{4.5mm}{\scalebox{1}{\scriptsize $\mathstrut\!\!\lhd\!\!$}}}}}
\putnotew{115}{73}{\hyperlink{para10pg2}{\fbox{\Ctab{4.5mm}{\scalebox{1}{\scriptsize $\mathstrut\!\rhd\!$}}}}}
\putnotew{120}{73}{\hyperlink{para10pg2}{\fbox{\Ctab{2.5mm}{\scalebox{1}{\scriptsize $\mathstrut \!\rhd\!\!|$}}}}}
\putnotew{125}{73}{\hyperlink{para11pg1}{\fbox{\Ctab{2.5mm}{\scalebox{1}{\scriptsize $\mathstrut \!\rhd\!\!||$}}}}}
\putnotee{126}{73}{\scriptsize\color{blue} 2/2}
\end{layer}

\slidepage

\begin{layer}{120}{0}
\putnotese{85}{9}{%%% /Users/takatoosetsuo/Dropbox/2018polytec/lecture/0910/presen/fig/bibun3.tex 
%%% Generator=presen0910.cdy 
{\unitlength=7.5mm%
\begin{picture}%
(5.5,4.5)(-0.5,-0.5)%
\special{pn 8}%
%
\Large\bf\boldmath%
\normalsize%
\special{pn 16}%
\special{pa  -148  -197}\special{pa   -87  -206}\special{pa   -21  -218}\special{pa    42  -230}%
\special{pa   101  -242}\special{pa   156  -255}\special{pa   206  -268}\special{pa   253  -281}%
\special{pa   295  -295}\special{pa   356  -317}\special{pa   416  -341}\special{pa   475  -367}%
\special{pa   535  -394}\special{pa   594  -423}\special{pa   653  -454}\special{pa   711  -486}%
\special{pa   770  -520}\special{pa   828  -555}\special{pa   886  -591}\special{pa   943  -627}%
\special{pa  1000  -665}\special{pa  1056  -704}\special{pa  1112  -745}\special{pa  1168  -787}%
\special{pa  1224  -831}\special{pa  1279  -876}\special{pa  1333  -922}\special{pa  1387  -970}%
\special{pa  1440 -1020}\special{pa  1476 -1057}\special{pa  1476 -1057}%
\special{fp}%
\special{pn 8}%
\special{pa 0 -295}\special{pa 35 -295}\special{fp}\special{pa 69 -295}\special{pa 104 -295}\special{fp}%
\special{pa 139 -295}\special{pa 174 -295}\special{fp}\special{pa 208 -295}\special{pa 243 -295}\special{fp}%
\special{pa 278 -295}\special{pa 295 -295}\special{pa 295 -278}\special{fp}\special{pa 295 -243}\special{pa 295 -208}\special{fp}%
\special{pa 295 -174}\special{pa 295 -139}\special{fp}\special{pa 295 -104}\special{pa 295 -69}\special{fp}%
\special{pa 295 -35}\special{pa 295 0}\special{fp}%
%
\special{pa 0 -591}\special{pa 38 -591}\special{fp}\special{pa 76 -591}\special{pa 114 -591}\special{fp}%
\special{pa 151 -591}\special{pa 189 -591}\special{fp}\special{pa 227 -591}\special{pa 265 -591}\special{fp}%
\special{pa 303 -591}\special{pa 341 -591}\special{fp}\special{pa 379 -591}\special{pa 416 -591}\special{fp}%
\special{pa 454 -591}\special{pa 492 -591}\special{fp}\special{pa 530 -591}\special{pa 568 -591}\special{fp}%
\special{pa 606 -591}\special{pa 644 -591}\special{fp}\special{pa 681 -591}\special{pa 719 -591}\special{fp}%
\special{pa 757 -591}\special{pa 795 -591}\special{fp}\special{pa 833 -591}\special{pa 871 -591}\special{fp}%
\special{pa 886 -568}\special{pa 886 -530}\special{fp}\special{pa 886 -492}\special{pa 886 -454}\special{fp}%
\special{pa 886 -416}\special{pa 886 -379}\special{fp}\special{pa 886 -341}\special{pa 886 -303}\special{fp}%
\special{pa 886 -265}\special{pa 886 -227}\special{fp}\special{pa 886 -189}\special{pa 886 -151}\special{fp}%
\special{pa 886 -114}\special{pa 886 -76}\special{fp}\special{pa 886 -38}\special{pa 886 0}\special{fp}%
%
%
\special{pa    20  -295}\special{pa   -20  -295}%
\special{fp}%
\settowidth{\Width}{$f(x)$}\setlength{\Width}{-1\Width}%
\settoheight{\Height}{$f(x)$}\settodepth{\Depth}{$f(x)$}\setlength{\Height}{-0.5\Height}\setlength{\Depth}{0.5\Depth}\addtolength{\Height}{\Depth}%
\put(-0.1333333,1.0000000){\hspace*{\Width}\raisebox{\Height}{$f(x)$}}%
%
\special{pa    20  -591}\special{pa   -20  -591}%
\special{fp}%
\settowidth{\Width}{$f(\xi)$}\setlength{\Width}{-1\Width}%
\settoheight{\Height}{$f(\xi)$}\settodepth{\Depth}{$f(\xi)$}\setlength{\Height}{-0.5\Height}\setlength{\Depth}{0.5\Depth}\addtolength{\Height}{\Depth}%
\put(-0.1333333,2.0000000){\hspace*{\Width}\raisebox{\Height}{$f(\xi)$}}%
%
\special{pa   295   -20}\special{pa   295    20}%
\special{fp}%
\settowidth{\Width}{$x$}\setlength{\Width}{-0.5\Width}%
\settoheight{\Height}{$x$}\settodepth{\Depth}{$x$}\setlength{\Height}{-\Height}%
\put(1.0000000,-0.1333333){\hspace*{\Width}\raisebox{\Height}{$x$}}%
%
\special{pa   886   -20}\special{pa   886    20}%
\special{fp}%
\settowidth{\Width}{$\xi$}\setlength{\Width}{-0.5\Width}%
\settoheight{\Height}{$\xi$}\settodepth{\Depth}{$\xi$}\setlength{\Height}{-\Height}%
\put(3.0000000,-0.1333333){\hspace*{\Width}\raisebox{\Height}{$\xi$}}%
%
\settowidth{\Width}{$\varDelta x$}\setlength{\Width}{-0.5\Width}%
\settoheight{\Height}{$\varDelta x$}\settodepth{\Depth}{$\varDelta x$}\setlength{\Height}{-0.5\Height}\setlength{\Depth}{0.5\Depth}\addtolength{\Height}{\Depth}%
\put(2.0000000,0.3300000){\hspace*{\Width}\raisebox{\Height}{$\varDelta x$}}%
%
\special{pa   886    -0}\special{pa   875    -5}\special{pa   863    -9}\special{pa   852   -13}%
\special{pa   841   -17}\special{pa   829   -21}\special{pa   817   -25}\special{pa   806   -28}%
\special{pa   794   -32}\special{pa   782   -35}\special{pa   771   -38}\special{pa   759   -40}%
\special{pa   747   -43}\special{pa   735   -45}\special{pa   723   -48}\special{pa   711   -50}%
\special{pa   699   -51}\special{pa   687   -53}\special{pa   675   -54}\special{pa   663   -56}%
\special{pa   651   -57}\special{pa   639   -58}\special{pa   627   -58}\special{pa   615   -59}%
\special{pa   603   -59}\special{pa   591   -59}\special{pa   578   -59}\special{pa   566   -59}%
\special{pa   554   -58}\special{pa   542   -58}\special{pa   530   -57}\special{pa   518   -56}%
\special{pa   506   -54}\special{pa   494   -53}\special{pa   482   -51}\special{pa   470   -50}%
\special{pa   458   -48}\special{pa   446   -45}\special{pa   434   -43}\special{pa   422   -40}%
\special{pa   410   -38}\special{pa   399   -35}\special{pa   387   -32}\special{pa   375   -28}%
\special{pa   364   -25}\special{pa   352   -21}\special{pa   341   -17}\special{pa   329   -13}%
\special{pa   318    -9}\special{pa   307    -5}\special{pa   295     0}%
\special{fp}%
\settowidth{\Width}{$\varDelta y$}\setlength{\Width}{-0.5\Width}%
\settoheight{\Height}{$\varDelta y$}\settodepth{\Depth}{$\varDelta y$}\setlength{\Height}{-0.5\Height}\setlength{\Depth}{0.5\Depth}\addtolength{\Height}{\Depth}%
\put(0.3700000,1.5000000){\hspace*{\Width}\raisebox{\Height}{$\varDelta y$}}%
%
\special{pa     0  -295}\special{pa     2  -301}\special{pa     4  -307}\special{pa     7  -312}%
\special{pa     9  -318}\special{pa    11  -324}\special{pa    12  -329}\special{pa    14  -335}%
\special{pa    16  -341}\special{pa    17  -347}\special{pa    19  -353}\special{pa    20  -359}%
\special{pa    21  -365}\special{pa    23  -371}\special{pa    24  -377}\special{pa    25  -383}%
\special{pa    26  -389}\special{pa    26  -395}\special{pa    27  -401}\special{pa    28  -407}%
\special{pa    28  -413}\special{pa    29  -419}\special{pa    29  -425}\special{pa    29  -431}%
\special{pa    29  -437}\special{pa    30  -443}\special{pa    29  -449}\special{pa    29  -455}%
\special{pa    29  -461}\special{pa    29  -467}\special{pa    28  -473}\special{pa    28  -479}%
\special{pa    27  -485}\special{pa    26  -491}\special{pa    26  -497}\special{pa    25  -503}%
\special{pa    24  -509}\special{pa    23  -515}\special{pa    21  -521}\special{pa    20  -527}%
\special{pa    19  -533}\special{pa    17  -539}\special{pa    16  -545}\special{pa    14  -551}%
\special{pa    12  -556}\special{pa    11  -562}\special{pa     9  -568}\special{pa     7  -574}%
\special{pa     4  -579}\special{pa     2  -585}\special{pa     0  -591}%
\special{fp}%
{%
\color[rgb]{0,0,1}%
\special{pa  -148  -141}\special{pa  1476  -708}%
\special{fp}%
}%
{%
\color[rgb]{1,0,0}%
\special{pa   295  -295}\special{pa   886  -295}\special{pa   886  -502}\special{pa   295  -295}%
\special{fp}%
}%
\settowidth{\Width}{${\color{red}dx}$}\setlength{\Width}{-0.5\Width}%
\settoheight{\Height}{${\color{red}dx}$}\settodepth{\Depth}{${\color{red}dx}$}\setlength{\Height}{-0.5\Height}\setlength{\Depth}{0.5\Depth}\addtolength{\Height}{\Depth}%
\put(2.0000000,0.6700000){\hspace*{\Width}\raisebox{\Height}{${\color{red}dx}$}}%
%
\special{pa   295  -295}\special{pa   307  -291}\special{pa   318  -286}\special{pa   329  -282}%
\special{pa   341  -278}\special{pa   352  -274}\special{pa   364  -271}\special{pa   375  -267}%
\special{pa   387  -264}\special{pa   399  -261}\special{pa   410  -258}\special{pa   422  -255}%
\special{pa   434  -252}\special{pa   446  -250}\special{pa   458  -248}\special{pa   470  -246}%
\special{pa   482  -244}\special{pa   494  -242}\special{pa   506  -241}\special{pa   518  -240}%
\special{pa   530  -239}\special{pa   542  -238}\special{pa   554  -237}\special{pa   566  -237}%
\special{pa   578  -236}\special{pa   591  -236}\special{pa   603  -236}\special{pa   615  -237}%
\special{pa   627  -237}\special{pa   639  -238}\special{pa   651  -239}\special{pa   663  -240}%
\special{pa   675  -241}\special{pa   687  -242}\special{pa   699  -244}\special{pa   711  -246}%
\special{pa   723  -248}\special{pa   735  -250}\special{pa   747  -252}\special{pa   759  -255}%
\special{pa   771  -258}\special{pa   782  -261}\special{pa   794  -264}\special{pa   806  -267}%
\special{pa   817  -271}\special{pa   829  -274}\special{pa   841  -278}\special{pa   852  -282}%
\special{pa   863  -286}\special{pa   875  -291}\special{pa   886  -295}%
\special{fp}%
\settowidth{\Width}{${\color{red}dy}$}\setlength{\Width}{-0.5\Width}%
\settoheight{\Height}{${\color{red}dy}$}\settodepth{\Depth}{${\color{red}dy}$}\setlength{\Height}{-0.5\Height}\setlength{\Depth}{0.5\Depth}\addtolength{\Height}{\Depth}%
\put(3.4100000,1.3500000){\hspace*{\Width}\raisebox{\Height}{${\color{red}dy}$}}%
%
\special{pa   886  -295}\special{pa   889  -299}\special{pa   892  -302}\special{pa   895  -306}%
\special{pa   898  -309}\special{pa   900  -313}\special{pa   903  -317}\special{pa   905  -321}%
\special{pa   907  -324}\special{pa   910  -328}\special{pa   912  -332}\special{pa   914  -337}%
\special{pa   916  -341}\special{pa   917  -345}\special{pa   919  -349}\special{pa   920  -354}%
\special{pa   921  -358}\special{pa   923  -362}\special{pa   924  -367}\special{pa   925  -371}%
\special{pa   925  -376}\special{pa   926  -380}\special{pa   926  -385}\special{pa   927  -389}%
\special{pa   927  -394}\special{pa   927  -398}\special{pa   927  -403}\special{pa   927  -407}%
\special{pa   926  -412}\special{pa   926  -417}\special{pa   925  -421}\special{pa   925  -426}%
\special{pa   924  -430}\special{pa   923  -434}\special{pa   921  -439}\special{pa   920  -443}%
\special{pa   919  -448}\special{pa   917  -452}\special{pa   916  -456}\special{pa   914  -460}%
\special{pa   912  -464}\special{pa   910  -468}\special{pa   907  -472}\special{pa   905  -476}%
\special{pa   903  -480}\special{pa   900  -484}\special{pa   898  -488}\special{pa   895  -491}%
\special{pa   892  -495}\special{pa   889  -498}\special{pa   886  -502}%
\special{fp}%
\special{pa  -148    -0}\special{pa  1476    -0}%
\special{fp}%
\special{pa     0   148}\special{pa     0 -1181}%
\special{fp}%
\settowidth{\Width}{$x$}\setlength{\Width}{0\Width}%
\settoheight{\Height}{$x$}\settodepth{\Depth}{$x$}\setlength{\Height}{-0.5\Height}\setlength{\Depth}{0.5\Depth}\addtolength{\Height}{\Depth}%
\put(5.0666667,0.0000000){\hspace*{\Width}\raisebox{\Height}{$x$}}%
%
\settowidth{\Width}{$y$}\setlength{\Width}{-0.5\Width}%
\settoheight{\Height}{$y$}\settodepth{\Depth}{$y$}\setlength{\Height}{\Depth}%
\put(0.0000000,4.0666667){\hspace*{\Width}\raisebox{\Height}{$y$}}%
%
\settowidth{\Width}{O}\setlength{\Width}{-1\Width}%
\settoheight{\Height}{O}\settodepth{\Depth}{O}\setlength{\Height}{-\Height}%
\put(-0.0666667,-0.0666667){\hspace*{\Width}\raisebox{\Height}{O}}%
%
\end{picture}}%}
\end{layer}

\begin{itemize}
\item
1変数関数$y=f(x)$
\item
$dy=\bunsuu{dy}{dx}dx$
\item
$\varDelta x$が小さいとき\\
\hspace*{2zw}$\varDelta y\fallingdotseq \bunsuu{dy}{dx}\varDelta x$
\end{itemize}

\newslide{2変数関数の場合}

\vspace*{18mm}


\begin{layer}{120}{0}
\putnotew{96}{73}{\hyperlink{para10pg2}{\fbox{\Ctab{2.5mm}{\scalebox{1}{\scriptsize $\mathstrut||\!\lhd$}}}}}
\putnotew{101}{73}{\hyperlink{para11pg1}{\fbox{\Ctab{2.5mm}{\scalebox{1}{\scriptsize $\mathstrut|\!\lhd$}}}}}
\putnotew{108}{73}{\hyperlink{para11pg2}{\fbox{\Ctab{4.5mm}{\scalebox{1}{\scriptsize $\mathstrut\!\!\lhd\!\!$}}}}}
\putnotew{115}{73}{\hyperlink{para11pg3}{\fbox{\Ctab{4.5mm}{\scalebox{1}{\scriptsize $\mathstrut\!\rhd\!$}}}}}
\putnotew{120}{73}{\hyperlink{para11pg3}{\fbox{\Ctab{2.5mm}{\scalebox{1}{\scriptsize $\mathstrut \!\rhd\!\!|$}}}}}
\putnotew{125}{73}{\hyperlink{para12pg1}{\fbox{\Ctab{2.5mm}{\scalebox{1}{\scriptsize $\mathstrut \!\rhd\!\!||$}}}}}
\putnotee{126}{73}{\scriptsize\color{blue} 3/3}
\end{layer}

\slidepage

\begin{layer}{120}{0}
\putnotes{65}{2}{%%% /Users/takatoosetsuo/Dropbox/2018polytec/lecture/0920/presen/fig/henbibun2.tex 
%%% Generator=zenbibun180920.cdy 
{\unitlength=1cm%
\begin{picture}%
(8.69,6.79)(-2.75,-1.17)%
\special{pn 8}%
%
\settowidth{\Width}{$x$}\setlength{\Width}{-0.5\Width}%
\settoheight{\Height}{$x$}\settodepth{\Depth}{$x$}\setlength{\Height}{-0.5\Height}\setlength{\Depth}{0.5\Depth}\addtolength{\Height}{\Depth}%
\put(4.2800000,-0.7800000){\hspace*{\Width}\raisebox{\Height}{$x$}}%
%
\settowidth{\Width}{$y$}\setlength{\Width}{-0.5\Width}%
\settoheight{\Height}{$y$}\settodepth{\Depth}{$y$}\setlength{\Height}{-0.5\Height}\setlength{\Depth}{0.5\Depth}\addtolength{\Height}{\Depth}%
\put(3.0500000,1.1300000){\hspace*{\Width}\raisebox{\Height}{$y$}}%
%
\settowidth{\Width}{$z$}\setlength{\Width}{-0.5\Width}%
\settoheight{\Height}{$z$}\settodepth{\Depth}{$z$}\setlength{\Height}{-0.5\Height}\setlength{\Depth}{0.5\Depth}\addtolength{\Height}{\Depth}%
\put(0.0000000,5.0200000){\hspace*{\Width}\raisebox{\Height}{$z$}}%
%
\special{pa     0    -0}\special{pa  1613   292}%
\special{fp}%
\special{pa     0    -0}\special{pa   382  -141}%
\special{fp}%
\special{pa   441  -163}\special{pa   776  -287}%
\special{fp}%
\special{pa   836  -309}\special{pa   946  -350}%
\special{fp}%
\special{pa  1005  -372}\special{pa  1129  -417}%
\special{fp}%
\special{pa     0    -0}\special{pa     0 -1901}%
\special{fp}%
\special{pa   976    84}\special{pa   976 -1274}%
\special{fp}%
\special{pa  1371   -63}\special{pa  1371 -1706}%
\special{fp}%
\special{pa   806  -165}\special{pa   806 -1114}%
\special{fp}%
\special{pa   806 -1175}\special{pa   806 -1218}%
\special{fp}%
\special{pa   806 -1279}\special{pa   806 -1573}%
\special{fp}%
\special{pa   411 -1216}\special{pa   524 -1221}\special{pa   637 -1230}\special{pa   750 -1242}%
\special{pa   863 -1256}\special{pa   976 -1274}%
\special{fp}%
\special{pa   411 -1216}\special{pa   490 -1273}\special{pa   569 -1338}\special{pa   648 -1409}%
\special{pa   727 -1488}\special{pa   806 -1573}%
\special{fp}%
\special{pa   806 -1573}\special{pa   919 -1597}\special{pa  1032 -1622}\special{pa  1145 -1648}%
\special{pa  1258 -1676}\special{pa  1371 -1706}%
\special{fp}%
\special{pa   976 -1274}\special{pa  1055 -1344}\special{pa  1134 -1422}\special{pa  1213 -1509}%
\special{pa  1292 -1604}\special{pa  1371 -1706}%
\special{fp}%
\special{pa   411   -19}\special{pa   976    84}\special{pa  1371   -63}\special{pa  1005  -129}%
\special{fp}%
\special{pa   946  -140}\special{pa   806  -165}\special{pa   411   -19}%
\special{fp}%
\special{pa   411   -19}\special{pa   411 -1216}%
\special{fp}%
\special{pa   411 -1216}\special{pa   976 -1114}\special{pa  1371 -1260}\special{pa  1132 -1303}%
\special{fp}%
\special{pa   994 -1328}\special{pa   806 -1362}\special{pa   411 -1216}%
\special{fp}%
\special{pa   976    84}\special{pa   976 -1114}%
\special{fp}%
\special{pa  1371   -63}\special{pa  1371 -1260}%
\special{fp}%
\special{pa   806  -165}\special{pa   806 -1114}%
\special{fp}%
\special{pa   806 -1175}\special{pa   806 -1218}%
\special{fp}%
\special{pa   806 -1279}\special{pa   806 -1362}%
\special{fp}%
\special{pa   976    84}\special{pa   976 -1232}%
\special{fp}%
\special{pa  1371   -63}\special{pa  1371 -1497}%
\special{fp}%
\special{pa   806  -165}\special{pa   806 -1114}%
\special{fp}%
\special{pa   806 -1175}\special{pa   806 -1218}%
\special{fp}%
\special{pa   806 -1279}\special{pa   806 -1482}%
\special{fp}%
\color[rgb]{0,0,1}%
\special{pa   411 -1216}\special{pa   976 -1232}\special{pa  1371 -1497}\special{pa  1237 -1493}%
\special{fp}%
\special{pa  1159 -1491}\special{pa   806 -1482}\special{pa   411 -1216}%
\special{fp}%
\color[rgb]{0,0,0}%
\end{picture}}%}
\end{layer}


\newslide{2変数関数の場合}

\vspace*{18mm}


\begin{layer}{120}{0}
\putnotes{65}{2}{%%% /Users/takatoosetsuo/Dropbox/2018polytec/lecture/0521/presen/fig/sinecurve/p008.tex 
%%% Generator=presen0521.cdy 
{\unitlength=12.5mm%
\begin{picture}%
(9.2,2.4)(-2.2,-1.2)%
\special{pn 8}%
%
\Large\bf\boldmath%
\small%
\special{pa     0    -0}\special{pa    -4   -62}\special{pa   -15  -122}\special{pa   -35  -181}%
\special{pa   -61  -237}\special{pa   -94  -289}\special{pa  -133  -337}\special{pa  -178  -379}%
\special{pa  -228  -416}\special{pa  -283  -445}\special{pa  -340  -468}\special{pa  -400  -483}%
\special{pa  -461  -491}\special{pa  -523  -491}\special{pa  -584  -483}\special{pa  -644  -468}%
\special{pa  -702  -445}\special{pa  -756  -416}\special{pa  -806  -379}\special{pa  -851  -337}%
\special{pa  -890  -289}\special{pa  -923  -237}\special{pa  -950  -181}\special{pa  -969  -122}%
\special{pa  -980   -62}\special{pa  -984     0}\special{pa  -980    62}\special{pa  -969   122}%
\special{pa  -950   181}\special{pa  -923   237}\special{pa  -890   289}\special{pa  -851   337}%
\special{pa  -806   379}\special{pa  -756   416}\special{pa  -702   445}\special{pa  -644   468}%
\special{pa  -584   483}\special{pa  -523   491}\special{pa  -461   491}\special{pa  -400   483}%
\special{pa  -340   468}\special{pa  -283   445}\special{pa  -228   416}\special{pa  -178   379}%
\special{pa  -133   337}\special{pa   -94   289}\special{pa   -61   237}\special{pa   -35   181}%
\special{pa   -15   122}\special{pa    -4    62}\special{pa     0     0}%
\special{fp}%
\special{pa  -492    -0}\special{pa     0    -0}%
\special{fp}%
\special{pa  -492    -0}\special{pa  -584  -483}%
\special{fp}%
\special{pa 866 -483}\special{pa 866 -446}\special{fp}\special{pa 866 -409}\special{pa 866 -372}\special{fp}%
\special{pa 866 -335}\special{pa 866 -297}\special{fp}\special{pa 866 -260}\special{pa 866 -223}\special{fp}%
\special{pa 866 -186}\special{pa 866 -149}\special{fp}\special{pa 866 -112}\special{pa 866 -74}\special{fp}%
\special{pa 866 -37}\special{pa 866 0}\special{fp}%
%
\special{pa -584 -483}\special{pa -545 -483}\special{fp}\special{pa -506 -483}\special{pa -467 -483}\special{fp}%
\special{pa -428 -483}\special{pa -388 -483}\special{fp}\special{pa -349 -483}\special{pa -310 -483}\special{fp}%
\special{pa -271 -483}\special{pa -232 -483}\special{fp}\special{pa -192 -483}\special{pa -153 -483}\special{fp}%
\special{pa -114 -483}\special{pa -75 -483}\special{fp}\special{pa -36 -483}\special{pa 4 -483}\special{fp}%
\special{pa 43 -483}\special{pa 82 -483}\special{fp}\special{pa 121 -483}\special{pa 160 -483}\special{fp}%
\special{pa 200 -483}\special{pa 239 -483}\special{fp}\special{pa 278 -483}\special{pa 317 -483}\special{fp}%
\special{pa 356 -483}\special{pa 395 -483}\special{fp}\special{pa 435 -483}\special{pa 474 -483}\special{fp}%
\special{pa 513 -483}\special{pa 552 -483}\special{fp}\special{pa 591 -483}\special{pa 631 -483}\special{fp}%
\special{pa 670 -483}\special{pa 709 -483}\special{fp}\special{pa 748 -483}\special{pa 787 -483}\special{fp}%
\special{pa 827 -483}\special{pa 866 -483}\special{fp}%
%
\settowidth{\Width}{$x$}\setlength{\Width}{-0.5\Width}%
\settoheight{\Height}{$x$}\settodepth{\Depth}{$x$}\setlength{\Height}{-0.5\Height}\setlength{\Depth}{0.5\Depth}\addtolength{\Height}{\Depth}%
\put(-0.6700000,0.4000000){\hspace*{\Width}\raisebox{\Height}{$x$}}%
%
\special{pa  -320    -0}\special{pa  -321   -22}\special{pa  -325   -43}\special{pa  -332   -63}%
\special{pa  -341   -83}\special{pa  -353  -101}\special{pa  -367  -118}\special{pa  -382  -133}%
\special{pa  -400  -145}\special{pa  -419  -156}\special{pa  -439  -164}\special{pa  -460  -169}%
\special{pa  -481  -172}\special{pa  -503  -172}\special{pa  -524  -169}%
\special{fp}%
\color[cmyk]{0,1,1,0}%
\special{pa     0    -0}\special{pa    17   -17}\special{pa    35   -35}\special{pa    52   -52}%
\special{pa    69   -69}\special{pa    87   -86}\special{pa   104  -103}\special{pa   121  -120}%
\special{pa   139  -137}\special{pa   156  -153}\special{pa   173  -170}\special{pa   190  -186}%
\special{pa   208  -202}\special{pa   225  -217}\special{pa   242  -233}\special{pa   260  -248}%
\special{pa   277  -263}\special{pa   294  -277}\special{pa   312  -291}\special{pa   329  -305}%
\special{pa   346  -318}\special{pa   364  -331}\special{pa   381  -344}\special{pa   398  -356}%
\special{pa   416  -368}\special{pa   433  -379}\special{pa   450  -390}\special{pa   468  -400}%
\special{pa   485  -410}\special{pa   502  -419}\special{pa   519  -428}\special{pa   537  -437}%
\special{pa   554  -444}\special{pa   571  -451}\special{pa   589  -458}\special{pa   606  -464}%
\special{pa   623  -470}\special{pa   641  -474}\special{pa   658  -479}\special{pa   675  -482}%
\special{pa   693  -486}\special{pa   710  -488}\special{pa   727  -490}\special{pa   745  -491}%
\special{pa   762  -492}\special{pa   779  -492}\special{pa   797  -492}\special{pa   814  -490}%
\special{pa   831  -489}\special{pa   848  -486}\special{pa   866  -483}%
\special{fp}%
\color[cmyk]{0,0,0,1}%
\special{pa   773   -20}\special{pa   773    20}%
\special{fp}%
\settowidth{\Width}{$\frac{\pi}{2}$}\setlength{\Width}{-0.5\Width}%
\settoheight{\Height}{$\frac{\pi}{2}$}\settodepth{\Depth}{$\frac{\pi}{2}$}\setlength{\Height}{\Depth}%
\put(1.5707960,0.0800000){\hspace*{\Width}\raisebox{\Height}{$\frac{\pi}{2}$}}%
%
%
\special{pa  1546   -20}\special{pa  1546    20}%
\special{fp}%
\settowidth{\Width}{$\pi$}\setlength{\Width}{-0.5\Width}%
\settoheight{\Height}{$\pi$}\settodepth{\Depth}{$\pi$}\setlength{\Height}{\Depth}%
\put(3.1415930,0.0800000){\hspace*{\Width}\raisebox{\Height}{$\pi$}}%
%
%
\special{pa  3092   -20}\special{pa  3092    20}%
\special{fp}%
\settowidth{\Width}{$2\pi$}\setlength{\Width}{-0.5\Width}%
\settoheight{\Height}{$2\pi$}\settodepth{\Depth}{$2\pi$}\setlength{\Height}{\Depth}%
\put(6.2831850,0.0800000){\hspace*{\Width}\raisebox{\Height}{$2\pi$}}%
%
%
\special{pa    20   492}\special{pa   -20   492}%
\special{fp}%
\settowidth{\Width}{$-1$}\setlength{\Width}{-1\Width}%
\settoheight{\Height}{$-1$}\settodepth{\Depth}{$-1$}\setlength{\Height}{-0.5\Height}\setlength{\Depth}{0.5\Depth}\addtolength{\Height}{\Depth}%
\put(-0.0800000,-1.0000000){\hspace*{\Width}\raisebox{\Height}{$-1$}}%
%
%
\special{pa    20  -492}\special{pa   -20  -492}%
\special{fp}%
\settowidth{\Width}{$1$}\setlength{\Width}{-1\Width}%
\settoheight{\Height}{$1$}\settodepth{\Depth}{$1$}\setlength{\Height}{-0.5\Height}\setlength{\Depth}{0.5\Depth}\addtolength{\Height}{\Depth}%
\put(-0.0800000,1.0000000){\hspace*{\Width}\raisebox{\Height}{$1$}}%
%
%
\special{pa -1083    -0}\special{pa  3445    -0}%
\special{fp}%
\special{pa     0   591}\special{pa     0  -591}%
\special{fp}%
\settowidth{\Width}{$x$}\setlength{\Width}{0\Width}%
\settoheight{\Height}{$x$}\settodepth{\Depth}{$x$}\setlength{\Height}{-0.5\Height}\setlength{\Depth}{0.5\Depth}\addtolength{\Height}{\Depth}%
\put(7.0400000,0.0000000){\hspace*{\Width}\raisebox{\Height}{$x$}}%
%
\settowidth{\Width}{$y$}\setlength{\Width}{-0.5\Width}%
\settoheight{\Height}{$y$}\settodepth{\Depth}{$y$}\setlength{\Height}{\Depth}%
\put(0.0000000,1.2400000){\hspace*{\Width}\raisebox{\Height}{$y$}}%
%
\settowidth{\Width}{O}\setlength{\Width}{0\Width}%
\settoheight{\Height}{O}\settodepth{\Depth}{O}\setlength{\Height}{-\Height}%
\put(0.0400000,-0.0400000){\hspace*{\Width}\raisebox{\Height}{O}}%
%
\end{picture}}%}
\putnotew{96}{73}{\hyperlink{para11pg3}{\fbox{\Ctab{2.5mm}{\scalebox{1}{\scriptsize $\mathstrut||\!\lhd$}}}}}
\putnotew{101}{73}{\hyperlink{para12pg1}{\fbox{\Ctab{2.5mm}{\scalebox{1}{\scriptsize $\mathstrut|\!\lhd$}}}}}
\putnotew{108}{73}{\hyperlink{para12pg7}{\fbox{\Ctab{4.5mm}{\scalebox{1}{\scriptsize $\mathstrut\!\!\lhd\!\!$}}}}}
\putnotew{115}{73}{\hyperlink{para12pg8}{\fbox{\Ctab{4.5mm}{\scalebox{1}{\scriptsize $\mathstrut\!\rhd\!$}}}}}
\putnotew{120}{73}{\hyperlink{para12pg8}{\fbox{\Ctab{2.5mm}{\scalebox{1}{\scriptsize $\mathstrut \!\rhd\!\!|$}}}}}
\putnotew{125}{73}{\hyperlink{para13pg1}{\fbox{\Ctab{2.5mm}{\scalebox{1}{\scriptsize $\mathstrut \!\rhd\!\!||$}}}}}
\putnotee{126}{73}{\scriptsize\color{blue} 8/8}
\end{layer}

\slidepage

\newslide{2変数関数の偏微分}

\vspace*{18mm}


\begin{layer}{120}{0}
\putnotew{96}{73}{\hyperlink{para12pg8}{\fbox{\Ctab{2.5mm}{\scalebox{1}{\scriptsize $\mathstrut||\!\lhd$}}}}}
\putnotew{101}{73}{\hyperlink{para13pg1}{\fbox{\Ctab{2.5mm}{\scalebox{1}{\scriptsize $\mathstrut|\!\lhd$}}}}}
\putnotew{108}{73}{\hyperlink{para13pg3}{\fbox{\Ctab{4.5mm}{\scalebox{1}{\scriptsize $\mathstrut\!\!\lhd\!\!$}}}}}
\putnotew{115}{73}{\hyperlink{para13pg4}{\fbox{\Ctab{4.5mm}{\scalebox{1}{\scriptsize $\mathstrut\!\rhd\!$}}}}}
\putnotew{120}{73}{\hyperlink{para13pg4}{\fbox{\Ctab{2.5mm}{\scalebox{1}{\scriptsize $\mathstrut \!\rhd\!\!|$}}}}}
\putnotew{125}{73}{\hyperlink{para14pg1}{\fbox{\Ctab{2.5mm}{\scalebox{1}{\scriptsize $\mathstrut \!\rhd\!\!||$}}}}}
\putnotee{126}{73}{\scriptsize\color{blue} 4/4}
\end{layer}

\slidepage

\begin{layer}{120}{0}
\putnotese{85}{9}{%%% /Users/takatoosetsuo/Dropbox/2021polytech/108/fig/bibun4.tex 
%%% Generator=doukansuu2.cdy 
{\unitlength=7mm%
\begin{picture}%
(8,8)(-4,-4)%
\special{pn 8}%
%
\special{pn 12}%
\special{pa -1102  -413}\special{pa  1102  -413}%
\special{fp}%
\special{pn 8}%
\special{pa    20  -413}\special{pa   -20  -413}%
\special{fp}%
\settowidth{\Width}{$c$}\setlength{\Width}{-1\Width}%
\settoheight{\Height}{$c$}\settodepth{\Depth}{$c$}\setlength{\Height}{-\Height}%
\put(-0.0714286,1.4285714){\hspace*{\Width}\raisebox{\Height}{$c$}}%
%
\special{pa -1102    -0}\special{pa  1102    -0}%
\special{fp}%
\special{pa     0  1102}\special{pa     0 -1102}%
\special{fp}%
\settowidth{\Width}{$x$}\setlength{\Width}{0\Width}%
\settoheight{\Height}{$x$}\settodepth{\Depth}{$x$}\setlength{\Height}{-0.5\Height}\setlength{\Depth}{0.5\Depth}\addtolength{\Height}{\Depth}%
\put(4.0714286,0.0000000){\hspace*{\Width}\raisebox{\Height}{$x$}}%
%
\settowidth{\Width}{$y$}\setlength{\Width}{-0.5\Width}%
\settoheight{\Height}{$y$}\settodepth{\Depth}{$y$}\setlength{\Height}{\Depth}%
\put(0.0000000,4.0714286){\hspace*{\Width}\raisebox{\Height}{$y$}}%
%
\settowidth{\Width}{O}\setlength{\Width}{0\Width}%
\settoheight{\Height}{O}\settodepth{\Depth}{O}\setlength{\Height}{-\Height}%
\put(0.0714286,-0.0714286){\hspace*{\Width}\raisebox{\Height}{O}}%
%
\end{picture}}%}
\end{layer}

\begin{itemize}
\item
2変数関数$z=f(x,y)$
\item
$x$だけを変化させる\vspace{-2mm}
\item
$\varDelta z_1=f(\xi,y)-f(x,y)$\vspace{-2mm}
\item
$dz_1=\bunsuu{\partial z}{\partial x}dx$\vspace{-2mm}
\item
$\varDelta x$が小さいとき\vspace{-1mm}
\hspace*{1zw}$\varDelta z_1\fallingdotseq \bunsuu{\partial z}{\partial x}\varDelta x$\vspace{-1mm}
\item
$y$についても同様 $dz_2=\bunsuu{\partial z}{\partial y}dy$
\end{itemize}
\fi

\newslide{2変数関数の偏微分}

\vspace*{18mm}


\begin{layer}{120}{0}
\putnotew{96}{73}{\hyperlink{para13pg4}{\fbox{\Ctab{2.5mm}{\scalebox{1}{\scriptsize $\mathstrut||\!\lhd$}}}}}
\putnotew{101}{73}{\hyperlink{para14pg1}{\fbox{\Ctab{2.5mm}{\scalebox{1}{\scriptsize $\mathstrut|\!\lhd$}}}}}
\putnotew{108}{73}{\hyperlink{para14pg5}{\fbox{\Ctab{4.5mm}{\scalebox{1}{\scriptsize $\mathstrut\!\!\lhd\!\!$}}}}}
\putnotew{115}{73}{\hyperlink{para14pg6}{\fbox{\Ctab{4.5mm}{\scalebox{1}{\scriptsize $\mathstrut\!\rhd\!$}}}}}
\putnotew{120}{73}{\hyperlink{para14pg6}{\fbox{\Ctab{2.5mm}{\scalebox{1}{\scriptsize $\mathstrut \!\rhd\!\!|$}}}}}
\putnotew{125}{73}{\hyperlink{para15pg1}{\fbox{\Ctab{2.5mm}{\scalebox{1}{\scriptsize $\mathstrut \!\rhd\!\!||$}}}}}
\putnotee{126}{73}{\scriptsize\color{blue} 6/6}
\end{layer}

\slidepage

\begin{layer}{120}{0}
\putnotese{86}{15}{%%% /Users/takatoosetsuo/Dropbox/2021polytech/108/fig/bibun4.tex 
%%% Generator=doukansuu2.cdy 
{\unitlength=7mm%
\begin{picture}%
(8,8)(-4,-4)%
\special{pn 8}%
%
\special{pn 12}%
\special{pa -1102  -413}\special{pa  1102  -413}%
\special{fp}%
\special{pn 8}%
\special{pa    20  -413}\special{pa   -20  -413}%
\special{fp}%
\settowidth{\Width}{$c$}\setlength{\Width}{-1\Width}%
\settoheight{\Height}{$c$}\settodepth{\Depth}{$c$}\setlength{\Height}{-\Height}%
\put(-0.0714286,1.4285714){\hspace*{\Width}\raisebox{\Height}{$c$}}%
%
\special{pa -1102    -0}\special{pa  1102    -0}%
\special{fp}%
\special{pa     0  1102}\special{pa     0 -1102}%
\special{fp}%
\settowidth{\Width}{$x$}\setlength{\Width}{0\Width}%
\settoheight{\Height}{$x$}\settodepth{\Depth}{$x$}\setlength{\Height}{-0.5\Height}\setlength{\Depth}{0.5\Depth}\addtolength{\Height}{\Depth}%
\put(4.0714286,0.0000000){\hspace*{\Width}\raisebox{\Height}{$x$}}%
%
\settowidth{\Width}{$y$}\setlength{\Width}{-0.5\Width}%
\settoheight{\Height}{$y$}\settodepth{\Depth}{$y$}\setlength{\Height}{\Depth}%
\put(0.0000000,4.0714286){\hspace*{\Width}\raisebox{\Height}{$y$}}%
%
\settowidth{\Width}{O}\setlength{\Width}{0\Width}%
\settoheight{\Height}{O}\settodepth{\Depth}{O}\setlength{\Height}{-\Height}%
\put(0.0714286,-0.0714286){\hspace*{\Width}\raisebox{\Height}{O}}%
%
\end{picture}}%}
\end{layer}

\begin{itemize}
\item
2変数関数$z=f(x,y)$
\item
$\bunsuu{\partial z}{\partial x}$は
$x$だけが変化したときの変化率\vspace{-2mm}
\item
[]変化量\;$\varDelta z_1=f(x',y)-f(x,y)$\vspace{-2mm}
\item
[] $\Longrightarrow\ dz_1=\bunsuu{\partial z}{\partial x}dx$で近似\vspace{-2mm}
\item
$y$についても同様 $dz_2=\bunsuu{\partial z}{\partial y}dy$
\end{itemize}

\newslide{全微分}

\vspace*{18mm}


\begin{layer}{120}{0}
\putnotew{96}{73}{\hyperlink{para14pg6}{\fbox{\Ctab{2.5mm}{\scalebox{1}{\scriptsize $\mathstrut||\!\lhd$}}}}}
\putnotew{101}{73}{\hyperlink{para15pg1}{\fbox{\Ctab{2.5mm}{\scalebox{1}{\scriptsize $\mathstrut|\!\lhd$}}}}}
\putnotew{108}{73}{\hyperlink{para15pg7}{\fbox{\Ctab{4.5mm}{\scalebox{1}{\scriptsize $\mathstrut\!\!\lhd\!\!$}}}}}
\putnotew{115}{73}{\hyperlink{para15pg8}{\fbox{\Ctab{4.5mm}{\scalebox{1}{\scriptsize $\mathstrut\!\rhd\!$}}}}}
\putnotew{120}{73}{\hyperlink{para15pg8}{\fbox{\Ctab{2.5mm}{\scalebox{1}{\scriptsize $\mathstrut \!\rhd\!\!|$}}}}}
\putnotew{125}{73}{\hyperlink{para16pg1}{\fbox{\Ctab{2.5mm}{\scalebox{1}{\scriptsize $\mathstrut \!\rhd\!\!||$}}}}}
\putnotee{126}{73}{\scriptsize\color{blue} 8/8}
\end{layer}

\slidepage
\vspace{3mm}

$x,\ y$の両方を$dx,\ dy$だけ変えたとき,$z$の変化量$dz$は?
\vspace{2mm}


\begin{layer}{120}{0}
\putnotes{20}{-3}{%%% /Users/takatoosetsuo/Dropbox/2018polytec/lecture/0702/presen/fig/bibunkeisuu/p001.tex 
%%% Generator=presen0702.cdy 
{\unitlength=1cm%
\begin{picture}%
(6,6)(-1,-1)%
\special{pn 8}%
%
\Large\bf\boldmath%
\small%
\special{pn 12}%
\special{pa  -195  -265}\special{pa  -136  -264}\special{pa   -76  -266}\special{pa   -17  -271}%
\special{pa    42  -279}\special{pa   101  -290}\special{pa   159  -305}\special{pa   218  -322}%
\special{pa   277  -343}\special{pa   335  -367}\special{pa   394  -394}\special{pa   482  -437}%
\special{pa   572  -483}\special{pa   662  -531}\special{pa   753  -582}\special{pa   844  -637}%
\special{pa   934  -696}\special{pa  1024  -760}\special{pa  1111  -829}\special{pa  1197  -903}%
\special{pa  1280  -984}\special{pa  1333 -1043}\special{pa  1386 -1107}\special{pa  1437 -1176}%
\special{pa  1488 -1251}\special{pa  1537 -1330}\special{pa  1585 -1415}\special{pa  1632 -1506}%
\special{pa  1677 -1601}\special{pa  1722 -1702}\special{pa  1765 -1808}%
\special{fp}%
\special{pn 8}%
\color[rgb]{1,0,0}%
\special{pa  -394   131}\special{pa  1969 -1444}%
\special{fp}%
\color[rgb]{0,0,0}%
\special{pa 394 -394}\special{pa 394 -358}\special{fp}\special{pa 394 -322}\special{pa 394 -286}\special{fp}%
\special{pa 394 -251}\special{pa 394 -215}\special{fp}\special{pa 394 -179}\special{pa 394 -143}\special{fp}%
\special{pa 394 -107}\special{pa 394 -72}\special{fp}\special{pa 394 -36}\special{pa 394 0}\special{fp}%
%
%
\special{pa 1280 -984}\special{pa 1280 -945}\special{fp}\special{pa 1280 -906}\special{pa 1280 -866}\special{fp}%
\special{pa 1280 -827}\special{pa 1280 -787}\special{fp}\special{pa 1280 -748}\special{pa 1280 -709}\special{fp}%
\special{pa 1280 -669}\special{pa 1280 -630}\special{fp}\special{pa 1280 -591}\special{pa 1280 -551}\special{fp}%
\special{pa 1280 -512}\special{pa 1280 -472}\special{fp}\special{pa 1280 -433}\special{pa 1280 -394}\special{fp}%
\special{pa 1280 -354}\special{pa 1280 -315}\special{fp}\special{pa 1280 -276}\special{pa 1280 -236}\special{fp}%
\special{pa 1280 -197}\special{pa 1280 -157}\special{fp}\special{pa 1280 -118}\special{pa 1280 -79}\special{fp}%
\special{pa 1280 -39}\special{pa 1280 0}\special{fp}%
%
\color[rgb]{0,0,1}%
\special{pa  -394   131}\special{pa  1969 -1444}%
\special{fp}%
\color[rgb]{0,0,0}%
\color[rgb]{0,0,1}%
\special{pa 1280 0}\special{pa 1280 -39}\special{fp}\special{pa 1280 -79}\special{pa 1280 -118}\special{fp}%
\special{pa 1280 -157}\special{pa 1280 -197}\special{fp}\special{pa 1280 -236}\special{pa 1280 -276}\special{fp}%
\special{pa 1280 -315}\special{pa 1280 -354}\special{fp}\special{pa 1280 -394}\special{pa 1280 -433}\special{fp}%
\special{pa 1280 -472}\special{pa 1280 -512}\special{fp}\special{pa 1280 -551}\special{pa 1280 -591}\special{fp}%
\special{pa 1280 -630}\special{pa 1280 -669}\special{fp}\special{pa 1280 -709}\special{pa 1280 -748}\special{fp}%
\special{pa 1280 -787}\special{pa 1280 -827}\special{fp}\special{pa 1280 -866}\special{pa 1280 -906}\special{fp}%
\special{pa 1280 -945}\special{pa 1280 -984}\special{fp}%
%
\color[rgb]{0,0,0}%
\settowidth{\Width}{P}\setlength{\Width}{-1\Width}%
\settoheight{\Height}{P}\settodepth{\Depth}{P}\setlength{\Height}{\Depth}%
\put(3.2000000,2.5500000){\hspace*{\Width}\raisebox{\Height}{P}}%
%
\settowidth{\Width}{$x$}\setlength{\Width}{-0.5\Width}%
\settoheight{\Height}{$x$}\settodepth{\Depth}{$x$}\setlength{\Height}{-\Height}%
\put(3.2500000,-0.1000000){\hspace*{\Width}\raisebox{\Height}{$x$}}%
%
\settowidth{\Width}{$a$}\setlength{\Width}{-0.5\Width}%
\settoheight{\Height}{$a$}\settodepth{\Depth}{$a$}\setlength{\Height}{-\Height}%
\put(1.0000000,-0.1000000){\hspace*{\Width}\raisebox{\Height}{$a$}}%
%
\settowidth{\Width}{A}\setlength{\Width}{0\Width}%
\settoheight{\Height}{A}\settodepth{\Depth}{A}\setlength{\Height}{-\Height}%
\put(1.0500000,0.9500000){\hspace*{\Width}\raisebox{\Height}{A}}%
%
\settowidth{\Width}{B}\setlength{\Width}{0\Width}%
\settoheight{\Height}{B}\settodepth{\Depth}{B}\setlength{\Height}{-\Height}%
\put(3.3000000,2.4500000){\hspace*{\Width}\raisebox{\Height}{B}}%
%
\special{pa  -394    -0}\special{pa  1969    -0}%
\special{fp}%
\special{pa     0   394}\special{pa     0 -1969}%
\special{fp}%
\settowidth{\Width}{$x$}\setlength{\Width}{0\Width}%
\settoheight{\Height}{$x$}\settodepth{\Depth}{$x$}\setlength{\Height}{-0.5\Height}\setlength{\Depth}{0.5\Depth}\addtolength{\Height}{\Depth}%
\put(5.0500000,0.0000000){\hspace*{\Width}\raisebox{\Height}{$x$}}%
%
\settowidth{\Width}{$y$}\setlength{\Width}{-0.5\Width}%
\settoheight{\Height}{$y$}\settodepth{\Depth}{$y$}\setlength{\Height}{\Depth}%
\put(0.0000000,5.0500000){\hspace*{\Width}\raisebox{\Height}{$y$}}%
%
\settowidth{\Width}{O}\setlength{\Width}{-1\Width}%
\settoheight{\Height}{O}\settodepth{\Depth}{O}\setlength{\Height}{-\Height}%
\put(-0.0500000,-0.0500000){\hspace*{\Width}\raisebox{\Height}{O}}%
%
\end{picture}}%}
\putnotes{85}{0}{%%% /Users/takatoosetsuo/Dropbox/2018polytec/lecture/0920/presen/fig/henbibun4.tex 
%%% Generator=zenbibun180920.cdy 
{\unitlength=20mm%
\begin{picture}%
(3.17,1.51)(0.65,2.5)%
\special{pn 8}%
%
\settowidth{\Width}{$$}\setlength{\Width}{-0.5\Width}%
\settoheight{\Height}{$$}\settodepth{\Depth}{$$}\setlength{\Height}{-0.5\Height}\setlength{\Depth}{0.5\Depth}\addtolength{\Height}{\Depth}%
\put(4.1900000,-0.7600000){\hspace*{\Width}\raisebox{\Height}{$$}}%
%
\settowidth{\Width}{$$}\setlength{\Width}{-0.5\Width}%
\settoheight{\Height}{$$}\settodepth{\Depth}{$$}\setlength{\Height}{-0.5\Height}\setlength{\Depth}{0.5\Depth}\addtolength{\Height}{\Depth}%
\put(2.9600000,1.0900000){\hspace*{\Width}\raisebox{\Height}{$$}}%
%
\settowidth{\Width}{$$}\setlength{\Width}{-0.5\Width}%
\settoheight{\Height}{$$}\settodepth{\Depth}{$$}\setlength{\Height}{-0.5\Height}\setlength{\Depth}{0.5\Depth}\addtolength{\Height}{\Depth}%
\put(0.0000000,4.9200000){\hspace*{\Width}\raisebox{\Height}{$$}}%
%
\special{pa  1951 -1969}\special{pa  1951 -2463}%
\special{fp}%
\special{pa  2742 -1969}\special{pa  2742 -2994}%
\special{fp}%
\special{pa  1613 -1969}\special{pa  1613 -2259}%
\special{fp}%
\special{pa  1613 -2320}\special{pa  1613 -2423}%
\special{fp}%
\special{pa  1613 -2484}\special{pa  1613 -2963}%
\special{fp}%
\special{pa   822 -1969}\special{pa   822 -2433}%
\special{fp}%
\special{pa   822 -2433}\special{pa  1951 -2228}\special{pa  2742 -2520}\special{pa  2226 -2614}%
\special{fp}%
\special{pa  2147 -2628}\special{pa  1613 -2725}\special{pa   822 -2433}%
\special{fp}%
\color[rgb]{0,0,1}%
\special{pa   822 -2433}\special{pa  1951 -2463}\special{pa  2742 -2994}\special{pa  1613 -2963}%
\special{pa   822 -2433}%
\special{fp}%
\color[rgb]{0,0,0}%
\end{picture}}%}
\putnotesw{23}{54}{\normalsize\color{red}$dx$}
\putnotenw{20}{51}{\normalsize\color{red}$dy$}
\putnotesw{76}{21}{\normalsize\color{red}$dx$}
\putnotenw{73}{15}{\normalsize\color{red}$dy$}
\putnotee{91}{20}{\normalsize\color{red}$dz_1$}
\putnotee{82.5}{8}{\normalsize\color{red}$dz_2$}
\putnotee{111}{10}{\normalsize\color{red}$dz$}
\end{layer}

\vspace*{35mm}

\hspace*{65mm}{\color{red}$dz=dz_1+dz_2$}\vspace{2mm}\\
\hspace*{65mm}{\fbox{\color{red}$dz=\bunsuu{\partial z}{\partial x}dx+\bunsuu{\partial z}{\partial y}dy$}}

\newslide{全微分}

\vspace*{18mm}


\begin{layer}{120}{0}
\putnotew{96}{73}{\hyperlink{para15pg8}{\fbox{\Ctab{2.5mm}{\scalebox{1}{\scriptsize $\mathstrut||\!\lhd$}}}}}
\putnotew{101}{73}{\hyperlink{para16pg1}{\fbox{\Ctab{2.5mm}{\scalebox{1}{\scriptsize $\mathstrut|\!\lhd$}}}}}
\putnotew{108}{73}{\hyperlink{para16pg4}{\fbox{\Ctab{4.5mm}{\scalebox{1}{\scriptsize $\mathstrut\!\!\lhd\!\!$}}}}}
\putnotew{115}{73}{\hyperlink{para16pg5}{\fbox{\Ctab{4.5mm}{\scalebox{1}{\scriptsize $\mathstrut\!\rhd\!$}}}}}
\putnotew{120}{73}{\hyperlink{para16pg5}{\fbox{\Ctab{2.5mm}{\scalebox{1}{\scriptsize $\mathstrut \!\rhd\!\!|$}}}}}
\putnotew{125}{73}{\hyperlink{para17pg1}{\fbox{\Ctab{2.5mm}{\scalebox{1}{\scriptsize $\mathstrut \!\rhd\!\!||$}}}}}
\putnotee{126}{73}{\scriptsize\color{blue} 5/5}
\end{layer}

\slidepage
\seteda{60}
\begin{itemize}
\item
[例]$z=x^2+5y^3$の全微分
\item
[解]$\dpar{z}{x}=2x$
\item
[]\ \ $\dpar{z}{y}=15y^2$
\item
[]\ \  $dz=2x\,dx+15y^2\,dy$
\item
[課題]\monban 次の関数の全微分を求めよ.\\
\eda{$z=2x+y$}
\eda{$z=xy$}
\end{itemize}
\label{pageend}\mbox{}

\end{document}

%%% タイトル presen22207
\documentclass[landscape,10pt]{jarticle}
\special{papersize=\the\paperwidth,\the\paperheight}
\usepackage{ketpic,ketlayer}
\usepackage{ketslide}
\usepackage{amsmath,amssymb}
\usepackage{bm,enumerate}
\usepackage[dvipdfmx]{graphicx}
\usepackage{color}
\definecolor{slidecolora}{cmyk}{0.98,0.13,0,0.43}
\definecolor{slidecolorb}{cmyk}{0.2,0,0,0}
\definecolor{slidecolorc}{cmyk}{0.2,0,0,0}
\definecolor{slidecolord}{cmyk}{0.2,0,0,0}
\definecolor{slidecolore}{cmyk}{0,0,0,0.5}
\definecolor{slidecolorf}{cmyk}{0,0,0,0.5}
\definecolor{slidecolori}{cmyk}{0.98,0.13,0,0.43}
\def\setthin#1{\def\thin{#1}}
\setthin{0}
\newcommand{\slidepage}[1][s]{%
\setcounter{ketpicctra}{18}%
\if#1m \setcounter{ketpicctra}{1}\fi
\hypersetup{linkcolor=black}%

\begin{layer}{118}{0}
\putnotee{122}{-\theketpicctra.05}{\small\thepage/\pageref{pageend}}
\end{layer}\hypersetup{linkcolor=blue}

}
\usepackage{emath}
\usepackage{pict2e}
\usepackage{ketlayermorewith2e}
\usepackage[dvipdfmx,colorlinks=true,linkcolor=blue,filecolor=blue]{hyperref}
\newcommand{\hiduke}{0926}
\newcommand{\hako}[2][1]{\fbox{\raisebox{#1mm}{\mbox{}}\raisebox{-#1mm}{\mbox{}}\,\phantom{#2}\,}}
\newcommand{\hakoa}[2][1]{\fbox{\raisebox{#1mm}{\mbox{}}\raisebox{-#1mm}{\mbox{}}\,#2\,}}
\newcommand{\hakom}[2][1]{\hako[#1]{$#2$}}
\newcommand{\hakoma}[2][1]{\hakoa[#1]{$#2$}}
\def\rad{\;\mathrm{rad}}
\def\deg#1{#1^{\circ}}
\newcommand{\sbunsuu}[2]{\scalebox{0.6}{$\bunsuu{#1}{#2}$}}
\def\pow{$\hspace{-1.5mm}^\hspace{-1mm}$}
\def\dlim{\displaystyle\lim}
\newcommand{\brd}[2][1]{\scalebox{#1}{\color{red}\fbox{\color{black}$#2$}}}
\newcommand\down[1][0.5zw]{\vspace{#1}\\}
\newcommand{\sfrac}[3][0.65]{\scalebox{#1}{$\frac{#2}{#3}$}}
\newcommand{\phn}[1]{\phantom{#1}}
\newcommand{\scb}[2][0.6]{\scalebox{#1}{#2}}
\newcommand{\dsum}{\displaystyle\sum}
\def\pow{$\hspace{-1.5mm}^\hspace{-1mm}$}
\def\dlim{\displaystyle\lim}
\def\dint{\displaystyle\int}

\setmargin{25}{145}{15}{100}

\ketslideinit

\pagestyle{empty}

\begin{document}

\begin{layer}{120}{0}
\putnotese{0}{0}{{\Large\bf
\color[cmyk]{1,1,0,0}

\begin{layer}{120}{0}
{\Huge \putnotes{60}{20}{微分積分の応用}}
\putnotes{60}{70}{2022.9.26}
\end{layer}

}
}
\end{layer}

\def\mainslidetitley{22}
\def\ketcletter{slidecolora}
\def\ketcbox{slidecolorb}
\def\ketdbox{slidecolorc}
\def\ketcframe{slidecolord}
\def\ketcshadow{slidecolore}
\def\ketdshadow{slidecolorf}
\def\slidetitlex{6}
\def\slidetitlesize{1.3}
\def\mketcletter{slidecolori}
\def\mketcbox{yellow}
\def\mketdbox{yellow}
\def\mketcframe{yellow}
\def\mslidetitlex{62}
\def\mslidetitlesize{2}

\color{black}
\Large\bf\boldmath
\addtocounter{page}{-1}

\def\MARU{}
\renewcommand{\MARU}[1]{{\ooalign{\hfil$#1$\/\hfil\crcr\raise.167ex\hbox{\mathhexbox20D}}}}
\renewcommand{\slidepage}[1][s]{%
\setcounter{ketpicctra}{18}%
\if#1m \setcounter{ketpicctra}{1}\fi
\hypersetup{linkcolor=black}%
\begin{layer}{118}{0}
\putnotee{115}{-\theketpicctra.05}{\small\hiduke-\thepage/\pageref{pageend}}
\end{layer}\hypersetup{linkcolor=blue}
}
\newcounter{ban}
\setcounter{ban}{1}
\newcommand{\monban}[1][\hiduke]{%
#1-\theban\ %
\addtocounter{ban}{1}%
}
\newcommand{\monbannoadd}[1][\hiduke]{%
#1-\theban\ %
}
\newcommand{\addban}{%
\addtocounter{ban}{1}%
}
\newcounter{edawidth}
\newcounter{edactr}
\newcommand{\seteda}[1]{% 20220708 modified
\setcounter{edawidth}{#1}
\setcounter{edactr}{1}
}
\newcommand{\eda}[2][\theedawidth]{%
\Ltab{#1 mm}{[\theedactr]\ #2}%
\addtocounter{edactr}{1}%
}
%%%%%%%%%%%%

%%%%%%%%%%%%%%%%%%%%

\mainslide{ 微分の応用}


\slidepage[m]
%%%%%%%%%%%%%

%%%%%%%%%%%%%%%%%%%%

\newslide{関数の増減}

\vspace*{18mm}


\begin{layer}{120}{0}
\putnotew{96}{73}{\hyperlink{para0pg0}{\fbox{\Ctab{2.5mm}{\scalebox{1}{\scriptsize $\mathstrut||\!\lhd$}}}}}
\putnotew{101}{73}{\hyperlink{para1pg1}{\fbox{\Ctab{2.5mm}{\scalebox{1}{\scriptsize $\mathstrut|\!\lhd$}}}}}
\putnotew{108}{73}{\hyperlink{para1pg1}{\fbox{\Ctab{4.5mm}{\scalebox{1}{\scriptsize $\mathstrut\!\!\lhd\!\!$}}}}}
\putnotew{115}{73}{\hyperlink{para1pg2}{\fbox{\Ctab{4.5mm}{\scalebox{1}{\scriptsize $\mathstrut\!\rhd\!$}}}}}
\putnotew{120}{73}{\hyperlink{para1pg2}{\fbox{\Ctab{2.5mm}{\scalebox{1}{\scriptsize $\mathstrut \!\rhd\!\!|$}}}}}
\putnotew{125}{73}{\hyperlink{para2pg1}{\fbox{\Ctab{2.5mm}{\scalebox{1}{\scriptsize $\mathstrut \!\rhd\!\!||$}}}}}
\putnotee{126}{73}{\scriptsize\color{blue} 2/2}
\end{layer}

\slidepage

\begin{layer}{120}{0}
\putnotese{70}{5}{%%% /Users/takatoosetsuo/polytech22.git/205-0926/presen/fig/zougen.tex 
%%% Generator=graph22207.cdy 
{\unitlength=6mm%
\begin{picture}%
(10.23,7.5)(-5.11,-4.24)%
\linethickness{0.008in}%%
\linethickness{0.016in}%%
\polyline(-4.85881,1.16837)(-4.75563,0.72408)(-4.65075,0.31260)(-4.54417,-0.06607)%
(-4.43589,-0.41194)(-4.32590,-0.72499)(-4.21420,-1.00524)(-4.10081,-1.25268)(-3.98571,-1.46730)%
(-3.86890,-1.64912)(-3.75040,-1.79814)(-3.56373,-1.98288)(-3.36334,-2.13996)(-3.15152,-2.27035)%
(-2.93056,-2.37501)(-2.70275,-2.45492)(-2.47039,-2.51106)(-2.23577,-2.54440)(-2.00118,-2.55592)%
(-1.76891,-2.54659)(-1.54126,-2.51739)(-1.30956,-2.45946)(-1.09097,-2.37109)(-0.88361,-2.25653)%
(-0.68561,-2.12003)(-0.49509,-1.96587)(-0.31017,-1.79828)(-0.12898,-1.62153)(0.05037,-1.43988)%
(0.22973,-1.25758)(0.41100,-1.07888)(0.59227,-0.87994)(0.74478,-0.66780)(0.87447,-0.45051)%
(0.98732,-0.23611)(1.08927,-0.03264)(1.18627,0.15187)(1.28430,0.30937)(1.38929,0.43183)%
(1.50721,0.51120)(1.64401,0.53944)(1.80627,0.50501)(1.93586,0.40983)(2.04114,0.26606)%
(2.13047,0.08587)(2.21223,-0.11857)(2.29477,-0.33511)(2.38644,-0.55157)(2.49563,-0.75580)%
(2.63067,-0.93562)(2.79995,-1.07888)(2.93827,-1.15452)(3.07892,-1.20789)(3.22064,-1.24082)%
(3.36215,-1.25514)(3.50219,-1.25270)(3.63947,-1.23532)(3.77274,-1.20485)(3.90072,-1.16311)%
(4.02213,-1.11195)(4.13571,-1.05319)(4.20781,-1.00500)(4.27844,-0.94300)(4.34760,-0.86720)%
(4.41531,-0.77760)(4.48154,-0.67419)(4.54632,-0.55698)(4.60963,-0.42596)(4.67147,-0.28115)%
(4.73185,-0.12253)(4.79077,0.04990)%
%
\linethickness{0.008in}%%
\polygon*(-4.80181,1.16837)(-4.80226,1.17552)(-4.80360,1.18255)(-4.80581,1.18936)%
(-4.80886,1.19583)(-4.81269,1.20188)(-4.81726,1.20739)(-4.82248,1.21229)(-4.82827,1.21650)%
(-4.83454,1.21995)(-4.84119,1.22258)(-4.84813,1.22436)(-4.85523,1.22526)(-4.86239,1.22526)%
(-4.86949,1.22436)(-4.87642,1.22258)(-4.88308,1.21995)(-4.88935,1.21650)(-4.89514,1.21229)%
(-4.90036,1.20739)(-4.90492,1.20188)(-4.90876,1.19583)(-4.91181,1.18936)(-4.91402,1.18255)%
(-4.91536,1.17552)(-4.91581,1.16837)(-4.91536,1.16123)(-4.91402,1.15420)(-4.91181,1.14739)%
(-4.90876,1.14091)(-4.90492,1.13487)(-4.90036,1.12935)(-4.89514,1.12445)(-4.88935,1.12025)%
(-4.88308,1.11680)(-4.87642,1.11416)(-4.86949,1.11238)(-4.86239,1.11148)(-4.85523,1.11148)%
(-4.84813,1.11238)(-4.84119,1.11416)(-4.83454,1.11680)(-4.82827,1.12025)(-4.82248,1.12445)%
(-4.81726,1.12935)(-4.81269,1.13487)(-4.80886,1.14091)(-4.80581,1.14739)(-4.80360,1.15420)%
(-4.80226,1.16123)(-4.80181,1.16837)(-4.80181,1.16837)\polyline(-4.80181,1.16837)(-4.80226,1.17552)(-4.80360,1.18255)(-4.80581,1.18936)%
(-4.80886,1.19583)(-4.81269,1.20188)(-4.81726,1.20739)(-4.82248,1.21229)(-4.82827,1.21650)%
(-4.83454,1.21995)(-4.84119,1.22258)(-4.84813,1.22436)(-4.85523,1.22526)(-4.86239,1.22526)%
(-4.86949,1.22436)(-4.87642,1.22258)(-4.88308,1.21995)(-4.88935,1.21650)(-4.89514,1.21229)%
(-4.90036,1.20739)(-4.90492,1.20188)(-4.90876,1.19583)(-4.91181,1.18936)(-4.91402,1.18255)%
(-4.91536,1.17552)(-4.91581,1.16837)(-4.91536,1.16123)(-4.91402,1.15420)(-4.91181,1.14739)%
(-4.90876,1.14091)(-4.90492,1.13487)(-4.90036,1.12935)(-4.89514,1.12445)(-4.88935,1.12025)%
(-4.88308,1.11680)(-4.87642,1.11416)(-4.86949,1.11238)(-4.86239,1.11148)(-4.85523,1.11148)%
(-4.84813,1.11238)(-4.84119,1.11416)(-4.83454,1.11680)(-4.82827,1.12025)(-4.82248,1.12445)%
(-4.81726,1.12935)(-4.81269,1.13487)(-4.80886,1.14091)(-4.80581,1.14739)(-4.80360,1.15420)%
(-4.80226,1.16123)(-4.80181,1.16837)%
%
\polygon*(-3.69340,-1.79814)(-3.69385,-1.79099)(-3.69519,-1.78396)(-3.69740,-1.77715)%
(-3.70045,-1.77068)(-3.70428,-1.76463)(-3.70885,-1.75912)(-3.71406,-1.75422)(-3.71985,-1.75001)%
(-3.72613,-1.74656)(-3.73278,-1.74392)(-3.73972,-1.74214)(-3.74682,-1.74125)(-3.75398,-1.74125)%
(-3.76108,-1.74214)(-3.76801,-1.74392)(-3.77467,-1.74656)(-3.78094,-1.75001)(-3.78673,-1.75422)%
(-3.79195,-1.75912)(-3.79651,-1.76463)(-3.80035,-1.77068)(-3.80339,-1.77715)(-3.80561,-1.78396)%
(-3.80695,-1.79099)(-3.80740,-1.79814)(-3.80695,-1.80528)(-3.80561,-1.81231)(-3.80339,-1.81912)%
(-3.80035,-1.82560)(-3.79651,-1.83164)(-3.79195,-1.83715)(-3.78673,-1.84205)(-3.78094,-1.84626)%
(-3.77467,-1.84971)(-3.76801,-1.85235)(-3.76108,-1.85413)(-3.75398,-1.85502)(-3.74682,-1.85502)%
(-3.73972,-1.85413)(-3.73278,-1.85235)(-3.72613,-1.84971)(-3.71985,-1.84626)(-3.71406,-1.84205)%
(-3.70885,-1.83715)(-3.70428,-1.83164)(-3.70045,-1.82560)(-3.69740,-1.81912)(-3.69519,-1.81231)%
(-3.69385,-1.80528)(-3.69340,-1.79814)(-3.69340,-1.79814)\polyline(-3.69340,-1.79814)(-3.69385,-1.79099)(-3.69519,-1.78396)(-3.69740,-1.77715)%
(-3.70045,-1.77068)(-3.70428,-1.76463)(-3.70885,-1.75912)(-3.71406,-1.75422)(-3.71985,-1.75001)%
(-3.72613,-1.74656)(-3.73278,-1.74392)(-3.73972,-1.74214)(-3.74682,-1.74125)(-3.75398,-1.74125)%
(-3.76108,-1.74214)(-3.76801,-1.74392)(-3.77467,-1.74656)(-3.78094,-1.75001)(-3.78673,-1.75422)%
(-3.79195,-1.75912)(-3.79651,-1.76463)(-3.80035,-1.77068)(-3.80339,-1.77715)(-3.80561,-1.78396)%
(-3.80695,-1.79099)(-3.80740,-1.79814)(-3.80695,-1.80528)(-3.80561,-1.81231)(-3.80339,-1.81912)%
(-3.80035,-1.82560)(-3.79651,-1.83164)(-3.79195,-1.83715)(-3.78673,-1.84205)(-3.78094,-1.84626)%
(-3.77467,-1.84971)(-3.76801,-1.85235)(-3.76108,-1.85413)(-3.75398,-1.85502)(-3.74682,-1.85502)%
(-3.73972,-1.85413)(-3.73278,-1.85235)(-3.72613,-1.84971)(-3.71985,-1.84626)(-3.71406,-1.84205)%
(-3.70885,-1.83715)(-3.70428,-1.83164)(-3.70045,-1.82560)(-3.69740,-1.81912)(-3.69519,-1.81231)%
(-3.69385,-1.80528)(-3.69340,-1.79814)%
%
\polygon*(-1.48426,-2.51739)(-1.48471,-2.51025)(-1.48605,-2.50321)(-1.48826,-2.49641)%
(-1.49131,-2.48993)(-1.49514,-2.48389)(-1.49971,-2.47837)(-1.50493,-2.47347)(-1.51072,-2.46926)%
(-1.51699,-2.46581)(-1.52364,-2.46318)(-1.53058,-2.46140)(-1.53768,-2.46050)(-1.54484,-2.46050)%
(-1.55194,-2.46140)(-1.55887,-2.46318)(-1.56553,-2.46581)(-1.57180,-2.46926)(-1.57759,-2.47347)%
(-1.58281,-2.47837)(-1.58737,-2.48389)(-1.59121,-2.48993)(-1.59426,-2.49641)(-1.59647,-2.50321)%
(-1.59781,-2.51025)(-1.59826,-2.51739)(-1.59781,-2.52453)(-1.59647,-2.53156)(-1.59426,-2.53837)%
(-1.59121,-2.54485)(-1.58737,-2.55089)(-1.58281,-2.55641)(-1.57759,-2.56131)(-1.57180,-2.56552)%
(-1.56553,-2.56896)(-1.55887,-2.57160)(-1.55194,-2.57338)(-1.54484,-2.57428)(-1.53768,-2.57428)%
(-1.53058,-2.57338)(-1.52364,-2.57160)(-1.51699,-2.56896)(-1.51072,-2.56552)(-1.50493,-2.56131)%
(-1.49971,-2.55641)(-1.49514,-2.55089)(-1.49131,-2.54485)(-1.48826,-2.53837)(-1.48605,-2.53156)%
(-1.48471,-2.52453)(-1.48426,-2.51739)(-1.48426,-2.51739)\polyline(-1.48426,-2.51739)(-1.48471,-2.51025)(-1.48605,-2.50321)(-1.48826,-2.49641)%
(-1.49131,-2.48993)(-1.49514,-2.48389)(-1.49971,-2.47837)(-1.50493,-2.47347)(-1.51072,-2.46926)%
(-1.51699,-2.46581)(-1.52364,-2.46318)(-1.53058,-2.46140)(-1.53768,-2.46050)(-1.54484,-2.46050)%
(-1.55194,-2.46140)(-1.55887,-2.46318)(-1.56553,-2.46581)(-1.57180,-2.46926)(-1.57759,-2.47347)%
(-1.58281,-2.47837)(-1.58737,-2.48389)(-1.59121,-2.48993)(-1.59426,-2.49641)(-1.59647,-2.50321)%
(-1.59781,-2.51025)(-1.59826,-2.51739)(-1.59781,-2.52453)(-1.59647,-2.53156)(-1.59426,-2.53837)%
(-1.59121,-2.54485)(-1.58737,-2.55089)(-1.58281,-2.55641)(-1.57759,-2.56131)(-1.57180,-2.56552)%
(-1.56553,-2.56896)(-1.55887,-2.57160)(-1.55194,-2.57338)(-1.54484,-2.57428)(-1.53768,-2.57428)%
(-1.53058,-2.57338)(-1.52364,-2.57160)(-1.51699,-2.56896)(-1.51072,-2.56552)(-1.50493,-2.56131)%
(-1.49971,-2.55641)(-1.49514,-2.55089)(-1.49131,-2.54485)(-1.48826,-2.53837)(-1.48605,-2.53156)%
(-1.48471,-2.52453)(-1.48426,-2.51739)%
%
\polygon*(0.46800,-1.07888)(0.46755,-1.07174)(0.46621,-1.06471)(0.46400,-1.05790)%
(0.46095,-1.05142)(0.45712,-1.04538)(0.45255,-1.03986)(0.44734,-1.03496)(0.44154,-1.03075)%
(0.43527,-1.02731)(0.42862,-1.02467)(0.42168,-1.02289)(0.41458,-1.02199)(0.40742,-1.02199)%
(0.40032,-1.02289)(0.39339,-1.02467)(0.38673,-1.02731)(0.38046,-1.03075)(0.37467,-1.03496)%
(0.36945,-1.03986)(0.36489,-1.04538)(0.36105,-1.05142)(0.35801,-1.05790)(0.35579,-1.06471)%
(0.35445,-1.07174)(0.35400,-1.07888)(0.35445,-1.08603)(0.35579,-1.09306)(0.35801,-1.09986)%
(0.36105,-1.10634)(0.36489,-1.11238)(0.36945,-1.11790)(0.37467,-1.12280)(0.38046,-1.12701)%
(0.38673,-1.13046)(0.39339,-1.13309)(0.40032,-1.13487)(0.40742,-1.13577)(0.41458,-1.13577)%
(0.42168,-1.13487)(0.42862,-1.13309)(0.43527,-1.13046)(0.44154,-1.12701)(0.44734,-1.12280)%
(0.45255,-1.11790)(0.45712,-1.11238)(0.46095,-1.10634)(0.46400,-1.09986)(0.46621,-1.09306)%
(0.46755,-1.08603)(0.46800,-1.07888)(0.46800,-1.07888)\polyline(0.46800,-1.07888)(0.46755,-1.07174)(0.46621,-1.06471)(0.46400,-1.05790)%
(0.46095,-1.05142)(0.45712,-1.04538)(0.45255,-1.03986)(0.44734,-1.03496)(0.44154,-1.03075)%
(0.43527,-1.02731)(0.42862,-1.02467)(0.42168,-1.02289)(0.41458,-1.02199)(0.40742,-1.02199)%
(0.40032,-1.02289)(0.39339,-1.02467)(0.38673,-1.02731)(0.38046,-1.03075)(0.37467,-1.03496)%
(0.36945,-1.03986)(0.36489,-1.04538)(0.36105,-1.05142)(0.35801,-1.05790)(0.35579,-1.06471)%
(0.35445,-1.07174)(0.35400,-1.07888)(0.35445,-1.08603)(0.35579,-1.09306)(0.35801,-1.09986)%
(0.36105,-1.10634)(0.36489,-1.11238)(0.36945,-1.11790)(0.37467,-1.12280)(0.38046,-1.12701)%
(0.38673,-1.13046)(0.39339,-1.13309)(0.40032,-1.13487)(0.40742,-1.13577)(0.41458,-1.13577)%
(0.42168,-1.13487)(0.42862,-1.13309)(0.43527,-1.13046)(0.44154,-1.12701)(0.44734,-1.12280)%
(0.45255,-1.11790)(0.45712,-1.11238)(0.46095,-1.10634)(0.46400,-1.09986)(0.46621,-1.09306)%
(0.46755,-1.08603)(0.46800,-1.07888)%
%
\polygon*(1.70101,0.53944)(1.70056,0.54658)(1.69922,0.55362)(1.69701,0.56042)(1.69396,0.56690)%
(1.69012,0.57294)(1.68556,0.57846)(1.68034,0.58336)(1.67455,0.58757)(1.66828,0.59102)%
(1.66162,0.59365)(1.65469,0.59543)(1.64759,0.59633)(1.64043,0.59633)(1.63333,0.59543)%
(1.62640,0.59365)(1.61974,0.59102)(1.61347,0.58757)(1.60768,0.58336)(1.60246,0.57846)%
(1.59790,0.57294)(1.59406,0.56690)(1.59101,0.56042)(1.58880,0.55362)(1.58746,0.54658)%
(1.58701,0.53944)(1.58746,0.53230)(1.58880,0.52527)(1.59101,0.51846)(1.59406,0.51198)%
(1.59790,0.50594)(1.60246,0.50042)(1.60768,0.49552)(1.61347,0.49131)(1.61974,0.48787)%
(1.62640,0.48523)(1.63333,0.48345)(1.64043,0.48255)(1.64759,0.48255)(1.65469,0.48345)%
(1.66162,0.48523)(1.66828,0.48787)(1.67455,0.49131)(1.68034,0.49552)(1.68556,0.50042)%
(1.69012,0.50594)(1.69396,0.51198)(1.69701,0.51846)(1.69922,0.52527)(1.70056,0.53230)%
(1.70101,0.53944)(1.70101,0.53944)\polyline(1.70101,0.53944)(1.70056,0.54658)(1.69922,0.55362)(1.69701,0.56042)(1.69396,0.56690)%
(1.69012,0.57294)(1.68556,0.57846)(1.68034,0.58336)(1.67455,0.58757)(1.66828,0.59102)%
(1.66162,0.59365)(1.65469,0.59543)(1.64759,0.59633)(1.64043,0.59633)(1.63333,0.59543)%
(1.62640,0.59365)(1.61974,0.59102)(1.61347,0.58757)(1.60768,0.58336)(1.60246,0.57846)%
(1.59790,0.57294)(1.59406,0.56690)(1.59101,0.56042)(1.58880,0.55362)(1.58746,0.54658)%
(1.58701,0.53944)(1.58746,0.53230)(1.58880,0.52527)(1.59101,0.51846)(1.59406,0.51198)%
(1.59790,0.50594)(1.60246,0.50042)(1.60768,0.49552)(1.61347,0.49131)(1.61974,0.48787)%
(1.62640,0.48523)(1.63333,0.48345)(1.64043,0.48255)(1.64759,0.48255)(1.65469,0.48345)%
(1.66162,0.48523)(1.66828,0.48787)(1.67455,0.49131)(1.68034,0.49552)(1.68556,0.50042)%
(1.69012,0.50594)(1.69396,0.51198)(1.69701,0.51846)(1.69922,0.52527)(1.70056,0.53230)%
(1.70101,0.53944)%
%
\polygon*(2.85695,-1.07888)(2.85650,-1.07174)(2.85516,-1.06471)(2.85295,-1.05790)%
(2.84990,-1.05142)(2.84607,-1.04538)(2.84150,-1.03986)(2.83629,-1.03496)(2.83050,-1.03075)%
(2.82422,-1.02731)(2.81757,-1.02467)(2.81063,-1.02289)(2.80353,-1.02199)(2.79637,-1.02199)%
(2.78927,-1.02289)(2.78234,-1.02467)(2.77568,-1.02731)(2.76941,-1.03075)(2.76362,-1.03496)%
(2.75840,-1.03986)(2.75384,-1.04538)(2.75000,-1.05142)(2.74696,-1.05790)(2.74474,-1.06471)%
(2.74340,-1.07174)(2.74295,-1.07888)(2.74340,-1.08603)(2.74474,-1.09306)(2.74696,-1.09986)%
(2.75000,-1.10634)(2.75384,-1.11238)(2.75840,-1.11790)(2.76362,-1.12280)(2.76941,-1.12701)%
(2.77568,-1.13046)(2.78234,-1.13309)(2.78927,-1.13487)(2.79637,-1.13577)(2.80353,-1.13577)%
(2.81063,-1.13487)(2.81757,-1.13309)(2.82422,-1.13046)(2.83050,-1.12701)(2.83629,-1.12280)%
(2.84150,-1.11790)(2.84607,-1.11238)(2.84990,-1.10634)(2.85295,-1.09986)(2.85516,-1.09306)%
(2.85650,-1.08603)(2.85695,-1.07888)(2.85695,-1.07888)\polyline(2.85695,-1.07888)(2.85650,-1.07174)(2.85516,-1.06471)(2.85295,-1.05790)%
(2.84990,-1.05142)(2.84607,-1.04538)(2.84150,-1.03986)(2.83629,-1.03496)(2.83050,-1.03075)%
(2.82422,-1.02731)(2.81757,-1.02467)(2.81063,-1.02289)(2.80353,-1.02199)(2.79637,-1.02199)%
(2.78927,-1.02289)(2.78234,-1.02467)(2.77568,-1.02731)(2.76941,-1.03075)(2.76362,-1.03496)%
(2.75840,-1.03986)(2.75384,-1.04538)(2.75000,-1.05142)(2.74696,-1.05790)(2.74474,-1.06471)%
(2.74340,-1.07174)(2.74295,-1.07888)(2.74340,-1.08603)(2.74474,-1.09306)(2.74696,-1.09986)%
(2.75000,-1.10634)(2.75384,-1.11238)(2.75840,-1.11790)(2.76362,-1.12280)(2.76941,-1.12701)%
(2.77568,-1.13046)(2.78234,-1.13309)(2.78927,-1.13487)(2.79637,-1.13577)(2.80353,-1.13577)%
(2.81063,-1.13487)(2.81757,-1.13309)(2.82422,-1.13046)(2.83050,-1.12701)(2.83629,-1.12280)%
(2.84150,-1.11790)(2.84607,-1.11238)(2.84990,-1.10634)(2.85295,-1.09986)(2.85516,-1.09306)%
(2.85650,-1.08603)(2.85695,-1.07888)%
%
\polygon*(4.19271,-1.05319)(4.19226,-1.04605)(4.19092,-1.03902)(4.18871,-1.03221)%
(4.18566,-1.02573)(4.18182,-1.01969)(4.17726,-1.01417)(4.17204,-1.00927)(4.16625,-1.00507)%
(4.15998,-1.00162)(4.15332,-0.99898)(4.14639,-0.99720)(4.13929,-0.99631)(4.13213,-0.99631)%
(4.12503,-0.99720)(4.11810,-0.99898)(4.11144,-1.00162)(4.10517,-1.00507)(4.09938,-1.00927)%
(4.09416,-1.01417)(4.08960,-1.01969)(4.08576,-1.02573)(4.08271,-1.03221)(4.08050,-1.03902)%
(4.07916,-1.04605)(4.07871,-1.05319)(4.07916,-1.06034)(4.08050,-1.06737)(4.08271,-1.07418)%
(4.08576,-1.08065)(4.08960,-1.08670)(4.09416,-1.09221)(4.09938,-1.09711)(4.10517,-1.10132)%
(4.11144,-1.10477)(4.11810,-1.10740)(4.12503,-1.10918)(4.13213,-1.11008)(4.13929,-1.11008)%
(4.14639,-1.10918)(4.15332,-1.10740)(4.15998,-1.10477)(4.16625,-1.10132)(4.17204,-1.09711)%
(4.17726,-1.09221)(4.18182,-1.08670)(4.18566,-1.08065)(4.18871,-1.07418)(4.19092,-1.06737)%
(4.19226,-1.06034)(4.19271,-1.05319)(4.19271,-1.05319)\polyline(4.19271,-1.05319)(4.19226,-1.04605)(4.19092,-1.03902)(4.18871,-1.03221)%
(4.18566,-1.02573)(4.18182,-1.01969)(4.17726,-1.01417)(4.17204,-1.00927)(4.16625,-1.00507)%
(4.15998,-1.00162)(4.15332,-0.99898)(4.14639,-0.99720)(4.13929,-0.99631)(4.13213,-0.99631)%
(4.12503,-0.99720)(4.11810,-0.99898)(4.11144,-1.00162)(4.10517,-1.00507)(4.09938,-1.00927)%
(4.09416,-1.01417)(4.08960,-1.01969)(4.08576,-1.02573)(4.08271,-1.03221)(4.08050,-1.03902)%
(4.07916,-1.04605)(4.07871,-1.05319)(4.07916,-1.06034)(4.08050,-1.06737)(4.08271,-1.07418)%
(4.08576,-1.08065)(4.08960,-1.08670)(4.09416,-1.09221)(4.09938,-1.09711)(4.10517,-1.10132)%
(4.11144,-1.10477)(4.11810,-1.10740)(4.12503,-1.10918)(4.13213,-1.11008)(4.13929,-1.11008)%
(4.14639,-1.10918)(4.15332,-1.10740)(4.15998,-1.10477)(4.16625,-1.10132)(4.17204,-1.09711)%
(4.17726,-1.09221)(4.18182,-1.08670)(4.18566,-1.08065)(4.18871,-1.07418)(4.19092,-1.06737)%
(4.19226,-1.06034)(4.19271,-1.05319)%
%
\polygon*(4.84777,0.04990)(4.84732,0.05704)(4.84598,0.06407)(4.84376,0.07088)(4.84072,0.07736)%
(4.83688,0.08340)(4.83232,0.08892)(4.82710,0.09382)(4.82131,0.09803)(4.81504,0.10147)%
(4.80838,0.10411)(4.80145,0.10589)(4.79435,0.10679)(4.78719,0.10679)(4.78009,0.10589)%
(4.77315,0.10411)(4.76650,0.10147)(4.76023,0.09803)(4.75443,0.09382)(4.74922,0.08892)%
(4.74465,0.08340)(4.74082,0.07736)(4.73777,0.07088)(4.73556,0.06407)(4.73422,0.05704)%
(4.73377,0.04990)(4.73422,0.04275)(4.73556,0.03572)(4.73777,0.02892)(4.74082,0.02244)%
(4.74465,0.01639)(4.74922,0.01088)(4.75443,0.00598)(4.76023,0.00177)(4.76650,-0.00168)%
(4.77315,-0.00431)(4.78009,-0.00609)(4.78719,-0.00699)(4.79435,-0.00699)(4.80145,-0.00609)%
(4.80838,-0.00431)(4.81504,-0.00168)(4.82131,0.00177)(4.82710,0.00598)(4.83232,0.01088)%
(4.83688,0.01639)(4.84072,0.02244)(4.84376,0.02892)(4.84598,0.03572)(4.84732,0.04275)%
(4.84777,0.04990)(4.84777,0.04990)\polyline(4.84777,0.04990)(4.84732,0.05704)(4.84598,0.06407)(4.84376,0.07088)(4.84072,0.07736)%
(4.83688,0.08340)(4.83232,0.08892)(4.82710,0.09382)(4.82131,0.09803)(4.81504,0.10147)%
(4.80838,0.10411)(4.80145,0.10589)(4.79435,0.10679)(4.78719,0.10679)(4.78009,0.10589)%
(4.77315,0.10411)(4.76650,0.10147)(4.76023,0.09803)(4.75443,0.09382)(4.74922,0.08892)%
(4.74465,0.08340)(4.74082,0.07736)(4.73777,0.07088)(4.73556,0.06407)(4.73422,0.05704)%
(4.73377,0.04990)(4.73422,0.04275)(4.73556,0.03572)(4.73777,0.02892)(4.74082,0.02244)%
(4.74465,0.01639)(4.74922,0.01088)(4.75443,0.00598)(4.76023,0.00177)(4.76650,-0.00168)%
(4.77315,-0.00431)(4.78009,-0.00609)(4.78719,-0.00699)(4.79435,-0.00699)(4.80145,-0.00609)%
(4.80838,-0.00431)(4.81504,-0.00168)(4.82131,0.00177)(4.82710,0.00598)(4.83232,0.01088)%
(4.83688,0.01639)(4.84072,0.02244)(4.84376,0.02892)(4.84598,0.03572)(4.84732,0.04275)%
(4.84777,0.04990)%
%
\polyline(-5.10574,0.00000)(5.11794,0.00000)%
%
\polyline(0.00000,-4.23973)(0.00000,3.26000)%
%
\settowidth{\Width}{$x$}\setlength{\Width}{0\Width}%
\settoheight{\Height}{$x$}\settodepth{\Depth}{$x$}\setlength{\Height}{-0.5\Height}\setlength{\Depth}{0.5\Depth}\addtolength{\Height}{\Depth}%
\put(5.2033333,0.0000000){\hspace*{\Width}\raisebox{\Height}{$x$}}%
%
\settowidth{\Width}{$y$}\setlength{\Width}{-0.5\Width}%
\settoheight{\Height}{$y$}\settodepth{\Depth}{$y$}\setlength{\Height}{\Depth}%
\put(0.0000000,3.3433333){\hspace*{\Width}\raisebox{\Height}{$y$}}%
%
\settowidth{\Width}{O}\setlength{\Width}{-1\Width}%
\settoheight{\Height}{O}\settodepth{\Depth}{O}\setlength{\Height}{-\Height}%
\put(-0.0833333,-0.0833333){\hspace*{\Width}\raisebox{\Height}{O}}%
%
\end{picture}}%}
\putnotese{70}{5}{%%% /Users/takatoosetsuo/polytech22.git/205-0926/presen/fig/zougen2.tex 
%%% Generator=graph22207.cdy 
{\unitlength=6mm%
\begin{picture}%
(10.23,7.5)(-5.11,-4.24)%
\linethickness{0.008in}%%
\linethickness{0.016in}%%
\polyline(-4.85881,1.16837)(-4.75563,0.72408)(-4.65075,0.31260)(-4.54417,-0.06607)%
(-4.43589,-0.41194)(-4.32590,-0.72499)(-4.21420,-1.00524)(-4.10081,-1.25268)(-3.98571,-1.46730)%
(-3.86890,-1.64912)(-3.75040,-1.79814)(-3.56373,-1.98288)(-3.36334,-2.13996)(-3.15152,-2.27035)%
(-2.93056,-2.37501)(-2.70275,-2.45492)(-2.47039,-2.51106)(-2.23577,-2.54440)(-2.00118,-2.55592)%
(-1.76891,-2.54659)(-1.54126,-2.51739)(-1.30956,-2.45946)(-1.09097,-2.37109)(-0.88361,-2.25653)%
(-0.68561,-2.12003)(-0.49509,-1.96587)(-0.31017,-1.79828)(-0.12898,-1.62153)(0.05037,-1.43988)%
(0.22973,-1.25758)(0.41100,-1.07888)(0.59227,-0.87994)(0.74478,-0.66780)(0.87447,-0.45051)%
(0.98732,-0.23611)(1.08927,-0.03264)(1.18627,0.15187)(1.28430,0.30937)(1.38929,0.43183)%
(1.50721,0.51120)(1.64401,0.53944)(1.80627,0.50501)(1.93586,0.40983)(2.04114,0.26606)%
(2.13047,0.08587)(2.21223,-0.11857)(2.29477,-0.33511)(2.38644,-0.55157)(2.49563,-0.75580)%
(2.63067,-0.93562)(2.79995,-1.07888)(2.93827,-1.15452)(3.07892,-1.20789)(3.22064,-1.24082)%
(3.36215,-1.25514)(3.50219,-1.25270)(3.63947,-1.23532)(3.77274,-1.20485)(3.90072,-1.16311)%
(4.02213,-1.11195)(4.13571,-1.05319)(4.20781,-1.00500)(4.27844,-0.94300)(4.34760,-0.86720)%
(4.41531,-0.77760)(4.48154,-0.67419)(4.54632,-0.55698)(4.60963,-0.42596)(4.67147,-0.28115)%
(4.73185,-0.12253)(4.79077,0.04990)%
%
\linethickness{0.008in}%%
\polygon*(-4.80181,1.16837)(-4.80226,1.17552)(-4.80360,1.18255)(-4.80581,1.18936)%
(-4.80886,1.19583)(-4.81269,1.20188)(-4.81726,1.20739)(-4.82248,1.21229)(-4.82827,1.21650)%
(-4.83454,1.21995)(-4.84119,1.22258)(-4.84813,1.22436)(-4.85523,1.22526)(-4.86239,1.22526)%
(-4.86949,1.22436)(-4.87642,1.22258)(-4.88308,1.21995)(-4.88935,1.21650)(-4.89514,1.21229)%
(-4.90036,1.20739)(-4.90492,1.20188)(-4.90876,1.19583)(-4.91181,1.18936)(-4.91402,1.18255)%
(-4.91536,1.17552)(-4.91581,1.16837)(-4.91536,1.16123)(-4.91402,1.15420)(-4.91181,1.14739)%
(-4.90876,1.14091)(-4.90492,1.13487)(-4.90036,1.12935)(-4.89514,1.12445)(-4.88935,1.12025)%
(-4.88308,1.11680)(-4.87642,1.11416)(-4.86949,1.11238)(-4.86239,1.11148)(-4.85523,1.11148)%
(-4.84813,1.11238)(-4.84119,1.11416)(-4.83454,1.11680)(-4.82827,1.12025)(-4.82248,1.12445)%
(-4.81726,1.12935)(-4.81269,1.13487)(-4.80886,1.14091)(-4.80581,1.14739)(-4.80360,1.15420)%
(-4.80226,1.16123)(-4.80181,1.16837)(-4.80181,1.16837)\polyline(-4.80181,1.16837)(-4.80226,1.17552)(-4.80360,1.18255)(-4.80581,1.18936)%
(-4.80886,1.19583)(-4.81269,1.20188)(-4.81726,1.20739)(-4.82248,1.21229)(-4.82827,1.21650)%
(-4.83454,1.21995)(-4.84119,1.22258)(-4.84813,1.22436)(-4.85523,1.22526)(-4.86239,1.22526)%
(-4.86949,1.22436)(-4.87642,1.22258)(-4.88308,1.21995)(-4.88935,1.21650)(-4.89514,1.21229)%
(-4.90036,1.20739)(-4.90492,1.20188)(-4.90876,1.19583)(-4.91181,1.18936)(-4.91402,1.18255)%
(-4.91536,1.17552)(-4.91581,1.16837)(-4.91536,1.16123)(-4.91402,1.15420)(-4.91181,1.14739)%
(-4.90876,1.14091)(-4.90492,1.13487)(-4.90036,1.12935)(-4.89514,1.12445)(-4.88935,1.12025)%
(-4.88308,1.11680)(-4.87642,1.11416)(-4.86949,1.11238)(-4.86239,1.11148)(-4.85523,1.11148)%
(-4.84813,1.11238)(-4.84119,1.11416)(-4.83454,1.11680)(-4.82827,1.12025)(-4.82248,1.12445)%
(-4.81726,1.12935)(-4.81269,1.13487)(-4.80886,1.14091)(-4.80581,1.14739)(-4.80360,1.15420)%
(-4.80226,1.16123)(-4.80181,1.16837)%
%
\polygon*(-3.69340,-1.79814)(-3.69385,-1.79099)(-3.69519,-1.78396)(-3.69740,-1.77715)%
(-3.70045,-1.77068)(-3.70428,-1.76463)(-3.70885,-1.75912)(-3.71406,-1.75422)(-3.71985,-1.75001)%
(-3.72613,-1.74656)(-3.73278,-1.74392)(-3.73972,-1.74214)(-3.74682,-1.74125)(-3.75398,-1.74125)%
(-3.76108,-1.74214)(-3.76801,-1.74392)(-3.77467,-1.74656)(-3.78094,-1.75001)(-3.78673,-1.75422)%
(-3.79195,-1.75912)(-3.79651,-1.76463)(-3.80035,-1.77068)(-3.80339,-1.77715)(-3.80561,-1.78396)%
(-3.80695,-1.79099)(-3.80740,-1.79814)(-3.80695,-1.80528)(-3.80561,-1.81231)(-3.80339,-1.81912)%
(-3.80035,-1.82560)(-3.79651,-1.83164)(-3.79195,-1.83715)(-3.78673,-1.84205)(-3.78094,-1.84626)%
(-3.77467,-1.84971)(-3.76801,-1.85235)(-3.76108,-1.85413)(-3.75398,-1.85502)(-3.74682,-1.85502)%
(-3.73972,-1.85413)(-3.73278,-1.85235)(-3.72613,-1.84971)(-3.71985,-1.84626)(-3.71406,-1.84205)%
(-3.70885,-1.83715)(-3.70428,-1.83164)(-3.70045,-1.82560)(-3.69740,-1.81912)(-3.69519,-1.81231)%
(-3.69385,-1.80528)(-3.69340,-1.79814)(-3.69340,-1.79814)\polyline(-3.69340,-1.79814)(-3.69385,-1.79099)(-3.69519,-1.78396)(-3.69740,-1.77715)%
(-3.70045,-1.77068)(-3.70428,-1.76463)(-3.70885,-1.75912)(-3.71406,-1.75422)(-3.71985,-1.75001)%
(-3.72613,-1.74656)(-3.73278,-1.74392)(-3.73972,-1.74214)(-3.74682,-1.74125)(-3.75398,-1.74125)%
(-3.76108,-1.74214)(-3.76801,-1.74392)(-3.77467,-1.74656)(-3.78094,-1.75001)(-3.78673,-1.75422)%
(-3.79195,-1.75912)(-3.79651,-1.76463)(-3.80035,-1.77068)(-3.80339,-1.77715)(-3.80561,-1.78396)%
(-3.80695,-1.79099)(-3.80740,-1.79814)(-3.80695,-1.80528)(-3.80561,-1.81231)(-3.80339,-1.81912)%
(-3.80035,-1.82560)(-3.79651,-1.83164)(-3.79195,-1.83715)(-3.78673,-1.84205)(-3.78094,-1.84626)%
(-3.77467,-1.84971)(-3.76801,-1.85235)(-3.76108,-1.85413)(-3.75398,-1.85502)(-3.74682,-1.85502)%
(-3.73972,-1.85413)(-3.73278,-1.85235)(-3.72613,-1.84971)(-3.71985,-1.84626)(-3.71406,-1.84205)%
(-3.70885,-1.83715)(-3.70428,-1.83164)(-3.70045,-1.82560)(-3.69740,-1.81912)(-3.69519,-1.81231)%
(-3.69385,-1.80528)(-3.69340,-1.79814)%
%
\polygon*(-1.48426,-2.51739)(-1.48471,-2.51025)(-1.48605,-2.50321)(-1.48826,-2.49641)%
(-1.49131,-2.48993)(-1.49514,-2.48389)(-1.49971,-2.47837)(-1.50493,-2.47347)(-1.51072,-2.46926)%
(-1.51699,-2.46581)(-1.52364,-2.46318)(-1.53058,-2.46140)(-1.53768,-2.46050)(-1.54484,-2.46050)%
(-1.55194,-2.46140)(-1.55887,-2.46318)(-1.56553,-2.46581)(-1.57180,-2.46926)(-1.57759,-2.47347)%
(-1.58281,-2.47837)(-1.58737,-2.48389)(-1.59121,-2.48993)(-1.59426,-2.49641)(-1.59647,-2.50321)%
(-1.59781,-2.51025)(-1.59826,-2.51739)(-1.59781,-2.52453)(-1.59647,-2.53156)(-1.59426,-2.53837)%
(-1.59121,-2.54485)(-1.58737,-2.55089)(-1.58281,-2.55641)(-1.57759,-2.56131)(-1.57180,-2.56552)%
(-1.56553,-2.56896)(-1.55887,-2.57160)(-1.55194,-2.57338)(-1.54484,-2.57428)(-1.53768,-2.57428)%
(-1.53058,-2.57338)(-1.52364,-2.57160)(-1.51699,-2.56896)(-1.51072,-2.56552)(-1.50493,-2.56131)%
(-1.49971,-2.55641)(-1.49514,-2.55089)(-1.49131,-2.54485)(-1.48826,-2.53837)(-1.48605,-2.53156)%
(-1.48471,-2.52453)(-1.48426,-2.51739)(-1.48426,-2.51739)\polyline(-1.48426,-2.51739)(-1.48471,-2.51025)(-1.48605,-2.50321)(-1.48826,-2.49641)%
(-1.49131,-2.48993)(-1.49514,-2.48389)(-1.49971,-2.47837)(-1.50493,-2.47347)(-1.51072,-2.46926)%
(-1.51699,-2.46581)(-1.52364,-2.46318)(-1.53058,-2.46140)(-1.53768,-2.46050)(-1.54484,-2.46050)%
(-1.55194,-2.46140)(-1.55887,-2.46318)(-1.56553,-2.46581)(-1.57180,-2.46926)(-1.57759,-2.47347)%
(-1.58281,-2.47837)(-1.58737,-2.48389)(-1.59121,-2.48993)(-1.59426,-2.49641)(-1.59647,-2.50321)%
(-1.59781,-2.51025)(-1.59826,-2.51739)(-1.59781,-2.52453)(-1.59647,-2.53156)(-1.59426,-2.53837)%
(-1.59121,-2.54485)(-1.58737,-2.55089)(-1.58281,-2.55641)(-1.57759,-2.56131)(-1.57180,-2.56552)%
(-1.56553,-2.56896)(-1.55887,-2.57160)(-1.55194,-2.57338)(-1.54484,-2.57428)(-1.53768,-2.57428)%
(-1.53058,-2.57338)(-1.52364,-2.57160)(-1.51699,-2.56896)(-1.51072,-2.56552)(-1.50493,-2.56131)%
(-1.49971,-2.55641)(-1.49514,-2.55089)(-1.49131,-2.54485)(-1.48826,-2.53837)(-1.48605,-2.53156)%
(-1.48471,-2.52453)(-1.48426,-2.51739)%
%
\polygon*(0.46800,-1.07888)(0.46755,-1.07174)(0.46621,-1.06471)(0.46400,-1.05790)%
(0.46095,-1.05142)(0.45712,-1.04538)(0.45255,-1.03986)(0.44734,-1.03496)(0.44154,-1.03075)%
(0.43527,-1.02731)(0.42862,-1.02467)(0.42168,-1.02289)(0.41458,-1.02199)(0.40742,-1.02199)%
(0.40032,-1.02289)(0.39339,-1.02467)(0.38673,-1.02731)(0.38046,-1.03075)(0.37467,-1.03496)%
(0.36945,-1.03986)(0.36489,-1.04538)(0.36105,-1.05142)(0.35801,-1.05790)(0.35579,-1.06471)%
(0.35445,-1.07174)(0.35400,-1.07888)(0.35445,-1.08603)(0.35579,-1.09306)(0.35801,-1.09986)%
(0.36105,-1.10634)(0.36489,-1.11238)(0.36945,-1.11790)(0.37467,-1.12280)(0.38046,-1.12701)%
(0.38673,-1.13046)(0.39339,-1.13309)(0.40032,-1.13487)(0.40742,-1.13577)(0.41458,-1.13577)%
(0.42168,-1.13487)(0.42862,-1.13309)(0.43527,-1.13046)(0.44154,-1.12701)(0.44734,-1.12280)%
(0.45255,-1.11790)(0.45712,-1.11238)(0.46095,-1.10634)(0.46400,-1.09986)(0.46621,-1.09306)%
(0.46755,-1.08603)(0.46800,-1.07888)(0.46800,-1.07888)\polyline(0.46800,-1.07888)(0.46755,-1.07174)(0.46621,-1.06471)(0.46400,-1.05790)%
(0.46095,-1.05142)(0.45712,-1.04538)(0.45255,-1.03986)(0.44734,-1.03496)(0.44154,-1.03075)%
(0.43527,-1.02731)(0.42862,-1.02467)(0.42168,-1.02289)(0.41458,-1.02199)(0.40742,-1.02199)%
(0.40032,-1.02289)(0.39339,-1.02467)(0.38673,-1.02731)(0.38046,-1.03075)(0.37467,-1.03496)%
(0.36945,-1.03986)(0.36489,-1.04538)(0.36105,-1.05142)(0.35801,-1.05790)(0.35579,-1.06471)%
(0.35445,-1.07174)(0.35400,-1.07888)(0.35445,-1.08603)(0.35579,-1.09306)(0.35801,-1.09986)%
(0.36105,-1.10634)(0.36489,-1.11238)(0.36945,-1.11790)(0.37467,-1.12280)(0.38046,-1.12701)%
(0.38673,-1.13046)(0.39339,-1.13309)(0.40032,-1.13487)(0.40742,-1.13577)(0.41458,-1.13577)%
(0.42168,-1.13487)(0.42862,-1.13309)(0.43527,-1.13046)(0.44154,-1.12701)(0.44734,-1.12280)%
(0.45255,-1.11790)(0.45712,-1.11238)(0.46095,-1.10634)(0.46400,-1.09986)(0.46621,-1.09306)%
(0.46755,-1.08603)(0.46800,-1.07888)%
%
\polygon*(1.70101,0.53944)(1.70056,0.54658)(1.69922,0.55362)(1.69701,0.56042)(1.69396,0.56690)%
(1.69012,0.57294)(1.68556,0.57846)(1.68034,0.58336)(1.67455,0.58757)(1.66828,0.59102)%
(1.66162,0.59365)(1.65469,0.59543)(1.64759,0.59633)(1.64043,0.59633)(1.63333,0.59543)%
(1.62640,0.59365)(1.61974,0.59102)(1.61347,0.58757)(1.60768,0.58336)(1.60246,0.57846)%
(1.59790,0.57294)(1.59406,0.56690)(1.59101,0.56042)(1.58880,0.55362)(1.58746,0.54658)%
(1.58701,0.53944)(1.58746,0.53230)(1.58880,0.52527)(1.59101,0.51846)(1.59406,0.51198)%
(1.59790,0.50594)(1.60246,0.50042)(1.60768,0.49552)(1.61347,0.49131)(1.61974,0.48787)%
(1.62640,0.48523)(1.63333,0.48345)(1.64043,0.48255)(1.64759,0.48255)(1.65469,0.48345)%
(1.66162,0.48523)(1.66828,0.48787)(1.67455,0.49131)(1.68034,0.49552)(1.68556,0.50042)%
(1.69012,0.50594)(1.69396,0.51198)(1.69701,0.51846)(1.69922,0.52527)(1.70056,0.53230)%
(1.70101,0.53944)(1.70101,0.53944)\polyline(1.70101,0.53944)(1.70056,0.54658)(1.69922,0.55362)(1.69701,0.56042)(1.69396,0.56690)%
(1.69012,0.57294)(1.68556,0.57846)(1.68034,0.58336)(1.67455,0.58757)(1.66828,0.59102)%
(1.66162,0.59365)(1.65469,0.59543)(1.64759,0.59633)(1.64043,0.59633)(1.63333,0.59543)%
(1.62640,0.59365)(1.61974,0.59102)(1.61347,0.58757)(1.60768,0.58336)(1.60246,0.57846)%
(1.59790,0.57294)(1.59406,0.56690)(1.59101,0.56042)(1.58880,0.55362)(1.58746,0.54658)%
(1.58701,0.53944)(1.58746,0.53230)(1.58880,0.52527)(1.59101,0.51846)(1.59406,0.51198)%
(1.59790,0.50594)(1.60246,0.50042)(1.60768,0.49552)(1.61347,0.49131)(1.61974,0.48787)%
(1.62640,0.48523)(1.63333,0.48345)(1.64043,0.48255)(1.64759,0.48255)(1.65469,0.48345)%
(1.66162,0.48523)(1.66828,0.48787)(1.67455,0.49131)(1.68034,0.49552)(1.68556,0.50042)%
(1.69012,0.50594)(1.69396,0.51198)(1.69701,0.51846)(1.69922,0.52527)(1.70056,0.53230)%
(1.70101,0.53944)%
%
\polygon*(2.85695,-1.07888)(2.85650,-1.07174)(2.85516,-1.06471)(2.85295,-1.05790)%
(2.84990,-1.05142)(2.84607,-1.04538)(2.84150,-1.03986)(2.83629,-1.03496)(2.83050,-1.03075)%
(2.82422,-1.02731)(2.81757,-1.02467)(2.81063,-1.02289)(2.80353,-1.02199)(2.79637,-1.02199)%
(2.78927,-1.02289)(2.78234,-1.02467)(2.77568,-1.02731)(2.76941,-1.03075)(2.76362,-1.03496)%
(2.75840,-1.03986)(2.75384,-1.04538)(2.75000,-1.05142)(2.74696,-1.05790)(2.74474,-1.06471)%
(2.74340,-1.07174)(2.74295,-1.07888)(2.74340,-1.08603)(2.74474,-1.09306)(2.74696,-1.09986)%
(2.75000,-1.10634)(2.75384,-1.11238)(2.75840,-1.11790)(2.76362,-1.12280)(2.76941,-1.12701)%
(2.77568,-1.13046)(2.78234,-1.13309)(2.78927,-1.13487)(2.79637,-1.13577)(2.80353,-1.13577)%
(2.81063,-1.13487)(2.81757,-1.13309)(2.82422,-1.13046)(2.83050,-1.12701)(2.83629,-1.12280)%
(2.84150,-1.11790)(2.84607,-1.11238)(2.84990,-1.10634)(2.85295,-1.09986)(2.85516,-1.09306)%
(2.85650,-1.08603)(2.85695,-1.07888)(2.85695,-1.07888)\polyline(2.85695,-1.07888)(2.85650,-1.07174)(2.85516,-1.06471)(2.85295,-1.05790)%
(2.84990,-1.05142)(2.84607,-1.04538)(2.84150,-1.03986)(2.83629,-1.03496)(2.83050,-1.03075)%
(2.82422,-1.02731)(2.81757,-1.02467)(2.81063,-1.02289)(2.80353,-1.02199)(2.79637,-1.02199)%
(2.78927,-1.02289)(2.78234,-1.02467)(2.77568,-1.02731)(2.76941,-1.03075)(2.76362,-1.03496)%
(2.75840,-1.03986)(2.75384,-1.04538)(2.75000,-1.05142)(2.74696,-1.05790)(2.74474,-1.06471)%
(2.74340,-1.07174)(2.74295,-1.07888)(2.74340,-1.08603)(2.74474,-1.09306)(2.74696,-1.09986)%
(2.75000,-1.10634)(2.75384,-1.11238)(2.75840,-1.11790)(2.76362,-1.12280)(2.76941,-1.12701)%
(2.77568,-1.13046)(2.78234,-1.13309)(2.78927,-1.13487)(2.79637,-1.13577)(2.80353,-1.13577)%
(2.81063,-1.13487)(2.81757,-1.13309)(2.82422,-1.13046)(2.83050,-1.12701)(2.83629,-1.12280)%
(2.84150,-1.11790)(2.84607,-1.11238)(2.84990,-1.10634)(2.85295,-1.09986)(2.85516,-1.09306)%
(2.85650,-1.08603)(2.85695,-1.07888)%
%
\polygon*(4.19271,-1.05319)(4.19226,-1.04605)(4.19092,-1.03902)(4.18871,-1.03221)%
(4.18566,-1.02573)(4.18182,-1.01969)(4.17726,-1.01417)(4.17204,-1.00927)(4.16625,-1.00507)%
(4.15998,-1.00162)(4.15332,-0.99898)(4.14639,-0.99720)(4.13929,-0.99631)(4.13213,-0.99631)%
(4.12503,-0.99720)(4.11810,-0.99898)(4.11144,-1.00162)(4.10517,-1.00507)(4.09938,-1.00927)%
(4.09416,-1.01417)(4.08960,-1.01969)(4.08576,-1.02573)(4.08271,-1.03221)(4.08050,-1.03902)%
(4.07916,-1.04605)(4.07871,-1.05319)(4.07916,-1.06034)(4.08050,-1.06737)(4.08271,-1.07418)%
(4.08576,-1.08065)(4.08960,-1.08670)(4.09416,-1.09221)(4.09938,-1.09711)(4.10517,-1.10132)%
(4.11144,-1.10477)(4.11810,-1.10740)(4.12503,-1.10918)(4.13213,-1.11008)(4.13929,-1.11008)%
(4.14639,-1.10918)(4.15332,-1.10740)(4.15998,-1.10477)(4.16625,-1.10132)(4.17204,-1.09711)%
(4.17726,-1.09221)(4.18182,-1.08670)(4.18566,-1.08065)(4.18871,-1.07418)(4.19092,-1.06737)%
(4.19226,-1.06034)(4.19271,-1.05319)(4.19271,-1.05319)\polyline(4.19271,-1.05319)(4.19226,-1.04605)(4.19092,-1.03902)(4.18871,-1.03221)%
(4.18566,-1.02573)(4.18182,-1.01969)(4.17726,-1.01417)(4.17204,-1.00927)(4.16625,-1.00507)%
(4.15998,-1.00162)(4.15332,-0.99898)(4.14639,-0.99720)(4.13929,-0.99631)(4.13213,-0.99631)%
(4.12503,-0.99720)(4.11810,-0.99898)(4.11144,-1.00162)(4.10517,-1.00507)(4.09938,-1.00927)%
(4.09416,-1.01417)(4.08960,-1.01969)(4.08576,-1.02573)(4.08271,-1.03221)(4.08050,-1.03902)%
(4.07916,-1.04605)(4.07871,-1.05319)(4.07916,-1.06034)(4.08050,-1.06737)(4.08271,-1.07418)%
(4.08576,-1.08065)(4.08960,-1.08670)(4.09416,-1.09221)(4.09938,-1.09711)(4.10517,-1.10132)%
(4.11144,-1.10477)(4.11810,-1.10740)(4.12503,-1.10918)(4.13213,-1.11008)(4.13929,-1.11008)%
(4.14639,-1.10918)(4.15332,-1.10740)(4.15998,-1.10477)(4.16625,-1.10132)(4.17204,-1.09711)%
(4.17726,-1.09221)(4.18182,-1.08670)(4.18566,-1.08065)(4.18871,-1.07418)(4.19092,-1.06737)%
(4.19226,-1.06034)(4.19271,-1.05319)%
%
\polygon*(4.84777,0.04990)(4.84732,0.05704)(4.84598,0.06407)(4.84376,0.07088)(4.84072,0.07736)%
(4.83688,0.08340)(4.83232,0.08892)(4.82710,0.09382)(4.82131,0.09803)(4.81504,0.10147)%
(4.80838,0.10411)(4.80145,0.10589)(4.79435,0.10679)(4.78719,0.10679)(4.78009,0.10589)%
(4.77315,0.10411)(4.76650,0.10147)(4.76023,0.09803)(4.75443,0.09382)(4.74922,0.08892)%
(4.74465,0.08340)(4.74082,0.07736)(4.73777,0.07088)(4.73556,0.06407)(4.73422,0.05704)%
(4.73377,0.04990)(4.73422,0.04275)(4.73556,0.03572)(4.73777,0.02892)(4.74082,0.02244)%
(4.74465,0.01639)(4.74922,0.01088)(4.75443,0.00598)(4.76023,0.00177)(4.76650,-0.00168)%
(4.77315,-0.00431)(4.78009,-0.00609)(4.78719,-0.00699)(4.79435,-0.00699)(4.80145,-0.00609)%
(4.80838,-0.00431)(4.81504,-0.00168)(4.82131,0.00177)(4.82710,0.00598)(4.83232,0.01088)%
(4.83688,0.01639)(4.84072,0.02244)(4.84376,0.02892)(4.84598,0.03572)(4.84732,0.04275)%
(4.84777,0.04990)(4.84777,0.04990)\polyline(4.84777,0.04990)(4.84732,0.05704)(4.84598,0.06407)(4.84376,0.07088)(4.84072,0.07736)%
(4.83688,0.08340)(4.83232,0.08892)(4.82710,0.09382)(4.82131,0.09803)(4.81504,0.10147)%
(4.80838,0.10411)(4.80145,0.10589)(4.79435,0.10679)(4.78719,0.10679)(4.78009,0.10589)%
(4.77315,0.10411)(4.76650,0.10147)(4.76023,0.09803)(4.75443,0.09382)(4.74922,0.08892)%
(4.74465,0.08340)(4.74082,0.07736)(4.73777,0.07088)(4.73556,0.06407)(4.73422,0.05704)%
(4.73377,0.04990)(4.73422,0.04275)(4.73556,0.03572)(4.73777,0.02892)(4.74082,0.02244)%
(4.74465,0.01639)(4.74922,0.01088)(4.75443,0.00598)(4.76023,0.00177)(4.76650,-0.00168)%
(4.77315,-0.00431)(4.78009,-0.00609)(4.78719,-0.00699)(4.79435,-0.00699)(4.80145,-0.00609)%
(4.80838,-0.00431)(4.81504,-0.00168)(4.82131,0.00177)(4.82710,0.00598)(4.83232,0.01088)%
(4.83688,0.01639)(4.84072,0.02244)(4.84376,0.02892)(4.84598,0.03572)(4.84732,0.04275)%
(4.84777,0.04990)%
%
\settowidth{\Width}{1}\setlength{\Width}{-0.5\Width}%
\settoheight{\Height}{1}\settodepth{\Depth}{1}\setlength{\Height}{-\Height}%
\put(-3.7500000,-1.8833333){\hspace*{\Width}\raisebox{\Height}{1}}%
%
\settowidth{\Width}{2}\setlength{\Width}{-0.5\Width}%
\settoheight{\Height}{2}\settodepth{\Depth}{2}\setlength{\Height}{-\Height}%
\put(-1.5400000,-2.6033333){\hspace*{\Width}\raisebox{\Height}{2}}%
%
\settowidth{\Width}{3}\setlength{\Width}{0\Width}%
\settoheight{\Height}{3}\settodepth{\Depth}{3}\setlength{\Height}{-\Height}%
\put(0.4933333,-1.1633333){\hspace*{\Width}\raisebox{\Height}{3}}%
%
\settowidth{\Width}{4}\setlength{\Width}{-0.5\Width}%
\settoheight{\Height}{4}\settodepth{\Depth}{4}\setlength{\Height}{\Depth}%
\put(1.6400000,0.6233333){\hspace*{\Width}\raisebox{\Height}{4}}%
%
\settowidth{\Width}{5}\setlength{\Width}{0\Width}%
\settoheight{\Height}{5}\settodepth{\Depth}{5}\setlength{\Height}{\Depth}%
\put(2.8833333,-0.9966667){\hspace*{\Width}\raisebox{\Height}{5}}%
%
\settowidth{\Width}{6}\setlength{\Width}{-1\Width}%
\settoheight{\Height}{6}\settodepth{\Depth}{6}\setlength{\Height}{\Depth}%
\put(4.0566667,-0.9666667){\hspace*{\Width}\raisebox{\Height}{6}}%
%
\polyline(-5.10574,0.00000)(5.11794,0.00000)%
%
\polyline(0.00000,-4.23973)(0.00000,3.26000)%
%
\settowidth{\Width}{$x$}\setlength{\Width}{0\Width}%
\settoheight{\Height}{$x$}\settodepth{\Depth}{$x$}\setlength{\Height}{-0.5\Height}\setlength{\Depth}{0.5\Depth}\addtolength{\Height}{\Depth}%
\put(5.2033333,0.0000000){\hspace*{\Width}\raisebox{\Height}{$x$}}%
%
\settowidth{\Width}{$y$}\setlength{\Width}{-0.5\Width}%
\settoheight{\Height}{$y$}\settodepth{\Depth}{$y$}\setlength{\Height}{\Depth}%
\put(0.0000000,3.3433333){\hspace*{\Width}\raisebox{\Height}{$y$}}%
%
\settowidth{\Width}{O}\setlength{\Width}{-1\Width}%
\settoheight{\Height}{O}\settodepth{\Depth}{O}\setlength{\Height}{-\Height}%
\put(-0.0833333,-0.0833333){\hspace*{\Width}\raisebox{\Height}{O}}%
%
\end{picture}}%}
\end{layer}

\begin{itemize}
\item
関数$y=f(x)$
\item
$x$が右に動くにつれて\\
(i) $y$の値が増えていくとき\\
\hspace*{4zw}{\color{red}増加の状態}\\
(ii) $y$の値が減っていくとき\\
\hspace*{4zw}{\color{red}減少の状態}
\item
[課題]\monban 図で次の状態である点の番号をすべてあげよ.\seteda{38}\\
\eda{増加の状態}\eda{減少の状態}\eda{どちらでもない}
\end{itemize}

\newslide{関数の増減と微分}

\vspace*{18mm}


\begin{layer}{120}{0}
\putnotew{96}{73}{\hyperlink{para1pg2}{\fbox{\Ctab{2.5mm}{\scalebox{1}{\scriptsize $\mathstrut||\!\lhd$}}}}}
\putnotew{101}{73}{\hyperlink{para2pg1}{\fbox{\Ctab{2.5mm}{\scalebox{1}{\scriptsize $\mathstrut|\!\lhd$}}}}}
\putnotew{108}{73}{\hyperlink{para2pg5}{\fbox{\Ctab{4.5mm}{\scalebox{1}{\scriptsize $\mathstrut\!\!\lhd\!\!$}}}}}
\putnotew{115}{73}{\hyperlink{para2pg6}{\fbox{\Ctab{4.5mm}{\scalebox{1}{\scriptsize $\mathstrut\!\rhd\!$}}}}}
\putnotew{120}{73}{\hyperlink{para2pg6}{\fbox{\Ctab{2.5mm}{\scalebox{1}{\scriptsize $\mathstrut \!\rhd\!\!|$}}}}}
\putnotew{125}{73}{\hyperlink{para3pg1}{\fbox{\Ctab{2.5mm}{\scalebox{1}{\scriptsize $\mathstrut \!\rhd\!\!||$}}}}}
\putnotee{126}{73}{\scriptsize\color{blue} 6/6}
\end{layer}

\slidepage

\begin{layer}{120}{0}
\putnotese{70}{5}{%%% /Users/takatoosetsuo/polytech22.git/205-0926/presen/fig/zougen.tex 
%%% Generator=graph22207.cdy 
{\unitlength=6mm%
\begin{picture}%
(10.23,7.5)(-5.11,-4.24)%
\linethickness{0.008in}%%
\linethickness{0.016in}%%
\polyline(-4.85881,1.16837)(-4.75563,0.72408)(-4.65075,0.31260)(-4.54417,-0.06607)%
(-4.43589,-0.41194)(-4.32590,-0.72499)(-4.21420,-1.00524)(-4.10081,-1.25268)(-3.98571,-1.46730)%
(-3.86890,-1.64912)(-3.75040,-1.79814)(-3.56373,-1.98288)(-3.36334,-2.13996)(-3.15152,-2.27035)%
(-2.93056,-2.37501)(-2.70275,-2.45492)(-2.47039,-2.51106)(-2.23577,-2.54440)(-2.00118,-2.55592)%
(-1.76891,-2.54659)(-1.54126,-2.51739)(-1.30956,-2.45946)(-1.09097,-2.37109)(-0.88361,-2.25653)%
(-0.68561,-2.12003)(-0.49509,-1.96587)(-0.31017,-1.79828)(-0.12898,-1.62153)(0.05037,-1.43988)%
(0.22973,-1.25758)(0.41100,-1.07888)(0.59227,-0.87994)(0.74478,-0.66780)(0.87447,-0.45051)%
(0.98732,-0.23611)(1.08927,-0.03264)(1.18627,0.15187)(1.28430,0.30937)(1.38929,0.43183)%
(1.50721,0.51120)(1.64401,0.53944)(1.80627,0.50501)(1.93586,0.40983)(2.04114,0.26606)%
(2.13047,0.08587)(2.21223,-0.11857)(2.29477,-0.33511)(2.38644,-0.55157)(2.49563,-0.75580)%
(2.63067,-0.93562)(2.79995,-1.07888)(2.93827,-1.15452)(3.07892,-1.20789)(3.22064,-1.24082)%
(3.36215,-1.25514)(3.50219,-1.25270)(3.63947,-1.23532)(3.77274,-1.20485)(3.90072,-1.16311)%
(4.02213,-1.11195)(4.13571,-1.05319)(4.20781,-1.00500)(4.27844,-0.94300)(4.34760,-0.86720)%
(4.41531,-0.77760)(4.48154,-0.67419)(4.54632,-0.55698)(4.60963,-0.42596)(4.67147,-0.28115)%
(4.73185,-0.12253)(4.79077,0.04990)%
%
\linethickness{0.008in}%%
\polygon*(-4.80181,1.16837)(-4.80226,1.17552)(-4.80360,1.18255)(-4.80581,1.18936)%
(-4.80886,1.19583)(-4.81269,1.20188)(-4.81726,1.20739)(-4.82248,1.21229)(-4.82827,1.21650)%
(-4.83454,1.21995)(-4.84119,1.22258)(-4.84813,1.22436)(-4.85523,1.22526)(-4.86239,1.22526)%
(-4.86949,1.22436)(-4.87642,1.22258)(-4.88308,1.21995)(-4.88935,1.21650)(-4.89514,1.21229)%
(-4.90036,1.20739)(-4.90492,1.20188)(-4.90876,1.19583)(-4.91181,1.18936)(-4.91402,1.18255)%
(-4.91536,1.17552)(-4.91581,1.16837)(-4.91536,1.16123)(-4.91402,1.15420)(-4.91181,1.14739)%
(-4.90876,1.14091)(-4.90492,1.13487)(-4.90036,1.12935)(-4.89514,1.12445)(-4.88935,1.12025)%
(-4.88308,1.11680)(-4.87642,1.11416)(-4.86949,1.11238)(-4.86239,1.11148)(-4.85523,1.11148)%
(-4.84813,1.11238)(-4.84119,1.11416)(-4.83454,1.11680)(-4.82827,1.12025)(-4.82248,1.12445)%
(-4.81726,1.12935)(-4.81269,1.13487)(-4.80886,1.14091)(-4.80581,1.14739)(-4.80360,1.15420)%
(-4.80226,1.16123)(-4.80181,1.16837)(-4.80181,1.16837)\polyline(-4.80181,1.16837)(-4.80226,1.17552)(-4.80360,1.18255)(-4.80581,1.18936)%
(-4.80886,1.19583)(-4.81269,1.20188)(-4.81726,1.20739)(-4.82248,1.21229)(-4.82827,1.21650)%
(-4.83454,1.21995)(-4.84119,1.22258)(-4.84813,1.22436)(-4.85523,1.22526)(-4.86239,1.22526)%
(-4.86949,1.22436)(-4.87642,1.22258)(-4.88308,1.21995)(-4.88935,1.21650)(-4.89514,1.21229)%
(-4.90036,1.20739)(-4.90492,1.20188)(-4.90876,1.19583)(-4.91181,1.18936)(-4.91402,1.18255)%
(-4.91536,1.17552)(-4.91581,1.16837)(-4.91536,1.16123)(-4.91402,1.15420)(-4.91181,1.14739)%
(-4.90876,1.14091)(-4.90492,1.13487)(-4.90036,1.12935)(-4.89514,1.12445)(-4.88935,1.12025)%
(-4.88308,1.11680)(-4.87642,1.11416)(-4.86949,1.11238)(-4.86239,1.11148)(-4.85523,1.11148)%
(-4.84813,1.11238)(-4.84119,1.11416)(-4.83454,1.11680)(-4.82827,1.12025)(-4.82248,1.12445)%
(-4.81726,1.12935)(-4.81269,1.13487)(-4.80886,1.14091)(-4.80581,1.14739)(-4.80360,1.15420)%
(-4.80226,1.16123)(-4.80181,1.16837)%
%
\polygon*(-3.69340,-1.79814)(-3.69385,-1.79099)(-3.69519,-1.78396)(-3.69740,-1.77715)%
(-3.70045,-1.77068)(-3.70428,-1.76463)(-3.70885,-1.75912)(-3.71406,-1.75422)(-3.71985,-1.75001)%
(-3.72613,-1.74656)(-3.73278,-1.74392)(-3.73972,-1.74214)(-3.74682,-1.74125)(-3.75398,-1.74125)%
(-3.76108,-1.74214)(-3.76801,-1.74392)(-3.77467,-1.74656)(-3.78094,-1.75001)(-3.78673,-1.75422)%
(-3.79195,-1.75912)(-3.79651,-1.76463)(-3.80035,-1.77068)(-3.80339,-1.77715)(-3.80561,-1.78396)%
(-3.80695,-1.79099)(-3.80740,-1.79814)(-3.80695,-1.80528)(-3.80561,-1.81231)(-3.80339,-1.81912)%
(-3.80035,-1.82560)(-3.79651,-1.83164)(-3.79195,-1.83715)(-3.78673,-1.84205)(-3.78094,-1.84626)%
(-3.77467,-1.84971)(-3.76801,-1.85235)(-3.76108,-1.85413)(-3.75398,-1.85502)(-3.74682,-1.85502)%
(-3.73972,-1.85413)(-3.73278,-1.85235)(-3.72613,-1.84971)(-3.71985,-1.84626)(-3.71406,-1.84205)%
(-3.70885,-1.83715)(-3.70428,-1.83164)(-3.70045,-1.82560)(-3.69740,-1.81912)(-3.69519,-1.81231)%
(-3.69385,-1.80528)(-3.69340,-1.79814)(-3.69340,-1.79814)\polyline(-3.69340,-1.79814)(-3.69385,-1.79099)(-3.69519,-1.78396)(-3.69740,-1.77715)%
(-3.70045,-1.77068)(-3.70428,-1.76463)(-3.70885,-1.75912)(-3.71406,-1.75422)(-3.71985,-1.75001)%
(-3.72613,-1.74656)(-3.73278,-1.74392)(-3.73972,-1.74214)(-3.74682,-1.74125)(-3.75398,-1.74125)%
(-3.76108,-1.74214)(-3.76801,-1.74392)(-3.77467,-1.74656)(-3.78094,-1.75001)(-3.78673,-1.75422)%
(-3.79195,-1.75912)(-3.79651,-1.76463)(-3.80035,-1.77068)(-3.80339,-1.77715)(-3.80561,-1.78396)%
(-3.80695,-1.79099)(-3.80740,-1.79814)(-3.80695,-1.80528)(-3.80561,-1.81231)(-3.80339,-1.81912)%
(-3.80035,-1.82560)(-3.79651,-1.83164)(-3.79195,-1.83715)(-3.78673,-1.84205)(-3.78094,-1.84626)%
(-3.77467,-1.84971)(-3.76801,-1.85235)(-3.76108,-1.85413)(-3.75398,-1.85502)(-3.74682,-1.85502)%
(-3.73972,-1.85413)(-3.73278,-1.85235)(-3.72613,-1.84971)(-3.71985,-1.84626)(-3.71406,-1.84205)%
(-3.70885,-1.83715)(-3.70428,-1.83164)(-3.70045,-1.82560)(-3.69740,-1.81912)(-3.69519,-1.81231)%
(-3.69385,-1.80528)(-3.69340,-1.79814)%
%
\polygon*(-1.48426,-2.51739)(-1.48471,-2.51025)(-1.48605,-2.50321)(-1.48826,-2.49641)%
(-1.49131,-2.48993)(-1.49514,-2.48389)(-1.49971,-2.47837)(-1.50493,-2.47347)(-1.51072,-2.46926)%
(-1.51699,-2.46581)(-1.52364,-2.46318)(-1.53058,-2.46140)(-1.53768,-2.46050)(-1.54484,-2.46050)%
(-1.55194,-2.46140)(-1.55887,-2.46318)(-1.56553,-2.46581)(-1.57180,-2.46926)(-1.57759,-2.47347)%
(-1.58281,-2.47837)(-1.58737,-2.48389)(-1.59121,-2.48993)(-1.59426,-2.49641)(-1.59647,-2.50321)%
(-1.59781,-2.51025)(-1.59826,-2.51739)(-1.59781,-2.52453)(-1.59647,-2.53156)(-1.59426,-2.53837)%
(-1.59121,-2.54485)(-1.58737,-2.55089)(-1.58281,-2.55641)(-1.57759,-2.56131)(-1.57180,-2.56552)%
(-1.56553,-2.56896)(-1.55887,-2.57160)(-1.55194,-2.57338)(-1.54484,-2.57428)(-1.53768,-2.57428)%
(-1.53058,-2.57338)(-1.52364,-2.57160)(-1.51699,-2.56896)(-1.51072,-2.56552)(-1.50493,-2.56131)%
(-1.49971,-2.55641)(-1.49514,-2.55089)(-1.49131,-2.54485)(-1.48826,-2.53837)(-1.48605,-2.53156)%
(-1.48471,-2.52453)(-1.48426,-2.51739)(-1.48426,-2.51739)\polyline(-1.48426,-2.51739)(-1.48471,-2.51025)(-1.48605,-2.50321)(-1.48826,-2.49641)%
(-1.49131,-2.48993)(-1.49514,-2.48389)(-1.49971,-2.47837)(-1.50493,-2.47347)(-1.51072,-2.46926)%
(-1.51699,-2.46581)(-1.52364,-2.46318)(-1.53058,-2.46140)(-1.53768,-2.46050)(-1.54484,-2.46050)%
(-1.55194,-2.46140)(-1.55887,-2.46318)(-1.56553,-2.46581)(-1.57180,-2.46926)(-1.57759,-2.47347)%
(-1.58281,-2.47837)(-1.58737,-2.48389)(-1.59121,-2.48993)(-1.59426,-2.49641)(-1.59647,-2.50321)%
(-1.59781,-2.51025)(-1.59826,-2.51739)(-1.59781,-2.52453)(-1.59647,-2.53156)(-1.59426,-2.53837)%
(-1.59121,-2.54485)(-1.58737,-2.55089)(-1.58281,-2.55641)(-1.57759,-2.56131)(-1.57180,-2.56552)%
(-1.56553,-2.56896)(-1.55887,-2.57160)(-1.55194,-2.57338)(-1.54484,-2.57428)(-1.53768,-2.57428)%
(-1.53058,-2.57338)(-1.52364,-2.57160)(-1.51699,-2.56896)(-1.51072,-2.56552)(-1.50493,-2.56131)%
(-1.49971,-2.55641)(-1.49514,-2.55089)(-1.49131,-2.54485)(-1.48826,-2.53837)(-1.48605,-2.53156)%
(-1.48471,-2.52453)(-1.48426,-2.51739)%
%
\polygon*(0.46800,-1.07888)(0.46755,-1.07174)(0.46621,-1.06471)(0.46400,-1.05790)%
(0.46095,-1.05142)(0.45712,-1.04538)(0.45255,-1.03986)(0.44734,-1.03496)(0.44154,-1.03075)%
(0.43527,-1.02731)(0.42862,-1.02467)(0.42168,-1.02289)(0.41458,-1.02199)(0.40742,-1.02199)%
(0.40032,-1.02289)(0.39339,-1.02467)(0.38673,-1.02731)(0.38046,-1.03075)(0.37467,-1.03496)%
(0.36945,-1.03986)(0.36489,-1.04538)(0.36105,-1.05142)(0.35801,-1.05790)(0.35579,-1.06471)%
(0.35445,-1.07174)(0.35400,-1.07888)(0.35445,-1.08603)(0.35579,-1.09306)(0.35801,-1.09986)%
(0.36105,-1.10634)(0.36489,-1.11238)(0.36945,-1.11790)(0.37467,-1.12280)(0.38046,-1.12701)%
(0.38673,-1.13046)(0.39339,-1.13309)(0.40032,-1.13487)(0.40742,-1.13577)(0.41458,-1.13577)%
(0.42168,-1.13487)(0.42862,-1.13309)(0.43527,-1.13046)(0.44154,-1.12701)(0.44734,-1.12280)%
(0.45255,-1.11790)(0.45712,-1.11238)(0.46095,-1.10634)(0.46400,-1.09986)(0.46621,-1.09306)%
(0.46755,-1.08603)(0.46800,-1.07888)(0.46800,-1.07888)\polyline(0.46800,-1.07888)(0.46755,-1.07174)(0.46621,-1.06471)(0.46400,-1.05790)%
(0.46095,-1.05142)(0.45712,-1.04538)(0.45255,-1.03986)(0.44734,-1.03496)(0.44154,-1.03075)%
(0.43527,-1.02731)(0.42862,-1.02467)(0.42168,-1.02289)(0.41458,-1.02199)(0.40742,-1.02199)%
(0.40032,-1.02289)(0.39339,-1.02467)(0.38673,-1.02731)(0.38046,-1.03075)(0.37467,-1.03496)%
(0.36945,-1.03986)(0.36489,-1.04538)(0.36105,-1.05142)(0.35801,-1.05790)(0.35579,-1.06471)%
(0.35445,-1.07174)(0.35400,-1.07888)(0.35445,-1.08603)(0.35579,-1.09306)(0.35801,-1.09986)%
(0.36105,-1.10634)(0.36489,-1.11238)(0.36945,-1.11790)(0.37467,-1.12280)(0.38046,-1.12701)%
(0.38673,-1.13046)(0.39339,-1.13309)(0.40032,-1.13487)(0.40742,-1.13577)(0.41458,-1.13577)%
(0.42168,-1.13487)(0.42862,-1.13309)(0.43527,-1.13046)(0.44154,-1.12701)(0.44734,-1.12280)%
(0.45255,-1.11790)(0.45712,-1.11238)(0.46095,-1.10634)(0.46400,-1.09986)(0.46621,-1.09306)%
(0.46755,-1.08603)(0.46800,-1.07888)%
%
\polygon*(1.70101,0.53944)(1.70056,0.54658)(1.69922,0.55362)(1.69701,0.56042)(1.69396,0.56690)%
(1.69012,0.57294)(1.68556,0.57846)(1.68034,0.58336)(1.67455,0.58757)(1.66828,0.59102)%
(1.66162,0.59365)(1.65469,0.59543)(1.64759,0.59633)(1.64043,0.59633)(1.63333,0.59543)%
(1.62640,0.59365)(1.61974,0.59102)(1.61347,0.58757)(1.60768,0.58336)(1.60246,0.57846)%
(1.59790,0.57294)(1.59406,0.56690)(1.59101,0.56042)(1.58880,0.55362)(1.58746,0.54658)%
(1.58701,0.53944)(1.58746,0.53230)(1.58880,0.52527)(1.59101,0.51846)(1.59406,0.51198)%
(1.59790,0.50594)(1.60246,0.50042)(1.60768,0.49552)(1.61347,0.49131)(1.61974,0.48787)%
(1.62640,0.48523)(1.63333,0.48345)(1.64043,0.48255)(1.64759,0.48255)(1.65469,0.48345)%
(1.66162,0.48523)(1.66828,0.48787)(1.67455,0.49131)(1.68034,0.49552)(1.68556,0.50042)%
(1.69012,0.50594)(1.69396,0.51198)(1.69701,0.51846)(1.69922,0.52527)(1.70056,0.53230)%
(1.70101,0.53944)(1.70101,0.53944)\polyline(1.70101,0.53944)(1.70056,0.54658)(1.69922,0.55362)(1.69701,0.56042)(1.69396,0.56690)%
(1.69012,0.57294)(1.68556,0.57846)(1.68034,0.58336)(1.67455,0.58757)(1.66828,0.59102)%
(1.66162,0.59365)(1.65469,0.59543)(1.64759,0.59633)(1.64043,0.59633)(1.63333,0.59543)%
(1.62640,0.59365)(1.61974,0.59102)(1.61347,0.58757)(1.60768,0.58336)(1.60246,0.57846)%
(1.59790,0.57294)(1.59406,0.56690)(1.59101,0.56042)(1.58880,0.55362)(1.58746,0.54658)%
(1.58701,0.53944)(1.58746,0.53230)(1.58880,0.52527)(1.59101,0.51846)(1.59406,0.51198)%
(1.59790,0.50594)(1.60246,0.50042)(1.60768,0.49552)(1.61347,0.49131)(1.61974,0.48787)%
(1.62640,0.48523)(1.63333,0.48345)(1.64043,0.48255)(1.64759,0.48255)(1.65469,0.48345)%
(1.66162,0.48523)(1.66828,0.48787)(1.67455,0.49131)(1.68034,0.49552)(1.68556,0.50042)%
(1.69012,0.50594)(1.69396,0.51198)(1.69701,0.51846)(1.69922,0.52527)(1.70056,0.53230)%
(1.70101,0.53944)%
%
\polygon*(2.85695,-1.07888)(2.85650,-1.07174)(2.85516,-1.06471)(2.85295,-1.05790)%
(2.84990,-1.05142)(2.84607,-1.04538)(2.84150,-1.03986)(2.83629,-1.03496)(2.83050,-1.03075)%
(2.82422,-1.02731)(2.81757,-1.02467)(2.81063,-1.02289)(2.80353,-1.02199)(2.79637,-1.02199)%
(2.78927,-1.02289)(2.78234,-1.02467)(2.77568,-1.02731)(2.76941,-1.03075)(2.76362,-1.03496)%
(2.75840,-1.03986)(2.75384,-1.04538)(2.75000,-1.05142)(2.74696,-1.05790)(2.74474,-1.06471)%
(2.74340,-1.07174)(2.74295,-1.07888)(2.74340,-1.08603)(2.74474,-1.09306)(2.74696,-1.09986)%
(2.75000,-1.10634)(2.75384,-1.11238)(2.75840,-1.11790)(2.76362,-1.12280)(2.76941,-1.12701)%
(2.77568,-1.13046)(2.78234,-1.13309)(2.78927,-1.13487)(2.79637,-1.13577)(2.80353,-1.13577)%
(2.81063,-1.13487)(2.81757,-1.13309)(2.82422,-1.13046)(2.83050,-1.12701)(2.83629,-1.12280)%
(2.84150,-1.11790)(2.84607,-1.11238)(2.84990,-1.10634)(2.85295,-1.09986)(2.85516,-1.09306)%
(2.85650,-1.08603)(2.85695,-1.07888)(2.85695,-1.07888)\polyline(2.85695,-1.07888)(2.85650,-1.07174)(2.85516,-1.06471)(2.85295,-1.05790)%
(2.84990,-1.05142)(2.84607,-1.04538)(2.84150,-1.03986)(2.83629,-1.03496)(2.83050,-1.03075)%
(2.82422,-1.02731)(2.81757,-1.02467)(2.81063,-1.02289)(2.80353,-1.02199)(2.79637,-1.02199)%
(2.78927,-1.02289)(2.78234,-1.02467)(2.77568,-1.02731)(2.76941,-1.03075)(2.76362,-1.03496)%
(2.75840,-1.03986)(2.75384,-1.04538)(2.75000,-1.05142)(2.74696,-1.05790)(2.74474,-1.06471)%
(2.74340,-1.07174)(2.74295,-1.07888)(2.74340,-1.08603)(2.74474,-1.09306)(2.74696,-1.09986)%
(2.75000,-1.10634)(2.75384,-1.11238)(2.75840,-1.11790)(2.76362,-1.12280)(2.76941,-1.12701)%
(2.77568,-1.13046)(2.78234,-1.13309)(2.78927,-1.13487)(2.79637,-1.13577)(2.80353,-1.13577)%
(2.81063,-1.13487)(2.81757,-1.13309)(2.82422,-1.13046)(2.83050,-1.12701)(2.83629,-1.12280)%
(2.84150,-1.11790)(2.84607,-1.11238)(2.84990,-1.10634)(2.85295,-1.09986)(2.85516,-1.09306)%
(2.85650,-1.08603)(2.85695,-1.07888)%
%
\polygon*(4.19271,-1.05319)(4.19226,-1.04605)(4.19092,-1.03902)(4.18871,-1.03221)%
(4.18566,-1.02573)(4.18182,-1.01969)(4.17726,-1.01417)(4.17204,-1.00927)(4.16625,-1.00507)%
(4.15998,-1.00162)(4.15332,-0.99898)(4.14639,-0.99720)(4.13929,-0.99631)(4.13213,-0.99631)%
(4.12503,-0.99720)(4.11810,-0.99898)(4.11144,-1.00162)(4.10517,-1.00507)(4.09938,-1.00927)%
(4.09416,-1.01417)(4.08960,-1.01969)(4.08576,-1.02573)(4.08271,-1.03221)(4.08050,-1.03902)%
(4.07916,-1.04605)(4.07871,-1.05319)(4.07916,-1.06034)(4.08050,-1.06737)(4.08271,-1.07418)%
(4.08576,-1.08065)(4.08960,-1.08670)(4.09416,-1.09221)(4.09938,-1.09711)(4.10517,-1.10132)%
(4.11144,-1.10477)(4.11810,-1.10740)(4.12503,-1.10918)(4.13213,-1.11008)(4.13929,-1.11008)%
(4.14639,-1.10918)(4.15332,-1.10740)(4.15998,-1.10477)(4.16625,-1.10132)(4.17204,-1.09711)%
(4.17726,-1.09221)(4.18182,-1.08670)(4.18566,-1.08065)(4.18871,-1.07418)(4.19092,-1.06737)%
(4.19226,-1.06034)(4.19271,-1.05319)(4.19271,-1.05319)\polyline(4.19271,-1.05319)(4.19226,-1.04605)(4.19092,-1.03902)(4.18871,-1.03221)%
(4.18566,-1.02573)(4.18182,-1.01969)(4.17726,-1.01417)(4.17204,-1.00927)(4.16625,-1.00507)%
(4.15998,-1.00162)(4.15332,-0.99898)(4.14639,-0.99720)(4.13929,-0.99631)(4.13213,-0.99631)%
(4.12503,-0.99720)(4.11810,-0.99898)(4.11144,-1.00162)(4.10517,-1.00507)(4.09938,-1.00927)%
(4.09416,-1.01417)(4.08960,-1.01969)(4.08576,-1.02573)(4.08271,-1.03221)(4.08050,-1.03902)%
(4.07916,-1.04605)(4.07871,-1.05319)(4.07916,-1.06034)(4.08050,-1.06737)(4.08271,-1.07418)%
(4.08576,-1.08065)(4.08960,-1.08670)(4.09416,-1.09221)(4.09938,-1.09711)(4.10517,-1.10132)%
(4.11144,-1.10477)(4.11810,-1.10740)(4.12503,-1.10918)(4.13213,-1.11008)(4.13929,-1.11008)%
(4.14639,-1.10918)(4.15332,-1.10740)(4.15998,-1.10477)(4.16625,-1.10132)(4.17204,-1.09711)%
(4.17726,-1.09221)(4.18182,-1.08670)(4.18566,-1.08065)(4.18871,-1.07418)(4.19092,-1.06737)%
(4.19226,-1.06034)(4.19271,-1.05319)%
%
\polygon*(4.84777,0.04990)(4.84732,0.05704)(4.84598,0.06407)(4.84376,0.07088)(4.84072,0.07736)%
(4.83688,0.08340)(4.83232,0.08892)(4.82710,0.09382)(4.82131,0.09803)(4.81504,0.10147)%
(4.80838,0.10411)(4.80145,0.10589)(4.79435,0.10679)(4.78719,0.10679)(4.78009,0.10589)%
(4.77315,0.10411)(4.76650,0.10147)(4.76023,0.09803)(4.75443,0.09382)(4.74922,0.08892)%
(4.74465,0.08340)(4.74082,0.07736)(4.73777,0.07088)(4.73556,0.06407)(4.73422,0.05704)%
(4.73377,0.04990)(4.73422,0.04275)(4.73556,0.03572)(4.73777,0.02892)(4.74082,0.02244)%
(4.74465,0.01639)(4.74922,0.01088)(4.75443,0.00598)(4.76023,0.00177)(4.76650,-0.00168)%
(4.77315,-0.00431)(4.78009,-0.00609)(4.78719,-0.00699)(4.79435,-0.00699)(4.80145,-0.00609)%
(4.80838,-0.00431)(4.81504,-0.00168)(4.82131,0.00177)(4.82710,0.00598)(4.83232,0.01088)%
(4.83688,0.01639)(4.84072,0.02244)(4.84376,0.02892)(4.84598,0.03572)(4.84732,0.04275)%
(4.84777,0.04990)(4.84777,0.04990)\polyline(4.84777,0.04990)(4.84732,0.05704)(4.84598,0.06407)(4.84376,0.07088)(4.84072,0.07736)%
(4.83688,0.08340)(4.83232,0.08892)(4.82710,0.09382)(4.82131,0.09803)(4.81504,0.10147)%
(4.80838,0.10411)(4.80145,0.10589)(4.79435,0.10679)(4.78719,0.10679)(4.78009,0.10589)%
(4.77315,0.10411)(4.76650,0.10147)(4.76023,0.09803)(4.75443,0.09382)(4.74922,0.08892)%
(4.74465,0.08340)(4.74082,0.07736)(4.73777,0.07088)(4.73556,0.06407)(4.73422,0.05704)%
(4.73377,0.04990)(4.73422,0.04275)(4.73556,0.03572)(4.73777,0.02892)(4.74082,0.02244)%
(4.74465,0.01639)(4.74922,0.01088)(4.75443,0.00598)(4.76023,0.00177)(4.76650,-0.00168)%
(4.77315,-0.00431)(4.78009,-0.00609)(4.78719,-0.00699)(4.79435,-0.00699)(4.80145,-0.00609)%
(4.80838,-0.00431)(4.81504,-0.00168)(4.82131,0.00177)(4.82710,0.00598)(4.83232,0.01088)%
(4.83688,0.01639)(4.84072,0.02244)(4.84376,0.02892)(4.84598,0.03572)(4.84732,0.04275)%
(4.84777,0.04990)%
%
\polyline(-5.10574,0.00000)(5.11794,0.00000)%
%
\polyline(0.00000,-4.23973)(0.00000,3.26000)%
%
\settowidth{\Width}{$x$}\setlength{\Width}{0\Width}%
\settoheight{\Height}{$x$}\settodepth{\Depth}{$x$}\setlength{\Height}{-0.5\Height}\setlength{\Depth}{0.5\Depth}\addtolength{\Height}{\Depth}%
\put(5.2033333,0.0000000){\hspace*{\Width}\raisebox{\Height}{$x$}}%
%
\settowidth{\Width}{$y$}\setlength{\Width}{-0.5\Width}%
\settoheight{\Height}{$y$}\settodepth{\Depth}{$y$}\setlength{\Height}{\Depth}%
\put(0.0000000,3.3433333){\hspace*{\Width}\raisebox{\Height}{$y$}}%
%
\settowidth{\Width}{O}\setlength{\Width}{-1\Width}%
\settoheight{\Height}{O}\settodepth{\Depth}{O}\setlength{\Height}{-\Height}%
\put(-0.0833333,-0.0833333){\hspace*{\Width}\raisebox{\Height}{O}}%
%
\end{picture}}%}
\putnotee{93}{53.7}{その近くでは最大}
\putnotee{93}{62}{その近くでは最小}
\end{layer}

\begin{itemize}
\item
接線の傾きは導関数の値(微分係数)\\
(i) 増加の状態のとき\\
\hspace*{1zw}接線の傾きは正\ {\color{red}$y'>0$}\\
(ii) 減少の状態のとき\\
\hspace*{1zw}接線の傾きは負\ {\color{red}$y'<0$}\vspace{-2mm}
\item
増加減少が変化する点\\
(i) 増加から減少に変わるとき {\color{red}極大}\\
(ii) 減少から増加に変わるとき {\color{red}極小}
\end{itemize}

\newslide{極値点と$y'$の値}

\vspace*{18mm}


\begin{layer}{120}{0}
\putnotew{96}{73}{\hyperlink{para2pg6}{\fbox{\Ctab{2.5mm}{\scalebox{1}{\scriptsize $\mathstrut||\!\lhd$}}}}}
\putnotew{101}{73}{\hyperlink{para3pg1}{\fbox{\Ctab{2.5mm}{\scalebox{1}{\scriptsize $\mathstrut|\!\lhd$}}}}}
\putnotew{108}{73}{\hyperlink{para3pg8}{\fbox{\Ctab{4.5mm}{\scalebox{1}{\scriptsize $\mathstrut\!\!\lhd\!\!$}}}}}
\putnotew{115}{73}{\hyperlink{para3pg9}{\fbox{\Ctab{4.5mm}{\scalebox{1}{\scriptsize $\mathstrut\!\rhd\!$}}}}}
\putnotew{120}{73}{\hyperlink{para3pg9}{\fbox{\Ctab{2.5mm}{\scalebox{1}{\scriptsize $\mathstrut \!\rhd\!\!|$}}}}}
\putnotew{125}{73}{\hyperlink{para4pg1}{\fbox{\Ctab{2.5mm}{\scalebox{1}{\scriptsize $\mathstrut \!\rhd\!\!||$}}}}}
\putnotee{126}{73}{\scriptsize\color{blue} 9/9}
\end{layer}

\slidepage

\begin{layer}{120}{0}
\putnotese{85}{5}{%%% /Users/takatoosetsuo/polytech22.git/205-0926/presen/fig/kyokuchi1.tex 
%%% Generator=graph22207.cdy 
{\unitlength=6mm%
\begin{picture}%
(6.78,4.98)(-2.14,-1.86)%
\linethickness{0.008in}%%
\linethickness{0.012in}%%
\polyline(-1.72510,-1.86000)(-1.64320,-0.75807)(-1.56040,0.15790)(-1.47760,0.89554)%
(-1.39480,1.47347)(-1.31200,1.90920)(-1.22920,2.21912)(-1.14640,2.41859)(-1.06360,2.52194)%
(-0.98080,2.54250)(-0.89800,2.49264)(-0.81520,2.38379)(-0.73240,2.22648)(-0.64960,2.03037)%
(-0.56680,1.80430)(-0.48400,1.55627)(-0.40120,1.29351)(-0.31840,1.02253)(-0.23560,0.74909)%
(-0.15280,0.47830)(-0.07000,0.21458)(0.01280,-0.03823)(0.09560,-0.27693)(0.17840,-0.49882)%
(0.26120,-0.70178)(0.34400,-0.88416)(0.42680,-1.04478)(0.50960,-1.18290)(0.59240,-1.29818)%
(0.67520,-1.39065)(0.75800,-1.46069)(0.84080,-1.50897)(0.92360,-1.53647)(1.00640,-1.54439)%
(1.08920,-1.53416)(1.17200,-1.50740)(1.25480,-1.46586)(1.33760,-1.41145)(1.42040,-1.34615)%
(1.50320,-1.27199)(1.58600,-1.19107)(1.66880,-1.10543)(1.75160,-1.01714)(1.83440,-0.92816)%
(1.91720,-0.84038)(2.00000,-0.75556)%
%
\linethickness{0.008in}%%
{%
\color[cmyk]{0,1,1,0}%
\polygon*(-0.94300,2.53763)(-0.94345,2.54478)(-0.94479,2.55181)(-0.94700,2.55861)%
(-0.95005,2.56509)(-0.95389,2.57114)(-0.95845,2.57665)(-0.96367,2.58155)(-0.96946,2.58576)%
(-0.97573,2.58921)(-0.98239,2.59184)(-0.98932,2.59362)(-0.99642,2.59452)(-1.00358,2.59452)%
(-1.01068,2.59362)(-1.01761,2.59184)(-1.02427,2.58921)(-1.03054,2.58576)(-1.03633,2.58155)%
(-1.04155,2.57665)(-1.04611,2.57114)(-1.04995,2.56509)(-1.05300,2.55861)(-1.05521,2.55181)%
(-1.05655,2.54478)(-1.05700,2.53763)(-1.05655,2.53049)(-1.05521,2.52346)(-1.05300,2.51665)%
(-1.04995,2.51017)(-1.04611,2.50413)(-1.04155,2.49861)(-1.03633,2.49371)(-1.03054,2.48950)%
(-1.02427,2.48606)(-1.01761,2.48342)(-1.01068,2.48164)(-1.00358,2.48074)(-0.99642,2.48074)%
(-0.98932,2.48164)(-0.98239,2.48342)(-0.97573,2.48606)(-0.96946,2.48950)(-0.96367,2.49371)%
(-0.95845,2.49861)(-0.95389,2.50413)(-0.95005,2.51017)(-0.94700,2.51665)(-0.94479,2.52346)%
(-0.94345,2.53049)(-0.94300,2.53763)(-0.94300,2.53763)}%
\polyline(-0.94300,2.53763)(-0.94345,2.54478)(-0.94479,2.55181)(-0.94700,2.55861)%
(-0.95005,2.56509)(-0.95389,2.57114)(-0.95845,2.57665)(-0.96367,2.58155)(-0.96946,2.58576)%
(-0.97573,2.58921)(-0.98239,2.59184)(-0.98932,2.59362)(-0.99642,2.59452)(-1.00358,2.59452)%
(-1.01068,2.59362)(-1.01761,2.59184)(-1.02427,2.58921)(-1.03054,2.58576)(-1.03633,2.58155)%
(-1.04155,2.57665)(-1.04611,2.57114)(-1.04995,2.56509)(-1.05300,2.55861)(-1.05521,2.55181)%
(-1.05655,2.54478)(-1.05700,2.53763)(-1.05655,2.53049)(-1.05521,2.52346)(-1.05300,2.51665)%
(-1.04995,2.51017)(-1.04611,2.50413)(-1.04155,2.49861)(-1.03633,2.49371)(-1.03054,2.48950)%
(-1.02427,2.48606)(-1.01761,2.48342)(-1.01068,2.48164)(-1.00358,2.48074)(-0.99642,2.48074)%
(-0.98932,2.48164)(-0.98239,2.48342)(-0.97573,2.48606)(-0.96946,2.48950)(-0.96367,2.49371)%
(-0.95845,2.49861)(-0.95389,2.50413)(-0.95005,2.51017)(-0.94700,2.51665)(-0.94479,2.52346)%
(-0.94345,2.53049)(-0.94300,2.53763)%
%
{%
\color[cmyk]{0,1,1,0}%
\polygon*(1.05700,-1.54377)(1.05655,-1.53662)(1.05521,-1.52959)(1.05300,-1.52278)%
(1.04995,-1.51631)(1.04611,-1.51026)(1.04155,-1.50475)(1.03633,-1.49985)(1.03054,-1.49564)%
(1.02427,-1.49219)(1.01761,-1.48956)(1.01068,-1.48778)(1.00358,-1.48688)(0.99642,-1.48688)%
(0.98932,-1.48778)(0.98239,-1.48956)(0.97573,-1.49219)(0.96946,-1.49564)(0.96367,-1.49985)%
(0.95845,-1.50475)(0.95389,-1.51026)(0.95005,-1.51631)(0.94700,-1.52278)(0.94479,-1.52959)%
(0.94345,-1.53662)(0.94300,-1.54377)(0.94345,-1.55091)(0.94479,-1.55794)(0.94700,-1.56475)%
(0.95005,-1.57123)(0.95389,-1.57727)(0.95845,-1.58278)(0.96367,-1.58769)(0.96946,-1.59189)%
(0.97573,-1.59534)(0.98239,-1.59798)(0.98932,-1.59976)(0.99642,-1.60065)(1.00358,-1.60065)%
(1.01068,-1.59976)(1.01761,-1.59798)(1.02427,-1.59534)(1.03054,-1.59189)(1.03633,-1.58769)%
(1.04155,-1.58278)(1.04611,-1.57727)(1.04995,-1.57123)(1.05300,-1.56475)(1.05521,-1.55794)%
(1.05655,-1.55091)(1.05700,-1.54377)(1.05700,-1.54377)}%
\polyline(1.05700,-1.54377)(1.05655,-1.53662)(1.05521,-1.52959)(1.05300,-1.52278)%
(1.04995,-1.51631)(1.04611,-1.51026)(1.04155,-1.50475)(1.03633,-1.49985)(1.03054,-1.49564)%
(1.02427,-1.49219)(1.01761,-1.48956)(1.01068,-1.48778)(1.00358,-1.48688)(0.99642,-1.48688)%
(0.98932,-1.48778)(0.98239,-1.48956)(0.97573,-1.49219)(0.96946,-1.49564)(0.96367,-1.49985)%
(0.95845,-1.50475)(0.95389,-1.51026)(0.95005,-1.51631)(0.94700,-1.52278)(0.94479,-1.52959)%
(0.94345,-1.53662)(0.94300,-1.54377)(0.94345,-1.55091)(0.94479,-1.55794)(0.94700,-1.56475)%
(0.95005,-1.57123)(0.95389,-1.57727)(0.95845,-1.58278)(0.96367,-1.58769)(0.96946,-1.59189)%
(0.97573,-1.59534)(0.98239,-1.59798)(0.98932,-1.59976)(0.99642,-1.60065)(1.00358,-1.60065)%
(1.01068,-1.59976)(1.01761,-1.59798)(1.02427,-1.59534)(1.03054,-1.59189)(1.03633,-1.58769)%
(1.04155,-1.58278)(1.04611,-1.57727)(1.04995,-1.57123)(1.05300,-1.56475)(1.05521,-1.55794)%
(1.05655,-1.55091)(1.05700,-1.54377)%
%
\polyline(-2.13938,0.00000)(4.63725,0.00000)%
%
\polyline(0.00000,-1.85809)(0.00000,3.12000)%
%
\settowidth{\Width}{$x$}\setlength{\Width}{0\Width}%
\settoheight{\Height}{$x$}\settodepth{\Depth}{$x$}\setlength{\Height}{-0.5\Height}\setlength{\Depth}{0.5\Depth}\addtolength{\Height}{\Depth}%
\put(4.7233333,0.0000000){\hspace*{\Width}\raisebox{\Height}{$x$}}%
%
\settowidth{\Width}{$y$}\setlength{\Width}{-0.5\Width}%
\settoheight{\Height}{$y$}\settodepth{\Depth}{$y$}\setlength{\Height}{\Depth}%
\put(0.0000000,3.2033333){\hspace*{\Width}\raisebox{\Height}{$y$}}%
%
\settowidth{\Width}{O}\setlength{\Width}{-1\Width}%
\settoheight{\Height}{O}\settodepth{\Depth}{O}\setlength{\Height}{-\Height}%
\put(-0.0833333,-0.0833333){\hspace*{\Width}\raisebox{\Height}{O}}%
%
\end{picture}}%}
\putnotese{85}{5}{%%% /Users/takatoosetsuo/polytech22.git/205-0926/presen/fig/kyokuchi2.tex 
%%% Generator=graph22207.cdy 
{\unitlength=6mm%
\begin{picture}%
(6.78,4.98)(-2.14,-1.86)%
\linethickness{0.008in}%%
\linethickness{0.012in}%%
\polyline(-1.72510,-1.86000)(-1.64320,-0.75807)(-1.56040,0.15790)(-1.47760,0.89554)%
(-1.39480,1.47347)(-1.31200,1.90920)(-1.22920,2.21912)(-1.14640,2.41859)(-1.06360,2.52194)%
(-0.98080,2.54250)(-0.89800,2.49264)(-0.81520,2.38379)(-0.73240,2.22648)(-0.64960,2.03037)%
(-0.56680,1.80430)(-0.48400,1.55627)(-0.40120,1.29351)(-0.31840,1.02253)(-0.23560,0.74909)%
(-0.15280,0.47830)(-0.07000,0.21458)(0.01280,-0.03823)(0.09560,-0.27693)(0.17840,-0.49882)%
(0.26120,-0.70178)(0.34400,-0.88416)(0.42680,-1.04478)(0.50960,-1.18290)(0.59240,-1.29818)%
(0.67520,-1.39065)(0.75800,-1.46069)(0.84080,-1.50897)(0.92360,-1.53647)(1.00640,-1.54439)%
(1.08920,-1.53416)(1.17200,-1.50740)(1.25480,-1.46586)(1.33760,-1.41145)(1.42040,-1.34615)%
(1.50320,-1.27199)(1.58600,-1.19107)(1.66880,-1.10543)(1.75160,-1.01714)(1.83440,-0.92816)%
(1.91720,-0.84038)(2.00000,-0.75556)%
%
\linethickness{0.008in}%%
{%
\color[cmyk]{0,1,1,0}%
\polygon*(-0.94300,2.53763)(-0.94345,2.54478)(-0.94479,2.55181)(-0.94700,2.55861)%
(-0.95005,2.56509)(-0.95389,2.57114)(-0.95845,2.57665)(-0.96367,2.58155)(-0.96946,2.58576)%
(-0.97573,2.58921)(-0.98239,2.59184)(-0.98932,2.59362)(-0.99642,2.59452)(-1.00358,2.59452)%
(-1.01068,2.59362)(-1.01761,2.59184)(-1.02427,2.58921)(-1.03054,2.58576)(-1.03633,2.58155)%
(-1.04155,2.57665)(-1.04611,2.57114)(-1.04995,2.56509)(-1.05300,2.55861)(-1.05521,2.55181)%
(-1.05655,2.54478)(-1.05700,2.53763)(-1.05655,2.53049)(-1.05521,2.52346)(-1.05300,2.51665)%
(-1.04995,2.51017)(-1.04611,2.50413)(-1.04155,2.49861)(-1.03633,2.49371)(-1.03054,2.48950)%
(-1.02427,2.48606)(-1.01761,2.48342)(-1.01068,2.48164)(-1.00358,2.48074)(-0.99642,2.48074)%
(-0.98932,2.48164)(-0.98239,2.48342)(-0.97573,2.48606)(-0.96946,2.48950)(-0.96367,2.49371)%
(-0.95845,2.49861)(-0.95389,2.50413)(-0.95005,2.51017)(-0.94700,2.51665)(-0.94479,2.52346)%
(-0.94345,2.53049)(-0.94300,2.53763)(-0.94300,2.53763)}%
\polyline(-0.94300,2.53763)(-0.94345,2.54478)(-0.94479,2.55181)(-0.94700,2.55861)%
(-0.95005,2.56509)(-0.95389,2.57114)(-0.95845,2.57665)(-0.96367,2.58155)(-0.96946,2.58576)%
(-0.97573,2.58921)(-0.98239,2.59184)(-0.98932,2.59362)(-0.99642,2.59452)(-1.00358,2.59452)%
(-1.01068,2.59362)(-1.01761,2.59184)(-1.02427,2.58921)(-1.03054,2.58576)(-1.03633,2.58155)%
(-1.04155,2.57665)(-1.04611,2.57114)(-1.04995,2.56509)(-1.05300,2.55861)(-1.05521,2.55181)%
(-1.05655,2.54478)(-1.05700,2.53763)(-1.05655,2.53049)(-1.05521,2.52346)(-1.05300,2.51665)%
(-1.04995,2.51017)(-1.04611,2.50413)(-1.04155,2.49861)(-1.03633,2.49371)(-1.03054,2.48950)%
(-1.02427,2.48606)(-1.01761,2.48342)(-1.01068,2.48164)(-1.00358,2.48074)(-0.99642,2.48074)%
(-0.98932,2.48164)(-0.98239,2.48342)(-0.97573,2.48606)(-0.96946,2.48950)(-0.96367,2.49371)%
(-0.95845,2.49861)(-0.95389,2.50413)(-0.95005,2.51017)(-0.94700,2.51665)(-0.94479,2.52346)%
(-0.94345,2.53049)(-0.94300,2.53763)%
%
{%
\color[cmyk]{0,1,1,0}%
\polygon*(1.05700,-1.54377)(1.05655,-1.53662)(1.05521,-1.52959)(1.05300,-1.52278)%
(1.04995,-1.51631)(1.04611,-1.51026)(1.04155,-1.50475)(1.03633,-1.49985)(1.03054,-1.49564)%
(1.02427,-1.49219)(1.01761,-1.48956)(1.01068,-1.48778)(1.00358,-1.48688)(0.99642,-1.48688)%
(0.98932,-1.48778)(0.98239,-1.48956)(0.97573,-1.49219)(0.96946,-1.49564)(0.96367,-1.49985)%
(0.95845,-1.50475)(0.95389,-1.51026)(0.95005,-1.51631)(0.94700,-1.52278)(0.94479,-1.52959)%
(0.94345,-1.53662)(0.94300,-1.54377)(0.94345,-1.55091)(0.94479,-1.55794)(0.94700,-1.56475)%
(0.95005,-1.57123)(0.95389,-1.57727)(0.95845,-1.58278)(0.96367,-1.58769)(0.96946,-1.59189)%
(0.97573,-1.59534)(0.98239,-1.59798)(0.98932,-1.59976)(0.99642,-1.60065)(1.00358,-1.60065)%
(1.01068,-1.59976)(1.01761,-1.59798)(1.02427,-1.59534)(1.03054,-1.59189)(1.03633,-1.58769)%
(1.04155,-1.58278)(1.04611,-1.57727)(1.04995,-1.57123)(1.05300,-1.56475)(1.05521,-1.55794)%
(1.05655,-1.55091)(1.05700,-1.54377)(1.05700,-1.54377)}%
\polyline(1.05700,-1.54377)(1.05655,-1.53662)(1.05521,-1.52959)(1.05300,-1.52278)%
(1.04995,-1.51631)(1.04611,-1.51026)(1.04155,-1.50475)(1.03633,-1.49985)(1.03054,-1.49564)%
(1.02427,-1.49219)(1.01761,-1.48956)(1.01068,-1.48778)(1.00358,-1.48688)(0.99642,-1.48688)%
(0.98932,-1.48778)(0.98239,-1.48956)(0.97573,-1.49219)(0.96946,-1.49564)(0.96367,-1.49985)%
(0.95845,-1.50475)(0.95389,-1.51026)(0.95005,-1.51631)(0.94700,-1.52278)(0.94479,-1.52959)%
(0.94345,-1.53662)(0.94300,-1.54377)(0.94345,-1.55091)(0.94479,-1.55794)(0.94700,-1.56475)%
(0.95005,-1.57123)(0.95389,-1.57727)(0.95845,-1.58278)(0.96367,-1.58769)(0.96946,-1.59189)%
(0.97573,-1.59534)(0.98239,-1.59798)(0.98932,-1.59976)(0.99642,-1.60065)(1.00358,-1.60065)%
(1.01068,-1.59976)(1.01761,-1.59798)(1.02427,-1.59534)(1.03054,-1.59189)(1.03633,-1.58769)%
(1.04155,-1.58278)(1.04611,-1.57727)(1.04995,-1.57123)(1.05300,-1.56475)(1.05521,-1.55794)%
(1.05655,-1.55091)(1.05700,-1.54377)%
%
\linethickness{0.012in}%%
\polyline(2.00000,-0.75556)(2.05280,-0.70375)(2.10560,-0.65417)(2.15840,-0.60717)%
(2.21120,-0.56305)(2.26400,-0.52207)(2.31680,-0.48445)(2.36960,-0.45036)(2.42240,-0.41992)%
(2.47520,-0.39321)(2.52800,-0.37022)(2.58080,-0.35090)(2.63360,-0.33515)(2.68640,-0.32278)%
(2.73920,-0.31354)(2.79200,-0.30709)(2.84480,-0.30304)(2.89760,-0.30090)(2.95040,-0.30011)%
(3.00320,-0.30000)(3.05600,-0.29984)(3.10880,-0.29878)(3.16160,-0.29590)(3.21440,-0.29015)%
(3.26720,-0.28040)(3.32000,-0.26541)(3.37280,-0.24381)(3.42560,-0.21414)(3.47840,-0.17482)%
(3.53120,-0.12413)(3.58400,-0.06027)(3.63680,0.01874)(3.68960,0.11497)(3.74240,0.23064)%
(3.79520,0.36809)(3.84800,0.52983)(3.90080,0.71849)(3.95360,0.93684)(4.00640,1.18782)%
(4.05920,1.47450)(4.11200,1.80013)(4.16480,2.16810)(4.21760,2.58197)(4.27040,3.04547)%
(4.27801,3.12000)%
%
\linethickness{0.008in}%%
{%
\color[cmyk]{0,1,1,0}%
\polygon*(3.05700,-0.30000)(3.05655,-0.29286)(3.05521,-0.28583)(3.05300,-0.27902)%
(3.04995,-0.27254)(3.04611,-0.26650)(3.04155,-0.26099)(3.03633,-0.25609)(3.03054,-0.25188)%
(3.02427,-0.24843)(3.01761,-0.24579)(3.01068,-0.24401)(3.00358,-0.24312)(2.99642,-0.24312)%
(2.98932,-0.24401)(2.98239,-0.24579)(2.97573,-0.24843)(2.96946,-0.25188)(2.96367,-0.25609)%
(2.95845,-0.26099)(2.95389,-0.26650)(2.95005,-0.27254)(2.94700,-0.27902)(2.94479,-0.28583)%
(2.94345,-0.29286)(2.94300,-0.30000)(2.94345,-0.30715)(2.94479,-0.31418)(2.94700,-0.32099)%
(2.95005,-0.32746)(2.95389,-0.33351)(2.95845,-0.33902)(2.96367,-0.34392)(2.96946,-0.34813)%
(2.97573,-0.35158)(2.98239,-0.35421)(2.98932,-0.35599)(2.99642,-0.35689)(3.00358,-0.35689)%
(3.01068,-0.35599)(3.01761,-0.35421)(3.02427,-0.35158)(3.03054,-0.34813)(3.03633,-0.34392)%
(3.04155,-0.33902)(3.04611,-0.33351)(3.04995,-0.32746)(3.05300,-0.32099)(3.05521,-0.31418)%
(3.05655,-0.30715)(3.05700,-0.30000)(3.05700,-0.30000)}%
\polyline(3.05700,-0.30000)(3.05655,-0.29286)(3.05521,-0.28583)(3.05300,-0.27902)%
(3.04995,-0.27254)(3.04611,-0.26650)(3.04155,-0.26099)(3.03633,-0.25609)(3.03054,-0.25188)%
(3.02427,-0.24843)(3.01761,-0.24579)(3.01068,-0.24401)(3.00358,-0.24312)(2.99642,-0.24312)%
(2.98932,-0.24401)(2.98239,-0.24579)(2.97573,-0.24843)(2.96946,-0.25188)(2.96367,-0.25609)%
(2.95845,-0.26099)(2.95389,-0.26650)(2.95005,-0.27254)(2.94700,-0.27902)(2.94479,-0.28583)%
(2.94345,-0.29286)(2.94300,-0.30000)(2.94345,-0.30715)(2.94479,-0.31418)(2.94700,-0.32099)%
(2.95005,-0.32746)(2.95389,-0.33351)(2.95845,-0.33902)(2.96367,-0.34392)(2.96946,-0.34813)%
(2.97573,-0.35158)(2.98239,-0.35421)(2.98932,-0.35599)(2.99642,-0.35689)(3.00358,-0.35689)%
(3.01068,-0.35599)(3.01761,-0.35421)(3.02427,-0.35158)(3.03054,-0.34813)(3.03633,-0.34392)%
(3.04155,-0.33902)(3.04611,-0.33351)(3.04995,-0.32746)(3.05300,-0.32099)(3.05521,-0.31418)%
(3.05655,-0.30715)(3.05700,-0.30000)%
%
\polyline(-2.13938,0.00000)(4.63725,0.00000)%
%
\polyline(0.00000,-1.85809)(0.00000,3.12000)%
%
\settowidth{\Width}{$x$}\setlength{\Width}{0\Width}%
\settoheight{\Height}{$x$}\settodepth{\Depth}{$x$}\setlength{\Height}{-0.5\Height}\setlength{\Depth}{0.5\Depth}\addtolength{\Height}{\Depth}%
\put(4.7233333,0.0000000){\hspace*{\Width}\raisebox{\Height}{$x$}}%
%
\settowidth{\Width}{$y$}\setlength{\Width}{-0.5\Width}%
\settoheight{\Height}{$y$}\settodepth{\Depth}{$y$}\setlength{\Height}{\Depth}%
\put(0.0000000,3.2033333){\hspace*{\Width}\raisebox{\Height}{$y$}}%
%
\settowidth{\Width}{O}\setlength{\Width}{-1\Width}%
\settoheight{\Height}{O}\settodepth{\Depth}{O}\setlength{\Height}{-\Height}%
\put(-0.0833333,-0.0833333){\hspace*{\Width}\raisebox{\Height}{O}}%
%
\end{picture}}%}
\putnoten{92}{8}{\color{red}\normalsize 極大}
\putnotes{104}{34}{\color{red}\normalsize 極小}
\putnotes{116}{26}{\color{red}\normalsize 極値でない}
\end{layer}

\begin{itemize}
\item
極大(極小)となる点を極値点\vspace{-2mm}
\item
極値点では $y'=0$\\
\hspace*{1zw}極大点 増加から減少に変わる\\
\hspace*{1zw}極小点 減少から増加に変わる\vspace{-2mm}
\item
$y'=0$でも極値点でない点もある\vspace{-2mm}
\item
[例]$y=x^3-3x^2$\\
 「導関数の意味」を用いてグラフを描けばよい.\\
 計算では $y'=3x^2-6x=3x(x-2)$\\
 $y'=0$となる点は,$x(x-2)=0$より $x=0,\ 2$
\end{itemize}

\newslide{極値点(課題1)}

\vspace*{18mm}


\begin{layer}{120}{0}
\putnotew{96}{73}{\hyperlink{para3pg9}{\fbox{\Ctab{2.5mm}{\scalebox{1}{\scriptsize $\mathstrut||\!\lhd$}}}}}
\putnotew{125}{73}{\hyperlink{para5pg1}{\fbox{\Ctab{2.5mm}{\scalebox{1}{\scriptsize $\mathstrut \!\rhd\!\!||$}}}}}
\putnotee{126}{73}{\scriptsize\color{blue} 1/1}
\end{layer}

\slidepage
\vspace{1zw}

アプリ「導関数の意味」を用いよ
\begin{itemize}
\item
[課題]\monban 次の関数について,極値点を求めよ.\seteda{70}\\
\eda{$y=x^3-6x^2+9x$}\\
\eda{$y=x^4-2x^2+1$}\\
\eda{$y=x^2 e^{x}$}\\
\end{itemize}
%%%%%%%%%%%%%

%%%%%%%%%%%%%%%%%%%%


\newslide{極値点(課題2)}

\vspace*{18mm}


\begin{layer}{120}{0}
\putnotew{96}{73}{\hyperlink{para4pg1}{\fbox{\Ctab{2.5mm}{\scalebox{1}{\scriptsize $\mathstrut||\!\lhd$}}}}}
\putnotew{125}{73}{\hyperlink{para6pg1}{\fbox{\Ctab{2.5mm}{\scalebox{1}{\scriptsize $\mathstrut \!\rhd\!\!||$}}}}}
\putnotee{126}{73}{\scriptsize\color{blue} 1/1}
\end{layer}

\slidepage
\vspace{1zw}

アプリ「導関数の意味」を用いよ
\begin{itemize}
\item
[課題]\monban 次の関数について,極値点を求めよ.\seteda{70}\\
\eda{$y=\log x-x\ (0<x\leqq 3)$}\\
\hspace*{4zw}入力書式 \verb|log(x)-x^(2)x=+,5|\\
\eda{$y=\sin x+\cos x\ (0 \leqq x \leqq 2\pi)$}\\
\hspace*{4zw}入力書式 \verb|sin(x)+cos(x)x=0,2pi|\\
\eda{$y=\sin^2 x\ (0 \leqq x \leqq \pi)$}\\
\hspace*{4zw}入力書式 \verb|sin(2,x)x=0,pi|\\
\hspace*{4zw}\verb|[2][3]|解は$\pi$を整数で割った形
\end{itemize}
%%%%%%%%%%%%%

%%%%%%%%%%%%%%%%%%%%


\newslide{増減表}

\vspace*{18mm}


\begin{layer}{120}{0}
\putnotew{96}{73}{\hyperlink{para5pg1}{\fbox{\Ctab{2.5mm}{\scalebox{1}{\scriptsize $\mathstrut||\!\lhd$}}}}}
\putnotew{101}{73}{\hyperlink{para6pg1}{\fbox{\Ctab{2.5mm}{\scalebox{1}{\scriptsize $\mathstrut|\!\lhd$}}}}}
\putnotew{108}{73}{\hyperlink{para6pg10}{\fbox{\Ctab{4.5mm}{\scalebox{1}{\scriptsize $\mathstrut\!\!\lhd\!\!$}}}}}
\putnotew{115}{73}{\hyperlink{para6pg11}{\fbox{\Ctab{4.5mm}{\scalebox{1}{\scriptsize $\mathstrut\!\rhd\!$}}}}}
\putnotew{120}{73}{\hyperlink{para6pg11}{\fbox{\Ctab{2.5mm}{\scalebox{1}{\scriptsize $\mathstrut \!\rhd\!\!|$}}}}}
\putnotew{125}{73}{\hyperlink{para7pg1}{\fbox{\Ctab{2.5mm}{\scalebox{1}{\scriptsize $\mathstrut \!\rhd\!\!||$}}}}}
\putnotee{126}{73}{\scriptsize\color{blue} 11/11}
\end{layer}

\slidepage

\begin{layer}{120}{0}
\putnotese{80}{53}{\scalebox{1.5}{%%% /Users/takatoosetsuo/polytech22.git/205-0926/presen/fig/zougenhyo1.tex 
%%% Generator=graph22207.cdy 
{\unitlength=6mm%
\begin{picture}%
(4.8,1.8)(0,0)%
\linethickness{0.008in}%%
\polyline(0.00000,1.80000)(0.00000,0.00000)%
%
\polyline(0.80000,1.80000)(0.80000,0.00000)%
%
\polyline(1.60000,1.80000)(1.60000,0.00000)%
%
\polyline(2.40000,1.80000)(2.40000,0.00000)%
%
\polyline(3.20000,1.80000)(3.20000,0.00000)%
%
\polyline(4.00000,1.80000)(4.00000,0.00000)%
%
\polyline(4.80000,1.80000)(4.80000,0.00000)%
%
\polyline(0.00000,1.80000)(4.80000,1.80000)%
%
\polyline(0.00000,1.20000)(4.80000,1.20000)%
%
\polyline(0.00000,0.60000)(4.80000,0.60000)%
%
\polyline(0.00000,0.00000)(4.80000,0.00000)%
%
\small%
\settowidth{\Width}{$x$}\setlength{\Width}{-0.5\Width}%
\settoheight{\Height}{$x$}\settodepth{\Depth}{$x$}\setlength{\Height}{-0.5\Height}\setlength{\Depth}{0.5\Depth}\addtolength{\Height}{\Depth}%
\put(0.4000000,1.5000000){\hspace*{\Width}\raisebox{\Height}{$x$}}%
%
\settowidth{\Width}{$y'$}\setlength{\Width}{-0.5\Width}%
\settoheight{\Height}{$y'$}\settodepth{\Depth}{$y'$}\setlength{\Height}{-0.5\Height}\setlength{\Depth}{0.5\Depth}\addtolength{\Height}{\Depth}%
\put(0.4000000,0.9000000){\hspace*{\Width}\raisebox{\Height}{$y'$}}%
%
\settowidth{\Width}{$y$}\setlength{\Width}{-0.5\Width}%
\settoheight{\Height}{$y$}\settodepth{\Depth}{$y$}\setlength{\Height}{-0.5\Height}\setlength{\Depth}{0.5\Depth}\addtolength{\Height}{\Depth}%
\put(0.4000000,0.3000000){\hspace*{\Width}\raisebox{\Height}{$y$}}%
%
\end{picture}}%}}
\putnotese{80}{53}{\scalebox{1.5}{%%% /Users/takatoosetsuo/polytech22.git/205-0926/presen/fig/zougenhyo2.tex 
%%% Generator=graph22207.cdy 
{\unitlength=6mm%
\begin{picture}%
(4.8,1.8)(0,0)%
\linethickness{0.008in}%%
\polyline(0.00000,1.80000)(0.00000,0.00000)%
%
\polyline(0.80000,1.80000)(0.80000,0.00000)%
%
\polyline(1.60000,1.80000)(1.60000,0.00000)%
%
\polyline(2.40000,1.80000)(2.40000,0.00000)%
%
\polyline(3.20000,1.80000)(3.20000,0.00000)%
%
\polyline(4.00000,1.80000)(4.00000,0.00000)%
%
\polyline(4.80000,1.80000)(4.80000,0.00000)%
%
\polyline(0.00000,1.80000)(4.80000,1.80000)%
%
\polyline(0.00000,1.20000)(4.80000,1.20000)%
%
\polyline(0.00000,0.60000)(4.80000,0.60000)%
%
\polyline(0.00000,0.00000)(4.80000,0.00000)%
%
\small%
\settowidth{\Width}{$x$}\setlength{\Width}{-0.5\Width}%
\settoheight{\Height}{$x$}\settodepth{\Depth}{$x$}\setlength{\Height}{-0.5\Height}\setlength{\Depth}{0.5\Depth}\addtolength{\Height}{\Depth}%
\put(0.4000000,1.5000000){\hspace*{\Width}\raisebox{\Height}{$x$}}%
%
\settowidth{\Width}{$y'$}\setlength{\Width}{-0.5\Width}%
\settoheight{\Height}{$y'$}\settodepth{\Depth}{$y'$}\setlength{\Height}{-0.5\Height}\setlength{\Depth}{0.5\Depth}\addtolength{\Height}{\Depth}%
\put(0.4000000,0.9000000){\hspace*{\Width}\raisebox{\Height}{$y'$}}%
%
\settowidth{\Width}{$y$}\setlength{\Width}{-0.5\Width}%
\settoheight{\Height}{$y$}\settodepth{\Depth}{$y$}\setlength{\Height}{-0.5\Height}\setlength{\Depth}{0.5\Depth}\addtolength{\Height}{\Depth}%
\put(0.4000000,0.3000000){\hspace*{\Width}\raisebox{\Height}{$y$}}%
%
\settowidth{\Width}{$$}\setlength{\Width}{-0.5\Width}%
\settoheight{\Height}{$$}\settodepth{\Depth}{$$}\setlength{\Height}{-0.5\Height}\setlength{\Depth}{0.5\Depth}\addtolength{\Height}{\Depth}%
\put(0.4000000,1.5000000){\hspace*{\Width}\raisebox{\Height}{$$}}%
%
\settowidth{\Width}{$$}\setlength{\Width}{-0.5\Width}%
\settoheight{\Height}{$$}\settodepth{\Depth}{$$}\setlength{\Height}{-0.5\Height}\setlength{\Depth}{0.5\Depth}\addtolength{\Height}{\Depth}%
\put(1.2000000,1.5000000){\hspace*{\Width}\raisebox{\Height}{$$}}%
%
\settowidth{\Width}{$0$}\setlength{\Width}{-0.5\Width}%
\settoheight{\Height}{$0$}\settodepth{\Depth}{$0$}\setlength{\Height}{-0.5\Height}\setlength{\Depth}{0.5\Depth}\addtolength{\Height}{\Depth}%
\put(2.0000000,1.5000000){\hspace*{\Width}\raisebox{\Height}{$0$}}%
%
\settowidth{\Width}{$$}\setlength{\Width}{-0.5\Width}%
\settoheight{\Height}{$$}\settodepth{\Depth}{$$}\setlength{\Height}{-0.5\Height}\setlength{\Depth}{0.5\Depth}\addtolength{\Height}{\Depth}%
\put(2.8000000,1.5000000){\hspace*{\Width}\raisebox{\Height}{$$}}%
%
\settowidth{\Width}{$2$}\setlength{\Width}{-0.5\Width}%
\settoheight{\Height}{$2$}\settodepth{\Depth}{$2$}\setlength{\Height}{-0.5\Height}\setlength{\Depth}{0.5\Depth}\addtolength{\Height}{\Depth}%
\put(3.6000000,1.5000000){\hspace*{\Width}\raisebox{\Height}{$2$}}%
%
\settowidth{\Width}{$$}\setlength{\Width}{-0.5\Width}%
\settoheight{\Height}{$$}\settodepth{\Depth}{$$}\setlength{\Height}{-0.5\Height}\setlength{\Depth}{0.5\Depth}\addtolength{\Height}{\Depth}%
\put(4.4000000,1.5000000){\hspace*{\Width}\raisebox{\Height}{$$}}%
%
\end{picture}}%}}
\putnotese{80}{53}{\scalebox{1.5}{%%% /Users/takatoosetsuo/polytech22.git/205-0926/presen/fig/zougenhyo3.tex 
%%% Generator=graph22207.cdy 
{\unitlength=6mm%
\begin{picture}%
(4.8,1.8)(0,0)%
\linethickness{0.008in}%%
\polyline(0.00000,1.80000)(0.00000,0.00000)%
%
\polyline(0.80000,1.80000)(0.80000,0.00000)%
%
\polyline(1.60000,1.80000)(1.60000,0.00000)%
%
\polyline(2.40000,1.80000)(2.40000,0.00000)%
%
\polyline(3.20000,1.80000)(3.20000,0.00000)%
%
\polyline(4.00000,1.80000)(4.00000,0.00000)%
%
\polyline(4.80000,1.80000)(4.80000,0.00000)%
%
\polyline(0.00000,1.80000)(4.80000,1.80000)%
%
\polyline(0.00000,1.20000)(4.80000,1.20000)%
%
\polyline(0.00000,0.60000)(4.80000,0.60000)%
%
\polyline(0.00000,0.00000)(4.80000,0.00000)%
%
\small%
\settowidth{\Width}{$x$}\setlength{\Width}{-0.5\Width}%
\settoheight{\Height}{$x$}\settodepth{\Depth}{$x$}\setlength{\Height}{-0.5\Height}\setlength{\Depth}{0.5\Depth}\addtolength{\Height}{\Depth}%
\put(0.4000000,1.5000000){\hspace*{\Width}\raisebox{\Height}{$x$}}%
%
\settowidth{\Width}{$y'$}\setlength{\Width}{-0.5\Width}%
\settoheight{\Height}{$y'$}\settodepth{\Depth}{$y'$}\setlength{\Height}{-0.5\Height}\setlength{\Depth}{0.5\Depth}\addtolength{\Height}{\Depth}%
\put(0.4000000,0.9000000){\hspace*{\Width}\raisebox{\Height}{$y'$}}%
%
\settowidth{\Width}{$y$}\setlength{\Width}{-0.5\Width}%
\settoheight{\Height}{$y$}\settodepth{\Depth}{$y$}\setlength{\Height}{-0.5\Height}\setlength{\Depth}{0.5\Depth}\addtolength{\Height}{\Depth}%
\put(0.4000000,0.3000000){\hspace*{\Width}\raisebox{\Height}{$y$}}%
%
\settowidth{\Width}{$$}\setlength{\Width}{-0.5\Width}%
\settoheight{\Height}{$$}\settodepth{\Depth}{$$}\setlength{\Height}{-0.5\Height}\setlength{\Depth}{0.5\Depth}\addtolength{\Height}{\Depth}%
\put(0.4000000,1.5000000){\hspace*{\Width}\raisebox{\Height}{$$}}%
%
\settowidth{\Width}{$$}\setlength{\Width}{-0.5\Width}%
\settoheight{\Height}{$$}\settodepth{\Depth}{$$}\setlength{\Height}{-0.5\Height}\setlength{\Depth}{0.5\Depth}\addtolength{\Height}{\Depth}%
\put(1.2000000,1.5000000){\hspace*{\Width}\raisebox{\Height}{$$}}%
%
\settowidth{\Width}{$0$}\setlength{\Width}{-0.5\Width}%
\settoheight{\Height}{$0$}\settodepth{\Depth}{$0$}\setlength{\Height}{-0.5\Height}\setlength{\Depth}{0.5\Depth}\addtolength{\Height}{\Depth}%
\put(2.0000000,1.5000000){\hspace*{\Width}\raisebox{\Height}{$0$}}%
%
\settowidth{\Width}{$$}\setlength{\Width}{-0.5\Width}%
\settoheight{\Height}{$$}\settodepth{\Depth}{$$}\setlength{\Height}{-0.5\Height}\setlength{\Depth}{0.5\Depth}\addtolength{\Height}{\Depth}%
\put(2.8000000,1.5000000){\hspace*{\Width}\raisebox{\Height}{$$}}%
%
\settowidth{\Width}{$2$}\setlength{\Width}{-0.5\Width}%
\settoheight{\Height}{$2$}\settodepth{\Depth}{$2$}\setlength{\Height}{-0.5\Height}\setlength{\Depth}{0.5\Depth}\addtolength{\Height}{\Depth}%
\put(3.6000000,1.5000000){\hspace*{\Width}\raisebox{\Height}{$2$}}%
%
\settowidth{\Width}{$$}\setlength{\Width}{-0.5\Width}%
\settoheight{\Height}{$$}\settodepth{\Depth}{$$}\setlength{\Height}{-0.5\Height}\setlength{\Depth}{0.5\Depth}\addtolength{\Height}{\Depth}%
\put(4.4000000,1.5000000){\hspace*{\Width}\raisebox{\Height}{$$}}%
%
\settowidth{\Width}{$$}\setlength{\Width}{-0.5\Width}%
\settoheight{\Height}{$$}\settodepth{\Depth}{$$}\setlength{\Height}{-0.5\Height}\setlength{\Depth}{0.5\Depth}\addtolength{\Height}{\Depth}%
\put(0.4000000,1.5000000){\hspace*{\Width}\raisebox{\Height}{$$}}%
%
\settowidth{\Width}{$\cdots$}\setlength{\Width}{-0.5\Width}%
\settoheight{\Height}{$\cdots$}\settodepth{\Depth}{$\cdots$}\setlength{\Height}{-0.5\Height}\setlength{\Depth}{0.5\Depth}\addtolength{\Height}{\Depth}%
\put(1.2000000,1.5000000){\hspace*{\Width}\raisebox{\Height}{$\cdots$}}%
%
\settowidth{\Width}{$$}\setlength{\Width}{-0.5\Width}%
\settoheight{\Height}{$$}\settodepth{\Depth}{$$}\setlength{\Height}{-0.5\Height}\setlength{\Depth}{0.5\Depth}\addtolength{\Height}{\Depth}%
\put(2.0000000,1.5000000){\hspace*{\Width}\raisebox{\Height}{$$}}%
%
\settowidth{\Width}{$\cdots$}\setlength{\Width}{-0.5\Width}%
\settoheight{\Height}{$\cdots$}\settodepth{\Depth}{$\cdots$}\setlength{\Height}{-0.5\Height}\setlength{\Depth}{0.5\Depth}\addtolength{\Height}{\Depth}%
\put(2.8000000,1.5000000){\hspace*{\Width}\raisebox{\Height}{$\cdots$}}%
%
\settowidth{\Width}{$$}\setlength{\Width}{-0.5\Width}%
\settoheight{\Height}{$$}\settodepth{\Depth}{$$}\setlength{\Height}{-0.5\Height}\setlength{\Depth}{0.5\Depth}\addtolength{\Height}{\Depth}%
\put(3.6000000,1.5000000){\hspace*{\Width}\raisebox{\Height}{$$}}%
%
\settowidth{\Width}{$\cdots$}\setlength{\Width}{-0.5\Width}%
\settoheight{\Height}{$\cdots$}\settodepth{\Depth}{$\cdots$}\setlength{\Height}{-0.5\Height}\setlength{\Depth}{0.5\Depth}\addtolength{\Height}{\Depth}%
\put(4.4000000,1.5000000){\hspace*{\Width}\raisebox{\Height}{$\cdots$}}%
%
\end{picture}}%}}
\putnotese{80}{53}{\scalebox{1.5}{%%% /Users/takatoosetsuo/polytech22.git/205-0926/presen/fig/zougenhyo4.tex 
%%% Generator=graph22207.cdy 
{\unitlength=6mm%
\begin{picture}%
(4.8,1.8)(0,0)%
\linethickness{0.008in}%%
\polyline(0.00000,1.80000)(0.00000,0.00000)%
%
\polyline(0.80000,1.80000)(0.80000,0.00000)%
%
\polyline(1.60000,1.80000)(1.60000,0.00000)%
%
\polyline(2.40000,1.80000)(2.40000,0.00000)%
%
\polyline(3.20000,1.80000)(3.20000,0.00000)%
%
\polyline(4.00000,1.80000)(4.00000,0.00000)%
%
\polyline(4.80000,1.80000)(4.80000,0.00000)%
%
\polyline(0.00000,1.80000)(4.80000,1.80000)%
%
\polyline(0.00000,1.20000)(4.80000,1.20000)%
%
\polyline(0.00000,0.60000)(4.80000,0.60000)%
%
\polyline(0.00000,0.00000)(4.80000,0.00000)%
%
\small%
\settowidth{\Width}{$x$}\setlength{\Width}{-0.5\Width}%
\settoheight{\Height}{$x$}\settodepth{\Depth}{$x$}\setlength{\Height}{-0.5\Height}\setlength{\Depth}{0.5\Depth}\addtolength{\Height}{\Depth}%
\put(0.4000000,1.5000000){\hspace*{\Width}\raisebox{\Height}{$x$}}%
%
\settowidth{\Width}{$y'$}\setlength{\Width}{-0.5\Width}%
\settoheight{\Height}{$y'$}\settodepth{\Depth}{$y'$}\setlength{\Height}{-0.5\Height}\setlength{\Depth}{0.5\Depth}\addtolength{\Height}{\Depth}%
\put(0.4000000,0.9000000){\hspace*{\Width}\raisebox{\Height}{$y'$}}%
%
\settowidth{\Width}{$y$}\setlength{\Width}{-0.5\Width}%
\settoheight{\Height}{$y$}\settodepth{\Depth}{$y$}\setlength{\Height}{-0.5\Height}\setlength{\Depth}{0.5\Depth}\addtolength{\Height}{\Depth}%
\put(0.4000000,0.3000000){\hspace*{\Width}\raisebox{\Height}{$y$}}%
%
\settowidth{\Width}{$$}\setlength{\Width}{-0.5\Width}%
\settoheight{\Height}{$$}\settodepth{\Depth}{$$}\setlength{\Height}{-0.5\Height}\setlength{\Depth}{0.5\Depth}\addtolength{\Height}{\Depth}%
\put(0.4000000,1.5000000){\hspace*{\Width}\raisebox{\Height}{$$}}%
%
\settowidth{\Width}{$$}\setlength{\Width}{-0.5\Width}%
\settoheight{\Height}{$$}\settodepth{\Depth}{$$}\setlength{\Height}{-0.5\Height}\setlength{\Depth}{0.5\Depth}\addtolength{\Height}{\Depth}%
\put(1.2000000,1.5000000){\hspace*{\Width}\raisebox{\Height}{$$}}%
%
\settowidth{\Width}{$0$}\setlength{\Width}{-0.5\Width}%
\settoheight{\Height}{$0$}\settodepth{\Depth}{$0$}\setlength{\Height}{-0.5\Height}\setlength{\Depth}{0.5\Depth}\addtolength{\Height}{\Depth}%
\put(2.0000000,1.5000000){\hspace*{\Width}\raisebox{\Height}{$0$}}%
%
\settowidth{\Width}{$$}\setlength{\Width}{-0.5\Width}%
\settoheight{\Height}{$$}\settodepth{\Depth}{$$}\setlength{\Height}{-0.5\Height}\setlength{\Depth}{0.5\Depth}\addtolength{\Height}{\Depth}%
\put(2.8000000,1.5000000){\hspace*{\Width}\raisebox{\Height}{$$}}%
%
\settowidth{\Width}{$2$}\setlength{\Width}{-0.5\Width}%
\settoheight{\Height}{$2$}\settodepth{\Depth}{$2$}\setlength{\Height}{-0.5\Height}\setlength{\Depth}{0.5\Depth}\addtolength{\Height}{\Depth}%
\put(3.6000000,1.5000000){\hspace*{\Width}\raisebox{\Height}{$2$}}%
%
\settowidth{\Width}{$$}\setlength{\Width}{-0.5\Width}%
\settoheight{\Height}{$$}\settodepth{\Depth}{$$}\setlength{\Height}{-0.5\Height}\setlength{\Depth}{0.5\Depth}\addtolength{\Height}{\Depth}%
\put(4.4000000,1.5000000){\hspace*{\Width}\raisebox{\Height}{$$}}%
%
\settowidth{\Width}{$$}\setlength{\Width}{-0.5\Width}%
\settoheight{\Height}{$$}\settodepth{\Depth}{$$}\setlength{\Height}{-0.5\Height}\setlength{\Depth}{0.5\Depth}\addtolength{\Height}{\Depth}%
\put(0.4000000,1.5000000){\hspace*{\Width}\raisebox{\Height}{$$}}%
%
\settowidth{\Width}{$\cdots$}\setlength{\Width}{-0.5\Width}%
\settoheight{\Height}{$\cdots$}\settodepth{\Depth}{$\cdots$}\setlength{\Height}{-0.5\Height}\setlength{\Depth}{0.5\Depth}\addtolength{\Height}{\Depth}%
\put(1.2000000,1.5000000){\hspace*{\Width}\raisebox{\Height}{$\cdots$}}%
%
\settowidth{\Width}{$$}\setlength{\Width}{-0.5\Width}%
\settoheight{\Height}{$$}\settodepth{\Depth}{$$}\setlength{\Height}{-0.5\Height}\setlength{\Depth}{0.5\Depth}\addtolength{\Height}{\Depth}%
\put(2.0000000,1.5000000){\hspace*{\Width}\raisebox{\Height}{$$}}%
%
\settowidth{\Width}{$\cdots$}\setlength{\Width}{-0.5\Width}%
\settoheight{\Height}{$\cdots$}\settodepth{\Depth}{$\cdots$}\setlength{\Height}{-0.5\Height}\setlength{\Depth}{0.5\Depth}\addtolength{\Height}{\Depth}%
\put(2.8000000,1.5000000){\hspace*{\Width}\raisebox{\Height}{$\cdots$}}%
%
\settowidth{\Width}{$$}\setlength{\Width}{-0.5\Width}%
\settoheight{\Height}{$$}\settodepth{\Depth}{$$}\setlength{\Height}{-0.5\Height}\setlength{\Depth}{0.5\Depth}\addtolength{\Height}{\Depth}%
\put(3.6000000,1.5000000){\hspace*{\Width}\raisebox{\Height}{$$}}%
%
\settowidth{\Width}{$\cdots$}\setlength{\Width}{-0.5\Width}%
\settoheight{\Height}{$\cdots$}\settodepth{\Depth}{$\cdots$}\setlength{\Height}{-0.5\Height}\setlength{\Depth}{0.5\Depth}\addtolength{\Height}{\Depth}%
\put(4.4000000,1.5000000){\hspace*{\Width}\raisebox{\Height}{$\cdots$}}%
%
\settowidth{\Width}{$$}\setlength{\Width}{-0.5\Width}%
\settoheight{\Height}{$$}\settodepth{\Depth}{$$}\setlength{\Height}{-0.5\Height}\setlength{\Depth}{0.5\Depth}\addtolength{\Height}{\Depth}%
\put(0.4000000,0.9000000){\hspace*{\Width}\raisebox{\Height}{$$}}%
%
\settowidth{\Width}{$$}\setlength{\Width}{-0.5\Width}%
\settoheight{\Height}{$$}\settodepth{\Depth}{$$}\setlength{\Height}{-0.5\Height}\setlength{\Depth}{0.5\Depth}\addtolength{\Height}{\Depth}%
\put(1.2000000,0.9000000){\hspace*{\Width}\raisebox{\Height}{$$}}%
%
\settowidth{\Width}{$0$}\setlength{\Width}{-0.5\Width}%
\settoheight{\Height}{$0$}\settodepth{\Depth}{$0$}\setlength{\Height}{-0.5\Height}\setlength{\Depth}{0.5\Depth}\addtolength{\Height}{\Depth}%
\put(2.0000000,0.9000000){\hspace*{\Width}\raisebox{\Height}{$0$}}%
%
\settowidth{\Width}{$$}\setlength{\Width}{-0.5\Width}%
\settoheight{\Height}{$$}\settodepth{\Depth}{$$}\setlength{\Height}{-0.5\Height}\setlength{\Depth}{0.5\Depth}\addtolength{\Height}{\Depth}%
\put(2.8000000,0.9000000){\hspace*{\Width}\raisebox{\Height}{$$}}%
%
\settowidth{\Width}{$0$}\setlength{\Width}{-0.5\Width}%
\settoheight{\Height}{$0$}\settodepth{\Depth}{$0$}\setlength{\Height}{-0.5\Height}\setlength{\Depth}{0.5\Depth}\addtolength{\Height}{\Depth}%
\put(3.6000000,0.9000000){\hspace*{\Width}\raisebox{\Height}{$0$}}%
%
\settowidth{\Width}{$$}\setlength{\Width}{-0.5\Width}%
\settoheight{\Height}{$$}\settodepth{\Depth}{$$}\setlength{\Height}{-0.5\Height}\setlength{\Depth}{0.5\Depth}\addtolength{\Height}{\Depth}%
\put(4.4000000,0.9000000){\hspace*{\Width}\raisebox{\Height}{$$}}%
%
\end{picture}}%}}
\putnotese{80}{53}{\scalebox{1.5}{%%% /Users/takatoosetsuo/polytech22.git/205-0926/presen/fig/zougenhyo5.tex 
%%% Generator=graph22207.cdy 
{\unitlength=6mm%
\begin{picture}%
(4.8,1.8)(0,0)%
\linethickness{0.008in}%%
\polyline(0.00000,1.80000)(0.00000,0.00000)%
%
\polyline(0.80000,1.80000)(0.80000,0.00000)%
%
\polyline(1.60000,1.80000)(1.60000,0.00000)%
%
\polyline(2.40000,1.80000)(2.40000,0.00000)%
%
\polyline(3.20000,1.80000)(3.20000,0.00000)%
%
\polyline(4.00000,1.80000)(4.00000,0.00000)%
%
\polyline(4.80000,1.80000)(4.80000,0.00000)%
%
\polyline(0.00000,1.80000)(4.80000,1.80000)%
%
\polyline(0.00000,1.20000)(4.80000,1.20000)%
%
\polyline(0.00000,0.60000)(4.80000,0.60000)%
%
\polyline(0.00000,0.00000)(4.80000,0.00000)%
%
\small%
\settowidth{\Width}{$x$}\setlength{\Width}{-0.5\Width}%
\settoheight{\Height}{$x$}\settodepth{\Depth}{$x$}\setlength{\Height}{-0.5\Height}\setlength{\Depth}{0.5\Depth}\addtolength{\Height}{\Depth}%
\put(0.4000000,1.5000000){\hspace*{\Width}\raisebox{\Height}{$x$}}%
%
\settowidth{\Width}{$y'$}\setlength{\Width}{-0.5\Width}%
\settoheight{\Height}{$y'$}\settodepth{\Depth}{$y'$}\setlength{\Height}{-0.5\Height}\setlength{\Depth}{0.5\Depth}\addtolength{\Height}{\Depth}%
\put(0.4000000,0.9000000){\hspace*{\Width}\raisebox{\Height}{$y'$}}%
%
\settowidth{\Width}{$y$}\setlength{\Width}{-0.5\Width}%
\settoheight{\Height}{$y$}\settodepth{\Depth}{$y$}\setlength{\Height}{-0.5\Height}\setlength{\Depth}{0.5\Depth}\addtolength{\Height}{\Depth}%
\put(0.4000000,0.3000000){\hspace*{\Width}\raisebox{\Height}{$y$}}%
%
\settowidth{\Width}{$$}\setlength{\Width}{-0.5\Width}%
\settoheight{\Height}{$$}\settodepth{\Depth}{$$}\setlength{\Height}{-0.5\Height}\setlength{\Depth}{0.5\Depth}\addtolength{\Height}{\Depth}%
\put(0.4000000,1.5000000){\hspace*{\Width}\raisebox{\Height}{$$}}%
%
\settowidth{\Width}{$$}\setlength{\Width}{-0.5\Width}%
\settoheight{\Height}{$$}\settodepth{\Depth}{$$}\setlength{\Height}{-0.5\Height}\setlength{\Depth}{0.5\Depth}\addtolength{\Height}{\Depth}%
\put(1.2000000,1.5000000){\hspace*{\Width}\raisebox{\Height}{$$}}%
%
\settowidth{\Width}{$0$}\setlength{\Width}{-0.5\Width}%
\settoheight{\Height}{$0$}\settodepth{\Depth}{$0$}\setlength{\Height}{-0.5\Height}\setlength{\Depth}{0.5\Depth}\addtolength{\Height}{\Depth}%
\put(2.0000000,1.5000000){\hspace*{\Width}\raisebox{\Height}{$0$}}%
%
\settowidth{\Width}{$$}\setlength{\Width}{-0.5\Width}%
\settoheight{\Height}{$$}\settodepth{\Depth}{$$}\setlength{\Height}{-0.5\Height}\setlength{\Depth}{0.5\Depth}\addtolength{\Height}{\Depth}%
\put(2.8000000,1.5000000){\hspace*{\Width}\raisebox{\Height}{$$}}%
%
\settowidth{\Width}{$2$}\setlength{\Width}{-0.5\Width}%
\settoheight{\Height}{$2$}\settodepth{\Depth}{$2$}\setlength{\Height}{-0.5\Height}\setlength{\Depth}{0.5\Depth}\addtolength{\Height}{\Depth}%
\put(3.6000000,1.5000000){\hspace*{\Width}\raisebox{\Height}{$2$}}%
%
\settowidth{\Width}{$$}\setlength{\Width}{-0.5\Width}%
\settoheight{\Height}{$$}\settodepth{\Depth}{$$}\setlength{\Height}{-0.5\Height}\setlength{\Depth}{0.5\Depth}\addtolength{\Height}{\Depth}%
\put(4.4000000,1.5000000){\hspace*{\Width}\raisebox{\Height}{$$}}%
%
\settowidth{\Width}{$$}\setlength{\Width}{-0.5\Width}%
\settoheight{\Height}{$$}\settodepth{\Depth}{$$}\setlength{\Height}{-0.5\Height}\setlength{\Depth}{0.5\Depth}\addtolength{\Height}{\Depth}%
\put(0.4000000,1.5000000){\hspace*{\Width}\raisebox{\Height}{$$}}%
%
\settowidth{\Width}{$\cdots$}\setlength{\Width}{-0.5\Width}%
\settoheight{\Height}{$\cdots$}\settodepth{\Depth}{$\cdots$}\setlength{\Height}{-0.5\Height}\setlength{\Depth}{0.5\Depth}\addtolength{\Height}{\Depth}%
\put(1.2000000,1.5000000){\hspace*{\Width}\raisebox{\Height}{$\cdots$}}%
%
\settowidth{\Width}{$$}\setlength{\Width}{-0.5\Width}%
\settoheight{\Height}{$$}\settodepth{\Depth}{$$}\setlength{\Height}{-0.5\Height}\setlength{\Depth}{0.5\Depth}\addtolength{\Height}{\Depth}%
\put(2.0000000,1.5000000){\hspace*{\Width}\raisebox{\Height}{$$}}%
%
\settowidth{\Width}{$\cdots$}\setlength{\Width}{-0.5\Width}%
\settoheight{\Height}{$\cdots$}\settodepth{\Depth}{$\cdots$}\setlength{\Height}{-0.5\Height}\setlength{\Depth}{0.5\Depth}\addtolength{\Height}{\Depth}%
\put(2.8000000,1.5000000){\hspace*{\Width}\raisebox{\Height}{$\cdots$}}%
%
\settowidth{\Width}{$$}\setlength{\Width}{-0.5\Width}%
\settoheight{\Height}{$$}\settodepth{\Depth}{$$}\setlength{\Height}{-0.5\Height}\setlength{\Depth}{0.5\Depth}\addtolength{\Height}{\Depth}%
\put(3.6000000,1.5000000){\hspace*{\Width}\raisebox{\Height}{$$}}%
%
\settowidth{\Width}{$\cdots$}\setlength{\Width}{-0.5\Width}%
\settoheight{\Height}{$\cdots$}\settodepth{\Depth}{$\cdots$}\setlength{\Height}{-0.5\Height}\setlength{\Depth}{0.5\Depth}\addtolength{\Height}{\Depth}%
\put(4.4000000,1.5000000){\hspace*{\Width}\raisebox{\Height}{$\cdots$}}%
%
\settowidth{\Width}{$$}\setlength{\Width}{-0.5\Width}%
\settoheight{\Height}{$$}\settodepth{\Depth}{$$}\setlength{\Height}{-0.5\Height}\setlength{\Depth}{0.5\Depth}\addtolength{\Height}{\Depth}%
\put(0.4000000,0.9000000){\hspace*{\Width}\raisebox{\Height}{$$}}%
%
\settowidth{\Width}{$$}\setlength{\Width}{-0.5\Width}%
\settoheight{\Height}{$$}\settodepth{\Depth}{$$}\setlength{\Height}{-0.5\Height}\setlength{\Depth}{0.5\Depth}\addtolength{\Height}{\Depth}%
\put(1.2000000,0.9000000){\hspace*{\Width}\raisebox{\Height}{$$}}%
%
\settowidth{\Width}{$0$}\setlength{\Width}{-0.5\Width}%
\settoheight{\Height}{$0$}\settodepth{\Depth}{$0$}\setlength{\Height}{-0.5\Height}\setlength{\Depth}{0.5\Depth}\addtolength{\Height}{\Depth}%
\put(2.0000000,0.9000000){\hspace*{\Width}\raisebox{\Height}{$0$}}%
%
\settowidth{\Width}{$$}\setlength{\Width}{-0.5\Width}%
\settoheight{\Height}{$$}\settodepth{\Depth}{$$}\setlength{\Height}{-0.5\Height}\setlength{\Depth}{0.5\Depth}\addtolength{\Height}{\Depth}%
\put(2.8000000,0.9000000){\hspace*{\Width}\raisebox{\Height}{$$}}%
%
\settowidth{\Width}{$0$}\setlength{\Width}{-0.5\Width}%
\settoheight{\Height}{$0$}\settodepth{\Depth}{$0$}\setlength{\Height}{-0.5\Height}\setlength{\Depth}{0.5\Depth}\addtolength{\Height}{\Depth}%
\put(3.6000000,0.9000000){\hspace*{\Width}\raisebox{\Height}{$0$}}%
%
\settowidth{\Width}{$$}\setlength{\Width}{-0.5\Width}%
\settoheight{\Height}{$$}\settodepth{\Depth}{$$}\setlength{\Height}{-0.5\Height}\setlength{\Depth}{0.5\Depth}\addtolength{\Height}{\Depth}%
\put(4.4000000,0.9000000){\hspace*{\Width}\raisebox{\Height}{$$}}%
%
\settowidth{\Width}{$$}\setlength{\Width}{-0.5\Width}%
\settoheight{\Height}{$$}\settodepth{\Depth}{$$}\setlength{\Height}{-0.5\Height}\setlength{\Depth}{0.5\Depth}\addtolength{\Height}{\Depth}%
\put(0.4000000,0.9000000){\hspace*{\Width}\raisebox{\Height}{$$}}%
%
\settowidth{\Width}{$+$}\setlength{\Width}{-0.5\Width}%
\settoheight{\Height}{$+$}\settodepth{\Depth}{$+$}\setlength{\Height}{-0.5\Height}\setlength{\Depth}{0.5\Depth}\addtolength{\Height}{\Depth}%
\put(1.2000000,0.9000000){\hspace*{\Width}\raisebox{\Height}{$+$}}%
%
\settowidth{\Width}{$$}\setlength{\Width}{-0.5\Width}%
\settoheight{\Height}{$$}\settodepth{\Depth}{$$}\setlength{\Height}{-0.5\Height}\setlength{\Depth}{0.5\Depth}\addtolength{\Height}{\Depth}%
\put(2.0000000,0.9000000){\hspace*{\Width}\raisebox{\Height}{$$}}%
%
\settowidth{\Width}{$-$}\setlength{\Width}{-0.5\Width}%
\settoheight{\Height}{$-$}\settodepth{\Depth}{$-$}\setlength{\Height}{-0.5\Height}\setlength{\Depth}{0.5\Depth}\addtolength{\Height}{\Depth}%
\put(2.8000000,0.9000000){\hspace*{\Width}\raisebox{\Height}{$-$}}%
%
\settowidth{\Width}{$$}\setlength{\Width}{-0.5\Width}%
\settoheight{\Height}{$$}\settodepth{\Depth}{$$}\setlength{\Height}{-0.5\Height}\setlength{\Depth}{0.5\Depth}\addtolength{\Height}{\Depth}%
\put(3.6000000,0.9000000){\hspace*{\Width}\raisebox{\Height}{$$}}%
%
\settowidth{\Width}{$+$}\setlength{\Width}{-0.5\Width}%
\settoheight{\Height}{$+$}\settodepth{\Depth}{$+$}\setlength{\Height}{-0.5\Height}\setlength{\Depth}{0.5\Depth}\addtolength{\Height}{\Depth}%
\put(4.4000000,0.9000000){\hspace*{\Width}\raisebox{\Height}{$+$}}%
%
\end{picture}}%}}
\putnotese{80}{53}{\scalebox{1.5}{%%% /Users/takatoosetsuo/polytech22.git/205-0926/presen/fig/zougenhyo6.tex 
%%% Generator=graph22207.cdy 
{\unitlength=6mm%
\begin{picture}%
(4.8,1.8)(0,0)%
\linethickness{0.008in}%%
\polyline(0.00000,1.80000)(0.00000,0.00000)%
%
\polyline(0.80000,1.80000)(0.80000,0.00000)%
%
\polyline(1.60000,1.80000)(1.60000,0.00000)%
%
\polyline(2.40000,1.80000)(2.40000,0.00000)%
%
\polyline(3.20000,1.80000)(3.20000,0.00000)%
%
\polyline(4.00000,1.80000)(4.00000,0.00000)%
%
\polyline(4.80000,1.80000)(4.80000,0.00000)%
%
\polyline(0.00000,1.80000)(4.80000,1.80000)%
%
\polyline(0.00000,1.20000)(4.80000,1.20000)%
%
\polyline(0.00000,0.60000)(4.80000,0.60000)%
%
\polyline(0.00000,0.00000)(4.80000,0.00000)%
%
\small%
\settowidth{\Width}{$x$}\setlength{\Width}{-0.5\Width}%
\settoheight{\Height}{$x$}\settodepth{\Depth}{$x$}\setlength{\Height}{-0.5\Height}\setlength{\Depth}{0.5\Depth}\addtolength{\Height}{\Depth}%
\put(0.4000000,1.5000000){\hspace*{\Width}\raisebox{\Height}{$x$}}%
%
\settowidth{\Width}{$y'$}\setlength{\Width}{-0.5\Width}%
\settoheight{\Height}{$y'$}\settodepth{\Depth}{$y'$}\setlength{\Height}{-0.5\Height}\setlength{\Depth}{0.5\Depth}\addtolength{\Height}{\Depth}%
\put(0.4000000,0.9000000){\hspace*{\Width}\raisebox{\Height}{$y'$}}%
%
\settowidth{\Width}{$y$}\setlength{\Width}{-0.5\Width}%
\settoheight{\Height}{$y$}\settodepth{\Depth}{$y$}\setlength{\Height}{-0.5\Height}\setlength{\Depth}{0.5\Depth}\addtolength{\Height}{\Depth}%
\put(0.4000000,0.3000000){\hspace*{\Width}\raisebox{\Height}{$y$}}%
%
\settowidth{\Width}{$$}\setlength{\Width}{-0.5\Width}%
\settoheight{\Height}{$$}\settodepth{\Depth}{$$}\setlength{\Height}{-0.5\Height}\setlength{\Depth}{0.5\Depth}\addtolength{\Height}{\Depth}%
\put(0.4000000,1.5000000){\hspace*{\Width}\raisebox{\Height}{$$}}%
%
\settowidth{\Width}{$$}\setlength{\Width}{-0.5\Width}%
\settoheight{\Height}{$$}\settodepth{\Depth}{$$}\setlength{\Height}{-0.5\Height}\setlength{\Depth}{0.5\Depth}\addtolength{\Height}{\Depth}%
\put(1.2000000,1.5000000){\hspace*{\Width}\raisebox{\Height}{$$}}%
%
\settowidth{\Width}{$0$}\setlength{\Width}{-0.5\Width}%
\settoheight{\Height}{$0$}\settodepth{\Depth}{$0$}\setlength{\Height}{-0.5\Height}\setlength{\Depth}{0.5\Depth}\addtolength{\Height}{\Depth}%
\put(2.0000000,1.5000000){\hspace*{\Width}\raisebox{\Height}{$0$}}%
%
\settowidth{\Width}{$$}\setlength{\Width}{-0.5\Width}%
\settoheight{\Height}{$$}\settodepth{\Depth}{$$}\setlength{\Height}{-0.5\Height}\setlength{\Depth}{0.5\Depth}\addtolength{\Height}{\Depth}%
\put(2.8000000,1.5000000){\hspace*{\Width}\raisebox{\Height}{$$}}%
%
\settowidth{\Width}{$2$}\setlength{\Width}{-0.5\Width}%
\settoheight{\Height}{$2$}\settodepth{\Depth}{$2$}\setlength{\Height}{-0.5\Height}\setlength{\Depth}{0.5\Depth}\addtolength{\Height}{\Depth}%
\put(3.6000000,1.5000000){\hspace*{\Width}\raisebox{\Height}{$2$}}%
%
\settowidth{\Width}{$$}\setlength{\Width}{-0.5\Width}%
\settoheight{\Height}{$$}\settodepth{\Depth}{$$}\setlength{\Height}{-0.5\Height}\setlength{\Depth}{0.5\Depth}\addtolength{\Height}{\Depth}%
\put(4.4000000,1.5000000){\hspace*{\Width}\raisebox{\Height}{$$}}%
%
\settowidth{\Width}{$$}\setlength{\Width}{-0.5\Width}%
\settoheight{\Height}{$$}\settodepth{\Depth}{$$}\setlength{\Height}{-0.5\Height}\setlength{\Depth}{0.5\Depth}\addtolength{\Height}{\Depth}%
\put(0.4000000,1.5000000){\hspace*{\Width}\raisebox{\Height}{$$}}%
%
\settowidth{\Width}{$\cdots$}\setlength{\Width}{-0.5\Width}%
\settoheight{\Height}{$\cdots$}\settodepth{\Depth}{$\cdots$}\setlength{\Height}{-0.5\Height}\setlength{\Depth}{0.5\Depth}\addtolength{\Height}{\Depth}%
\put(1.2000000,1.5000000){\hspace*{\Width}\raisebox{\Height}{$\cdots$}}%
%
\settowidth{\Width}{$$}\setlength{\Width}{-0.5\Width}%
\settoheight{\Height}{$$}\settodepth{\Depth}{$$}\setlength{\Height}{-0.5\Height}\setlength{\Depth}{0.5\Depth}\addtolength{\Height}{\Depth}%
\put(2.0000000,1.5000000){\hspace*{\Width}\raisebox{\Height}{$$}}%
%
\settowidth{\Width}{$\cdots$}\setlength{\Width}{-0.5\Width}%
\settoheight{\Height}{$\cdots$}\settodepth{\Depth}{$\cdots$}\setlength{\Height}{-0.5\Height}\setlength{\Depth}{0.5\Depth}\addtolength{\Height}{\Depth}%
\put(2.8000000,1.5000000){\hspace*{\Width}\raisebox{\Height}{$\cdots$}}%
%
\settowidth{\Width}{$$}\setlength{\Width}{-0.5\Width}%
\settoheight{\Height}{$$}\settodepth{\Depth}{$$}\setlength{\Height}{-0.5\Height}\setlength{\Depth}{0.5\Depth}\addtolength{\Height}{\Depth}%
\put(3.6000000,1.5000000){\hspace*{\Width}\raisebox{\Height}{$$}}%
%
\settowidth{\Width}{$\cdots$}\setlength{\Width}{-0.5\Width}%
\settoheight{\Height}{$\cdots$}\settodepth{\Depth}{$\cdots$}\setlength{\Height}{-0.5\Height}\setlength{\Depth}{0.5\Depth}\addtolength{\Height}{\Depth}%
\put(4.4000000,1.5000000){\hspace*{\Width}\raisebox{\Height}{$\cdots$}}%
%
\settowidth{\Width}{$$}\setlength{\Width}{-0.5\Width}%
\settoheight{\Height}{$$}\settodepth{\Depth}{$$}\setlength{\Height}{-0.5\Height}\setlength{\Depth}{0.5\Depth}\addtolength{\Height}{\Depth}%
\put(0.4000000,0.9000000){\hspace*{\Width}\raisebox{\Height}{$$}}%
%
\settowidth{\Width}{$$}\setlength{\Width}{-0.5\Width}%
\settoheight{\Height}{$$}\settodepth{\Depth}{$$}\setlength{\Height}{-0.5\Height}\setlength{\Depth}{0.5\Depth}\addtolength{\Height}{\Depth}%
\put(1.2000000,0.9000000){\hspace*{\Width}\raisebox{\Height}{$$}}%
%
\settowidth{\Width}{$0$}\setlength{\Width}{-0.5\Width}%
\settoheight{\Height}{$0$}\settodepth{\Depth}{$0$}\setlength{\Height}{-0.5\Height}\setlength{\Depth}{0.5\Depth}\addtolength{\Height}{\Depth}%
\put(2.0000000,0.9000000){\hspace*{\Width}\raisebox{\Height}{$0$}}%
%
\settowidth{\Width}{$$}\setlength{\Width}{-0.5\Width}%
\settoheight{\Height}{$$}\settodepth{\Depth}{$$}\setlength{\Height}{-0.5\Height}\setlength{\Depth}{0.5\Depth}\addtolength{\Height}{\Depth}%
\put(2.8000000,0.9000000){\hspace*{\Width}\raisebox{\Height}{$$}}%
%
\settowidth{\Width}{$0$}\setlength{\Width}{-0.5\Width}%
\settoheight{\Height}{$0$}\settodepth{\Depth}{$0$}\setlength{\Height}{-0.5\Height}\setlength{\Depth}{0.5\Depth}\addtolength{\Height}{\Depth}%
\put(3.6000000,0.9000000){\hspace*{\Width}\raisebox{\Height}{$0$}}%
%
\settowidth{\Width}{$$}\setlength{\Width}{-0.5\Width}%
\settoheight{\Height}{$$}\settodepth{\Depth}{$$}\setlength{\Height}{-0.5\Height}\setlength{\Depth}{0.5\Depth}\addtolength{\Height}{\Depth}%
\put(4.4000000,0.9000000){\hspace*{\Width}\raisebox{\Height}{$$}}%
%
\settowidth{\Width}{$$}\setlength{\Width}{-0.5\Width}%
\settoheight{\Height}{$$}\settodepth{\Depth}{$$}\setlength{\Height}{-0.5\Height}\setlength{\Depth}{0.5\Depth}\addtolength{\Height}{\Depth}%
\put(0.4000000,0.9000000){\hspace*{\Width}\raisebox{\Height}{$$}}%
%
\settowidth{\Width}{$+$}\setlength{\Width}{-0.5\Width}%
\settoheight{\Height}{$+$}\settodepth{\Depth}{$+$}\setlength{\Height}{-0.5\Height}\setlength{\Depth}{0.5\Depth}\addtolength{\Height}{\Depth}%
\put(1.2000000,0.9000000){\hspace*{\Width}\raisebox{\Height}{$+$}}%
%
\settowidth{\Width}{$$}\setlength{\Width}{-0.5\Width}%
\settoheight{\Height}{$$}\settodepth{\Depth}{$$}\setlength{\Height}{-0.5\Height}\setlength{\Depth}{0.5\Depth}\addtolength{\Height}{\Depth}%
\put(2.0000000,0.9000000){\hspace*{\Width}\raisebox{\Height}{$$}}%
%
\settowidth{\Width}{$-$}\setlength{\Width}{-0.5\Width}%
\settoheight{\Height}{$-$}\settodepth{\Depth}{$-$}\setlength{\Height}{-0.5\Height}\setlength{\Depth}{0.5\Depth}\addtolength{\Height}{\Depth}%
\put(2.8000000,0.9000000){\hspace*{\Width}\raisebox{\Height}{$-$}}%
%
\settowidth{\Width}{$$}\setlength{\Width}{-0.5\Width}%
\settoheight{\Height}{$$}\settodepth{\Depth}{$$}\setlength{\Height}{-0.5\Height}\setlength{\Depth}{0.5\Depth}\addtolength{\Height}{\Depth}%
\put(3.6000000,0.9000000){\hspace*{\Width}\raisebox{\Height}{$$}}%
%
\settowidth{\Width}{$+$}\setlength{\Width}{-0.5\Width}%
\settoheight{\Height}{$+$}\settodepth{\Depth}{$+$}\setlength{\Height}{-0.5\Height}\setlength{\Depth}{0.5\Depth}\addtolength{\Height}{\Depth}%
\put(4.4000000,0.9000000){\hspace*{\Width}\raisebox{\Height}{$+$}}%
%
\settowidth{\Width}{$$}\setlength{\Width}{-0.5\Width}%
\settoheight{\Height}{$$}\settodepth{\Depth}{$$}\setlength{\Height}{-0.5\Height}\setlength{\Depth}{0.5\Depth}\addtolength{\Height}{\Depth}%
\put(0.4000000,0.3000000){\hspace*{\Width}\raisebox{\Height}{$$}}%
%
\settowidth{\Width}{$\NEarrow$}\setlength{\Width}{-0.5\Width}%
\settoheight{\Height}{$\NEarrow$}\settodepth{\Depth}{$\NEarrow$}\setlength{\Height}{-0.5\Height}\setlength{\Depth}{0.5\Depth}\addtolength{\Height}{\Depth}%
\put(1.2000000,0.3000000){\hspace*{\Width}\raisebox{\Height}{$\NEarrow$}}%
%
\settowidth{\Width}{$$}\setlength{\Width}{-0.5\Width}%
\settoheight{\Height}{$$}\settodepth{\Depth}{$$}\setlength{\Height}{-0.5\Height}\setlength{\Depth}{0.5\Depth}\addtolength{\Height}{\Depth}%
\put(2.0000000,0.3000000){\hspace*{\Width}\raisebox{\Height}{$$}}%
%
\settowidth{\Width}{$\SEarrow$}\setlength{\Width}{-0.5\Width}%
\settoheight{\Height}{$\SEarrow$}\settodepth{\Depth}{$\SEarrow$}\setlength{\Height}{-0.5\Height}\setlength{\Depth}{0.5\Depth}\addtolength{\Height}{\Depth}%
\put(2.8000000,0.3000000){\hspace*{\Width}\raisebox{\Height}{$\SEarrow$}}%
%
\settowidth{\Width}{$$}\setlength{\Width}{-0.5\Width}%
\settoheight{\Height}{$$}\settodepth{\Depth}{$$}\setlength{\Height}{-0.5\Height}\setlength{\Depth}{0.5\Depth}\addtolength{\Height}{\Depth}%
\put(3.6000000,0.3000000){\hspace*{\Width}\raisebox{\Height}{$$}}%
%
\settowidth{\Width}{$\NEarrow$}\setlength{\Width}{-0.5\Width}%
\settoheight{\Height}{$\NEarrow$}\settodepth{\Depth}{$\NEarrow$}\setlength{\Height}{-0.5\Height}\setlength{\Depth}{0.5\Depth}\addtolength{\Height}{\Depth}%
\put(4.4000000,0.3000000){\hspace*{\Width}\raisebox{\Height}{$\NEarrow$}}%
%
\end{picture}}%}}
\end{layer}

\begin{itemize}
\item
$y'=0$となる点と間の範囲での増減の様子を書いた表
\item
[例]$y=x^3-3x^2$
\begin{enumerate}[(1)]
\item $y'$を求める.$y'=3x^2-6x=3x(x-2)$\vspace{-2mm}
\item $y'=0$となる点を求める.$y'=0$より\ $x=0,2$\vspace{-2mm}
\item 増減表を書き,$y'=0$となる点を書き入れる\vspace{-2mm}
\item (3)の$x$の下に$0$を書く\vspace{-2mm}
\item 各範囲の$y'$の符号を書く\\
\hspace*{1zw}$x=-1$,\ $y'>0,\ ...$\vspace{-2mm}
\item $+$は増加,$-$は減少
\end{enumerate}
\end{itemize}

\newslide{増減表(課題)}

\vspace*{18mm}


\begin{layer}{120}{0}
\putnotew{96}{73}{\hyperlink{para6pg11}{\fbox{\Ctab{2.5mm}{\scalebox{1}{\scriptsize $\mathstrut||\!\lhd$}}}}}
\putnotew{125}{73}{\hyperlink{para8pg1}{\fbox{\Ctab{2.5mm}{\scalebox{1}{\scriptsize $\mathstrut \!\rhd\!\!||$}}}}}
\putnotee{126}{73}{\scriptsize\color{blue} 1/1}
\end{layer}

\slidepage

\begin{layer}{120}{0}
\putnotese{80}{53}{\scalebox{1.5}{%%% /Users/takatoosetsuo/polytech22.git/205-0926/presen/fig/zougenhyo1.tex 
%%% Generator=graph22207.cdy 
{\unitlength=6mm%
\begin{picture}%
(4.8,1.8)(0,0)%
\linethickness{0.008in}%%
\polyline(0.00000,1.80000)(0.00000,0.00000)%
%
\polyline(0.80000,1.80000)(0.80000,0.00000)%
%
\polyline(1.60000,1.80000)(1.60000,0.00000)%
%
\polyline(2.40000,1.80000)(2.40000,0.00000)%
%
\polyline(3.20000,1.80000)(3.20000,0.00000)%
%
\polyline(4.00000,1.80000)(4.00000,0.00000)%
%
\polyline(4.80000,1.80000)(4.80000,0.00000)%
%
\polyline(0.00000,1.80000)(4.80000,1.80000)%
%
\polyline(0.00000,1.20000)(4.80000,1.20000)%
%
\polyline(0.00000,0.60000)(4.80000,0.60000)%
%
\polyline(0.00000,0.00000)(4.80000,0.00000)%
%
\small%
\settowidth{\Width}{$x$}\setlength{\Width}{-0.5\Width}%
\settoheight{\Height}{$x$}\settodepth{\Depth}{$x$}\setlength{\Height}{-0.5\Height}\setlength{\Depth}{0.5\Depth}\addtolength{\Height}{\Depth}%
\put(0.4000000,1.5000000){\hspace*{\Width}\raisebox{\Height}{$x$}}%
%
\settowidth{\Width}{$y'$}\setlength{\Width}{-0.5\Width}%
\settoheight{\Height}{$y'$}\settodepth{\Depth}{$y'$}\setlength{\Height}{-0.5\Height}\setlength{\Depth}{0.5\Depth}\addtolength{\Height}{\Depth}%
\put(0.4000000,0.9000000){\hspace*{\Width}\raisebox{\Height}{$y'$}}%
%
\settowidth{\Width}{$y$}\setlength{\Width}{-0.5\Width}%
\settoheight{\Height}{$y$}\settodepth{\Depth}{$y$}\setlength{\Height}{-0.5\Height}\setlength{\Depth}{0.5\Depth}\addtolength{\Height}{\Depth}%
\put(0.4000000,0.3000000){\hspace*{\Width}\raisebox{\Height}{$y$}}%
%
\end{picture}}%}}
\end{layer}

\begin{itemize}
\item
[課題]\monban 関数$y=x^4-4x^3$について,問いに答えよ.\seteda{65}\\
\eda{$y'=4x^2(x-3)$となることを示せ}\\
\eda{$y'=0$となる$x$を求めよ}\\
\eda{増減の1行目に入れる数式記号を左から順に書け}\\
\eda{増減の2行目に入れる数式記号を左から順に書け}\\
\eda{増減の3行目に入れる矢印記号を左から順に書け}\\
\hspace*{2zw}${\cdots}$ 点々 \verb|{\nearrow}|\\
hspace*{2zw}${\nearrow}$ 右上 \verb|{\nearrow}|\\
\hspace*{2zw}${\searrow}$ 右下 \verb|{\searrow}|
\end{itemize}
%%%%%%%%%%%%

%%%%%%%%%%%%%%%%%%%%


\mainslide{ 積分の応用}


\slidepage[m]
%%%%%%%%%%%%%

%%%%%%%%%%%%%%%%%%%%

\newslide{定積分と面積}

\vspace*{18mm}


\begin{layer}{120}{0}
\putnotew{96}{73}{\hyperlink{para7pg1}{\fbox{\Ctab{2.5mm}{\scalebox{1}{\scriptsize $\mathstrut||\!\lhd$}}}}}
\putnotew{101}{73}{\hyperlink{para8pg1}{\fbox{\Ctab{2.5mm}{\scalebox{1}{\scriptsize $\mathstrut|\!\lhd$}}}}}
\putnotew{108}{73}{\hyperlink{para8pg3}{\fbox{\Ctab{4.5mm}{\scalebox{1}{\scriptsize $\mathstrut\!\!\lhd\!\!$}}}}}
\putnotew{115}{73}{\hyperlink{para8pg4}{\fbox{\Ctab{4.5mm}{\scalebox{1}{\scriptsize $\mathstrut\!\rhd\!$}}}}}
\putnotew{120}{73}{\hyperlink{para8pg4}{\fbox{\Ctab{2.5mm}{\scalebox{1}{\scriptsize $\mathstrut \!\rhd\!\!|$}}}}}
\putnotew{125}{73}{\hyperlink{para9pg1}{\fbox{\Ctab{2.5mm}{\scalebox{1}{\scriptsize $\mathstrut \!\rhd\!\!||$}}}}}
\putnotee{126}{73}{\scriptsize\color{blue} 4/4}
\end{layer}

\slidepage

\begin{layer}{120}{0}
\putnotese{100}{5}{\scalebox{0.8}{%%% /Users/takatoosetsuo/polytech22.git/205-0912/presen/fig/sekibun1.tex 
%%% Generator=presen22206.cdy 
{\unitlength=1cm%
\begin{picture}%
(3.5,4)(-0.5,-1.5)%
\linethickness{0.008in}%%
\Large\bf\boldmath%
\small%
\polyline(-0.50000,1.25000)(-0.43000,1.04490)(-0.36000,0.84960)(-0.29000,0.66410)%
(-0.22000,0.48840)(-0.15000,0.32250)(-0.08000,0.16640)(-0.01000,0.02010)(0.06000,-0.11640)%
(0.13000,-0.24310)(0.20000,-0.36000)(0.27000,-0.46710)(0.34000,-0.56440)(0.41000,-0.65190)%
(0.48000,-0.72960)(0.55000,-0.79750)(0.62000,-0.85560)(0.69000,-0.90390)(0.76000,-0.94240)%
(0.83000,-0.97110)(0.90000,-0.99000)(0.97000,-0.99910)(1.04000,-0.99840)(1.11000,-0.98790)%
(1.18000,-0.96760)(1.25000,-0.93750)(1.32000,-0.89760)(1.39000,-0.84790)(1.46000,-0.78840)%
(1.53000,-0.71910)(1.60000,-0.64000)(1.67000,-0.55110)(1.74000,-0.45240)(1.81000,-0.34390)%
(1.88000,-0.22560)(1.95000,-0.09750)(2.02000,0.04040)(2.09000,0.18810)(2.16000,0.34560)%
(2.23000,0.51290)(2.30000,0.69000)(2.37000,0.87690)(2.44000,1.07360)(2.51000,1.28010)%
(2.58000,1.49640)(2.65000,1.72250)(2.72000,1.95840)(2.79000,2.20410)(2.86000,2.45960)%
(2.87066,2.50000)%
%
\polyline(2.00000,0.05000)(2.00000,-0.05000)%
%
\settowidth{\Width}{$2$}\setlength{\Width}{0\Width}%
\settoheight{\Height}{$2$}\settodepth{\Depth}{$2$}\setlength{\Height}{-\Height}%
\put(2.0500000,-0.1000000){\hspace*{\Width}\raisebox{\Height}{$2$}}%
%
\polyline(-0.50000,0.00000)(3.00000,0.00000)%
%
\polyline(0.00000,-1.50000)(0.00000,2.50000)%
%
\settowidth{\Width}{$x$}\setlength{\Width}{0\Width}%
\settoheight{\Height}{$x$}\settodepth{\Depth}{$x$}\setlength{\Height}{-0.5\Height}\setlength{\Depth}{0.5\Depth}\addtolength{\Height}{\Depth}%
\put(3.0500000,0.0000000){\hspace*{\Width}\raisebox{\Height}{$x$}}%
%
\settowidth{\Width}{$y$}\setlength{\Width}{-0.5\Width}%
\settoheight{\Height}{$y$}\settodepth{\Depth}{$y$}\setlength{\Height}{\Depth}%
\put(0.0000000,2.5500000){\hspace*{\Width}\raisebox{\Height}{$y$}}%
%
\settowidth{\Width}{O}\setlength{\Width}{-1\Width}%
\settoheight{\Height}{O}\settodepth{\Depth}{O}\setlength{\Height}{-\Height}%
\put(-0.0500000,-0.0500000){\hspace*{\Width}\raisebox{\Height}{O}}%
%
\end{picture}}%}}
\putnotese{100}{5}{\scalebox{0.8}{%%% /Users/takatoosetsuo/polytech22.git/205-0912/presen/fig/sekibun2.tex 
%%% Generator=presen22206.cdy 
{\unitlength=15mm%
\begin{picture}%
(3,3)(-1.5,-1.5)%
\linethickness{0.008in}%%
\Large\bf\boldmath%
\small%
\polyline(-1.14449,-1.50000)(-1.14000,-1.48154)(-1.08000,-1.25971)(-1.02000,-1.06121)%
(-0.96000,-0.88474)(-0.90000,-0.72900)(-0.84000,-0.59270)(-0.78000,-0.47455)(-0.72000,-0.37325)%
(-0.66000,-0.28750)(-0.60000,-0.21600)(-0.54000,-0.15746)(-0.48000,-0.11059)(-0.42000,-0.07409)%
(-0.36000,-0.04666)(-0.30000,-0.02700)(-0.24000,-0.01382)(-0.18000,-0.00583)(-0.12000,-0.00173)%
(-0.06000,-0.00022)(0.00000,0.00000)(0.06000,0.00022)(0.12000,0.00173)(0.18000,0.00583)%
(0.24000,0.01382)(0.30000,0.02700)(0.36000,0.04666)(0.42000,0.07409)(0.48000,0.11059)%
(0.54000,0.15746)(0.60000,0.21600)(0.66000,0.28750)(0.72000,0.37325)(0.78000,0.47455)%
(0.84000,0.59270)(0.90000,0.72900)(0.96000,0.88474)(1.02000,1.06121)(1.08000,1.25971)%
(1.14000,1.48154)(1.14449,1.50000)%
%
\polyline(-1.00000,0.03333)(-1.00000,-0.03333)%
%
\settowidth{\Width}{$-1$}\setlength{\Width}{-0.5\Width}%
\settoheight{\Height}{$-1$}\settodepth{\Depth}{$-1$}\setlength{\Height}{\Depth}%
\put(-1.0000000,0.0666667){\hspace*{\Width}\raisebox{\Height}{$-1$}}%
%
\polyline(1.00000,0.03333)(1.00000,-0.03333)%
%
\settowidth{\Width}{$1$}\setlength{\Width}{-0.5\Width}%
\settoheight{\Height}{$1$}\settodepth{\Depth}{$1$}\setlength{\Height}{-\Height}%
\put(1.0000000,-0.0666667){\hspace*{\Width}\raisebox{\Height}{$1$}}%
%
\polyline(-1.00000,-1.00000)(-1.00000,0.00000)%
%
\polyline(-1.00000,-1.00000)(-0.99877,-0.99877)%
%
\polyline(-1.00000,-0.88215)(-0.93331,-0.81546)%
%
\polyline(-0.11957,-0.00172)(-0.11785,0.00000)%
%
\polyline(-1.00000,-0.76430)(-0.84912,-0.61342)%
%
\polyline(-0.25221,-0.01650)(-0.23570,0.00000)%
%
\polyline(-1.00000,-0.64645)(-0.70415,-0.35060)%
%
\polyline(-0.43951,-0.08596)(-0.35355,0.00000)%
%
\polyline(-1.00000,-0.52860)(-0.47140,0.00000)%
%
\polyline(-1.00000,-0.41074)(-0.58926,0.00000)%
%
\polyline(-1.00000,-0.29289)(-0.70711,0.00000)%
%
\polyline(-1.00000,-0.17504)(-0.82496,0.00000)%
%
\polyline(-1.00000,-0.05719)(-0.94281,0.00000)%
%
\polyline(1.00000,0.00000)(1.00000,1.00000)%
%
\polyline(0.94281,0.00000)(1.00000,0.05719)%
%
\polyline(0.82496,0.00000)(1.00000,0.17504)%
%
\polyline(0.70711,0.00000)(1.00000,0.29289)%
%
\polyline(0.58926,0.00000)(1.00000,0.41074)%
%
\polyline(0.47140,0.00000)(1.00000,0.52860)%
%
\polyline(0.35355,0.00000)(0.43951,0.08596)%
%
\polyline(0.70415,0.35060)(1.00000,0.64645)%
%
\polyline(0.23570,0.00000)(0.25221,0.01650)%
%
\polyline(0.84912,0.61342)(1.00000,0.76430)%
%
\polyline(0.11785,0.00000)(0.11957,0.00172)%
%
\polyline(0.93331,0.81546)(1.00000,0.88215)%
%
\polyline(0.99877,0.99877)(1.00000,1.00000)%
%
\settowidth{\Width}{$y=x^3$}\setlength{\Width}{0\Width}%
\settoheight{\Height}{$y=x^3$}\settodepth{\Depth}{$y=x^3$}\setlength{\Height}{-0.5\Height}\setlength{\Depth}{0.5\Depth}\addtolength{\Height}{\Depth}%
\put(1.2633333,1.6200000){\hspace*{\Width}\raisebox{\Height}{$y=x^3$}}%
%
\polyline(-1.50000,0.00000)(1.50000,0.00000)%
%
\polyline(0.00000,-1.50000)(0.00000,1.50000)%
%
\settowidth{\Width}{$x$}\setlength{\Width}{0\Width}%
\settoheight{\Height}{$x$}\settodepth{\Depth}{$x$}\setlength{\Height}{-0.5\Height}\setlength{\Depth}{0.5\Depth}\addtolength{\Height}{\Depth}%
\put(1.5333333,0.0000000){\hspace*{\Width}\raisebox{\Height}{$x$}}%
%
\settowidth{\Width}{$y$}\setlength{\Width}{-0.5\Width}%
\settoheight{\Height}{$y$}\settodepth{\Depth}{$y$}\setlength{\Height}{\Depth}%
\put(0.0000000,1.5333333){\hspace*{\Width}\raisebox{\Height}{$y$}}%
%
\settowidth{\Width}{O}\setlength{\Width}{-1\Width}%
\settoheight{\Height}{O}\settodepth{\Depth}{O}\setlength{\Height}{-\Height}%
\put(-0.0333333,-0.0333333){\hspace*{\Width}\raisebox{\Height}{O}}%
%
\end{picture}}%}}
\end{layer}

\vspace{1zw}

面積$S$,定積分$I=\dint_a^b f(x)\,dx$
\begin{itemize}
\item
$a\leqq x \leqq b$で$f(x)$が正のとき\ \fbox{$S=I$}\\
\hspace*{1zw}{\normalsize$S=\dint_1^2 (4-x^2)\,dx=\Bigl[4x-\bunsuu{1}{3}x^3\Bigr]_1^2=\bunsuu{5}{3}$}
\item
$a\leqq x \leqq b$で$f(x)$が負のとき\ \fbox{$S=-I$}\\
\hspace*{1zw}{\normalsize$I=\dint_2^3 (4-x^2)\,dx=\Bigl[4x-\bunsuu{1}{3}x^3\Bigr]_2^3=-\bunsuu{7}{3}$}\\
\hspace*{1zw}{\normalsize$S=-I=\bunsuu{7}{3}$}
\end{itemize}

\newslide{2曲線で囲まれる図形の面積}

\vspace*{18mm}


\begin{layer}{120}{0}
\putnotew{96}{73}{\hyperlink{para8pg4}{\fbox{\Ctab{2.5mm}{\scalebox{1}{\scriptsize $\mathstrut||\!\lhd$}}}}}
\putnotew{101}{73}{\hyperlink{para9pg1}{\fbox{\Ctab{2.5mm}{\scalebox{1}{\scriptsize $\mathstrut|\!\lhd$}}}}}
\putnotew{108}{73}{\hyperlink{para9pg8}{\fbox{\Ctab{4.5mm}{\scalebox{1}{\scriptsize $\mathstrut\!\!\lhd\!\!$}}}}}
\putnotew{115}{73}{\hyperlink{para9pg9}{\fbox{\Ctab{4.5mm}{\scalebox{1}{\scriptsize $\mathstrut\!\rhd\!$}}}}}
\putnotew{120}{73}{\hyperlink{para9pg9}{\fbox{\Ctab{2.5mm}{\scalebox{1}{\scriptsize $\mathstrut \!\rhd\!\!|$}}}}}
\putnotew{125}{73}{\hyperlink{para10pg1}{\fbox{\Ctab{2.5mm}{\scalebox{1}{\scriptsize $\mathstrut \!\rhd\!\!||$}}}}}
\putnotee{126}{73}{\scriptsize\color{blue} 9/9}
\end{layer}

\slidepage

\begin{layer}{120}{0}
\putnotese{5}{25}{\scalebox{0.7}{%%% /Users/takatoosetsuo/polytech22.git/205-0926/presen/fig/2kyokusen1.tex 
%%% Generator=graph22207.cdy 
{\unitlength=6mm%
\begin{picture}%
(3.98,3.86)(-0.5,-0.5)%
\linethickness{0.008in}%%
\linethickness{0.012in}%%
\polyline(0.50000,2.00000)(0.58335,2.12148)(0.66498,2.23238)(0.74489,2.33268)(0.82310,2.42241)%
(0.89959,2.50154)(0.97436,2.57009)(1.04743,2.62806)(1.11877,2.67543)(1.18841,2.71223)%
(1.25633,2.73843)(1.35715,2.75725)(1.45836,2.75431)(1.55983,2.73464)(1.66142,2.70327)%
(1.76300,2.66523)(1.86446,2.62554)(1.96566,2.58922)(2.06648,2.56132)(2.16678,2.54685)%
(2.26644,2.55084)(2.33335,2.56427)(2.40170,2.58469)(2.47148,2.61212)(2.54269,2.64654)%
(2.61533,2.68796)(2.68940,2.73637)(2.76490,2.79178)(2.84184,2.85419)(2.92020,2.92360)%
(3.00000,3.00000)%
%
\linethickness{0.008in}%%
\linethickness{0.012in}%%
\polyline(0.50000,1.00000)(0.62643,1.01935)(0.74703,1.03807)(0.86179,1.05618)(0.97071,1.07366)%
(1.07379,1.09053)(1.17103,1.10677)(1.26244,1.12240)(1.34801,1.13740)(1.42774,1.15178)%
(1.50164,1.16554)(1.60694,1.18533)(1.70986,1.20452)(1.81034,1.22336)(1.90831,1.24213)%
(2.00370,1.26108)(2.09645,1.28046)(2.18648,1.30054)(2.27372,1.32158)(2.35812,1.34383)%
(2.43960,1.36756)(2.49280,1.38475)(2.54571,1.40357)(2.59835,1.42402)(2.65072,1.44610)%
(2.70281,1.46981)(2.75462,1.49516)(2.80615,1.52214)(2.85741,1.55075)(2.90839,1.58100)%
(2.95909,1.61288)%
%
\linethickness{0.008in}%%
\polyline(0.50000,2.00000)(0.50000,0.00000)%
%
\polyline(3.00000,3.00000)(3.00000,0.00000)%
%
\polyline(2.71361,1.47510)(2.95909,1.72058)%
%
\polyline(1.88738,1.23812)(2.95909,2.30983)%
%
\polyline(1.16592,1.10592)(2.95909,2.89909)%
%
\polyline(0.50000,1.02926)(2.03952,2.56878)%
%
\polyline(0.50000,1.61851)(1.60284,2.72136)%
%
\linethickness{0.012in}%%
\polyline(0.50000,2.00000)(0.58335,2.12148)(0.66498,2.23238)(0.74489,2.33268)(0.82310,2.42241)%
(0.89959,2.50154)(0.97436,2.57009)(1.04743,2.62806)(1.11877,2.67543)(1.18841,2.71223)%
(1.25633,2.73843)(1.35715,2.75725)(1.45836,2.75431)(1.55983,2.73464)(1.66142,2.70327)%
(1.76300,2.66523)(1.86446,2.62554)(1.96566,2.58922)(2.06648,2.56132)(2.16678,2.54685)%
(2.26644,2.55084)(2.33335,2.56427)(2.40170,2.58469)(2.47148,2.61212)(2.54269,2.64654)%
(2.61533,2.68796)(2.68940,2.73637)(2.76490,2.79178)(2.84184,2.85419)(2.92020,2.92360)%
(3.00000,3.00000)%
%
\linethickness{0.008in}%%
\linethickness{0.012in}%%
\polyline(0.50000,1.00000)(0.62643,1.01935)(0.74703,1.03807)(0.86179,1.05618)(0.97071,1.07366)%
(1.07379,1.09053)(1.17103,1.10677)(1.26244,1.12240)(1.34801,1.13740)(1.42774,1.15178)%
(1.50164,1.16554)(1.60694,1.18533)(1.70986,1.20452)(1.81034,1.22336)(1.90831,1.24213)%
(2.00370,1.26108)(2.09645,1.28046)(2.18648,1.30054)(2.27372,1.32158)(2.35812,1.34383)%
(2.43960,1.36756)(2.49280,1.38475)(2.54571,1.40357)(2.59835,1.42402)(2.65072,1.44610)%
(2.70281,1.46981)(2.75462,1.49516)(2.80615,1.52214)(2.85741,1.55075)(2.90839,1.58100)%
(2.95909,1.61288)%
%
\linethickness{0.008in}%%
\polyline(0.50000,2.00000)(0.50000,0.00000)%
%
\polyline(3.00000,3.00000)(3.00000,0.00000)%
%
\polyline(-0.50000,0.00000)(3.47858,0.00000)%
%
\polyline(0.00000,-0.50000)(0.00000,3.35893)%
%
\settowidth{\Width}{$x$}\setlength{\Width}{0\Width}%
\settoheight{\Height}{$x$}\settodepth{\Depth}{$x$}\setlength{\Height}{-0.5\Height}\setlength{\Depth}{0.5\Depth}\addtolength{\Height}{\Depth}%
\put(3.5633333,0.0000000){\hspace*{\Width}\raisebox{\Height}{$x$}}%
%
\settowidth{\Width}{$y$}\setlength{\Width}{-0.5\Width}%
\settoheight{\Height}{$y$}\settodepth{\Depth}{$y$}\setlength{\Height}{\Depth}%
\put(0.0000000,3.4433333){\hspace*{\Width}\raisebox{\Height}{$y$}}%
%
\settowidth{\Width}{O}\setlength{\Width}{-1\Width}%
\settoheight{\Height}{O}\settodepth{\Depth}{O}\setlength{\Height}{-\Height}%
\put(-0.0833333,-0.0833333){\hspace*{\Width}\raisebox{\Height}{O}}%
%
\end{picture}}%}}
\putnotese{30}{25}{\scalebox{0.7}{%%% /Users/takatoosetsuo/polytech22.git/205-0926/presen/fig/2kyokusen2.tex 
%%% Generator=graph22207.cdy 
{\unitlength=6mm%
\begin{picture}%
(3.98,3.86)(-0.5,-0.5)%
\linethickness{0.008in}%%
\linethickness{0.012in}%%
\polyline(0.50000,2.00000)(0.58335,2.12148)(0.66498,2.23238)(0.74489,2.33268)(0.82310,2.42241)%
(0.89959,2.50154)(0.97436,2.57009)(1.04743,2.62806)(1.11877,2.67543)(1.18841,2.71223)%
(1.25633,2.73843)(1.35715,2.75725)(1.45836,2.75431)(1.55983,2.73464)(1.66142,2.70327)%
(1.76300,2.66523)(1.86446,2.62554)(1.96566,2.58922)(2.06648,2.56132)(2.16678,2.54685)%
(2.26644,2.55084)(2.33335,2.56427)(2.40170,2.58469)(2.47148,2.61212)(2.54269,2.64654)%
(2.61533,2.68796)(2.68940,2.73637)(2.76490,2.79178)(2.84184,2.85419)(2.92020,2.92360)%
(3.00000,3.00000)%
%
\linethickness{0.008in}%%
\linethickness{0.012in}%%
\polyline(0.50000,1.00000)(0.62643,1.01935)(0.74703,1.03807)(0.86179,1.05618)(0.97071,1.07366)%
(1.07379,1.09053)(1.17103,1.10677)(1.26244,1.12240)(1.34801,1.13740)(1.42774,1.15178)%
(1.50164,1.16554)(1.60694,1.18533)(1.70986,1.20452)(1.81034,1.22336)(1.90831,1.24213)%
(2.00370,1.26108)(2.09645,1.28046)(2.18648,1.30054)(2.27372,1.32158)(2.35812,1.34383)%
(2.43960,1.36756)(2.49280,1.38475)(2.54571,1.40357)(2.59835,1.42402)(2.65072,1.44610)%
(2.70281,1.46981)(2.75462,1.49516)(2.80615,1.52214)(2.85741,1.55075)(2.90839,1.58100)%
(2.95909,1.61288)%
%
\linethickness{0.008in}%%
\polyline(0.50000,2.00000)(0.50000,0.00000)%
%
\polyline(3.00000,3.00000)(3.00000,0.00000)%
%
\polyline(2.41702,-0.00000)(3.00000,0.58298)%
%
\polyline(1.82777,0.00000)(3.00000,1.17223)%
%
\polyline(1.23851,-0.00000)(3.00000,1.76149)%
%
\polyline(0.64926,-0.00000)(3.00000,2.35074)%
%
\polyline(0.50000,0.44000)(3.00000,2.94000)%
%
\polyline(0.50000,1.02926)(2.03952,2.56878)%
%
\polyline(0.50000,1.61851)(1.60284,2.72136)%
%
\polyline(-0.50000,0.00000)(3.47858,0.00000)%
%
\polyline(0.00000,-0.50000)(0.00000,3.35893)%
%
\settowidth{\Width}{$x$}\setlength{\Width}{0\Width}%
\settoheight{\Height}{$x$}\settodepth{\Depth}{$x$}\setlength{\Height}{-0.5\Height}\setlength{\Depth}{0.5\Depth}\addtolength{\Height}{\Depth}%
\put(3.5633333,0.0000000){\hspace*{\Width}\raisebox{\Height}{$x$}}%
%
\settowidth{\Width}{$y$}\setlength{\Width}{-0.5\Width}%
\settoheight{\Height}{$y$}\settodepth{\Depth}{$y$}\setlength{\Height}{\Depth}%
\put(0.0000000,3.4433333){\hspace*{\Width}\raisebox{\Height}{$y$}}%
%
\settowidth{\Width}{O}\setlength{\Width}{-1\Width}%
\settoheight{\Height}{O}\settodepth{\Depth}{O}\setlength{\Height}{-\Height}%
\put(-0.0833333,-0.0833333){\hspace*{\Width}\raisebox{\Height}{O}}%
%
\end{picture}}%}}
\putnotese{53}{25}{\scalebox{0.7}{%%% /Users/takatoosetsuo/polytech22.git/205-0926/presen/fig/2kyokusen3.tex 
%%% Generator=graph22207.cdy 
{\unitlength=6mm%
\begin{picture}%
(3.98,3.86)(-0.5,-0.5)%
\linethickness{0.008in}%%
\linethickness{0.012in}%%
\polyline(0.50000,2.00000)(0.58335,2.12148)(0.66498,2.23238)(0.74489,2.33268)(0.82310,2.42241)%
(0.89959,2.50154)(0.97436,2.57009)(1.04743,2.62806)(1.11877,2.67543)(1.18841,2.71223)%
(1.25633,2.73843)(1.35715,2.75725)(1.45836,2.75431)(1.55983,2.73464)(1.66142,2.70327)%
(1.76300,2.66523)(1.86446,2.62554)(1.96566,2.58922)(2.06648,2.56132)(2.16678,2.54685)%
(2.26644,2.55084)(2.33335,2.56427)(2.40170,2.58469)(2.47148,2.61212)(2.54269,2.64654)%
(2.61533,2.68796)(2.68940,2.73637)(2.76490,2.79178)(2.84184,2.85419)(2.92020,2.92360)%
(3.00000,3.00000)%
%
\linethickness{0.008in}%%
\linethickness{0.012in}%%
\polyline(0.50000,1.00000)(0.62643,1.01935)(0.74703,1.03807)(0.86179,1.05618)(0.97071,1.07366)%
(1.07379,1.09053)(1.17103,1.10677)(1.26244,1.12240)(1.34801,1.13740)(1.42774,1.15178)%
(1.50164,1.16554)(1.60694,1.18533)(1.70986,1.20452)(1.81034,1.22336)(1.90831,1.24213)%
(2.00370,1.26108)(2.09645,1.28046)(2.18648,1.30054)(2.27372,1.32158)(2.35812,1.34383)%
(2.43960,1.36756)(2.49280,1.38475)(2.54571,1.40357)(2.59835,1.42402)(2.65072,1.44610)%
(2.70281,1.46981)(2.75462,1.49516)(2.80615,1.52214)(2.85741,1.55075)(2.90839,1.58100)%
(2.95909,1.61288)%
%
\linethickness{0.008in}%%
\polyline(0.50000,2.00000)(0.50000,0.00000)%
%
\polyline(3.00000,3.00000)(3.00000,0.00000)%
%
\polyline(2.41702,-0.00000)(2.95909,0.54207)%
%
\polyline(1.82777,0.00000)(2.95909,1.13132)%
%
\polyline(1.23851,-0.00000)(2.71361,1.47510)%
%
\polyline(0.64926,-0.00000)(1.88738,1.23812)%
%
\polyline(0.50000,0.44000)(1.16592,1.10592)%
%
\polyline(-0.50000,0.00000)(3.47858,0.00000)%
%
\polyline(0.00000,-0.50000)(0.00000,3.35893)%
%
\settowidth{\Width}{$x$}\setlength{\Width}{0\Width}%
\settoheight{\Height}{$x$}\settodepth{\Depth}{$x$}\setlength{\Height}{-0.5\Height}\setlength{\Depth}{0.5\Depth}\addtolength{\Height}{\Depth}%
\put(3.5633333,0.0000000){\hspace*{\Width}\raisebox{\Height}{$x$}}%
%
\settowidth{\Width}{$y$}\setlength{\Width}{-0.5\Width}%
\settoheight{\Height}{$y$}\settodepth{\Depth}{$y$}\setlength{\Height}{\Depth}%
\put(0.0000000,3.4433333){\hspace*{\Width}\raisebox{\Height}{$y$}}%
%
\settowidth{\Width}{O}\setlength{\Width}{-1\Width}%
\settoheight{\Height}{O}\settodepth{\Depth}{O}\setlength{\Height}{-\Height}%
\put(-0.0833333,-0.0833333){\hspace*{\Width}\raisebox{\Height}{O}}%
%
\end{picture}}%}}
\putnotec{26}{31}{$=$}
\putnotec{50}{31}{$-$}
\putnotee{72}{31}{\normalsize$S=S_1-S_2=\dint_a^b(f(x)-g(x))\,dx$}
\putnotese{5}{50}{\scalebox{0.7}{%%% /Users/takatoosetsuo/polytech22.git/205-0926/presen/fig/2kyokusen8_1.tex 
%%% Generator=graph22207.cdy 
{\unitlength=6mm%
\begin{picture}%
(3.98,4.86)(-0.5,-1.5)%
\linethickness{0.008in}%%
\linethickness{0.012in}%%
\polyline(0.50000,2.00000)(0.58335,2.12148)(0.66498,2.23238)(0.74489,2.33268)(0.82310,2.42241)%
(0.89959,2.50154)(0.97436,2.57009)(1.04743,2.62806)(1.11877,2.67543)(1.18841,2.71223)%
(1.25633,2.73843)(1.35715,2.75725)(1.45836,2.75431)(1.55983,2.73464)(1.66142,2.70327)%
(1.76300,2.66523)(1.86446,2.62554)(1.96566,2.58922)(2.06648,2.56132)(2.16678,2.54685)%
(2.26644,2.55084)(2.33335,2.56427)(2.40170,2.58469)(2.47148,2.61212)(2.54269,2.64654)%
(2.61533,2.68796)(2.68940,2.73637)(2.76490,2.79178)(2.84184,2.85419)(2.92020,2.92360)%
(3.00000,3.00000)%
%
\linethickness{0.008in}%%
\linethickness{0.012in}%%
\polyline(0.50000,-1.00000)(0.62029,-0.95083)(0.73419,-0.90648)(0.84169,-0.86694)%
(0.94279,-0.83220)(1.03749,-0.80228)(1.12580,-0.77717)(1.20771,-0.75686)(1.28322,-0.74137)%
(1.35234,-0.73069)(1.41506,-0.72482)(1.50354,-0.72649)(1.59054,-0.73984)(1.67620,-0.76180)%
(1.76069,-0.78928)(1.84418,-0.81922)(1.92681,-0.84853)(2.00875,-0.87414)(2.09015,-0.89297)%
(2.17119,-0.90194)(2.25201,-0.89798)(2.30795,-0.88713)(2.36808,-0.86984)(2.43240,-0.84612)%
(2.50092,-0.81597)(2.57362,-0.77939)(2.65051,-0.73638)(2.73160,-0.68693)(2.81688,-0.63105)%
(2.90634,-0.56874)(3.00000,-0.50000)%
%
\linethickness{0.008in}%%
\polyline(0.50000,2.00000)(0.50000,-1.00000)%
%
\polyline(3.00000,3.00000)(3.00000,-0.50000)%
%
\polyline(2.03647,-0.88055)(3.00000,0.08298)%
%
\polyline(1.58828,-0.73949)(3.00000,0.67223)%
%
\polyline(0.88722,-0.85129)(3.00000,1.26149)%
%
\polyline(0.50000,-0.64926)(3.00000,1.85074)%
%
\polyline(0.50000,-0.06000)(3.00000,2.44000)%
%
\polyline(0.50000,0.52926)(2.75598,2.78523)%
%
\polyline(0.50000,1.11851)(1.96962,2.58813)%
%
\polyline(0.50000,1.70777)(1.53222,2.73999)%
%
\polyline(-0.50000,0.00000)(3.47858,0.00000)%
%
\polyline(0.00000,-1.50000)(0.00000,3.35893)%
%
\settowidth{\Width}{$x$}\setlength{\Width}{0\Width}%
\settoheight{\Height}{$x$}\settodepth{\Depth}{$x$}\setlength{\Height}{-0.5\Height}\setlength{\Depth}{0.5\Depth}\addtolength{\Height}{\Depth}%
\put(3.5633333,0.0000000){\hspace*{\Width}\raisebox{\Height}{$x$}}%
%
\settowidth{\Width}{$y$}\setlength{\Width}{-0.5\Width}%
\settoheight{\Height}{$y$}\settodepth{\Depth}{$y$}\setlength{\Height}{\Depth}%
\put(0.0000000,3.4433333){\hspace*{\Width}\raisebox{\Height}{$y$}}%
%
\settowidth{\Width}{O}\setlength{\Width}{-1\Width}%
\settoheight{\Height}{O}\settodepth{\Depth}{O}\setlength{\Height}{-\Height}%
\put(-0.0833333,-0.0833333){\hspace*{\Width}\raisebox{\Height}{O}}%
%
\end{picture}}%}}
\putnotese{30}{50}{\scalebox{0.7}{%%% /Users/takatoosetsuo/polytech22.git/205-0926/presen/fig/2kyokusen8_2.tex 
%%% Generator=graph22207.cdy 
{\unitlength=6mm%
\begin{picture}%
(3.98,4.86)(-0.5,-1.5)%
\linethickness{0.008in}%%
\linethickness{0.012in}%%
\polyline(0.50000,2.00000)(0.58335,2.12148)(0.66498,2.23238)(0.74489,2.33268)(0.82310,2.42241)%
(0.89959,2.50154)(0.97436,2.57009)(1.04743,2.62806)(1.11877,2.67543)(1.18841,2.71223)%
(1.25633,2.73843)(1.35715,2.75725)(1.45836,2.75431)(1.55983,2.73464)(1.66142,2.70327)%
(1.76300,2.66523)(1.86446,2.62554)(1.96566,2.58922)(2.06648,2.56132)(2.16678,2.54685)%
(2.26644,2.55084)(2.33335,2.56427)(2.40170,2.58469)(2.47148,2.61212)(2.54269,2.64654)%
(2.61533,2.68796)(2.68940,2.73637)(2.76490,2.79178)(2.84184,2.85419)(2.92020,2.92360)%
(3.00000,3.00000)%
%
\linethickness{0.008in}%%
\linethickness{0.012in}%%
\polyline(0.50000,-1.00000)(0.62029,-0.95083)(0.73419,-0.90648)(0.84169,-0.86694)%
(0.94279,-0.83220)(1.03749,-0.80228)(1.12580,-0.77717)(1.20771,-0.75686)(1.28322,-0.74137)%
(1.35234,-0.73069)(1.41506,-0.72482)(1.50354,-0.72649)(1.59054,-0.73984)(1.67620,-0.76180)%
(1.76069,-0.78928)(1.84418,-0.81922)(1.92681,-0.84853)(2.00875,-0.87414)(2.09015,-0.89297)%
(2.17119,-0.90194)(2.25201,-0.89798)(2.30795,-0.88713)(2.36808,-0.86984)(2.43240,-0.84612)%
(2.50092,-0.81597)(2.57362,-0.77939)(2.65051,-0.73638)(2.73160,-0.68693)(2.81688,-0.63105)%
(2.90634,-0.56874)(3.00000,-0.50000)%
%
\linethickness{0.008in}%%
\polyline(0.50000,2.00000)(0.50000,-1.00000)%
%
\polyline(3.00000,3.00000)(3.00000,-0.50000)%
%
\polyline(2.91702,0.00000)(3.00000,0.08298)%
%
\polyline(2.32777,-0.00000)(3.00000,0.67223)%
%
\polyline(1.73851,0.00000)(3.00000,1.26149)%
%
\polyline(1.14926,0.00000)(3.00000,1.85074)%
%
\polyline(0.56000,0.00000)(3.00000,2.44000)%
%
\polyline(0.50000,0.52926)(2.75598,2.78523)%
%
\polyline(0.50000,1.11851)(1.96962,2.58813)%
%
\polyline(0.50000,1.70777)(1.53222,2.73999)%
%
\polyline(-0.50000,0.00000)(3.47858,0.00000)%
%
\polyline(0.00000,-1.50000)(0.00000,3.35893)%
%
\settowidth{\Width}{$x$}\setlength{\Width}{0\Width}%
\settoheight{\Height}{$x$}\settodepth{\Depth}{$x$}\setlength{\Height}{-0.5\Height}\setlength{\Depth}{0.5\Depth}\addtolength{\Height}{\Depth}%
\put(3.5633333,0.0000000){\hspace*{\Width}\raisebox{\Height}{$x$}}%
%
\settowidth{\Width}{$y$}\setlength{\Width}{-0.5\Width}%
\settoheight{\Height}{$y$}\settodepth{\Depth}{$y$}\setlength{\Height}{\Depth}%
\put(0.0000000,3.4433333){\hspace*{\Width}\raisebox{\Height}{$y$}}%
%
\settowidth{\Width}{O}\setlength{\Width}{-1\Width}%
\settoheight{\Height}{O}\settodepth{\Depth}{O}\setlength{\Height}{-\Height}%
\put(-0.0833333,-0.0833333){\hspace*{\Width}\raisebox{\Height}{O}}%
%
\end{picture}}%}}
\putnotese{53}{50}{\scalebox{0.7}{%%% /Users/takatoosetsuo/polytech22.git/205-0926/presen/fig/2kyokusen8_3.tex 
%%% Generator=graph22207.cdy 
{\unitlength=6mm%
\begin{picture}%
(3.98,4.86)(-0.5,-1.5)%
\linethickness{0.008in}%%
\linethickness{0.012in}%%
\polyline(0.50000,2.00000)(0.58335,2.12148)(0.66498,2.23238)(0.74489,2.33268)(0.82310,2.42241)%
(0.89959,2.50154)(0.97436,2.57009)(1.04743,2.62806)(1.11877,2.67543)(1.18841,2.71223)%
(1.25633,2.73843)(1.35715,2.75725)(1.45836,2.75431)(1.55983,2.73464)(1.66142,2.70327)%
(1.76300,2.66523)(1.86446,2.62554)(1.96566,2.58922)(2.06648,2.56132)(2.16678,2.54685)%
(2.26644,2.55084)(2.33335,2.56427)(2.40170,2.58469)(2.47148,2.61212)(2.54269,2.64654)%
(2.61533,2.68796)(2.68940,2.73637)(2.76490,2.79178)(2.84184,2.85419)(2.92020,2.92360)%
(3.00000,3.00000)%
%
\linethickness{0.008in}%%
\linethickness{0.012in}%%
\polyline(0.50000,-1.00000)(0.62029,-0.95083)(0.73419,-0.90648)(0.84169,-0.86694)%
(0.94279,-0.83220)(1.03749,-0.80228)(1.12580,-0.77717)(1.20771,-0.75686)(1.28322,-0.74137)%
(1.35234,-0.73069)(1.41506,-0.72482)(1.50354,-0.72649)(1.59054,-0.73984)(1.67620,-0.76180)%
(1.76069,-0.78928)(1.84418,-0.81922)(1.92681,-0.84853)(2.00875,-0.87414)(2.09015,-0.89297)%
(2.17119,-0.90194)(2.25201,-0.89798)(2.30795,-0.88713)(2.36808,-0.86984)(2.43240,-0.84612)%
(2.50092,-0.81597)(2.57362,-0.77939)(2.65051,-0.73638)(2.73160,-0.68693)(2.81688,-0.63105)%
(2.90634,-0.56874)(3.00000,-0.50000)%
%
\linethickness{0.008in}%%
\polyline(0.50000,2.00000)(0.50000,-1.00000)%
%
\polyline(3.00000,3.00000)(3.00000,-0.50000)%
%
\polyline(2.03647,-0.88055)(2.91702,0.00000)%
%
\polyline(1.58828,-0.73949)(2.32777,-0.00000)%
%
\polyline(0.88722,-0.85129)(1.73851,0.00000)%
%
\polyline(0.50000,-0.64926)(1.14926,0.00000)%
%
\polyline(0.50000,-0.06000)(0.56000,0.00000)%
%
\polyline(-0.50000,0.00000)(3.47858,0.00000)%
%
\polyline(0.00000,-1.50000)(0.00000,3.35893)%
%
\settowidth{\Width}{$x$}\setlength{\Width}{0\Width}%
\settoheight{\Height}{$x$}\settodepth{\Depth}{$x$}\setlength{\Height}{-0.5\Height}\setlength{\Depth}{0.5\Depth}\addtolength{\Height}{\Depth}%
\put(3.5633333,0.0000000){\hspace*{\Width}\raisebox{\Height}{$x$}}%
%
\settowidth{\Width}{$y$}\setlength{\Width}{-0.5\Width}%
\settoheight{\Height}{$y$}\settodepth{\Depth}{$y$}\setlength{\Height}{\Depth}%
\put(0.0000000,3.4433333){\hspace*{\Width}\raisebox{\Height}{$y$}}%
%
\settowidth{\Width}{O}\setlength{\Width}{-1\Width}%
\settoheight{\Height}{O}\settodepth{\Depth}{O}\setlength{\Height}{-\Height}%
\put(-0.0833333,-0.0833333){\hspace*{\Width}\raisebox{\Height}{O}}%
%
\end{picture}}%}}
\putnotec{26}{58}{$=$}
\putnotec{50}{58}{$+$}
\putnotee{72}{58}{\normalsize$S=S_1+S_2=\dint_a^b(f(x)-g(x))\,dx$}
\putnotee{70}{45}{\color{red}上から下の関数を引いて積分}
\end{layer}

\vspace{1zw}

区間$a\leqq x\leqq b$で$f(x)\geqq g(x)$とする

\newslide{例題(2曲線で囲まれる図形)}

\vspace*{18mm}


\begin{layer}{120}{0}
\putnotew{96}{73}{\hyperlink{para9pg9}{\fbox{\Ctab{2.5mm}{\scalebox{1}{\scriptsize $\mathstrut||\!\lhd$}}}}}
\putnotew{101}{73}{\hyperlink{para10pg1}{\fbox{\Ctab{2.5mm}{\scalebox{1}{\scriptsize $\mathstrut|\!\lhd$}}}}}
\putnotew{108}{73}{\hyperlink{para10pg8}{\fbox{\Ctab{4.5mm}{\scalebox{1}{\scriptsize $\mathstrut\!\!\lhd\!\!$}}}}}
\putnotew{115}{73}{\hyperlink{para10pg9}{\fbox{\Ctab{4.5mm}{\scalebox{1}{\scriptsize $\mathstrut\!\rhd\!$}}}}}
\putnotew{120}{73}{\hyperlink{para10pg9}{\fbox{\Ctab{2.5mm}{\scalebox{1}{\scriptsize $\mathstrut \!\rhd\!\!|$}}}}}
\putnotew{125}{73}{\hyperlink{para11pg1}{\fbox{\Ctab{2.5mm}{\scalebox{1}{\scriptsize $\mathstrut \!\rhd\!\!||$}}}}}
\putnotee{126}{73}{\scriptsize\color{blue} 9/9}
\end{layer}

\slidepage

\begin{layer}{120}{0}
\putnotese{90}{20}{%%% /Users/takatoosetsuo/polytech22.git/205-0926/presen/fig/2kyokusen9.tex 
%%% Generator=graph22207.cdy 
{\unitlength=6mm%
\begin{picture}%
(3.97,4.62)(-0.49,-1.26)%
\linethickness{0.008in}%%
\polyline(-0.49000,1.22010)(-0.41060,0.98979)(-0.33120,0.77209)(-0.25180,0.56700)%
(-0.17240,0.37452)(-0.09300,0.19465)(-0.01360,0.02738)(0.06580,-0.12727)(0.14520,-0.26932)%
(0.22460,-0.39875)(0.30400,-0.51558)(0.38340,-0.61980)(0.46280,-0.71142)(0.54220,-0.79042)%
(0.62160,-0.85681)(0.70100,-0.91060)(0.78040,-0.95178)(0.85980,-0.98034)(0.93920,-0.99630)%
(1.01860,-0.99965)(1.09800,-0.99040)(1.17740,-0.96853)(1.25680,-0.93405)(1.33620,-0.88697)%
(1.41560,-0.82728)(1.49500,-0.75497)(1.57440,-0.67006)(1.65380,-0.57255)(1.73320,-0.46242)%
(1.81260,-0.33968)(1.89200,-0.20434)(1.97140,-0.05638)(2.05080,0.10418)(2.13020,0.27735)%
(2.20960,0.46313)(2.28900,0.66152)(2.36840,0.87252)(2.44780,1.09612)(2.52720,1.33234)%
(2.60660,1.58116)(2.68600,1.84260)(2.76540,2.11664)(2.84480,2.40329)(2.92420,2.70255)%
(3.00360,3.01441)(3.08300,3.33889)(3.08797,3.36000)%
%
\polyline(0.87000,-1.26000)(3.18000,3.36000)%
%
\polyline(1.00000,-1.00000)(1.00000,-0.85714)\polyline(1.00000,-0.71429)(1.00000,-0.57143)%
\polyline(1.00000,-0.42857)(1.00000,-0.28571)\polyline(1.00000,-0.14286)(1.00000,0.00000)%
%
%
\polyline(3.00000,0.00000)(3.00000,0.16000)\polyline(3.00000,0.32000)(3.00000,0.48000)%
\polyline(3.00000,0.64000)(3.00000,0.80000)\polyline(3.00000,0.96000)(3.00000,1.12000)%
\polyline(3.00000,1.28000)(3.00000,1.44000)\polyline(3.00000,1.60000)(3.00000,1.76000)%
\polyline(3.00000,1.92000)(3.00000,2.08000)\polyline(3.00000,2.24000)(3.00000,2.40000)%
\polyline(3.00000,2.56000)(3.00000,2.72000)\polyline(3.00000,2.88000)(3.00000,3.04000)%
\polyline(3.00000,3.20000)(3.00000,3.36000)%
%
\polyline(1.00000,0.08333)(1.00000,-0.08333)%
%
\settowidth{\Width}{$1$}\setlength{\Width}{-0.5\Width}%
\settoheight{\Height}{$1$}\settodepth{\Depth}{$1$}\setlength{\Height}{\Depth}%
\put(1.0000000,0.1666667){\hspace*{\Width}\raisebox{\Height}{$1$}}%
%
\polyline(3.00000,0.08333)(3.00000,-0.08333)%
%
\settowidth{\Width}{$3$}\setlength{\Width}{-0.5\Width}%
\settoheight{\Height}{$3$}\settodepth{\Depth}{$3$}\setlength{\Height}{-\Height}%
\put(3.0000000,-0.1666667){\hspace*{\Width}\raisebox{\Height}{$3$}}%
%
\polyline(1.06266,-0.87468)(1.16102,-0.97304)%
%
\polyline(1.16087,-0.67826)(1.35525,-0.87265)%
%
\polyline(1.25908,-0.48184)(1.51298,-0.73575)%
%
\polyline(1.35729,-0.28543)(1.64958,-0.57772)%
%
\polyline(1.45550,-0.08901)(1.77079,-0.40431)%
%
\polyline(1.55371,0.10741)(1.88218,-0.22107)%
%
\polyline(1.65191,0.30383)(1.98488,-0.02913)%
%
\polyline(1.75012,0.50025)(2.08079,0.16958)%
%
\polyline(1.84833,0.69667)(2.17135,0.37365)%
%
\polyline(1.94654,0.89309)(2.25730,0.58232)%
%
\polyline(2.04475,1.08950)(2.33924,0.79502)%
%
\polyline(2.14296,1.28592)(2.41765,1.01123)%
%
\polyline(2.24117,1.48234)(2.49298,1.23053)%
%
\polyline(2.33938,1.67876)(2.56557,1.45257)%
%
\polyline(2.43759,1.87518)(2.63572,1.67705)%
%
\polyline(2.53580,2.07160)(2.70370,1.90369)%
%
\polyline(2.63401,2.26802)(2.76974,2.13229)%
%
\polyline(2.73222,2.46443)(2.83364,2.36301)%
%
\polyline(2.83043,2.66085)(2.89579,2.59548)%
%
\polyline(2.92864,2.85727)(2.95650,2.82941)%
%
\polyline(-0.48973,0.00000)(3.47858,0.00000)%
%
\polyline(0.00000,-1.25874)(0.00000,3.35893)%
%
\settowidth{\Width}{$x$}\setlength{\Width}{0\Width}%
\settoheight{\Height}{$x$}\settodepth{\Depth}{$x$}\setlength{\Height}{-0.5\Height}\setlength{\Depth}{0.5\Depth}\addtolength{\Height}{\Depth}%
\put(3.5633333,0.0000000){\hspace*{\Width}\raisebox{\Height}{$x$}}%
%
\settowidth{\Width}{$y$}\setlength{\Width}{-0.5\Width}%
\settoheight{\Height}{$y$}\settodepth{\Depth}{$y$}\setlength{\Height}{\Depth}%
\put(0.0000000,3.4433333){\hspace*{\Width}\raisebox{\Height}{$y$}}%
%
\settowidth{\Width}{O}\setlength{\Width}{-1\Width}%
\settoheight{\Height}{O}\settodepth{\Depth}{O}\setlength{\Height}{-\Height}%
\put(-0.0833333,-0.0833333){\hspace*{\Width}\raisebox{\Height}{O}}%
%
\end{picture}}%}
\end{layer}

\begin{itemize}
\item
[例題)]$y=x^2-2x$と$y=2x-3$で囲まれる図形\vspace{-2mm}
\item
[解)]交点を求める\\
$x^2-2x-(2x-3)=0$より\\
 $x^2-4x+3=0\\
 (x-1)(x-3)=0$\\
これから $x=1,\ 3$\\
図より,$1\leqq x\leqq 3$のとき $2x-3 \leqq x^2-2x$\\
したがって $S=\dint_1^3(2x-3-x^2+2x)\,dx=\bunsuu{4}{3}$
\end{itemize}

\newslide{課題(曲線で囲まれる図形の面積1)}

\vspace*{18mm}


\begin{layer}{120}{0}
\putnotew{96}{73}{\hyperlink{para10pg9}{\fbox{\Ctab{2.5mm}{\scalebox{1}{\scriptsize $\mathstrut||\!\lhd$}}}}}
\putnotew{125}{73}{\hyperlink{para12pg1}{\fbox{\Ctab{2.5mm}{\scalebox{1}{\scriptsize $\mathstrut \!\rhd\!\!||$}}}}}
\putnotee{126}{73}{\scriptsize\color{blue} 1/1}
\end{layer}

\slidepage
\vspace{3mm}

アプリ「関数のグラフ」を用いよ.\vspace{-2mm}
\begin{itemize}
\item
[課題]\monban 曲線$y=x^2-4x$について,問いに答えよ\seteda{65}\\
\eda{曲線と$x$軸との交点の$x$座標を求めよ}\\
\eda{曲線と$x$軸で囲まれる図形の面積$S$を求めよ}
\item
[課題]\monban $y=-x^2+2$と$y=x$で囲まれる図形を考える\seteda{65}\\
\eda{曲線と直線の交点の$x$座標を求めよ}\\
\eda{面積$S$を積分で表せ}\\
\eda{面積$S$を求めよ}
\end{itemize}
%%%%%%%%%%%%%

%%%%%%%%%%%%%%%%%%%%


\newslide{課題(曲線で囲まれる図形の面積2)}

\vspace*{18mm}


\begin{layer}{120}{0}
\putnotew{96}{73}{\hyperlink{para11pg1}{\fbox{\Ctab{2.5mm}{\scalebox{1}{\scriptsize $\mathstrut||\!\lhd$}}}}}
\putnotew{125}{73}{\hyperlink{para13pg1}{\fbox{\Ctab{2.5mm}{\scalebox{1}{\scriptsize $\mathstrut \!\rhd\!\!||$}}}}}
\putnotee{126}{73}{\scriptsize\color{blue} 1/1}
\end{layer}

\slidepage
\vspace{3mm}

アプリ「関数のグラフ」を用いよ.\vspace{-2mm}
\begin{itemize}
\item
[課題]\monban $y=\sin x$と$y=\cos x$で囲まれる図形を考える.
ただし,$0\leqq x \leqq 2\pi$とする.\seteda{65}\\
\eda{2曲線の交点の$x$座標を求めよ}\\
\eda{面積$S$を積分で表せ}\\
\eda{面積$S$を求めよ}
\item
[注)]次の積分公式とアプリ「三角関数の値」を用いよ\\
\hspace*{1zw}$\dint \sin x\,dx=-\cos x,\ \dint \cos x\,dx=\sin x$
\end{itemize}
%%%%%%%%%%%%%

%%%%%%%%%%%%%%%%%%%%


\newslide{課題(曲線で囲まれる図形の面積3)}

\vspace*{18mm}


\begin{layer}{120}{0}
\putnotew{96}{73}{\hyperlink{para12pg1}{\fbox{\Ctab{2.5mm}{\scalebox{1}{\scriptsize $\mathstrut||\!\lhd$}}}}}
\putnotew{125}{73}{\hyperlink{para14pg1}{\fbox{\Ctab{2.5mm}{\scalebox{1}{\scriptsize $\mathstrut \!\rhd\!\!||$}}}}}
\putnotee{126}{73}{\scriptsize\color{blue} 1/1}
\end{layer}

\slidepage
\vspace{3mm}

アプリ「関数のグラフ」を用いよ.\vspace{-2mm}
\begin{itemize}
\item
[課題]\monban 曲線$y=e^x$,$y=e^{-x}$とy軸に平行な直線$x=1$で囲まれる図形を考える.
\seteda{65}\\
\eda{2曲線の交点の$x$座標を求めよ}\\
\eda{面積$S$を積分で表せ}\\
\eda{面積$S$を求めよ}
\item
[注)]次の積分公式を用いよ\\
\hspace*{1zw}$\dint e^x\,dx=e^x,\ \dint e^{-x}\,dx=-e^{-x}$
\end{itemize}
\label{pageend}\mbox{}

\end{document}

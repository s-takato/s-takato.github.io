%%% Title presen22106
\documentclass[landscape,10pt]{ujarticle}
\special{papersize=\the\paperwidth,\the\paperheight}
\usepackage{ketpic,ketlayer}
\usepackage{ketslide}
\usepackage{amsmath,amssymb}
\usepackage{bm,enumerate}
\usepackage[dvipdfmx]{graphicx}
\usepackage{color}
\definecolor{slidecolora}{cmyk}{0.98,0.13,0,0.43}
\definecolor{slidecolorb}{cmyk}{0.2,0,0,0}
\definecolor{slidecolorc}{cmyk}{0.2,0,0,0}
\definecolor{slidecolord}{cmyk}{0.2,0,0,0}
\definecolor{slidecolore}{cmyk}{0,0,0,0.5}
\definecolor{slidecolorf}{cmyk}{0,0,0,0.5}
\definecolor{slidecolori}{cmyk}{0.98,0.13,0,0.43}
\def\setthin#1{\def\thin{#1}}
\setthin{0}
\newcommand{\slidepage}[1][s]{%
\setcounter{ketpicctra}{18}%
\if#1m \setcounter{ketpicctra}{1}\fi
\hypersetup{linkcolor=black}%

\begin{layer}{118}{0}
\putnotee{122}{-\theketpicctra.05}{\small\thepage/\pageref{pageend}}
\end{layer}\hypersetup{linkcolor=blue}

}
\usepackage{emath}
\usepackage{emathEy}
\usepackage{emathMw}
\usepackage{pict2e}
\usepackage{ketlayermorewith2e}
\usepackage[dvipdfmx,colorlinks=true,linkcolor=blue,filecolor=blue]{hyperref}
\newcommand{\hiduke}{0523}
\newcommand{\hako}[2][1]{\fbox{\raisebox{#1mm}{\mbox{}}\raisebox{-#1mm}{\mbox{}}\,\phantom{#2}\,}}
\newcommand{\hakoa}[2][1]{\fbox{\raisebox{#1mm}{\mbox{}}\raisebox{-#1mm}{\mbox{}}\,#2\,}}
\newcommand{\hakom}[2][1]{\hako[#1]{$#2$}}
\newcommand{\hakoma}[2][1]{\hakoa[#1]{$#2$}}
\def\rad{\;\mathrm{rad}}
\def\deg#1{#1^{\circ}}
\newcommand{\sbunsuu}[2]{\scalebox{0.6}{$\bunsuu{#1}{#2}$}}
\def\pow{$\hspace{-1.5mm}^\hspace{-1mm}$}
\def\dlim{\displaystyle\lim}
\newcommand{\brd}[2][1]{\scalebox{#1}{\color{red}\fbox{\color{black}$#2$}}}
\newcommand\down[1][0.5zw]{\vspace{#1}\\}
\newcommand{\sfrac}[3][0.65]{\scalebox{#1}{$\frac{#2}{#3}$}}

\setmargin{25}{145}{15}{100}

\ketslideinit

\pagestyle{empty}

\begin{document}

\begin{layer}{120}{0}
\putnotese{0}{0}{{\Large\bf
\color[cmyk]{1,1,0,0}

\begin{layer}{120}{0}
{\Huge \putnotes{60}{20}{指数・対数関数}}
\putnotes{60}{70}{2022.05.23}
\end{layer}

}
}
\end{layer}

\def\mainslidetitley{22}
\def\ketcletter{slidecolora}
\def\ketcbox{slidecolorb}
\def\ketdbox{slidecolorc}
\def\ketcframe{slidecolord}
\def\ketcshadow{slidecolore}
\def\ketdshadow{slidecolorf}
\def\slidetitlex{6}
\def\slidetitlesize{1.3}
\def\mketcletter{slidecolori}
\def\mketcbox{yellow}
\def\mketdbox{yellow}
\def\mketcframe{yellow}
\def\mslidetitlex{62}
\def\mslidetitlesize{2}

\color{black}
\Large\bf\boldmath
\addtocounter{page}{-1}

\def\MARU{}
\renewcommand{\MARU}[1]{{\ooalign{\hfil$#1$\/\hfil\crcr\raise.167ex\hbox{\mathhexbox20D}}}}
\renewcommand{\slidepage}[1][s]{%
\setcounter{ketpicctra}{18}%
\if#1m \setcounter{ketpicctra}{1}\fi
\hypersetup{linkcolor=black}%
\begin{layer}{118}{0}
\putnotee{115}{-\theketpicctra.05}{\small\hiduke-\thepage/\pageref{pageend}}
\end{layer}\hypersetup{linkcolor=blue}
}
\newcounter{ban}
\setcounter{ban}{1}
\newcommand{\monban}[1][\hiduke]{%
#1-\theban\ %
\addtocounter{ban}{1}%
}
\newcommand{\monbannoadd}[1][\hiduke]{%
#1-\theban\ %
}
\newcommand{\addban}{%
\addtocounter{ban}{1}%%210614
}
\newcounter{edawidth}
\newcounter{edactr}
\newcommand{\seteda}[1]{
\setcounter{edawidth}{#1}
\setcounter{edactr}{1}
}
\newcommand{\eda}[2][\theedawidth ]{%
\noindent\Ltab{#1 mm}{[\theedactr]\ #2}%
\addtocounter{edactr}{1}%
}
%%%%%%%%%%%%

%%%%%%%%%%%%%%%%%%%%

\mainslide{指数関数}


\slidepage[m]
%%%%%%%%%%%%

%%%%%%%%%%%%%%%%%%%%

\newslide{指数関数$y=a^x$}

\vspace*{18mm}


\begin{layer}{120}{0}
\putnotew{96}{73}{\hyperlink{para0pg0}{\fbox{\Ctab{2.5mm}{\scalebox{1}{\scriptsize $\mathstrut||\!\lhd$}}}}}
\putnotew{101}{73}{\hyperlink{para1pg1}{\fbox{\Ctab{2.5mm}{\scalebox{1}{\scriptsize $\mathstrut|\!\lhd$}}}}}
\putnotew{108}{73}{\hyperlink{para1pg13}{\fbox{\Ctab{4.5mm}{\scalebox{1}{\scriptsize $\mathstrut\!\!\lhd\!\!$}}}}}
\putnotew{115}{73}{\hyperlink{para1pg14}{\fbox{\Ctab{4.5mm}{\scalebox{1}{\scriptsize $\mathstrut\!\rhd\!$}}}}}
\putnotew{120}{73}{\hyperlink{para1pg14}{\fbox{\Ctab{2.5mm}{\scalebox{1}{\scriptsize $\mathstrut \!\rhd\!\!|$}}}}}
\putnotew{125}{73}{\hyperlink{para2pg1}{\fbox{\Ctab{2.5mm}{\scalebox{1}{\scriptsize $\mathstrut \!\rhd\!\!||$}}}}}
\putnotee{126}{73}{\scriptsize\color{blue} 14/14}
\end{layer}

\slidepage
{\color{blue}

\begin{layer}{120}{0}
\putnotec{21}{47}{$2$}
\putnotec{29}{47}{$4$}
\putnotec{37}{47}{$8$}
\putnotec{46}{47}{$16$}
\putnotec{56}{47}{$32$}
\putnotec{66}{47}{$64$}
\putnotec{77}{47}{$128$}
\putnotec{89}{47}{$256$}
\putnotec{101}{47}{$512$}
\putnotec{113}{47}{$1024$}
\end{layer}

}
\begin{itemize}
\item
$a$は正の定数,$x$は変数
\item
$a$を{\color{red}底},$x$を{\color{red}(べき)指数}という.
\item
[(例)]$y=2^x$\vspace{2mm}\\
%%% /Users/takatoosetsuo/Dropbox/2021polytech/105/fig/sisuuhyo1.tex 
%%% Generator=105sisuu.cdy 
{\unitlength=1cm%
\begin{picture}%
(11,1.6)(0,0)%
\special{pn 8}%
%
\special{pa     0  -630}\special{pa     0    -0}%
\special{fp}%
\special{pa   315  -630}\special{pa   315    -0}%
\special{fp}%
\special{pa   630  -630}\special{pa   630    -0}%
\special{fp}%
\special{pa   945  -630}\special{pa   945    -0}%
\special{fp}%
\special{pa  1260  -630}\special{pa  1260    -0}%
\special{fp}%
\special{pa  1654  -630}\special{pa  1654    -0}%
\special{fp}%
\special{pa  2047  -630}\special{pa  2047    -0}%
\special{fp}%
\special{pa  2441  -630}\special{pa  2441    -0}%
\special{fp}%
\special{pa  2913  -630}\special{pa  2913    -0}%
\special{fp}%
\special{pa  3386  -630}\special{pa  3386    -0}%
\special{fp}%
\special{pa  3858  -630}\special{pa  3858    -0}%
\special{fp}%
\special{pa  4331  -630}\special{pa  4331    -0}%
\special{fp}%
\special{pa     0  -630}\special{pa  4331  -630}%
\special{fp}%
\special{pa     0  -315}\special{pa  4331  -315}%
\special{fp}%
\special{pa     0    -0}\special{pa  4331    -0}%
\special{fp}%
\settowidth{\Width}{$x$}\setlength{\Width}{-0.5\Width}%
\settoheight{\Height}{$x$}\settodepth{\Depth}{$x$}\setlength{\Height}{-0.5\Height}\setlength{\Depth}{0.5\Depth}\addtolength{\Height}{\Depth}%
\put(0.4000000,1.2000000){\hspace*{\Width}\raisebox{\Height}{$x$}}%
%
\settowidth{\Width}{$1$}\setlength{\Width}{-0.5\Width}%
\settoheight{\Height}{$1$}\settodepth{\Depth}{$1$}\setlength{\Height}{-0.5\Height}\setlength{\Depth}{0.5\Depth}\addtolength{\Height}{\Depth}%
\put(1.2000000,1.2000000){\hspace*{\Width}\raisebox{\Height}{$1$}}%
%
\settowidth{\Width}{$2$}\setlength{\Width}{-0.5\Width}%
\settoheight{\Height}{$2$}\settodepth{\Depth}{$2$}\setlength{\Height}{-0.5\Height}\setlength{\Depth}{0.5\Depth}\addtolength{\Height}{\Depth}%
\put(2.0000000,1.2000000){\hspace*{\Width}\raisebox{\Height}{$2$}}%
%
\settowidth{\Width}{$3$}\setlength{\Width}{-0.5\Width}%
\settoheight{\Height}{$3$}\settodepth{\Depth}{$3$}\setlength{\Height}{-0.5\Height}\setlength{\Depth}{0.5\Depth}\addtolength{\Height}{\Depth}%
\put(2.8000000,1.2000000){\hspace*{\Width}\raisebox{\Height}{$3$}}%
%
\settowidth{\Width}{$4$}\setlength{\Width}{-0.5\Width}%
\settoheight{\Height}{$4$}\settodepth{\Depth}{$4$}\setlength{\Height}{-0.5\Height}\setlength{\Depth}{0.5\Depth}\addtolength{\Height}{\Depth}%
\put(3.7000000,1.2000000){\hspace*{\Width}\raisebox{\Height}{$4$}}%
%
\settowidth{\Width}{$5$}\setlength{\Width}{-0.5\Width}%
\settoheight{\Height}{$5$}\settodepth{\Depth}{$5$}\setlength{\Height}{-0.5\Height}\setlength{\Depth}{0.5\Depth}\addtolength{\Height}{\Depth}%
\put(4.7000000,1.2000000){\hspace*{\Width}\raisebox{\Height}{$5$}}%
%
\settowidth{\Width}{$6$}\setlength{\Width}{-0.5\Width}%
\settoheight{\Height}{$6$}\settodepth{\Depth}{$6$}\setlength{\Height}{-0.5\Height}\setlength{\Depth}{0.5\Depth}\addtolength{\Height}{\Depth}%
\put(5.7000000,1.2000000){\hspace*{\Width}\raisebox{\Height}{$6$}}%
%
\settowidth{\Width}{$7$}\setlength{\Width}{-0.5\Width}%
\settoheight{\Height}{$7$}\settodepth{\Depth}{$7$}\setlength{\Height}{-0.5\Height}\setlength{\Depth}{0.5\Depth}\addtolength{\Height}{\Depth}%
\put(6.8000000,1.2000000){\hspace*{\Width}\raisebox{\Height}{$7$}}%
%
\settowidth{\Width}{$8$}\setlength{\Width}{-0.5\Width}%
\settoheight{\Height}{$8$}\settodepth{\Depth}{$8$}\setlength{\Height}{-0.5\Height}\setlength{\Depth}{0.5\Depth}\addtolength{\Height}{\Depth}%
\put(8.0000000,1.2000000){\hspace*{\Width}\raisebox{\Height}{$8$}}%
%
\settowidth{\Width}{$9$}\setlength{\Width}{-0.5\Width}%
\settoheight{\Height}{$9$}\settodepth{\Depth}{$9$}\setlength{\Height}{-0.5\Height}\setlength{\Depth}{0.5\Depth}\addtolength{\Height}{\Depth}%
\put(9.2000000,1.2000000){\hspace*{\Width}\raisebox{\Height}{$9$}}%
%
\settowidth{\Width}{$10$}\setlength{\Width}{-0.5\Width}%
\settoheight{\Height}{$10$}\settodepth{\Depth}{$10$}\setlength{\Height}{-0.5\Height}\setlength{\Depth}{0.5\Depth}\addtolength{\Height}{\Depth}%
\put(10.4000000,1.2000000){\hspace*{\Width}\raisebox{\Height}{$10$}}%
%
\settowidth{\Width}{$$}\setlength{\Width}{-0.5\Width}%
\settoheight{\Height}{$$}\settodepth{\Depth}{$$}\setlength{\Height}{-0.5\Height}\setlength{\Depth}{0.5\Depth}\addtolength{\Height}{\Depth}%
\put(0.4000000,1.2000000){\hspace*{\Width}\raisebox{\Height}{$$}}%
%
\settowidth{\Width}{$y$}\setlength{\Width}{-0.5\Width}%
\settoheight{\Height}{$y$}\settodepth{\Depth}{$y$}\setlength{\Height}{-0.5\Height}\setlength{\Depth}{0.5\Depth}\addtolength{\Height}{\Depth}%
\put(0.4000000,0.4000000){\hspace*{\Width}\raisebox{\Height}{$y$}}%
%
\end{picture}}%
\item
$x$が正の整数以外の場合でも$a^x$の値を定める
\end{itemize}

\newslide{指数法則}

\vspace*{18mm}


\begin{layer}{120}{0}
\putnotew{96}{73}{\hyperlink{para1pg14}{\fbox{\Ctab{2.5mm}{\scalebox{1}{\scriptsize $\mathstrut||\!\lhd$}}}}}
\putnotew{101}{73}{\hyperlink{para2pg1}{\fbox{\Ctab{2.5mm}{\scalebox{1}{\scriptsize $\mathstrut|\!\lhd$}}}}}
\putnotew{108}{73}{\hyperlink{para2pg13}{\fbox{\Ctab{4.5mm}{\scalebox{1}{\scriptsize $\mathstrut\!\!\lhd\!\!$}}}}}
\putnotew{115}{73}{\hyperlink{para2pg14}{\fbox{\Ctab{4.5mm}{\scalebox{1}{\scriptsize $\mathstrut\!\rhd\!$}}}}}
\putnotew{120}{73}{\hyperlink{para2pg14}{\fbox{\Ctab{2.5mm}{\scalebox{1}{\scriptsize $\mathstrut \!\rhd\!\!|$}}}}}
\putnotew{125}{73}{\hyperlink{para3pg1}{\fbox{\Ctab{2.5mm}{\scalebox{1}{\scriptsize $\mathstrut \!\rhd\!\!||$}}}}}
\putnotee{126}{73}{\scriptsize\color{blue} 14/14}
\end{layer}

\slidepage

\begin{layer}{120}{0}
\putnotese{30}{33}{
\fbox{$\begin{array}{l}
(1)\ \ a^p a^{q}=a^{p+q}\\
(2)\ \ (a^p)^{q}=a^{pq}\\
(3)\ \ (ab)^p=a^p b^p
\end{array}$}}
\putnotese{85}{43}{(指数法則)}
\end{layer}

\vspace*{-2mm}

\begin{itemize}
\item
元になるのは,指数の性質(指数法則)\\
\hspace*{0.5zw}$a^3a^2$
$=(aaa)(aa)$
$=a^5=a^{3+2}$
\ ({\color{red}指数の足し算})\\
\hspace*{0.5zw}$(a^3)^2$
$=(aaa)(aaa)$
$=a^6=a^{3\times 2}$
\ ({\color{red}指数の掛け算})
\hspace*{0.5zw}$(ab)^3$
$=(ab)(ab)(ab)$
$=a^3b^3$
\vspace{23mm}

\item
[課題]\monban 指数法則の具体例を書け\seteda{33}\\
\eda{(1)の例}\eda{(2)の例}\eda{(3)の例}
\end{itemize}

\newslide{指数の拡張1 $a$の0乗}

\vspace*{18mm}


\begin{layer}{120}{0}
\putnotew{96}{73}{\hyperlink{para2pg14}{\fbox{\Ctab{2.5mm}{\scalebox{1}{\scriptsize $\mathstrut||\!\lhd$}}}}}
\putnotew{101}{73}{\hyperlink{para3pg1}{\fbox{\Ctab{2.5mm}{\scalebox{1}{\scriptsize $\mathstrut|\!\lhd$}}}}}
\putnotew{108}{73}{\hyperlink{para3pg11}{\fbox{\Ctab{4.5mm}{\scalebox{1}{\scriptsize $\mathstrut\!\!\lhd\!\!$}}}}}
\putnotew{115}{73}{\hyperlink{para3pg12}{\fbox{\Ctab{4.5mm}{\scalebox{1}{\scriptsize $\mathstrut\!\rhd\!$}}}}}
\putnotew{120}{73}{\hyperlink{para3pg12}{\fbox{\Ctab{2.5mm}{\scalebox{1}{\scriptsize $\mathstrut \!\rhd\!\!|$}}}}}
\putnotew{125}{73}{\hyperlink{para4pg1}{\fbox{\Ctab{2.5mm}{\scalebox{1}{\scriptsize $\mathstrut \!\rhd\!\!||$}}}}}
\putnotee{126}{73}{\scriptsize\color{blue} 12/12}
\end{layer}

\slidepage
{\color{blue}

\begin{layer}{120}{0}
\lineseg{18}{43}{5}{-45}
\lineseg{36}{43}{5}{-45}
\end{layer}

}
\begin{itemize}
\item
指数法則が成り立つように正の整数以外の指数を定める
\item
$a\neqq 0$とする.
\item
$a^p a^{q}=a^{p+q}$
 $p=1,\ q=0$とすると\\
\hspace*{2zw}$a^1a^0=a^{1+0}$\\
\hspace*{2zw}$a\;a^0=a$\\
$a$は0でないから,両辺を$a$で割って\\
\hspace*{2zw}{\color{red}\fbox{$a^0=1$}}
\item
[(例)]$2^0=1,\ 3^0=1,\ 10^0=1$
\end{itemize}

\newslide{指数の拡張2 $a$のマイナス乗}

\vspace*{18mm}


\begin{layer}{120}{0}
\putnotew{96}{73}{\hyperlink{para3pg12}{\fbox{\Ctab{2.5mm}{\scalebox{1}{\scriptsize $\mathstrut||\!\lhd$}}}}}
\putnotew{101}{73}{\hyperlink{para4pg1}{\fbox{\Ctab{2.5mm}{\scalebox{1}{\scriptsize $\mathstrut|\!\lhd$}}}}}
\putnotew{108}{73}{\hyperlink{para4pg12}{\fbox{\Ctab{4.5mm}{\scalebox{1}{\scriptsize $\mathstrut\!\!\lhd\!\!$}}}}}
\putnotew{115}{73}{\hyperlink{para4pg13}{\fbox{\Ctab{4.5mm}{\scalebox{1}{\scriptsize $\mathstrut\!\rhd\!$}}}}}
\putnotew{120}{73}{\hyperlink{para4pg13}{\fbox{\Ctab{2.5mm}{\scalebox{1}{\scriptsize $\mathstrut \!\rhd\!\!|$}}}}}
\putnotew{125}{73}{\hyperlink{para5pg1}{\fbox{\Ctab{2.5mm}{\scalebox{1}{\scriptsize $\mathstrut \!\rhd\!\!||$}}}}}
\putnotee{126}{73}{\scriptsize\color{blue} 13/13}
\end{layer}

\slidepage
\begin{itemize}
\item
$a\neqq 0$とする.
\item
$a^p a^{q}=a^{p+q}$
 $q=-p$とすると\\
\hspace*{2zw}$a^p a^{-p}=a^{p+(-p)}$
$=a^0=1$\\
\hspace*{2zw}$a^p a^{-q}=1$\\
両辺を$a^p$で割って\hspace*{1zw}{\color{red}\fbox{$a^{-p}=\bunsuu{1}{a^p}$}}
\item
[(例)]$2^{-1}=\bunsuu{1}{2},\ 3^{-2}=\bunsuu{1}{3^2}=\bunsuu{1}{9}$
\item
[課題]\monban $5^0,4^{-1},2^{-2},3^{-3}$の値を求めよ.
\end{itemize}

\newslide{指数関数の表}

\vspace*{18mm}


\begin{layer}{120}{0}
\putnotew{96}{73}{\hyperlink{para4pg13}{\fbox{\Ctab{2.5mm}{\scalebox{1}{\scriptsize $\mathstrut||\!\lhd$}}}}}
\putnotew{101}{73}{\hyperlink{para5pg1}{\fbox{\Ctab{2.5mm}{\scalebox{1}{\scriptsize $\mathstrut|\!\lhd$}}}}}
\putnotew{108}{73}{\hyperlink{para5pg12}{\fbox{\Ctab{4.5mm}{\scalebox{1}{\scriptsize $\mathstrut\!\!\lhd\!\!$}}}}}
\putnotew{115}{73}{\hyperlink{para5pg13}{\fbox{\Ctab{4.5mm}{\scalebox{1}{\scriptsize $\mathstrut\!\rhd\!$}}}}}
\putnotew{120}{73}{\hyperlink{para5pg13}{\fbox{\Ctab{2.5mm}{\scalebox{1}{\scriptsize $\mathstrut \!\rhd\!\!|$}}}}}
\putnotew{125}{73}{\hyperlink{para6pg1}{\fbox{\Ctab{2.5mm}{\scalebox{1}{\scriptsize $\mathstrut \!\rhd\!\!||$}}}}}
\putnotee{126}{73}{\scriptsize\color{blue} 13/13}
\end{layer}

\slidepage

\begin{layer}{120}{0}
\putnotese{5}{5}{%%% /Users/takatoosetsuo/Dropbox/2021polytech/105/fig/sisuuhyo1.tex 
%%% Generator=105sisuu.cdy 
{\unitlength=1cm%
\begin{picture}%
(11,1.6)(0,0)%
\special{pn 8}%
%
\special{pa     0  -630}\special{pa     0    -0}%
\special{fp}%
\special{pa   315  -630}\special{pa   315    -0}%
\special{fp}%
\special{pa   630  -630}\special{pa   630    -0}%
\special{fp}%
\special{pa   945  -630}\special{pa   945    -0}%
\special{fp}%
\special{pa  1260  -630}\special{pa  1260    -0}%
\special{fp}%
\special{pa  1654  -630}\special{pa  1654    -0}%
\special{fp}%
\special{pa  2047  -630}\special{pa  2047    -0}%
\special{fp}%
\special{pa  2441  -630}\special{pa  2441    -0}%
\special{fp}%
\special{pa  2913  -630}\special{pa  2913    -0}%
\special{fp}%
\special{pa  3386  -630}\special{pa  3386    -0}%
\special{fp}%
\special{pa  3858  -630}\special{pa  3858    -0}%
\special{fp}%
\special{pa  4331  -630}\special{pa  4331    -0}%
\special{fp}%
\special{pa     0  -630}\special{pa  4331  -630}%
\special{fp}%
\special{pa     0  -315}\special{pa  4331  -315}%
\special{fp}%
\special{pa     0    -0}\special{pa  4331    -0}%
\special{fp}%
\settowidth{\Width}{$x$}\setlength{\Width}{-0.5\Width}%
\settoheight{\Height}{$x$}\settodepth{\Depth}{$x$}\setlength{\Height}{-0.5\Height}\setlength{\Depth}{0.5\Depth}\addtolength{\Height}{\Depth}%
\put(0.4000000,1.2000000){\hspace*{\Width}\raisebox{\Height}{$x$}}%
%
\settowidth{\Width}{$1$}\setlength{\Width}{-0.5\Width}%
\settoheight{\Height}{$1$}\settodepth{\Depth}{$1$}\setlength{\Height}{-0.5\Height}\setlength{\Depth}{0.5\Depth}\addtolength{\Height}{\Depth}%
\put(1.2000000,1.2000000){\hspace*{\Width}\raisebox{\Height}{$1$}}%
%
\settowidth{\Width}{$2$}\setlength{\Width}{-0.5\Width}%
\settoheight{\Height}{$2$}\settodepth{\Depth}{$2$}\setlength{\Height}{-0.5\Height}\setlength{\Depth}{0.5\Depth}\addtolength{\Height}{\Depth}%
\put(2.0000000,1.2000000){\hspace*{\Width}\raisebox{\Height}{$2$}}%
%
\settowidth{\Width}{$3$}\setlength{\Width}{-0.5\Width}%
\settoheight{\Height}{$3$}\settodepth{\Depth}{$3$}\setlength{\Height}{-0.5\Height}\setlength{\Depth}{0.5\Depth}\addtolength{\Height}{\Depth}%
\put(2.8000000,1.2000000){\hspace*{\Width}\raisebox{\Height}{$3$}}%
%
\settowidth{\Width}{$4$}\setlength{\Width}{-0.5\Width}%
\settoheight{\Height}{$4$}\settodepth{\Depth}{$4$}\setlength{\Height}{-0.5\Height}\setlength{\Depth}{0.5\Depth}\addtolength{\Height}{\Depth}%
\put(3.7000000,1.2000000){\hspace*{\Width}\raisebox{\Height}{$4$}}%
%
\settowidth{\Width}{$5$}\setlength{\Width}{-0.5\Width}%
\settoheight{\Height}{$5$}\settodepth{\Depth}{$5$}\setlength{\Height}{-0.5\Height}\setlength{\Depth}{0.5\Depth}\addtolength{\Height}{\Depth}%
\put(4.7000000,1.2000000){\hspace*{\Width}\raisebox{\Height}{$5$}}%
%
\settowidth{\Width}{$6$}\setlength{\Width}{-0.5\Width}%
\settoheight{\Height}{$6$}\settodepth{\Depth}{$6$}\setlength{\Height}{-0.5\Height}\setlength{\Depth}{0.5\Depth}\addtolength{\Height}{\Depth}%
\put(5.7000000,1.2000000){\hspace*{\Width}\raisebox{\Height}{$6$}}%
%
\settowidth{\Width}{$7$}\setlength{\Width}{-0.5\Width}%
\settoheight{\Height}{$7$}\settodepth{\Depth}{$7$}\setlength{\Height}{-0.5\Height}\setlength{\Depth}{0.5\Depth}\addtolength{\Height}{\Depth}%
\put(6.8000000,1.2000000){\hspace*{\Width}\raisebox{\Height}{$7$}}%
%
\settowidth{\Width}{$8$}\setlength{\Width}{-0.5\Width}%
\settoheight{\Height}{$8$}\settodepth{\Depth}{$8$}\setlength{\Height}{-0.5\Height}\setlength{\Depth}{0.5\Depth}\addtolength{\Height}{\Depth}%
\put(8.0000000,1.2000000){\hspace*{\Width}\raisebox{\Height}{$8$}}%
%
\settowidth{\Width}{$9$}\setlength{\Width}{-0.5\Width}%
\settoheight{\Height}{$9$}\settodepth{\Depth}{$9$}\setlength{\Height}{-0.5\Height}\setlength{\Depth}{0.5\Depth}\addtolength{\Height}{\Depth}%
\put(9.2000000,1.2000000){\hspace*{\Width}\raisebox{\Height}{$9$}}%
%
\settowidth{\Width}{$10$}\setlength{\Width}{-0.5\Width}%
\settoheight{\Height}{$10$}\settodepth{\Depth}{$10$}\setlength{\Height}{-0.5\Height}\setlength{\Depth}{0.5\Depth}\addtolength{\Height}{\Depth}%
\put(10.4000000,1.2000000){\hspace*{\Width}\raisebox{\Height}{$10$}}%
%
\settowidth{\Width}{$$}\setlength{\Width}{-0.5\Width}%
\settoheight{\Height}{$$}\settodepth{\Depth}{$$}\setlength{\Height}{-0.5\Height}\setlength{\Depth}{0.5\Depth}\addtolength{\Height}{\Depth}%
\put(0.4000000,1.2000000){\hspace*{\Width}\raisebox{\Height}{$$}}%
%
\settowidth{\Width}{$y$}\setlength{\Width}{-0.5\Width}%
\settoheight{\Height}{$y$}\settodepth{\Depth}{$y$}\setlength{\Height}{-0.5\Height}\setlength{\Depth}{0.5\Depth}\addtolength{\Height}{\Depth}%
\put(0.4000000,0.4000000){\hspace*{\Width}\raisebox{\Height}{$y$}}%
%
\end{picture}}%}
\putnotese{5}{25}{%%% /Users/takatoosetsuo/Dropbox/2021polytech/105/fig/sisuuhyo2.tex 
%%% Generator=105sisuu.cdy 
{\unitlength=1cm%
\begin{picture}%
(11.8,2)(0,0)%
\special{pn 8}%
%
\special{pa     0  -787}\special{pa     0    -0}%
\special{fp}%
\special{pa   315  -787}\special{pa   315    -0}%
\special{fp}%
\special{pa   787  -787}\special{pa   787    -0}%
\special{fp}%
\special{pa  1260  -787}\special{pa  1260    -0}%
\special{fp}%
\special{pa  1732  -787}\special{pa  1732    -0}%
\special{fp}%
\special{pa  2205  -787}\special{pa  2205    -0}%
\special{fp}%
\special{pa  2598  -787}\special{pa  2598    -0}%
\special{fp}%
\special{pa  2992  -787}\special{pa  2992    -0}%
\special{fp}%
\special{pa  3386  -787}\special{pa  3386    -0}%
\special{fp}%
\special{pa  3701  -787}\special{pa  3701    -0}%
\special{fp}%
\special{pa  4016  -787}\special{pa  4016    -0}%
\special{fp}%
\special{pa  4331  -787}\special{pa  4331    -0}%
\special{fp}%
\special{pa  4646  -787}\special{pa  4646    -0}%
\special{fp}%
\special{pa     0  -787}\special{pa  4646  -787}%
\special{fp}%
\special{pa     0  -472}\special{pa  4646  -472}%
\special{fp}%
\special{pa     0    -0}\special{pa  4646    -0}%
\special{fp}%
\settowidth{\Width}{$x$}\setlength{\Width}{-0.5\Width}%
\settoheight{\Height}{$x$}\settodepth{\Depth}{$x$}\setlength{\Height}{-0.5\Height}\setlength{\Depth}{0.5\Depth}\addtolength{\Height}{\Depth}%
\put(0.4000000,1.6000000){\hspace*{\Width}\raisebox{\Height}{$x$}}%
%
\settowidth{\Width}{$-10$}\setlength{\Width}{-0.5\Width}%
\settoheight{\Height}{$-10$}\settodepth{\Depth}{$-10$}\setlength{\Height}{-0.5\Height}\setlength{\Depth}{0.5\Depth}\addtolength{\Height}{\Depth}%
\put(1.4000000,1.6000000){\hspace*{\Width}\raisebox{\Height}{$-10$}}%
%
\settowidth{\Width}{$-9$}\setlength{\Width}{-0.5\Width}%
\settoheight{\Height}{$-9$}\settodepth{\Depth}{$-9$}\setlength{\Height}{-0.5\Height}\setlength{\Depth}{0.5\Depth}\addtolength{\Height}{\Depth}%
\put(2.6000000,1.6000000){\hspace*{\Width}\raisebox{\Height}{$-9$}}%
%
\settowidth{\Width}{$-8$}\setlength{\Width}{-0.5\Width}%
\settoheight{\Height}{$-8$}\settodepth{\Depth}{$-8$}\setlength{\Height}{-0.5\Height}\setlength{\Depth}{0.5\Depth}\addtolength{\Height}{\Depth}%
\put(3.8000000,1.6000000){\hspace*{\Width}\raisebox{\Height}{$-8$}}%
%
\settowidth{\Width}{$-7$}\setlength{\Width}{-0.5\Width}%
\settoheight{\Height}{$-7$}\settodepth{\Depth}{$-7$}\setlength{\Height}{-0.5\Height}\setlength{\Depth}{0.5\Depth}\addtolength{\Height}{\Depth}%
\put(5.0000000,1.6000000){\hspace*{\Width}\raisebox{\Height}{$-7$}}%
%
\settowidth{\Width}{$-6$}\setlength{\Width}{-0.5\Width}%
\settoheight{\Height}{$-6$}\settodepth{\Depth}{$-6$}\setlength{\Height}{-0.5\Height}\setlength{\Depth}{0.5\Depth}\addtolength{\Height}{\Depth}%
\put(6.1000000,1.6000000){\hspace*{\Width}\raisebox{\Height}{$-6$}}%
%
\settowidth{\Width}{$-5$}\setlength{\Width}{-0.5\Width}%
\settoheight{\Height}{$-5$}\settodepth{\Depth}{$-5$}\setlength{\Height}{-0.5\Height}\setlength{\Depth}{0.5\Depth}\addtolength{\Height}{\Depth}%
\put(7.1000000,1.6000000){\hspace*{\Width}\raisebox{\Height}{$-5$}}%
%
\settowidth{\Width}{$-4$}\setlength{\Width}{-0.5\Width}%
\settoheight{\Height}{$-4$}\settodepth{\Depth}{$-4$}\setlength{\Height}{-0.5\Height}\setlength{\Depth}{0.5\Depth}\addtolength{\Height}{\Depth}%
\put(8.1000000,1.6000000){\hspace*{\Width}\raisebox{\Height}{$-4$}}%
%
\settowidth{\Width}{$-3$}\setlength{\Width}{-0.5\Width}%
\settoheight{\Height}{$-3$}\settodepth{\Depth}{$-3$}\setlength{\Height}{-0.5\Height}\setlength{\Depth}{0.5\Depth}\addtolength{\Height}{\Depth}%
\put(9.0000000,1.6000000){\hspace*{\Width}\raisebox{\Height}{$-3$}}%
%
\settowidth{\Width}{$-2$}\setlength{\Width}{-0.5\Width}%
\settoheight{\Height}{$-2$}\settodepth{\Depth}{$-2$}\setlength{\Height}{-0.5\Height}\setlength{\Depth}{0.5\Depth}\addtolength{\Height}{\Depth}%
\put(9.8000000,1.6000000){\hspace*{\Width}\raisebox{\Height}{$-2$}}%
%
\settowidth{\Width}{$-1$}\setlength{\Width}{-0.5\Width}%
\settoheight{\Height}{$-1$}\settodepth{\Depth}{$-1$}\setlength{\Height}{-0.5\Height}\setlength{\Depth}{0.5\Depth}\addtolength{\Height}{\Depth}%
\put(10.6000000,1.6000000){\hspace*{\Width}\raisebox{\Height}{$-1$}}%
%
\settowidth{\Width}{$0$}\setlength{\Width}{-0.5\Width}%
\settoheight{\Height}{$0$}\settodepth{\Depth}{$0$}\setlength{\Height}{-0.5\Height}\setlength{\Depth}{0.5\Depth}\addtolength{\Height}{\Depth}%
\put(11.4000000,1.6000000){\hspace*{\Width}\raisebox{\Height}{$0$}}%
%
\settowidth{\Width}{$$}\setlength{\Width}{-0.5\Width}%
\settoheight{\Height}{$$}\settodepth{\Depth}{$$}\setlength{\Height}{-0.5\Height}\setlength{\Depth}{0.5\Depth}\addtolength{\Height}{\Depth}%
\put(0.4000000,1.6000000){\hspace*{\Width}\raisebox{\Height}{$$}}%
%
\settowidth{\Width}{$y$}\setlength{\Width}{-0.5\Width}%
\settoheight{\Height}{$y$}\settodepth{\Depth}{$y$}\setlength{\Height}{-0.5\Height}\setlength{\Depth}{0.5\Depth}\addtolength{\Height}{\Depth}%
\put(0.4000000,0.6000000){\hspace*{\Width}\raisebox{\Height}{$y$}}%
%
\end{picture}}%}
\end{layer}

{\color{blue}

\begin{layer}{120}{0}
\putnotec{17}{17}{$2$}
\putnotec{25}{17}{$4$}
\putnotec{34}{17}{$8$}
\putnotec{42}{17}{$16$}
\putnotec{52}{17}{$32$}
\putnotec{62}{17}{$64$}
\putnotec{73}{17}{$128$}
\putnotec{85}{17}{$256$}
\putnotec{97}{17}{$512$}
\putnotec{109}{17}{$1024$}
\putnotec{119}{39}{$1$}
\putnotec{111}{39}{$\bunsuu{1}{2}$}
\putnotec{103}{39}{$\bunsuu{1}{4}$}
\putnotec{95}{39}{$\bunsuu{1}{8}$}
\putnotec{86}{39}{$\bunsuu{1}{16}$}
\putnotec{76}{39}{$\bunsuu{1}{32}$}
\putnotec{66}{39}{$\bunsuu{1}{64}$}
\putnotec{55}{39}{$\bunsuu{1}{128}$}
\putnotec{43}{39}{$\bunsuu{1}{256}$}
\putnotec{31}{39}{$\bunsuu{1}{512}$}
\putnotec{19}{39}{$\bunsuu{1}{1024}$}
\end{layer}

}

\newslide{指数関数のグラフ}

\vspace*{18mm}


\begin{layer}{120}{0}
\putnotew{96}{73}{\hyperlink{para5pg13}{\fbox{\Ctab{2.5mm}{\scalebox{1}{\scriptsize $\mathstrut||\!\lhd$}}}}}
\putnotew{101}{73}{\hyperlink{para6pg1}{\fbox{\Ctab{2.5mm}{\scalebox{1}{\scriptsize $\mathstrut|\!\lhd$}}}}}
\putnotew{108}{73}{\hyperlink{para6pg1}{\fbox{\Ctab{4.5mm}{\scalebox{1}{\scriptsize $\mathstrut\!\!\lhd\!\!$}}}}}
\putnotew{115}{73}{\hyperlink{para6pg2}{\fbox{\Ctab{4.5mm}{\scalebox{1}{\scriptsize $\mathstrut\!\rhd\!$}}}}}
\putnotew{120}{73}{\hyperlink{para6pg2}{\fbox{\Ctab{2.5mm}{\scalebox{1}{\scriptsize $\mathstrut \!\rhd\!\!|$}}}}}
\putnotew{125}{73}{\hyperlink{para7pg1}{\fbox{\Ctab{2.5mm}{\scalebox{1}{\scriptsize $\mathstrut \!\rhd\!\!||$}}}}}
\putnotee{126}{73}{\scriptsize\color{blue} 2/2}
\end{layer}

\slidepage
\begin{itemize}
\item
アプリ「指数関数のグラフ」を用いる
\begin{enumerate}[(1)]
\item
下にある点を$y=2^x$の上に動かそう.
\item
$y=2^x$のグラフをかこう.
\end{enumerate}
\item
[課題]\monban $2^{-2},2^{-1},2^0,2^1,2^2,2^3$の値を書け
\end{itemize}

\newslide{$x$が整数でない場合(グラフから)}

\vspace*{18mm}


\begin{layer}{120}{0}
\putnotew{96}{73}{\hyperlink{para6pg2}{\fbox{\Ctab{2.5mm}{\scalebox{1}{\scriptsize $\mathstrut||\!\lhd$}}}}}
\putnotew{101}{73}{\hyperlink{para7pg1}{\fbox{\Ctab{2.5mm}{\scalebox{1}{\scriptsize $\mathstrut|\!\lhd$}}}}}
\putnotew{108}{73}{\hyperlink{para7pg3}{\fbox{\Ctab{4.5mm}{\scalebox{1}{\scriptsize $\mathstrut\!\!\lhd\!\!$}}}}}
\putnotew{115}{73}{\hyperlink{para7pg4}{\fbox{\Ctab{4.5mm}{\scalebox{1}{\scriptsize $\mathstrut\!\rhd\!$}}}}}
\putnotew{120}{73}{\hyperlink{para7pg4}{\fbox{\Ctab{2.5mm}{\scalebox{1}{\scriptsize $\mathstrut \!\rhd\!\!|$}}}}}
\putnotew{125}{73}{\hyperlink{para8pg1}{\fbox{\Ctab{2.5mm}{\scalebox{1}{\scriptsize $\mathstrut \!\rhd\!\!||$}}}}}
\putnotee{126}{73}{\scriptsize\color{blue} 4/4}
\end{layer}

\slidepage

\begin{layer}{120}{0}
\putnotes{90}{5}{\scalebox{0.9}{%%% /polytech.git/n104/fig/sisuugraph2.tex 
%%% Generator=n104sisuu.cdy 
{\unitlength=1cm%
\begin{picture}%
(8,7.5)(-4,-0.5)%
\special{pn 8}%
%
\special{pn 4}%
\special{pa -1575    -0}\special{pa -1575 -2756}%
\special{fp}%
\special{pn 8}%
\special{pn 4}%
\special{pa -1181    -0}\special{pa -1181 -2756}%
\special{fp}%
\special{pn 8}%
\special{pn 4}%
\special{pa  -787    -0}\special{pa  -787 -2756}%
\special{fp}%
\special{pn 8}%
\special{pn 4}%
\special{pa  -394    -0}\special{pa  -394 -2756}%
\special{fp}%
\special{pn 8}%
\special{pn 4}%
\special{pa     0    -0}\special{pa     0 -2756}%
\special{fp}%
\special{pn 8}%
\special{pn 4}%
\special{pa   394    -0}\special{pa   394 -2756}%
\special{fp}%
\special{pn 8}%
\special{pn 4}%
\special{pa   787    -0}\special{pa   787 -2756}%
\special{fp}%
\special{pn 8}%
\special{pn 4}%
\special{pa  1181    -0}\special{pa  1181 -2756}%
\special{fp}%
\special{pn 8}%
\special{pn 4}%
\special{pa  1575    -0}\special{pa  1575 -2756}%
\special{fp}%
\special{pn 8}%
\special{pn 4}%
\special{pa -1575  -394}\special{pa  1575  -394}%
\special{fp}%
\special{pn 8}%
\special{pn 4}%
\special{pa -1575  -787}\special{pa  1575  -787}%
\special{fp}%
\special{pn 8}%
\special{pn 4}%
\special{pa -1575 -1181}\special{pa  1575 -1181}%
\special{fp}%
\special{pn 8}%
\special{pn 4}%
\special{pa -1575 -1575}\special{pa  1575 -1575}%
\special{fp}%
\special{pn 8}%
\special{pn 4}%
\special{pa -1575 -1969}\special{pa  1575 -1969}%
\special{fp}%
\special{pn 8}%
\special{pn 4}%
\special{pa -1575 -2362}\special{pa  1575 -2362}%
\special{fp}%
\special{pn 8}%
{%
\color[cmyk]{0,1,1,0}%
\special{pa -1559 -24}\special{pa -1561 -30}\special{pa -1565 -36}\special{pa -1571 -39}%
\special{pa -1578 -39}\special{pa -1585 -36}\special{pa -1589 -30}\special{pa -1591 -24}%
\special{pa -1589 -17}\special{pa -1585 -11}\special{pa -1578 -8}\special{pa -1571 -8}%
\special{pa -1565 -11}\special{pa -1561 -17}\special{pa -1559 -24}\special{pa -1559 -24}%
\special{sh 1}\special{ip}%
}%
\special{pn 4}%
\special{pa -1575 -2756}\special{pa  1575 -2756}%
\special{fp}%
\special{pn 8}%
{%
\color[cmyk]{0,1,1,0}%
\special{pa -1559   -24}\special{pa -1561   -30}\special{pa -1565   -36}\special{pa -1571   -39}%
\special{pa -1575   -39}%
\special{fp}%
\special{pa -1575    -8}\special{pa -1571    -8}\special{pa -1565   -11}\special{pa -1561   -17}%
\special{pa -1559   -24}%
\special{fp}%
{%
\color[cmyk]{0,1,1,0}%
\special{pa -1165 -47}\special{pa -1167 -54}\special{pa -1171 -60}\special{pa -1178 -63}%
\special{pa -1185 -63}\special{pa -1191 -60}\special{pa -1195 -54}\special{pa -1197 -47}%
\special{pa -1195 -40}\special{pa -1191 -35}\special{pa -1185 -32}\special{pa -1178 -32}%
\special{pa -1171 -35}\special{pa -1167 -40}\special{pa -1165 -47}\special{pa -1165 -47}%
\special{sh 1}\special{ip}%
}%
}%
{%
\color[cmyk]{0,1,1,0}%
\special{pa -1165   -47}\special{pa -1167   -54}\special{pa -1171   -60}\special{pa -1178   -63}%
\special{pa -1185   -63}\special{pa -1191   -60}\special{pa -1195   -54}\special{pa -1197   -47}%
\special{pa -1195   -40}\special{pa -1191   -35}\special{pa -1185   -32}\special{pa -1178   -32}%
\special{pa -1171   -35}\special{pa -1167   -40}\special{pa -1165   -47}%
\special{fp}%
{%
\color[cmyk]{0,1,1,0}%
\special{pa -772 -98}\special{pa -773 -105}\special{pa -778 -111}\special{pa -784 -114}%
\special{pa -791 -114}\special{pa -797 -111}\special{pa -802 -105}\special{pa -803 -98}%
\special{pa -802 -92}\special{pa -797 -86}\special{pa -791 -83}\special{pa -784 -83}%
\special{pa -778 -86}\special{pa -773 -92}\special{pa -772 -98}\special{pa -772 -98}%
\special{sh 1}\special{ip}%
}%
}%
{%
\color[cmyk]{0,1,1,0}%
\special{pa  -772   -98}\special{pa  -773  -105}\special{pa  -778  -111}\special{pa  -784  -114}%
\special{pa  -791  -114}\special{pa  -797  -111}\special{pa  -802  -105}\special{pa  -803   -98}%
\special{pa  -802   -92}\special{pa  -797   -86}\special{pa  -791   -83}\special{pa  -784   -83}%
\special{pa  -778   -86}\special{pa  -773   -92}\special{pa  -772   -98}%
\special{fp}%
{%
\color[cmyk]{0,1,1,0}%
\special{pa -378 -197}\special{pa -380 -204}\special{pa -384 -209}\special{pa -390 -212}%
\special{pa -397 -212}\special{pa -404 -209}\special{pa -408 -204}\special{pa -409 -197}%
\special{pa -408 -190}\special{pa -404 -185}\special{pa -397 -181}\special{pa -390 -181}%
\special{pa -384 -185}\special{pa -380 -190}\special{pa -378 -197}\special{pa -378 -197}%
\special{sh 1}\special{ip}%
}%
}%
{%
\color[cmyk]{0,1,1,0}%
\special{pa  -378  -197}\special{pa  -380  -204}\special{pa  -384  -209}\special{pa  -390  -212}%
\special{pa  -397  -212}\special{pa  -404  -209}\special{pa  -408  -204}\special{pa  -409  -197}%
\special{pa  -408  -190}\special{pa  -404  -185}\special{pa  -397  -181}\special{pa  -390  -181}%
\special{pa  -384  -185}\special{pa  -380  -190}\special{pa  -378  -197}%
\special{fp}%
{%
\color[cmyk]{0,1,1,0}%
\special{pa 16 -394}\special{pa 14 -401}\special{pa 10 -406}\special{pa 4 -409}\special{pa -4 -409}%
\special{pa -10 -406}\special{pa -14 -401}\special{pa -16 -394}\special{pa -14 -387}%
\special{pa -10 -381}\special{pa -4 -378}\special{pa 4 -378}\special{pa 10 -381}\special{pa 14 -387}%
\special{pa 16 -394}\special{pa 16 -394}\special{sh 1}\special{ip}%
}%
}%
{%
\color[cmyk]{0,1,1,0}%
\special{pa    16  -394}\special{pa    14  -401}\special{pa    10  -406}\special{pa     4  -409}%
\special{pa    -4  -409}\special{pa   -10  -406}\special{pa   -14  -401}\special{pa   -16  -394}%
\special{pa   -14  -387}\special{pa   -10  -381}\special{pa    -4  -378}\special{pa     4  -378}%
\special{pa    10  -381}\special{pa    14  -387}\special{pa    16  -394}%
\special{fp}%
{%
\color[cmyk]{0,1,1,0}%
\special{pa 409 -787}\special{pa 408 -794}\special{pa 404 -800}\special{pa 397 -803}%
\special{pa 390 -803}\special{pa 384 -800}\special{pa 380 -794}\special{pa 378 -787}%
\special{pa 380 -781}\special{pa 384 -775}\special{pa 390 -772}\special{pa 397 -772}%
\special{pa 404 -775}\special{pa 408 -781}\special{pa 409 -787}\special{pa 409 -787}%
\special{sh 1}\special{ip}%
}%
}%
{%
\color[cmyk]{0,1,1,0}%
\special{pa   409  -787}\special{pa   408  -794}\special{pa   404  -800}\special{pa   397  -803}%
\special{pa   390  -803}\special{pa   384  -800}\special{pa   380  -794}\special{pa   378  -787}%
\special{pa   380  -781}\special{pa   384  -775}\special{pa   390  -772}\special{pa   397  -772}%
\special{pa   404  -775}\special{pa   408  -781}\special{pa   409  -787}%
\special{fp}%
{%
\color[cmyk]{0,1,1,0}%
\special{pa 803 -1575}\special{pa 802 -1582}\special{pa 797 -1587}\special{pa 791 -1590}%
\special{pa 784 -1590}\special{pa 778 -1587}\special{pa 773 -1582}\special{pa 772 -1575}%
\special{pa 773 -1568}\special{pa 778 -1562}\special{pa 784 -1559}\special{pa 791 -1559}%
\special{pa 797 -1562}\special{pa 802 -1568}\special{pa 803 -1575}\special{pa 803 -1575}%
\special{sh 1}\special{ip}%
}%
}%
{%
\color[cmyk]{0,1,1,0}%
\special{pa   803 -1575}\special{pa   802 -1582}\special{pa   797 -1587}\special{pa   791 -1590}%
\special{pa   784 -1590}\special{pa   778 -1587}\special{pa   773 -1582}\special{pa   772 -1575}%
\special{pa   773 -1568}\special{pa   778 -1562}\special{pa   784 -1559}\special{pa   791 -1559}%
\special{pa   797 -1562}\special{pa   802 -1568}\special{pa   803 -1575}%
\special{fp}%
{%
\color[cmyk]{0,1,1,0}%
\special{pa 1197 -3150}\special{pa 1195 -3156}\special{pa 1191 -3162}\special{pa 1185 -3165}%
\special{pa 1178 -3165}\special{pa 1171 -3162}\special{pa 1167 -3156}\special{pa 1165 -3150}%
\special{pa 1167 -3143}\special{pa 1171 -3137}\special{pa 1178 -3134}\special{pa 1185 -3134}%
\special{pa 1191 -3137}\special{pa 1195 -3143}\special{pa 1197 -3150}\special{pa 1197 -3150}%
\special{sh 1}\special{ip}%
}%
}%
{%
\color[cmyk]{0,1,1,0}%
{%
\color[cmyk]{0,1,1,0}%
\special{pa 1591 -6299}\special{pa 1589 -6306}\special{pa 1585 -6312}\special{pa 1578 -6315}%
\special{pa 1571 -6315}\special{pa 1565 -6312}\special{pa 1561 -6306}\special{pa 1559 -6299}%
\special{pa 1561 -6292}\special{pa 1565 -6287}\special{pa 1571 -6284}\special{pa 1578 -6284}%
\special{pa 1585 -6287}\special{pa 1589 -6292}\special{pa 1591 -6299}\special{pa 1591 -6299}%
\special{sh 1}\special{ip}%
}%
}%
{%
\color[cmyk]{0,1,1,0}%
}%
\special{pa -1575   -20}\special{pa -1575    20}%
\special{fp}%
\settowidth{\Width}{$-4$}\setlength{\Width}{-0.5\Width}%
\settoheight{\Height}{$-4$}\settodepth{\Depth}{$-4$}\setlength{\Height}{-\Height}%
\put(-4.0000000,-0.1000000){\hspace*{\Width}\raisebox{\Height}{$-4$}}%
%
\special{pa -1181   -20}\special{pa -1181    20}%
\special{fp}%
\settowidth{\Width}{$-3$}\setlength{\Width}{-0.5\Width}%
\settoheight{\Height}{$-3$}\settodepth{\Depth}{$-3$}\setlength{\Height}{-\Height}%
\put(-3.0000000,-0.1000000){\hspace*{\Width}\raisebox{\Height}{$-3$}}%
%
\special{pa  -787   -20}\special{pa  -787    20}%
\special{fp}%
\settowidth{\Width}{$-2$}\setlength{\Width}{-0.5\Width}%
\settoheight{\Height}{$-2$}\settodepth{\Depth}{$-2$}\setlength{\Height}{-\Height}%
\put(-2.0000000,-0.1000000){\hspace*{\Width}\raisebox{\Height}{$-2$}}%
%
\special{pa  -394   -20}\special{pa  -394    20}%
\special{fp}%
\settowidth{\Width}{$-1$}\setlength{\Width}{-0.5\Width}%
\settoheight{\Height}{$-1$}\settodepth{\Depth}{$-1$}\setlength{\Height}{-\Height}%
\put(-1.0000000,-0.1000000){\hspace*{\Width}\raisebox{\Height}{$-1$}}%
%
\special{pa   394   -20}\special{pa   394    20}%
\special{fp}%
\settowidth{\Width}{$1$}\setlength{\Width}{-0.5\Width}%
\settoheight{\Height}{$1$}\settodepth{\Depth}{$1$}\setlength{\Height}{-\Height}%
\put(1.0000000,-0.1000000){\hspace*{\Width}\raisebox{\Height}{$1$}}%
%
\special{pa   787   -20}\special{pa   787    20}%
\special{fp}%
\settowidth{\Width}{$2$}\setlength{\Width}{-0.5\Width}%
\settoheight{\Height}{$2$}\settodepth{\Depth}{$2$}\setlength{\Height}{-\Height}%
\put(2.0000000,-0.1000000){\hspace*{\Width}\raisebox{\Height}{$2$}}%
%
\special{pa  1181   -20}\special{pa  1181    20}%
\special{fp}%
\settowidth{\Width}{$3$}\setlength{\Width}{-0.5\Width}%
\settoheight{\Height}{$3$}\settodepth{\Depth}{$3$}\setlength{\Height}{-\Height}%
\put(3.0000000,-0.1000000){\hspace*{\Width}\raisebox{\Height}{$3$}}%
%
\special{pa  1575   -20}\special{pa  1575    20}%
\special{fp}%
\settowidth{\Width}{$4$}\setlength{\Width}{-0.5\Width}%
\settoheight{\Height}{$4$}\settodepth{\Depth}{$4$}\setlength{\Height}{-\Height}%
\put(4.0000000,-0.1000000){\hspace*{\Width}\raisebox{\Height}{$4$}}%
%
\special{pa    20  -394}\special{pa   -20  -394}%
\special{fp}%
\settowidth{\Width}{$1$}\setlength{\Width}{-1\Width}%
\settoheight{\Height}{$1$}\settodepth{\Depth}{$1$}\setlength{\Height}{-0.5\Height}\setlength{\Depth}{0.5\Depth}\addtolength{\Height}{\Depth}%
\put(-0.1000000,1.0000000){\hspace*{\Width}\raisebox{\Height}{$1$}}%
%
\special{pa    20  -787}\special{pa   -20  -787}%
\special{fp}%
\settowidth{\Width}{$2$}\setlength{\Width}{-1\Width}%
\settoheight{\Height}{$2$}\settodepth{\Depth}{$2$}\setlength{\Height}{-0.5\Height}\setlength{\Depth}{0.5\Depth}\addtolength{\Height}{\Depth}%
\put(-0.1000000,2.0000000){\hspace*{\Width}\raisebox{\Height}{$2$}}%
%
\special{pa    20 -1181}\special{pa   -20 -1181}%
\special{fp}%
\settowidth{\Width}{$3$}\setlength{\Width}{-1\Width}%
\settoheight{\Height}{$3$}\settodepth{\Depth}{$3$}\setlength{\Height}{-0.5\Height}\setlength{\Depth}{0.5\Depth}\addtolength{\Height}{\Depth}%
\put(-0.1000000,3.0000000){\hspace*{\Width}\raisebox{\Height}{$3$}}%
%
\special{pa    20 -1575}\special{pa   -20 -1575}%
\special{fp}%
\settowidth{\Width}{$4$}\setlength{\Width}{-1\Width}%
\settoheight{\Height}{$4$}\settodepth{\Depth}{$4$}\setlength{\Height}{-0.5\Height}\setlength{\Depth}{0.5\Depth}\addtolength{\Height}{\Depth}%
\put(-0.1000000,4.0000000){\hspace*{\Width}\raisebox{\Height}{$4$}}%
%
\special{pa    20 -1969}\special{pa   -20 -1969}%
\special{fp}%
\settowidth{\Width}{$5$}\setlength{\Width}{-1\Width}%
\settoheight{\Height}{$5$}\settodepth{\Depth}{$5$}\setlength{\Height}{-0.5\Height}\setlength{\Depth}{0.5\Depth}\addtolength{\Height}{\Depth}%
\put(-0.1000000,5.0000000){\hspace*{\Width}\raisebox{\Height}{$5$}}%
%
\special{pa    20 -2362}\special{pa   -20 -2362}%
\special{fp}%
\settowidth{\Width}{$6$}\setlength{\Width}{-1\Width}%
\settoheight{\Height}{$6$}\settodepth{\Depth}{$6$}\setlength{\Height}{-0.5\Height}\setlength{\Depth}{0.5\Depth}\addtolength{\Height}{\Depth}%
\put(-0.1000000,6.0000000){\hspace*{\Width}\raisebox{\Height}{$6$}}%
%
\special{pa    20 -2756}\special{pa   -20 -2756}%
\special{fp}%
\settowidth{\Width}{$7$}\setlength{\Width}{-1\Width}%
\settoheight{\Height}{$7$}\settodepth{\Depth}{$7$}\setlength{\Height}{-0.5\Height}\setlength{\Depth}{0.5\Depth}\addtolength{\Height}{\Depth}%
\put(-0.1000000,7.0000000){\hspace*{\Width}\raisebox{\Height}{$7$}}%
%
{%
\color[rgb]{0,0,0}%
\special{pa -1575   -25}\special{pa -1564   -25}\special{pa -1554   -26}\special{pa -1543   -26}%
\special{pa -1533   -26}\special{pa -1522   -27}\special{pa -1512   -27}\special{pa -1501   -28}%
\special{pa -1491   -29}\special{pa -1480   -29}\special{pa -1470   -30}\special{pa -1459   -30}%
\special{pa -1449   -31}\special{pa -1438   -31}\special{pa -1428   -32}\special{pa -1417   -32}%
\special{pa -1407   -33}\special{pa -1396   -34}\special{pa -1386   -34}\special{pa -1375   -35}%
\special{pa -1365   -36}\special{pa -1354   -36}\special{pa -1344   -37}\special{pa -1333   -38}%
\special{pa -1323   -38}\special{pa -1312   -39}\special{pa -1302   -40}\special{pa -1291   -41}%
\special{pa -1281   -41}\special{pa -1270   -42}\special{pa -1260   -43}\special{pa -1249   -44}%
\special{pa -1239   -44}\special{pa -1228   -45}\special{pa -1218   -46}\special{pa -1207   -47}%
\special{pa -1197   -48}\special{pa -1186   -49}\special{pa -1176   -50}\special{pa -1165   -51}%
\special{pa -1155   -52}\special{pa -1144   -53}\special{pa -1134   -53}\special{pa -1123   -54}%
\special{pa -1113   -55}\special{pa -1102   -57}\special{pa -1092   -58}\special{pa -1081   -59}%
\special{pa -1071   -60}\special{pa -1060   -61}\special{pa -1050   -62}\special{pa -1039   -63}%
\special{pa -1029   -64}\special{pa -1018   -66}\special{pa -1008   -67}\special{pa  -997   -68}%
\special{pa  -987   -69}\special{pa  -976   -71}\special{pa  -966   -72}\special{pa  -955   -73}%
\special{pa  -945   -75}\special{pa  -934   -76}\special{pa  -924   -77}\special{pa  -913   -79}%
\special{pa  -903   -80}\special{pa  -892   -82}\special{pa  -882   -83}\special{pa  -871   -85}%
\special{pa  -861   -86}\special{pa  -850   -88}\special{pa  -840   -90}\special{pa  -829   -91}%
\special{pa  -819   -93}\special{pa  -808   -95}\special{pa  -798   -97}\special{pa  -787   -98}%
\special{pa  -777  -100}\special{pa  -766  -102}\special{pa  -756  -104}\special{pa  -745  -106}%
\special{pa  -735  -108}\special{pa  -724  -110}\special{pa  -714  -112}\special{pa  -703  -114}%
\special{pa  -693  -116}\special{pa  -682  -118}\special{pa  -672  -121}\special{pa  -661  -123}%
\special{pa  -651  -125}\special{pa  -640  -127}\special{pa  -630  -130}\special{pa  -619  -132}%
\special{pa  -609  -135}\special{pa  -598  -137}\special{pa  -588  -140}\special{pa  -577  -142}%
\special{pa  -567  -145}\special{pa  -556  -148}\special{pa  -546  -151}\special{pa  -535  -153}%
\special{pa  -525  -156}\special{pa  -514  -159}\special{pa  -504  -162}\special{pa  -493  -165}%
\special{pa  -483  -168}\special{pa  -472  -171}\special{pa  -462  -175}\special{pa  -451  -178}%
\special{pa  -441  -181}\special{pa  -430  -185}\special{pa  -420  -188}\special{pa  -409  -191}%
\special{pa  -399  -195}\special{pa  -388  -199}\special{pa  -378  -202}\special{pa  -367  -206}%
\special{pa  -357  -210}\special{pa  -346  -214}\special{pa  -336  -218}\special{pa  -325  -222}%
\special{pa  -315  -226}\special{pa  -304  -230}\special{pa  -294  -235}\special{pa  -283  -239}%
\special{pa  -273  -243}\special{pa  -262  -248}\special{pa  -252  -253}\special{pa  -241  -257}%
\special{pa  -231  -262}\special{pa  -220  -267}\special{pa  -210  -272}\special{pa  -199  -277}%
\special{pa  -189  -282}\special{pa  -178  -288}\special{pa  -168  -293}\special{pa  -157  -298}%
\special{pa  -147  -304}\special{pa  -136  -310}\special{pa  -126  -315}\special{pa  -115  -321}%
\special{pa  -105  -327}\special{pa   -94  -333}\special{pa   -84  -340}\special{pa   -73  -346}%
\special{pa   -63  -352}\special{pa   -52  -359}\special{pa   -42  -366}\special{pa   -31  -372}%
\special{pa   -21  -379}\special{pa   -10  -386}\special{pa     0  -394}\special{pa    10  -401}%
\special{pa    21  -409}\special{pa    31  -416}\special{pa    42  -424}\special{pa    52  -432}%
\special{pa    63  -440}\special{pa    73  -448}\special{pa    84  -456}\special{pa    94  -465}%
\special{pa   105  -474}\special{pa   115  -482}\special{pa   126  -491}\special{pa   136  -501}%
\special{pa   147  -510}\special{pa   157  -519}\special{pa   168  -529}\special{pa   178  -539}%
\special{pa   189  -549}\special{pa   199  -559}\special{pa   210  -570}\special{pa   220  -580}%
\special{pa   231  -591}\special{pa   241  -602}\special{pa   252  -614}\special{pa   262  -625}%
\special{pa   273  -637}\special{pa   283  -648}\special{pa   294  -661}\special{pa   304  -673}%
\special{pa   315  -685}\special{pa   325  -698}\special{pa   336  -711}\special{pa   346  -725}%
\special{pa   357  -738}\special{pa   367  -752}\special{pa   378  -766}\special{pa   388  -780}%
\special{pa   399  -795}\special{pa   409  -810}\special{pa   420  -825}\special{pa   430  -840}%
\special{pa   441  -856}\special{pa   451  -872}\special{pa   462  -888}\special{pa   472  -904}%
\special{pa   483  -921}\special{pa   493  -939}\special{pa   504  -956}\special{pa   514  -974}%
\special{pa   525  -992}\special{pa   535 -1011}\special{pa   546 -1029}\special{pa   556 -1049}%
\special{pa   567 -1068}\special{pa   577 -1088}\special{pa   588 -1108}\special{pa   598 -1129}%
\special{pa   609 -1150}\special{pa   619 -1172}\special{pa   630 -1193}\special{pa   640 -1216}%
\special{pa   651 -1238}\special{pa   661 -1262}\special{pa   672 -1285}\special{pa   682 -1309}%
\special{pa   693 -1333}\special{pa   703 -1358}\special{pa   714 -1384}\special{pa   724 -1409}%
\special{pa   735 -1436}\special{pa   745 -1463}\special{pa   756 -1490}\special{pa   766 -1518}%
\special{pa   777 -1546}\special{pa   787 -1575}\special{pa   798 -1604}\special{pa   808 -1634}%
\special{pa   819 -1665}\special{pa   829 -1696}\special{pa   840 -1727}\special{pa   850 -1760}%
\special{pa   861 -1792}\special{pa   871 -1826}\special{pa   882 -1860}\special{pa   892 -1895}%
\special{pa   903 -1930}\special{pa   913 -1966}\special{pa   924 -2003}\special{pa   934 -2040}%
\special{pa   945 -2078}\special{pa   955 -2117}\special{pa   966 -2156}\special{pa   976 -2196}%
\special{pa   987 -2237}\special{pa   997 -2279}\special{pa  1008 -2322}\special{pa  1018 -2365}%
\special{pa  1029 -2409}\special{pa  1039 -2454}\special{pa  1050 -2500}\special{pa  1060 -2546}%
\special{pa  1071 -2594}\special{pa  1081 -2642}\special{pa  1092 -2692}\special{pa  1102 -2742}%
\special{pa  1105 -2756}%
\special{fp}%
}%
\special{pa -1575    -0}\special{pa  1575    -0}%
\special{fp}%
\special{pa     0   197}\special{pa     0 -2756}%
\special{fp}%
\settowidth{\Width}{$x$}\setlength{\Width}{0\Width}%
\settoheight{\Height}{$x$}\settodepth{\Depth}{$x$}\setlength{\Height}{-0.5\Height}\setlength{\Depth}{0.5\Depth}\addtolength{\Height}{\Depth}%
\put(4.0500000,0.0000000){\hspace*{\Width}\raisebox{\Height}{$x$}}%
%
\settowidth{\Width}{$y$}\setlength{\Width}{-0.5\Width}%
\settoheight{\Height}{$y$}\settodepth{\Depth}{$y$}\setlength{\Height}{\Depth}%
\put(0.0000000,7.0500000){\hspace*{\Width}\raisebox{\Height}{$y$}}%
%
\settowidth{\Width}{O}\setlength{\Width}{-1\Width}%
\settoheight{\Height}{O}\settodepth{\Depth}{O}\setlength{\Height}{-\Height}%
\put(-0.0500000,-0.0500000){\hspace*{\Width}\raisebox{\Height}{O}}%
%
\end{picture}}%}}
\putnotes{90}{5}{\scalebox{0.9}{%%% /polytech.git/n104/fig/sisuugraph3.tex 
%%% Generator=n104sisuu.cdy 
{\unitlength=1cm%
\begin{picture}%
(8,7.5)(-4,-0.5)%
\special{pn 8}%
%
\special{pn 4}%
\special{pa -1575    -0}\special{pa -1575 -2756}%
\special{fp}%
\special{pn 8}%
\special{pn 4}%
\special{pa -1181    -0}\special{pa -1181 -2756}%
\special{fp}%
\special{pn 8}%
\special{pn 4}%
\special{pa  -787    -0}\special{pa  -787 -2756}%
\special{fp}%
\special{pn 8}%
\special{pn 4}%
\special{pa  -394    -0}\special{pa  -394 -2756}%
\special{fp}%
\special{pn 8}%
\special{pn 4}%
\special{pa     0    -0}\special{pa     0 -2756}%
\special{fp}%
\special{pn 8}%
\special{pn 4}%
\special{pa   394    -0}\special{pa   394 -2756}%
\special{fp}%
\special{pn 8}%
\special{pn 4}%
\special{pa   787    -0}\special{pa   787 -2756}%
\special{fp}%
\special{pn 8}%
\special{pn 4}%
\special{pa  1181    -0}\special{pa  1181 -2756}%
\special{fp}%
\special{pn 8}%
\special{pn 4}%
\special{pa  1575    -0}\special{pa  1575 -2756}%
\special{fp}%
\special{pn 8}%
\special{pn 4}%
\special{pa -1575  -394}\special{pa  1575  -394}%
\special{fp}%
\special{pn 8}%
\special{pn 4}%
\special{pa -1575  -787}\special{pa  1575  -787}%
\special{fp}%
\special{pn 8}%
\special{pn 4}%
\special{pa -1575 -1181}\special{pa  1575 -1181}%
\special{fp}%
\special{pn 8}%
\special{pn 4}%
\special{pa -1575 -1575}\special{pa  1575 -1575}%
\special{fp}%
\special{pn 8}%
\special{pn 4}%
\special{pa -1575 -1969}\special{pa  1575 -1969}%
\special{fp}%
\special{pn 8}%
\special{pn 4}%
\special{pa -1575 -2362}\special{pa  1575 -2362}%
\special{fp}%
\special{pn 8}%
{%
\color[cmyk]{0,1,1,0}%
\special{pa -1559 -24}\special{pa -1561 -30}\special{pa -1565 -36}\special{pa -1571 -39}%
\special{pa -1578 -39}\special{pa -1585 -36}\special{pa -1589 -30}\special{pa -1591 -24}%
\special{pa -1589 -17}\special{pa -1585 -11}\special{pa -1578 -8}\special{pa -1571 -8}%
\special{pa -1565 -11}\special{pa -1561 -17}\special{pa -1559 -24}\special{pa -1559 -24}%
\special{sh 1}\special{ip}%
}%
\special{pn 4}%
\special{pa -1575 -2756}\special{pa  1575 -2756}%
\special{fp}%
\special{pn 8}%
{%
\color[cmyk]{0,1,1,0}%
\special{pa -1559   -24}\special{pa -1561   -30}\special{pa -1565   -36}\special{pa -1571   -39}%
\special{pa -1575   -39}%
\special{fp}%
\special{pa -1575    -8}\special{pa -1571    -8}\special{pa -1565   -11}\special{pa -1561   -17}%
\special{pa -1559   -24}%
\special{fp}%
{%
\color[cmyk]{0,1,1,0}%
\special{pa -1165 -47}\special{pa -1167 -54}\special{pa -1171 -60}\special{pa -1178 -63}%
\special{pa -1185 -63}\special{pa -1191 -60}\special{pa -1195 -54}\special{pa -1197 -47}%
\special{pa -1195 -40}\special{pa -1191 -35}\special{pa -1185 -32}\special{pa -1178 -32}%
\special{pa -1171 -35}\special{pa -1167 -40}\special{pa -1165 -47}\special{pa -1165 -47}%
\special{sh 1}\special{ip}%
}%
}%
{%
\color[cmyk]{0,1,1,0}%
\special{pa -1165   -47}\special{pa -1167   -54}\special{pa -1171   -60}\special{pa -1178   -63}%
\special{pa -1185   -63}\special{pa -1191   -60}\special{pa -1195   -54}\special{pa -1197   -47}%
\special{pa -1195   -40}\special{pa -1191   -35}\special{pa -1185   -32}\special{pa -1178   -32}%
\special{pa -1171   -35}\special{pa -1167   -40}\special{pa -1165   -47}%
\special{fp}%
{%
\color[cmyk]{0,1,1,0}%
\special{pa -772 -98}\special{pa -773 -105}\special{pa -778 -111}\special{pa -784 -114}%
\special{pa -791 -114}\special{pa -797 -111}\special{pa -802 -105}\special{pa -803 -98}%
\special{pa -802 -92}\special{pa -797 -86}\special{pa -791 -83}\special{pa -784 -83}%
\special{pa -778 -86}\special{pa -773 -92}\special{pa -772 -98}\special{pa -772 -98}%
\special{sh 1}\special{ip}%
}%
}%
{%
\color[cmyk]{0,1,1,0}%
\special{pa  -772   -98}\special{pa  -773  -105}\special{pa  -778  -111}\special{pa  -784  -114}%
\special{pa  -791  -114}\special{pa  -797  -111}\special{pa  -802  -105}\special{pa  -803   -98}%
\special{pa  -802   -92}\special{pa  -797   -86}\special{pa  -791   -83}\special{pa  -784   -83}%
\special{pa  -778   -86}\special{pa  -773   -92}\special{pa  -772   -98}%
\special{fp}%
{%
\color[cmyk]{0,1,1,0}%
\special{pa -378 -197}\special{pa -380 -204}\special{pa -384 -209}\special{pa -390 -212}%
\special{pa -397 -212}\special{pa -404 -209}\special{pa -408 -204}\special{pa -409 -197}%
\special{pa -408 -190}\special{pa -404 -185}\special{pa -397 -181}\special{pa -390 -181}%
\special{pa -384 -185}\special{pa -380 -190}\special{pa -378 -197}\special{pa -378 -197}%
\special{sh 1}\special{ip}%
}%
}%
{%
\color[cmyk]{0,1,1,0}%
\special{pa  -378  -197}\special{pa  -380  -204}\special{pa  -384  -209}\special{pa  -390  -212}%
\special{pa  -397  -212}\special{pa  -404  -209}\special{pa  -408  -204}\special{pa  -409  -197}%
\special{pa  -408  -190}\special{pa  -404  -185}\special{pa  -397  -181}\special{pa  -390  -181}%
\special{pa  -384  -185}\special{pa  -380  -190}\special{pa  -378  -197}%
\special{fp}%
{%
\color[cmyk]{0,1,1,0}%
\special{pa 16 -394}\special{pa 14 -401}\special{pa 10 -406}\special{pa 4 -409}\special{pa -4 -409}%
\special{pa -10 -406}\special{pa -14 -401}\special{pa -16 -394}\special{pa -14 -387}%
\special{pa -10 -381}\special{pa -4 -378}\special{pa 4 -378}\special{pa 10 -381}\special{pa 14 -387}%
\special{pa 16 -394}\special{pa 16 -394}\special{sh 1}\special{ip}%
}%
}%
{%
\color[cmyk]{0,1,1,0}%
\special{pa    16  -394}\special{pa    14  -401}\special{pa    10  -406}\special{pa     4  -409}%
\special{pa    -4  -409}\special{pa   -10  -406}\special{pa   -14  -401}\special{pa   -16  -394}%
\special{pa   -14  -387}\special{pa   -10  -381}\special{pa    -4  -378}\special{pa     4  -378}%
\special{pa    10  -381}\special{pa    14  -387}\special{pa    16  -394}%
\special{fp}%
{%
\color[cmyk]{0,1,1,0}%
\special{pa 409 -787}\special{pa 408 -794}\special{pa 404 -800}\special{pa 397 -803}%
\special{pa 390 -803}\special{pa 384 -800}\special{pa 380 -794}\special{pa 378 -787}%
\special{pa 380 -781}\special{pa 384 -775}\special{pa 390 -772}\special{pa 397 -772}%
\special{pa 404 -775}\special{pa 408 -781}\special{pa 409 -787}\special{pa 409 -787}%
\special{sh 1}\special{ip}%
}%
}%
{%
\color[cmyk]{0,1,1,0}%
\special{pa   409  -787}\special{pa   408  -794}\special{pa   404  -800}\special{pa   397  -803}%
\special{pa   390  -803}\special{pa   384  -800}\special{pa   380  -794}\special{pa   378  -787}%
\special{pa   380  -781}\special{pa   384  -775}\special{pa   390  -772}\special{pa   397  -772}%
\special{pa   404  -775}\special{pa   408  -781}\special{pa   409  -787}%
\special{fp}%
{%
\color[cmyk]{0,1,1,0}%
\special{pa 803 -1575}\special{pa 802 -1582}\special{pa 797 -1587}\special{pa 791 -1590}%
\special{pa 784 -1590}\special{pa 778 -1587}\special{pa 773 -1582}\special{pa 772 -1575}%
\special{pa 773 -1568}\special{pa 778 -1562}\special{pa 784 -1559}\special{pa 791 -1559}%
\special{pa 797 -1562}\special{pa 802 -1568}\special{pa 803 -1575}\special{pa 803 -1575}%
\special{sh 1}\special{ip}%
}%
}%
{%
\color[cmyk]{0,1,1,0}%
\special{pa   803 -1575}\special{pa   802 -1582}\special{pa   797 -1587}\special{pa   791 -1590}%
\special{pa   784 -1590}\special{pa   778 -1587}\special{pa   773 -1582}\special{pa   772 -1575}%
\special{pa   773 -1568}\special{pa   778 -1562}\special{pa   784 -1559}\special{pa   791 -1559}%
\special{pa   797 -1562}\special{pa   802 -1568}\special{pa   803 -1575}%
\special{fp}%
{%
\color[cmyk]{0,1,1,0}%
\special{pa 1197 -3150}\special{pa 1195 -3156}\special{pa 1191 -3162}\special{pa 1185 -3165}%
\special{pa 1178 -3165}\special{pa 1171 -3162}\special{pa 1167 -3156}\special{pa 1165 -3150}%
\special{pa 1167 -3143}\special{pa 1171 -3137}\special{pa 1178 -3134}\special{pa 1185 -3134}%
\special{pa 1191 -3137}\special{pa 1195 -3143}\special{pa 1197 -3150}\special{pa 1197 -3150}%
\special{sh 1}\special{ip}%
}%
}%
{%
\color[cmyk]{0,1,1,0}%
{%
\color[cmyk]{0,1,1,0}%
\special{pa 1591 -6299}\special{pa 1589 -6306}\special{pa 1585 -6312}\special{pa 1578 -6315}%
\special{pa 1571 -6315}\special{pa 1565 -6312}\special{pa 1561 -6306}\special{pa 1559 -6299}%
\special{pa 1561 -6292}\special{pa 1565 -6287}\special{pa 1571 -6284}\special{pa 1578 -6284}%
\special{pa 1585 -6287}\special{pa 1589 -6292}\special{pa 1591 -6299}\special{pa 1591 -6299}%
\special{sh 1}\special{ip}%
}%
}%
{%
\color[cmyk]{0,1,1,0}%
}%
\special{pa -1575   -20}\special{pa -1575    20}%
\special{fp}%
\settowidth{\Width}{$-4$}\setlength{\Width}{-0.5\Width}%
\settoheight{\Height}{$-4$}\settodepth{\Depth}{$-4$}\setlength{\Height}{-\Height}%
\put(-4.0000000,-0.1000000){\hspace*{\Width}\raisebox{\Height}{$-4$}}%
%
\special{pa -1181   -20}\special{pa -1181    20}%
\special{fp}%
\settowidth{\Width}{$-3$}\setlength{\Width}{-0.5\Width}%
\settoheight{\Height}{$-3$}\settodepth{\Depth}{$-3$}\setlength{\Height}{-\Height}%
\put(-3.0000000,-0.1000000){\hspace*{\Width}\raisebox{\Height}{$-3$}}%
%
\special{pa  -787   -20}\special{pa  -787    20}%
\special{fp}%
\settowidth{\Width}{$-2$}\setlength{\Width}{-0.5\Width}%
\settoheight{\Height}{$-2$}\settodepth{\Depth}{$-2$}\setlength{\Height}{-\Height}%
\put(-2.0000000,-0.1000000){\hspace*{\Width}\raisebox{\Height}{$-2$}}%
%
\special{pa  -394   -20}\special{pa  -394    20}%
\special{fp}%
\settowidth{\Width}{$-1$}\setlength{\Width}{-0.5\Width}%
\settoheight{\Height}{$-1$}\settodepth{\Depth}{$-1$}\setlength{\Height}{-\Height}%
\put(-1.0000000,-0.1000000){\hspace*{\Width}\raisebox{\Height}{$-1$}}%
%
\special{pa   394   -20}\special{pa   394    20}%
\special{fp}%
\settowidth{\Width}{$1$}\setlength{\Width}{-0.5\Width}%
\settoheight{\Height}{$1$}\settodepth{\Depth}{$1$}\setlength{\Height}{-\Height}%
\put(1.0000000,-0.1000000){\hspace*{\Width}\raisebox{\Height}{$1$}}%
%
\special{pa   787   -20}\special{pa   787    20}%
\special{fp}%
\settowidth{\Width}{$2$}\setlength{\Width}{-0.5\Width}%
\settoheight{\Height}{$2$}\settodepth{\Depth}{$2$}\setlength{\Height}{-\Height}%
\put(2.0000000,-0.1000000){\hspace*{\Width}\raisebox{\Height}{$2$}}%
%
\special{pa  1181   -20}\special{pa  1181    20}%
\special{fp}%
\settowidth{\Width}{$3$}\setlength{\Width}{-0.5\Width}%
\settoheight{\Height}{$3$}\settodepth{\Depth}{$3$}\setlength{\Height}{-\Height}%
\put(3.0000000,-0.1000000){\hspace*{\Width}\raisebox{\Height}{$3$}}%
%
\special{pa  1575   -20}\special{pa  1575    20}%
\special{fp}%
\settowidth{\Width}{$4$}\setlength{\Width}{-0.5\Width}%
\settoheight{\Height}{$4$}\settodepth{\Depth}{$4$}\setlength{\Height}{-\Height}%
\put(4.0000000,-0.1000000){\hspace*{\Width}\raisebox{\Height}{$4$}}%
%
\special{pa    20  -394}\special{pa   -20  -394}%
\special{fp}%
\settowidth{\Width}{$1$}\setlength{\Width}{-1\Width}%
\settoheight{\Height}{$1$}\settodepth{\Depth}{$1$}\setlength{\Height}{-0.5\Height}\setlength{\Depth}{0.5\Depth}\addtolength{\Height}{\Depth}%
\put(-0.1000000,1.0000000){\hspace*{\Width}\raisebox{\Height}{$1$}}%
%
\special{pa    20  -787}\special{pa   -20  -787}%
\special{fp}%
\settowidth{\Width}{$2$}\setlength{\Width}{-1\Width}%
\settoheight{\Height}{$2$}\settodepth{\Depth}{$2$}\setlength{\Height}{-0.5\Height}\setlength{\Depth}{0.5\Depth}\addtolength{\Height}{\Depth}%
\put(-0.1000000,2.0000000){\hspace*{\Width}\raisebox{\Height}{$2$}}%
%
\special{pa    20 -1181}\special{pa   -20 -1181}%
\special{fp}%
\settowidth{\Width}{$3$}\setlength{\Width}{-1\Width}%
\settoheight{\Height}{$3$}\settodepth{\Depth}{$3$}\setlength{\Height}{-0.5\Height}\setlength{\Depth}{0.5\Depth}\addtolength{\Height}{\Depth}%
\put(-0.1000000,3.0000000){\hspace*{\Width}\raisebox{\Height}{$3$}}%
%
\special{pa    20 -1575}\special{pa   -20 -1575}%
\special{fp}%
\settowidth{\Width}{$4$}\setlength{\Width}{-1\Width}%
\settoheight{\Height}{$4$}\settodepth{\Depth}{$4$}\setlength{\Height}{-0.5\Height}\setlength{\Depth}{0.5\Depth}\addtolength{\Height}{\Depth}%
\put(-0.1000000,4.0000000){\hspace*{\Width}\raisebox{\Height}{$4$}}%
%
\special{pa    20 -1969}\special{pa   -20 -1969}%
\special{fp}%
\settowidth{\Width}{$5$}\setlength{\Width}{-1\Width}%
\settoheight{\Height}{$5$}\settodepth{\Depth}{$5$}\setlength{\Height}{-0.5\Height}\setlength{\Depth}{0.5\Depth}\addtolength{\Height}{\Depth}%
\put(-0.1000000,5.0000000){\hspace*{\Width}\raisebox{\Height}{$5$}}%
%
\special{pa    20 -2362}\special{pa   -20 -2362}%
\special{fp}%
\settowidth{\Width}{$6$}\setlength{\Width}{-1\Width}%
\settoheight{\Height}{$6$}\settodepth{\Depth}{$6$}\setlength{\Height}{-0.5\Height}\setlength{\Depth}{0.5\Depth}\addtolength{\Height}{\Depth}%
\put(-0.1000000,6.0000000){\hspace*{\Width}\raisebox{\Height}{$6$}}%
%
\special{pa    20 -2756}\special{pa   -20 -2756}%
\special{fp}%
\settowidth{\Width}{$7$}\setlength{\Width}{-1\Width}%
\settoheight{\Height}{$7$}\settodepth{\Depth}{$7$}\setlength{\Height}{-0.5\Height}\setlength{\Depth}{0.5\Depth}\addtolength{\Height}{\Depth}%
\put(-0.1000000,7.0000000){\hspace*{\Width}\raisebox{\Height}{$7$}}%
%
{%
\color[rgb]{0,0,0}%
\special{pa -1575   -25}\special{pa -1564   -25}\special{pa -1554   -26}\special{pa -1543   -26}%
\special{pa -1533   -26}\special{pa -1522   -27}\special{pa -1512   -27}\special{pa -1501   -28}%
\special{pa -1491   -29}\special{pa -1480   -29}\special{pa -1470   -30}\special{pa -1459   -30}%
\special{pa -1449   -31}\special{pa -1438   -31}\special{pa -1428   -32}\special{pa -1417   -32}%
\special{pa -1407   -33}\special{pa -1396   -34}\special{pa -1386   -34}\special{pa -1375   -35}%
\special{pa -1365   -36}\special{pa -1354   -36}\special{pa -1344   -37}\special{pa -1333   -38}%
\special{pa -1323   -38}\special{pa -1312   -39}\special{pa -1302   -40}\special{pa -1291   -41}%
\special{pa -1281   -41}\special{pa -1270   -42}\special{pa -1260   -43}\special{pa -1249   -44}%
\special{pa -1239   -44}\special{pa -1228   -45}\special{pa -1218   -46}\special{pa -1207   -47}%
\special{pa -1197   -48}\special{pa -1186   -49}\special{pa -1176   -50}\special{pa -1165   -51}%
\special{pa -1155   -52}\special{pa -1144   -53}\special{pa -1134   -53}\special{pa -1123   -54}%
\special{pa -1113   -55}\special{pa -1102   -57}\special{pa -1092   -58}\special{pa -1081   -59}%
\special{pa -1071   -60}\special{pa -1060   -61}\special{pa -1050   -62}\special{pa -1039   -63}%
\special{pa -1029   -64}\special{pa -1018   -66}\special{pa -1008   -67}\special{pa  -997   -68}%
\special{pa  -987   -69}\special{pa  -976   -71}\special{pa  -966   -72}\special{pa  -955   -73}%
\special{pa  -945   -75}\special{pa  -934   -76}\special{pa  -924   -77}\special{pa  -913   -79}%
\special{pa  -903   -80}\special{pa  -892   -82}\special{pa  -882   -83}\special{pa  -871   -85}%
\special{pa  -861   -86}\special{pa  -850   -88}\special{pa  -840   -90}\special{pa  -829   -91}%
\special{pa  -819   -93}\special{pa  -808   -95}\special{pa  -798   -97}\special{pa  -787   -98}%
\special{pa  -777  -100}\special{pa  -766  -102}\special{pa  -756  -104}\special{pa  -745  -106}%
\special{pa  -735  -108}\special{pa  -724  -110}\special{pa  -714  -112}\special{pa  -703  -114}%
\special{pa  -693  -116}\special{pa  -682  -118}\special{pa  -672  -121}\special{pa  -661  -123}%
\special{pa  -651  -125}\special{pa  -640  -127}\special{pa  -630  -130}\special{pa  -619  -132}%
\special{pa  -609  -135}\special{pa  -598  -137}\special{pa  -588  -140}\special{pa  -577  -142}%
\special{pa  -567  -145}\special{pa  -556  -148}\special{pa  -546  -151}\special{pa  -535  -153}%
\special{pa  -525  -156}\special{pa  -514  -159}\special{pa  -504  -162}\special{pa  -493  -165}%
\special{pa  -483  -168}\special{pa  -472  -171}\special{pa  -462  -175}\special{pa  -451  -178}%
\special{pa  -441  -181}\special{pa  -430  -185}\special{pa  -420  -188}\special{pa  -409  -191}%
\special{pa  -399  -195}\special{pa  -388  -199}\special{pa  -378  -202}\special{pa  -367  -206}%
\special{pa  -357  -210}\special{pa  -346  -214}\special{pa  -336  -218}\special{pa  -325  -222}%
\special{pa  -315  -226}\special{pa  -304  -230}\special{pa  -294  -235}\special{pa  -283  -239}%
\special{pa  -273  -243}\special{pa  -262  -248}\special{pa  -252  -253}\special{pa  -241  -257}%
\special{pa  -231  -262}\special{pa  -220  -267}\special{pa  -210  -272}\special{pa  -199  -277}%
\special{pa  -189  -282}\special{pa  -178  -288}\special{pa  -168  -293}\special{pa  -157  -298}%
\special{pa  -147  -304}\special{pa  -136  -310}\special{pa  -126  -315}\special{pa  -115  -321}%
\special{pa  -105  -327}\special{pa   -94  -333}\special{pa   -84  -340}\special{pa   -73  -346}%
\special{pa   -63  -352}\special{pa   -52  -359}\special{pa   -42  -366}\special{pa   -31  -372}%
\special{pa   -21  -379}\special{pa   -10  -386}\special{pa     0  -394}\special{pa    10  -401}%
\special{pa    21  -409}\special{pa    31  -416}\special{pa    42  -424}\special{pa    52  -432}%
\special{pa    63  -440}\special{pa    73  -448}\special{pa    84  -456}\special{pa    94  -465}%
\special{pa   105  -474}\special{pa   115  -482}\special{pa   126  -491}\special{pa   136  -501}%
\special{pa   147  -510}\special{pa   157  -519}\special{pa   168  -529}\special{pa   178  -539}%
\special{pa   189  -549}\special{pa   199  -559}\special{pa   210  -570}\special{pa   220  -580}%
\special{pa   231  -591}\special{pa   241  -602}\special{pa   252  -614}\special{pa   262  -625}%
\special{pa   273  -637}\special{pa   283  -648}\special{pa   294  -661}\special{pa   304  -673}%
\special{pa   315  -685}\special{pa   325  -698}\special{pa   336  -711}\special{pa   346  -725}%
\special{pa   357  -738}\special{pa   367  -752}\special{pa   378  -766}\special{pa   388  -780}%
\special{pa   399  -795}\special{pa   409  -810}\special{pa   420  -825}\special{pa   430  -840}%
\special{pa   441  -856}\special{pa   451  -872}\special{pa   462  -888}\special{pa   472  -904}%
\special{pa   483  -921}\special{pa   493  -939}\special{pa   504  -956}\special{pa   514  -974}%
\special{pa   525  -992}\special{pa   535 -1011}\special{pa   546 -1029}\special{pa   556 -1049}%
\special{pa   567 -1068}\special{pa   577 -1088}\special{pa   588 -1108}\special{pa   598 -1129}%
\special{pa   609 -1150}\special{pa   619 -1172}\special{pa   630 -1193}\special{pa   640 -1216}%
\special{pa   651 -1238}\special{pa   661 -1262}\special{pa   672 -1285}\special{pa   682 -1309}%
\special{pa   693 -1333}\special{pa   703 -1358}\special{pa   714 -1384}\special{pa   724 -1409}%
\special{pa   735 -1436}\special{pa   745 -1463}\special{pa   756 -1490}\special{pa   766 -1518}%
\special{pa   777 -1546}\special{pa   787 -1575}\special{pa   798 -1604}\special{pa   808 -1634}%
\special{pa   819 -1665}\special{pa   829 -1696}\special{pa   840 -1727}\special{pa   850 -1760}%
\special{pa   861 -1792}\special{pa   871 -1826}\special{pa   882 -1860}\special{pa   892 -1895}%
\special{pa   903 -1930}\special{pa   913 -1966}\special{pa   924 -2003}\special{pa   934 -2040}%
\special{pa   945 -2078}\special{pa   955 -2117}\special{pa   966 -2156}\special{pa   976 -2196}%
\special{pa   987 -2237}\special{pa   997 -2279}\special{pa  1008 -2322}\special{pa  1018 -2365}%
\special{pa  1029 -2409}\special{pa  1039 -2454}\special{pa  1050 -2500}\special{pa  1060 -2546}%
\special{pa  1071 -2594}\special{pa  1081 -2642}\special{pa  1092 -2692}\special{pa  1102 -2742}%
\special{pa  1105 -2756}%
\special{fp}%
}%
\special{pn 12}%
\special{pa 197 6}\special{pa 197 -6}\special{fp}\special{pa 197 -34}\special{pa 197 -46}\special{fp}%
\special{pa 197 -73}\special{pa 197 -85}\special{fp}\special{pa 197 -113}\special{pa 197 -125}\special{fp}%
\special{pa 197 -153}\special{pa 197 -165}\special{fp}\special{pa 197 -192}\special{pa 197 -204}\special{fp}%
\special{pa 197 -232}\special{pa 197 -244}\special{fp}\special{pa 197 -272}\special{pa 197 -284}\special{fp}%
\special{pa 197 -311}\special{pa 197 -323}\special{fp}\special{pa 197 -351}\special{pa 197 -363}\special{fp}%
\special{pa 197 -391}\special{pa 197 -403}\special{fp}\special{pa 197 -430}\special{pa 197 -442}\special{fp}%
\special{pa 197 -470}\special{pa 197 -482}\special{fp}\special{pa 197 -510}\special{pa 197 -522}\special{fp}%
\special{pa 201 -551}\special{pa 193 -560}\special{fp}\special{pa 165 -557}\special{pa 153 -557}\special{fp}%
\special{pa 125 -557}\special{pa 113 -557}\special{fp}\special{pa 85 -557}\special{pa 73 -557}\special{fp}%
\special{pa 46 -557}\special{pa 34 -557}\special{fp}\special{pa 6 -557}\special{pa -6 -557}\special{fp}%
\special{pn 8}%
\special{pn 12}%
\special{pa 591 6}\special{pa 591 -6}\special{fp}\special{pa 591 -34}\special{pa 591 -46}\special{fp}%
\special{pa 591 -73}\special{pa 591 -85}\special{fp}\special{pa 591 -113}\special{pa 591 -125}\special{fp}%
\special{pa 591 -153}\special{pa 591 -165}\special{fp}\special{pa 591 -192}\special{pa 591 -204}\special{fp}%
\special{pa 591 -232}\special{pa 591 -244}\special{fp}\special{pa 591 -271}\special{pa 591 -283}\special{fp}%
\special{pa 591 -311}\special{pa 591 -323}\special{fp}\special{pa 591 -351}\special{pa 591 -363}\special{fp}%
\special{pa 591 -390}\special{pa 591 -402}\special{fp}\special{pa 591 -430}\special{pa 591 -442}\special{fp}%
\special{pa 591 -470}\special{pa 591 -482}\special{fp}\special{pa 591 -509}\special{pa 591 -521}\special{fp}%
\special{pa 591 -549}\special{pa 591 -561}\special{fp}\special{pa 591 -588}\special{pa 591 -600}\special{fp}%
\special{pa 591 -628}\special{pa 591 -640}\special{fp}\special{pa 591 -668}\special{pa 591 -680}\special{fp}%
\special{pa 591 -707}\special{pa 591 -719}\special{fp}\special{pa 591 -747}\special{pa 591 -759}\special{fp}%
\special{pa 591 -787}\special{pa 591 -799}\special{fp}\special{pa 591 -826}\special{pa 591 -838}\special{fp}%
\special{pa 591 -866}\special{pa 591 -878}\special{fp}\special{pa 591 -905}\special{pa 591 -917}\special{fp}%
\special{pa 591 -945}\special{pa 591 -957}\special{fp}\special{pa 591 -985}\special{pa 591 -997}\special{fp}%
\special{pa 591 -1024}\special{pa 591 -1036}\special{fp}\special{pa 591 -1064}\special{pa 591 -1076}\special{fp}%
\special{pa 595 -1105}\special{pa 587 -1114}\special{fp}\special{pa 561 -1113}\special{pa 549 -1114}\special{fp}%
\special{pa 521 -1114}\special{pa 509 -1114}\special{fp}\special{pa 482 -1114}\special{pa 470 -1114}\special{fp}%
\special{pa 442 -1114}\special{pa 430 -1114}\special{fp}\special{pa 402 -1114}\special{pa 390 -1114}\special{fp}%
\special{pa 363 -1114}\special{pa 351 -1114}\special{fp}\special{pa 323 -1114}\special{pa 311 -1114}\special{fp}%
\special{pa 283 -1114}\special{pa 271 -1114}\special{fp}\special{pa 244 -1114}\special{pa 232 -1114}\special{fp}%
\special{pa 204 -1114}\special{pa 192 -1114}\special{fp}\special{pa 165 -1114}\special{pa 153 -1114}\special{fp}%
\special{pa 125 -1114}\special{pa 113 -1114}\special{fp}\special{pa 85 -1114}\special{pa 73 -1114}\special{fp}%
\special{pa 46 -1114}\special{pa 34 -1114}\special{fp}\special{pa 6 -1114}\special{pa -6 -1114}\special{fp}%
\special{pn 8}%
\special{pn 12}%
\special{pa 984 6}\special{pa 984 -6}\special{fp}\special{pa 984 -33}\special{pa 984 -45}\special{fp}%
\special{pa 984 -72}\special{pa 984 -84}\special{fp}\special{pa 984 -111}\special{pa 984 -123}\special{fp}%
\special{pa 984 -151}\special{pa 984 -163}\special{fp}\special{pa 984 -190}\special{pa 984 -202}\special{fp}%
\special{pa 984 -229}\special{pa 984 -241}\special{fp}\special{pa 984 -268}\special{pa 984 -280}\special{fp}%
\special{pa 984 -307}\special{pa 984 -319}\special{fp}\special{pa 984 -346}\special{pa 984 -358}\special{fp}%
\special{pa 984 -386}\special{pa 984 -398}\special{fp}\special{pa 984 -425}\special{pa 984 -437}\special{fp}%
\special{pa 984 -464}\special{pa 984 -476}\special{fp}\special{pa 984 -503}\special{pa 984 -515}\special{fp}%
\special{pa 984 -542}\special{pa 984 -554}\special{fp}\special{pa 984 -581}\special{pa 984 -593}\special{fp}%
\special{pa 984 -621}\special{pa 984 -633}\special{fp}\special{pa 984 -660}\special{pa 984 -672}\special{fp}%
\special{pa 984 -699}\special{pa 984 -711}\special{fp}\special{pa 984 -738}\special{pa 984 -750}\special{fp}%
\special{pa 984 -777}\special{pa 984 -789}\special{fp}\special{pa 984 -816}\special{pa 984 -828}\special{fp}%
\special{pa 984 -856}\special{pa 984 -868}\special{fp}\special{pa 984 -895}\special{pa 984 -907}\special{fp}%
\special{pa 984 -934}\special{pa 984 -946}\special{fp}\special{pa 984 -973}\special{pa 984 -985}\special{fp}%
\special{pa 984 -1012}\special{pa 984 -1024}\special{fp}\special{pa 984 -1051}\special{pa 984 -1063}\special{fp}%
\special{pa 984 -1091}\special{pa 984 -1103}\special{fp}\special{pa 984 -1130}\special{pa 984 -1142}\special{fp}%
\special{pa 984 -1169}\special{pa 984 -1181}\special{fp}\special{pa 984 -1208}\special{pa 984 -1220}\special{fp}%
\special{pa 984 -1247}\special{pa 984 -1259}\special{fp}\special{pa 984 -1286}\special{pa 984 -1298}\special{fp}%
\special{pa 984 -1326}\special{pa 984 -1338}\special{fp}\special{pa 984 -1365}\special{pa 984 -1377}\special{fp}%
\special{pa 984 -1404}\special{pa 984 -1416}\special{fp}\special{pa 984 -1443}\special{pa 984 -1455}\special{fp}%
\special{pa 984 -1482}\special{pa 984 -1494}\special{fp}\special{pa 984 -1521}\special{pa 984 -1533}\special{fp}%
\special{pa 984 -1561}\special{pa 984 -1573}\special{fp}\special{pa 984 -1600}\special{pa 984 -1612}\special{fp}%
\special{pa 984 -1639}\special{pa 984 -1651}\special{fp}\special{pa 984 -1678}\special{pa 984 -1690}\special{fp}%
\special{pa 984 -1717}\special{pa 984 -1729}\special{fp}\special{pa 984 -1756}\special{pa 984 -1768}\special{fp}%
\special{pa 984 -1795}\special{pa 984 -1807}\special{fp}\special{pa 984 -1835}\special{pa 984 -1847}\special{fp}%
\special{pa 984 -1874}\special{pa 984 -1886}\special{fp}\special{pa 984 -1913}\special{pa 984 -1925}\special{fp}%
\special{pa 984 -1952}\special{pa 984 -1964}\special{fp}\special{pa 984 -1991}\special{pa 984 -2003}\special{fp}%
\special{pa 984 -2030}\special{pa 984 -2042}\special{fp}\special{pa 984 -2070}\special{pa 984 -2082}\special{fp}%
\special{pa 984 -2109}\special{pa 984 -2121}\special{fp}\special{pa 984 -2148}\special{pa 984 -2160}\special{fp}%
\special{pa 985 -2187}\special{pa 984 -2199}\special{fp}\special{pa 984 -2223}\special{pa 975 -2231}\special{fp}%
\special{pa 946 -2227}\special{pa 934 -2227}\special{fp}\special{pa 907 -2227}\special{pa 895 -2227}\special{fp}%
\special{pa 868 -2227}\special{pa 856 -2227}\special{fp}\special{pa 828 -2227}\special{pa 816 -2227}\special{fp}%
\special{pa 789 -2227}\special{pa 777 -2227}\special{fp}\special{pa 750 -2227}\special{pa 738 -2227}\special{fp}%
\special{pa 711 -2227}\special{pa 699 -2227}\special{fp}\special{pa 672 -2227}\special{pa 660 -2227}\special{fp}%
\special{pa 633 -2227}\special{pa 621 -2227}\special{fp}\special{pa 593 -2227}\special{pa 581 -2227}\special{fp}%
\special{pa 554 -2227}\special{pa 542 -2227}\special{fp}\special{pa 515 -2227}\special{pa 503 -2227}\special{fp}%
\special{pa 476 -2227}\special{pa 464 -2227}\special{fp}\special{pa 437 -2227}\special{pa 425 -2227}\special{fp}%
\special{pa 398 -2227}\special{pa 386 -2227}\special{fp}\special{pa 358 -2227}\special{pa 346 -2227}\special{fp}%
\special{pa 319 -2227}\special{pa 307 -2227}\special{fp}\special{pa 280 -2227}\special{pa 268 -2227}\special{fp}%
\special{pa 241 -2227}\special{pa 229 -2227}\special{fp}\special{pa 202 -2227}\special{pa 190 -2227}\special{fp}%
\special{pa 163 -2227}\special{pa 151 -2227}\special{fp}\special{pa 123 -2227}\special{pa 111 -2227}\special{fp}%
\special{pa 84 -2227}\special{pa 72 -2227}\special{fp}\special{pa 45 -2227}\special{pa 33 -2227}\special{fp}%
\special{pa 6 -2227}\special{pa -6 -2227}\special{fp}\special{pn 8}%
\special{pa -1575    -0}\special{pa  1575    -0}%
\special{fp}%
\special{pa     0   197}\special{pa     0 -2756}%
\special{fp}%
\settowidth{\Width}{$x$}\setlength{\Width}{0\Width}%
\settoheight{\Height}{$x$}\settodepth{\Depth}{$x$}\setlength{\Height}{-0.5\Height}\setlength{\Depth}{0.5\Depth}\addtolength{\Height}{\Depth}%
\put(4.0500000,0.0000000){\hspace*{\Width}\raisebox{\Height}{$x$}}%
%
\settowidth{\Width}{$y$}\setlength{\Width}{-0.5\Width}%
\settoheight{\Height}{$y$}\settodepth{\Depth}{$y$}\setlength{\Height}{\Depth}%
\put(0.0000000,7.0500000){\hspace*{\Width}\raisebox{\Height}{$y$}}%
%
\settowidth{\Width}{O}\setlength{\Width}{-1\Width}%
\settoheight{\Height}{O}\settodepth{\Depth}{O}\setlength{\Height}{-\Height}%
\put(-0.0500000,-0.0500000){\hspace*{\Width}\raisebox{\Height}{O}}%
%
\end{picture}}%}}
\end{layer}

\begin{itemize}
\item
[課題]\monbannoadd\\\hspace*{0.5zw}$2^{0.5},2^{1.5},2^{2.5}$\\はどうなりそうか
\item
[]{\color{blue}$2^{0.5}=\sqrt{2}$}
\item
[]{\color{blue}$2^{1.5}=2\sqrt{2}$}
\item
[]{\color{blue}$2^{2.5}=4\sqrt{2}$}
\end{itemize}
\addban

\newslide{$x$が整数でない場合(指数法則から)}

\vspace*{18mm}


\begin{layer}{120}{0}
\putnotew{96}{73}{\hyperlink{para7pg4}{\fbox{\Ctab{2.5mm}{\scalebox{1}{\scriptsize $\mathstrut||\!\lhd$}}}}}
\putnotew{101}{73}{\hyperlink{para8pg1}{\fbox{\Ctab{2.5mm}{\scalebox{1}{\scriptsize $\mathstrut|\!\lhd$}}}}}
\putnotew{108}{73}{\hyperlink{para8pg15}{\fbox{\Ctab{4.5mm}{\scalebox{1}{\scriptsize $\mathstrut\!\!\lhd\!\!$}}}}}
\putnotew{115}{73}{\hyperlink{para8pg16}{\fbox{\Ctab{4.5mm}{\scalebox{1}{\scriptsize $\mathstrut\!\rhd\!$}}}}}
\putnotew{120}{73}{\hyperlink{para8pg16}{\fbox{\Ctab{2.5mm}{\scalebox{1}{\scriptsize $\mathstrut \!\rhd\!\!|$}}}}}
\putnotew{125}{73}{\hyperlink{para9pg1}{\fbox{\Ctab{2.5mm}{\scalebox{1}{\scriptsize $\mathstrut \!\rhd\!\!||$}}}}}
\putnotee{126}{73}{\scriptsize\color{blue} 16/16}
\end{layer}

\slidepage
\begin{itemize}
\item
$(a^p)^{q}=a^{pq}$で,$p=0.5,\ q=2$とする\\
\hspace*{2zw}$(2^{0.5})^2$
$=a^{0.5\times 2}$
$=2^1=2$
\item
[]$2^{0.5}$は2乗すると2になる数だから,
$\pm\sqrt{2}$のどちらか
\\$2^{0.5}>0$と決めると
\\\hspace*{2zw}\fbox{\color{red}$2^{0.5}=\sqrt{2}$}\hspace{1zw}または\hspace{1zw}\fbox{\color{red}$2^{\sfrac{1}{2}}=\sqrt{2}$}
\item
$2^{1.5}=2^{1+0.5}=2^1\cdot 2^{0.5}=2\sqrt{2}$
\item
$2^{2.5}=2^{2+0.5}=2^2\cdot 2^{0.5}=4\sqrt{2}$
\end{itemize}

\newslide{指数法則と$n$乗根}

\vspace*{18mm}


\begin{layer}{120}{0}
\putnotew{96}{73}{\hyperlink{para8pg16}{\fbox{\Ctab{2.5mm}{\scalebox{1}{\scriptsize $\mathstrut||\!\lhd$}}}}}
\putnotew{101}{73}{\hyperlink{para9pg1}{\fbox{\Ctab{2.5mm}{\scalebox{1}{\scriptsize $\mathstrut|\!\lhd$}}}}}
\putnotew{108}{73}{\hyperlink{para9pg4}{\fbox{\Ctab{4.5mm}{\scalebox{1}{\scriptsize $\mathstrut\!\!\lhd\!\!$}}}}}
\putnotew{115}{73}{\hyperlink{para9pg5}{\fbox{\Ctab{4.5mm}{\scalebox{1}{\scriptsize $\mathstrut\!\rhd\!$}}}}}
\putnotew{120}{73}{\hyperlink{para9pg5}{\fbox{\Ctab{2.5mm}{\scalebox{1}{\scriptsize $\mathstrut \!\rhd\!\!|$}}}}}
\putnotew{125}{73}{\hyperlink{para10pg1}{\fbox{\Ctab{2.5mm}{\scalebox{1}{\scriptsize $\mathstrut \!\rhd\!\!||$}}}}}
\putnotee{126}{73}{\scriptsize\color{blue} 5/5}
\end{layer}

\slidepage
\begin{itemize}
\item
\fbox{$\begin{array}{l}
(1)\ \ a^p a^{q}=a^{p+q}\\
(2)\ \ (a^p)^{q}=a^{pq}\\
(3)\ \ (ab)^p=a^p b^p
\end{array}$}($a>0,\ b>0$)
\item
$a^{\sfrac{1}{3}}$は3乗すると$a$になる正の数
\item
[]これを$\sqrt[3]{a}$と書く(3乗根)\\
\hspace*{2zw}\fbox{\color{red}$2^{\frac{1}{3}}=\sqrt[3]{2}$}
\item
[課題]\monban $3^{\sfrac{1}{2}},\ 5^{\sfrac{1}{2}},\ 4^{\sfrac{1}{3}},\ 8^{\sfrac{1}{3}}$を求めよ
\end{itemize}

\newslide{指数の計算(TextP188)}

\vspace*{18mm}


\begin{layer}{120}{0}
\putnotew{96}{73}{\hyperlink{para9pg5}{\fbox{\Ctab{2.5mm}{\scalebox{1}{\scriptsize $\mathstrut||\!\lhd$}}}}}
\putnotew{101}{73}{\hyperlink{para10pg1}{\fbox{\Ctab{2.5mm}{\scalebox{1}{\scriptsize $\mathstrut|\!\lhd$}}}}}
\putnotew{108}{73}{\hyperlink{para10pg12}{\fbox{\Ctab{4.5mm}{\scalebox{1}{\scriptsize $\mathstrut\!\!\lhd\!\!$}}}}}
\putnotew{115}{73}{\hyperlink{para10pg13}{\fbox{\Ctab{4.5mm}{\scalebox{1}{\scriptsize $\mathstrut\!\rhd\!$}}}}}
\putnotew{120}{73}{\hyperlink{para10pg13}{\fbox{\Ctab{2.5mm}{\scalebox{1}{\scriptsize $\mathstrut \!\rhd\!\!|$}}}}}
\putnotew{125}{73}{\hyperlink{para11pg1}{\fbox{\Ctab{2.5mm}{\scalebox{1}{\scriptsize $\mathstrut \!\rhd\!\!||$}}}}}
\putnotee{126}{73}{\scriptsize\color{blue} 13/13}
\end{layer}

\slidepage
\begin{enumerate}[(1)]
\item
$8^{\sfrac{2}{3}}=(2^3)^{\sfrac{2}{3}}$
$=2^{3\cdot \sfrac{2}{3}}$
$=2^2=4$
\item
$(64^{\sfrac{1}{2}})^{\sfrac{1}{3}}=((2^6)^{\sfrac{1}{2}})^{\sfrac{1}{3}}$
$=2^{6\times\sfrac{1}{2}\times\sfrac{1}{3}}$
$=2^1=2$
\item
$(8^{\sfrac{1}{6}})^{-2}=((2^3)^{\sfrac{1}{6}})^{-2}$
$=2^{3\times\sfrac{1}{6}\times(-2)}$
=$2^{-1}$
$=\bunsuu{1}{2}$
\item
[課題]\monban 計算せよ\hfill TextP188\seteda{60}\\
\eda{$32^{\tfrac{2}{5}}$}\eda{$\sqrt[3]{27}$}\\
\eda{$(\sqrt[2]{4})^{\tfrac{1}{2}}$}\eda{$(\sqrt[2]{4})^{-\sfrac{1}{2}}$}
\end{enumerate}

\newslide{指数関数のグラフの特徴}

\vspace*{18mm}


\begin{layer}{120}{0}
\putnotew{96}{73}{\hyperlink{para10pg13}{\fbox{\Ctab{2.5mm}{\scalebox{1}{\scriptsize $\mathstrut||\!\lhd$}}}}}
\putnotew{125}{73}{\hyperlink{para12pg1}{\fbox{\Ctab{2.5mm}{\scalebox{1}{\scriptsize $\mathstrut \!\rhd\!\!||$}}}}}
\putnotee{126}{73}{\scriptsize\color{blue} 1/1}
\end{layer}

\slidepage
\begin{itemize}
\item
[課題]\monban $y=2^x,\ y=3^x,\ y=(\frac{1}{2})^x,\ y=1^x$のグラフをかき、( )に当てはまる言葉を入れよ.
 
\item
指数関数$y=a^x$の特徴
\begin{itemize}
\item
[[1\!\!]] $y$の値はいつでも(\hspace{4zw})
\item
[[2\!\!]] $a>1$のとき,グラフは右(\hspace{4zw})
\item
[[3\!\!]] $0<a<1$のとき,グラフは右(\hspace{4zw})
\item
[[4\!\!]] $a=1$のとき,グラフは(\hspace{4zw})
\end{itemize}
\end{itemize}
%%%%%%%%%%%%

%%%%%%%%%%%%%%%%%%%%


\newslide{指数方程式(TextP191)}

\vspace*{18mm}


\begin{layer}{120}{0}
\putnotew{96}{73}{\hyperlink{para11pg1}{\fbox{\Ctab{2.5mm}{\scalebox{1}{\scriptsize $\mathstrut||\!\lhd$}}}}}
\putnotew{101}{73}{\hyperlink{para12pg1}{\fbox{\Ctab{2.5mm}{\scalebox{1}{\scriptsize $\mathstrut|\!\lhd$}}}}}
\putnotew{108}{73}{\hyperlink{para12pg9}{\fbox{\Ctab{4.5mm}{\scalebox{1}{\scriptsize $\mathstrut\!\!\lhd\!\!$}}}}}
\putnotew{115}{73}{\hyperlink{para12pg10}{\fbox{\Ctab{4.5mm}{\scalebox{1}{\scriptsize $\mathstrut\!\rhd\!$}}}}}
\putnotew{120}{73}{\hyperlink{para12pg10}{\fbox{\Ctab{2.5mm}{\scalebox{1}{\scriptsize $\mathstrut \!\rhd\!\!|$}}}}}
\putnotew{125}{73}{\hyperlink{para13pg1}{\fbox{\Ctab{2.5mm}{\scalebox{1}{\scriptsize $\mathstrut \!\rhd\!\!||$}}}}}
\putnotee{126}{73}{\scriptsize\color{blue} 10/10}
\end{layer}

\slidepage
\vspace{3mm}

\begin{minipage}[t]{60mm}
(1)\ $16^x=8$\\
\hspace*{2zw}$(2^4)^x=2^3$\\
\hspace*{2zw}$2^{4x}=2^3$\\
\hspace*{1zw}指数を等しいとおいて\\
\hspace*{2zw}$4x=3$\\
\hspace*{1zw}よって $x=\dfrac{3}{4}$
\end{minipage}
\hspace*{5mm}\begin{minipage}[t]{60mm}
(2)\ $8^x=2^{x+1}$\\
\hspace*{2zw}$(2^3)^x=2^{x+1}$\\
\hspace*{2zw}$2^{3x}=2^{x+1}$\\
\hspace*{1zw}指数を等しいとおいて\\
\hspace*{2zw}$3x=x+1$\\
\hspace*{1zw}よって $x=\dfrac{1}{2}$
\end{minipage}
\vspace{-5mm}

\seteda{55}
\begin{itemize}
\item
[課題]\monban 次の方程式を解け\hfill TextP191\\
\eda{$8^{x}=\bunsuu{1}{32}$}\eda{$81^x=3^{3-2x}$}
\end{itemize}

\mainslide{対数関数}


\slidepage[m]
%%%%%%%%%%%%

%%%%%%%%%%%%%%%%%%%%

\newslide{対数の定義}

\vspace*{18mm}


\begin{layer}{120}{0}
\putnotew{96}{73}{\hyperlink{para12pg10}{\fbox{\Ctab{2.5mm}{\scalebox{1}{\scriptsize $\mathstrut||\!\lhd$}}}}}
\putnotew{101}{73}{\hyperlink{para13pg1}{\fbox{\Ctab{2.5mm}{\scalebox{1}{\scriptsize $\mathstrut|\!\lhd$}}}}}
\putnotew{108}{73}{\hyperlink{para13pg9}{\fbox{\Ctab{4.5mm}{\scalebox{1}{\scriptsize $\mathstrut\!\!\lhd\!\!$}}}}}
\putnotew{115}{73}{\hyperlink{para13pg10}{\fbox{\Ctab{4.5mm}{\scalebox{1}{\scriptsize $\mathstrut\!\rhd\!$}}}}}
\putnotew{120}{73}{\hyperlink{para13pg10}{\fbox{\Ctab{2.5mm}{\scalebox{1}{\scriptsize $\mathstrut \!\rhd\!\!|$}}}}}
\putnotew{125}{73}{\hyperlink{para14pg1}{\fbox{\Ctab{2.5mm}{\scalebox{1}{\scriptsize $\mathstrut \!\rhd\!\!||$}}}}}
\putnotee{126}{73}{\scriptsize\color{blue} 10/10}
\end{layer}

\slidepage
\begin{itemize}
\item
$y=\log_a x$\\
\hspace*{4zw}$a$を{\color{red}底},$x$を{\color{red}真数}\\
\hspace*{4zw}$y$を$a$を底とする$x$の{\color{red}対数}という.
\item
対数$y$は,$a$を何乗したら$x$になるかという数\\
\hspace*{4zw}$a^{\fbox{$y$}}=x$となる\;\fbox{$\mathstrut y$}\;のこと
\item
[例)]$y=\log_3 9$\\
\hspace*{1zw}$3^{\fbox{$y$}}=9$となる$y$のこと\\
\hspace*{1zw}$3^2=9$だから
\hspace*{1zw}$y=\log_3 9=2$
\end{itemize}

\newslide{対数の値を求める}

\vspace*{18mm}


\begin{layer}{120}{0}
\putnotew{96}{73}{\hyperlink{para13pg10}{\fbox{\Ctab{2.5mm}{\scalebox{1}{\scriptsize $\mathstrut||\!\lhd$}}}}}
\putnotew{101}{73}{\hyperlink{para14pg1}{\fbox{\Ctab{2.5mm}{\scalebox{1}{\scriptsize $\mathstrut|\!\lhd$}}}}}
\putnotew{108}{73}{\hyperlink{para14pg5}{\fbox{\Ctab{4.5mm}{\scalebox{1}{\scriptsize $\mathstrut\!\!\lhd\!\!$}}}}}
\putnotew{115}{73}{\hyperlink{para14pg6}{\fbox{\Ctab{4.5mm}{\scalebox{1}{\scriptsize $\mathstrut\!\rhd\!$}}}}}
\putnotew{120}{73}{\hyperlink{para14pg6}{\fbox{\Ctab{2.5mm}{\scalebox{1}{\scriptsize $\mathstrut \!\rhd\!\!|$}}}}}
\putnotew{125}{73}{\hyperlink{para15pg1}{\fbox{\Ctab{2.5mm}{\scalebox{1}{\scriptsize $\mathstrut \!\rhd\!\!||$}}}}}
\putnotee{126}{73}{\scriptsize\color{blue} 6/6}
\end{layer}

\slidepage
\begin{itemize}
\item
\fbox{$y=\log_a x \Longleftrightarrow a^y=x$}
\item
[(例)]$y=\log_2 16$
$\Longleftrightarrow 2^y=16$
$=2^4$\\
 $y=4$となるから $\log_2 16=4$
\seteda{30}
\item
[課題]\monban 次の値を求めよ.\\
\eda{$\log_2 8$ }\eda{$\log_3 3$}\eda{$\log_5 \bunsuu{1}{5}$}\eda{$\log_2 \bunsuu{1}{4}$}
\end{itemize}

\newslide{対数法則}

\vspace*{18mm}


\begin{layer}{120}{0}
\putnotew{96}{73}{\hyperlink{para14pg6}{\fbox{\Ctab{2.5mm}{\scalebox{1}{\scriptsize $\mathstrut||\!\lhd$}}}}}
\putnotew{101}{73}{\hyperlink{para15pg1}{\fbox{\Ctab{2.5mm}{\scalebox{1}{\scriptsize $\mathstrut|\!\lhd$}}}}}
\putnotew{108}{73}{\hyperlink{para15pg9}{\fbox{\Ctab{4.5mm}{\scalebox{1}{\scriptsize $\mathstrut\!\!\lhd\!\!$}}}}}
\putnotew{115}{73}{\hyperlink{para15pg10}{\fbox{\Ctab{4.5mm}{\scalebox{1}{\scriptsize $\mathstrut\!\rhd\!$}}}}}
\putnotew{120}{73}{\hyperlink{para15pg10}{\fbox{\Ctab{2.5mm}{\scalebox{1}{\scriptsize $\mathstrut \!\rhd\!\!|$}}}}}
\putnotew{125}{73}{\hyperlink{para16pg1}{\fbox{\Ctab{2.5mm}{\scalebox{1}{\scriptsize $\mathstrut \!\rhd\!\!||$}}}}}
\putnotee{126}{73}{\scriptsize\color{blue} 10/10}
\end{layer}

\slidepage
{\color{blue}

\begin{layer}{120}{0}
\putnotee{80}{10}{\normalsize$\log_4 2+\log_4 8=\log_4 16=2$}
\putnotee{80}{21}{\normalsize$\log_3 15-\log_3 5=\log_3\dfrac{15}{5}=1$}
\putnotee{80}{32}{\normalsize$\log_2 4^5=5\log_2 2=5$}
\end{layer}

}
\begin{itemize}
\item
[(1)]$\log_a b+\log_a c=\log_a(bc)$\vspace{-1mm}
\item
[(2)]$\log_a b-\log_a c=\log_a\bunsuu{\ b\ }{c}$\vspace{-1mm}
\item
[(3)]$\log_a b^{\,p}=p\log_a b$\vspace{-1mm}
\item
[課題]\monbannoadd 対数法則の具体例を書け\seteda{33}\\
\eda{(1)の例}\eda{(2)の例}\eda{(3)の例}
\item
[証明](1)
$\log_a b=x,\log_a c=x$とおくと \ %
$a^x=b,a^y=c$\\
\phantom{(1)} $a^{x+y}=a^x a^y=bc$となるから\ %
$x+y=\log_a{bc}$
\end{itemize}
\addban

\newslide{対数の計算}

\vspace*{18mm}


\begin{layer}{120}{0}
\putnotew{96}{73}{\hyperlink{para15pg10}{\fbox{\Ctab{2.5mm}{\scalebox{1}{\scriptsize $\mathstrut||\!\lhd$}}}}}
\putnotew{101}{73}{\hyperlink{para16pg1}{\fbox{\Ctab{2.5mm}{\scalebox{1}{\scriptsize $\mathstrut|\!\lhd$}}}}}
\putnotew{108}{73}{\hyperlink{para16pg11}{\fbox{\Ctab{4.5mm}{\scalebox{1}{\scriptsize $\mathstrut\!\!\lhd\!\!$}}}}}
\putnotew{115}{73}{\hyperlink{para16pg12}{\fbox{\Ctab{4.5mm}{\scalebox{1}{\scriptsize $\mathstrut\!\rhd\!$}}}}}
\putnotew{120}{73}{\hyperlink{para16pg12}{\fbox{\Ctab{2.5mm}{\scalebox{1}{\scriptsize $\mathstrut \!\rhd\!\!|$}}}}}
\putnotew{125}{73}{\hyperlink{para17pg1}{\fbox{\Ctab{2.5mm}{\scalebox{1}{\scriptsize $\mathstrut \!\rhd\!\!||$}}}}}
\putnotee{126}{73}{\scriptsize\color{blue} 12/12}
\end{layer}

\slidepage
\begin{itemize}
\item
[(1)]$\log_{10} 5+\log_{10} 2$
\item
[]与式$=\log_{10}(5\times 2)$
$=\log_{10} 10$
$=1$
\item
[(2)]$\log_2 12-\log_2 3$
\item
[]与式$=\log_2(\bunsuu{12}{3})$
$=\log_2 4$
$=2$
\item
[(3)]$2\log_3 4+\log_3 4-\log_3 8$
\item
[]与式$=\log_3 4^2+\log_3 4-\log_3 8$\\
\phantom{与式}$=\log_3 \frac{16\times 4}{8}$
$=\log_3 8$
\end{itemize}

\newslide{対数の計算(課題)}

\vspace*{18mm}

%%repeat=4
\slidepage
\begin{itemize}
\item
[課題]\monban 次の計算をせよ.\hfill TextP192\seteda{100}\\
\eda{$2\log_4 3-\log_4 36$}\\
\eda{$\log_3 \bunsuu{3}{4}+\log_3 24-\log_3 2$}\\
\eda{$\log_3 18+\log_3 8-4\log_3 2$}\\
\eda{$\log_3 4+\log_3 18- 3\log_3 2$}
\end{itemize}
%%%%%%%%%%%%

%%%%%%%%%%%%%%%%%%%%

\newslide{底・真数・対数の条件}

\vspace*{18mm}


\begin{layer}{120}{0}
\putnotew{96}{73}{\hyperlink{para16pg12}{\fbox{\Ctab{2.5mm}{\scalebox{1}{\scriptsize $\mathstrut||\!\lhd$}}}}}
\putnotew{101}{73}{\hyperlink{para17pg1}{\fbox{\Ctab{2.5mm}{\scalebox{1}{\scriptsize $\mathstrut|\!\lhd$}}}}}
\putnotew{108}{73}{\hyperlink{para17pg9}{\fbox{\Ctab{4.5mm}{\scalebox{1}{\scriptsize $\mathstrut\!\!\lhd\!\!$}}}}}
\putnotew{115}{73}{\hyperlink{para17pg10}{\fbox{\Ctab{4.5mm}{\scalebox{1}{\scriptsize $\mathstrut\!\rhd\!$}}}}}
\putnotew{120}{73}{\hyperlink{para17pg10}{\fbox{\Ctab{2.5mm}{\scalebox{1}{\scriptsize $\mathstrut \!\rhd\!\!|$}}}}}
\putnotew{125}{73}{\hyperlink{para18pg1}{\fbox{\Ctab{2.5mm}{\scalebox{1}{\scriptsize $\mathstrut \!\rhd\!\!||$}}}}}
\putnotee{126}{73}{\scriptsize\color{blue} 10/10}
\end{layer}

\slidepage
\begin{itemize}
\item
$y=\log_a x \Longleftrightarrow a^y=x$
\item
$\log_a 1$
$=0$
\item
$\log_a a$
$=1$
\item
底$a$の条件は
\hspace{1zw}\fbox{$a>0,a\neqq 1$}
\item
真数$x$の条件は
\hspace{1zw}\fbox{$x>0$}
\item
対数$y=\log_a x$の範囲は
\hspace{1zw}\fbox{実数全部}
\end{itemize}

\newslide{対数関数のグラフ}

\vspace*{18mm}

\slidepage
\begin{itemize}
\item
$y=\log_a x$\hspace*{2zw}KeTMathでは\ \verb|log(a,x)|
\item
アプリ「関数のグラフ」でかいてみよう
\item
$x$の範囲が全実数でない \verb|x=0.01,10|などとする.
\item
[課題]\monban グラフをかいて,問いに答えよ\seteda{90}\\
\eda{$y=\log_2 x,\ \log_4 x,\ \log_{\frac{1}{2}}x,\ \log_2(-x)$}\\
\eda{$y=\log_a x$の$a$を変えるとどうなるか}\\
\eda{$y=\log_a x$と$y=\log_a(-x)$はどのような関係か}
\end{itemize}
\label{pageend}\mbox{}

\end{document}

\documentclass[10pt,dvipdfmx]{jarticle}
\usepackage{amsmath,amssymb}
\usepackage{graphicx}
\usepackage{color}
\usepackage[dvipdfmx]{pict2e}
\usepackage{ketpic2e}
\usepackage{ketlayer2e}
\usepackage{ketlayermorewith2e}
\setmargin{15}{15}{15}{15}
\pagestyle{headings}
\begin{document}
\begin{center}
\verb|question0711-01.txt|\\
\end{center}

$Q1\;2$\\
$[]\;\text{導関数の定義式を書け}$\\
\text{Sheet} 
$[]\;f'(x)=\;\;::2::-1$ 
\\
\text{Ans}\\
$[]\;f'(x)=\displaystyle\lim_{z \to z} (\dfrac{f(z)-f(x)}{z-x})$\\
$Q2\;2$\\
$[]\;x^p\text{の微分公式を書け}$\\
\text{Sheet} 
$[]\;(x^p)'=\;\;::2$ 
\\
\text{Ans}\\
$[]\;px^{p-1}$\\
$Q3\;6$\\
$\text{微分せよ}$\\
$[1]\;y=x$\\
$[2]\;y=x^3$\\
$[3]\;y=x^6$\\
\text{Sheet} 
$[1]\;y'=\;::2$ 
$[2]\;y'=\;::2$ 
$[3]\;y'=\;::2$ 
\\
\text{Ans}\\
$[1]\;y'=1$\\
$[2]\;y'=3x^2$\\
$[3]\;y'=6x^5$\\
$Q4\;6$\\
$\text{微分せよ}$\\
$[1]\;y=x^{\frac{1}{2}}$\\
$[2]\;y=x^{-1}$\\
$[3]\;y=\dfrac{1}{x^2}$\\
\text{Sheet} 
$[1]\;y'=\;::2$ 
$[2]\;y'=\;::2$ 
$[3]\;y'=\;::2$ 
\\
\text{Ans}\\
$[1]\;y'=\dfrac{1}{2}x^{-\frac{1}{2}}$\\
$[2]\;y'=-\dfrac{1}{x^2}$\\
$[3]\;y'=-\dfrac{2}{x^3}$\\
$Q5\;4$\\
$\text{微分せよ}$\\
$[1]\;y=x+\dfrac{1}{x}$\\
$[2]\;y=\sqrt{x}+\dfrac{1}{\sqrt{x}}$\\
\text{Sheet} 
$[1]\;y'=\;::2$ 
$[2]\;y'=\;::2$ 
\\
\text{Ans}\\
$[1]\;y'=1-\dfrac{1}{x^2}$\\
$[2]\;y'=\dfrac{1}{2\sqrt{x}}-\dfrac{1}{2x\sqrt{x}}$\\
\newpage
\begin{center}
\verb|question0711-06.txt|\\
\end{center}
$Q6\;4$\\
$\sin x ,\cos x \text{の微分公式を書け}$\\
$[1]\;(\sin x )'=$\\
$[2]\;(\cos x )'=$\\
\text{Sheet} 
$[1]\;(\sin x )'=\;\;::2$ 
$[2]\;(\cos x )'=\;\;::2$ 
\\
\text{Ans}\\
$[1]\;\cos x $\\
$[2]\;-\sin x \;$\\
$Q7\;4$\\
${\cos^2x}\text{の意味は次のどれか}$\\
$[]\;1\;\cos x^2 \text{ }2\;(\cos x )^2\text{ }3\;(cos)^2x$\\
\text{Sheet} 
$[]\;\text{番号}=\;\;::4$ 
\\
\text{Ans}\\
$[]\;2$\\
$Q8\;4$\\
$\tan x \text{の微分公式を書け}$\\
$[]\;(\tan x )'=$\\
\text{Sheet} 
$[]\;\;(\tan x )'=\;\;::4::-1$ 
\\
\text{Ans}\\
$[]\;\dfrac{1}{\cos^{2}\! x }$\\
\newpage
\begin{center}
\verb|question0711-09.txt|\\
\end{center}
$Q9\;4$\\
$\text{積の微分公式を書け}$\\
$[]\;(fg)'=$\\
\text{Sheet} 
$[]\;=\;\;::4::-1$ 
\\
\text{Ans}\\
$[]\;(fg)'=f'g+fg'$\\
$Q10\;4$\\
$\text{商の微分公式を書け}$\\
$[]\;\left(\begin{array}{c}\dfrac{f}{g}\end{array}\right)'=$\\
\text{Sheet} 
$[]\;=\;\;::4::-1$ 
\\
\text{Ans}\\
$[]\;=\dfrac{f'g-fg'}{g^2}$\\
$Q11\;6$\\
$\text{次の関数を微分せよ}$\\
$[1]\;y=x^2\;\sin x $\\
$[2]\;y=\sin^{2}\! x $\\
$[3]\;y=\dfrac{\cos x }{\sin x }$\\
\text{Sheet} 
$[1]\;=\;::2::-1$ 
$[2]\;=\;::2::-1$ 
$[3]\;=\;::2::-1$ 
\\
\text{Ans}\\
$[1]\;2x\sin x +x^2\cos x $\\
$[2]\;2\sin x \cos x $\\
$[3]\;-\dfrac{1}{\sin^{2}\! x }$\\
$Q12\;4$\\
$\text{次の関数を微分せよ}$\\
$[1]\;y=(-3x+4)^5$\\
$[2]\;y=\sin (2x+\dfrac{{\pi}}{4}) $\\
\text{Sheet} 
$[1]\;=\;::2$ 
$[2]\;=\;::2::-1$ 
\\
\text{Ans}\\
$[1]\;-15(-3x+4)^4$\\
$[2]\;2\cos (2x+\dfrac{{\pi}}{4}) $\\
\newpage
\begin{center}
\verb|question0711-13.txt|\\
\end{center}
$Q13$\\
$y=a^x\text{の}(0,1)\text{における接線の傾きがちょうど}1\text{となるとき.}$\\
$[]\;a\text{を求めよ.}$\\
\text{Sheet} 
$[]\;a=\;\;::4$ 
\\
\text{Ans}\\
$[]\;a=2.7182818284$\\
\newpage
\begin{center}
\verb|question0711-14.txt|\\
\end{center}
$Q14$\\
$\text{微分せよ}$\\
$[1]\;y=e^{5x}$\\
$[2]\;y=e^{-2x}$\\
$[3]\;y=e^{3x+1}$\\
$[4]\;y=\dfrac{e^x+e^{-x}}{2}$\\
\text{Sheet} 
$[1]\;y'=\;::2$ 
$[2]\;y'=\;::2$ 
$[3]\;y'=\;::2$ 
$[4]\;y'=\;::2$ 
\\
\text{Ans}\\
$[1]\;y'=5e^{5x}$\\
$[2]\;y'=-2e^{-2x}$\\
$[3]\;y'=3e^{3x+1}$\\
$[4]\;y'=\dfrac{e^x-e^{-x}}{2}$\\
\newpage
\begin{center}
\verb|question0711-15.txt|\\
\end{center}
$Q15$\\
$\text{微分せよ}$\\
$[1]\;y=\log (-x) $\\
$[2]\;y=\log 2x $\\
$[3]\;y=\log (x+5) $\\
\text{Sheet} 
$[1]\;y'=\;::2$ 
$[2]\;y'=\;::2$ 
$[3]\;y'=\;::2$ 
\\
\text{Ans}\\
$[1]\;y'=\dfrac{1}{x}$\\
$[2]\;y'=\dfrac{1}{x}$\\
$[3]\;y'=\dfrac{1}{x+5}$\\
\newpage
\end{document}

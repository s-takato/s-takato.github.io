%%% タイトル presen22201
\documentclass[landscape,10pt]{jarticle}
\special{papersize=\the\paperwidth,\the\paperheight}
\usepackage{ketpic,ketlayer}
\usepackage{ketslide}
\usepackage{amsmath,amssymb}
\usepackage{bm,enumerate}
\usepackage[dvipdfmx]{graphicx}
\usepackage{color}
\definecolor{slidecolora}{cmyk}{0.98,0.13,0,0.43}
\definecolor{slidecolorb}{cmyk}{0.2,0,0,0}
\definecolor{slidecolorc}{cmyk}{0.2,0,0,0}
\definecolor{slidecolord}{cmyk}{0.2,0,0,0}
\definecolor{slidecolore}{cmyk}{0,0,0,0.5}
\definecolor{slidecolorf}{cmyk}{0,0,0,0.5}
\definecolor{slidecolori}{cmyk}{0.98,0.13,0,0.43}
\def\setthin#1{\def\thin{#1}}
\setthin{0}
\newcommand{\slidepage}[1][s]{%
\setcounter{ketpicctra}{18}%
\if#1m \setcounter{ketpicctra}{1}\fi
\hypersetup{linkcolor=black}%

\begin{layer}{118}{0}
\putnotee{122}{-\theketpicctra.05}{\small\thepage/\pageref{pageend}}
\end{layer}\hypersetup{linkcolor=blue}

}
\usepackage{emath}
\usepackage{pict2e}
\usepackage{ketlayermorewith2e}
\usepackage[dvipdfmx,colorlinks=true,linkcolor=blue,filecolor=blue]{hyperref}
\newcommand{\hiduke}{0627}
\newcommand{\hako}[2][1]{\fbox{\raisebox{#1mm}{\mbox{}}\raisebox{-#1mm}{\mbox{}}\,\phantom{#2}\,}}
\newcommand{\hakoa}[2][1]{\fbox{\raisebox{#1mm}{\mbox{}}\raisebox{-#1mm}{\mbox{}}\,#2\,}}
\newcommand{\hakom}[2][1]{\hako[#1]{$#2$}}
\newcommand{\hakoma}[2][1]{\hakoa[#1]{$#2$}}
\def\rad{\;\mathrm{rad}}
\def\deg#1{#1^{\circ}}
\newcommand{\sbunsuu}[2]{\scalebox{0.6}{$\bunsuu{#1}{#2}$}}
\def\pow{$\hspace{-1.5mm}^\hspace{-1mm}$}
\def\dlim{\displaystyle\lim}
\newcommand{\brd}[2][1]{\scalebox{#1}{\color{red}\fbox{\color{black}$#2$}}}
\newcommand\down[1][0.5zw]{\vspace{#1}\\}
\newcommand{\sfrac}[3][0.65]{\scalebox{#1}{$\frac{#2}{#3}$}}
\newcommand{\phn}[1]{\phantom{#1}}
\newcommand{\scb}[2][0.6]{\scalebox{#1}{#2}}
\newcommand{\dsum}{\displaystyle\sum}

\setmargin{25}{145}{15}{100}

\ketslideinit

\pagestyle{empty}

\begin{document}

\begin{layer}{120}{0}
\putnotese{0}{0}{{\Large\bf
\color[cmyk]{1,1,0,0}

\begin{layer}{120}{0}
{\Huge \putnotes{60}{20}{三角関数の性質}}
\putnotes{60}{70}{2022.05.16}
\end{layer}

}
}
\end{layer}

\def\mainslidetitley{22}
\def\ketcletter{slidecolora}
\def\ketcbox{slidecolorb}
\def\ketdbox{slidecolorc}
\def\ketcframe{slidecolord}
\def\ketcshadow{slidecolore}
\def\ketdshadow{slidecolorf}
\def\slidetitlex{6}
\def\slidetitlesize{1.3}
\def\mketcletter{slidecolori}
\def\mketcbox{yellow}
\def\mketdbox{yellow}
\def\mketcframe{yellow}
\def\mslidetitlex{62}
\def\mslidetitlesize{2}

\color{black}
\Large\bf\boldmath
\addtocounter{page}{-1}

\def\MARU{}
\renewcommand{\MARU}[1]{{\ooalign{\hfil$#1$\/\hfil\crcr\raise.167ex\hbox{\mathhexbox20D}}}}
\renewcommand{\slidepage}[1][s]{%
\setcounter{ketpicctra}{18}%
\if#1m \setcounter{ketpicctra}{1}\fi
\hypersetup{linkcolor=black}%
\begin{layer}{118}{0}
\putnotee{115}{-\theketpicctra.05}{\small\hiduke-\thepage/\pageref{pageend}}
\end{layer}\hypersetup{linkcolor=blue}
}
\newcounter{ban}
\setcounter{ban}{1}
\newcommand{\monban}[1][\hiduke]{%
#1-\theban\ %
\addtocounter{ban}{1}%
}
\newcommand{\monbannoadd}[1][\hiduke]{%
#1-\theban\ %
}
\newcommand{\addban}{%
\addtocounter{ban}{1}%
}
\newcounter{edawidth}
\newcounter{edactr}
\newcommand{\seteda}[1]{
\setcounter{edawidth}{#1}
\setcounter{edactr}{1}
}
\newcommand{\eda}[1]{%
\Ltab{\theedawidth mm}{[\theedactr]\ #1}%
\addtocounter{edactr}{1}%
}
%%%%%%%%%%%%

%%%%%%%%%%%%%%%%%%%%

\mainslide{ 平均変化率}


\slidepage[m]
%%%%%%%%%%%%%

%%%%%%%%%%%%%%%%%%%%

\newslide{平均変化率の意味}

\vspace*{18mm}


\begin{layer}{120}{0}
\putnotew{96}{73}{\hyperlink{para0pg0}{\fbox{\Ctab{2.5mm}{\scalebox{1}{\scriptsize $\mathstrut||\!\lhd$}}}}}
\putnotew{101}{73}{\hyperlink{para1pg1}{\fbox{\Ctab{2.5mm}{\scalebox{1}{\scriptsize $\mathstrut|\!\lhd$}}}}}
\putnotew{108}{73}{\hyperlink{para1pg6}{\fbox{\Ctab{4.5mm}{\scalebox{1}{\scriptsize $\mathstrut\!\!\lhd\!\!$}}}}}
\putnotew{115}{73}{\hyperlink{para1pg7}{\fbox{\Ctab{4.5mm}{\scalebox{1}{\scriptsize $\mathstrut\!\rhd\!$}}}}}
\putnotew{120}{73}{\hyperlink{para1pg7}{\fbox{\Ctab{2.5mm}{\scalebox{1}{\scriptsize $\mathstrut \!\rhd\!\!|$}}}}}
\putnotew{125}{73}{\hyperlink{para2pg1}{\fbox{\Ctab{2.5mm}{\scalebox{1}{\scriptsize $\mathstrut \!\rhd\!\!||$}}}}}
\putnotee{126}{73}{\scriptsize\color{blue} 7/7}
\end{layer}

\slidepage

\begin{layer}{120}{0}
\putnotese{80}{10}{\scalebox{0.7}{%%% /polytech.git/n106/fig/henka1.tex 
%%% Generator=henkaritu.cdy 
{\unitlength=1cm%
\begin{picture}%
(6,6)(-1,-1)%
\special{pn 8}%
%
\special{pn 12}%
\special{pa  -195  -265}\special{pa  -136  -264}\special{pa   -76  -266}\special{pa   -17  -271}%
\special{pa    42  -279}\special{pa   101  -290}\special{pa   159  -305}\special{pa   218  -322}%
\special{pa   277  -343}\special{pa   335  -367}\special{pa   394  -394}\special{pa   482  -437}%
\special{pa   572  -483}\special{pa   662  -531}\special{pa   753  -582}\special{pa   844  -637}%
\special{pa   934  -696}\special{pa  1024  -760}\special{pa  1111  -829}\special{pa  1197  -903}%
\special{pa  1280  -984}\special{pa  1333 -1043}\special{pa  1386 -1107}\special{pa  1437 -1176}%
\special{pa  1488 -1251}\special{pa  1537 -1330}\special{pa  1585 -1415}\special{pa  1632 -1506}%
\special{pa  1677 -1601}\special{pa  1722 -1702}\special{pa  1765 -1808}%
\special{fp}%
\special{pn 8}%
\special{pa 0 -394}\special{pa 37 -394}\special{fp}\special{pa 75 -394}\special{pa 112 -394}\special{fp}%
\special{pa 150 -394}\special{pa 187 -394}\special{fp}\special{pa 225 -394}\special{pa 262 -394}\special{fp}%
\special{pa 300 -394}\special{pa 337 -394}\special{fp}\special{pa 375 -394}\special{pa 394 -394}\special{pa 394 -375}\special{fp}%
\special{pa 394 -337}\special{pa 394 -300}\special{fp}\special{pa 394 -262}\special{pa 394 -225}\special{fp}%
\special{pa 394 -187}\special{pa 394 -150}\special{fp}\special{pa 394 -112}\special{pa 394 -75}\special{fp}%
\special{pa 394 -37}\special{pa 394 0}\special{fp}%
%
\special{pa 0 -984}\special{pa 38 -984}\special{fp}\special{pa 77 -984}\special{pa 115 -984}\special{fp}%
\special{pa 153 -984}\special{pa 192 -984}\special{fp}\special{pa 230 -984}\special{pa 269 -984}\special{fp}%
\special{pa 307 -984}\special{pa 345 -984}\special{fp}\special{pa 384 -984}\special{pa 422 -984}\special{fp}%
\special{pa 460 -984}\special{pa 499 -984}\special{fp}\special{pa 537 -984}\special{pa 576 -984}\special{fp}%
\special{pa 614 -984}\special{pa 652 -984}\special{fp}\special{pa 691 -984}\special{pa 729 -984}\special{fp}%
\special{pa 767 -984}\special{pa 806 -984}\special{fp}\special{pa 844 -984}\special{pa 882 -984}\special{fp}%
\special{pa 921 -984}\special{pa 959 -984}\special{fp}\special{pa 998 -984}\special{pa 1036 -984}\special{fp}%
\special{pa 1074 -984}\special{pa 1113 -984}\special{fp}\special{pa 1151 -984}\special{pa 1189 -984}\special{fp}%
\special{pa 1228 -984}\special{pa 1266 -984}\special{fp}\special{pa 1280 -959}\special{pa 1280 -921}\special{fp}%
\special{pa 1280 -882}\special{pa 1280 -844}\special{fp}\special{pa 1280 -806}\special{pa 1280 -767}\special{fp}%
\special{pa 1280 -729}\special{pa 1280 -691}\special{fp}\special{pa 1280 -652}\special{pa 1280 -614}\special{fp}%
\special{pa 1280 -576}\special{pa 1280 -537}\special{fp}\special{pa 1280 -499}\special{pa 1280 -460}\special{fp}%
\special{pa 1280 -422}\special{pa 1280 -384}\special{fp}\special{pa 1280 -345}\special{pa 1280 -307}\special{fp}%
\special{pa 1280 -269}\special{pa 1280 -230}\special{fp}\special{pa 1280 -192}\special{pa 1280 -153}\special{fp}%
\special{pa 1280 -115}\special{pa 1280 -77}\special{fp}\special{pa 1280 -38}\special{pa 1280 0}\special{fp}%
%
%
\special{pa   394   -20}\special{pa   394    20}%
\special{fp}%
\settowidth{\Width}{$a$}\setlength{\Width}{-0.5\Width}%
\settoheight{\Height}{$a$}\settodepth{\Depth}{$a$}\setlength{\Height}{-\Height}%
\put(1.0000000,-0.1000000){\hspace*{\Width}\raisebox{\Height}{$a$}}%
%
\special{pa  1280   -20}\special{pa  1280    20}%
\special{fp}%
\settowidth{\Width}{$b$}\setlength{\Width}{-0.5\Width}%
\settoheight{\Height}{$b$}\settodepth{\Depth}{$b$}\setlength{\Height}{-\Height}%
\put(3.2500000,-0.1000000){\hspace*{\Width}\raisebox{\Height}{$b$}}%
%
\special{pa    20  -394}\special{pa   -20  -394}%
\special{fp}%
\settowidth{\Width}{$f(a)$}\setlength{\Width}{-1\Width}%
\settoheight{\Height}{$f(a)$}\settodepth{\Depth}{$f(a)$}\setlength{\Height}{-0.5\Height}\setlength{\Depth}{0.5\Depth}\addtolength{\Height}{\Depth}%
\put(-0.1000000,1.0000000){\hspace*{\Width}\raisebox{\Height}{$f(a)$}}%
%
\special{pa    20  -984}\special{pa   -20  -984}%
\special{fp}%
\settowidth{\Width}{$f(b)$}\setlength{\Width}{-1\Width}%
\settoheight{\Height}{$f(b)$}\settodepth{\Depth}{$f(b)$}\setlength{\Height}{-0.5\Height}\setlength{\Depth}{0.5\Depth}\addtolength{\Height}{\Depth}%
\put(-0.1000000,2.5000000){\hspace*{\Width}\raisebox{\Height}{$f(b)$}}%
%
\settowidth{\Width}{$y=f(x)$}\setlength{\Width}{-0.5\Width}%
\settoheight{\Height}{$y=f(x)$}\settodepth{\Depth}{$y=f(x)$}\setlength{\Height}{\Depth}%
\put(4.4800000,4.6400000){\hspace*{\Width}\raisebox{\Height}{$y=f(x)$}}%
%
\special{pa  -394    -0}\special{pa  1969    -0}%
\special{fp}%
\special{pa     0   394}\special{pa     0 -1969}%
\special{fp}%
\settowidth{\Width}{$x$}\setlength{\Width}{0\Width}%
\settoheight{\Height}{$x$}\settodepth{\Depth}{$x$}\setlength{\Height}{-0.5\Height}\setlength{\Depth}{0.5\Depth}\addtolength{\Height}{\Depth}%
\put(5.0500000,0.0000000){\hspace*{\Width}\raisebox{\Height}{$x$}}%
%
\settowidth{\Width}{$y$}\setlength{\Width}{-0.5\Width}%
\settoheight{\Height}{$y$}\settodepth{\Depth}{$y$}\setlength{\Height}{\Depth}%
\put(0.0000000,5.0500000){\hspace*{\Width}\raisebox{\Height}{$y$}}%
%
\settowidth{\Width}{O}\setlength{\Width}{-1\Width}%
\settoheight{\Height}{O}\settodepth{\Depth}{O}\setlength{\Height}{-\Height}%
\put(-0.0500000,-0.0500000){\hspace*{\Width}\raisebox{\Height}{O}}%
%
\end{picture}}%}}
\putnotese{80}{10}{\scalebox{0.7}{%%% /polytech.git/n106/fig/henka2.tex 
%%% Generator=henkaritu.cdy 
{\unitlength=1cm%
\begin{picture}%
(6,6)(-1,-1)%
\special{pn 8}%
%
\special{pn 12}%
\special{pa  -195  -265}\special{pa  -136  -264}\special{pa   -76  -266}\special{pa   -17  -271}%
\special{pa    42  -279}\special{pa   101  -290}\special{pa   159  -305}\special{pa   218  -322}%
\special{pa   277  -343}\special{pa   335  -367}\special{pa   394  -394}\special{pa   482  -437}%
\special{pa   572  -483}\special{pa   662  -531}\special{pa   753  -582}\special{pa   844  -637}%
\special{pa   934  -696}\special{pa  1024  -760}\special{pa  1111  -829}\special{pa  1197  -903}%
\special{pa  1280  -984}\special{pa  1333 -1043}\special{pa  1386 -1107}\special{pa  1437 -1176}%
\special{pa  1488 -1251}\special{pa  1537 -1330}\special{pa  1585 -1415}\special{pa  1632 -1506}%
\special{pa  1677 -1601}\special{pa  1722 -1702}\special{pa  1765 -1808}%
\special{fp}%
\special{pn 8}%
\special{pa 0 -394}\special{pa 37 -394}\special{fp}\special{pa 75 -394}\special{pa 112 -394}\special{fp}%
\special{pa 150 -394}\special{pa 187 -394}\special{fp}\special{pa 225 -394}\special{pa 262 -394}\special{fp}%
\special{pa 300 -394}\special{pa 337 -394}\special{fp}\special{pa 375 -394}\special{pa 394 -394}\special{pa 394 -375}\special{fp}%
\special{pa 394 -337}\special{pa 394 -300}\special{fp}\special{pa 394 -262}\special{pa 394 -225}\special{fp}%
\special{pa 394 -187}\special{pa 394 -150}\special{fp}\special{pa 394 -112}\special{pa 394 -75}\special{fp}%
\special{pa 394 -37}\special{pa 394 0}\special{fp}%
%
\special{pa 0 -984}\special{pa 38 -984}\special{fp}\special{pa 77 -984}\special{pa 115 -984}\special{fp}%
\special{pa 153 -984}\special{pa 192 -984}\special{fp}\special{pa 230 -984}\special{pa 269 -984}\special{fp}%
\special{pa 307 -984}\special{pa 345 -984}\special{fp}\special{pa 384 -984}\special{pa 422 -984}\special{fp}%
\special{pa 460 -984}\special{pa 499 -984}\special{fp}\special{pa 537 -984}\special{pa 576 -984}\special{fp}%
\special{pa 614 -984}\special{pa 652 -984}\special{fp}\special{pa 691 -984}\special{pa 729 -984}\special{fp}%
\special{pa 767 -984}\special{pa 806 -984}\special{fp}\special{pa 844 -984}\special{pa 882 -984}\special{fp}%
\special{pa 921 -984}\special{pa 959 -984}\special{fp}\special{pa 998 -984}\special{pa 1036 -984}\special{fp}%
\special{pa 1074 -984}\special{pa 1113 -984}\special{fp}\special{pa 1151 -984}\special{pa 1189 -984}\special{fp}%
\special{pa 1228 -984}\special{pa 1266 -984}\special{fp}\special{pa 1280 -959}\special{pa 1280 -921}\special{fp}%
\special{pa 1280 -882}\special{pa 1280 -844}\special{fp}\special{pa 1280 -806}\special{pa 1280 -767}\special{fp}%
\special{pa 1280 -729}\special{pa 1280 -691}\special{fp}\special{pa 1280 -652}\special{pa 1280 -614}\special{fp}%
\special{pa 1280 -576}\special{pa 1280 -537}\special{fp}\special{pa 1280 -499}\special{pa 1280 -460}\special{fp}%
\special{pa 1280 -422}\special{pa 1280 -384}\special{fp}\special{pa 1280 -345}\special{pa 1280 -307}\special{fp}%
\special{pa 1280 -269}\special{pa 1280 -230}\special{fp}\special{pa 1280 -192}\special{pa 1280 -153}\special{fp}%
\special{pa 1280 -115}\special{pa 1280 -77}\special{fp}\special{pa 1280 -38}\special{pa 1280 0}\special{fp}%
%
%
\special{pa   394   -20}\special{pa   394    20}%
\special{fp}%
\settowidth{\Width}{$a$}\setlength{\Width}{-0.5\Width}%
\settoheight{\Height}{$a$}\settodepth{\Depth}{$a$}\setlength{\Height}{-\Height}%
\put(1.0000000,-0.1000000){\hspace*{\Width}\raisebox{\Height}{$a$}}%
%
\special{pa  1280   -20}\special{pa  1280    20}%
\special{fp}%
\settowidth{\Width}{$b$}\setlength{\Width}{-0.5\Width}%
\settoheight{\Height}{$b$}\settodepth{\Depth}{$b$}\setlength{\Height}{-\Height}%
\put(3.2500000,-0.1000000){\hspace*{\Width}\raisebox{\Height}{$b$}}%
%
\special{pa    20  -394}\special{pa   -20  -394}%
\special{fp}%
\settowidth{\Width}{$f(a)$}\setlength{\Width}{-1\Width}%
\settoheight{\Height}{$f(a)$}\settodepth{\Depth}{$f(a)$}\setlength{\Height}{-0.5\Height}\setlength{\Depth}{0.5\Depth}\addtolength{\Height}{\Depth}%
\put(-0.1000000,1.0000000){\hspace*{\Width}\raisebox{\Height}{$f(a)$}}%
%
\special{pa    20  -984}\special{pa   -20  -984}%
\special{fp}%
\settowidth{\Width}{$f(b)$}\setlength{\Width}{-1\Width}%
\settoheight{\Height}{$f(b)$}\settodepth{\Depth}{$f(b)$}\setlength{\Height}{-0.5\Height}\setlength{\Depth}{0.5\Depth}\addtolength{\Height}{\Depth}%
\put(-0.1000000,2.5000000){\hspace*{\Width}\raisebox{\Height}{$f(b)$}}%
%
\settowidth{\Width}{$y=f(x)$}\setlength{\Width}{-0.5\Width}%
\settoheight{\Height}{$y=f(x)$}\settodepth{\Depth}{$y=f(x)$}\setlength{\Height}{\Depth}%
\put(4.4800000,4.6400000){\hspace*{\Width}\raisebox{\Height}{$y=f(x)$}}%
%
{%
\color[cmyk]{0,1,1,0}%
\special{pa   394  -394}\special{pa  1280  -984}%
\special{fp}%
}%
\special{pa 394 -394}\special{pa 432 -394}\special{fp}\special{pa 471 -394}\special{pa 509 -394}\special{fp}%
\special{pa 548 -394}\special{pa 586 -394}\special{fp}\special{pa 625 -394}\special{pa 663 -394}\special{fp}%
\special{pa 702 -394}\special{pa 740 -394}\special{fp}\special{pa 779 -394}\special{pa 817 -394}\special{fp}%
\special{pa 856 -394}\special{pa 894 -394}\special{fp}\special{pa 933 -394}\special{pa 971 -394}\special{fp}%
\special{pa 1010 -394}\special{pa 1048 -394}\special{fp}\special{pa 1087 -394}\special{pa 1125 -394}\special{fp}%
\special{pa 1164 -394}\special{pa 1202 -394}\special{fp}\special{pa 1241 -394}\special{pa 1280 -394}\special{fp}%
%
%
\special{pa   394  -394}\special{pa   411  -387}\special{pa   427  -380}\special{pa   445  -374}%
\special{pa   462  -368}\special{pa   479  -362}\special{pa   496  -357}\special{pa   514  -351}%
\special{pa   531  -346}\special{pa   549  -342}\special{pa   566  -337}\special{pa   584  -333}%
\special{pa   602  -329}\special{pa   620  -326}\special{pa   638  -322}\special{pa   656  -319}%
\special{pa   673  -317}\special{pa   692  -314}\special{pa   710  -312}\special{pa   728  -310}%
\special{pa   746  -309}\special{pa   764  -307}\special{pa   782  -306}\special{pa   800  -306}%
\special{pa   818  -305}\special{pa   837  -305}\special{pa   855  -305}\special{pa   873  -306}%
\special{pa   891  -306}\special{pa   909  -307}\special{pa   927  -309}\special{pa   946  -310}%
\special{pa   964  -312}\special{pa   982  -314}\special{pa  1000  -317}\special{pa  1018  -319}%
\special{pa  1036  -322}\special{pa  1054  -326}\special{pa  1071  -329}\special{pa  1089  -333}%
\special{pa  1107  -337}\special{pa  1124  -342}\special{pa  1142  -346}\special{pa  1160  -351}%
\special{pa  1177  -357}\special{pa  1194  -362}\special{pa  1212  -368}\special{pa  1229  -374}%
\special{pa  1246  -380}\special{pa  1263  -387}\special{pa  1280  -394}%
\special{fp}%
\settowidth{\Width}{$b-a$}\setlength{\Width}{-0.5\Width}%
\settoheight{\Height}{$b-a$}\settodepth{\Depth}{$b-a$}\setlength{\Height}{-0.5\Height}\setlength{\Depth}{0.5\Depth}\addtolength{\Height}{\Depth}%
\put(2.1200000,0.6700000){\hspace*{\Width}\raisebox{\Height}{$b-a$}}%
%
\special{pa  1280  -394}\special{pa  1284  -405}\special{pa  1288  -416}\special{pa  1293  -428}%
\special{pa  1297  -439}\special{pa  1301  -451}\special{pa  1304  -462}\special{pa  1308  -474}%
\special{pa  1311  -485}\special{pa  1314  -497}\special{pa  1317  -509}\special{pa  1320  -521}%
\special{pa  1322  -532}\special{pa  1325  -544}\special{pa  1327  -556}\special{pa  1329  -568}%
\special{pa  1331  -580}\special{pa  1332  -592}\special{pa  1334  -604}\special{pa  1335  -616}%
\special{pa  1336  -628}\special{pa  1337  -641}\special{pa  1338  -653}\special{pa  1338  -665}%
\special{pa  1338  -677}\special{pa  1339  -689}\special{pa  1338  -701}\special{pa  1338  -713}%
\special{pa  1338  -725}\special{pa  1337  -737}\special{pa  1336  -750}\special{pa  1335  -762}%
\special{pa  1334  -774}\special{pa  1332  -786}\special{pa  1331  -798}\special{pa  1329  -810}%
\special{pa  1327  -822}\special{pa  1325  -834}\special{pa  1322  -845}\special{pa  1320  -857}%
\special{pa  1317  -869}\special{pa  1314  -881}\special{pa  1311  -893}\special{pa  1308  -904}%
\special{pa  1304  -916}\special{pa  1301  -927}\special{pa  1297  -939}\special{pa  1293  -950}%
\special{pa  1288  -962}\special{pa  1284  -973}\special{pa  1280  -984}%
\special{fp}%
\settowidth{\Width}{$f(b)-f(a)$}\setlength{\Width}{0\Width}%
\settoheight{\Height}{$f(b)-f(a)$}\settodepth{\Depth}{$f(b)-f(a)$}\setlength{\Height}{-0.5\Height}\setlength{\Depth}{0.5\Depth}\addtolength{\Height}{\Depth}%
\put(3.5000000,1.7500000){\hspace*{\Width}\raisebox{\Height}{$f(b)-f(a)$}}%
%
\settowidth{\Width}{A}\setlength{\Width}{-1\Width}%
\settoheight{\Height}{A}\settodepth{\Depth}{A}\setlength{\Height}{\Depth}%
\put(0.9500000,1.0500000){\hspace*{\Width}\raisebox{\Height}{A}}%
%
\settowidth{\Width}{B}\setlength{\Width}{-1\Width}%
\settoheight{\Height}{B}\settodepth{\Depth}{B}\setlength{\Height}{\Depth}%
\put(3.2000000,2.5500000){\hspace*{\Width}\raisebox{\Height}{B}}%
%
\special{pa  -394    -0}\special{pa  1969    -0}%
\special{fp}%
\special{pa     0   394}\special{pa     0 -1969}%
\special{fp}%
\settowidth{\Width}{$x$}\setlength{\Width}{0\Width}%
\settoheight{\Height}{$x$}\settodepth{\Depth}{$x$}\setlength{\Height}{-0.5\Height}\setlength{\Depth}{0.5\Depth}\addtolength{\Height}{\Depth}%
\put(5.0500000,0.0000000){\hspace*{\Width}\raisebox{\Height}{$x$}}%
%
\settowidth{\Width}{$y$}\setlength{\Width}{-0.5\Width}%
\settoheight{\Height}{$y$}\settodepth{\Depth}{$y$}\setlength{\Height}{\Depth}%
\put(0.0000000,5.0500000){\hspace*{\Width}\raisebox{\Height}{$y$}}%
%
\settowidth{\Width}{O}\setlength{\Width}{-1\Width}%
\settoheight{\Height}{O}\settodepth{\Depth}{O}\setlength{\Height}{-\Height}%
\put(-0.0500000,-0.0500000){\hspace*{\Width}\raisebox{\Height}{O}}%
%
\end{picture}}%}}
\end{layer}

\begin{itemize}
\item
関数$y=f(x)$,\ 区間$[a,\ b]$
\item
$f(x)$の$[a,\ b]$での変化量は\\
\hspace*{2zw}$f(b)-f(a)$
\item
[]区間幅$b-a$で割る\\
\hspace*{2zw}$\bunsuu{f(b)-f(a)}{b-a}$
\item
[]これを{\color{red}平均変化率}という
\item
平均変化率は直線ABの傾き
\end{itemize}

\newslide{平均変化率の計算例}

\vspace*{18mm}


\begin{layer}{120}{0}
\putnotew{96}{73}{\hyperlink{para1pg7}{\fbox{\Ctab{2.5mm}{\scalebox{1}{\scriptsize $\mathstrut||\!\lhd$}}}}}
\putnotew{101}{73}{\hyperlink{para2pg1}{\fbox{\Ctab{2.5mm}{\scalebox{1}{\scriptsize $\mathstrut|\!\lhd$}}}}}
\putnotew{108}{73}{\hyperlink{para2pg8}{\fbox{\Ctab{4.5mm}{\scalebox{1}{\scriptsize $\mathstrut\!\!\lhd\!\!$}}}}}
\putnotew{115}{73}{\hyperlink{para2pg9}{\fbox{\Ctab{4.5mm}{\scalebox{1}{\scriptsize $\mathstrut\!\rhd\!$}}}}}
\putnotew{120}{73}{\hyperlink{para2pg9}{\fbox{\Ctab{2.5mm}{\scalebox{1}{\scriptsize $\mathstrut \!\rhd\!\!|$}}}}}
\putnotew{125}{73}{\hyperlink{para3pg1}{\fbox{\Ctab{2.5mm}{\scalebox{1}{\scriptsize $\mathstrut \!\rhd\!\!||$}}}}}
\putnotee{126}{73}{\scriptsize\color{blue} 9/9}
\end{layer}

\slidepage
\begin{itemize}
\item
$f(x)=x^2$の$[1,3]$での平均変化率($r$とおく)\\
\hspace*{1zw}$r=\bunsuu{f(3)-f(1)}{3-1}=$
$\bunsuu{3^2-1^2}{3-1}=\bunsuu{9-1}{3-1}=4$
\item
$f(x)=x^2$の$[a,b]$での平均変化率\\
\hspace*{1zw}$r=\bunsuu{b^2-a^2}{b-a}=$
$\bunsuu{(b-a)(b+a)}{b-a}=b+a$
\end{itemize}

\newslide{$b$を$a$に近づけたときの変化率}

\vspace*{18mm}


\begin{layer}{120}{0}
\putnotew{96}{73}{\hyperlink{para2pg9}{\fbox{\Ctab{2.5mm}{\scalebox{1}{\scriptsize $\mathstrut||\!\lhd$}}}}}
\putnotew{101}{73}{\hyperlink{para3pg1}{\fbox{\Ctab{2.5mm}{\scalebox{1}{\scriptsize $\mathstrut|\!\lhd$}}}}}
\putnotew{108}{73}{\hyperlink{para3pg8}{\fbox{\Ctab{4.5mm}{\scalebox{1}{\scriptsize $\mathstrut\!\!\lhd\!\!$}}}}}
\putnotew{115}{73}{\hyperlink{para3pg9}{\fbox{\Ctab{4.5mm}{\scalebox{1}{\scriptsize $\mathstrut\!\rhd\!$}}}}}
\putnotew{120}{73}{\hyperlink{para3pg9}{\fbox{\Ctab{2.5mm}{\scalebox{1}{\scriptsize $\mathstrut \!\rhd\!\!|$}}}}}
\putnotew{125}{73}{\hyperlink{para4pg1}{\fbox{\Ctab{2.5mm}{\scalebox{1}{\scriptsize $\mathstrut \!\rhd\!\!||$}}}}}
\putnotee{126}{73}{\scriptsize\color{blue} 9/9}
\end{layer}

\slidepage

\begin{layer}{120}{0}
\putnotese{83}{11}{\scalebox{0.8}{%%% /polytech.git/n106/fig/henka3.tex 
%%% Generator=henkaritu.cdy 
{\unitlength=1cm%
\begin{picture}%
(6,6)(-1,-1)%
\special{pn 8}%
%
\special{pn 12}%
\special{pa  -394  -394}\special{pa  -346  -305}\special{pa  -299  -227}\special{pa  -252  -161}%
\special{pa  -205  -106}\special{pa  -157   -63}\special{pa  -110   -31}\special{pa   -63   -10}%
\special{pa   -16    -1}\special{pa    31    -3}\special{pa    79   -16}\special{pa   126   -40}%
\special{pa   173   -76}\special{pa   220  -123}\special{pa   268  -182}\special{pa   315  -252}%
\special{pa   362  -333}\special{pa   409  -426}\special{pa   457  -530}\special{pa   504  -645}%
\special{pa   551  -772}\special{pa   598  -910}\special{pa   646 -1059}\special{pa   693 -1220}%
\special{pa   740 -1391}\special{pa   787 -1575}\special{pa   835 -1769}\special{pa   880 -1969}%
\special{fp}%
\special{pn 8}%
{%
\color[cmyk]{0,1,1,0}%
\special{pa   394  -394}\special{pa   787 -1575}%
\special{fp}%
}%
\special{pn 8}%
\special{pa 394 -398}\special{pa 394 -390}\special{fp}\special{pa 394 -358}\special{pa 394 -350}\special{fp}%
\special{pa 394 -319}\special{pa 394 -311}\special{fp}\special{pa 394 -280}\special{pa 394 -272}\special{fp}%
\special{pa 394 -240}\special{pa 394 -232}\special{fp}\special{pa 394 -201}\special{pa 394 -193}\special{fp}%
\special{pa 394 -161}\special{pa 394 -153}\special{fp}\special{pa 394 -122}\special{pa 394 -114}\special{fp}%
\special{pa 394 -83}\special{pa 394 -75}\special{fp}\special{pa 394 -43}\special{pa 394 -35}\special{fp}%
\special{pa 394 -4}\special{pa 394 4}\special{fp}\special{pn 8}%
\special{pn 8}%
\special{pa 787 -1579}\special{pa 787 -1571}\special{fp}\special{pa 787 -1539}\special{pa 787 -1531}\special{fp}%
\special{pa 787 -1500}\special{pa 787 -1492}\special{fp}\special{pa 787 -1461}\special{pa 787 -1453}\special{fp}%
\special{pa 787 -1421}\special{pa 787 -1413}\special{fp}\special{pa 787 -1382}\special{pa 787 -1374}\special{fp}%
\special{pa 787 -1343}\special{pa 787 -1335}\special{fp}\special{pa 787 -1303}\special{pa 787 -1295}\special{fp}%
\special{pa 787 -1264}\special{pa 787 -1256}\special{fp}\special{pa 787 -1224}\special{pa 787 -1216}\special{fp}%
\special{pa 787 -1185}\special{pa 787 -1177}\special{fp}\special{pa 787 -1146}\special{pa 787 -1138}\special{fp}%
\special{pa 787 -1106}\special{pa 787 -1098}\special{fp}\special{pa 787 -1067}\special{pa 787 -1059}\special{fp}%
\special{pa 787 -1028}\special{pa 787 -1020}\special{fp}\special{pa 787 -988}\special{pa 787 -980}\special{fp}%
\special{pa 787 -949}\special{pa 787 -941}\special{fp}\special{pa 787 -910}\special{pa 787 -902}\special{fp}%
\special{pa 787 -870}\special{pa 787 -862}\special{fp}\special{pa 787 -831}\special{pa 787 -823}\special{fp}%
\special{pa 787 -791}\special{pa 787 -783}\special{fp}\special{pa 787 -752}\special{pa 787 -744}\special{fp}%
\special{pa 787 -713}\special{pa 787 -705}\special{fp}\special{pa 787 -673}\special{pa 787 -665}\special{fp}%
\special{pa 787 -634}\special{pa 787 -626}\special{fp}\special{pa 787 -595}\special{pa 787 -587}\special{fp}%
\special{pa 787 -555}\special{pa 787 -547}\special{fp}\special{pa 787 -516}\special{pa 787 -508}\special{fp}%
\special{pa 787 -476}\special{pa 787 -468}\special{fp}\special{pa 787 -437}\special{pa 787 -429}\special{fp}%
\special{pa 787 -398}\special{pa 787 -390}\special{fp}\special{pa 787 -358}\special{pa 787 -350}\special{fp}%
\special{pa 787 -319}\special{pa 787 -311}\special{fp}\special{pa 787 -280}\special{pa 787 -272}\special{fp}%
\special{pa 787 -240}\special{pa 787 -232}\special{fp}\special{pa 787 -201}\special{pa 787 -193}\special{fp}%
\special{pa 787 -161}\special{pa 787 -153}\special{fp}\special{pa 787 -122}\special{pa 787 -114}\special{fp}%
\special{pa 787 -83}\special{pa 787 -75}\special{fp}\special{pa 787 -43}\special{pa 787 -35}\special{fp}%
\special{pa 787 -4}\special{pa 787 4}\special{fp}\special{pn 8}%
\special{pn 8}%
\special{pa 390 -394}\special{pa 398 -394}\special{fp}\special{pa 429 -394}\special{pa 437 -394}\special{fp}%
\special{pa 468 -394}\special{pa 476 -394}\special{fp}\special{pa 508 -394}\special{pa 516 -394}\special{fp}%
\special{pa 547 -394}\special{pa 555 -394}\special{fp}\special{pa 587 -394}\special{pa 595 -394}\special{fp}%
\special{pa 626 -394}\special{pa 634 -394}\special{fp}\special{pa 665 -394}\special{pa 673 -394}\special{fp}%
\special{pa 705 -394}\special{pa 713 -394}\special{fp}\special{pa 744 -394}\special{pa 752 -394}\special{fp}%
\special{pa 783 -394}\special{pa 791 -394}\special{fp}\special{pn 8}%
\special{pa   394   -20}\special{pa   394    20}%
\special{fp}%
\settowidth{\Width}{$1$}\setlength{\Width}{-0.5\Width}%
\settoheight{\Height}{$1$}\settodepth{\Depth}{$1$}\setlength{\Height}{-\Height}%
\put(1.0000000,-0.1000000){\hspace*{\Width}\raisebox{\Height}{$1$}}%
%
\special{pa   787   -20}\special{pa   787    20}%
\special{fp}%
\settowidth{\Width}{$2$}\setlength{\Width}{-0.5\Width}%
\settoheight{\Height}{$2$}\settodepth{\Depth}{$2$}\setlength{\Height}{-\Height}%
\put(2.0000000,-0.1000000){\hspace*{\Width}\raisebox{\Height}{$2$}}%
%
%
\settowidth{\Width}{$y=x^2$}\setlength{\Width}{0\Width}%
\settoheight{\Height}{$y=x^2$}\settodepth{\Depth}{$y=x^2$}\setlength{\Height}{-0.5\Height}\setlength{\Depth}{0.5\Depth}\addtolength{\Height}{\Depth}%
\put(2.2900000,5.0000000){\hspace*{\Width}\raisebox{\Height}{$y=x^2$}}%
%
\special{pa  -394    -0}\special{pa  1969    -0}%
\special{fp}%
\special{pa     0   394}\special{pa     0 -1969}%
\special{fp}%
\settowidth{\Width}{$x$}\setlength{\Width}{0\Width}%
\settoheight{\Height}{$x$}\settodepth{\Depth}{$x$}\setlength{\Height}{-0.5\Height}\setlength{\Depth}{0.5\Depth}\addtolength{\Height}{\Depth}%
\put(5.0500000,0.0000000){\hspace*{\Width}\raisebox{\Height}{$x$}}%
%
\settowidth{\Width}{$y$}\setlength{\Width}{-0.5\Width}%
\settoheight{\Height}{$y$}\settodepth{\Depth}{$y$}\setlength{\Height}{\Depth}%
\put(0.0000000,5.0500000){\hspace*{\Width}\raisebox{\Height}{$y$}}%
%
\settowidth{\Width}{O}\setlength{\Width}{-1\Width}%
\settoheight{\Height}{O}\settodepth{\Depth}{O}\setlength{\Height}{-\Height}%
\put(-0.0500000,-0.0500000){\hspace*{\Width}\raisebox{\Height}{O}}%
%
\end{picture}}%}}
\end{layer}

\begin{itemize}
\item
関数$y=x^2$の$[a,b]$での平均変化率\\
\hspace*{1zw}$r=\bunsuu{b^2-a^2}{b-a}$
\item
[]$[1,b]$のとき\\
\hspace*{1zw}$r=\bunsuu{b^2-1}{b-1}$
\item
$b=2$のとき $r=\hakoa{3}$
\item
[課題]\monban \href{https://s-takato.github.io/polytech22/offlineapp/henka3mainoff.html}%
{1点における変化率}を動かして答えよ\\
・$b=1$における$r$はどうなるか
\end{itemize}

\newslide{割り算(分数)の意味}

\vspace*{18mm}


\begin{layer}{120}{0}
\putnotew{96}{73}{\hyperlink{para3pg9}{\fbox{\Ctab{2.5mm}{\scalebox{1}{\scriptsize $\mathstrut||\!\lhd$}}}}}
\putnotew{101}{73}{\hyperlink{para4pg1}{\fbox{\Ctab{2.5mm}{\scalebox{1}{\scriptsize $\mathstrut|\!\lhd$}}}}}
\putnotew{108}{73}{\hyperlink{para4pg5}{\fbox{\Ctab{4.5mm}{\scalebox{1}{\scriptsize $\mathstrut\!\!\lhd\!\!$}}}}}
\putnotew{115}{73}{\hyperlink{para4pg6}{\fbox{\Ctab{4.5mm}{\scalebox{1}{\scriptsize $\mathstrut\!\rhd\!$}}}}}
\putnotew{120}{73}{\hyperlink{para4pg6}{\fbox{\Ctab{2.5mm}{\scalebox{1}{\scriptsize $\mathstrut \!\rhd\!\!|$}}}}}
\putnotew{125}{73}{\hyperlink{para5pg1}{\fbox{\Ctab{2.5mm}{\scalebox{1}{\scriptsize $\mathstrut \!\rhd\!\!||$}}}}}
\putnotee{126}{73}{\scriptsize\color{blue} 6/6}
\end{layer}

\slidepage
\begin{itemize}
\item
$a \div b\ \ \Bigl(\bunsuu{a}{b}\Bigr)$とは
\item
[]例)$x=\bunsuu{6}{2}$
$\Longleftrightarrow\ 2x=6$となる$x$のこと
\item
[]例)$x=\bunsuu{3}{5}$
$\Longleftrightarrow\ 5x=3$となる$x$のこと
\item
{\color{red}$x=\bunsuu{a}{b}\ \Longleftrightarrow\ bx=a$となる$x$のこと}
\end{itemize}

\newslide{分母が$0$になると?}

\vspace*{18mm}


\begin{layer}{120}{0}
\putnotew{96}{73}{\hyperlink{para4pg6}{\fbox{\Ctab{2.5mm}{\scalebox{1}{\scriptsize $\mathstrut||\!\lhd$}}}}}
\putnotew{101}{73}{\hyperlink{para5pg1}{\fbox{\Ctab{2.5mm}{\scalebox{1}{\scriptsize $\mathstrut|\!\lhd$}}}}}
\putnotew{108}{73}{\hyperlink{para5pg7}{\fbox{\Ctab{4.5mm}{\scalebox{1}{\scriptsize $\mathstrut\!\!\lhd\!\!$}}}}}
\putnotew{115}{73}{\hyperlink{para5pg8}{\fbox{\Ctab{4.5mm}{\scalebox{1}{\scriptsize $\mathstrut\!\rhd\!$}}}}}
\putnotew{120}{73}{\hyperlink{para5pg8}{\fbox{\Ctab{2.5mm}{\scalebox{1}{\scriptsize $\mathstrut \!\rhd\!\!|$}}}}}
\putnotew{125}{73}{\hyperlink{para6pg1}{\fbox{\Ctab{2.5mm}{\scalebox{1}{\scriptsize $\mathstrut \!\rhd\!\!||$}}}}}
\putnotee{126}{73}{\scriptsize\color{blue} 8/8}
\end{layer}

\slidepage
\begin{itemize}
\item
[(1)]$\bunsuu{1}{0}$は\;\hakoa{求まらない}
\item
[(2)]$\bunsuu{0}{0}$は\;\hakoa{決まらない}
{\normalsize\color{blue}
\item
[]$x=\bunsuu{1}{0}\ \Longleftrightarrow\  \hakoa{0}\;x= \hakoa{1}$
\item
[]$x=\bunsuu{0}{0}\ \Longleftrightarrow\  \hakoa{0}\;x= \hakoa{0}$
}
\item
{\color{red}分母が0となる分数は考えない}
\end{itemize}

\newslide{1点における変化率}

\vspace*{18mm}


\begin{layer}{120}{0}
\putnotew{96}{73}{\hyperlink{para5pg8}{\fbox{\Ctab{2.5mm}{\scalebox{1}{\scriptsize $\mathstrut||\!\lhd$}}}}}
\putnotew{101}{73}{\hyperlink{para6pg1}{\fbox{\Ctab{2.5mm}{\scalebox{1}{\scriptsize $\mathstrut|\!\lhd$}}}}}
\putnotew{108}{73}{\hyperlink{para6pg2}{\fbox{\Ctab{4.5mm}{\scalebox{1}{\scriptsize $\mathstrut\!\!\lhd\!\!$}}}}}
\putnotew{115}{73}{\hyperlink{para6pg3}{\fbox{\Ctab{4.5mm}{\scalebox{1}{\scriptsize $\mathstrut\!\rhd\!$}}}}}
\putnotew{120}{73}{\hyperlink{para6pg3}{\fbox{\Ctab{2.5mm}{\scalebox{1}{\scriptsize $\mathstrut \!\rhd\!\!|$}}}}}
\putnotew{125}{73}{\hyperlink{para7pg1}{\fbox{\Ctab{2.5mm}{\scalebox{1}{\scriptsize $\mathstrut \!\rhd\!\!||$}}}}}
\putnotee{126}{73}{\scriptsize\color{blue} 3/3}
\end{layer}

\slidepage
\begin{itemize}
\item
区間$[a,b]$の平均変化率$r=\bunsuu{f(b)-f(a)}{b-a}$
\item
1点$a$における変化率$r=\bunsuu{f(a)-f(a)}{a-a}$\\
\hspace*{3zw}{\color{red}分母が0になってしまう}
\item
1点における変化率はどうやって求めればいいか
\end{itemize}

\mainslide{微分係数}


\slidepage[m]
%%%%%%%%%%%%%

%%%%%%%%%%%%%%%%%%%%

\newslide{関数の極限}

\vspace*{18mm}


\begin{layer}{120}{0}
\putnotew{96}{73}{\hyperlink{para6pg3}{\fbox{\Ctab{2.5mm}{\scalebox{1}{\scriptsize $\mathstrut||\!\lhd$}}}}}
\putnotew{101}{73}{\hyperlink{para7pg1}{\fbox{\Ctab{2.5mm}{\scalebox{1}{\scriptsize $\mathstrut|\!\lhd$}}}}}
\putnotew{108}{73}{\hyperlink{para7pg5}{\fbox{\Ctab{4.5mm}{\scalebox{1}{\scriptsize $\mathstrut\!\!\lhd\!\!$}}}}}
\putnotew{115}{73}{\hyperlink{para7pg6}{\fbox{\Ctab{4.5mm}{\scalebox{1}{\scriptsize $\mathstrut\!\rhd\!$}}}}}
\putnotew{120}{73}{\hyperlink{para7pg6}{\fbox{\Ctab{2.5mm}{\scalebox{1}{\scriptsize $\mathstrut \!\rhd\!\!|$}}}}}
\putnotew{125}{73}{\hyperlink{para8pg1}{\fbox{\Ctab{2.5mm}{\scalebox{1}{\scriptsize $\mathstrut \!\rhd\!\!||$}}}}}
\putnotee{126}{73}{\scriptsize\color{blue} 6/6}
\end{layer}

\slidepage
\begin{itemize}
\item
$x$が$a$に\underline{限りなく近づく}とする({\color{red}$x\to a$})\\
\hspace*{2zw}{\color{blue}$a$に等しくはないが,いくらでも近くなること}\vspace{-1mm}
\item
$f(x)$が$\alpha$に近づくとき,$\alpha$を{\color{red}極限値}という\\
\hspace*{2zw}{\color{blue}$\dlim_{x \to a}f(x)=\alpha$}と書く\vspace{-1mm}
\item
[例]$\dlim_{x \to 1}(2x+3)=5$
\end{itemize}

\newslide{$y=\bunsuu{x^2-4}{x-2}$のグラフ}

\vspace*{18mm}


\begin{layer}{120}{0}
\putnotew{96}{73}{\hyperlink{para7pg6}{\fbox{\Ctab{2.5mm}{\scalebox{1}{\scriptsize $\mathstrut||\!\lhd$}}}}}
\putnotew{101}{73}{\hyperlink{para8pg1}{\fbox{\Ctab{2.5mm}{\scalebox{1}{\scriptsize $\mathstrut|\!\lhd$}}}}}
\putnotew{108}{73}{\hyperlink{para8pg4}{\fbox{\Ctab{4.5mm}{\scalebox{1}{\scriptsize $\mathstrut\!\!\lhd\!\!$}}}}}
\putnotew{115}{73}{\hyperlink{para8pg5}{\fbox{\Ctab{4.5mm}{\scalebox{1}{\scriptsize $\mathstrut\!\rhd\!$}}}}}
\putnotew{120}{73}{\hyperlink{para8pg5}{\fbox{\Ctab{2.5mm}{\scalebox{1}{\scriptsize $\mathstrut \!\rhd\!\!|$}}}}}
\putnotew{125}{73}{\hyperlink{para9pg1}{\fbox{\Ctab{2.5mm}{\scalebox{1}{\scriptsize $\mathstrut \!\rhd\!\!||$}}}}}
\putnotee{126}{73}{\scriptsize\color{blue} 5/5}
\end{layer}

\slidepage

\begin{layer}{120}{0}
\putnotese{85}{25}{%%% /Users/takatoosetsuo/Dropbox/2021polytech/107/fig/graphfr2.tex 
%%% Generator=graphfr.cdy 
{\unitlength=5mm%
\begin{picture}%
(8,8)(-4,-2)%
\special{pn 8}%
%
\normalsize%
\special{pa  -787   -20}\special{pa  -787    20}%
\special{fp}%
\settowidth{\Width}{$-4$}\setlength{\Width}{-0.5\Width}%
\settoheight{\Height}{$-4$}\settodepth{\Depth}{$-4$}\setlength{\Height}{-\Height}%
\put(-4.0000000,-0.2000000){\hspace*{\Width}\raisebox{\Height}{$-4$}}%
%
\special{pa  -591   -20}\special{pa  -591    20}%
\special{fp}%
\settowidth{\Width}{$-3$}\setlength{\Width}{-0.5\Width}%
\settoheight{\Height}{$-3$}\settodepth{\Depth}{$-3$}\setlength{\Height}{-\Height}%
\put(-3.0000000,-0.2000000){\hspace*{\Width}\raisebox{\Height}{$-3$}}%
%
\special{pa  -394   -20}\special{pa  -394    20}%
\special{fp}%
\settowidth{\Width}{$-2$}\setlength{\Width}{-0.5\Width}%
\settoheight{\Height}{$-2$}\settodepth{\Depth}{$-2$}\setlength{\Height}{-\Height}%
\put(-2.0000000,-0.2000000){\hspace*{\Width}\raisebox{\Height}{$-2$}}%
%
\special{pa  -197   -20}\special{pa  -197    20}%
\special{fp}%
\settowidth{\Width}{$-1$}\setlength{\Width}{-0.5\Width}%
\settoheight{\Height}{$-1$}\settodepth{\Depth}{$-1$}\setlength{\Height}{-\Height}%
\put(-1.0000000,-0.2000000){\hspace*{\Width}\raisebox{\Height}{$-1$}}%
%
\special{pa   197   -20}\special{pa   197    20}%
\special{fp}%
\settowidth{\Width}{$1$}\setlength{\Width}{-0.5\Width}%
\settoheight{\Height}{$1$}\settodepth{\Depth}{$1$}\setlength{\Height}{-\Height}%
\put(1.0000000,-0.2000000){\hspace*{\Width}\raisebox{\Height}{$1$}}%
%
\special{pa   394   -20}\special{pa   394    20}%
\special{fp}%
\settowidth{\Width}{$2$}\setlength{\Width}{-0.5\Width}%
\settoheight{\Height}{$2$}\settodepth{\Depth}{$2$}\setlength{\Height}{-\Height}%
\put(2.0000000,-0.2000000){\hspace*{\Width}\raisebox{\Height}{$2$}}%
%
\special{pa   591   -20}\special{pa   591    20}%
\special{fp}%
\settowidth{\Width}{$3$}\setlength{\Width}{-0.5\Width}%
\settoheight{\Height}{$3$}\settodepth{\Depth}{$3$}\setlength{\Height}{-\Height}%
\put(3.0000000,-0.2000000){\hspace*{\Width}\raisebox{\Height}{$3$}}%
%
\special{pa    20   394}\special{pa   -20   394}%
\special{fp}%
\settowidth{\Width}{$-2$}\setlength{\Width}{-1\Width}%
\settoheight{\Height}{$-2$}\settodepth{\Depth}{$-2$}\setlength{\Height}{-0.5\Height}\setlength{\Depth}{0.5\Depth}\addtolength{\Height}{\Depth}%
\put(-0.2000000,-2.0000000){\hspace*{\Width}\raisebox{\Height}{$-2$}}%
%
\special{pa    20   197}\special{pa   -20   197}%
\special{fp}%
\settowidth{\Width}{$-1$}\setlength{\Width}{-1\Width}%
\settoheight{\Height}{$-1$}\settodepth{\Depth}{$-1$}\setlength{\Height}{-0.5\Height}\setlength{\Depth}{0.5\Depth}\addtolength{\Height}{\Depth}%
\put(-0.2000000,-1.0000000){\hspace*{\Width}\raisebox{\Height}{$-1$}}%
%
\special{pa    20  -197}\special{pa   -20  -197}%
\special{fp}%
\settowidth{\Width}{$1$}\setlength{\Width}{-1\Width}%
\settoheight{\Height}{$1$}\settodepth{\Depth}{$1$}\setlength{\Height}{-0.5\Height}\setlength{\Depth}{0.5\Depth}\addtolength{\Height}{\Depth}%
\put(-0.2000000,1.0000000){\hspace*{\Width}\raisebox{\Height}{$1$}}%
%
\special{pa    20  -394}\special{pa   -20  -394}%
\special{fp}%
\settowidth{\Width}{$2$}\setlength{\Width}{-1\Width}%
\settoheight{\Height}{$2$}\settodepth{\Depth}{$2$}\setlength{\Height}{-0.5\Height}\setlength{\Depth}{0.5\Depth}\addtolength{\Height}{\Depth}%
\put(-0.2000000,2.0000000){\hspace*{\Width}\raisebox{\Height}{$2$}}%
%
\special{pa    20  -591}\special{pa   -20  -591}%
\special{fp}%
\settowidth{\Width}{$3$}\setlength{\Width}{-1\Width}%
\settoheight{\Height}{$3$}\settodepth{\Depth}{$3$}\setlength{\Height}{-0.5\Height}\setlength{\Depth}{0.5\Depth}\addtolength{\Height}{\Depth}%
\put(-0.2000000,3.0000000){\hspace*{\Width}\raisebox{\Height}{$3$}}%
%
\special{pa    20  -787}\special{pa   -20  -787}%
\special{fp}%
\settowidth{\Width}{$4$}\setlength{\Width}{-1\Width}%
\settoheight{\Height}{$4$}\settodepth{\Depth}{$4$}\setlength{\Height}{-0.5\Height}\setlength{\Depth}{0.5\Depth}\addtolength{\Height}{\Depth}%
\put(-0.2000000,4.0000000){\hspace*{\Width}\raisebox{\Height}{$4$}}%
%
\special{pa    20  -984}\special{pa   -20  -984}%
\special{fp}%
\settowidth{\Width}{$5$}\setlength{\Width}{-1\Width}%
\settoheight{\Height}{$5$}\settodepth{\Depth}{$5$}\setlength{\Height}{-0.5\Height}\setlength{\Depth}{0.5\Depth}\addtolength{\Height}{\Depth}%
\put(-0.2000000,5.0000000){\hspace*{\Width}\raisebox{\Height}{$5$}}%
%
\special{pa  -787    -0}\special{pa   787    -0}%
\special{fp}%
\special{pa     0   394}\special{pa     0 -1181}%
\special{fp}%
\settowidth{\Width}{$x$}\setlength{\Width}{0\Width}%
\settoheight{\Height}{$x$}\settodepth{\Depth}{$x$}\setlength{\Height}{-0.5\Height}\setlength{\Depth}{0.5\Depth}\addtolength{\Height}{\Depth}%
\put(4.1000000,0.0000000){\hspace*{\Width}\raisebox{\Height}{$x$}}%
%
\settowidth{\Width}{$y$}\setlength{\Width}{-0.5\Width}%
\settoheight{\Height}{$y$}\settodepth{\Depth}{$y$}\setlength{\Height}{\Depth}%
\put(0.0000000,6.1000000){\hspace*{\Width}\raisebox{\Height}{$y$}}%
%
\settowidth{\Width}{O}\setlength{\Width}{-1\Width}%
\settoheight{\Height}{O}\settodepth{\Depth}{O}\setlength{\Height}{-\Height}%
\put(-0.1000000,-0.1000000){\hspace*{\Width}\raisebox{\Height}{O}}%
%
{%
\color[cmyk]{1,1,0,0}%
\special{pa  -787   394}\special{pa   787 -1181}%
\special{fp}%
}%
{%
\color[cmyk]{0,0,0,0}%
\special{pa 409 -787}\special{pa 409 -789}\special{pa 408 -791}\special{pa 408 -793}%
\special{pa 407 -795}\special{pa 406 -796}\special{pa 405 -798}\special{pa 403 -799}%
\special{pa 402 -800}\special{pa 400 -801}\special{pa 398 -802}\special{pa 397 -802}%
\special{pa 395 -802}\special{pa 393 -802}\special{pa 391 -802}\special{pa 389 -802}%
\special{pa 387 -801}\special{pa 386 -800}\special{pa 384 -799}\special{pa 383 -798}%
\special{pa 382 -796}\special{pa 381 -795}\special{pa 380 -793}\special{pa 379 -791}%
\special{pa 379 -789}\special{pa 379 -787}\special{pa 379 -786}\special{pa 379 -784}%
\special{pa 380 -782}\special{pa 381 -780}\special{pa 382 -779}\special{pa 383 -777}%
\special{pa 384 -776}\special{pa 386 -775}\special{pa 387 -774}\special{pa 389 -773}%
\special{pa 391 -773}\special{pa 393 -772}\special{pa 395 -772}\special{pa 397 -773}%
\special{pa 398 -773}\special{pa 400 -774}\special{pa 402 -775}\special{pa 403 -776}%
\special{pa 405 -777}\special{pa 406 -779}\special{pa 407 -780}\special{pa 408 -782}%
\special{pa 408 -784}\special{pa 409 -786}\special{pa 409 -787}\special{pa 409 -787}%
\special{sh 1}\special{ip}%
}%
\special{pa   409  -787}\special{pa   409  -789}\special{pa   408  -791}\special{pa   408  -793}%
\special{pa   407  -795}\special{pa   406  -796}\special{pa   405  -798}\special{pa   403  -799}%
\special{pa   402  -800}\special{pa   400  -801}\special{pa   398  -802}\special{pa   397  -802}%
\special{pa   395  -802}\special{pa   393  -802}\special{pa   391  -802}\special{pa   389  -802}%
\special{pa   387  -801}\special{pa   386  -800}\special{pa   384  -799}\special{pa   383  -798}%
\special{pa   382  -796}\special{pa   381  -795}\special{pa   380  -793}\special{pa   379  -791}%
\special{pa   379  -789}\special{pa   379  -787}\special{pa   379  -786}\special{pa   379  -784}%
\special{pa   380  -782}\special{pa   381  -780}\special{pa   382  -779}\special{pa   383  -777}%
\special{pa   384  -776}\special{pa   386  -775}\special{pa   387  -774}\special{pa   389  -773}%
\special{pa   391  -773}\special{pa   393  -772}\special{pa   395  -772}\special{pa   397  -773}%
\special{pa   398  -773}\special{pa   400  -774}\special{pa   402  -775}\special{pa   403  -776}%
\special{pa   405  -777}\special{pa   406  -779}\special{pa   407  -780}\special{pa   408  -782}%
\special{pa   408  -784}\special{pa   409  -786}\special{pa   409  -787}%
\special{fp}%
\end{picture}}%}
\end{layer}

\vspace{6mm}
\begin{itemize}
\item
$y=\bunsuu{x^2-4}{x-2}=$
$\bunsuu{(x-2)(x+2)}{x-2}$\vspace{2mm}\\
$\phantom{y}=x+2$
\item
[課題]\monbannoadd $y=\bunsuu{x^2-4}{x-2}$のグラフとして\\図は正しくない.理由を述べよ.
\item
正しくは\ $\Longrightarrow$
\end{itemize}
\addban

\newslide{分母が0に近づくときの極限}

\vspace*{18mm}


\begin{layer}{120}{0}
\putnotew{96}{73}{\hyperlink{para8pg5}{\fbox{\Ctab{2.5mm}{\scalebox{1}{\scriptsize $\mathstrut||\!\lhd$}}}}}
\putnotew{101}{73}{\hyperlink{para9pg1}{\fbox{\Ctab{2.5mm}{\scalebox{1}{\scriptsize $\mathstrut|\!\lhd$}}}}}
\putnotew{108}{73}{\hyperlink{para9pg5}{\fbox{\Ctab{4.5mm}{\scalebox{1}{\scriptsize $\mathstrut\!\!\lhd\!\!$}}}}}
\putnotew{115}{73}{\hyperlink{para9pg6}{\fbox{\Ctab{4.5mm}{\scalebox{1}{\scriptsize $\mathstrut\!\rhd\!$}}}}}
\putnotew{120}{73}{\hyperlink{para9pg6}{\fbox{\Ctab{2.5mm}{\scalebox{1}{\scriptsize $\mathstrut \!\rhd\!\!|$}}}}}
\putnotew{125}{73}{\hyperlink{para10pg1}{\fbox{\Ctab{2.5mm}{\scalebox{1}{\scriptsize $\mathstrut \!\rhd\!\!||$}}}}}
\putnotee{126}{73}{\scriptsize\color{blue} 6/6}
\end{layer}

\slidepage
\begin{itemize}
\item
[例]\hspace{-1mm}2)\ $\dlim_{x \to 2}\bunsuu{x^2-4}{x-2}$\ \
\item
$x\to 2$とすると,分母も分子も0に近づく
\item
そのままでは極限値がわからない
\item
$x^2-4=(x-2)(x+2)$と因数分解して\\
\hspace*{2zw}$\dlim_{x \to 2}\bunsuu{(x-2)(x+2)}{x-2}$
$=\dlim_{x \to 2}\bunsuu{x+2}{1}=\hakoa{4}$
\end{itemize}

\newslide{$a$における変化率}

\vspace*{18mm}


\begin{layer}{120}{0}
\putnotew{96}{73}{\hyperlink{para9pg6}{\fbox{\Ctab{2.5mm}{\scalebox{1}{\scriptsize $\mathstrut||\!\lhd$}}}}}
\putnotew{101}{73}{\hyperlink{para10pg1}{\fbox{\Ctab{2.5mm}{\scalebox{1}{\scriptsize $\mathstrut|\!\lhd$}}}}}
\putnotew{108}{73}{\hyperlink{para10pg3}{\fbox{\Ctab{4.5mm}{\scalebox{1}{\scriptsize $\mathstrut\!\!\lhd\!\!$}}}}}
\putnotew{115}{73}{\hyperlink{para10pg4}{\fbox{\Ctab{4.5mm}{\scalebox{1}{\scriptsize $\mathstrut\!\rhd\!$}}}}}
\putnotew{120}{73}{\hyperlink{para10pg4}{\fbox{\Ctab{2.5mm}{\scalebox{1}{\scriptsize $\mathstrut \!\rhd\!\!|$}}}}}
\putnotew{125}{73}{\hyperlink{para11pg1}{\fbox{\Ctab{2.5mm}{\scalebox{1}{\scriptsize $\mathstrut \!\rhd\!\!||$}}}}}
\putnotee{126}{73}{\scriptsize\color{blue} 4/4}
\end{layer}

\slidepage

\begin{layer}{120}{0}
\putnotese{70}{5}{\scalebox{0.8}{%%% /polytech.git/n107/fig/shunkanhenkaritsu.tex 
%%% Generator=henkaritu.cdy 
{\unitlength=1cm%
\begin{picture}%
(6,6)(-1,-1)%
\special{pn 8}%
%
\special{pn 12}%
\special{pa  -257  -267}\special{pa  -156  -277}\special{pa   -61  -287}\special{pa    27  -297}%
\special{pa   109  -309}\special{pa   186  -320}\special{pa   256  -333}\special{pa   320  -346}%
\special{pa   378  -359}\special{pa   431  -373}\special{pa   477  -388}\special{pa   541  -411}%
\special{pa   604  -437}\special{pa   666  -465}\special{pa   727  -495}\special{pa   786  -528}%
\special{pa   845  -563}\special{pa   903  -600}\special{pa   959  -639}\special{pa  1014  -680}%
\special{pa  1068  -723}\special{pa  1128  -773}\special{pa  1185  -824}\special{pa  1241  -877}%
\special{pa  1295  -932}\special{pa  1347  -988}\special{pa  1396 -1045}\special{pa  1444 -1103}%
\special{pa  1489 -1164}\special{pa  1531 -1225}\special{pa  1571 -1288}\special{pa  1596 -1333}%
\special{pa  1620 -1380}\special{pa  1643 -1431}\special{pa  1665 -1486}\special{pa  1685 -1544}%
\special{pa  1704 -1605}\special{pa  1723 -1670}\special{pa  1740 -1738}\special{pa  1755 -1810}%
\special{pa  1770 -1885}%
\special{fp}%
\special{pn 8}%
\special{pa   477   -20}\special{pa   477    20}%
\special{fp}%
\settowidth{\Width}{$a$}\setlength{\Width}{-0.5\Width}%
\settoheight{\Height}{$a$}\settodepth{\Depth}{$a$}\setlength{\Height}{-\Height}%
\put(1.2100000,-0.1000000){\hspace*{\Width}\raisebox{\Height}{$a$}}%
%
\special{pa   981   -20}\special{pa   981    20}%
\special{fp}%
\settowidth{\Width}{$z$}\setlength{\Width}{-0.5\Width}%
\settoheight{\Height}{$z$}\settodepth{\Depth}{$z$}\setlength{\Height}{-\Height}%
\put(2.4900000,-0.1000000){\hspace*{\Width}\raisebox{\Height}{$z$}}%
%
\special{pa    20  -388}\special{pa   -20  -388}%
\special{fp}%
\settowidth{\Width}{$f(a)$}\setlength{\Width}{-1\Width}%
\settoheight{\Height}{$f(a)$}\settodepth{\Depth}{$f(a)$}\setlength{\Height}{-0.5\Height}\setlength{\Depth}{0.5\Depth}\addtolength{\Height}{\Depth}%
\put(-0.1000000,0.9800000){\hspace*{\Width}\raisebox{\Height}{$f(a)$}}%
%
\special{pa    20  -655}\special{pa   -20  -655}%
\special{fp}%
\settowidth{\Width}{$f(z)$}\setlength{\Width}{-1\Width}%
\settoheight{\Height}{$f(z)$}\settodepth{\Depth}{$f(z)$}\setlength{\Height}{-0.5\Height}\setlength{\Depth}{0.5\Depth}\addtolength{\Height}{\Depth}%
\put(-0.1000000,1.6600000){\hspace*{\Width}\raisebox{\Height}{$f(z)$}}%
%
\special{pn 8}%
\special{pa 477 4}\special{pa 477 -4}\special{fp}\special{pa 477 -35}\special{pa 477 -43}\special{fp}%
\special{pa 477 -75}\special{pa 477 -83}\special{fp}\special{pa 477 -114}\special{pa 477 -122}\special{fp}%
\special{pa 477 -153}\special{pa 477 -161}\special{fp}\special{pa 477 -192}\special{pa 477 -200}\special{fp}%
\special{pa 477 -232}\special{pa 477 -240}\special{fp}\special{pa 477 -271}\special{pa 477 -279}\special{fp}%
\special{pa 477 -310}\special{pa 477 -318}\special{fp}\special{pa 477 -350}\special{pa 476 -357}\special{fp}%
\special{pa 474 -385}\special{pa 468 -390}\special{fp}\special{pa 436 -388}\special{pa 428 -388}\special{fp}%
\special{pa 397 -388}\special{pa 389 -388}\special{fp}\special{pa 357 -388}\special{pa 349 -388}\special{fp}%
\special{pa 318 -388}\special{pa 310 -388}\special{fp}\special{pa 279 -388}\special{pa 271 -388}\special{fp}%
\special{pa 240 -388}\special{pa 232 -388}\special{fp}\special{pa 200 -388}\special{pa 192 -388}\special{fp}%
\special{pa 161 -388}\special{pa 153 -388}\special{fp}\special{pa 122 -388}\special{pa 114 -388}\special{fp}%
\special{pa 83 -388}\special{pa 75 -388}\special{fp}\special{pa 43 -388}\special{pa 35 -388}\special{fp}%
\special{pa 4 -388}\special{pa -4 -388}\special{fp}\special{pn 8}%
\special{pn 8}%
\special{pa 981 4}\special{pa 981 -4}\special{fp}\special{pa 981 -35}\special{pa 981 -43}\special{fp}%
\special{pa 981 -74}\special{pa 981 -82}\special{fp}\special{pa 981 -113}\special{pa 981 -121}\special{fp}%
\special{pa 981 -152}\special{pa 981 -160}\special{fp}\special{pa 981 -191}\special{pa 981 -199}\special{fp}%
\special{pa 981 -230}\special{pa 981 -238}\special{fp}\special{pa 981 -269}\special{pa 981 -277}\special{fp}%
\special{pa 981 -308}\special{pa 981 -316}\special{fp}\special{pa 981 -347}\special{pa 981 -355}\special{fp}%
\special{pa 981 -385}\special{pa 981 -393}\special{fp}\special{pa 981 -424}\special{pa 981 -432}\special{fp}%
\special{pa 981 -463}\special{pa 981 -471}\special{fp}\special{pa 981 -502}\special{pa 981 -510}\special{fp}%
\special{pa 981 -541}\special{pa 981 -549}\special{fp}\special{pa 981 -580}\special{pa 981 -588}\special{fp}%
\special{pa 981 -619}\special{pa 980 -627}\special{fp}\special{pa 977 -652}\special{pa 971 -657}\special{fp}%
\special{pa 939 -655}\special{pa 931 -655}\special{fp}\special{pa 900 -655}\special{pa 892 -655}\special{fp}%
\special{pa 861 -655}\special{pa 853 -655}\special{fp}\special{pa 822 -655}\special{pa 814 -655}\special{fp}%
\special{pa 783 -655}\special{pa 775 -655}\special{fp}\special{pa 744 -655}\special{pa 736 -655}\special{fp}%
\special{pa 705 -655}\special{pa 697 -655}\special{fp}\special{pa 666 -655}\special{pa 658 -655}\special{fp}%
\special{pa 627 -655}\special{pa 619 -655}\special{fp}\special{pa 588 -655}\special{pa 580 -655}\special{fp}%
\special{pa 549 -655}\special{pa 541 -655}\special{fp}\special{pa 510 -655}\special{pa 502 -655}\special{fp}%
\special{pa 471 -655}\special{pa 463 -655}\special{fp}\special{pa 432 -655}\special{pa 424 -655}\special{fp}%
\special{pa 393 -655}\special{pa 385 -655}\special{fp}\special{pa 355 -655}\special{pa 347 -655}\special{fp}%
\special{pa 316 -655}\special{pa 308 -655}\special{fp}\special{pa 277 -655}\special{pa 269 -655}\special{fp}%
\special{pa 238 -655}\special{pa 230 -655}\special{fp}\special{pa 199 -655}\special{pa 191 -655}\special{fp}%
\special{pa 160 -655}\special{pa 152 -655}\special{fp}\special{pa 121 -655}\special{pa 113 -655}\special{fp}%
\special{pa 82 -655}\special{pa 74 -655}\special{fp}\special{pa 43 -655}\special{pa 35 -655}\special{fp}%
\special{pa 4 -655}\special{pa -4 -655}\special{fp}\special{pn 8}%
\settowidth{\Width}{$y=f(x)$}\setlength{\Width}{0\Width}%
\settoheight{\Height}{$y=f(x)$}\settodepth{\Depth}{$y=f(x)$}\setlength{\Height}{-\Height}%
\put(4.1400000,3.2200000){\hspace*{\Width}\raisebox{\Height}{$y=f(x)$}}%
%
\special{pa  -394    -0}\special{pa  1969    -0}%
\special{fp}%
\special{pa     0   394}\special{pa     0 -1969}%
\special{fp}%
\settowidth{\Width}{$x$}\setlength{\Width}{0\Width}%
\settoheight{\Height}{$x$}\settodepth{\Depth}{$x$}\setlength{\Height}{-0.5\Height}\setlength{\Depth}{0.5\Depth}\addtolength{\Height}{\Depth}%
\put(5.0500000,0.0000000){\hspace*{\Width}\raisebox{\Height}{$x$}}%
%
\settowidth{\Width}{$y$}\setlength{\Width}{-0.5\Width}%
\settoheight{\Height}{$y$}\settodepth{\Depth}{$y$}\setlength{\Height}{\Depth}%
\put(0.0000000,5.0500000){\hspace*{\Width}\raisebox{\Height}{$y$}}%
%
\settowidth{\Width}{O}\setlength{\Width}{-1\Width}%
\settoheight{\Height}{O}\settodepth{\Depth}{O}\setlength{\Height}{-\Height}%
\put(-0.0500000,-0.0500000){\hspace*{\Width}\raisebox{\Height}{O}}%
%
\end{picture}}%}}
\end{layer}

\begin{itemize}
\item
$a$の近くに$z$をとる
\item
$[a,\ z]$での平均変化率は\\
\hspace*{2zw}$\bunsuu{f(z)-f(a)}{z-a}$
\item
$z \to a$の極限値\\
\hspace*{2zw}$\dlim_{z\to a}\bunsuu{f(z)-f(a)}{z-a}$
\item
これを$a$における{\color{red}微分係数}といい,$f'(a)$と書く
\end{itemize}

\newslide{微分係数の計算例}

\vspace*{18mm}


\begin{layer}{120}{0}
\putnotew{96}{73}{\hyperlink{para10pg4}{\fbox{\Ctab{2.5mm}{\scalebox{1}{\scriptsize $\mathstrut||\!\lhd$}}}}}
\putnotew{101}{73}{\hyperlink{para11pg1}{\fbox{\Ctab{2.5mm}{\scalebox{1}{\scriptsize $\mathstrut|\!\lhd$}}}}}
\putnotew{108}{73}{\hyperlink{para11pg5}{\fbox{\Ctab{4.5mm}{\scalebox{1}{\scriptsize $\mathstrut\!\!\lhd\!\!$}}}}}
\putnotew{115}{73}{\hyperlink{para11pg6}{\fbox{\Ctab{4.5mm}{\scalebox{1}{\scriptsize $\mathstrut\!\rhd\!$}}}}}
\putnotew{120}{73}{\hyperlink{para11pg6}{\fbox{\Ctab{2.5mm}{\scalebox{1}{\scriptsize $\mathstrut \!\rhd\!\!|$}}}}}
\putnotew{125}{73}{\hyperlink{para12pg1}{\fbox{\Ctab{2.5mm}{\scalebox{1}{\scriptsize $\mathstrut \!\rhd\!\!||$}}}}}
\putnotee{126}{73}{\scriptsize\color{blue} 6/6}
\end{layer}

\slidepage
{\color[cmyk]{1,0,0,0}

\begin{layer}{120}{0}
\lineseg{93}{29}{13}{-13}
\lineseg{85}{36}{13}{-13}
\end{layer}

}
\begin{itemize}
\item
[例]$f(x)=x^2$の$1$における微分係数$f'(1)$
\item
[]$f'(1)=\dlim_{z\to 1}\bunsuu{f(z)-f(1)}{z-1}$\\
$\phantom{f'(1)}=\dlim_{z\to 1}\bunsuu{z^2-1}{z-1}=$
$\dlim_{z\to 1}\bunsuu{(z+1)(z-1)}{z-1}$\vspace{2mm}
\hspace*{2zw}\\$\phantom{f'(1)}=\dlim_{z\to 1}(z+1)$
$=2$
\item
[課題]\monbannoadd 次を求めよ\seteda{100}\\
\eda{$f(x)=2x^2$のとき,$f'(1)$}\\
\eda{$f(x)=3x$のとき,$f'(2)$}
\end{itemize}
\addban

\newslide{微分係数の図形的意味}

\vspace*{18mm}


\begin{layer}{120}{0}
\putnotew{96}{73}{\hyperlink{para11pg6}{\fbox{\Ctab{2.5mm}{\scalebox{1}{\scriptsize $\mathstrut||\!\lhd$}}}}}
\putnotew{101}{73}{\hyperlink{para12pg1}{\fbox{\Ctab{2.5mm}{\scalebox{1}{\scriptsize $\mathstrut|\!\lhd$}}}}}
\putnotew{108}{73}{\hyperlink{para12pg2}{\fbox{\Ctab{4.5mm}{\scalebox{1}{\scriptsize $\mathstrut\!\!\lhd\!\!$}}}}}
\putnotew{115}{73}{\hyperlink{para12pg3}{\fbox{\Ctab{4.5mm}{\scalebox{1}{\scriptsize $\mathstrut\!\rhd\!$}}}}}
\putnotew{120}{73}{\hyperlink{para12pg3}{\fbox{\Ctab{2.5mm}{\scalebox{1}{\scriptsize $\mathstrut \!\rhd\!\!|$}}}}}
\putnotew{125}{73}{\hyperlink{para13pg1}{\fbox{\Ctab{2.5mm}{\scalebox{1}{\scriptsize $\mathstrut \!\rhd\!\!||$}}}}}
\putnotee{126}{73}{\scriptsize\color{blue} 3/3}
\end{layer}

\slidepage
\begin{itemize}
\item
[課題]\monbannoadd \href{https://s-takato.github.io/polytech22/offlineapp/bibunkeisuumainoff.html}%
{微分係数の意味}を動かせ\\
・$a$における微分係数の値$f'(a)$を求めよ
\item
微分係数$f'(a)$は\\
\hspace*{1zw}Aにおける接線の傾き
\end{itemize}
\addban

\newslide{微分係数の定義式の別形}

\vspace*{18mm}


\begin{layer}{120}{0}
\putnotew{96}{73}{\hyperlink{para12pg3}{\fbox{\Ctab{2.5mm}{\scalebox{1}{\scriptsize $\mathstrut||\!\lhd$}}}}}
\putnotew{101}{73}{\hyperlink{para13pg1}{\fbox{\Ctab{2.5mm}{\scalebox{1}{\scriptsize $\mathstrut|\!\lhd$}}}}}
\putnotew{108}{73}{\hyperlink{para13pg3}{\fbox{\Ctab{4.5mm}{\scalebox{1}{\scriptsize $\mathstrut\!\!\lhd\!\!$}}}}}
\putnotew{115}{73}{\hyperlink{para13pg4}{\fbox{\Ctab{4.5mm}{\scalebox{1}{\scriptsize $\mathstrut\!\rhd\!$}}}}}
\putnotew{120}{73}{\hyperlink{para13pg4}{\fbox{\Ctab{2.5mm}{\scalebox{1}{\scriptsize $\mathstrut \!\rhd\!\!|$}}}}}
\putnotew{125}{73}{\hyperlink{para14pg1}{\fbox{\Ctab{2.5mm}{\scalebox{1}{\scriptsize $\mathstrut \!\rhd\!\!||$}}}}}
\putnotee{126}{73}{\scriptsize\color{blue} 4/4}
\end{layer}

\slidepage
\begin{itemize}
\item
$f'(a)=\dlim_{x\to a}\bunsuu{f(z)-f(a)}{z-a}$\hfill(1)
\item
$z-a=h$とおくと $z=a+h,\ h \to 0$\vspace{1mm}\\
\hspace*{1zw}$f'(a)=\dlim_{h\to 0}\bunsuu{f(a+h)-f(a)}{h}$\hfill(2)
\item
(2)はよく用いられるが,(1)がおすすめ
\end{itemize}

\mainslide{導関数}


\slidepage[m]
%%%%%%%%%%%%%

%%%%%%%%%%%%%%%%%%%%

\newslide{導関数}

\vspace*{18mm}


\begin{layer}{120}{0}
\putnotese{80}{14}{\scalebox{0.7}{%%% /Users/takatoosetsuo/Dropbox/2018polytec/lecture/0521/presen/fig/sinecurve/p017.tex 
%%% Generator=presen0521.cdy 
{\unitlength=12.5mm%
\begin{picture}%
(9.2,2.4)(-2.2,-1.2)%
\special{pn 8}%
%
\Large\bf\boldmath%
\small%
\special{pa     0    -0}\special{pa    -4   -62}\special{pa   -15  -122}\special{pa   -35  -181}%
\special{pa   -61  -237}\special{pa   -94  -289}\special{pa  -133  -337}\special{pa  -178  -379}%
\special{pa  -228  -416}\special{pa  -283  -445}\special{pa  -340  -468}\special{pa  -400  -483}%
\special{pa  -461  -491}\special{pa  -523  -491}\special{pa  -584  -483}\special{pa  -644  -468}%
\special{pa  -702  -445}\special{pa  -756  -416}\special{pa  -806  -379}\special{pa  -851  -337}%
\special{pa  -890  -289}\special{pa  -923  -237}\special{pa  -950  -181}\special{pa  -969  -122}%
\special{pa  -980   -62}\special{pa  -984     0}\special{pa  -980    62}\special{pa  -969   122}%
\special{pa  -950   181}\special{pa  -923   237}\special{pa  -890   289}\special{pa  -851   337}%
\special{pa  -806   379}\special{pa  -756   416}\special{pa  -702   445}\special{pa  -644   468}%
\special{pa  -584   483}\special{pa  -523   491}\special{pa  -461   491}\special{pa  -400   483}%
\special{pa  -340   468}\special{pa  -283   445}\special{pa  -228   416}\special{pa  -178   379}%
\special{pa  -133   337}\special{pa   -94   289}\special{pa   -61   237}\special{pa   -35   181}%
\special{pa   -15   122}\special{pa    -4    62}\special{pa     0     0}%
\special{fp}%
\special{pa  -492    -0}\special{pa     0    -0}%
\special{fp}%
\special{pa  -492    -0}\special{pa  -806   379}%
\special{fp}%
\special{pa 1979 379}\special{pa 1979 345}\special{fp}\special{pa 1979 310}\special{pa 1979 276}\special{fp}%
\special{pa 1979 241}\special{pa 1979 207}\special{fp}\special{pa 1979 172}\special{pa 1979 138}\special{fp}%
\special{pa 1979 103}\special{pa 1979 69}\special{fp}\special{pa 1979 34}\special{pa 1979 0}\special{fp}%
%
%
\special{pa -806 379}\special{pa -767 379}\special{fp}\special{pa -727 379}\special{pa -688 379}\special{fp}%
\special{pa -649 379}\special{pa -610 379}\special{fp}\special{pa -570 379}\special{pa -531 379}\special{fp}%
\special{pa -492 379}\special{pa -453 379}\special{fp}\special{pa -414 379}\special{pa -374 379}\special{fp}%
\special{pa -335 379}\special{pa -296 379}\special{fp}\special{pa -257 379}\special{pa -217 379}\special{fp}%
\special{pa -178 379}\special{pa -139 379}\special{fp}\special{pa -100 379}\special{pa -61 379}\special{fp}%
\special{pa -21 379}\special{pa 18 379}\special{fp}\special{pa 57 379}\special{pa 96 379}\special{fp}%
\special{pa 136 379}\special{pa 175 379}\special{fp}\special{pa 214 379}\special{pa 253 379}\special{fp}%
\special{pa 292 379}\special{pa 332 379}\special{fp}\special{pa 371 379}\special{pa 410 379}\special{fp}%
\special{pa 449 379}\special{pa 489 379}\special{fp}\special{pa 528 379}\special{pa 567 379}\special{fp}%
\special{pa 606 379}\special{pa 645 379}\special{fp}\special{pa 685 379}\special{pa 724 379}\special{fp}%
\special{pa 763 379}\special{pa 802 379}\special{fp}\special{pa 842 379}\special{pa 881 379}\special{fp}%
\special{pa 920 379}\special{pa 959 379}\special{fp}\special{pa 998 379}\special{pa 1038 379}\special{fp}%
\special{pa 1077 379}\special{pa 1116 379}\special{fp}\special{pa 1155 379}\special{pa 1195 379}\special{fp}%
\special{pa 1234 379}\special{pa 1273 379}\special{fp}\special{pa 1312 379}\special{pa 1351 379}\special{fp}%
\special{pa 1391 379}\special{pa 1430 379}\special{fp}\special{pa 1469 379}\special{pa 1508 379}\special{fp}%
\special{pa 1548 379}\special{pa 1587 379}\special{fp}\special{pa 1626 379}\special{pa 1665 379}\special{fp}%
\special{pa 1704 379}\special{pa 1744 379}\special{fp}\special{pa 1783 379}\special{pa 1822 379}\special{fp}%
\special{pa 1861 379}\special{pa 1901 379}\special{fp}\special{pa 1940 379}\special{pa 1979 379}\special{fp}%
%
%
\settowidth{\Width}{$x$}\setlength{\Width}{-0.5\Width}%
\settoheight{\Height}{$x$}\settodepth{\Depth}{$x$}\setlength{\Height}{-0.5\Height}\setlength{\Depth}{0.5\Depth}\addtolength{\Height}{\Depth}%
\put(-1.2200000,0.4800000){\hspace*{\Width}\raisebox{\Height}{$x$}}%
%
\special{pa  -320    -0}\special{pa  -321   -22}\special{pa  -325   -43}\special{pa  -332   -63}%
\special{pa  -341   -83}\special{pa  -353  -101}\special{pa  -367  -118}\special{pa  -382  -133}%
\special{pa  -400  -145}\special{pa  -419  -156}\special{pa  -439  -164}\special{pa  -460  -169}%
\special{pa  -481  -172}\special{pa  -503  -172}\special{pa  -524  -169}\special{pa  -545  -164}%
\special{pa  -565  -156}\special{pa  -584  -145}\special{pa  -602  -133}\special{pa  -618  -118}%
\special{pa  -631  -101}\special{pa  -643   -83}\special{pa  -652   -63}\special{pa  -659   -43}%
\special{pa  -663   -22}\special{pa  -664     0}\special{pa  -663    22}\special{pa  -659    43}%
\special{pa  -652    63}\special{pa  -643    83}\special{pa  -631   101}\special{pa  -618   118}%
\special{pa  -602   133}%
\special{fp}%
\color[cmyk]{0,1,1,0}%
\special{pa     0    -0}\special{pa    40   -40}\special{pa    79   -79}\special{pa   119  -118}%
\special{pa   158  -156}\special{pa   198  -193}\special{pa   237  -228}\special{pa   277  -263}%
\special{pa   317  -295}\special{pa   356  -326}\special{pa   396  -354}\special{pa   435  -381}%
\special{pa   475  -405}\special{pa   515  -426}\special{pa   554  -444}\special{pa   594  -460}%
\special{pa   633  -472}\special{pa   673  -482}\special{pa   712  -488}\special{pa   752  -492}%
\special{pa   792  -492}\special{pa   831  -489}\special{pa   871  -482}\special{pa   910  -473}%
\special{pa   950  -461}\special{pa   989  -445}\special{pa  1029  -427}\special{pa  1069  -406}%
\special{pa  1108  -382}\special{pa  1148  -356}\special{pa  1187  -328}\special{pa  1227  -297}%
\special{pa  1267  -265}\special{pa  1306  -231}\special{pa  1346  -195}\special{pa  1385  -158}%
\special{pa  1425  -120}\special{pa  1464   -81}\special{pa  1504   -42}\special{pa  1544    -2}%
\special{pa  1583    37}\special{pa  1623    76}\special{pa  1662   115}\special{pa  1702   153}%
\special{pa  1741   190}\special{pa  1781   226}\special{pa  1821   261}\special{pa  1860   293}%
\special{pa  1900   324}\special{pa  1939   353}\special{pa  1979   379}%
\special{fp}%
\color[cmyk]{0,0,0,1}%
\special{pa   773   -20}\special{pa   773    20}%
\special{fp}%
\settowidth{\Width}{$\frac{\pi}{2}$}\setlength{\Width}{-0.5\Width}%
\settoheight{\Height}{$\frac{\pi}{2}$}\settodepth{\Depth}{$\frac{\pi}{2}$}\setlength{\Height}{\Depth}%
\put(1.5707960,0.0800000){\hspace*{\Width}\raisebox{\Height}{$\frac{\pi}{2}$}}%
%
%
\special{pa  1546   -20}\special{pa  1546    20}%
\special{fp}%
\settowidth{\Width}{$\pi$}\setlength{\Width}{-0.5\Width}%
\settoheight{\Height}{$\pi$}\settodepth{\Depth}{$\pi$}\setlength{\Height}{\Depth}%
\put(3.1415930,0.0800000){\hspace*{\Width}\raisebox{\Height}{$\pi$}}%
%
%
\special{pa  3092   -20}\special{pa  3092    20}%
\special{fp}%
\settowidth{\Width}{$2\pi$}\setlength{\Width}{-0.5\Width}%
\settoheight{\Height}{$2\pi$}\settodepth{\Depth}{$2\pi$}\setlength{\Height}{\Depth}%
\put(6.2831850,0.0800000){\hspace*{\Width}\raisebox{\Height}{$2\pi$}}%
%
%
\special{pa    20   492}\special{pa   -20   492}%
\special{fp}%
\settowidth{\Width}{$-1$}\setlength{\Width}{-1\Width}%
\settoheight{\Height}{$-1$}\settodepth{\Depth}{$-1$}\setlength{\Height}{-0.5\Height}\setlength{\Depth}{0.5\Depth}\addtolength{\Height}{\Depth}%
\put(-0.0800000,-1.0000000){\hspace*{\Width}\raisebox{\Height}{$-1$}}%
%
%
\special{pa    20  -492}\special{pa   -20  -492}%
\special{fp}%
\settowidth{\Width}{$1$}\setlength{\Width}{-1\Width}%
\settoheight{\Height}{$1$}\settodepth{\Depth}{$1$}\setlength{\Height}{-0.5\Height}\setlength{\Depth}{0.5\Depth}\addtolength{\Height}{\Depth}%
\put(-0.0800000,1.0000000){\hspace*{\Width}\raisebox{\Height}{$1$}}%
%
%
\special{pa -1083    -0}\special{pa  3445    -0}%
\special{fp}%
\special{pa     0   591}\special{pa     0  -591}%
\special{fp}%
\settowidth{\Width}{$x$}\setlength{\Width}{0\Width}%
\settoheight{\Height}{$x$}\settodepth{\Depth}{$x$}\setlength{\Height}{-0.5\Height}\setlength{\Depth}{0.5\Depth}\addtolength{\Height}{\Depth}%
\put(7.0400000,0.0000000){\hspace*{\Width}\raisebox{\Height}{$x$}}%
%
\settowidth{\Width}{$y$}\setlength{\Width}{-0.5\Width}%
\settoheight{\Height}{$y$}\settodepth{\Depth}{$y$}\setlength{\Height}{\Depth}%
\put(0.0000000,1.2400000){\hspace*{\Width}\raisebox{\Height}{$y$}}%
%
\settowidth{\Width}{O}\setlength{\Width}{0\Width}%
\settoheight{\Height}{O}\settodepth{\Depth}{O}\setlength{\Height}{-\Height}%
\put(0.0400000,-0.0400000){\hspace*{\Width}\raisebox{\Height}{O}}%
%
\end{picture}}%}}
\putnotew{96}{73}{\hyperlink{para13pg4}{\fbox{\Ctab{2.5mm}{\scalebox{1}{\scriptsize $\mathstrut||\!\lhd$}}}}}
\putnotew{101}{73}{\hyperlink{para14pg1}{\fbox{\Ctab{2.5mm}{\scalebox{1}{\scriptsize $\mathstrut|\!\lhd$}}}}}
\putnotew{108}{73}{\hyperlink{para14pg16}{\fbox{\Ctab{4.5mm}{\scalebox{1}{\scriptsize $\mathstrut\!\!\lhd\!\!$}}}}}
\putnotew{115}{73}{\hyperlink{para14pg17}{\fbox{\Ctab{4.5mm}{\scalebox{1}{\scriptsize $\mathstrut\!\rhd\!$}}}}}
\putnotew{120}{73}{\hyperlink{para14pg17}{\fbox{\Ctab{2.5mm}{\scalebox{1}{\scriptsize $\mathstrut \!\rhd\!\!|$}}}}}
\putnotew{125}{73}{\hyperlink{para15pg1}{\fbox{\Ctab{2.5mm}{\scalebox{1}{\scriptsize $\mathstrut \!\rhd\!\!||$}}}}}
\putnotee{126}{73}{\scriptsize\color{blue} 17/17}
\end{layer}

\slidepage
\begin{itemize}
\item
微分係数$f'(a)$
\item
$a$を動かすと,$f'(a)$も変わる\\
\hspace*{2zw}$\Longrightarrow$\ $f'(a)$は$a$の関数
\end{itemize}

\newslide{導関数2}

\vspace*{18mm}


\begin{layer}{120}{0}
\putnotew{96}{73}{\hyperlink{para14pg17}{\fbox{\Ctab{2.5mm}{\scalebox{1}{\scriptsize $\mathstrut||\!\lhd$}}}}}
\putnotew{101}{73}{\hyperlink{para15pg1}{\fbox{\Ctab{2.5mm}{\scalebox{1}{\scriptsize $\mathstrut|\!\lhd$}}}}}
\putnotew{108}{73}{\hyperlink{para15pg2}{\fbox{\Ctab{4.5mm}{\scalebox{1}{\scriptsize $\mathstrut\!\!\lhd\!\!$}}}}}
\putnotew{115}{73}{\hyperlink{para15pg3}{\fbox{\Ctab{4.5mm}{\scalebox{1}{\scriptsize $\mathstrut\!\rhd\!$}}}}}
\putnotew{120}{73}{\hyperlink{para15pg3}{\fbox{\Ctab{2.5mm}{\scalebox{1}{\scriptsize $\mathstrut \!\rhd\!\!|$}}}}}
\putnotew{125}{73}{\hyperlink{para16pg1}{\fbox{\Ctab{2.5mm}{\scalebox{1}{\scriptsize $\mathstrut \!\rhd\!\!||$}}}}}
\putnotee{126}{73}{\scriptsize\color{blue} 3/3}
\end{layer}

\slidepage

\begin{layer}{120}{0}
\putnotese{80}{14}{\scalebox{0.7}{%%% /Users/takatoosetsuo/Dropbox/2018polytec/lecture/0521/presen/fig/sinecurve/p017.tex 
%%% Generator=presen0521.cdy 
{\unitlength=12.5mm%
\begin{picture}%
(9.2,2.4)(-2.2,-1.2)%
\special{pn 8}%
%
\Large\bf\boldmath%
\small%
\special{pa     0    -0}\special{pa    -4   -62}\special{pa   -15  -122}\special{pa   -35  -181}%
\special{pa   -61  -237}\special{pa   -94  -289}\special{pa  -133  -337}\special{pa  -178  -379}%
\special{pa  -228  -416}\special{pa  -283  -445}\special{pa  -340  -468}\special{pa  -400  -483}%
\special{pa  -461  -491}\special{pa  -523  -491}\special{pa  -584  -483}\special{pa  -644  -468}%
\special{pa  -702  -445}\special{pa  -756  -416}\special{pa  -806  -379}\special{pa  -851  -337}%
\special{pa  -890  -289}\special{pa  -923  -237}\special{pa  -950  -181}\special{pa  -969  -122}%
\special{pa  -980   -62}\special{pa  -984     0}\special{pa  -980    62}\special{pa  -969   122}%
\special{pa  -950   181}\special{pa  -923   237}\special{pa  -890   289}\special{pa  -851   337}%
\special{pa  -806   379}\special{pa  -756   416}\special{pa  -702   445}\special{pa  -644   468}%
\special{pa  -584   483}\special{pa  -523   491}\special{pa  -461   491}\special{pa  -400   483}%
\special{pa  -340   468}\special{pa  -283   445}\special{pa  -228   416}\special{pa  -178   379}%
\special{pa  -133   337}\special{pa   -94   289}\special{pa   -61   237}\special{pa   -35   181}%
\special{pa   -15   122}\special{pa    -4    62}\special{pa     0     0}%
\special{fp}%
\special{pa  -492    -0}\special{pa     0    -0}%
\special{fp}%
\special{pa  -492    -0}\special{pa  -806   379}%
\special{fp}%
\special{pa 1979 379}\special{pa 1979 345}\special{fp}\special{pa 1979 310}\special{pa 1979 276}\special{fp}%
\special{pa 1979 241}\special{pa 1979 207}\special{fp}\special{pa 1979 172}\special{pa 1979 138}\special{fp}%
\special{pa 1979 103}\special{pa 1979 69}\special{fp}\special{pa 1979 34}\special{pa 1979 0}\special{fp}%
%
%
\special{pa -806 379}\special{pa -767 379}\special{fp}\special{pa -727 379}\special{pa -688 379}\special{fp}%
\special{pa -649 379}\special{pa -610 379}\special{fp}\special{pa -570 379}\special{pa -531 379}\special{fp}%
\special{pa -492 379}\special{pa -453 379}\special{fp}\special{pa -414 379}\special{pa -374 379}\special{fp}%
\special{pa -335 379}\special{pa -296 379}\special{fp}\special{pa -257 379}\special{pa -217 379}\special{fp}%
\special{pa -178 379}\special{pa -139 379}\special{fp}\special{pa -100 379}\special{pa -61 379}\special{fp}%
\special{pa -21 379}\special{pa 18 379}\special{fp}\special{pa 57 379}\special{pa 96 379}\special{fp}%
\special{pa 136 379}\special{pa 175 379}\special{fp}\special{pa 214 379}\special{pa 253 379}\special{fp}%
\special{pa 292 379}\special{pa 332 379}\special{fp}\special{pa 371 379}\special{pa 410 379}\special{fp}%
\special{pa 449 379}\special{pa 489 379}\special{fp}\special{pa 528 379}\special{pa 567 379}\special{fp}%
\special{pa 606 379}\special{pa 645 379}\special{fp}\special{pa 685 379}\special{pa 724 379}\special{fp}%
\special{pa 763 379}\special{pa 802 379}\special{fp}\special{pa 842 379}\special{pa 881 379}\special{fp}%
\special{pa 920 379}\special{pa 959 379}\special{fp}\special{pa 998 379}\special{pa 1038 379}\special{fp}%
\special{pa 1077 379}\special{pa 1116 379}\special{fp}\special{pa 1155 379}\special{pa 1195 379}\special{fp}%
\special{pa 1234 379}\special{pa 1273 379}\special{fp}\special{pa 1312 379}\special{pa 1351 379}\special{fp}%
\special{pa 1391 379}\special{pa 1430 379}\special{fp}\special{pa 1469 379}\special{pa 1508 379}\special{fp}%
\special{pa 1548 379}\special{pa 1587 379}\special{fp}\special{pa 1626 379}\special{pa 1665 379}\special{fp}%
\special{pa 1704 379}\special{pa 1744 379}\special{fp}\special{pa 1783 379}\special{pa 1822 379}\special{fp}%
\special{pa 1861 379}\special{pa 1901 379}\special{fp}\special{pa 1940 379}\special{pa 1979 379}\special{fp}%
%
%
\settowidth{\Width}{$x$}\setlength{\Width}{-0.5\Width}%
\settoheight{\Height}{$x$}\settodepth{\Depth}{$x$}\setlength{\Height}{-0.5\Height}\setlength{\Depth}{0.5\Depth}\addtolength{\Height}{\Depth}%
\put(-1.2200000,0.4800000){\hspace*{\Width}\raisebox{\Height}{$x$}}%
%
\special{pa  -320    -0}\special{pa  -321   -22}\special{pa  -325   -43}\special{pa  -332   -63}%
\special{pa  -341   -83}\special{pa  -353  -101}\special{pa  -367  -118}\special{pa  -382  -133}%
\special{pa  -400  -145}\special{pa  -419  -156}\special{pa  -439  -164}\special{pa  -460  -169}%
\special{pa  -481  -172}\special{pa  -503  -172}\special{pa  -524  -169}\special{pa  -545  -164}%
\special{pa  -565  -156}\special{pa  -584  -145}\special{pa  -602  -133}\special{pa  -618  -118}%
\special{pa  -631  -101}\special{pa  -643   -83}\special{pa  -652   -63}\special{pa  -659   -43}%
\special{pa  -663   -22}\special{pa  -664     0}\special{pa  -663    22}\special{pa  -659    43}%
\special{pa  -652    63}\special{pa  -643    83}\special{pa  -631   101}\special{pa  -618   118}%
\special{pa  -602   133}%
\special{fp}%
\color[cmyk]{0,1,1,0}%
\special{pa     0    -0}\special{pa    40   -40}\special{pa    79   -79}\special{pa   119  -118}%
\special{pa   158  -156}\special{pa   198  -193}\special{pa   237  -228}\special{pa   277  -263}%
\special{pa   317  -295}\special{pa   356  -326}\special{pa   396  -354}\special{pa   435  -381}%
\special{pa   475  -405}\special{pa   515  -426}\special{pa   554  -444}\special{pa   594  -460}%
\special{pa   633  -472}\special{pa   673  -482}\special{pa   712  -488}\special{pa   752  -492}%
\special{pa   792  -492}\special{pa   831  -489}\special{pa   871  -482}\special{pa   910  -473}%
\special{pa   950  -461}\special{pa   989  -445}\special{pa  1029  -427}\special{pa  1069  -406}%
\special{pa  1108  -382}\special{pa  1148  -356}\special{pa  1187  -328}\special{pa  1227  -297}%
\special{pa  1267  -265}\special{pa  1306  -231}\special{pa  1346  -195}\special{pa  1385  -158}%
\special{pa  1425  -120}\special{pa  1464   -81}\special{pa  1504   -42}\special{pa  1544    -2}%
\special{pa  1583    37}\special{pa  1623    76}\special{pa  1662   115}\special{pa  1702   153}%
\special{pa  1741   190}\special{pa  1781   226}\special{pa  1821   261}\special{pa  1860   293}%
\special{pa  1900   324}\special{pa  1939   353}\special{pa  1979   379}%
\special{fp}%
\color[cmyk]{0,0,0,1}%
\special{pa   773   -20}\special{pa   773    20}%
\special{fp}%
\settowidth{\Width}{$\frac{\pi}{2}$}\setlength{\Width}{-0.5\Width}%
\settoheight{\Height}{$\frac{\pi}{2}$}\settodepth{\Depth}{$\frac{\pi}{2}$}\setlength{\Height}{\Depth}%
\put(1.5707960,0.0800000){\hspace*{\Width}\raisebox{\Height}{$\frac{\pi}{2}$}}%
%
%
\special{pa  1546   -20}\special{pa  1546    20}%
\special{fp}%
\settowidth{\Width}{$\pi$}\setlength{\Width}{-0.5\Width}%
\settoheight{\Height}{$\pi$}\settodepth{\Depth}{$\pi$}\setlength{\Height}{\Depth}%
\put(3.1415930,0.0800000){\hspace*{\Width}\raisebox{\Height}{$\pi$}}%
%
%
\special{pa  3092   -20}\special{pa  3092    20}%
\special{fp}%
\settowidth{\Width}{$2\pi$}\setlength{\Width}{-0.5\Width}%
\settoheight{\Height}{$2\pi$}\settodepth{\Depth}{$2\pi$}\setlength{\Height}{\Depth}%
\put(6.2831850,0.0800000){\hspace*{\Width}\raisebox{\Height}{$2\pi$}}%
%
%
\special{pa    20   492}\special{pa   -20   492}%
\special{fp}%
\settowidth{\Width}{$-1$}\setlength{\Width}{-1\Width}%
\settoheight{\Height}{$-1$}\settodepth{\Depth}{$-1$}\setlength{\Height}{-0.5\Height}\setlength{\Depth}{0.5\Depth}\addtolength{\Height}{\Depth}%
\put(-0.0800000,-1.0000000){\hspace*{\Width}\raisebox{\Height}{$-1$}}%
%
%
\special{pa    20  -492}\special{pa   -20  -492}%
\special{fp}%
\settowidth{\Width}{$1$}\setlength{\Width}{-1\Width}%
\settoheight{\Height}{$1$}\settodepth{\Depth}{$1$}\setlength{\Height}{-0.5\Height}\setlength{\Depth}{0.5\Depth}\addtolength{\Height}{\Depth}%
\put(-0.0800000,1.0000000){\hspace*{\Width}\raisebox{\Height}{$1$}}%
%
%
\special{pa -1083    -0}\special{pa  3445    -0}%
\special{fp}%
\special{pa     0   591}\special{pa     0  -591}%
\special{fp}%
\settowidth{\Width}{$x$}\setlength{\Width}{0\Width}%
\settoheight{\Height}{$x$}\settodepth{\Depth}{$x$}\setlength{\Height}{-0.5\Height}\setlength{\Depth}{0.5\Depth}\addtolength{\Height}{\Depth}%
\put(7.0400000,0.0000000){\hspace*{\Width}\raisebox{\Height}{$x$}}%
%
\settowidth{\Width}{$y$}\setlength{\Width}{-0.5\Width}%
\settoheight{\Height}{$y$}\settodepth{\Depth}{$y$}\setlength{\Height}{\Depth}%
\put(0.0000000,1.2400000){\hspace*{\Width}\raisebox{\Height}{$y$}}%
%
\settowidth{\Width}{O}\setlength{\Width}{0\Width}%
\settoheight{\Height}{O}\settodepth{\Depth}{O}\setlength{\Height}{-\Height}%
\put(0.0400000,-0.0400000){\hspace*{\Width}\raisebox{\Height}{O}}%
%
\end{picture}}%}}
\end{layer}

\begin{itemize}
\item
{\normalsize \url{s-takato.github.io/polytech/n107/doukansuumainoff.html}}
\item
微分係数$f'(a)$
\item
$a$を動かすと,$f'(a)$も変わる\\
\hspace*{2zw}$\Longrightarrow$\ $f'(a)$は$a$の関数
\item
$a$を$x$と置き換えて\\
$f'(x)$を$f(x)$の{\color{red}導関数}という
\item
導関数を求めることを「{\color{red}微分する}」
\end{itemize}

\newslide{導関数の定義式}

\vspace*{18mm}


\begin{layer}{120}{0}
\putnotew{96}{73}{\hyperlink{para15pg3}{\fbox{\Ctab{2.5mm}{\scalebox{1}{\scriptsize $\mathstrut||\!\lhd$}}}}}
\putnotew{101}{73}{\hyperlink{para16pg1}{\fbox{\Ctab{2.5mm}{\scalebox{1}{\scriptsize $\mathstrut|\!\lhd$}}}}}
\putnotew{108}{73}{\hyperlink{para16pg2}{\fbox{\Ctab{4.5mm}{\scalebox{1}{\scriptsize $\mathstrut\!\!\lhd\!\!$}}}}}
\putnotew{115}{73}{\hyperlink{para16pg3}{\fbox{\Ctab{4.5mm}{\scalebox{1}{\scriptsize $\mathstrut\!\rhd\!$}}}}}
\putnotew{120}{73}{\hyperlink{para16pg3}{\fbox{\Ctab{2.5mm}{\scalebox{1}{\scriptsize $\mathstrut \!\rhd\!\!|$}}}}}
\putnotew{125}{73}{\hyperlink{para17pg1}{\fbox{\Ctab{2.5mm}{\scalebox{1}{\scriptsize $\mathstrut \!\rhd\!\!||$}}}}}
\putnotee{126}{73}{\scriptsize\color{blue} 3/3}
\end{layer}

\slidepage
\begin{itemize}
\item
$f'(a)=\dlim_{z \to a}\bunsuu{f(z)-f(a)}{z-a}$
\item
$a$を$x$で置き換える\\
\hspace*{1zw}\fbox{$f'(x)=\dlim_{z \to x}\bunsuu{f(z)-f(x)}{z-x}$}
\end{itemize}

\newslide{導関数の意味(課題)}

\vspace*{18mm}


\begin{layer}{120}{0}
\putnotew{96}{73}{\hyperlink{para16pg3}{\fbox{\Ctab{2.5mm}{\scalebox{1}{\scriptsize $\mathstrut||\!\lhd$}}}}}
\putnotew{125}{73}{\hyperlink{para18pg1}{\fbox{\Ctab{2.5mm}{\scalebox{1}{\scriptsize $\mathstrut \!\rhd\!\!||$}}}}}
\putnotee{126}{73}{\scriptsize\color{blue} 1/1}
\end{layer}

\slidepage
\begin{itemize}
\item
[課題]\monban「導関数の意味」を実行して導関数を求めよ.\seteda{50}\\
\eda{$y=x^2-x$}\eda{$y=x^2-3$}\\
\eda{$y=x^3-x$}\eda{$y=x^3+2x^2+x$}
\end{itemize}
%%%%%%%%%%%%%

%%%%%%%%%%%%%%%%%%%%


\newslide{導関数の計算}

\vspace*{18mm}


\begin{layer}{120}{0}
\putnotew{96}{73}{\hyperlink{para17pg1}{\fbox{\Ctab{2.5mm}{\scalebox{1}{\scriptsize $\mathstrut||\!\lhd$}}}}}
\putnotew{101}{73}{\hyperlink{para18pg1}{\fbox{\Ctab{2.5mm}{\scalebox{1}{\scriptsize $\mathstrut|\!\lhd$}}}}}
\putnotew{108}{73}{\hyperlink{para18pg3}{\fbox{\Ctab{4.5mm}{\scalebox{1}{\scriptsize $\mathstrut\!\!\lhd\!\!$}}}}}
\putnotew{115}{73}{\hyperlink{para18pg4}{\fbox{\Ctab{4.5mm}{\scalebox{1}{\scriptsize $\mathstrut\!\rhd\!$}}}}}
\putnotew{120}{73}{\hyperlink{para18pg4}{\fbox{\Ctab{2.5mm}{\scalebox{1}{\scriptsize $\mathstrut \!\rhd\!\!|$}}}}}
\putnotew{125}{73}{\hyperlink{para19pg1}{\fbox{\Ctab{2.5mm}{\scalebox{1}{\scriptsize $\mathstrut \!\rhd\!\!||$}}}}}
\putnotee{126}{73}{\scriptsize\color{blue} 4/4}
\end{layer}

\slidepage
\begin{itemize}
\item
[例)]$f(x)=x^2$を微分する
\item
[]$f'(x)=\dlim_{z \to x}\bunsuu{f(z)-f(x)}{z-x}\\\phantom{f'(x)}=\lim_{z \to x}\bunsuu{z^2-x^2}{z-x}$\\
$\phantom{f'(x)}=\dlim_{z \to x}\bunsuu{(z+x)(z-x)}{z-x}\vspace{1mm}\\\phantom{f'(x)}=\lim_{z \to x}(z+x)=2x$\\
{\color{red}$(x^2)'=2x$}
\end{itemize}

\newslide{導関数の計算2}

\vspace*{18mm}


\begin{layer}{120}{0}
\putnotew{96}{73}{\hyperlink{para18pg4}{\fbox{\Ctab{2.5mm}{\scalebox{1}{\scriptsize $\mathstrut||\!\lhd$}}}}}
\putnotew{101}{73}{\hyperlink{para19pg1}{\fbox{\Ctab{2.5mm}{\scalebox{1}{\scriptsize $\mathstrut|\!\lhd$}}}}}
\putnotew{108}{73}{\hyperlink{para19pg2}{\fbox{\Ctab{4.5mm}{\scalebox{1}{\scriptsize $\mathstrut\!\!\lhd\!\!$}}}}}
\putnotew{115}{73}{\hyperlink{para19pg3}{\fbox{\Ctab{4.5mm}{\scalebox{1}{\scriptsize $\mathstrut\!\rhd\!$}}}}}
\putnotew{120}{73}{\hyperlink{para19pg3}{\fbox{\Ctab{2.5mm}{\scalebox{1}{\scriptsize $\mathstrut \!\rhd\!\!|$}}}}}
\putnotew{125}{73}{\hyperlink{para20pg1}{\fbox{\Ctab{2.5mm}{\scalebox{1}{\scriptsize $\mathstrut \!\rhd\!\!||$}}}}}
\putnotee{126}{73}{\scriptsize\color{blue} 3/3}
\end{layer}

\slidepage
\begin{itemize}
\item
[例)]$f(x)=x^3$を微分する
\item
因数分解$a^3-b^3=(a-b)(a^2+ab+b^2)$を用いる\\
{\color{blue}%
\hspace*{2zw}$a^3-b^3=a^2(a-b)+a^2b-b^3$\\
\hspace*{2zw}$\phantom{a^3-b^3}=(a-b)a^2+(a^2-b^2)b$\\
\hspace*{2zw}$\phantom{a^3-b^3}=(a-b)a^2+(a-b)(a+b)b$\\
\hspace*{2zw}$\phantom{a^3-b^3}=(a-b)\bigl(a^2+(a+b)b\bigr)$\\
\hspace*{2zw}$\phantom{a^3-b^3}=(a-b)(a^2+ab+b^2)$\\
}%
\end{itemize}

\newslide{導関数の計算2(続)}

\vspace*{18mm}


\begin{layer}{120}{0}
\putnotew{96}{73}{\hyperlink{para19pg3}{\fbox{\Ctab{2.5mm}{\scalebox{1}{\scriptsize $\mathstrut||\!\lhd$}}}}}
\putnotew{101}{73}{\hyperlink{para20pg1}{\fbox{\Ctab{2.5mm}{\scalebox{1}{\scriptsize $\mathstrut|\!\lhd$}}}}}
\putnotew{108}{73}{\hyperlink{para20pg4}{\fbox{\Ctab{4.5mm}{\scalebox{1}{\scriptsize $\mathstrut\!\!\lhd\!\!$}}}}}
\putnotew{115}{73}{\hyperlink{para20pg5}{\fbox{\Ctab{4.5mm}{\scalebox{1}{\scriptsize $\mathstrut\!\rhd\!$}}}}}
\putnotew{120}{73}{\hyperlink{para20pg5}{\fbox{\Ctab{2.5mm}{\scalebox{1}{\scriptsize $\mathstrut \!\rhd\!\!|$}}}}}
\putnotew{125}{73}{\hyperlink{para21pg1}{\fbox{\Ctab{2.5mm}{\scalebox{1}{\scriptsize $\mathstrut \!\rhd\!\!||$}}}}}
\putnotee{126}{73}{\scriptsize\color{blue} 5/5}
\end{layer}

\slidepage
\begin{itemize}
\item
[例)]$f(x)=x^3$
\item
因数分解$a^3-b^3=(a-b)(a^2+ab+b^2)$を用いる
\item
$f'(x)=\dlim_{z\to x}\bunsuu{z^3-x^3}{z-x}$
\item
[]$\phantom{f'(x)}=\dlim_{z\to x}\bunsuu{(z-x)(z^2+z x+x^2)}{z-x}$\vspace{2mm}\\
$\phantom{f'(x)}=\dlim_{z\to x}(z^2+z x+x^2)=3x^2$\\
{\color{red}$(x^3)'=3x^2$}
\end{itemize}

\newslide{導関数の定義式の別形}

\vspace*{18mm}


\begin{layer}{120}{0}
\putnotew{96}{73}{\hyperlink{para20pg5}{\fbox{\Ctab{2.5mm}{\scalebox{1}{\scriptsize $\mathstrut||\!\lhd$}}}}}
\putnotew{101}{73}{\hyperlink{para21pg1}{\fbox{\Ctab{2.5mm}{\scalebox{1}{\scriptsize $\mathstrut|\!\lhd$}}}}}
\putnotew{108}{73}{\hyperlink{para21pg3}{\fbox{\Ctab{4.5mm}{\scalebox{1}{\scriptsize $\mathstrut\!\!\lhd\!\!$}}}}}
\putnotew{115}{73}{\hyperlink{para21pg4}{\fbox{\Ctab{4.5mm}{\scalebox{1}{\scriptsize $\mathstrut\!\rhd\!$}}}}}
\putnotew{120}{73}{\hyperlink{para21pg4}{\fbox{\Ctab{2.5mm}{\scalebox{1}{\scriptsize $\mathstrut \!\rhd\!\!|$}}}}}
\putnotew{125}{73}{\hyperlink{para22pg1}{\fbox{\Ctab{2.5mm}{\scalebox{1}{\scriptsize $\mathstrut \!\rhd\!\!||$}}}}}
\putnotee{126}{73}{\scriptsize\color{blue} 4/4}
\end{layer}

\slidepage
\begin{itemize}
\item
$f'(x)=\dlim_{z \to x}\bunsuu{f(z)-f(x)}{z-x}$
$=\dlim_{\varDelta x \to 0}\bunsuu{\varDelta y}{\varDelta x}$
\item
$z-x=\varDelta x$とおく($x$の変化量でデルタ$x$と読む)
\item
$f(z)-f(x)=\varDelta y$とおく($y$の変化量)
\item
$z \to x$より $\varDelta x \to 0$
\item
$z=x+\varDelta x$より\\
\hspace*{2zw}$f'(x)=\dlim_{\varDelta x \to 0}\bunsuu{f(x+\varDelta x)-f(x)}{\varDelta x}$(教科書)
\end{itemize}

\newslide{微分の公式}

\vspace*{18mm}


\begin{layer}{120}{0}
\putnotew{96}{73}{\hyperlink{para21pg4}{\fbox{\Ctab{2.5mm}{\scalebox{1}{\scriptsize $\mathstrut||\!\lhd$}}}}}
\putnotew{101}{73}{\hyperlink{para22pg1}{\fbox{\Ctab{2.5mm}{\scalebox{1}{\scriptsize $\mathstrut|\!\lhd$}}}}}
\putnotew{108}{73}{\hyperlink{para22pg5}{\fbox{\Ctab{4.5mm}{\scalebox{1}{\scriptsize $\mathstrut\!\!\lhd\!\!$}}}}}
\putnotew{115}{73}{\hyperlink{para22pg6}{\fbox{\Ctab{4.5mm}{\scalebox{1}{\scriptsize $\mathstrut\!\rhd\!$}}}}}
\putnotew{120}{73}{\hyperlink{para22pg6}{\fbox{\Ctab{2.5mm}{\scalebox{1}{\scriptsize $\mathstrut \!\rhd\!\!|$}}}}}
\putnotew{125}{73}{\hyperlink{para23pg1}{\fbox{\Ctab{2.5mm}{\scalebox{1}{\scriptsize $\mathstrut \!\rhd\!\!||$}}}}}
\putnotee{126}{73}{\scriptsize\color{blue} 6/6}
\end{layer}

\slidepage

\begin{layer}{120}{0}
\putnotese{72}{5}{\scalebox{0.9}{%%% /Users/takatoosetsuo/Dropbox/2021polytech/108/fig/bibun4.tex 
%%% Generator=doukansuu2.cdy 
{\unitlength=7mm%
\begin{picture}%
(8,8)(-4,-4)%
\special{pn 8}%
%
\special{pn 12}%
\special{pa -1102  -413}\special{pa  1102  -413}%
\special{fp}%
\special{pn 8}%
\special{pa    20  -413}\special{pa   -20  -413}%
\special{fp}%
\settowidth{\Width}{$c$}\setlength{\Width}{-1\Width}%
\settoheight{\Height}{$c$}\settodepth{\Depth}{$c$}\setlength{\Height}{-\Height}%
\put(-0.0714286,1.4285714){\hspace*{\Width}\raisebox{\Height}{$c$}}%
%
\special{pa -1102    -0}\special{pa  1102    -0}%
\special{fp}%
\special{pa     0  1102}\special{pa     0 -1102}%
\special{fp}%
\settowidth{\Width}{$x$}\setlength{\Width}{0\Width}%
\settoheight{\Height}{$x$}\settodepth{\Depth}{$x$}\setlength{\Height}{-0.5\Height}\setlength{\Depth}{0.5\Depth}\addtolength{\Height}{\Depth}%
\put(4.0714286,0.0000000){\hspace*{\Width}\raisebox{\Height}{$x$}}%
%
\settowidth{\Width}{$y$}\setlength{\Width}{-0.5\Width}%
\settoheight{\Height}{$y$}\settodepth{\Depth}{$y$}\setlength{\Height}{\Depth}%
\put(0.0000000,4.0714286){\hspace*{\Width}\raisebox{\Height}{$y$}}%
%
\settowidth{\Width}{O}\setlength{\Width}{0\Width}%
\settoheight{\Height}{O}\settodepth{\Depth}{O}\setlength{\Height}{-\Height}%
\put(0.0714286,-0.0714286){\hspace*{\Width}\raisebox{\Height}{O}}%
%
\end{picture}}%}}
\end{layer}

\begin{itemize}
\item
定数関数$f(x)=c$($c$は定数)\\
\hspace*{1zw}$(c)'=0$
\item
$f(x)=x$\\
\hspace*{1zw}$(x)'=\dlim_{z \to x}\bunsuu{z-x}{z-x}=1$
\item
$(x^2)'=2x$
\item
$(x^3)'=3x^2$
\item
一般に $(x^n)'=\hakoa{$n x^{n-1}$}$
\end{itemize}

\newslide{微分の性質}

\vspace*{18mm}

\slidepage
\vspace{2mm}

\noindent
$f(x),\ g(x)$と定数$c$について
\begin{itemize}
\item
$(f+g)'=f'+g',\ (f-g)'=f'-g'$
\item
$(c f)'=c f'$
\item
[例]\hspace{-1mm})\ $(x^2+3x+4)'=(x^2)'+(3x)'+(4)'$
$=2x+3$
\item
[課題]\monbannoadd 次を微分せよ\hfill TextP7\seteda{58}\\
\eda{$y=3x^2+3x-3$}\eda{$y=2x^2-5x+4$}\\
\eda{$y=-4x^2+3x-2$}\eda{$y=\frac{5}{3}x^3+\frac{3}{4}x^2-\frac{1}{3}x$}\\
\end{itemize}
\addban

\newslide{導関数の書き方}

\vspace*{18mm}


\begin{layer}{120}{0}
\putnotew{96}{73}{\hyperlink{para22pg5}{\fbox{\Ctab{2.5mm}{\scalebox{1}{\scriptsize $\mathstrut||\!\lhd$}}}}}
\putnotew{101}{73}{\hyperlink{para23pg1}{\fbox{\Ctab{2.5mm}{\scalebox{1}{\scriptsize $\mathstrut|\!\lhd$}}}}}
\putnotew{108}{73}{\hyperlink{para23pg3}{\fbox{\Ctab{4.5mm}{\scalebox{1}{\scriptsize $\mathstrut\!\!\lhd\!\!$}}}}}
\putnotew{115}{73}{\hyperlink{para23pg4}{\fbox{\Ctab{4.5mm}{\scalebox{1}{\scriptsize $\mathstrut\!\rhd\!$}}}}}
\putnotew{120}{73}{\hyperlink{para23pg4}{\fbox{\Ctab{2.5mm}{\scalebox{1}{\scriptsize $\mathstrut \!\rhd\!\!|$}}}}}
\putnotew{125}{73}{\hyperlink{para24pg1}{\fbox{\Ctab{2.5mm}{\scalebox{1}{\scriptsize $\mathstrut \!\rhd\!\!||$}}}}}
\putnotee{126}{73}{\scriptsize\color{blue} 4/4}
\end{layer}

\slidepage
\begin{itemize}
\item
関数$y=f(x)$を変数$x$で微分する\\
\hspace*{1zw}$y',\ f'(x),\ f',\ \bigl(f(x)\bigr)'$(ラグランジュ)\vspace{2mm}\\
\hspace*{1zw}$\bunsuu{dy}{dx},\ \bunsuu{df}{dx},\ \bunsuu{d}{dx}\bigl(f(x)\bigr)$(ライプニッツ)
\item
[例]\hspace{-1mm})\ $y=f(x)=x^3$\\
\hspace*{1zw}$y'=f'(x)=f'=\bigl(x^3\bigr)'=3x^2$\vspace{2mm}\\
\hspace*{1zw}$\bunsuu{dy}{dx}=\bunsuu{df}{dx}=\bunsuu{d}{dx}f(x)=\bunsuu{d}{dx}(x^3)=3x^2$
\end{itemize}
\label{pageend}\mbox{}

\end{document}

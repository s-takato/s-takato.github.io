%%% タイトル presen22205
\documentclass[landscape,10pt]{jarticle}
\special{papersize=\the\paperwidth,\the\paperheight}
\usepackage{ketpic,ketlayer}
\usepackage{ketslide}
\usepackage{amsmath,amssymb}
\usepackage{bm,enumerate}
\usepackage[dvipdfmx]{graphicx}
\usepackage{color}
\definecolor{slidecolora}{cmyk}{0.98,0.13,0,0.43}
\definecolor{slidecolorb}{cmyk}{0.2,0,0,0}
\definecolor{slidecolorc}{cmyk}{0.2,0,0,0}
\definecolor{slidecolord}{cmyk}{0.2,0,0,0}
\definecolor{slidecolore}{cmyk}{0,0,0,0.5}
\definecolor{slidecolorf}{cmyk}{0,0,0,0.5}
\definecolor{slidecolori}{cmyk}{0.98,0.13,0,0.43}
\def\setthin#1{\def\thin{#1}}
\setthin{0}
\newcommand{\slidepage}[1][s]{%
\setcounter{ketpicctra}{18}%
\if#1m \setcounter{ketpicctra}{1}\fi
\hypersetup{linkcolor=black}%

\begin{layer}{118}{0}
\putnotee{122}{-\theketpicctra.05}{\small\thepage/\pageref{pageend}}
\end{layer}\hypersetup{linkcolor=blue}

}
\usepackage{emath}
\usepackage{pict2e}
\usepackage{ketlayermorewith2e}
\usepackage[dvipdfmx,colorlinks=true,linkcolor=blue,filecolor=blue]{hyperref}
\newcommand{\hiduke}{0905}
\newcommand{\hako}[2][1]{\fbox{\raisebox{#1mm}{\mbox{}}\raisebox{-#1mm}{\mbox{}}\,\phantom{#2}\,}}
\newcommand{\hakoa}[2][1]{\fbox{\raisebox{#1mm}{\mbox{}}\raisebox{-#1mm}{\mbox{}}\,#2\,}}
\newcommand{\hakom}[2][1]{\hako[#1]{$#2$}}
\newcommand{\hakoma}[2][1]{\hakoa[#1]{$#2$}}
\def\rad{\;\mathrm{rad}}
\def\deg#1{#1^{\circ}}
\newcommand{\sbunsuu}[2]{\scalebox{0.6}{$\bunsuu{#1}{#2}$}}
\def\pow{$\hspace{-1.5mm}^\hspace{-1mm}$}
\def\dlim{\displaystyle\lim}
\newcommand{\brd}[2][1]{\scalebox{#1}{\color{red}\fbox{\color{black}$#2$}}}
\newcommand\down[1][0.5zw]{\vspace{#1}\\}
\newcommand{\sfrac}[3][0.65]{\scalebox{#1}{$\frac{#2}{#3}$}}
\newcommand{\phn}[1]{\phantom{#1}}
\newcommand{\scb}[2][0.6]{\scalebox{#1}{#2}}
\newcommand{\dsum}{\displaystyle\sum}
\def\pow{$\hspace{-1.5mm}^\hspace{-1mm}$}
\def\dlim{\displaystyle\lim}
\def\dint{\displaystyle\int}

\setmargin{25}{145}{15}{100}

\ketslideinit

\pagestyle{empty}

\begin{document}

\begin{layer}{120}{0}
\putnotese{0}{0}{{\Large\bf
\color[cmyk]{1,1,0,0}

\begin{layer}{120}{0}
{\Huge \putnotes{60}{20}{定積分}}
\putnotes{60}{70}{2022.9.05}
\end{layer}

}
}
\end{layer}

\def\mainslidetitley{22}
\def\ketcletter{slidecolora}
\def\ketcbox{slidecolorb}
\def\ketdbox{slidecolorc}
\def\ketcframe{slidecolord}
\def\ketcshadow{slidecolore}
\def\ketdshadow{slidecolorf}
\def\slidetitlex{6}
\def\slidetitlesize{1.3}
\def\mketcletter{slidecolori}
\def\mketcbox{yellow}
\def\mketdbox{yellow}
\def\mketcframe{yellow}
\def\mslidetitlex{62}
\def\mslidetitlesize{2}

\color{black}
\Large\bf\boldmath
\addtocounter{page}{-1}

\def\MARU{}
\renewcommand{\MARU}[1]{{\ooalign{\hfil$#1$\/\hfil\crcr\raise.167ex\hbox{\mathhexbox20D}}}}
\renewcommand{\slidepage}[1][s]{%
\setcounter{ketpicctra}{18}%
\if#1m \setcounter{ketpicctra}{1}\fi
\hypersetup{linkcolor=black}%
\begin{layer}{118}{0}
\putnotee{115}{-\theketpicctra.05}{\small\hiduke-\thepage/\pageref{pageend}}
\end{layer}\hypersetup{linkcolor=blue}
}
\newcounter{ban}
\setcounter{ban}{1}
\newcommand{\monban}[1][\hiduke]{%
#1-\theban\ %
\addtocounter{ban}{1}%
}
\newcommand{\monbannoadd}[1][\hiduke]{%
#1-\theban\ %
}
\newcommand{\addban}{%
\addtocounter{ban}{1}%
}
\newcounter{edawidth}
\newcounter{edactr}
\newcommand{\seteda}[1]{% 20220708 modified
\setcounter{edawidth}{#1}
\setcounter{edactr}{1}
}
\newcommand{\eda}[2][\theedawidth]{%
\Ltab{#1 mm}{[\theedactr]\ #2}%
\addtocounter{edactr}{1}%
}
%%%%%%%%%%%%

%%%%%%%%%%%%%%%%%%%%

\mainslide{ 復習(微分と不定積分)}


\slidepage[m]
%%%%%%%%%%%%%

%%%%%%%%%%%%%%%%%%%%

\newslide{微分と不定積分}

\vspace*{18mm}


\begin{layer}{120}{0}
\putnotew{96}{73}{\hyperlink{para0pg0}{\fbox{\Ctab{2.5mm}{\scalebox{1}{\scriptsize $\mathstrut||\!\lhd$}}}}}
\putnotew{101}{73}{\hyperlink{para1pg1}{\fbox{\Ctab{2.5mm}{\scalebox{1}{\scriptsize $\mathstrut|\!\lhd$}}}}}
\putnotew{108}{73}{\hyperlink{para1pg6}{\fbox{\Ctab{4.5mm}{\scalebox{1}{\scriptsize $\mathstrut\!\!\lhd\!\!$}}}}}
\putnotew{115}{73}{\hyperlink{para1pg7}{\fbox{\Ctab{4.5mm}{\scalebox{1}{\scriptsize $\mathstrut\!\rhd\!$}}}}}
\putnotew{120}{73}{\hyperlink{para1pg7}{\fbox{\Ctab{2.5mm}{\scalebox{1}{\scriptsize $\mathstrut \!\rhd\!\!|$}}}}}
\putnotew{125}{73}{\hyperlink{para2pg1}{\fbox{\Ctab{2.5mm}{\scalebox{1}{\scriptsize $\mathstrut \!\rhd\!\!||$}}}}}
\putnotee{126}{73}{\scriptsize\color{blue} 7/7}
\end{layer}

\slidepage
\seteda{50}
\begin{itemize}
\item
微分\\
 関数の変化率(変化の割合)\\
    $(x^2)'=$
$2x$\\
 導関数の定義式\\
    $f'(x)=\dlim_{z\to x}\bunsuu{f(z)-f(x)}{z-x}$
\item
不定積分\\
 微分の逆(微分したら,そうなる関数)\\
    $\dint x^2 dx=$
$\bunsuu{1}{3}x^3+C$($C$は積分定数)
\end{itemize}

\newslide{不定積分(課題1)}

\vspace*{18mm}

%%repeat=7,para
\slidepage
\seteda{60}
\begin{itemize}
\item
[課題]\monban 次の\,\hakom{\bunsuu{1}{n}x^{n+1}}\,に公式を入れよ.\vspace{2mm}\\
\eda{$\dint 1\,dx=\hakom{\bunsuu{1}{n}x^{n+1}}+C$}\\%
\eda{$\dint x\,dx=\hakom{\bunsuu{1}{n}x^{n+1}}+C$}\\
\eda{$\dint x^2\,dx=\hakom{\bunsuu{1}{n}x^{n+1}}+C$}\\%
\eda{$\dint x^3\,dx=\hakom{\bunsuu{1}{n}x^{n+1}}+C$}\\
\end{itemize}
%%%%%%%%%%%%%

%%%%%%%%%%%%%%%%%%%%

\newslide{不定積分(課題2)}

\vspace*{18mm}

%%repeat=7,para
\slidepage
\begin{itemize}
\item
[課題]\monban 次の不定積分を求めよ.\seteda{50}\\
\eda{$\dint (x^2+4x)\,dx$}
\eda{$\dint (x^3-1)\,dx$}
\item
[課題]\monban 次の不定積分を求めよ.\seteda{50}\\
\eda{$\dint (x+1)^2\,dx$}
\eda{$\dint (x+1)(x+2)\,dx$}
\end{itemize}
%%%%%%%%%%%%%

%%%%%%%%%%%%%%%%%%%%

\mainslide{定積分}


%%%%%%%%%%%%%

%%%%%%%%%%%%%%%%%%%%

\newslide{$f(x)$の区間$[a,\ b]$での定積分}

\vspace*{18mm}


\begin{layer}{120}{0}
\putnotew{96}{73}{\hyperlink{para1pg7}{\fbox{\Ctab{2.5mm}{\scalebox{1}{\scriptsize $\mathstrut||\!\lhd$}}}}}
\putnotew{101}{73}{\hyperlink{para2pg1}{\fbox{\Ctab{2.5mm}{\scalebox{1}{\scriptsize $\mathstrut|\!\lhd$}}}}}
\putnotew{108}{73}{\hyperlink{para2pg1}{\fbox{\Ctab{4.5mm}{\scalebox{1}{\scriptsize $\mathstrut\!\!\lhd\!\!$}}}}}
\putnotew{115}{73}{\hyperlink{para2pg2}{\fbox{\Ctab{4.5mm}{\scalebox{1}{\scriptsize $\mathstrut\!\rhd\!$}}}}}
\putnotew{120}{73}{\hyperlink{para2pg2}{\fbox{\Ctab{2.5mm}{\scalebox{1}{\scriptsize $\mathstrut \!\rhd\!\!|$}}}}}
\putnotew{125}{73}{\hyperlink{para3pg1}{\fbox{\Ctab{2.5mm}{\scalebox{1}{\scriptsize $\mathstrut \!\rhd\!\!||$}}}}}
\putnotee{126}{73}{\scriptsize\color{blue} 2/2}
\end{layer}

\slidepage
\vspace{5mm}

\begin{layer}{120}{0}
\putnotese{65}{17}{%%% /Users/takatoosetsuo/Dropbox/2021polytech/204/fig/teisekiteigi1.tex 
%%% Generator=teiseki0723.cdy 
{\unitlength=1cm%
\begin{picture}%
(6,5)(-2,-1)%
\linethickness{0.008in}%%
\normalsize%
\linethickness{0.012in}%%
\polyline(-2.00000,1.10124)(-1.96203,1.14456)(-1.79969,1.30563)(-1.63161,1.44907)%
(-1.45780,1.57487)(-1.27824,1.68305)(-1.09295,1.77359)(-0.90192,1.84651)(-0.70516,1.90179)%
(-0.40093,1.94287)(-0.08815,1.93548)(0.23169,1.89277)(0.55709,1.82790)(0.88657,1.75400)%
(1.21863,1.68423)(1.55178,1.63175)(1.88453,1.60969)(2.21538,1.63122)(2.54284,1.70947)%
(2.75937,1.79637)(2.97527,1.90908)(3.19053,2.04758)(3.40516,2.21188)(3.61916,2.40198)%
(3.83253,2.61788)(4.00000,2.80815)%
%
\linethickness{0.008in}%%
\polyline(-0.70516,0.00000)(-0.70516,1.90179)%
%
\polyline(2.54284,1.70947)(2.54284,0.00000)%
%
\polyline(4.00000,0.00000)(-2.00000,0.00000)%
%
\polyline(2.32843,-0.00000)(2.54284,0.21441)%
%
\polyline(1.97487,0.00000)(2.54284,0.56797)%
%
\polyline(1.62132,0.00000)(2.54284,0.92152)%
%
\polyline(1.26777,-0.00000)(2.54284,1.27507)%
%
\polyline(0.91421,0.00000)(2.54284,1.62863)%
%
\polyline(0.56066,-0.00000)(2.19024,1.62958)%
%
\polyline(0.20711,-0.00000)(1.82101,1.61390)%
%
\polyline(-0.14645,0.00000)(1.49435,1.64080)%
%
\polyline(-0.50000,0.00000)(1.19020,1.69020)%
%
\polyline(-0.70516,0.14839)(0.89804,1.75159)%
%
\polyline(-0.70516,0.50195)(0.60912,1.81623)%
%
\polyline(-0.70516,0.85550)(0.31542,1.87608)%
%
\polyline(-0.70516,1.20905)(0.00838,1.92259)%
%
\polyline(-0.70516,1.56261)(-0.32665,1.94112)%
%
\settowidth{\Width}{$y=f(x)$}\setlength{\Width}{0\Width}%
\settoheight{\Height}{$y=f(x)$}\settodepth{\Depth}{$y=f(x)$}\setlength{\Height}{-0.5\Height}\setlength{\Depth}{0.5\Depth}\addtolength{\Height}{\Depth}%
\put(3.0900000,1.7100000){\hspace*{\Width}\raisebox{\Height}{$y=f(x)$}}%
%
\polyline(-0.70516,0.05000)(-0.70516,-0.05000)%
%
\settowidth{\Width}{$a$}\setlength{\Width}{-0.5\Width}%
\settoheight{\Height}{$a$}\settodepth{\Depth}{$a$}\setlength{\Height}{-\Height}%
\put(-0.7100000,-0.1000000){\hspace*{\Width}\raisebox{\Height}{$a$}}%
%
\polyline(2.54284,0.05000)(2.54284,-0.05000)%
%
\settowidth{\Width}{$b$}\setlength{\Width}{-0.5\Width}%
\settoheight{\Height}{$b$}\settodepth{\Depth}{$b$}\setlength{\Height}{-\Height}%
\put(2.5400000,-0.1000000){\hspace*{\Width}\raisebox{\Height}{$b$}}%
%
\polyline(-2.00000,0.00000)(4.00000,0.00000)%
%
\polyline(0.00000,-1.00000)(0.00000,4.00000)%
%
\settowidth{\Width}{$x$}\setlength{\Width}{0\Width}%
\settoheight{\Height}{$x$}\settodepth{\Depth}{$x$}\setlength{\Height}{-0.5\Height}\setlength{\Depth}{0.5\Depth}\addtolength{\Height}{\Depth}%
\put(4.0500000,0.0000000){\hspace*{\Width}\raisebox{\Height}{$x$}}%
%
\settowidth{\Width}{$y$}\setlength{\Width}{-0.5\Width}%
\settoheight{\Height}{$y$}\settodepth{\Depth}{$y$}\setlength{\Height}{\Depth}%
\put(0.0000000,4.0500000){\hspace*{\Width}\raisebox{\Height}{$y$}}%
%
\settowidth{\Width}{O}\setlength{\Width}{-1\Width}%
\settoheight{\Height}{O}\settodepth{\Depth}{O}\setlength{\Height}{-\Height}%
\put(-0.0500000,-0.0500000){\hspace*{\Width}\raisebox{\Height}{O}}%
%
\end{picture}}%}
\end{layer}

\begin{itemize}
\item
しばらく,$f(x)\geqq 0$とする.
\item
\begin{minipage}[t]{65mm}
$y=f(x)$と$x$軸と$x=a$と$x=b$で囲まれた部分の面積$S$を
$[a,\ b]$での{\color{red}定積分}といい\vspace{2mm}\\
$\displaystyle\int_a^b f(x)\,dx$と書く
\end{minipage}
\end{itemize}

\newslide{簡単な定積分の例}

\vspace*{18mm}


\begin{layer}{120}{0}
\putnotew{96}{73}{\hyperlink{para2pg2}{\fbox{\Ctab{2.5mm}{\scalebox{1}{\scriptsize $\mathstrut||\!\lhd$}}}}}
\putnotew{101}{73}{\hyperlink{para3pg1}{\fbox{\Ctab{2.5mm}{\scalebox{1}{\scriptsize $\mathstrut|\!\lhd$}}}}}
\putnotew{108}{73}{\hyperlink{para3pg4}{\fbox{\Ctab{4.5mm}{\scalebox{1}{\scriptsize $\mathstrut\!\!\lhd\!\!$}}}}}
\putnotew{115}{73}{\hyperlink{para3pg5}{\fbox{\Ctab{4.5mm}{\scalebox{1}{\scriptsize $\mathstrut\!\rhd\!$}}}}}
\putnotew{120}{73}{\hyperlink{para3pg5}{\fbox{\Ctab{2.5mm}{\scalebox{1}{\scriptsize $\mathstrut \!\rhd\!\!|$}}}}}
\putnotew{125}{73}{\hyperlink{para4pg1}{\fbox{\Ctab{2.5mm}{\scalebox{1}{\scriptsize $\mathstrut \!\rhd\!\!||$}}}}}
\putnotee{126}{73}{\scriptsize\color{blue} 5/5}
\end{layer}

\slidepage

\begin{layer}{120}{0}
\putnotese{65}{0}{%%% /Users/takatoosetsuo/Dropbox/2021polytech/204/fig/teisekiteigi2.tex 
%%% Generator=teiseki0723.cdy 
{\unitlength=1cm%
\begin{picture}%
(6,3)(-2,-1)%
\linethickness{0.008in}%%
\normalsize%
\linethickness{0.012in}%%
\polyline(-2.00000,1.00000)(4.00000,1.00000)%
%
\linethickness{0.008in}%%
\polyline(1.00000,0.00000)(1.00000,1.00000)%
%
\polyline(3.00000,1.00000)(3.00000,0.00000)%
%
\polyline(4.00000,0.00000)(-2.00000,0.00000)%
%
\polyline(2.97487,0.00000)(3.00000,0.02513)%
%
\polyline(2.62132,0.00000)(3.00000,0.37868)%
%
\polyline(2.26777,0.00000)(3.00000,0.73223)%
%
\polyline(1.91421,-0.00000)(2.91421,1.00000)%
%
\polyline(1.56066,0.00000)(2.56066,1.00000)%
%
\polyline(1.20711,0.00000)(2.20711,1.00000)%
%
\polyline(1.00000,0.14645)(1.85355,1.00000)%
%
\polyline(1.00000,0.50000)(1.50000,1.00000)%
%
\polyline(1.00000,0.85355)(1.14645,1.00000)%
%
\polyline(1.00000,0.05000)(1.00000,-0.05000)%
%
\settowidth{\Width}{$1$}\setlength{\Width}{-0.5\Width}%
\settoheight{\Height}{$1$}\settodepth{\Depth}{$1$}\setlength{\Height}{-\Height}%
\put(1.0000000,-0.1000000){\hspace*{\Width}\raisebox{\Height}{$1$}}%
%
\polyline(3.00000,0.05000)(3.00000,-0.05000)%
%
\settowidth{\Width}{$3$}\setlength{\Width}{-0.5\Width}%
\settoheight{\Height}{$3$}\settodepth{\Depth}{$3$}\setlength{\Height}{-\Height}%
\put(3.0000000,-0.1000000){\hspace*{\Width}\raisebox{\Height}{$3$}}%
%
\polyline(0.05000,1.00000)(-0.05000,1.00000)%
%
\settowidth{\Width}{$1$}\setlength{\Width}{-1\Width}%
\settoheight{\Height}{$1$}\settodepth{\Depth}{$1$}\setlength{\Height}{\Depth}%
\put(-0.0500000,1.0500000){\hspace*{\Width}\raisebox{\Height}{$1$}}%
%
\settowidth{\Width}{$y=1$}\setlength{\Width}{0\Width}%
\settoheight{\Height}{$y=1$}\settodepth{\Depth}{$y=1$}\setlength{\Height}{\Depth}%
\put(3.2500000,1.1500000){\hspace*{\Width}\raisebox{\Height}{$y=1$}}%
%
\polyline(-2.00000,0.00000)(4.00000,0.00000)%
%
\polyline(0.00000,-1.00000)(0.00000,2.00000)%
%
\settowidth{\Width}{$x$}\setlength{\Width}{0\Width}%
\settoheight{\Height}{$x$}\settodepth{\Depth}{$x$}\setlength{\Height}{-0.5\Height}\setlength{\Depth}{0.5\Depth}\addtolength{\Height}{\Depth}%
\put(4.0500000,0.0000000){\hspace*{\Width}\raisebox{\Height}{$x$}}%
%
\settowidth{\Width}{$y$}\setlength{\Width}{-0.5\Width}%
\settoheight{\Height}{$y$}\settodepth{\Depth}{$y$}\setlength{\Height}{\Depth}%
\put(0.0000000,2.0500000){\hspace*{\Width}\raisebox{\Height}{$y$}}%
%
\settowidth{\Width}{O}\setlength{\Width}{-1\Width}%
\settoheight{\Height}{O}\settodepth{\Depth}{O}\setlength{\Height}{-\Height}%
\put(-0.0500000,-0.0500000){\hspace*{\Width}\raisebox{\Height}{O}}%
%
\end{picture}}%}
\putnotese{65}{35}{%%% /Users/takatoosetsuo/Dropbox/2021polytech/204/fig/teisekiteigi3.tex 
%%% Generator=teiseki0723.cdy 
{\unitlength=1cm%
\begin{picture}%
(6,4)(-2,-1)%
\linethickness{0.008in}%%
\normalsize%
\linethickness{0.012in}%%
\polyline(-1.00000,-1.00000)(3.00000,3.00000)%
%
\linethickness{0.008in}%%
\polyline(2.00000,2.00000)(2.00000,0.00000)%
%
\polyline(4.00000,0.00000)(-2.00000,0.00000)%
%
\polyline(1.76777,0.00000)(2.00000,0.23223)%
%
\polyline(1.41421,0.00000)(2.00000,0.58579)%
%
\polyline(1.06066,0.00000)(2.00000,0.93934)%
%
\polyline(0.70711,0.00000)(2.00000,1.29289)%
%
\polyline(0.35355,0.00000)(2.00000,1.64645)%
%
\polyline(2.00000,0.05000)(2.00000,-0.05000)%
%
\settowidth{\Width}{$2$}\setlength{\Width}{-0.5\Width}%
\settoheight{\Height}{$2$}\settodepth{\Depth}{$2$}\setlength{\Height}{-\Height}%
\put(2.0000000,-0.1000000){\hspace*{\Width}\raisebox{\Height}{$2$}}%
%
\polyline(0.05000,2.00000)(-0.05000,2.00000)%
%
\settowidth{\Width}{$2$}\setlength{\Width}{-1\Width}%
\settoheight{\Height}{$2$}\settodepth{\Depth}{$2$}\setlength{\Height}{-0.5\Height}\setlength{\Depth}{0.5\Depth}\addtolength{\Height}{\Depth}%
\put(-0.1000000,2.0000000){\hspace*{\Width}\raisebox{\Height}{$2$}}%
%
\settowidth{\Width}{$y=x$}\setlength{\Width}{0\Width}%
\settoheight{\Height}{$y=x$}\settodepth{\Depth}{$y=x$}\setlength{\Height}{-\Height}%
\put(2.5500000,2.4500000){\hspace*{\Width}\raisebox{\Height}{$y=x$}}%
%
\polyline(-2.00000,0.00000)(4.00000,0.00000)%
%
\polyline(0.00000,-1.00000)(0.00000,3.00000)%
%
\settowidth{\Width}{$x$}\setlength{\Width}{0\Width}%
\settoheight{\Height}{$x$}\settodepth{\Depth}{$x$}\setlength{\Height}{-0.5\Height}\setlength{\Depth}{0.5\Depth}\addtolength{\Height}{\Depth}%
\put(4.0500000,0.0000000){\hspace*{\Width}\raisebox{\Height}{$x$}}%
%
\settowidth{\Width}{$y$}\setlength{\Width}{-0.5\Width}%
\settoheight{\Height}{$y$}\settodepth{\Depth}{$y$}\setlength{\Height}{\Depth}%
\put(0.0000000,3.0500000){\hspace*{\Width}\raisebox{\Height}{$y$}}%
%
\settowidth{\Width}{O}\setlength{\Width}{-1\Width}%
\settoheight{\Height}{O}\settodepth{\Depth}{O}\setlength{\Height}{-\Height}%
\put(-0.0500000,-0.0500000){\hspace*{\Width}\raisebox{\Height}{O}}%
%
\end{picture}}%}
\end{layer}

\seteda{50}
\begin{enumerate}[(1)]
\item
$\dint_1^3 1\,dx=\hakoma{2}$\vspace{2mm}
\item
$\dint_0^2 x\,dx=\hakoma{2}$\vspace{2mm}
\item
[課題]\monban 次の値を求めよ.\\
\eda{$\dint_0^3 1\,dx=$}\\
\eda{$\dint_0^1 x\,dx=$}
\end{enumerate}

\newslide{定積分と不定積分}

\vspace*{18mm}


\begin{layer}{120}{0}
\putnotew{96}{73}{\hyperlink{para3pg5}{\fbox{\Ctab{2.5mm}{\scalebox{1}{\scriptsize $\mathstrut||\!\lhd$}}}}}
\putnotew{101}{73}{\hyperlink{para4pg1}{\fbox{\Ctab{2.5mm}{\scalebox{1}{\scriptsize $\mathstrut|\!\lhd$}}}}}
\putnotew{108}{73}{\hyperlink{para4pg3}{\fbox{\Ctab{4.5mm}{\scalebox{1}{\scriptsize $\mathstrut\!\!\lhd\!\!$}}}}}
\putnotew{115}{73}{\hyperlink{para4pg4}{\fbox{\Ctab{4.5mm}{\scalebox{1}{\scriptsize $\mathstrut\!\rhd\!$}}}}}
\putnotew{120}{73}{\hyperlink{para4pg4}{\fbox{\Ctab{2.5mm}{\scalebox{1}{\scriptsize $\mathstrut \!\rhd\!\!|$}}}}}
\putnotew{125}{73}{\hyperlink{para5pg1}{\fbox{\Ctab{2.5mm}{\scalebox{1}{\scriptsize $\mathstrut \!\rhd\!\!||$}}}}}
\putnotee{126}{73}{\scriptsize\color{blue} 4/4}
\end{layer}

\slidepage

\begin{layer}{120}{0}
\putnotese{65}{4}{%%% /Users/takatoosetsuo/Dropbox/2018polytec/lecture/0720/presen/fig/teisekiteigi4.tex 
%%% Generator=presen0720.cdy 
{\unitlength=1cm%
\begin{picture}%
(6,4)(-2,-1)%
\special{pn 8}%
%
\Large\bf\boldmath%
\small%
\special{pn 12}%
\special{pa  -394   394}\special{pa  1181 -1181}%
\special{fp}%
\special{pn 8}%
{%
\color[rgb]{0,0,0}%
\special{pa   984  -984}\special{pa   984    -0}%
\special{fp}%
}%
{%
\color[rgb]{0,0,0}%
\special{pa  1575    -0}\special{pa  -787    -0}%
\special{fp}%
}%
{%
\color[rgb]{0,0,0}%
\special{pa 984 -984}\special{pa 945 -984}\special{fp}\special{pa 906 -984}\special{pa 866 -984}\special{fp}%
\special{pa 827 -984}\special{pa 787 -984}\special{fp}\special{pa 748 -984}\special{pa 709 -984}\special{fp}%
\special{pa 669 -984}\special{pa 630 -984}\special{fp}\special{pa 591 -984}\special{pa 551 -984}\special{fp}%
\special{pa 512 -984}\special{pa 472 -984}\special{fp}\special{pa 433 -984}\special{pa 394 -984}\special{fp}%
\special{pa 354 -984}\special{pa 315 -984}\special{fp}\special{pa 276 -984}\special{pa 236 -984}\special{fp}%
\special{pa 197 -984}\special{pa 157 -984}\special{fp}\special{pa 118 -984}\special{pa 79 -984}\special{fp}%
\special{pa 39 -984}\special{pa 0 -984}\special{fp}%
%
}%
\special{pa   974    -0}\special{pa   984   -10}%
\special{fp}%
\special{pa   835    -0}\special{pa   984  -149}%
\special{fp}%
\special{pa   696    -0}\special{pa   984  -288}%
\special{fp}%
\special{pa   557    -0}\special{pa   984  -427}%
\special{fp}%
\special{pa   418    -0}\special{pa   984  -567}%
\special{fp}%
\special{pa   278    -0}\special{pa   984  -706}%
\special{fp}%
\special{pa   139     0}\special{pa   984  -845}%
\special{fp}%
{%
\color[rgb]{0,0,0}%
\special{pa   984   -20}\special{pa   984    20}%
\special{fp}%
}%
\settowidth{\Width}{$x$}\setlength{\Width}{-0.5\Width}%
\settoheight{\Height}{$x$}\settodepth{\Depth}{$x$}\setlength{\Height}{-\Height}%
\put(2.5000000,-0.1000000){\hspace*{\Width}\raisebox{\Height}{$x$}}%
%
{%
\color[rgb]{0,0,0}%
\special{pa    20  -984}\special{pa   -20  -984}%
\special{fp}%
}%
\settowidth{\Width}{$x$}\setlength{\Width}{-1\Width}%
\settoheight{\Height}{$x$}\settodepth{\Depth}{$x$}\setlength{\Height}{-0.5\Height}\setlength{\Depth}{0.5\Depth}\addtolength{\Height}{\Depth}%
\put(-0.1000000,2.5000000){\hspace*{\Width}\raisebox{\Height}{$x$}}%
%
\special{pa  -787    -0}\special{pa  1575    -0}%
\special{fp}%
\special{pa     0   394}\special{pa     0 -1181}%
\special{fp}%
\settowidth{\Width}{$x$}\setlength{\Width}{0\Width}%
\settoheight{\Height}{$x$}\settodepth{\Depth}{$x$}\setlength{\Height}{-0.5\Height}\setlength{\Depth}{0.5\Depth}\addtolength{\Height}{\Depth}%
\put(4.0500000,0.0000000){\hspace*{\Width}\raisebox{\Height}{$x$}}%
%
\settowidth{\Width}{$y$}\setlength{\Width}{-0.5\Width}%
\settoheight{\Height}{$y$}\settodepth{\Depth}{$y$}\setlength{\Height}{\Depth}%
\put(0.0000000,3.0500000){\hspace*{\Width}\raisebox{\Height}{$y$}}%
%
\settowidth{\Width}{O}\setlength{\Width}{-1\Width}%
\settoheight{\Height}{O}\settodepth{\Depth}{O}\setlength{\Height}{-\Height}%
\put(-0.0500000,-0.0500000){\hspace*{\Width}\raisebox{\Height}{O}}%
%
\end{picture}}%}
\end{layer}

\begin{itemize}
\item
$\displaystyle\int_0^2 x\,dx=2$
\item
右端の値を変化させる\\
それを$x$と書く\vspace{1mm}\\
\hspace*{1.5zw}$\dint_0^x x\,dx=\hakoma{\bunsuu{1}{2}x^2}$\vspace{-1mm}
\item
[注)]積分の中の$x$と右端の$x$が重なるが\\
積分の中の$x$は気にせず,右端の$x$の関数と考える.
\end{itemize}

\newslide{定積分と不定積分2}

\vspace*{18mm}


\begin{layer}{120}{0}
\putnotew{96}{73}{\hyperlink{para4pg4}{\fbox{\Ctab{2.5mm}{\scalebox{1}{\scriptsize $\mathstrut||\!\lhd$}}}}}
\putnotew{101}{73}{\hyperlink{para5pg1}{\fbox{\Ctab{2.5mm}{\scalebox{1}{\scriptsize $\mathstrut|\!\lhd$}}}}}
\putnotew{108}{73}{\hyperlink{para5pg3}{\fbox{\Ctab{4.5mm}{\scalebox{1}{\scriptsize $\mathstrut\!\!\lhd\!\!$}}}}}
\putnotew{115}{73}{\hyperlink{para5pg4}{\fbox{\Ctab{4.5mm}{\scalebox{1}{\scriptsize $\mathstrut\!\rhd\!$}}}}}
\putnotew{120}{73}{\hyperlink{para5pg4}{\fbox{\Ctab{2.5mm}{\scalebox{1}{\scriptsize $\mathstrut \!\rhd\!\!|$}}}}}
\putnotew{125}{73}{\hyperlink{para6pg1}{\fbox{\Ctab{2.5mm}{\scalebox{1}{\scriptsize $\mathstrut \!\rhd\!\!||$}}}}}
\putnotee{126}{73}{\scriptsize\color{blue} 4/4}
\end{layer}

\slidepage

\begin{layer}{120}{0}
\end{layer}

\begin{enumerate}[(1)]
\item
$\displaystyle\int_0^x x\,dx=\bunsuu{x^2}{2}$
\item
一方,不定積分では$\dint x\,dx=\bunsuu{x^2}{2}+C$
\item
定積分(1)は不定積分(2)の1つ\vspace{2mm}\\
\hspace*{3zw}\fbox{\color{red}$\left(\dint_0^x x\,dx\right)'=x$}
\end{enumerate}

\newslide{微分積分の基本定理}

\vspace*{18mm}


\begin{layer}{120}{0}
\putnotew{96}{73}{\hyperlink{para5pg4}{\fbox{\Ctab{2.5mm}{\scalebox{1}{\scriptsize $\mathstrut||\!\lhd$}}}}}
\putnotew{101}{73}{\hyperlink{para6pg1}{\fbox{\Ctab{2.5mm}{\scalebox{1}{\scriptsize $\mathstrut|\!\lhd$}}}}}
\putnotew{108}{73}{\hyperlink{para6pg4}{\fbox{\Ctab{4.5mm}{\scalebox{1}{\scriptsize $\mathstrut\!\!\lhd\!\!$}}}}}
\putnotew{115}{73}{\hyperlink{para6pg5}{\fbox{\Ctab{4.5mm}{\scalebox{1}{\scriptsize $\mathstrut\!\rhd\!$}}}}}
\putnotew{120}{73}{\hyperlink{para6pg5}{\fbox{\Ctab{2.5mm}{\scalebox{1}{\scriptsize $\mathstrut \!\rhd\!\!|$}}}}}
\putnotew{125}{73}{\hyperlink{para7pg1}{\fbox{\Ctab{2.5mm}{\scalebox{1}{\scriptsize $\mathstrut \!\rhd\!\!||$}}}}}
\putnotee{126}{73}{\scriptsize\color{blue} 5/5}
\end{layer}

\slidepage
{\color{red}

\begin{layer}{120}{0}
\putnotec{22}{10.5}{\circlemark{1}}
\putnotec{46}{10.5}{\circlemark{1}}
\putnotec{19}{6}{\color{blue}\fbox{\phantom{\scriptsize x}}}
\putnotec{31}{23}{\color{blue}\fbox{\phantom{\small x}}}
\end{layer}

}
\begin{itemize}
\item
$\left(\dint_0^x x\,dx\right)'=x$だった
\item
[] $0$から$x$までの定積分を微分すると元の関数になる
\item
これは,一般の関数$f(x)$についても成り立つ\vspace{2mm}\\
\hspace*{3zw}\fbox{\color{red}$\left(\dint_a^x f(x)\,dx\right)'=f(x)$}\hspace{1zw}($a$は定数)
\item
微分積分において,最も重要な定理である
\end{itemize}

\newslide{基本定理の証明}

\vspace*{18mm}


\begin{layer}{120}{0}
\putnotew{96}{73}{\hyperlink{para6pg5}{\fbox{\Ctab{2.5mm}{\scalebox{1}{\scriptsize $\mathstrut||\!\lhd$}}}}}
\putnotew{101}{73}{\hyperlink{para7pg1}{\fbox{\Ctab{2.5mm}{\scalebox{1}{\scriptsize $\mathstrut|\!\lhd$}}}}}
\putnotew{108}{73}{\hyperlink{para7pg7}{\fbox{\Ctab{4.5mm}{\scalebox{1}{\scriptsize $\mathstrut\!\!\lhd\!\!$}}}}}
\putnotew{115}{73}{\hyperlink{para7pg8}{\fbox{\Ctab{4.5mm}{\scalebox{1}{\scriptsize $\mathstrut\!\rhd\!$}}}}}
\putnotew{120}{73}{\hyperlink{para7pg8}{\fbox{\Ctab{2.5mm}{\scalebox{1}{\scriptsize $\mathstrut \!\rhd\!\!|$}}}}}
\putnotew{125}{73}{\hyperlink{para8pg1}{\fbox{\Ctab{2.5mm}{\scalebox{1}{\scriptsize $\mathstrut \!\rhd\!\!||$}}}}}
\putnotee{126}{73}{\scriptsize\color{blue} 8/8}
\end{layer}

\slidepage

\begin{layer}{120}{0}
\putnotese{80}{25}{\scalebox{0.85}{%%% /Users/takatoosetsuo/Dropbox/2021polytech/204/fig/kihon3.tex 
%%% Generator=kihon0720.cdy 
{\unitlength=1cm%
\begin{picture}%
(5.5,4)(-1,-1)%
\linethickness{0.008in}%%
\normalsize%
\polyline(-1.00000,1.00957)(-0.82950,1.13936)(-0.62123,1.29368)(-0.42499,1.43466)%
(-0.24077,1.56227)(-0.06857,1.67653)(0.09161,1.77743)(0.23976,1.86498)(0.37589,1.93917)%
(0.50000,2.00000)(0.68016,2.08086)(0.86641,2.16002)(1.05884,2.23610)(1.25752,2.30768)%
(1.46253,2.37336)(1.67397,2.43175)(1.89191,2.48142)(2.11644,2.52100)(2.34763,2.54906)%
(2.58558,2.56421)(2.76222,2.56530)(2.97037,2.55760)(3.21001,2.54111)(3.48116,2.51583)%
(3.78380,2.48176)(4.11793,2.43890)(4.48357,2.38725)(4.50000,2.38475)%
%
\polyline(0.50000,2.00000)(0.50000,0.00000)%
%
\polyline(3.29073,2.53358)(3.29073,0.00000)%
%
\polyline(3.69673,2.49156)(3.69673,0.00000)%
%
{%
\color[cmyk]{0,1,1,0}%
\polyline(3.29073,2.53358)(3.29073,0.00000)(3.69673,0.00000)(3.69673,2.49156)(3.48116,2.51583)%
(3.29073,2.53358)%
%
}%
\settowidth{\Width}{$y=f(x)$}\setlength{\Width}{-0.5\Width}%
\settoheight{\Height}{$y=f(x)$}\settodepth{\Depth}{$y=f(x)$}\setlength{\Height}{\Depth}%
\put(2.5900000,2.6600000){\hspace*{\Width}\raisebox{\Height}{$y=f(x)$}}%
%
{%
\color[cmyk]{0,1,1,0}%
\polyline(3.29073,0.16638)(3.45711,0.00000)%
%
\polyline(3.29073,0.51993)(3.69673,0.11393)%
%
\polyline(3.29073,0.87348)(3.69673,0.46748)%
%
\polyline(3.29073,1.22704)(3.69673,0.82104)%
%
\polyline(3.29073,1.58059)(3.69673,1.17459)%
%
\polyline(3.29073,1.93414)(3.69673,1.52814)%
%
\polyline(3.29073,2.28770)(3.69673,1.88170)%
%
\polyline(3.40947,2.52251)(3.69673,2.23525)%
%
}%
\polyline(0.50000,0.05000)(0.50000,-0.05000)%
%
\settowidth{\Width}{$a$}\setlength{\Width}{-0.5\Width}%
\settoheight{\Height}{$a$}\settodepth{\Depth}{$a$}\setlength{\Height}{-\Height}%
\put(0.5000000,-0.1000000){\hspace*{\Width}\raisebox{\Height}{$a$}}%
%
\polyline(3.29073,0.05000)(3.29073,-0.05000)%
%
\settowidth{\Width}{$x$}\setlength{\Width}{-0.5\Width}%
\settoheight{\Height}{$x$}\settodepth{\Depth}{$x$}\setlength{\Height}{-\Height}%
\put(3.2900000,-0.1000000){\hspace*{\Width}\raisebox{\Height}{$x$}}%
%
\polyline(3.69673,0.05000)(3.69673,-0.05000)%
%
\settowidth{\Width}{$z$}\setlength{\Width}{-0.5\Width}%
\settoheight{\Height}{$z$}\settodepth{\Depth}{$z$}\setlength{\Height}{-\Height}%
\put(3.7000000,-0.1000000){\hspace*{\Width}\raisebox{\Height}{$z$}}%
%
\polyline(-1.00000,0.00000)(4.50000,0.00000)%
%
\polyline(0.00000,-1.00000)(0.00000,3.00000)%
%
\settowidth{\Width}{$x$}\setlength{\Width}{0\Width}%
\settoheight{\Height}{$x$}\settodepth{\Depth}{$x$}\setlength{\Height}{-0.5\Height}\setlength{\Depth}{0.5\Depth}\addtolength{\Height}{\Depth}%
\put(4.5500000,0.0000000){\hspace*{\Width}\raisebox{\Height}{$x$}}%
%
\settowidth{\Width}{$y$}\setlength{\Width}{-0.5\Width}%
\settoheight{\Height}{$y$}\settodepth{\Depth}{$y$}\setlength{\Height}{\Depth}%
\put(0.0000000,3.0500000){\hspace*{\Width}\raisebox{\Height}{$y$}}%
%
\settowidth{\Width}{O}\setlength{\Width}{-1\Width}%
\settoheight{\Height}{O}\settodepth{\Depth}{O}\setlength{\Height}{-\Height}%
\put(-0.0500000,-0.0500000){\hspace*{\Width}\raisebox{\Height}{O}}%
%
\end{picture}}%}}
\putnotee{15}{20}{$S(x)$は黒斜線の面積}
\putnotee{15}{54}{$S(z)$は赤斜線の面積}
\putnotee{15}{62}{$S(z)-S(x)$は赤斜線の面積}
\end{layer}

\begin{itemize}
\item
$S(x)=\dint_a^x f(x)\,dx$とおく\ $\Rightarrow$\ $S'(x)=f(x)$を示す\vspace{1.8zw}
\item
$S'(x)=$
$\dlim_{z \to x}\bunsuu{S(z)-S(x)}{z-x}$
\item
$S(z)-S(x)$は?\vspace{2mm}\\
\end{itemize}

\newslide{基本定理の証明(続)}

\vspace*{18mm}


\begin{layer}{120}{0}
\putnotew{96}{73}{\hyperlink{para7pg8}{\fbox{\Ctab{2.5mm}{\scalebox{1}{\scriptsize $\mathstrut||\!\lhd$}}}}}
\putnotew{101}{73}{\hyperlink{para8pg1}{\fbox{\Ctab{2.5mm}{\scalebox{1}{\scriptsize $\mathstrut|\!\lhd$}}}}}
\putnotew{108}{73}{\hyperlink{para8pg7}{\fbox{\Ctab{4.5mm}{\scalebox{1}{\scriptsize $\mathstrut\!\!\lhd\!\!$}}}}}
\putnotew{115}{73}{\hyperlink{para8pg8}{\fbox{\Ctab{4.5mm}{\scalebox{1}{\scriptsize $\mathstrut\!\rhd\!$}}}}}
\putnotew{120}{73}{\hyperlink{para8pg8}{\fbox{\Ctab{2.5mm}{\scalebox{1}{\scriptsize $\mathstrut \!\rhd\!\!|$}}}}}
\putnotew{125}{73}{\hyperlink{para9pg1}{\fbox{\Ctab{2.5mm}{\scalebox{1}{\scriptsize $\mathstrut \!\rhd\!\!||$}}}}}
\putnotee{126}{73}{\scriptsize\color{blue} 8/8}
\end{layer}

\slidepage

\begin{layer}{120}{0}
\putnotese{67}{24}{\scalebox{0.85}{%%% /Users/takatoosetsuo/Dropbox/2021polytech/204/fig/kihon4.tex 
%%% Generator=kihon0723.cdy 
{\unitlength=1cm%
\begin{picture}%
(5.5,4)(-1,-1)%
\linethickness{0.008in}%%
\normalsize%
\polyline(-1.00000,1.00957)(-0.82950,1.13936)(-0.62123,1.29368)(-0.42499,1.43466)%
(-0.24077,1.56227)(-0.06857,1.67653)(0.09161,1.77743)(0.23976,1.86498)(0.37589,1.93917)%
(0.50000,2.00000)(0.68016,2.08086)(0.86641,2.16002)(1.05884,2.23610)(1.25752,2.30768)%
(1.46253,2.37336)(1.67397,2.43175)(1.89191,2.48142)(2.11644,2.52100)(2.34763,2.54906)%
(2.58558,2.56421)(2.76222,2.56530)(2.97037,2.55760)(3.21001,2.54111)(3.48116,2.51583)%
(3.78380,2.48176)(4.11793,2.43890)(4.48357,2.38725)(4.50000,2.38475)%
%
\polyline(0.50000,2.00000)(0.50000,0.00000)%
%
\polyline(3.29073,2.53358)(3.29073,0.00000)%
%
\polyline(3.69673,2.49156)(3.69673,0.00000)%
%
{%
\color[cmyk]{0,1,1,0}%
\polyline(3.29073,2.53358)(3.29073,0.00000)(3.69673,0.00000)(3.69673,2.49156)(3.48116,2.51583)%
(3.29073,2.53358)%
%
}%
\settowidth{\Width}{$y=f(x)$}\setlength{\Width}{-0.5\Width}%
\settoheight{\Height}{$y=f(x)$}\settodepth{\Depth}{$y=f(x)$}\setlength{\Height}{\Depth}%
\put(2.5900000,2.6600000){\hspace*{\Width}\raisebox{\Height}{$y=f(x)$}}%
%
{%
\color[cmyk]{0,1,1,0}%
\polyline(3.29073,0.16638)(3.45711,0.00000)%
%
\polyline(3.29073,0.51993)(3.69673,0.11393)%
%
\polyline(3.29073,0.87348)(3.69673,0.46748)%
%
\polyline(3.29073,1.22704)(3.69673,0.82104)%
%
\polyline(3.29073,1.58059)(3.69673,1.17459)%
%
\polyline(3.29073,1.93414)(3.69673,1.52814)%
%
\polyline(3.29073,2.28770)(3.69673,1.88170)%
%
\polyline(3.40947,2.52251)(3.69673,2.23525)%
%
}%
\polyline(0.50000,0.05000)(0.50000,-0.05000)%
%
\settowidth{\Width}{$a$}\setlength{\Width}{-0.5\Width}%
\settoheight{\Height}{$a$}\settodepth{\Depth}{$a$}\setlength{\Height}{-\Height}%
\put(0.5000000,-0.1000000){\hspace*{\Width}\raisebox{\Height}{$a$}}%
%
\polyline(3.29073,0.05000)(3.29073,-0.05000)%
%
\settowidth{\Width}{$x$}\setlength{\Width}{-0.5\Width}%
\settoheight{\Height}{$x$}\settodepth{\Depth}{$x$}\setlength{\Height}{-\Height}%
\put(3.2900000,-0.1000000){\hspace*{\Width}\raisebox{\Height}{$x$}}%
%
\polyline(3.69673,0.05000)(3.69673,-0.05000)%
%
\settowidth{\Width}{$z$}\setlength{\Width}{-0.5\Width}%
\settoheight{\Height}{$z$}\settodepth{\Depth}{$z$}\setlength{\Height}{-\Height}%
\put(3.7000000,-0.1000000){\hspace*{\Width}\raisebox{\Height}{$z$}}%
%
\polyline(3.29073,2.53358)(3.28114,2.51457)(3.27169,2.49548)(3.26239,2.47632)(3.25323,2.45709)%
(3.24422,2.43779)(3.23536,2.41843)(3.22665,2.39899)(3.21808,2.37949)(3.20967,2.35993)%
(3.20140,2.34030)(3.19328,2.32061)(3.18532,2.30086)(3.17750,2.28104)(3.16984,2.26117)%
(3.16233,2.24124)(3.15497,2.22126)(3.14776,2.20122)(3.14071,2.18112)(3.13381,2.16097)%
(3.12706,2.14077)(3.12047,2.12052)(3.11403,2.10022)(3.10775,2.07987)(3.10163,2.05947)%
(3.09566,2.03903)(3.08984,2.01854)(3.08418,1.99800)(3.07868,1.97743)(3.07334,1.95681)%
(3.06815,1.93616)(3.06312,1.91546)(3.05825,1.89473)(3.05354,1.87396)(3.04899,1.85315)%
(3.04459,1.83231)(3.04036,1.81144)(3.03628,1.79054)(3.03236,1.76960)(3.02861,1.74864)%
(3.02501,1.72765)(3.02157,1.70663)(3.01830,1.68558)(3.01518,1.66452)(3.01223,1.64342)%
(3.00943,1.62231)(3.00680,1.60118)(3.00433,1.58002)(3.00202,1.55885)(2.99987,1.53766)%
(2.99789,1.51646)%
%
\polyline(2.99789,1.01712)(2.99987,0.99592)(3.00202,0.97473)(3.00433,0.95356)(3.00680,0.93240)%
(3.00943,0.91127)(3.01223,0.89016)(3.01518,0.86906)(3.01830,0.84800)(3.02157,0.82695)%
(3.02501,0.80593)(3.02861,0.78494)(3.03236,0.76398)(3.03628,0.74304)(3.04036,0.72214)%
(3.04459,0.70127)(3.04899,0.68043)(3.05354,0.65962)(3.05825,0.63885)(3.06312,0.61812)%
(3.06815,0.59742)(3.07334,0.57677)(3.07868,0.55615)(3.08418,0.53558)(3.08984,0.51504)%
(3.09566,0.49455)(3.10163,0.47411)(3.10775,0.45371)(3.11403,0.43336)(3.12047,0.41306)%
(3.12706,0.39281)(3.13381,0.37261)(3.14071,0.35246)(3.14776,0.33236)(3.15497,0.31232)%
(3.16233,0.29234)(3.16984,0.27241)(3.17750,0.25254)(3.18532,0.23272)(3.19328,0.21297)%
(3.20140,0.19328)(3.20967,0.17365)(3.21808,0.15409)(3.22665,0.13459)(3.23536,0.11515)%
(3.24422,0.09579)(3.25323,0.07649)(3.26239,0.05726)(3.27169,0.03810)(3.28114,0.01901)%
(3.29073,-0.00000)%
%
\settowidth{\Width}{$f(x)$}\setlength{\Width}{-0.5\Width}%
\settoheight{\Height}{$f(x)$}\settodepth{\Depth}{$f(x)$}\setlength{\Height}{-0.5\Height}\setlength{\Depth}{0.5\Depth}\addtolength{\Height}{\Depth}%
\put(2.8900000,1.2700000){\hspace*{\Width}\raisebox{\Height}{$f(x)$}}%
%
\polyline(-1.00000,0.00000)(4.50000,0.00000)%
%
\polyline(0.00000,-1.00000)(0.00000,3.00000)%
%
\settowidth{\Width}{$x$}\setlength{\Width}{0\Width}%
\settoheight{\Height}{$x$}\settodepth{\Depth}{$x$}\setlength{\Height}{-0.5\Height}\setlength{\Depth}{0.5\Depth}\addtolength{\Height}{\Depth}%
\put(4.5500000,0.0000000){\hspace*{\Width}\raisebox{\Height}{$x$}}%
%
\settowidth{\Width}{$y$}\setlength{\Width}{-0.5\Width}%
\settoheight{\Height}{$y$}\settodepth{\Depth}{$y$}\setlength{\Height}{\Depth}%
\put(0.0000000,3.0500000){\hspace*{\Width}\raisebox{\Height}{$y$}}%
%
\settowidth{\Width}{O}\setlength{\Width}{-1\Width}%
\settoheight{\Height}{O}\settodepth{\Depth}{O}\setlength{\Height}{-\Height}%
\put(-0.0500000,-0.0500000){\hspace*{\Width}\raisebox{\Height}{O}}%
%
\end{picture}}%}}
\putnotene{117}{51}{\scalebox{0.85}{%%% /Users/takatoosetsuo/Dropbox/2021polytech/204/fig/kihon4_2.tex 
%%% Generator=kihon0723.cdy 
{\unitlength=20mm%
\begin{picture}%
(0.91,2.73)(2.89,-0.1)%
\linethickness{0.008in}%%
\normalsize%
\polyline(2.89000,2.56057)(2.97037,2.55760)(3.21001,2.54111)(3.48116,2.51583)(3.78380,2.48176)%
(3.80000,2.47968)%
%
\polyline(3.29073,2.53358)(3.29073,0.00000)%
%
\polyline(3.69673,2.49156)(3.69673,0.00000)%
%
{%
\color[cmyk]{0,1,1,0}%
\polyline(3.29073,2.53358)(3.29073,0.00000)(3.69673,0.00000)(3.69673,2.49156)(3.48116,2.51583)%
(3.29073,2.53358)%
%
}%
\settowidth{\Width}{$y=f(x)$}\setlength{\Width}{-0.5\Width}%
\settoheight{\Height}{$y=f(x)$}\settodepth{\Depth}{$y=f(x)$}\setlength{\Height}{\Depth}%
\put(2.5900000,2.6100000){\hspace*{\Width}\raisebox{\Height}{$y=f(x)$}}%
%
{%
\color[cmyk]{0,1,1,0}%
\polyline(3.29073,0.25861)(3.54934,0.00000)%
%
\polyline(3.29073,0.61216)(3.69673,0.20616)%
%
\polyline(3.29073,0.96572)(3.69673,0.55972)%
%
\polyline(3.29073,1.31927)(3.69673,0.91327)%
%
\polyline(3.29073,1.67282)(3.69673,1.26682)%
%
\polyline(3.29073,2.02638)(3.69673,1.62038)%
%
\polyline(3.29073,2.37993)(3.69673,1.97393)%
%
\polyline(3.51184,2.51238)(3.69673,2.32748)%
%
}%
%\settowidth{\Width}{$a$}\setlength{\Width}{-0.5\Width}%
%\settoheight{\Height}{$a$}\settodepth{\Depth}{$a$}\setlength{\Height}{-\Height}%
%\put(0.5000000,-0.0500000){\hspace*{\Width}\raisebox{\Height}{$a$}}%
%
\polyline(3.29073,0.05000)(3.29073,-0.05000)%
%
\settowidth{\Width}{$x$}\setlength{\Width}{-0.5\Width}%
\settoheight{\Height}{$x$}\settodepth{\Depth}{$x$}\setlength{\Height}{-\Height}%
\put(3.2900000,-0.0500000){\hspace*{\Width}\raisebox{\Height}{$x$}}%
%
\polyline(3.69673,0.05000)(3.69673,-0.05000)%
%
\settowidth{\Width}{$z$}\setlength{\Width}{-0.5\Width}%
\settoheight{\Height}{$z$}\settodepth{\Depth}{$z$}\setlength{\Height}{-\Height}%
\put(3.7000000,-0.0500000){\hspace*{\Width}\raisebox{\Height}{$z$}}%
%
\polyline(3.29073,2.53358)(3.28114,2.51457)(3.27169,2.49548)(3.26239,2.47632)(3.25323,2.45709)%
(3.24422,2.43779)(3.23536,2.41843)(3.22665,2.39899)(3.21808,2.37949)(3.20967,2.35993)%
(3.20140,2.34030)(3.19328,2.32061)(3.18532,2.30086)(3.17750,2.28104)(3.16984,2.26117)%
(3.16233,2.24124)(3.15497,2.22126)(3.14776,2.20122)(3.14071,2.18112)(3.13381,2.16097)%
(3.12706,2.14077)(3.12047,2.12052)(3.11403,2.10022)(3.10775,2.07987)(3.10163,2.05947)%
(3.09566,2.03903)(3.08984,2.01854)(3.08418,1.99800)(3.07868,1.97743)(3.07334,1.95681)%
(3.06815,1.93616)(3.06312,1.91546)(3.05825,1.89473)(3.05354,1.87396)(3.04899,1.85315)%
(3.04459,1.83231)(3.04036,1.81144)(3.03628,1.79054)(3.03236,1.76960)(3.02861,1.74864)%
(3.02501,1.72765)(3.02157,1.70663)(3.01830,1.68558)(3.01518,1.66452)(3.01223,1.64342)%
(3.00943,1.62231)(3.00680,1.60118)(3.00433,1.58002)(3.00202,1.55885)(2.99987,1.53766)%
(2.99789,1.51646)%
%
\polyline(2.99789,1.01712)(2.99987,0.99592)(3.00202,0.97473)(3.00433,0.95356)(3.00680,0.93240)%
(3.00943,0.91127)(3.01223,0.89016)(3.01518,0.86906)(3.01830,0.84800)(3.02157,0.82695)%
(3.02501,0.80593)(3.02861,0.78494)(3.03236,0.76398)(3.03628,0.74304)(3.04036,0.72214)%
(3.04459,0.70127)(3.04899,0.68043)(3.05354,0.65962)(3.05825,0.63885)(3.06312,0.61812)%
(3.06815,0.59742)(3.07334,0.57677)(3.07868,0.55615)(3.08418,0.53558)(3.08984,0.51504)%
(3.09566,0.49455)(3.10163,0.47411)(3.10775,0.45371)(3.11403,0.43336)(3.12047,0.41306)%
(3.12706,0.39281)(3.13381,0.37261)(3.14071,0.35246)(3.14776,0.33236)(3.15497,0.31232)%
(3.16233,0.29234)(3.16984,0.27241)(3.17750,0.25254)(3.18532,0.23272)(3.19328,0.21297)%
(3.20140,0.19328)(3.20967,0.17365)(3.21808,0.15409)(3.22665,0.13459)(3.23536,0.11515)%
(3.24422,0.09579)(3.25323,0.07649)(3.26239,0.05726)(3.27169,0.03810)(3.28114,0.01901)%
(3.29073,-0.00000)%
%
\settowidth{\Width}{$f(x)$}\setlength{\Width}{-0.5\Width}%
\settoheight{\Height}{$f(x)$}\settodepth{\Depth}{$f(x)$}\setlength{\Height}{-0.5\Height}\setlength{\Depth}{0.5\Depth}\addtolength{\Height}{\Depth}%
\put(2.9400000,1.2700000){\hspace*{\Width}\raisebox{\Height}{$f(x)$}}%
%
{%
\color[cmyk]{0,0,0,0}%
\polygon*(2.18558,2.61421)(2.98558,2.61421)(2.98558,2.86421)(2.18558,2.86421)(2.18558,2.61421)%
(2.18558,2.61421)}%
{%
\color[cmyk]{0.7,0.7,0,0}%
\linethickness{0.016in}%%
\polyline(3.29073,2.51257)(3.29073,0.00000)(3.69673,0.00000)(3.69673,2.51257)(3.29073,2.51257)%
%
\linethickness{0.008in}%%
}%
{%
\color[cmyk]{0.7,0.7,0,0}%
\linethickness{0.016in}%%
\polyline(3.49373,2.51257)(3.49373,0.00000)%
%
\linethickness{0.008in}%%
}%
\settowidth{\Width}{$x_1$}\setlength{\Width}{-0.5\Width}%
\settoheight{\Height}{$x_1$}\settodepth{\Depth}{$x_1$}\setlength{\Height}{-\Height}%
\put(3.4900000,-0.0625000){\hspace*{\Width}\raisebox{\Height}{$x_1$}}%
%
\polyline(2.89073,0.00000)(3.79673,0.00000)%
%
\settowidth{\Width}{$ $}\setlength{\Width}{0\Width}%
\settoheight{\Height}{$ $}\settodepth{\Depth}{$ $}\setlength{\Height}{-0.5\Height}\setlength{\Depth}{0.5\Depth}\addtolength{\Height}{\Depth}%
\put(3.8250000,0.0000000){\hspace*{\Width}\raisebox{\Height}{$ $}}%
%
\settowidth{\Width}{$ $}\setlength{\Width}{-0.5\Width}%
\settoheight{\Height}{$ $}\settodepth{\Depth}{$ $}\setlength{\Height}{\Depth}%
\put(0.0000000,2.6550000){\hspace*{\Width}\raisebox{\Height}{$ $}}%
%
\settowidth{\Width}{ }\setlength{\Width}{-1\Width}%
\settoheight{\Height}{ }\settodepth{\Depth}{ }\setlength{\Height}{-\Height}%
\put(-0.0250000,-0.0250000){\hspace*{\Width}\raisebox{\Height}{ }}%
%
\end{picture}}%}}
\end{layer}

\begin{itemize}
\item
図より $S(z)-S(x)\fallingdotseq f(x)(z-x)$\\
$S(z)-S(x)=f(x_1)(z-x)$
\item
$S'(x)=\dlim_{z \to x}\bunsuu{S(z)-S(x)}{z-x}$\vspace{2mm}\\
\hspace*{5mm}$=\dlim_{z \to x}\bunsuu{f(x_1)(z-x)}{z-x}$\vspace{2mm}\\
\hspace*{5mm}$=\dlim_{z \to x}f(x_1)$
$=f(x)$\vspace{-1mm}
\item
したがって $S(x)=\dint_a^x f(x)\,dx$は$f(x)$の不定積分
\end{itemize}

\newslide{定積分の計算公式}

\vspace*{18mm}


\begin{layer}{120}{0}
\putnotew{96}{73}{\hyperlink{para8pg8}{\fbox{\Ctab{2.5mm}{\scalebox{1}{\scriptsize $\mathstrut||\!\lhd$}}}}}
\putnotew{101}{73}{\hyperlink{para9pg1}{\fbox{\Ctab{2.5mm}{\scalebox{1}{\scriptsize $\mathstrut|\!\lhd$}}}}}
\putnotew{108}{73}{\hyperlink{para9pg4}{\fbox{\Ctab{4.5mm}{\scalebox{1}{\scriptsize $\mathstrut\!\!\lhd\!\!$}}}}}
\putnotew{115}{73}{\hyperlink{para9pg5}{\fbox{\Ctab{4.5mm}{\scalebox{1}{\scriptsize $\mathstrut\!\rhd\!$}}}}}
\putnotew{120}{73}{\hyperlink{para9pg5}{\fbox{\Ctab{2.5mm}{\scalebox{1}{\scriptsize $\mathstrut \!\rhd\!\!|$}}}}}
\putnotew{125}{73}{\hyperlink{para10pg1}{\fbox{\Ctab{2.5mm}{\scalebox{1}{\scriptsize $\mathstrut \!\rhd\!\!||$}}}}}
\putnotee{126}{73}{\scriptsize\color{blue} 5/5}
\end{layer}

\slidepage

\begin{layer}{120}{0}
\end{layer}

\begin{itemize}
\item
$f(x)$の不定積分の1つを$F(x)$とおく
\item
基本定理より $S(x)=\displaystyle\int_a^x f(x)\,dx$も不定積分
\item
したがって $S(x)=F(x)+C$($C$は積分定数)
\item
$x$に$a$を代入すると $S(a)=F(a)+C$
\item
$S(a)=\displaystyle\int_a^a f(x)\,dx=0$より $F(a)+C=0$
\end{itemize}

\newslide{定積分の計算公式(続)}

\vspace*{18mm}


\begin{layer}{120}{0}
\putnotew{96}{73}{\hyperlink{para9pg5}{\fbox{\Ctab{2.5mm}{\scalebox{1}{\scriptsize $\mathstrut||\!\lhd$}}}}}
\putnotew{101}{73}{\hyperlink{para10pg1}{\fbox{\Ctab{2.5mm}{\scalebox{1}{\scriptsize $\mathstrut|\!\lhd$}}}}}
\putnotew{108}{73}{\hyperlink{para10pg4}{\fbox{\Ctab{4.5mm}{\scalebox{1}{\scriptsize $\mathstrut\!\!\lhd\!\!$}}}}}
\putnotew{115}{73}{\hyperlink{para10pg5}{\fbox{\Ctab{4.5mm}{\scalebox{1}{\scriptsize $\mathstrut\!\rhd\!$}}}}}
\putnotew{120}{73}{\hyperlink{para10pg5}{\fbox{\Ctab{2.5mm}{\scalebox{1}{\scriptsize $\mathstrut \!\rhd\!\!|$}}}}}
\putnotew{125}{73}{\hyperlink{para11pg1}{\fbox{\Ctab{2.5mm}{\scalebox{1}{\scriptsize $\mathstrut \!\rhd\!\!||$}}}}}
\putnotee{126}{73}{\scriptsize\color{blue} 5/5}
\end{layer}

\slidepage

\begin{layer}{120}{0}
\end{layer}

\begin{itemize}
\item
これから $C=-F(a)$
\item
したがって $S(x)=F(x)+C=F(x)-F(a)$
\item
$x$に$b$を代入すると $S(b)=F(b)-F(a)$
\item
よって \fbox{\color{red}$\displaystyle\int_a^b f(x)\,dx=F(b)-F(a)$}
\item
[]\hspace*{2zw}$F(b)-F(a)={\color{red}\Bigl[F(x)\Bigr]_a^b}$と書く
\end{itemize}

\newslide{定積分の計算1}

\vspace*{18mm}


\begin{layer}{120}{0}
\putnotew{96}{73}{\hyperlink{para10pg5}{\fbox{\Ctab{2.5mm}{\scalebox{1}{\scriptsize $\mathstrut||\!\lhd$}}}}}
\putnotew{101}{73}{\hyperlink{para11pg1}{\fbox{\Ctab{2.5mm}{\scalebox{1}{\scriptsize $\mathstrut|\!\lhd$}}}}}
\putnotew{108}{73}{\hyperlink{para11pg4}{\fbox{\Ctab{4.5mm}{\scalebox{1}{\scriptsize $\mathstrut\!\!\lhd\!\!$}}}}}
\putnotew{115}{73}{\hyperlink{para11pg5}{\fbox{\Ctab{4.5mm}{\scalebox{1}{\scriptsize $\mathstrut\!\rhd\!$}}}}}
\putnotew{120}{73}{\hyperlink{para11pg5}{\fbox{\Ctab{2.5mm}{\scalebox{1}{\scriptsize $\mathstrut \!\rhd\!\!|$}}}}}
\putnotew{125}{73}{\hyperlink{para12pg1}{\fbox{\Ctab{2.5mm}{\scalebox{1}{\scriptsize $\mathstrut \!\rhd\!\!||$}}}}}
\putnotee{126}{73}{\scriptsize\color{blue} 5/5}
\end{layer}

\slidepage

\begin{layer}{120}{0}
\putnotese{100}{35}{%%% /Users/takatoosetsuo/Dropbox/2021polytech/204/fig/teiseki1.tex 
%%% Generator=kihon0720.cdy 
{\unitlength=1cm%
\begin{picture}%
(3,3.5)(-1,-1)%
\linethickness{0.008in}%%
\normalsize%
\polyline(-1.00000,1.00000)(-0.94000,0.88360)(-0.88000,0.77440)(-0.82000,0.67240)%
(-0.76000,0.57760)(-0.70000,0.49000)(-0.64000,0.40960)(-0.58000,0.33640)(-0.52000,0.27040)%
(-0.46000,0.21160)(-0.40000,0.16000)(-0.34000,0.11560)(-0.28000,0.07840)(-0.22000,0.04840)%
(-0.16000,0.02560)(-0.10000,0.01000)(-0.04000,0.00160)(0.02000,0.00040)(0.08000,0.00640)%
(0.14000,0.01960)(0.20000,0.04000)(0.26000,0.06760)(0.32000,0.10240)(0.38000,0.14440)%
(0.44000,0.19360)(0.50000,0.25000)(0.56000,0.31360)(0.62000,0.38440)(0.68000,0.46240)%
(0.74000,0.54760)(0.80000,0.64000)(0.86000,0.73960)(0.92000,0.84640)(0.98000,0.96040)%
(1.04000,1.08160)(1.10000,1.21000)(1.16000,1.34560)(1.22000,1.48840)(1.28000,1.63840)%
(1.34000,1.79560)(1.40000,1.96000)(1.46000,2.13160)(1.52000,2.31040)(1.58000,2.49640)%
(1.58112,2.50000)%
%
\polyline(1.00000,1.00000)(1.00000,0.00000)%
%
\polyline(0.98744,0.00000)(1.00000,0.01256)%
%
\polyline(0.81066,0.00000)(0.99994,0.18928)%
%
\polyline(0.63388,0.00000)(0.99988,0.36600)%
%
\polyline(0.45711,0.00000)(0.99983,0.54272)%
%
\polyline(0.28033,0.00000)(0.99977,0.71944)%
%
\polyline(0.10355,0.00000)(0.11840,0.01485)%
%
\polyline(0.88160,0.77804)(0.99971,0.89616)%
%
\polyline(1.00000,0.05000)(1.00000,-0.05000)%
%
\settowidth{\Width}{$1$}\setlength{\Width}{-0.5\Width}%
\settoheight{\Height}{$1$}\settodepth{\Depth}{$1$}\setlength{\Height}{-\Height}%
\put(1.0000000,-0.1000000){\hspace*{\Width}\raisebox{\Height}{$1$}}%
%
\polyline(-1.00000,0.00000)(2.00000,0.00000)%
%
\polyline(0.00000,-1.00000)(0.00000,2.50000)%
%
\settowidth{\Width}{$x$}\setlength{\Width}{0\Width}%
\settoheight{\Height}{$x$}\settodepth{\Depth}{$x$}\setlength{\Height}{-0.5\Height}\setlength{\Depth}{0.5\Depth}\addtolength{\Height}{\Depth}%
\put(2.0500000,0.0000000){\hspace*{\Width}\raisebox{\Height}{$x$}}%
%
\settowidth{\Width}{$y$}\setlength{\Width}{-0.5\Width}%
\settoheight{\Height}{$y$}\settodepth{\Depth}{$y$}\setlength{\Height}{\Depth}%
\put(0.0000000,2.5500000){\hspace*{\Width}\raisebox{\Height}{$y$}}%
%
\settowidth{\Width}{O}\setlength{\Width}{-1\Width}%
\settoheight{\Height}{O}\settodepth{\Depth}{O}\setlength{\Height}{-\Height}%
\put(-0.0500000,-0.0500000){\hspace*{\Width}\raisebox{\Height}{O}}%
%
\end{picture}}%}
\end{layer}

\seteda{35}
\begin{itemize}
\item
[(例)]$\dint_0^1x^2\,dx$\vspace{-1mm}
\item
[]不定積分の公式より $\dint x^2\,dx=\bunsuu{1}{3}x^3+C$\vspace{-1mm}
\item
[]$\dint_0^1x^2\,dx=\Bigl[\bunsuu{1}{3}x^3\Bigr]_0^1$
$=\bunsuu{1}{3}1^3-\bunsuu{1}{3}0^3=\bunsuu{1}{3}$
\item
[課題]\monban 次の定積分を計算せよ.\\
\eda{$\dint_0^2 x^2\,dx$}\eda{$\dint_1^2 x^2\,dx$}
\end{itemize}

\newslide{定積分の性質}

\vspace*{18mm}


\begin{layer}{120}{0}
\putnotew{96}{73}{\hyperlink{para11pg5}{\fbox{\Ctab{2.5mm}{\scalebox{1}{\scriptsize $\mathstrut||\!\lhd$}}}}}
\putnotew{101}{73}{\hyperlink{para12pg1}{\fbox{\Ctab{2.5mm}{\scalebox{1}{\scriptsize $\mathstrut|\!\lhd$}}}}}
\putnotew{108}{73}{\hyperlink{para12pg1}{\fbox{\Ctab{4.5mm}{\scalebox{1}{\scriptsize $\mathstrut\!\!\lhd\!\!$}}}}}
\putnotew{115}{73}{\hyperlink{para12pg2}{\fbox{\Ctab{4.5mm}{\scalebox{1}{\scriptsize $\mathstrut\!\rhd\!$}}}}}
\putnotew{120}{73}{\hyperlink{para12pg2}{\fbox{\Ctab{2.5mm}{\scalebox{1}{\scriptsize $\mathstrut \!\rhd\!\!|$}}}}}
\putnotew{125}{73}{\hyperlink{para13pg1}{\fbox{\Ctab{2.5mm}{\scalebox{1}{\scriptsize $\mathstrut \!\rhd\!\!||$}}}}}
\putnotee{126}{73}{\scriptsize\color{blue} 2/2}
\end{layer}

\slidepage
\begin{itemize}
\item
$\displaystyle\int_a^b\bigl(f(x)+g(x)\bigr)\,dx=\int_a^b f(x)\,dx+\int_a^b g(x)\,dx$
\item
$\displaystyle\int_a^b\bigl(f(x)-g(x)\bigr)\,dx=\int_a^b f(x)\,dx-\int_a^b g(x)\,dx$
\item
$\displaystyle\int_a^b c f(x)\,dx=c\int_a^b f(x)\,dx$($c$は定数)
\item
[(注)]最初に不定積分を求めてから計算した方がいい
\end{itemize}

\newslide{定積分の計算2}

\vspace*{18mm}


\begin{layer}{120}{0}
\putnotew{96}{73}{\hyperlink{para12pg2}{\fbox{\Ctab{2.5mm}{\scalebox{1}{\scriptsize $\mathstrut||\!\lhd$}}}}}
\putnotew{101}{73}{\hyperlink{para13pg1}{\fbox{\Ctab{2.5mm}{\scalebox{1}{\scriptsize $\mathstrut|\!\lhd$}}}}}
\putnotew{108}{73}{\hyperlink{para13pg7}{\fbox{\Ctab{4.5mm}{\scalebox{1}{\scriptsize $\mathstrut\!\!\lhd\!\!$}}}}}
\putnotew{115}{73}{\hyperlink{para13pg8}{\fbox{\Ctab{4.5mm}{\scalebox{1}{\scriptsize $\mathstrut\!\rhd\!$}}}}}
\putnotew{120}{73}{\hyperlink{para13pg8}{\fbox{\Ctab{2.5mm}{\scalebox{1}{\scriptsize $\mathstrut \!\rhd\!\!|$}}}}}
\putnotew{125}{73}{\hyperlink{para14pg1}{\fbox{\Ctab{2.5mm}{\scalebox{1}{\scriptsize $\mathstrut \!\rhd\!\!||$}}}}}
\putnotee{126}{73}{\scriptsize\color{blue} 8/8}
\end{layer}

\slidepage
\begin{itemize}
\item
[(例1)]$\dint_1^2 (2 x+3)\;dx$
$=\Bigl[x^2+3x\Bigr]_1^2$\\
$\phantom{dint_1^2 (2 x+3)\;dx}=(2^2+3\cdot 2)-(1^2+3\cdot 1)$\\
$\phantom{dint_1^2 (2 x+3)\;dx}=6$
\item
[(例2)]$\dint_0^1 (3x^2+x)\,dx$
$=\Bigl[x^3+\bunsuu{1}{2}x^2\Bigr]_0^1$\\
$\phantom{\dint_0^1 (3x^2+x)\,dx}=(1+\bunsuu{1}{2})-(0+0)$\\
$\phantom{\dint_0^1 (3x^2+x)\,dx}=\bunsuu{3}{2}$
\end{itemize}

\newslide{定積分の計算(課題)}

\vspace*{18mm}


\begin{layer}{120}{0}
\putnotew{96}{73}{\hyperlink{para13pg8}{\fbox{\Ctab{2.5mm}{\scalebox{1}{\scriptsize $\mathstrut||\!\lhd$}}}}}
\putnotew{125}{73}{\hyperlink{para15pg1}{\fbox{\Ctab{2.5mm}{\scalebox{1}{\scriptsize $\mathstrut \!\rhd\!\!||$}}}}}
\putnotee{126}{73}{\scriptsize\color{blue} 1/1}
\end{layer}

\slidepage
\seteda{100}
\begin{itemize}
\item
[課題]\monban 次の定積分を計算せよ.\\
\eda{$\dint_0^1 (3x^2+1)\;dx$}\\
\eda{$\dint_{-1}^2(-x^2+x+2)\;dx$}\\
\eda{$\dint_0^1 (x^3+1)\;dx$}\\
\eda{$\dint_{-1}^1(x^4+x^3+2x^2)\;dx$}
\end{itemize}
\label{pageend}\mbox{}

\end{document}

%%% Title presen22101
\documentclass[landscape,10pt]{ujarticle}
\special{papersize=\the\paperwidth,\the\paperheight}
\usepackage{ketpic,ketlayer}
\usepackage{ketslide}
\usepackage{amsmath,amssymb}
\usepackage{bm,enumerate}
\usepackage[dvipdfmx]{graphicx}
\usepackage{color}
\definecolor{slidecolora}{cmyk}{0.98,0.13,0,0.43}
\definecolor{slidecolorb}{cmyk}{0.2,0,0,0}
\definecolor{slidecolorc}{cmyk}{0.2,0,0,0}
\definecolor{slidecolord}{cmyk}{0.2,0,0,0}
\definecolor{slidecolore}{cmyk}{0,0,0,0.5}
\definecolor{slidecolorf}{cmyk}{0,0,0,0.5}
\definecolor{slidecolori}{cmyk}{0.98,0.13,0,0.43}
\def\setthin#1{\def\thin{#1}}
\setthin{0}
\newcommand{\slidepage}[1][s]{%
\setcounter{ketpicctra}{18}%
\if#1m \setcounter{ketpicctra}{1}\fi
\hypersetup{linkcolor=black}%

\begin{layer}{118}{0}
\putnotee{122}{-\theketpicctra.05}{\small\thepage/\pageref{pageend}}
\end{layer}\hypersetup{linkcolor=blue}

}
\usepackage{pict2e}
\usepackage[dvipdfmx,colorlinks=true,linkcolor=blue,filecolor=blue]{hyperref}
\newcommand{\hiduke}{0411}
\newcommand{\hako}[2][1]{\fbox{\raisebox{#1mm}{\mbox{}}\raisebox{-#1mm}{\mbox{}}\,\phantom{#2}\,}}
\newcommand{\hakoa}[2][1]{\fbox{\raisebox{#1mm}{\mbox{}}\raisebox{-#1mm}{\mbox{}}\,#2\,}}
\newcommand{\hakom}[2][1]{\hako[#1]{$#2$}}
\newcommand{\hakoma}[2][1]{\hakoa[#1]{$#2$}}
\def\rad{\;\mathrm{rad}}
\def\deg#1{#1^{\circ}}
\newcommand{\sbunsuu}[2]{\scalebox{0.6}{$\bunsuu{#1}{#2}$}}
\def\pow{$\hspace{-1.5mm}^\hspace{-1mm}$}
\def\dlim{\displaystyle\lim}
\newcommand{\brd}[2][1]{\scalebox{#1}{\color{red}\fbox{\color{black}$#2$}}}

\setmargin{25}{145}{15}{100}

\ketslideinit

\pagestyle{empty}

\begin{document}

\begin{layer}{120}{0}
\putnotese{0}{0}{{\Large\bf
\color[cmyk]{1,1,0,0}

\begin{layer}{120}{0}
{\Huge \putnotes{60}{20}{この授業について}}
\putnotes{60}{50}{高遠節夫(たかとおせつお)}
\putnotes{60}{70}{2022.04.11}
\end{layer}

}
}
\end{layer}

\def\mainslidetitley{22}
\def\ketcletter{slidecolora}
\def\ketcbox{slidecolorb}
\def\ketdbox{slidecolorc}
\def\ketcframe{slidecolord}
\def\ketcshadow{slidecolore}
\def\ketdshadow{slidecolorf}
\def\slidetitlex{6}
\def\slidetitlesize{1.3}
\def\mketcletter{slidecolori}
\def\mketcbox{yellow}
\def\mketdbox{yellow}
\def\mketcframe{yellow}
\def\mslidetitlex{62}
\def\mslidetitlesize{2}

\color{black}
\Large\bf\boldmath
\addtocounter{page}{-1}

\def\MARU{}
\renewcommand{\MARU}[1]{{\ooalign{\hfil$#1$\/\hfil\crcr\raise.167ex\hbox{\mathhexbox20D}}}}
\renewcommand{\slidepage}[1][s]{%
\setcounter{ketpicctra}{18}%
\if#1m \setcounter{ketpicctra}{1}\fi
\hypersetup{linkcolor=black}%
\begin{layer}{118}{0}
\putnotee{115}{-\theketpicctra.05}{\small\hiduke-\thepage/\pageref{pageend}}
\end{layer}\hypersetup{linkcolor=blue}
}
\newcounter{ban}
\setcounter{ban}{1}
\newcommand{\monban}[1][\hiduke]{%
#1-\theban\ %
\addtocounter{ban}{1}%
}
\newcommand{\monbannoadd}[1][\hiduke]{%
#1-\theban\ %
}
\newcommand{\addban}{%
\addtocounter{ban}{1}%%210614
}
\newcounter{edawidth}
\newcounter{edactr}
\newcommand{\seteda}[1]{
\setcounter{edawidth}{#1}
\setcounter{edactr}{1}
}
\newcommand{\eda}[2][\theedawidth ]{%
\noindent\Ltab{#1 mm}{[\theedactr]\ #2}%
\addtocounter{edactr}{1}%
}
%%%%%%%%%%%%

%%%%%%%%%%%%%%%%%%%%

\mainslide{Google Classroom(GC)}


\slidepage[m]
%%%%%%%%%%%%

%%%%%%%%%%%%%%%%%%%%

\newslide{準備}

\vspace*{18mm}

\slidepage
\begin{enumerate}[(1)]
\item
Gmailのアカウントがない場合は作成する
\item
スマホ:GoogleClassroomのアプリを入手(無料)\\
PC :「GoogleClassroom ログイン」で検索
\item
\verb|Google Classroom|(以下GC)にログインする\\
\hspace*{1zw}注)アカウント名とパスワードが必要
\item
右上か右下の「+」を押して「クラスに参加」を選ぶ\\
\hspace*{1zw}クラスコード {\color{red}6mxdy4j} を入力
\end{enumerate}
%%%%%%%%%%%%

%%%%%%%%%%%%%%%%%%%%

\newslide{授業(polytech22)のページ}

\vspace*{18mm}

\slidepage

\begin{layer}{120}{0}
\putnotes{65}{2}{\includegraphics[bb=0.00 0.00 2210.00 1540.00,width=90mm]{fig/classroomtop.pdf}}
\end{layer}

%%%%%%%%%%%%

%%%%%%%%%%%%%%%%%%%%


\newslide{授業の方法}

\vspace*{18mm}

%%repeat=4
\slidepage
\begin{itemize}
\item
主にスライドを用いる
\item
ノートに要点をまとめ,問題を解く
\item
試験ではノートの持ち込みOK
\item
教材や課題の配付は,GCから%%\\
%%\hspace*{1zw}HTML教材のときは,スマホを横置きで使う
\item
{\color{red}授業中は,指示した時以外スマホを使わないこと}
\end{itemize}
%%%%%%%%%%%%

%%%%%%%%%%%%%%%%%%%%

\mainslide{数式のやりとり}


\slidepage[m]
%%%%%%%%%%%%

%%%%%%%%%%%%%%%%%%%%

\newslide{KeTMathの利用}

\vspace*{18mm}

\slidepage
\begin{itemize}
\item
普通の数式(2次元記法)は見やすい.\\
 $\dfrac{4}{9}$,\ $\sqrt{7}$,\ $5^3$
\item
しかし,オンラインでのやりとりには向かない\\
 =>1次元記法の方がよい
\item
KeTMath\\
 ・1次元数式を入力すると即時に2次元数式を表示\\
 ・問題の1次元数式を入力窓1にコピーして確認\\
 ・解答を入力窓2に入力してよければGCにコピー
\end{itemize}

\newslide{1次元数式の記法}

\vspace*{18mm}

%%repeat=4
\slidepage
\begin{itemize}
\item
\Ltab{8zw}{分数 (fraction)}$\dfrac{a}{b}\ \Longrightarrow$\ \verb|fr(a,b)|
\item
\Ltab{8zw}{割り算}$a\div b\ \Longrightarrow$\ \verb|a {\div} b|\\
\hfill(あまり使わない)
\item
\Ltab{6zw}{掛け算}$ab\ \Longrightarrow$\ \verb|ab, a*b, a(cdot)b, a(cross)b|
\item
\Ltab{10zw}{べき乗}$a^b\ \Longrightarrow$\ \verb|a^(b)|
\item
\Ltab{12zw}{平方根 (square root)}$\sqrt{a}\ \Longrightarrow$\ \verb|sq(a)|
\item
\Ltab{10zw}{円周率}$\pi\ \Longrightarrow$\ \verb|pi|
\end{itemize}
%%%%%%%%%%%%

%%%%%%%%%%%%%%%%%%%%

\newslide{GCとKeTMath}

\vspace*{18mm}

%%repeat=2
\slidepage
\begin{enumerate}[(1)]
\item
GCの質問にリンクがあるときはリンクをクリック\vspace{-2mm}
\item
課題を埋め込んだKeTMathが立ち上がる\vspace{-2mm}
\item
自分の番号を入れてOKを押す\vspace{-2mm}
\item
入力2に解答を入れる\vspace{-2mm}
\item
Recを押すと全ての解答が入力3に入る\vspace{-2mm}
\item
入力3の「すべてを選択」コピーする\vspace{-2mm}
\item
GCの回答欄にペーストして送信を押す
\end{enumerate}
%%%%%%%%%%%%

%%%%%%%%%%%%%%%%%%%%

\newslide{課題(KeTMathに慣れよう)}

\vspace*{18mm}

%%repeat=2
\slidepage
\begin{itemize}
\item
課題 \monban 次の1次元数式の2次元数式を書け.\seteda{55}\\
\eda{fr(1+4,3)}\eda{a+b/c+d}\\
\eda{3sq(6)}\eda{pir\^{}\!(2)}
\item
課題 \monban 次の数式を1次元数式で書け.\vspace{2mm}\seteda{55}\\
\eda{$-\dfrac{3}{5}$}\eda{$\dfrac{xy}{x+y}$}\vspace{2mm}\\
\eda{$\sqrt{3}-\sqrt{2}$}\eda{$\dfrac{\pi}{2}$}
\end{itemize}
%%%%%%%%%%%%

%%%%%%%%%%%%%%%%%%%%

\mainslide{数と式の計算}


\slidepage[m]
%%%%%%%%%%%%

%%%%%%%%%%%%%%%%%%%%

\newslide{正負の数の足し算と引き算}

\vspace*{18mm}

\slidepage
\begin{itemize}
\item
正の数$-$正の数\\
\hspace*{2zw}$14-6=8$\\
\hspace*{2zw}$6-8=-2$
\item
正の数$+$負の数\\
\hspace*{2zw}$12+(-3)=12-3=9$\\
\hspace*{2zw}$(-5)+3=3+(-5)=3-5=-2$
\item
負の数$+$負の数\\
\hspace*{2zw}$(-2)+(-3)=-(2+3)=-5$
\end{itemize}

\newslide{正負の数の掛け算(割り算)}

\vspace*{18mm}

\slidepage
\begin{itemize}
\item
正の数$\times$負の数$=$負の数\\
\hspace*{2zw}$6\times(-3)=-18$
\item
負の数$\times$負の数$=$正の数\\
\hspace*{2zw}$(-4)\times(-3)=12$\\
\hspace*{1zw}注)$(-4)(-3)$とか$(-4)\cdot(-3)$と書くこともある
\end{itemize}

\newslide{計算問題}

\vspace*{18mm}

%%repeat=2
\slidepage
\seteda{55}
\begin{itemize}
\item
課題\monban\\
\eda{$-6+5$}\eda{$8-(-2)$}\\
\eda{$(-7)(+8)$}\eda{$32\div(-4)\times 8$}
\seteda{55}
\item
課題\monban\\
\eda{$6-8\div (-4)$}\eda{$18\div(-6)-7\times(-2)$}\\
\eda{$54\div(3^2-3)$}\eda{$3\times 23 + 3\times 77$}
\end{itemize}
%%%%%%%%%%%%

%%%%%%%%%%%%%%%%%%%%

\newslide{分数の計算}

\vspace*{18mm}

\slidepage
\begin{itemize}
\item
約分 分母と分子を同じ数で割る\vspace{1mm}\\
\hspace*{2zw}$\dfrac{4}{6}=\dfrac{2}{3}$(分母と分子を$2$で割る)
\item
通分 2つの分数の分母を同じにする
\item
足し算(引き算) 通分してから分子どうしを計算\vspace{1mm}\\
\hspace*{2zw}$\dfrac{3}{4}+\dfrac{1}{6}=\dfrac{9}{12}+\dfrac{2}{12}=\dfrac{9+2}{12}=\dfrac{11}{12}$
\item
掛け算 分母どうし,分子どうしを掛ける\vspace{1mm}\\
\hspace*{2zw}$\dfrac{2}{5}\times\dfrac{5}{6}=\dfrac{10}{30}=\dfrac{1}{3}$
\end{itemize}

\newslide{分数の計算問題}

\vspace*{18mm}

%%repeat=2,para
\slidepage
\seteda{55}
\begin{itemize}
\item
課題\monban\vspace{1mm}\\
\eda{$\dfrac{1}{2}-\dfrac{1}{4}$}\eda{$\dfrac{2}{3}-\dfrac{5}{6}$}\vspace{1mm}\\
\eda{$\dfrac{3}{10}-\dfrac{3}{5}$}\eda{$\dfrac{1}{2}+\dfrac{1}{3}$}
\item
課題\monban\vspace{1mm}\seteda{55}\\
\eda{$\dfrac{4}{5}\times\dfrac{2}{3}$}\eda{$\dfrac{2}{5}\times\dfrac{3}{7}$}\vspace{1mm}\\
\eda{$\dfrac{4}{3}\times\dfrac{9}{8}$}\eda{$\dfrac{4}{9}\times\dfrac{6}{5}$}
\end{itemize}
%%%%%%%%%%%%

%%%%%%%%%%%%%%%%%%%%

\newslide{分数の割り算}

\vspace*{18mm}

%%repeat=1
\slidepage
\seteda{55}
\begin{itemize}
\item
割る方の分母と分子をひっくり返して掛ける\vspace{1mm}\\
\hspace*{2zw}$\dfrac{9}{26}\div\dfrac{3}{4}=\dfrac{9}{26}\times\dfrac{4}{3}=
\dfrac{9\times 4}{26\times 3}=\dfrac{3 \times 2}{13}=\dfrac{6}{13}$
\item
課題\monban\vspace{1mm}\\
\eda{$\dfrac{4}{7}\div\dfrac{2}{3}$}\eda{$\dfrac{2}{5}\div\dfrac{4}{7}$}\vspace{1mm}\\
\eda{$\dfrac{7}{12}\div\dfrac{3}{8}$}\eda{$\dfrac{5}{4}\div\dfrac{15}{7}$}
\end{itemize}
%%%%%%%%%%%%

%%%%%%%%%%%%%%%%%%%%

\newslide{文字式}

\vspace*{18mm}

\slidepage

\begin{layer}{120}{0}
\putnotese{70}{14}{%%% /Users/takatoosetsuo/Dropbox/2022polytech/101/fig/presen22101rope.tex 
%%% Generator=presen22101.cdy 
{\unitlength=1cm%
\begin{picture}%
(4,0.8)(0,0)%
\linethickness{0.008in}%%
\Large\bf\boldmath%
\small%
\linethickness{0.024in}%%
\polyline(0.00000,0.25000)(4.00000,0.25000)%
%
\linethickness{0.008in}%%
\polyline(4.00000,0.25000)(3.96322,0.26196)(3.92636,0.27369)(3.88943,0.28521)(3.85244,0.29650)%
(3.81538,0.30756)(3.77825,0.31840)(3.74105,0.32902)(3.70380,0.33942)(3.66648,0.34958)%
(3.62910,0.35953)(3.59166,0.36924)(3.55417,0.37874)(3.51661,0.38800)(3.47900,0.39704)%
(3.44134,0.40585)(3.40363,0.41443)(3.36586,0.42279)(3.32805,0.43092)(3.29018,0.43882)%
(3.25227,0.44649)(3.21431,0.45393)(3.17631,0.46114)(3.13827,0.46813)(3.10018,0.47488)%
(3.06206,0.48140)(3.02390,0.48770)(2.98570,0.49376)(2.94746,0.49959)(2.90919,0.50519)%
(2.87088,0.51056)(2.83255,0.51570)(2.79418,0.52061)(2.75578,0.52529)(2.71736,0.52973)%
(2.67891,0.53395)(2.64044,0.53793)(2.60194,0.54167)(2.56342,0.54519)(2.52488,0.54847)%
(2.48632,0.55152)(2.44775,0.55434)(2.40916,0.55693)(2.37055,0.55928)(2.33193,0.56140)%
(2.29329,0.56329)(2.25465,0.56494)(2.21600,0.56636)(2.17734,0.56755)(2.13867,0.56850)%
(2.10000,0.56922)%
%
\polyline(1.90000,0.56922)(1.86133,0.56850)(1.82266,0.56755)(1.78400,0.56636)(1.74535,0.56494)%
(1.70671,0.56329)(1.66807,0.56140)(1.62945,0.55928)(1.59084,0.55693)(1.55225,0.55434)%
(1.51368,0.55152)(1.47512,0.54847)(1.43658,0.54519)(1.39806,0.54167)(1.35956,0.53793)%
(1.32109,0.53395)(1.28264,0.52973)(1.24422,0.52529)(1.20582,0.52061)(1.16745,0.51570)%
(1.12912,0.51056)(1.09081,0.50519)(1.05254,0.49959)(1.01430,0.49376)(0.97610,0.48770)%
(0.93794,0.48140)(0.89982,0.47488)(0.86173,0.46813)(0.82369,0.46114)(0.78569,0.45393)%
(0.74773,0.44649)(0.70982,0.43882)(0.67195,0.43092)(0.63414,0.42279)(0.59637,0.41443)%
(0.55866,0.40585)(0.52100,0.39704)(0.48339,0.38800)(0.44583,0.37874)(0.40834,0.36924)%
(0.37090,0.35953)(0.33352,0.34958)(0.29620,0.33942)(0.25895,0.32902)(0.22175,0.31840)%
(0.18462,0.30756)(0.14756,0.29650)(0.11057,0.28521)(0.07364,0.27369)(0.03678,0.26196)%
(-0.00000,0.25000)%
%
\settowidth{\Width}{$a$}\setlength{\Width}{-0.5\Width}%
\settoheight{\Height}{$a$}\settodepth{\Depth}{$a$}\setlength{\Height}{-0.5\Height}\setlength{\Depth}{0.5\Depth}\addtolength{\Height}{\Depth}%
\put(2.0000000,0.5700000){\hspace*{\Width}\raisebox{\Height}{$a$}}%
%
\polyline(2.50000,0.25000)(2.51238,0.24489)(2.52480,0.23986)(2.53726,0.23493)(2.54974,0.23007)%
(2.56226,0.22530)(2.57481,0.22062)(2.58740,0.21603)(2.60001,0.21152)(2.61266,0.20710)%
(2.62534,0.20276)(2.63804,0.19851)(2.65078,0.19435)(2.66354,0.19028)(2.67633,0.18629)%
(2.68915,0.18240)(2.70199,0.17859)(2.71486,0.17487)(2.72775,0.17123)(2.74067,0.16769)%
(2.75362,0.16424)(2.76658,0.16087)(2.77957,0.15759)(2.79259,0.15441)(2.80562,0.15131)%
(2.81868,0.14830)(2.83175,0.14538)(2.84485,0.14255)(2.85796,0.13982)(2.87109,0.13717)%
(2.88424,0.13461)(2.89741,0.13214)(2.91059,0.12976)(2.92379,0.12748)(2.93701,0.12528)%
(2.95024,0.12318)(2.96349,0.12116)(2.97674,0.11924)(2.99001,0.11741)(3.00330,0.11567)%
(3.01659,0.11402)(3.02990,0.11246)(3.04321,0.11100)(3.05654,0.10962)(3.06988,0.10834)%
(3.08322,0.10715)(3.09657,0.10605)(3.10993,0.10504)(3.12330,0.10412)(3.13667,0.10330)%
(3.15004,0.10256)%
%
\polyline(3.34996,0.10256)(3.36333,0.10330)(3.37670,0.10412)(3.39007,0.10504)(3.40343,0.10605)%
(3.41678,0.10715)(3.43012,0.10834)(3.44346,0.10962)(3.45679,0.11100)(3.47010,0.11246)%
(3.48341,0.11402)(3.49670,0.11567)(3.50999,0.11741)(3.52326,0.11924)(3.53651,0.12116)%
(3.54976,0.12318)(3.56299,0.12528)(3.57621,0.12748)(3.58941,0.12976)(3.60259,0.13214)%
(3.61576,0.13461)(3.62891,0.13717)(3.64204,0.13982)(3.65515,0.14255)(3.66825,0.14538)%
(3.68132,0.14830)(3.69438,0.15131)(3.70741,0.15441)(3.72043,0.15759)(3.73342,0.16087)%
(3.74638,0.16424)(3.75933,0.16769)(3.77225,0.17123)(3.78514,0.17487)(3.79801,0.17859)%
(3.81085,0.18240)(3.82367,0.18629)(3.83646,0.19028)(3.84922,0.19435)(3.86196,0.19851)%
(3.87466,0.20276)(3.88734,0.20710)(3.89999,0.21152)(3.91260,0.21603)(3.92519,0.22062)%
(3.93774,0.22530)(3.95026,0.23007)(3.96274,0.23493)(3.97520,0.23986)(3.98762,0.24489)%
(4.00000,0.25000)%
%
\settowidth{\Width}{$x$}\setlength{\Width}{-0.5\Width}%
\settoheight{\Height}{$x$}\settodepth{\Depth}{$x$}\setlength{\Height}{-0.5\Height}\setlength{\Depth}{0.5\Depth}\addtolength{\Height}{\Depth}%
\put(3.2500000,0.1000000){\hspace*{\Width}\raisebox{\Height}{$x$}}%
%
\linethickness{0.004in}%%
\polyline(-0.00000,0.25000)(0.02148,0.24392)(0.04299,0.23795)(0.06453,0.23208)(0.08610,0.22632)%
(0.10770,0.22066)(0.12933,0.21511)(0.15098,0.20967)(0.17265,0.20433)(0.19436,0.19910)%
(0.21609,0.19398)(0.23784,0.18896)(0.25962,0.18405)(0.28142,0.17925)(0.30325,0.17455)%
(0.32510,0.16996)(0.34697,0.16548)(0.36886,0.16110)(0.39077,0.15683)(0.41271,0.15267)%
(0.43466,0.14862)(0.45664,0.14467)(0.47863,0.14084)(0.50064,0.13711)(0.52267,0.13349)%
(0.54472,0.12997)(0.56678,0.12657)(0.58886,0.12327)(0.61096,0.12008)(0.63307,0.11700)%
(0.65520,0.11403)(0.67734,0.11117)(0.69949,0.10841)(0.72166,0.10577)(0.74384,0.10323)%
(0.76604,0.10080)(0.78824,0.09848)(0.81046,0.09627)(0.83268,0.09417)(0.85492,0.09218)%
(0.87716,0.09030)(0.89942,0.08852)(0.92168,0.08686)(0.94395,0.08530)(0.96623,0.08385)%
(0.98852,0.08252)(1.01081,0.08129)(1.03311,0.08017)(1.05541,0.07916)(1.07772,0.07826)%
(1.10003,0.07747)%
%
\polyline(1.39997,0.07747)(1.42228,0.07826)(1.44459,0.07916)(1.46689,0.08017)(1.48919,0.08129)%
(1.51148,0.08252)(1.53377,0.08385)(1.55605,0.08530)(1.57832,0.08686)(1.60058,0.08852)%
(1.62284,0.09030)(1.64508,0.09218)(1.66732,0.09417)(1.68954,0.09627)(1.71176,0.09848)%
(1.73396,0.10080)(1.75616,0.10323)(1.77834,0.10577)(1.80051,0.10841)(1.82266,0.11117)%
(1.84480,0.11403)(1.86693,0.11700)(1.88904,0.12008)(1.91114,0.12327)(1.93322,0.12657)%
(1.95528,0.12997)(1.97733,0.13349)(1.99936,0.13711)(2.02137,0.14084)(2.04336,0.14467)%
(2.06534,0.14862)(2.08729,0.15267)(2.10923,0.15683)(2.13114,0.16110)(2.15303,0.16548)%
(2.17490,0.16996)(2.19675,0.17455)(2.21858,0.17925)(2.24038,0.18405)(2.26216,0.18896)%
(2.28391,0.19398)(2.30564,0.19910)(2.32735,0.20433)(2.34902,0.20967)(2.37067,0.21511)%
(2.39230,0.22066)(2.41390,0.22632)(2.43547,0.23208)(2.45701,0.23795)(2.47852,0.24392)%
(2.50000,0.25000)%
%
\linethickness{0.008in}%%
\settowidth{\Width}{$b$}\setlength{\Width}{-0.5\Width}%
\settoheight{\Height}{$b$}\settodepth{\Depth}{$b$}\setlength{\Height}{-0.5\Height}\setlength{\Depth}{0.5\Depth}\addtolength{\Height}{\Depth}%
\put(1.2500000,0.0800000){\hspace*{\Width}\raisebox{\Height}{$b$}}%
%
\end{picture}}%}
\end{layer}

\seteda{110}
\begin{itemize}
\item
長さ$a$のひもから長さ$x$のひもを切り取ったときの残りの長さを$b$とする.\\
\hspace*{4zw}$b=a-x$
\item
課題\monbannoadd 次を文字式で表せ\hfill(TextP4)\\
\eda{1辺の長さが$a$\;cmである正方形の面積$S=$}\\
\eda{円周率$\pi$,半径が$r$である円の円周$L=$}
\end{itemize}
\addban

\newslide{文字式の計算}

\vspace*{18mm}

\slidepage
\begin{itemize}
\item
掛け算記号は省略 $x\cdot y,\ x\times y\ \Longrightarrow\ xy$
\item
べき乗 $xx,\ xxx\ \Longrightarrow\ x^2,\ x^3$
\item
数は文字の前におく $x\cdot 3\cdot y\cdot 4=12xy$
\item
計算は,数の場合と同様\\
\hspace*{2zw}$3a\times (-7a^2)=3\cdot(-7) aa^2=-21 a^3$
\item
課題\monbannoadd 次の計算をせよ.\hfill(TextP5)\seteda{55}\\
\eda{$-\dfrac{9}{2}a\times\left(-\dfrac{5}{6}b\right)$}%
\eda{$\dfrac{2}{3}a\times(-3a)^2$}
\end{itemize}
\addban

\mainslide{関数}


\slidepage[m]
%%%%%%%%%%%%

%%%%%%%%%%%%%%%%%%%%

\newslide{関数}

\vspace*{18mm}

\slidepage
\begin{itemize}
\item
変数$x$の値を与えると変数$y$の値が求まる\\
\hspace*{1zw}例)$y=2x+1,\ y=x^2+2x+1$
\item
これを変数$x$の関数という
\item
変数$x$の関数であることを$f(x)$などで表す\\
 例1)$f(x)=2x+1$(1次関数)\\
 例2)$g(x)=x^2+2x+1$(2次関数)
\end{itemize}

\newslide{関数記号}

\vspace*{18mm}

\slidepage
\seteda{55}
\begin{itemize}
\item
関数$f(x)$の$x$に定数$a$を代入した値を$f(a)$で表す
\item
例)$f(x)=x^2+x-1$のとき\\
   $f(2)=2^2+2-1=5$
\item
課題\monban $f(x)=3x+1$のとき,次を求めよ.\\
\eda{$f(0)$}\eda{$f(2)$}\\
\eda{$f(-3)$}\eda{$f(a-1)$($a$は定数)}\\
\hfill(TextP80問1,2)\\
\end{itemize}

\newslide{関数のグラフ}

\vspace*{18mm}

%%repeat=8,para
\slidepage
\vspace{5mm}

関数$y=f(x)$
\begin{itemize}
\item
$x$を変えるとき,点$\bigl(x,\ f(x)\bigr)$も変わる.
\item
[]例) 1次関数$y=2x+1$\vspace{1mm}\\
%%% /polytech22.git/102(0418)/presen/fig/table1a.tex 
%%% Generator=presen22102.cdy 
{\unitlength=1cm%
\begin{picture}%
(9.6,1.2)(0,0)%
\linethickness{0.008in}%%
\Large\bf\boldmath%
\small%
\polyline(0.00000,1.20000)(0.00000,0.00000)%
%
\polyline(0.80000,1.20000)(0.80000,0.00000)%
%
\polyline(1.60000,1.20000)(1.60000,0.00000)%
%
\polyline(2.40000,1.20000)(2.40000,0.00000)%
%
\polyline(3.20000,1.20000)(3.20000,0.00000)%
%
\polyline(4.00000,1.20000)(4.00000,0.00000)%
%
\polyline(4.80000,1.20000)(4.80000,0.00000)%
%
\polyline(5.60000,1.20000)(5.60000,0.00000)%
%
\polyline(6.40000,1.20000)(6.40000,0.00000)%
%
\polyline(7.20000,1.20000)(7.20000,0.00000)%
%
\polyline(8.00000,1.20000)(8.00000,0.00000)%
%
\polyline(8.80000,1.20000)(8.80000,0.00000)%
%
\polyline(9.60000,1.20000)(9.60000,0.00000)%
%
\polyline(0.00000,1.20000)(9.60000,1.20000)%
%
\polyline(0.00000,0.60000)(9.60000,0.60000)%
%
\polyline(0.00000,0.00000)(9.60000,0.00000)%
%
\settowidth{\Width}{$x$}\setlength{\Width}{-0.5\Width}%
\settoheight{\Height}{$x$}\settodepth{\Depth}{$x$}\setlength{\Height}{-0.5\Height}\setlength{\Depth}{0.5\Depth}\addtolength{\Height}{\Depth}%
\put(0.4000000,0.9000000){\hspace*{\Width}\raisebox{\Height}{$x$}}%
%
\settowidth{\Width}{$-5$}\setlength{\Width}{-0.5\Width}%
\settoheight{\Height}{$-5$}\settodepth{\Depth}{$-5$}\setlength{\Height}{-0.5\Height}\setlength{\Depth}{0.5\Depth}\addtolength{\Height}{\Depth}%
\put(1.2000000,0.9000000){\hspace*{\Width}\raisebox{\Height}{$-5$}}%
%
\settowidth{\Width}{$-4$}\setlength{\Width}{-0.5\Width}%
\settoheight{\Height}{$-4$}\settodepth{\Depth}{$-4$}\setlength{\Height}{-0.5\Height}\setlength{\Depth}{0.5\Depth}\addtolength{\Height}{\Depth}%
\put(2.0000000,0.9000000){\hspace*{\Width}\raisebox{\Height}{$-4$}}%
%
\settowidth{\Width}{$-3$}\setlength{\Width}{-0.5\Width}%
\settoheight{\Height}{$-3$}\settodepth{\Depth}{$-3$}\setlength{\Height}{-0.5\Height}\setlength{\Depth}{0.5\Depth}\addtolength{\Height}{\Depth}%
\put(2.8000000,0.9000000){\hspace*{\Width}\raisebox{\Height}{$-3$}}%
%
\settowidth{\Width}{$-2$}\setlength{\Width}{-0.5\Width}%
\settoheight{\Height}{$-2$}\settodepth{\Depth}{$-2$}\setlength{\Height}{-0.5\Height}\setlength{\Depth}{0.5\Depth}\addtolength{\Height}{\Depth}%
\put(3.6000000,0.9000000){\hspace*{\Width}\raisebox{\Height}{$-2$}}%
%
\settowidth{\Width}{$-1$}\setlength{\Width}{-0.5\Width}%
\settoheight{\Height}{$-1$}\settodepth{\Depth}{$-1$}\setlength{\Height}{-0.5\Height}\setlength{\Depth}{0.5\Depth}\addtolength{\Height}{\Depth}%
\put(4.4000000,0.9000000){\hspace*{\Width}\raisebox{\Height}{$-1$}}%
%
\settowidth{\Width}{$0$}\setlength{\Width}{-0.5\Width}%
\settoheight{\Height}{$0$}\settodepth{\Depth}{$0$}\setlength{\Height}{-0.5\Height}\setlength{\Depth}{0.5\Depth}\addtolength{\Height}{\Depth}%
\put(5.2000000,0.9000000){\hspace*{\Width}\raisebox{\Height}{$0$}}%
%
\settowidth{\Width}{$1$}\setlength{\Width}{-0.5\Width}%
\settoheight{\Height}{$1$}\settodepth{\Depth}{$1$}\setlength{\Height}{-0.5\Height}\setlength{\Depth}{0.5\Depth}\addtolength{\Height}{\Depth}%
\put(6.0000000,0.9000000){\hspace*{\Width}\raisebox{\Height}{$1$}}%
%
\settowidth{\Width}{$2$}\setlength{\Width}{-0.5\Width}%
\settoheight{\Height}{$2$}\settodepth{\Depth}{$2$}\setlength{\Height}{-0.5\Height}\setlength{\Depth}{0.5\Depth}\addtolength{\Height}{\Depth}%
\put(6.8000000,0.9000000){\hspace*{\Width}\raisebox{\Height}{$2$}}%
%
\settowidth{\Width}{$3$}\setlength{\Width}{-0.5\Width}%
\settoheight{\Height}{$3$}\settodepth{\Depth}{$3$}\setlength{\Height}{-0.5\Height}\setlength{\Depth}{0.5\Depth}\addtolength{\Height}{\Depth}%
\put(7.6000000,0.9000000){\hspace*{\Width}\raisebox{\Height}{$3$}}%
%
\settowidth{\Width}{$4$}\setlength{\Width}{-0.5\Width}%
\settoheight{\Height}{$4$}\settodepth{\Depth}{$4$}\setlength{\Height}{-0.5\Height}\setlength{\Depth}{0.5\Depth}\addtolength{\Height}{\Depth}%
\put(8.4000000,0.9000000){\hspace*{\Width}\raisebox{\Height}{$4$}}%
%
\settowidth{\Width}{$5$}\setlength{\Width}{-0.5\Width}%
\settoheight{\Height}{$5$}\settodepth{\Depth}{$5$}\setlength{\Height}{-0.5\Height}\setlength{\Depth}{0.5\Depth}\addtolength{\Height}{\Depth}%
\put(9.2000000,0.9000000){\hspace*{\Width}\raisebox{\Height}{$5$}}%
%
\settowidth{\Width}{$y$}\setlength{\Width}{-0.5\Width}%
\settoheight{\Height}{$y$}\settodepth{\Depth}{$y$}\setlength{\Height}{-0.5\Height}\setlength{\Depth}{0.5\Depth}\addtolength{\Height}{\Depth}%
\put(0.4000000,0.3000000){\hspace*{\Width}\raisebox{\Height}{$y$}}%
%
\end{picture}}%
\item
この点の集まりを,その関数の{\color{red}グラフ}という.
\end{itemize}
\if 1=0 % to end
%%%%%%%%%%%%

%%%%%%%%%%%%%%%%%%%%

\newslide{1次関数のグラフ}

\vspace*{18mm}

\slidepage
例)$y=2x+1$

\begin{layer}{120}{0}
\putnotese{70}{-3}{\scalebox{0.5}{%%% /polytech.git/n101/fig/table1b.tex 
%%% Generator=presen0601.cdy 
{\unitlength=1cm%
\begin{picture}%
(9.6,1.2)(0,0)%
\special{pn 8}%
%
\Large\bf\boldmath%
\small%
{%
\color{red}%
\special{pa     0  -472}\special{pa     0    -0}%
\special{fp}%
}%
{%
\color{red}%
\special{pa   315  -472}\special{pa   315    -0}%
\special{fp}%
}%
{%
\color{red}%
\special{pa   630  -472}\special{pa   630    -0}%
\special{fp}%
}%
{%
\color{red}%
\special{pa   945  -472}\special{pa   945    -0}%
\special{fp}%
}%
{%
\color{red}%
\special{pa  1260  -472}\special{pa  1260    -0}%
\special{fp}%
}%
{%
\color{red}%
\special{pa  1575  -472}\special{pa  1575    -0}%
\special{fp}%
}%
{%
\color{red}%
\special{pa  1890  -472}\special{pa  1890    -0}%
\special{fp}%
}%
{%
\color{red}%
\special{pa  2205  -472}\special{pa  2205    -0}%
\special{fp}%
}%
{%
\color{red}%
\special{pa  2520  -472}\special{pa  2520    -0}%
\special{fp}%
}%
{%
\color{red}%
\special{pa  2835  -472}\special{pa  2835    -0}%
\special{fp}%
}%
{%
\color{red}%
\special{pa  3150  -472}\special{pa  3150    -0}%
\special{fp}%
}%
{%
\color{red}%
\special{pa  3465  -472}\special{pa  3465    -0}%
\special{fp}%
}%
{%
\color{red}%
\special{pa  3780  -472}\special{pa  3780    -0}%
\special{fp}%
}%
{%
\color{red}%
\special{pa     0  -472}\special{pa  3780  -472}%
\special{fp}%
}%
{%
\color{red}%
\special{pa     0  -236}\special{pa  3780  -236}%
\special{fp}%
}%
{%
\color{red}%
\special{pa     0    -0}\special{pa  3780    -0}%
\special{fp}%
}%
{%
\color{red}%
\settowidth{\Width}{$x$}\setlength{\Width}{-0.5\Width}%
\settoheight{\Height}{$x$}\settodepth{\Depth}{$x$}\setlength{\Height}{-0.5\Height}\setlength{\Depth}{0.5\Depth}\addtolength{\Height}{\Depth}%
\put(0.4000000,0.9000000){\hspace*{\Width}\raisebox{\Height}{$x$}}%
%
}%
{%
\color{red}%
\settowidth{\Width}{$-5$}\setlength{\Width}{-0.5\Width}%
\settoheight{\Height}{$-5$}\settodepth{\Depth}{$-5$}\setlength{\Height}{-0.5\Height}\setlength{\Depth}{0.5\Depth}\addtolength{\Height}{\Depth}%
\put(1.2000000,0.9000000){\hspace*{\Width}\raisebox{\Height}{$-5$}}%
%
}%
{%
\color{red}%
\settowidth{\Width}{$-4$}\setlength{\Width}{-0.5\Width}%
\settoheight{\Height}{$-4$}\settodepth{\Depth}{$-4$}\setlength{\Height}{-0.5\Height}\setlength{\Depth}{0.5\Depth}\addtolength{\Height}{\Depth}%
\put(2.0000000,0.9000000){\hspace*{\Width}\raisebox{\Height}{$-4$}}%
%
}%
{%
\color{red}%
\settowidth{\Width}{$-3$}\setlength{\Width}{-0.5\Width}%
\settoheight{\Height}{$-3$}\settodepth{\Depth}{$-3$}\setlength{\Height}{-0.5\Height}\setlength{\Depth}{0.5\Depth}\addtolength{\Height}{\Depth}%
\put(2.8000000,0.9000000){\hspace*{\Width}\raisebox{\Height}{$-3$}}%
%
}%
{%
\color{red}%
\settowidth{\Width}{$-2$}\setlength{\Width}{-0.5\Width}%
\settoheight{\Height}{$-2$}\settodepth{\Depth}{$-2$}\setlength{\Height}{-0.5\Height}\setlength{\Depth}{0.5\Depth}\addtolength{\Height}{\Depth}%
\put(3.6000000,0.9000000){\hspace*{\Width}\raisebox{\Height}{$-2$}}%
%
}%
{%
\color{red}%
\settowidth{\Width}{$-1$}\setlength{\Width}{-0.5\Width}%
\settoheight{\Height}{$-1$}\settodepth{\Depth}{$-1$}\setlength{\Height}{-0.5\Height}\setlength{\Depth}{0.5\Depth}\addtolength{\Height}{\Depth}%
\put(4.4000000,0.9000000){\hspace*{\Width}\raisebox{\Height}{$-1$}}%
%
}%
{%
\color{red}%
\settowidth{\Width}{$0$}\setlength{\Width}{-0.5\Width}%
\settoheight{\Height}{$0$}\settodepth{\Depth}{$0$}\setlength{\Height}{-0.5\Height}\setlength{\Depth}{0.5\Depth}\addtolength{\Height}{\Depth}%
\put(5.2000000,0.9000000){\hspace*{\Width}\raisebox{\Height}{$0$}}%
%
}%
{%
\color{red}%
\settowidth{\Width}{$1$}\setlength{\Width}{-0.5\Width}%
\settoheight{\Height}{$1$}\settodepth{\Depth}{$1$}\setlength{\Height}{-0.5\Height}\setlength{\Depth}{0.5\Depth}\addtolength{\Height}{\Depth}%
\put(6.0000000,0.9000000){\hspace*{\Width}\raisebox{\Height}{$1$}}%
%
}%
{%
\color{red}%
\settowidth{\Width}{$2$}\setlength{\Width}{-0.5\Width}%
\settoheight{\Height}{$2$}\settodepth{\Depth}{$2$}\setlength{\Height}{-0.5\Height}\setlength{\Depth}{0.5\Depth}\addtolength{\Height}{\Depth}%
\put(6.8000000,0.9000000){\hspace*{\Width}\raisebox{\Height}{$2$}}%
%
}%
{%
\color{red}%
\settowidth{\Width}{$3$}\setlength{\Width}{-0.5\Width}%
\settoheight{\Height}{$3$}\settodepth{\Depth}{$3$}\setlength{\Height}{-0.5\Height}\setlength{\Depth}{0.5\Depth}\addtolength{\Height}{\Depth}%
\put(7.6000000,0.9000000){\hspace*{\Width}\raisebox{\Height}{$3$}}%
%
}%
{%
\color{red}%
\settowidth{\Width}{$4$}\setlength{\Width}{-0.5\Width}%
\settoheight{\Height}{$4$}\settodepth{\Depth}{$4$}\setlength{\Height}{-0.5\Height}\setlength{\Depth}{0.5\Depth}\addtolength{\Height}{\Depth}%
\put(8.4000000,0.9000000){\hspace*{\Width}\raisebox{\Height}{$4$}}%
%
}%
{%
\color{red}%
\settowidth{\Width}{$5$}\setlength{\Width}{-0.5\Width}%
\settoheight{\Height}{$5$}\settodepth{\Depth}{$5$}\setlength{\Height}{-0.5\Height}\setlength{\Depth}{0.5\Depth}\addtolength{\Height}{\Depth}%
\put(9.2000000,0.9000000){\hspace*{\Width}\raisebox{\Height}{$5$}}%
%
}%
{%
\color{red}%
\settowidth{\Width}{$y$}\setlength{\Width}{-0.5\Width}%
\settoheight{\Height}{$y$}\settodepth{\Depth}{$y$}\setlength{\Height}{-0.5\Height}\setlength{\Depth}{0.5\Depth}\addtolength{\Height}{\Depth}%
\put(0.4000000,0.3000000){\hspace*{\Width}\raisebox{\Height}{$y$}}%
%
}%
{%
\color{red}%
\settowidth{\Width}{$-9$}\setlength{\Width}{-0.5\Width}%
\settoheight{\Height}{$-9$}\settodepth{\Depth}{$-9$}\setlength{\Height}{-0.5\Height}\setlength{\Depth}{0.5\Depth}\addtolength{\Height}{\Depth}%
\put(1.2000000,0.3000000){\hspace*{\Width}\raisebox{\Height}{$-9$}}%
%
}%
{%
\color{red}%
\settowidth{\Width}{$-7$}\setlength{\Width}{-0.5\Width}%
\settoheight{\Height}{$-7$}\settodepth{\Depth}{$-7$}\setlength{\Height}{-0.5\Height}\setlength{\Depth}{0.5\Depth}\addtolength{\Height}{\Depth}%
\put(2.0000000,0.3000000){\hspace*{\Width}\raisebox{\Height}{$-7$}}%
%
}%
{%
\color{red}%
\settowidth{\Width}{$-5$}\setlength{\Width}{-0.5\Width}%
\settoheight{\Height}{$-5$}\settodepth{\Depth}{$-5$}\setlength{\Height}{-0.5\Height}\setlength{\Depth}{0.5\Depth}\addtolength{\Height}{\Depth}%
\put(2.8000000,0.3000000){\hspace*{\Width}\raisebox{\Height}{$-5$}}%
%
}%
{%
\color{red}%
\settowidth{\Width}{$-3$}\setlength{\Width}{-0.5\Width}%
\settoheight{\Height}{$-3$}\settodepth{\Depth}{$-3$}\setlength{\Height}{-0.5\Height}\setlength{\Depth}{0.5\Depth}\addtolength{\Height}{\Depth}%
\put(3.6000000,0.3000000){\hspace*{\Width}\raisebox{\Height}{$-3$}}%
%
}%
{%
\color{red}%
\settowidth{\Width}{$-1$}\setlength{\Width}{-0.5\Width}%
\settoheight{\Height}{$-1$}\settodepth{\Depth}{$-1$}\setlength{\Height}{-0.5\Height}\setlength{\Depth}{0.5\Depth}\addtolength{\Height}{\Depth}%
\put(4.4000000,0.3000000){\hspace*{\Width}\raisebox{\Height}{$-1$}}%
%
}%
{%
\color{red}%
\settowidth{\Width}{$1$}\setlength{\Width}{-0.5\Width}%
\settoheight{\Height}{$1$}\settodepth{\Depth}{$1$}\setlength{\Height}{-0.5\Height}\setlength{\Depth}{0.5\Depth}\addtolength{\Height}{\Depth}%
\put(5.2000000,0.3000000){\hspace*{\Width}\raisebox{\Height}{$1$}}%
%
}%
{%
\color{red}%
\settowidth{\Width}{$3$}\setlength{\Width}{-0.5\Width}%
\settoheight{\Height}{$3$}\settodepth{\Depth}{$3$}\setlength{\Height}{-0.5\Height}\setlength{\Depth}{0.5\Depth}\addtolength{\Height}{\Depth}%
\put(6.0000000,0.3000000){\hspace*{\Width}\raisebox{\Height}{$3$}}%
%
}%
{%
\color{red}%
\settowidth{\Width}{$5$}\setlength{\Width}{-0.5\Width}%
\settoheight{\Height}{$5$}\settodepth{\Depth}{$5$}\setlength{\Height}{-0.5\Height}\setlength{\Depth}{0.5\Depth}\addtolength{\Height}{\Depth}%
\put(6.8000000,0.3000000){\hspace*{\Width}\raisebox{\Height}{$5$}}%
%
}%
{%
\color{red}%
\settowidth{\Width}{$7$}\setlength{\Width}{-0.5\Width}%
\settoheight{\Height}{$7$}\settodepth{\Depth}{$7$}\setlength{\Height}{-0.5\Height}\setlength{\Depth}{0.5\Depth}\addtolength{\Height}{\Depth}%
\put(7.6000000,0.3000000){\hspace*{\Width}\raisebox{\Height}{$7$}}%
%
}%
{%
\color{red}%
\settowidth{\Width}{$9$}\setlength{\Width}{-0.5\Width}%
\settoheight{\Height}{$9$}\settodepth{\Depth}{$9$}\setlength{\Height}{-0.5\Height}\setlength{\Depth}{0.5\Depth}\addtolength{\Height}{\Depth}%
\put(8.4000000,0.3000000){\hspace*{\Width}\raisebox{\Height}{$9$}}%
%
}%
{%
\color{red}%
\settowidth{\Width}{$11$}\setlength{\Width}{-0.5\Width}%
\settoheight{\Height}{$11$}\settodepth{\Depth}{$11$}\setlength{\Height}{-0.5\Height}\setlength{\Depth}{0.5\Depth}\addtolength{\Height}{\Depth}%
\put(9.2000000,0.3000000){\hspace*{\Width}\raisebox{\Height}{$11$}}%
%
}%
\end{picture}}%}}
\putnotes{60}{6}{\scalebox{0.5}{\input{fig/graphpaper3.tex}}}
\end{layer}


\newslide{1次関数のグラフ}

\vspace*{18mm}

%%repeat=3
\slidepage
資料の「関数のグラフ」で次の1次関数のグラフをかいてみよう.
\begin{itemize}
\item
$y=3x+3$(TextP81)
\item
$y=10-2x$(TextP81)
\item
$y=2x+2$(TextP81)
\item
$y=\dfrac{1}{2}x+1$
\end{itemize}
\fi %from 259
\label{pageend}\mbox{}

\end{document}

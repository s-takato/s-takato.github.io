%%% Title presen22105
\documentclass[landscape,10pt]{ujarticle}
\special{papersize=\the\paperwidth,\the\paperheight}
\usepackage{ketpic,ketlayer}
\usepackage{ketslide}
\usepackage{amsmath,amssymb}
\usepackage{bm,enumerate}
\usepackage[dvipdfmx]{graphicx}
\usepackage{color}
\definecolor{slidecolora}{cmyk}{0.98,0.13,0,0.43}
\definecolor{slidecolorb}{cmyk}{0.2,0,0,0}
\definecolor{slidecolorc}{cmyk}{0.2,0,0,0}
\definecolor{slidecolord}{cmyk}{0.2,0,0,0}
\definecolor{slidecolore}{cmyk}{0,0,0,0.5}
\definecolor{slidecolorf}{cmyk}{0,0,0,0.5}
\definecolor{slidecolori}{cmyk}{0.98,0.13,0,0.43}
\def\setthin#1{\def\thin{#1}}
\setthin{0}
\newcommand{\slidepage}[1][s]{%
\setcounter{ketpicctra}{18}%
\if#1m \setcounter{ketpicctra}{1}\fi
\hypersetup{linkcolor=black}%

\begin{layer}{118}{0}
\putnotee{122}{-\theketpicctra.05}{\small\thepage/\pageref{pageend}}
\end{layer}\hypersetup{linkcolor=blue}

}
\usepackage{emath}
\usepackage{emathEy}
\usepackage{emathMw}
\usepackage{pict2e}
\usepackage{ketlayermorewith2e}
\usepackage[dvipdfmx,colorlinks=true,linkcolor=blue,filecolor=blue]{hyperref}
\newcommand{\hiduke}{0516}
\newcommand{\hako}[2][1]{\fbox{\raisebox{#1mm}{\mbox{}}\raisebox{-#1mm}{\mbox{}}\,\phantom{#2}\,}}
\newcommand{\hakoa}[2][1]{\fbox{\raisebox{#1mm}{\mbox{}}\raisebox{-#1mm}{\mbox{}}\,#2\,}}
\newcommand{\hakom}[2][1]{\hako[#1]{$#2$}}
\newcommand{\hakoma}[2][1]{\hakoa[#1]{$#2$}}
\def\rad{\;\mathrm{rad}}
\def\deg#1{#1^{\circ}}
\newcommand{\sbunsuu}[2]{\scalebox{0.6}{$\bunsuu{#1}{#2}$}}
\def\pow{$\hspace{-1.5mm}^\hspace{-1mm}$}
\def\dlim{\displaystyle\lim}
\newcommand{\brd}[2][1]{\scalebox{#1}{\color{red}\fbox{\color{black}$#2$}}}
\newcommand\down[1][0.5zw]{\vspace{#1}\\}

\setmargin{25}{145}{15}{100}

\ketslideinit

\pagestyle{empty}

\begin{document}

\begin{layer}{120}{0}
\putnotese{0}{0}{{\Large\bf
\color[cmyk]{1,1,0,0}

\begin{layer}{120}{0}
{\Huge \putnotes{60}{20}{三角関数の性質}}
\putnotes{60}{70}{2022.05.16}
\end{layer}

}
}
\end{layer}

\def\mainslidetitley{22}
\def\ketcletter{slidecolora}
\def\ketcbox{slidecolorb}
\def\ketdbox{slidecolorc}
\def\ketcframe{slidecolord}
\def\ketcshadow{slidecolore}
\def\ketdshadow{slidecolorf}
\def\slidetitlex{6}
\def\slidetitlesize{1.3}
\def\mketcletter{slidecolori}
\def\mketcbox{yellow}
\def\mketdbox{yellow}
\def\mketcframe{yellow}
\def\mslidetitlex{62}
\def\mslidetitlesize{2}

\color{black}
\Large\bf\boldmath
\addtocounter{page}{-1}

\def\MARU{}
\renewcommand{\MARU}[1]{{\ooalign{\hfil$#1$\/\hfil\crcr\raise.167ex\hbox{\mathhexbox20D}}}}
\renewcommand{\slidepage}[1][s]{%
\setcounter{ketpicctra}{18}%
\if#1m \setcounter{ketpicctra}{1}\fi
\hypersetup{linkcolor=black}%
\begin{layer}{118}{0}
\putnotee{115}{-\theketpicctra.05}{\small\hiduke-\thepage/\pageref{pageend}}
\end{layer}\hypersetup{linkcolor=blue}
}
\newcounter{ban}
\setcounter{ban}{1}
\newcommand{\monban}[1][\hiduke]{%
#1-\theban\ %
\addtocounter{ban}{1}%
}
\newcommand{\monbannoadd}[1][\hiduke]{%
#1-\theban\ %
}
\newcommand{\addban}{%
\addtocounter{ban}{1}%%210614
}
\newcounter{edawidth}
\newcounter{edactr}
\newcommand{\seteda}[1]{
\setcounter{edawidth}{#1}
\setcounter{edactr}{1}
}
\newcommand{\eda}[2][\theedawidth ]{%
\noindent\Ltab{#1 mm}{[\theedactr]\ #2}%
\addtocounter{edactr}{1}%
}
%%%%%%%%%%%%

%%%%%%%%%%%%%%%%%%%%

\mainslide{三角関数のグラフ}


\slidepage[m]
%%%%%%%%%%%%

%%%%%%%%%%%%%%%%%%%%

\newslide{$y=\sin x$のグラフ}

\vspace*{18mm}

\slidepage

\begin{layer}{120}{0}
\putnotese{75}{3}{\scalebox{0.9}{%%% /polytech22.git/104-0509/presen/fig/fig221046b.tex 
%%% Generator=fig22104.cdy 
{\unitlength=24mm%
\begin{picture}%
(2.4,2.4)(-1.2,-1.2)%
\linethickness{0.008in}%%
\polyline(1.00000,0.00000)(0.99211,0.12533)(0.96858,0.24869)(0.92978,0.36812)(0.87631,0.48175)%
(0.80902,0.58779)(0.72897,0.68455)(0.63742,0.77051)(0.53583,0.84433)(0.42578,0.90483)%
(0.30902,0.95106)(0.18738,0.98229)(0.06279,0.99803)(-0.06279,0.99803)(-0.18738,0.98229)%
(-0.30902,0.95106)(-0.42578,0.90483)(-0.53583,0.84433)(-0.63742,0.77051)(-0.72897,0.68455)%
(-0.80902,0.58779)(-0.87631,0.48175)(-0.92978,0.36812)(-0.96858,0.24869)(-0.99211,0.12533)%
(-1.00000,-0.00000)(-0.99211,-0.12533)(-0.96858,-0.24869)(-0.92978,-0.36812)(-0.87631,-0.48175)%
(-0.80902,-0.58779)(-0.72897,-0.68455)(-0.63742,-0.77051)(-0.53583,-0.84433)(-0.42578,-0.90483)%
(-0.30902,-0.95106)(-0.18738,-0.98229)(-0.06279,-0.99803)(0.06279,-0.99803)(0.18738,-0.98229)%
(0.30902,-0.95106)(0.42578,-0.90483)(0.53583,-0.84433)(0.63742,-0.77051)(0.72897,-0.68455)%
(0.80902,-0.58779)(0.87631,-0.48175)(0.92978,-0.36812)(0.96858,-0.24869)(0.99211,-0.12533)%
(1.00000,-0.00000)%
%
\polyline(0.00000,0.00000)(0.58779,0.80902)%
%
\settowidth{\Width}{$\mbox{P}(X,Y)$}\setlength{\Width}{0\Width}%
\settoheight{\Height}{$\mbox{P}(X,Y)$}\settodepth{\Depth}{$\mbox{P}(X,Y)$}\setlength{\Height}{\Depth}%
\put(0.6108333,0.8308333){\hspace*{\Width}\raisebox{\Height}{$\mbox{P}(X,Y)$}}%
%
\polyline(0.25000,0.00000)(0.24803,0.03133)(0.24215,0.06217)(0.23244,0.09203)(0.21908,0.12044)%
(0.20225,0.14695)(0.18224,0.17114)(0.15936,0.19263)(0.14666,0.20186)%
%
\settowidth{\Width}{$x$}\setlength{\Width}{-0.5\Width}%
\settoheight{\Height}{$x$}\settodepth{\Depth}{$x$}\setlength{\Height}{-0.5\Height}\setlength{\Depth}{0.5\Depth}\addtolength{\Height}{\Depth}%
\put(0.3200000,0.1600000){\hspace*{\Width}\raisebox{\Height}{$x$}}%
%
\polyline(-1.00000,0.02083)(-1.00000,-0.02083)%
%
\settowidth{\Width}{$-1$}\setlength{\Width}{-1\Width}%
\settoheight{\Height}{$-1$}\settodepth{\Depth}{$-1$}\setlength{\Height}{-\Height}%
\put(-1.0208333,-0.0208333){\hspace*{\Width}\raisebox{\Height}{$-1$}}%
%
\polyline(1.00000,0.02083)(1.00000,-0.02083)%
%
\settowidth{\Width}{$1$}\setlength{\Width}{0\Width}%
\settoheight{\Height}{$1$}\settodepth{\Depth}{$1$}\setlength{\Height}{-\Height}%
\put(1.0208333,-0.0208333){\hspace*{\Width}\raisebox{\Height}{$1$}}%
%
\polyline(0.02083,-1.00000)(-0.02083,-1.00000)%
%
\settowidth{\Width}{$-1$}\setlength{\Width}{-1\Width}%
\settoheight{\Height}{$-1$}\settodepth{\Depth}{$-1$}\setlength{\Height}{-\Height}%
\put(-0.0208333,-1.0208333){\hspace*{\Width}\raisebox{\Height}{$-1$}}%
%
\polyline(0.02083,1.00000)(-0.02083,1.00000)%
%
\settowidth{\Width}{$1$}\setlength{\Width}{-1\Width}%
\settoheight{\Height}{$1$}\settodepth{\Depth}{$1$}\setlength{\Height}{\Depth}%
\put(-0.0208333,1.0208333){\hspace*{\Width}\raisebox{\Height}{$1$}}%
%
{%
\color[cmyk]{0,1,1,0}%
\linethickness{0.016in}%%
\polyline(0.00000,0.00000)(0.58779,0.00000)%
%
\linethickness{0.008in}%%
}%
\settowidth{\Width}{(1)}\setlength{\Width}{-0.5\Width}%
\settoheight{\Height}{(1)}\settodepth{\Depth}{(1)}\setlength{\Height}{-\Height}%
\put(0.2900000,-0.0208333){\hspace*{\Width}\raisebox{\Height}{(1)}}%
%
{%
\color[cmyk]{1,0,0,0}%
\linethickness{0.016in}%%
\polyline(0.58779,0.80902)(0.58779,0.00000)%
%
\linethickness{0.008in}%%
}%
\settowidth{\Width}{(2)}\setlength{\Width}{-1\Width}%
\settoheight{\Height}{(2)}\settodepth{\Depth}{(2)}\setlength{\Height}{-0.5\Height}\setlength{\Depth}{0.5\Depth}\addtolength{\Height}{\Depth}%
\put(0.5900000,0.4000000){\hspace*{\Width}\raisebox{\Height}{(2)}}%
%
{%
\color[cmyk]{0,1,1,0}%
\linethickness{0.016in}%%
\polyline(0.58779,0.80902)(1.00000,0.00000)%
%
\linethickness{0.008in}%%
}%
\settowidth{\Width}{(3)}\setlength{\Width}{0\Width}%
\settoheight{\Height}{(3)}\settodepth{\Depth}{(3)}\setlength{\Height}{-\Height}%
\put(0.6525000,0.3791667){\hspace*{\Width}\raisebox{\Height}{(3)}}%
%
{%
\color[cmyk]{1,0,0,0}%
\linethickness{0.016in}%%
\polyline(1.00000,0.00000)(0.99982,0.01885)(0.99929,0.03769)(0.99840,0.05652)(0.99716,0.07533)%
(0.99556,0.09411)(0.99361,0.11286)(0.99131,0.13156)(0.98865,0.15023)(0.98564,0.16883)%
(0.98229,0.18738)(0.97858,0.20586)(0.97453,0.22427)(0.97013,0.24260)(0.96538,0.26084)%
(0.96029,0.27899)(0.95486,0.29704)(0.94910,0.31499)(0.94299,0.33282)(0.93655,0.35053)%
(0.92978,0.36812)(0.92267,0.38558)(0.91524,0.40291)(0.90748,0.42009)(0.89941,0.43712)%
(0.89101,0.45399)(0.88229,0.47070)(0.87326,0.48725)(0.86392,0.50362)(0.85428,0.51982)%
(0.84433,0.53583)(0.83408,0.55165)(0.82353,0.56727)(0.81269,0.58269)(0.80157,0.59790)%
(0.79016,0.61291)(0.77846,0.62769)(0.76649,0.64225)(0.75425,0.65659)(0.74174,0.67069)%
(0.72897,0.68455)(0.71594,0.69817)(0.70265,0.71154)(0.68911,0.72465)(0.67533,0.73751)%
(0.66131,0.75011)(0.64706,0.76244)(0.63257,0.77450)(0.61786,0.78629)(0.60293,0.79779)%
(0.58779,0.80902)%
%
\linethickness{0.008in}%%
}%
\settowidth{\Width}{(4)}\setlength{\Width}{0\Width}%
\settoheight{\Height}{(4)}\settodepth{\Depth}{(4)}\setlength{\Height}{\Depth}%
\put(0.8900000,0.4500000){\hspace*{\Width}\raisebox{\Height}{(4)}}%
%
\polyline(-1.20000,0.00000)(1.20000,0.00000)%
%
\polyline(0.00000,-1.20000)(0.00000,1.20000)%
%
\settowidth{\Width}{$ $}\setlength{\Width}{0\Width}%
\settoheight{\Height}{$ $}\settodepth{\Depth}{$ $}\setlength{\Height}{-0.5\Height}\setlength{\Depth}{0.5\Depth}\addtolength{\Height}{\Depth}%
\put(1.2208333,0.0000000){\hspace*{\Width}\raisebox{\Height}{$ $}}%
%
\settowidth{\Width}{$ $}\setlength{\Width}{-0.5\Width}%
\settoheight{\Height}{$ $}\settodepth{\Depth}{$ $}\setlength{\Height}{\Depth}%
\put(0.0000000,1.2208333){\hspace*{\Width}\raisebox{\Height}{$ $}}%
%
\settowidth{\Width}{O}\setlength{\Width}{-1\Width}%
\settoheight{\Height}{O}\settodepth{\Depth}{O}\setlength{\Height}{-\Height}%
\put(-0.0208333,-0.0208333){\hspace*{\Width}\raisebox{\Height}{O}}%
%
\end{picture}}%}}
\end{layer}

{\color{red}

\begin{layer}{120}{0}
\qarrowline[8]{44}{36}{33}{152}{40}
\circleline{46}{35}{1}
\qarrowline[8]{35}{55}{38}{122}{40}
\circleline{37}{55}{1}
\end{layer}

}
\begin{itemize}
\item
角$x$はラジアンで測る
\item
{\color{red}半径$1$}の円に点$\mathrm{P}(X,Y)$をとる
\item
[]\hspace*{3zw}$\sin x=\bunsuu{Y}{r}=Y$
\item
また弧の長さを$\ell$とすると\\
\hspace*{3zw}$x=\bunsuu{\ell}{r}=\ell$
\item
$x$は弧(4)の長さ,\ $\sin x$は(2)の長さ
\end{itemize}

\newslide{正弦曲線を描く}

\vspace*{18mm}

\slidepage
\begin{itemize}
\item
[]アプリ「$y=\sin x$のグラフ」を動かしてみよう\vspace{-2mm}
\item
[(1)]学生番号を入れてOKを押す\vspace{-2mm}
\item
[(2)]赤い点を動かして$x$を決め,「点を打つ」\\
 =>長さが$x$の弧を表示,$(x,\sin x)$に点が打たれる\vspace{-2mm}
\item
[(3)]いくつかの点を打って「点を結ぶ」\\
 正弦曲線との誤差が表示される\\
 さらに「点を打つ」,「点を結ぶ」を繰り返す.\vspace{-2mm}
\item
[課題]\monban 違いをできるだけ小さくして,{\color{red}点を結ぶ} を押してからRecを押し, 回答にコピペせよ.
\end{itemize}
%%%%%%%%%%%%

%%%%%%%%%%%%%%%%%%%%


\newslide{正弦曲線の特徴}

\vspace*{18mm}

\slidepage

\begin{layer}{120}{0}
\putnotes{60}{5}{%%% /polytech22.git/104-0509/presen/fig/graphsin.tex 
%%% Generator=graphsincos.cdy 
{\unitlength=1cm%
\begin{picture}%
(13,2.4)(-6.5,-1.2)%
\linethickness{0.008in}%%
\polyline(-6.50000,1.00000)(-6.40076,1.00000)\polyline(-6.30153,1.00000)(-6.20229,1.00000)%
\polyline(-6.10305,1.00000)(-6.00382,1.00000)\polyline(-5.90458,1.00000)(-5.80534,1.00000)%
\polyline(-5.70611,1.00000)(-5.60687,1.00000)\polyline(-5.50763,1.00000)(-5.40840,1.00000)%
\polyline(-5.30916,1.00000)(-5.20992,1.00000)\polyline(-5.11069,1.00000)(-5.01145,1.00000)%
\polyline(-4.91221,1.00000)(-4.81298,1.00000)\polyline(-4.71374,1.00000)(-4.61450,1.00000)%
\polyline(-4.51527,1.00000)(-4.41603,1.00000)\polyline(-4.31679,1.00000)(-4.21756,1.00000)%
\polyline(-4.11832,1.00000)(-4.01908,1.00000)\polyline(-3.91985,1.00000)(-3.82061,1.00000)%
\polyline(-3.72137,1.00000)(-3.62214,1.00000)\polyline(-3.52290,1.00000)(-3.42366,1.00000)%
\polyline(-3.32443,1.00000)(-3.22519,1.00000)\polyline(-3.12595,1.00000)(-3.02672,1.00000)%
\polyline(-2.92748,1.00000)(-2.82824,1.00000)\polyline(-2.72901,1.00000)(-2.62977,1.00000)%
\polyline(-2.53053,1.00000)(-2.43130,1.00000)\polyline(-2.33206,1.00000)(-2.23282,1.00000)%
\polyline(-2.13359,1.00000)(-2.03435,1.00000)\polyline(-1.93511,1.00000)(-1.83588,1.00000)%
\polyline(-1.73664,1.00000)(-1.63740,1.00000)\polyline(-1.53817,1.00000)(-1.43893,1.00000)%
\polyline(-1.33969,1.00000)(-1.24046,1.00000)\polyline(-1.14122,1.00000)(-1.04198,1.00000)%
\polyline(-0.94275,1.00000)(-0.84351,1.00000)\polyline(-0.74427,1.00000)(-0.64504,1.00000)%
\polyline(-0.54580,1.00000)(-0.44656,1.00000)\polyline(-0.34733,1.00000)(-0.24809,1.00000)%
\polyline(-0.14885,1.00000)(-0.04962,1.00000)\polyline(0.04962,1.00000)(0.14885,1.00000)%
\polyline(0.24809,1.00000)(0.34733,1.00000)\polyline(0.44656,1.00000)(0.54580,1.00000)%
\polyline(0.64504,1.00000)(0.74427,1.00000)\polyline(0.84351,1.00000)(0.94275,1.00000)%
\polyline(1.04198,1.00000)(1.14122,1.00000)\polyline(1.24046,1.00000)(1.33969,1.00000)%
\polyline(1.43893,1.00000)(1.53817,1.00000)\polyline(1.63740,1.00000)(1.73664,1.00000)%
\polyline(1.83588,1.00000)(1.93511,1.00000)\polyline(2.03435,1.00000)(2.13359,1.00000)%
\polyline(2.23282,1.00000)(2.33206,1.00000)\polyline(2.43130,1.00000)(2.53053,1.00000)%
\polyline(2.62977,1.00000)(2.72901,1.00000)\polyline(2.82824,1.00000)(2.92748,1.00000)%
\polyline(3.02672,1.00000)(3.12595,1.00000)\polyline(3.22519,1.00000)(3.32443,1.00000)%
\polyline(3.42366,1.00000)(3.52290,1.00000)\polyline(3.62214,1.00000)(3.72137,1.00000)%
\polyline(3.82061,1.00000)(3.91985,1.00000)\polyline(4.01908,1.00000)(4.11832,1.00000)%
\polyline(4.21756,1.00000)(4.31679,1.00000)\polyline(4.41603,1.00000)(4.51527,1.00000)%
\polyline(4.61450,1.00000)(4.71374,1.00000)\polyline(4.81298,1.00000)(4.91221,1.00000)%
\polyline(5.01145,1.00000)(5.11069,1.00000)\polyline(5.20992,1.00000)(5.30916,1.00000)%
\polyline(5.40840,1.00000)(5.50763,1.00000)\polyline(5.60687,1.00000)(5.70611,1.00000)%
\polyline(5.80534,1.00000)(5.90458,1.00000)\polyline(6.00382,1.00000)(6.10305,1.00000)%
\polyline(6.20229,1.00000)(6.30153,1.00000)\polyline(6.40076,1.00000)(6.50000,1.00000)%
%
%
\polyline(-6.50000,-1.00000)(-6.40076,-1.00000)\polyline(-6.30153,-1.00000)(-6.20229,-1.00000)%
\polyline(-6.10305,-1.00000)(-6.00382,-1.00000)\polyline(-5.90458,-1.00000)(-5.80534,-1.00000)%
\polyline(-5.70611,-1.00000)(-5.60687,-1.00000)\polyline(-5.50763,-1.00000)(-5.40840,-1.00000)%
\polyline(-5.30916,-1.00000)(-5.20992,-1.00000)\polyline(-5.11069,-1.00000)(-5.01145,-1.00000)%
\polyline(-4.91221,-1.00000)(-4.81298,-1.00000)\polyline(-4.71374,-1.00000)(-4.61450,-1.00000)%
\polyline(-4.51527,-1.00000)(-4.41603,-1.00000)\polyline(-4.31679,-1.00000)(-4.21756,-1.00000)%
\polyline(-4.11832,-1.00000)(-4.01908,-1.00000)\polyline(-3.91985,-1.00000)(-3.82061,-1.00000)%
\polyline(-3.72137,-1.00000)(-3.62214,-1.00000)\polyline(-3.52290,-1.00000)(-3.42366,-1.00000)%
\polyline(-3.32443,-1.00000)(-3.22519,-1.00000)\polyline(-3.12595,-1.00000)(-3.02672,-1.00000)%
\polyline(-2.92748,-1.00000)(-2.82824,-1.00000)\polyline(-2.72901,-1.00000)(-2.62977,-1.00000)%
\polyline(-2.53053,-1.00000)(-2.43130,-1.00000)\polyline(-2.33206,-1.00000)(-2.23282,-1.00000)%
\polyline(-2.13359,-1.00000)(-2.03435,-1.00000)\polyline(-1.93511,-1.00000)(-1.83588,-1.00000)%
\polyline(-1.73664,-1.00000)(-1.63740,-1.00000)\polyline(-1.53817,-1.00000)(-1.43893,-1.00000)%
\polyline(-1.33969,-1.00000)(-1.24046,-1.00000)\polyline(-1.14122,-1.00000)(-1.04198,-1.00000)%
\polyline(-0.94275,-1.00000)(-0.84351,-1.00000)\polyline(-0.74427,-1.00000)(-0.64504,-1.00000)%
\polyline(-0.54580,-1.00000)(-0.44656,-1.00000)\polyline(-0.34733,-1.00000)(-0.24809,-1.00000)%
\polyline(-0.14885,-1.00000)(-0.04962,-1.00000)\polyline(0.04962,-1.00000)(0.14885,-1.00000)%
\polyline(0.24809,-1.00000)(0.34733,-1.00000)\polyline(0.44656,-1.00000)(0.54580,-1.00000)%
\polyline(0.64504,-1.00000)(0.74427,-1.00000)\polyline(0.84351,-1.00000)(0.94275,-1.00000)%
\polyline(1.04198,-1.00000)(1.14122,-1.00000)\polyline(1.24046,-1.00000)(1.33969,-1.00000)%
\polyline(1.43893,-1.00000)(1.53817,-1.00000)\polyline(1.63740,-1.00000)(1.73664,-1.00000)%
\polyline(1.83588,-1.00000)(1.93511,-1.00000)\polyline(2.03435,-1.00000)(2.13359,-1.00000)%
\polyline(2.23282,-1.00000)(2.33206,-1.00000)\polyline(2.43130,-1.00000)(2.53053,-1.00000)%
\polyline(2.62977,-1.00000)(2.72901,-1.00000)\polyline(2.82824,-1.00000)(2.92748,-1.00000)%
\polyline(3.02672,-1.00000)(3.12595,-1.00000)\polyline(3.22519,-1.00000)(3.32443,-1.00000)%
\polyline(3.42366,-1.00000)(3.52290,-1.00000)\polyline(3.62214,-1.00000)(3.72137,-1.00000)%
\polyline(3.82061,-1.00000)(3.91985,-1.00000)\polyline(4.01908,-1.00000)(4.11832,-1.00000)%
\polyline(4.21756,-1.00000)(4.31679,-1.00000)\polyline(4.41603,-1.00000)(4.51527,-1.00000)%
\polyline(4.61450,-1.00000)(4.71374,-1.00000)\polyline(4.81298,-1.00000)(4.91221,-1.00000)%
\polyline(5.01145,-1.00000)(5.11069,-1.00000)\polyline(5.20992,-1.00000)(5.30916,-1.00000)%
\polyline(5.40840,-1.00000)(5.50763,-1.00000)\polyline(5.60687,-1.00000)(5.70611,-1.00000)%
\polyline(5.80534,-1.00000)(5.90458,-1.00000)\polyline(6.00382,-1.00000)(6.10305,-1.00000)%
\polyline(6.20229,-1.00000)(6.30153,-1.00000)\polyline(6.40076,-1.00000)(6.50000,-1.00000)%
%
%
\polyline(1.57080,0.05000)(1.57080,-0.05000)%
%
\settowidth{\Width}{$\tfrac{\pi}{2}$}\setlength{\Width}{-0.5\Width}%
\settoheight{\Height}{$\tfrac{\pi}{2}$}\settodepth{\Depth}{$\tfrac{\pi}{2}$}\setlength{\Height}{-\Height}%
\put(1.5700000,-0.1000000){\hspace*{\Width}\raisebox{\Height}{$\tfrac{\pi}{2}$}}%
%
\polyline(3.14159,0.05000)(3.14159,-0.05000)%
%
\settowidth{\Width}{$\pi$}\setlength{\Width}{-0.5\Width}%
\settoheight{\Height}{$\pi$}\settodepth{\Depth}{$\pi$}\setlength{\Height}{-\Height}%
\put(3.1400000,-0.1000000){\hspace*{\Width}\raisebox{\Height}{$\pi$}}%
%
\polyline(6.28319,0.05000)(6.28319,-0.05000)%
%
\settowidth{\Width}{$2\pi$}\setlength{\Width}{-0.5\Width}%
\settoheight{\Height}{$2\pi$}\settodepth{\Depth}{$2\pi$}\setlength{\Height}{-\Height}%
\put(6.2800000,-0.1000000){\hspace*{\Width}\raisebox{\Height}{$2\pi$}}%
%
\polyline(-1.57080,0.05000)(-1.57080,-0.05000)%
%
\settowidth{\Width}{$-\tfrac{\pi}{2}$}\setlength{\Width}{-0.5\Width}%
\settoheight{\Height}{$-\tfrac{\pi}{2}$}\settodepth{\Depth}{$-\tfrac{\pi}{2}$}\setlength{\Height}{-\Height}%
\put(-1.5700000,-0.1000000){\hspace*{\Width}\raisebox{\Height}{$-\tfrac{\pi}{2}$}}%
%
\polyline(-3.14159,0.05000)(-3.14159,-0.05000)%
%
\settowidth{\Width}{$-\pi$}\setlength{\Width}{-0.5\Width}%
\settoheight{\Height}{$-\pi$}\settodepth{\Depth}{$-\pi$}\setlength{\Height}{-\Height}%
\put(-3.1400000,-0.1000000){\hspace*{\Width}\raisebox{\Height}{$-\pi$}}%
%
\polyline(-6.28319,0.05000)(-6.28319,-0.05000)%
%
\settowidth{\Width}{$-2\pi$}\setlength{\Width}{-0.5\Width}%
\settoheight{\Height}{$-2\pi$}\settodepth{\Depth}{$-2\pi$}\setlength{\Height}{-\Height}%
\put(-6.2800000,-0.1000000){\hspace*{\Width}\raisebox{\Height}{$-2\pi$}}%
%
\polyline(0.05000,-1.00000)(-0.05000,-1.00000)%
%
\settowidth{\Width}{$-1$}\setlength{\Width}{-1\Width}%
\settoheight{\Height}{$-1$}\settodepth{\Depth}{$-1$}\setlength{\Height}{-0.5\Height}\setlength{\Depth}{0.5\Depth}\addtolength{\Height}{\Depth}%
\put(-0.1000000,-1.0000000){\hspace*{\Width}\raisebox{\Height}{$-1$}}%
%
\polyline(0.05000,1.00000)(-0.05000,1.00000)%
%
\settowidth{\Width}{$1$}\setlength{\Width}{-1\Width}%
\settoheight{\Height}{$1$}\settodepth{\Depth}{$1$}\setlength{\Height}{-0.5\Height}\setlength{\Depth}{0.5\Depth}\addtolength{\Height}{\Depth}%
\put(-0.1000000,1.0000000){\hspace*{\Width}\raisebox{\Height}{$1$}}%
%
\polyline(-6.50000,-0.21512)(-6.43500,-0.15123)(-6.37000,-0.08671)(-6.30500,-0.02181)%
(-6.24000,0.04317)(-6.17500,0.10797)(-6.11000,0.17232)(-6.04500,0.23594)(-5.98000,0.29856)%
(-5.91500,0.35992)(-5.85000,0.41976)(-5.78500,0.47783)(-5.72000,0.53388)(-5.65500,0.58768)%
(-5.59000,0.63899)(-5.52500,0.68760)(-5.46000,0.73332)(-5.39500,0.77593)(-5.33000,0.81526)%
(-5.26500,0.85116)(-5.20000,0.88345)(-5.13500,0.91202)(-5.07000,0.93674)(-5.00500,0.95749)%
(-4.94000,0.97421)(-4.87500,0.98681)(-4.81000,0.99524)(-4.74500,0.99947)(-4.68000,0.99948)%
(-4.61500,0.99526)(-4.55000,0.98684)(-4.48500,0.97426)(-4.42000,0.95756)(-4.35500,0.93681)%
(-4.29000,0.91211)(-4.22500,0.88356)(-4.16000,0.85127)(-4.09500,0.81539)(-4.03000,0.77607)%
(-3.96500,0.73347)(-3.90000,0.68777)(-3.83500,0.63916)(-3.77000,0.58786)(-3.70500,0.53407)%
(-3.64000,0.47803)(-3.57500,0.41997)(-3.51000,0.36013)(-3.44500,0.29877)(-3.38000,0.23616)%
(-3.31500,0.17254)(-3.25000,0.10820)(-3.18500,0.04339)(-3.12000,-0.02159)(-3.05500,-0.08648)%
(-2.99000,-0.15101)(-2.92500,-0.21490)(-2.86000,-0.27789)(-2.79500,-0.33970)(-2.73000,-0.40007)%
(-2.66500,-0.45875)(-2.60000,-0.51550)(-2.53500,-0.57007)(-2.47000,-0.62223)(-2.40500,-0.67177)%
(-2.34000,-0.71846)(-2.27500,-0.76213)(-2.21000,-0.80257)(-2.14500,-0.83963)(-2.08000,-0.87313)%
(-2.01500,-0.90295)(-1.95000,-0.92896)(-1.88500,-0.95104)(-1.82000,-0.96911)(-1.75500,-0.98308)%
(-1.69000,-0.99290)(-1.62500,-0.99853)(-1.56000,-0.99994)(-1.49500,-0.99713)(-1.43000,-0.99010)%
(-1.36500,-0.97890)(-1.30000,-0.96356)(-1.23500,-0.94415)(-1.17000,-0.92075)(-1.10500,-0.89346)%
(-1.04000,-0.86240)(-0.97500,-0.82770)(-0.91000,-0.78950)(-0.84500,-0.74797)(-0.78000,-0.70328)%
(-0.71500,-0.65562)(-0.65000,-0.60519)(-0.58500,-0.55220)(-0.52000,-0.49688)(-0.45500,-0.43946)%
(-0.39000,-0.38019)(-0.32500,-0.31931)(-0.26000,-0.25708)(-0.19500,-0.19377)(-0.13000,-0.12963)%
(-0.06500,-0.06495)(0.00000,0.00000)(0.06500,0.06495)(0.13000,0.12963)(0.19500,0.19377)%
(0.26000,0.25708)(0.32500,0.31931)(0.39000,0.38019)(0.45500,0.43946)(0.52000,0.49688)%
(0.58500,0.55220)(0.65000,0.60519)(0.71500,0.65562)(0.78000,0.70328)(0.84500,0.74797)%
(0.91000,0.78950)(0.97500,0.82770)(1.04000,0.86240)(1.10500,0.89346)(1.17000,0.92075)%
(1.23500,0.94415)(1.30000,0.96356)(1.36500,0.97890)(1.43000,0.99010)(1.49500,0.99713)%
(1.56000,0.99994)(1.62500,0.99853)(1.69000,0.99290)(1.75500,0.98308)(1.82000,0.96911)%
(1.88500,0.95104)(1.95000,0.92896)(2.01500,0.90295)(2.08000,0.87313)(2.14500,0.83963)%
(2.21000,0.80257)(2.27500,0.76213)(2.34000,0.71846)(2.40500,0.67177)(2.47000,0.62223)%
(2.53500,0.57007)(2.60000,0.51550)(2.66500,0.45875)(2.73000,0.40007)(2.79500,0.33970)%
(2.86000,0.27789)(2.92500,0.21490)(2.99000,0.15101)(3.05500,0.08648)(3.12000,0.02159)%
(3.18500,-0.04339)(3.25000,-0.10820)(3.31500,-0.17254)(3.38000,-0.23616)(3.44500,-0.29877)%
(3.51000,-0.36013)(3.57500,-0.41997)(3.64000,-0.47803)(3.70500,-0.53407)(3.77000,-0.58786)%
(3.83500,-0.63916)(3.90000,-0.68777)(3.96500,-0.73347)(4.03000,-0.77607)(4.09500,-0.81539)%
(4.16000,-0.85127)(4.22500,-0.88356)(4.29000,-0.91211)(4.35500,-0.93681)(4.42000,-0.95756)%
(4.48500,-0.97426)(4.55000,-0.98684)(4.61500,-0.99526)(4.68000,-0.99948)(4.74500,-0.99947)%
(4.81000,-0.99524)(4.87500,-0.98681)(4.94000,-0.97421)(5.00500,-0.95749)(5.07000,-0.93674)%
(5.13500,-0.91202)(5.20000,-0.88345)(5.26500,-0.85116)(5.33000,-0.81526)(5.39500,-0.77593)%
(5.46000,-0.73332)(5.52500,-0.68760)(5.59000,-0.63899)(5.65500,-0.58768)(5.72000,-0.53388)%
(5.78500,-0.47783)(5.85000,-0.41976)(5.91500,-0.35992)(5.98000,-0.29856)(6.04500,-0.23594)%
(6.11000,-0.17232)(6.17500,-0.10797)(6.24000,-0.04317)(6.30500,0.02181)(6.37000,0.08671)%
(6.43500,0.15123)(6.50000,0.21512)%
%
\polyline(-6.50000,0.00000)(6.50000,0.00000)%
%
\polyline(0.00000,-1.20000)(0.00000,1.20000)%
%
\settowidth{\Width}{$x$}\setlength{\Width}{0\Width}%
\settoheight{\Height}{$x$}\settodepth{\Depth}{$x$}\setlength{\Height}{-0.5\Height}\setlength{\Depth}{0.5\Depth}\addtolength{\Height}{\Depth}%
\put(6.5500000,0.0000000){\hspace*{\Width}\raisebox{\Height}{$x$}}%
%
\settowidth{\Width}{$y$}\setlength{\Width}{-0.5\Width}%
\settoheight{\Height}{$y$}\settodepth{\Depth}{$y$}\setlength{\Height}{\Depth}%
\put(0.0000000,1.2500000){\hspace*{\Width}\raisebox{\Height}{$y$}}%
%
\settowidth{\Width}{O}\setlength{\Width}{0\Width}%
\settoheight{\Height}{O}\settodepth{\Depth}{O}\setlength{\Height}{-\Height}%
\put(0.0500000,-0.0500000){\hspace*{\Width}\raisebox{\Height}{O}}%
%
\end{picture}}%}
\end{layer}

\vspace{30mm}
\begin{itemize}
\item
{\color{red}振幅}は$1$(値の範囲は$-1$から$1$)
\item
{\color{red}周期}は$2\pi$($2\pi$で元に戻る)
\item
原点対称
\end{itemize}

\newslide{正弦曲線}

\vspace*{18mm}

\slidepage
\begin{itemize}
\item
\href{https://s-takato.github.io/polytech22/offlineapp/drawgraphvjsoffL.html}{関数のグラフ}で振幅と周期を求めよう\seteda{50}\vspace{2mm}\\
\eda{$y=2\sin x$}\eda{$y=\bunsuu{1}{3}\sin x$}\\
\eda{$y=\sin 2x$}\eda{$y=4\sin\bunsuu{x}{2}$}\seteda{50}\\
\eda{振幅2, 周期$2\pi$}\eda{振幅$\bunsuu{1}{3}$, 周期$2\pi$}\\
\eda{振幅1, 周期$\pi$}\eda{振幅4, 周期$4\pi$}
\item
[課題]\monban $y=a\sin bx$の振幅と周期を求めよ\\
ただし,$a,\ b$は正の定数である.
\end{itemize}

\newslide{$y=\cos x$のグラフ}

\vspace*{18mm}

\slidepage

\begin{layer}{120}{0}
\putnotese{75}{3}{\scalebox{0.9}{%%% /polytech22.git/104-0509/presen/fig/fig221046b.tex 
%%% Generator=fig22104.cdy 
{\unitlength=24mm%
\begin{picture}%
(2.4,2.4)(-1.2,-1.2)%
\linethickness{0.008in}%%
\polyline(1.00000,0.00000)(0.99211,0.12533)(0.96858,0.24869)(0.92978,0.36812)(0.87631,0.48175)%
(0.80902,0.58779)(0.72897,0.68455)(0.63742,0.77051)(0.53583,0.84433)(0.42578,0.90483)%
(0.30902,0.95106)(0.18738,0.98229)(0.06279,0.99803)(-0.06279,0.99803)(-0.18738,0.98229)%
(-0.30902,0.95106)(-0.42578,0.90483)(-0.53583,0.84433)(-0.63742,0.77051)(-0.72897,0.68455)%
(-0.80902,0.58779)(-0.87631,0.48175)(-0.92978,0.36812)(-0.96858,0.24869)(-0.99211,0.12533)%
(-1.00000,-0.00000)(-0.99211,-0.12533)(-0.96858,-0.24869)(-0.92978,-0.36812)(-0.87631,-0.48175)%
(-0.80902,-0.58779)(-0.72897,-0.68455)(-0.63742,-0.77051)(-0.53583,-0.84433)(-0.42578,-0.90483)%
(-0.30902,-0.95106)(-0.18738,-0.98229)(-0.06279,-0.99803)(0.06279,-0.99803)(0.18738,-0.98229)%
(0.30902,-0.95106)(0.42578,-0.90483)(0.53583,-0.84433)(0.63742,-0.77051)(0.72897,-0.68455)%
(0.80902,-0.58779)(0.87631,-0.48175)(0.92978,-0.36812)(0.96858,-0.24869)(0.99211,-0.12533)%
(1.00000,-0.00000)%
%
\polyline(0.00000,0.00000)(0.58779,0.80902)%
%
\settowidth{\Width}{$\mbox{P}(X,Y)$}\setlength{\Width}{0\Width}%
\settoheight{\Height}{$\mbox{P}(X,Y)$}\settodepth{\Depth}{$\mbox{P}(X,Y)$}\setlength{\Height}{\Depth}%
\put(0.6108333,0.8308333){\hspace*{\Width}\raisebox{\Height}{$\mbox{P}(X,Y)$}}%
%
\polyline(0.25000,0.00000)(0.24803,0.03133)(0.24215,0.06217)(0.23244,0.09203)(0.21908,0.12044)%
(0.20225,0.14695)(0.18224,0.17114)(0.15936,0.19263)(0.14666,0.20186)%
%
\settowidth{\Width}{$x$}\setlength{\Width}{-0.5\Width}%
\settoheight{\Height}{$x$}\settodepth{\Depth}{$x$}\setlength{\Height}{-0.5\Height}\setlength{\Depth}{0.5\Depth}\addtolength{\Height}{\Depth}%
\put(0.3200000,0.1600000){\hspace*{\Width}\raisebox{\Height}{$x$}}%
%
\polyline(-1.00000,0.02083)(-1.00000,-0.02083)%
%
\settowidth{\Width}{$-1$}\setlength{\Width}{-1\Width}%
\settoheight{\Height}{$-1$}\settodepth{\Depth}{$-1$}\setlength{\Height}{-\Height}%
\put(-1.0208333,-0.0208333){\hspace*{\Width}\raisebox{\Height}{$-1$}}%
%
\polyline(1.00000,0.02083)(1.00000,-0.02083)%
%
\settowidth{\Width}{$1$}\setlength{\Width}{0\Width}%
\settoheight{\Height}{$1$}\settodepth{\Depth}{$1$}\setlength{\Height}{-\Height}%
\put(1.0208333,-0.0208333){\hspace*{\Width}\raisebox{\Height}{$1$}}%
%
\polyline(0.02083,-1.00000)(-0.02083,-1.00000)%
%
\settowidth{\Width}{$-1$}\setlength{\Width}{-1\Width}%
\settoheight{\Height}{$-1$}\settodepth{\Depth}{$-1$}\setlength{\Height}{-\Height}%
\put(-0.0208333,-1.0208333){\hspace*{\Width}\raisebox{\Height}{$-1$}}%
%
\polyline(0.02083,1.00000)(-0.02083,1.00000)%
%
\settowidth{\Width}{$1$}\setlength{\Width}{-1\Width}%
\settoheight{\Height}{$1$}\settodepth{\Depth}{$1$}\setlength{\Height}{\Depth}%
\put(-0.0208333,1.0208333){\hspace*{\Width}\raisebox{\Height}{$1$}}%
%
{%
\color[cmyk]{0,1,1,0}%
\linethickness{0.016in}%%
\polyline(0.00000,0.00000)(0.58779,0.00000)%
%
\linethickness{0.008in}%%
}%
\settowidth{\Width}{(1)}\setlength{\Width}{-0.5\Width}%
\settoheight{\Height}{(1)}\settodepth{\Depth}{(1)}\setlength{\Height}{-\Height}%
\put(0.2900000,-0.0208333){\hspace*{\Width}\raisebox{\Height}{(1)}}%
%
{%
\color[cmyk]{1,0,0,0}%
\linethickness{0.016in}%%
\polyline(0.58779,0.80902)(0.58779,0.00000)%
%
\linethickness{0.008in}%%
}%
\settowidth{\Width}{(2)}\setlength{\Width}{-1\Width}%
\settoheight{\Height}{(2)}\settodepth{\Depth}{(2)}\setlength{\Height}{-0.5\Height}\setlength{\Depth}{0.5\Depth}\addtolength{\Height}{\Depth}%
\put(0.5900000,0.4000000){\hspace*{\Width}\raisebox{\Height}{(2)}}%
%
{%
\color[cmyk]{0,1,1,0}%
\linethickness{0.016in}%%
\polyline(0.58779,0.80902)(1.00000,0.00000)%
%
\linethickness{0.008in}%%
}%
\settowidth{\Width}{(3)}\setlength{\Width}{0\Width}%
\settoheight{\Height}{(3)}\settodepth{\Depth}{(3)}\setlength{\Height}{-\Height}%
\put(0.6525000,0.3791667){\hspace*{\Width}\raisebox{\Height}{(3)}}%
%
{%
\color[cmyk]{1,0,0,0}%
\linethickness{0.016in}%%
\polyline(1.00000,0.00000)(0.99982,0.01885)(0.99929,0.03769)(0.99840,0.05652)(0.99716,0.07533)%
(0.99556,0.09411)(0.99361,0.11286)(0.99131,0.13156)(0.98865,0.15023)(0.98564,0.16883)%
(0.98229,0.18738)(0.97858,0.20586)(0.97453,0.22427)(0.97013,0.24260)(0.96538,0.26084)%
(0.96029,0.27899)(0.95486,0.29704)(0.94910,0.31499)(0.94299,0.33282)(0.93655,0.35053)%
(0.92978,0.36812)(0.92267,0.38558)(0.91524,0.40291)(0.90748,0.42009)(0.89941,0.43712)%
(0.89101,0.45399)(0.88229,0.47070)(0.87326,0.48725)(0.86392,0.50362)(0.85428,0.51982)%
(0.84433,0.53583)(0.83408,0.55165)(0.82353,0.56727)(0.81269,0.58269)(0.80157,0.59790)%
(0.79016,0.61291)(0.77846,0.62769)(0.76649,0.64225)(0.75425,0.65659)(0.74174,0.67069)%
(0.72897,0.68455)(0.71594,0.69817)(0.70265,0.71154)(0.68911,0.72465)(0.67533,0.73751)%
(0.66131,0.75011)(0.64706,0.76244)(0.63257,0.77450)(0.61786,0.78629)(0.60293,0.79779)%
(0.58779,0.80902)%
%
\linethickness{0.008in}%%
}%
\settowidth{\Width}{(4)}\setlength{\Width}{0\Width}%
\settoheight{\Height}{(4)}\settodepth{\Depth}{(4)}\setlength{\Height}{\Depth}%
\put(0.8900000,0.4500000){\hspace*{\Width}\raisebox{\Height}{(4)}}%
%
\polyline(-1.20000,0.00000)(1.20000,0.00000)%
%
\polyline(0.00000,-1.20000)(0.00000,1.20000)%
%
\settowidth{\Width}{$ $}\setlength{\Width}{0\Width}%
\settoheight{\Height}{$ $}\settodepth{\Depth}{$ $}\setlength{\Height}{-0.5\Height}\setlength{\Depth}{0.5\Depth}\addtolength{\Height}{\Depth}%
\put(1.2208333,0.0000000){\hspace*{\Width}\raisebox{\Height}{$ $}}%
%
\settowidth{\Width}{$ $}\setlength{\Width}{-0.5\Width}%
\settoheight{\Height}{$ $}\settodepth{\Depth}{$ $}\setlength{\Height}{\Depth}%
\put(0.0000000,1.2208333){\hspace*{\Width}\raisebox{\Height}{$ $}}%
%
\settowidth{\Width}{O}\setlength{\Width}{-1\Width}%
\settoheight{\Height}{O}\settodepth{\Depth}{O}\setlength{\Height}{-\Height}%
\put(-0.0208333,-0.0208333){\hspace*{\Width}\raisebox{\Height}{O}}%
%
\end{picture}}%}}
\end{layer}

{\color{red}

\begin{layer}{120}{0}
\qarrowline[8]{44}{36}{33}{152}{40}
\circleline{46}{35}{1}
\qarrowline[8]{35}{55}{38}{122}{40}
\circleline{37}{55}{1}
\end{layer}

}
\begin{itemize}
\item
角$x$はラジアンで測る
\item
{\color{red}半径$1$}の円に点$\mathrm{P}(X,Y)$をとる
\item
[]\hspace*{3zw}$\cos x=\bunsuu{X}{r}=X$
\item
また弧の長さを$\ell$とすると\\
\hspace*{3zw}$x=\bunsuu{\ell}{r}=\ell$
\item
$x$は弧(4)の長さ,\ $\cos x$は(1)の長さ
\end{itemize}

\newslide{$y=\cos x$のグラフ(余弦曲線)}

\vspace*{18mm}

\slidepage

\begin{layer}{120}{0}
\putnotes{62}{6}{%%% /polytech.git/n103/fig/graphcos.tex 
%%% Generator=graphsincos.cdy 
{\unitlength=1cm%
\begin{picture}%
(13,2.4)(-6.5,-1.2)%
\special{pn 8}%
%
\small%
{%
\color[rgb]{0,0,0}%
\special{pa -2559  -384}\special{pa -2533  -389}\special{pa -2508  -392}\special{pa -2482  -394}%
\special{pa -2457  -393}\special{pa -2431  -391}\special{pa -2406  -388}\special{pa -2380  -383}%
\special{pa -2354  -376}\special{pa -2329  -367}\special{pa -2303  -357}\special{pa -2278  -346}%
\special{pa -2252  -333}\special{pa -2226  -319}\special{pa -2201  -303}\special{pa -2175  -286}%
\special{pa -2150  -268}\special{pa -2124  -248}\special{pa -2098  -228}\special{pa -2073  -207}%
\special{pa -2047  -184}\special{pa -2022  -161}\special{pa -1996  -138}\special{pa -1970  -114}%
\special{pa -1945   -89}\special{pa -1919   -64}\special{pa -1894   -38}\special{pa -1868   -13}%
\special{pa -1843    13}\special{pa -1817    38}\special{pa -1791    64}\special{pa -1766    89}%
\special{pa -1740   113}\special{pa -1715   138}\special{pa -1689   161}\special{pa -1663   184}%
\special{pa -1638   207}\special{pa -1612   228}\special{pa -1587   248}\special{pa -1561   268}%
\special{pa -1535   286}\special{pa -1510   303}\special{pa -1484   318}\special{pa -1459   333}%
\special{pa -1433   346}\special{pa -1407   357}\special{pa -1382   367}\special{pa -1356   376}%
\special{pa -1331   383}\special{pa -1305   388}\special{pa -1280   391}\special{pa -1254   393}%
\special{pa -1228   394}\special{pa -1203   392}\special{pa -1177   389}\special{pa -1152   385}%
\special{pa -1126   378}\special{pa -1100   370}\special{pa -1075   361}\special{pa -1049   350}%
\special{pa -1024   337}\special{pa  -998   323}\special{pa  -972   308}\special{pa  -947   292}%
\special{pa  -921   274}\special{pa  -896   255}\special{pa  -870   235}\special{pa  -844   214}%
\special{pa  -819   192}\special{pa  -793   169}\special{pa  -768   146}\special{pa  -742   122}%
\special{pa  -717    97}\special{pa  -691    72}\special{pa  -665    47}\special{pa  -640    21}%
\special{pa  -614    -4}\special{pa  -589   -30}\special{pa  -563   -55}\special{pa  -537   -80}%
\special{pa  -512  -105}\special{pa  -486  -130}\special{pa  -461  -154}\special{pa  -435  -177}%
\special{pa  -409  -199}\special{pa  -384  -221}\special{pa  -358  -242}\special{pa  -333  -261}%
\special{pa  -307  -280}\special{pa  -281  -297}\special{pa  -256  -313}\special{pa  -230  -328}%
\special{pa  -205  -342}\special{pa  -179  -354}\special{pa  -154  -364}\special{pa  -128  -373}%
\special{pa  -102  -380}\special{pa   -77  -386}\special{pa   -51  -390}\special{pa   -26  -393}%
\special{pa     0  -394}\special{pa    26  -393}\special{pa    51  -390}\special{pa    77  -386}%
\special{pa   102  -380}\special{pa   128  -373}\special{pa   154  -364}\special{pa   179  -354}%
\special{pa   205  -342}\special{pa   230  -328}\special{pa   256  -313}\special{pa   281  -297}%
\special{pa   307  -280}\special{pa   333  -261}\special{pa   358  -242}\special{pa   384  -221}%
\special{pa   409  -199}\special{pa   435  -177}\special{pa   461  -154}\special{pa   486  -130}%
\special{pa   512  -105}\special{pa   537   -80}\special{pa   563   -55}\special{pa   589   -30}%
\special{pa   614    -4}\special{pa   640    21}\special{pa   665    47}\special{pa   691    72}%
\special{pa   717    97}\special{pa   742   122}\special{pa   768   146}\special{pa   793   169}%
\special{pa   819   192}\special{pa   844   214}\special{pa   870   235}\special{pa   896   255}%
\special{pa   921   274}\special{pa   947   292}\special{pa   972   308}\special{pa   998   323}%
\special{pa  1024   337}\special{pa  1049   350}\special{pa  1075   361}\special{pa  1100   370}%
\special{pa  1126   378}\special{pa  1152   385}\special{pa  1177   389}\special{pa  1203   392}%
\special{pa  1228   394}\special{pa  1254   393}\special{pa  1280   391}\special{pa  1305   388}%
\special{pa  1331   383}\special{pa  1356   376}\special{pa  1382   367}\special{pa  1407   357}%
\special{pa  1433   346}\special{pa  1459   333}\special{pa  1484   318}\special{pa  1510   303}%
\special{pa  1535   286}\special{pa  1561   268}\special{pa  1587   248}\special{pa  1612   228}%
\special{pa  1638   207}\special{pa  1663   184}\special{pa  1689   161}\special{pa  1715   138}%
\special{pa  1740   113}\special{pa  1766    89}\special{pa  1791    64}\special{pa  1817    38}%
\special{pa  1843    13}\special{pa  1868   -13}\special{pa  1894   -38}\special{pa  1919   -64}%
\special{pa  1945   -89}\special{pa  1970  -114}\special{pa  1996  -138}\special{pa  2022  -161}%
\special{pa  2047  -184}\special{pa  2073  -207}\special{pa  2098  -228}\special{pa  2124  -248}%
\special{pa  2150  -268}\special{pa  2175  -286}\special{pa  2201  -303}\special{pa  2226  -319}%
\special{pa  2252  -333}\special{pa  2278  -346}\special{pa  2303  -357}\special{pa  2329  -367}%
\special{pa  2354  -376}\special{pa  2380  -383}\special{pa  2406  -388}\special{pa  2431  -391}%
\special{pa  2457  -393}\special{pa  2482  -394}\special{pa  2508  -392}\special{pa  2533  -389}%
\special{pa  2559  -384}%
\special{fp}%
}%
{%
\color[rgb]{0,0,0}%
\special{pa -2559 -394}\special{pa -2520 -394}\special{fp}\special{pa -2481 -394}\special{pa -2442 -394}\special{fp}%
\special{pa -2403 -394}\special{pa -2364 -394}\special{fp}\special{pa -2325 -394}\special{pa -2286 -394}\special{fp}%
\special{pa -2246 -394}\special{pa -2207 -394}\special{fp}\special{pa -2168 -394}\special{pa -2129 -394}\special{fp}%
\special{pa -2090 -394}\special{pa -2051 -394}\special{fp}\special{pa -2012 -394}\special{pa -1973 -394}\special{fp}%
\special{pa -1934 -394}\special{pa -1895 -394}\special{fp}\special{pa -1856 -394}\special{pa -1817 -394}\special{fp}%
\special{pa -1778 -394}\special{pa -1739 -394}\special{fp}\special{pa -1700 -394}\special{pa -1660 -394}\special{fp}%
\special{pa -1621 -394}\special{pa -1582 -394}\special{fp}\special{pa -1543 -394}\special{pa -1504 -394}\special{fp}%
\special{pa -1465 -394}\special{pa -1426 -394}\special{fp}\special{pa -1387 -394}\special{pa -1348 -394}\special{fp}%
\special{pa -1309 -394}\special{pa -1270 -394}\special{fp}\special{pa -1231 -394}\special{pa -1192 -394}\special{fp}%
\special{pa -1153 -394}\special{pa -1113 -394}\special{fp}\special{pa -1074 -394}\special{pa -1035 -394}\special{fp}%
\special{pa -996 -394}\special{pa -957 -394}\special{fp}\special{pa -918 -394}\special{pa -879 -394}\special{fp}%
\special{pa -840 -394}\special{pa -801 -394}\special{fp}\special{pa -762 -394}\special{pa -723 -394}\special{fp}%
\special{pa -684 -394}\special{pa -645 -394}\special{fp}\special{pa -606 -394}\special{pa -567 -394}\special{fp}%
\special{pa -527 -394}\special{pa -488 -394}\special{fp}\special{pa -449 -394}\special{pa -410 -394}\special{fp}%
\special{pa -371 -394}\special{pa -332 -394}\special{fp}\special{pa -293 -394}\special{pa -254 -394}\special{fp}%
\special{pa -215 -394}\special{pa -176 -394}\special{fp}\special{pa -137 -394}\special{pa -98 -394}\special{fp}%
\special{pa -59 -394}\special{pa -20 -394}\special{fp}\special{pa 20 -394}\special{pa 59 -394}\special{fp}%
\special{pa 98 -394}\special{pa 137 -394}\special{fp}\special{pa 176 -394}\special{pa 215 -394}\special{fp}%
\special{pa 254 -394}\special{pa 293 -394}\special{fp}\special{pa 332 -394}\special{pa 371 -394}\special{fp}%
\special{pa 410 -394}\special{pa 449 -394}\special{fp}\special{pa 488 -394}\special{pa 527 -394}\special{fp}%
\special{pa 567 -394}\special{pa 606 -394}\special{fp}\special{pa 645 -394}\special{pa 684 -394}\special{fp}%
\special{pa 723 -394}\special{pa 762 -394}\special{fp}\special{pa 801 -394}\special{pa 840 -394}\special{fp}%
\special{pa 879 -394}\special{pa 918 -394}\special{fp}\special{pa 957 -394}\special{pa 996 -394}\special{fp}%
\special{pa 1035 -394}\special{pa 1074 -394}\special{fp}\special{pa 1113 -394}\special{pa 1153 -394}\special{fp}%
\special{pa 1192 -394}\special{pa 1231 -394}\special{fp}\special{pa 1270 -394}\special{pa 1309 -394}\special{fp}%
\special{pa 1348 -394}\special{pa 1387 -394}\special{fp}\special{pa 1426 -394}\special{pa 1465 -394}\special{fp}%
\special{pa 1504 -394}\special{pa 1543 -394}\special{fp}\special{pa 1582 -394}\special{pa 1621 -394}\special{fp}%
\special{pa 1660 -394}\special{pa 1700 -394}\special{fp}\special{pa 1739 -394}\special{pa 1778 -394}\special{fp}%
\special{pa 1817 -394}\special{pa 1856 -394}\special{fp}\special{pa 1895 -394}\special{pa 1934 -394}\special{fp}%
\special{pa 1973 -394}\special{pa 2012 -394}\special{fp}\special{pa 2051 -394}\special{pa 2090 -394}\special{fp}%
\special{pa 2129 -394}\special{pa 2168 -394}\special{fp}\special{pa 2207 -394}\special{pa 2246 -394}\special{fp}%
\special{pa 2286 -394}\special{pa 2325 -394}\special{fp}\special{pa 2364 -394}\special{pa 2403 -394}\special{fp}%
\special{pa 2442 -394}\special{pa 2481 -394}\special{fp}\special{pa 2520 -394}\special{pa 2559 -394}\special{fp}%
%
%
}%
{%
\color[rgb]{0,0,0}%
\special{pa -2559 394}\special{pa -2520 394}\special{fp}\special{pa -2481 394}\special{pa -2442 394}\special{fp}%
\special{pa -2403 394}\special{pa -2364 394}\special{fp}\special{pa -2325 394}\special{pa -2286 394}\special{fp}%
\special{pa -2246 394}\special{pa -2207 394}\special{fp}\special{pa -2168 394}\special{pa -2129 394}\special{fp}%
\special{pa -2090 394}\special{pa -2051 394}\special{fp}\special{pa -2012 394}\special{pa -1973 394}\special{fp}%
\special{pa -1934 394}\special{pa -1895 394}\special{fp}\special{pa -1856 394}\special{pa -1817 394}\special{fp}%
\special{pa -1778 394}\special{pa -1739 394}\special{fp}\special{pa -1700 394}\special{pa -1660 394}\special{fp}%
\special{pa -1621 394}\special{pa -1582 394}\special{fp}\special{pa -1543 394}\special{pa -1504 394}\special{fp}%
\special{pa -1465 394}\special{pa -1426 394}\special{fp}\special{pa -1387 394}\special{pa -1348 394}\special{fp}%
\special{pa -1309 394}\special{pa -1270 394}\special{fp}\special{pa -1231 394}\special{pa -1192 394}\special{fp}%
\special{pa -1153 394}\special{pa -1113 394}\special{fp}\special{pa -1074 394}\special{pa -1035 394}\special{fp}%
\special{pa -996 394}\special{pa -957 394}\special{fp}\special{pa -918 394}\special{pa -879 394}\special{fp}%
\special{pa -840 394}\special{pa -801 394}\special{fp}\special{pa -762 394}\special{pa -723 394}\special{fp}%
\special{pa -684 394}\special{pa -645 394}\special{fp}\special{pa -606 394}\special{pa -567 394}\special{fp}%
\special{pa -527 394}\special{pa -488 394}\special{fp}\special{pa -449 394}\special{pa -410 394}\special{fp}%
\special{pa -371 394}\special{pa -332 394}\special{fp}\special{pa -293 394}\special{pa -254 394}\special{fp}%
\special{pa -215 394}\special{pa -176 394}\special{fp}\special{pa -137 394}\special{pa -98 394}\special{fp}%
\special{pa -59 394}\special{pa -20 394}\special{fp}\special{pa 20 394}\special{pa 59 394}\special{fp}%
\special{pa 98 394}\special{pa 137 394}\special{fp}\special{pa 176 394}\special{pa 215 394}\special{fp}%
\special{pa 254 394}\special{pa 293 394}\special{fp}\special{pa 332 394}\special{pa 371 394}\special{fp}%
\special{pa 410 394}\special{pa 449 394}\special{fp}\special{pa 488 394}\special{pa 527 394}\special{fp}%
\special{pa 567 394}\special{pa 606 394}\special{fp}\special{pa 645 394}\special{pa 684 394}\special{fp}%
\special{pa 723 394}\special{pa 762 394}\special{fp}\special{pa 801 394}\special{pa 840 394}\special{fp}%
\special{pa 879 394}\special{pa 918 394}\special{fp}\special{pa 957 394}\special{pa 996 394}\special{fp}%
\special{pa 1035 394}\special{pa 1074 394}\special{fp}\special{pa 1113 394}\special{pa 1153 394}\special{fp}%
\special{pa 1192 394}\special{pa 1231 394}\special{fp}\special{pa 1270 394}\special{pa 1309 394}\special{fp}%
\special{pa 1348 394}\special{pa 1387 394}\special{fp}\special{pa 1426 394}\special{pa 1465 394}\special{fp}%
\special{pa 1504 394}\special{pa 1543 394}\special{fp}\special{pa 1582 394}\special{pa 1621 394}\special{fp}%
\special{pa 1660 394}\special{pa 1700 394}\special{fp}\special{pa 1739 394}\special{pa 1778 394}\special{fp}%
\special{pa 1817 394}\special{pa 1856 394}\special{fp}\special{pa 1895 394}\special{pa 1934 394}\special{fp}%
\special{pa 1973 394}\special{pa 2012 394}\special{fp}\special{pa 2051 394}\special{pa 2090 394}\special{fp}%
\special{pa 2129 394}\special{pa 2168 394}\special{fp}\special{pa 2207 394}\special{pa 2246 394}\special{fp}%
\special{pa 2286 394}\special{pa 2325 394}\special{fp}\special{pa 2364 394}\special{pa 2403 394}\special{fp}%
\special{pa 2442 394}\special{pa 2481 394}\special{fp}\special{pa 2520 394}\special{pa 2559 394}\special{fp}%
%
%
}%
{%
\color[rgb]{0,0,0}%
\special{pa   618   -20}\special{pa   618    20}%
\special{fp}%
}%
{%
\color[rgb]{0,0,0}%
\settowidth{\Width}{$\tfrac{\pi}{2}$}\setlength{\Width}{-0.5\Width}%
\settoheight{\Height}{$\tfrac{\pi}{2}$}\settodepth{\Depth}{$\tfrac{\pi}{2}$}\setlength{\Height}{-\Height}%
\put(1.5700000,-0.1000000){\hspace*{\Width}\raisebox{\Height}{$\tfrac{\pi}{2}$}}%
%
}%
{%
\color[rgb]{0,0,0}%
\special{pa  1237   -20}\special{pa  1237    20}%
\special{fp}%
}%
{%
\color[rgb]{0,0,0}%
\settowidth{\Width}{$\pi$}\setlength{\Width}{-0.5\Width}%
\settoheight{\Height}{$\pi$}\settodepth{\Depth}{$\pi$}\setlength{\Height}{-\Height}%
\put(3.1400000,-0.1000000){\hspace*{\Width}\raisebox{\Height}{$\pi$}}%
%
}%
{%
\color[rgb]{0,0,0}%
\special{pa  2474   -20}\special{pa  2474    20}%
\special{fp}%
}%
{%
\color[rgb]{0,0,0}%
\settowidth{\Width}{$2\pi$}\setlength{\Width}{-0.5\Width}%
\settoheight{\Height}{$2\pi$}\settodepth{\Depth}{$2\pi$}\setlength{\Height}{-\Height}%
\put(6.2800000,-0.1000000){\hspace*{\Width}\raisebox{\Height}{$2\pi$}}%
%
}%
{%
\color[rgb]{0,0,0}%
\special{pa  -618   -20}\special{pa  -618    20}%
\special{fp}%
}%
{%
\color[rgb]{0,0,0}%
\settowidth{\Width}{$-\tfrac{\pi}{2}$}\setlength{\Width}{-0.5\Width}%
\settoheight{\Height}{$-\tfrac{\pi}{2}$}\settodepth{\Depth}{$-\tfrac{\pi}{2}$}\setlength{\Height}{-\Height}%
\put(-1.5700000,-0.1000000){\hspace*{\Width}\raisebox{\Height}{$-\tfrac{\pi}{2}$}}%
%
}%
{%
\color[rgb]{0,0,0}%
\special{pa -1237   -20}\special{pa -1237    20}%
\special{fp}%
}%
{%
\color[rgb]{0,0,0}%
\settowidth{\Width}{$-\pi$}\setlength{\Width}{-0.5\Width}%
\settoheight{\Height}{$-\pi$}\settodepth{\Depth}{$-\pi$}\setlength{\Height}{-\Height}%
\put(-3.1400000,-0.1000000){\hspace*{\Width}\raisebox{\Height}{$-\pi$}}%
%
}%
{%
\color[rgb]{0,0,0}%
\special{pa -2474   -20}\special{pa -2474    20}%
\special{fp}%
}%
{%
\color[rgb]{0,0,0}%
\settowidth{\Width}{$-2\pi$}\setlength{\Width}{-0.5\Width}%
\settoheight{\Height}{$-2\pi$}\settodepth{\Depth}{$-2\pi$}\setlength{\Height}{-\Height}%
\put(-6.2800000,-0.1000000){\hspace*{\Width}\raisebox{\Height}{$-2\pi$}}%
%
}%
{%
\color[rgb]{0,0,0}%
\special{pa    20   394}\special{pa   -20   394}%
\special{fp}%
}%
{%
\color[rgb]{0,0,0}%
\settowidth{\Width}{$-1$}\setlength{\Width}{-1\Width}%
\settoheight{\Height}{$-1$}\settodepth{\Depth}{$-1$}\setlength{\Height}{-0.5\Height}\setlength{\Depth}{0.5\Depth}\addtolength{\Height}{\Depth}%
\put(-0.1000000,-1.0000000){\hspace*{\Width}\raisebox{\Height}{$-1$}}%
%
}%
{%
\color[rgb]{0,0,0}%
\special{pa    20  -394}\special{pa   -20  -394}%
\special{fp}%
}%
{%
\color[rgb]{0,0,0}%
\settowidth{\Width}{$1$}\setlength{\Width}{-1\Width}%
\settoheight{\Height}{$1$}\settodepth{\Depth}{$1$}\setlength{\Height}{-0.5\Height}\setlength{\Depth}{0.5\Depth}\addtolength{\Height}{\Depth}%
\put(-0.1000000,1.0000000){\hspace*{\Width}\raisebox{\Height}{$1$}}%
%
}%
\special{pa -2559    -0}\special{pa  2559    -0}%
\special{fp}%
\special{pa     0   472}\special{pa     0  -472}%
\special{fp}%
\settowidth{\Width}{$x$}\setlength{\Width}{0\Width}%
\settoheight{\Height}{$x$}\settodepth{\Depth}{$x$}\setlength{\Height}{-0.5\Height}\setlength{\Depth}{0.5\Depth}\addtolength{\Height}{\Depth}%
\put(6.5500000,0.0000000){\hspace*{\Width}\raisebox{\Height}{$x$}}%
%
\settowidth{\Width}{$y$}\setlength{\Width}{-0.5\Width}%
\settoheight{\Height}{$y$}\settodepth{\Depth}{$y$}\setlength{\Height}{\Depth}%
\put(0.0000000,1.2500000){\hspace*{\Width}\raisebox{\Height}{$y$}}%
%
\settowidth{\Width}{O}\setlength{\Width}{0\Width}%
\settoheight{\Height}{O}\settodepth{\Depth}{O}\setlength{\Height}{-\Height}%
\put(0.0500000,-0.0500000){\hspace*{\Width}\raisebox{\Height}{O}}%
%
\end{picture}}%}
\putnoten{118}{6}{\small $y=\cos x$}
\putnotes{62}{6}{%%% /polytech.git/n103/fig/graphsinadd.tex 
%%% Generator=graphsincos.cdy 
{\unitlength=1cm%
\begin{picture}%
(13,2.4)(-6.5,-1.2)%
\special{pn 8}%
%
\small%
\color[cmyk]{0,1,1,0}%
{%
\color[rgb]{1,0,0}%
\special{pa -2559    85}\special{pa -2533    60}\special{pa -2508    34}\special{pa -2482     9}%
\special{pa -2457   -17}\special{pa -2431   -43}\special{pa -2406   -68}\special{pa -2380   -93}%
\special{pa -2354  -118}\special{pa -2329  -142}\special{pa -2303  -165}\special{pa -2278  -188}%
\special{pa -2252  -210}\special{pa -2226  -231}\special{pa -2201  -252}\special{pa -2175  -271}%
\special{pa -2150  -289}\special{pa -2124  -305}\special{pa -2098  -321}\special{pa -2073  -335}%
\special{pa -2047  -348}\special{pa -2022  -359}\special{pa -1996  -369}\special{pa -1970  -377}%
\special{pa -1945  -384}\special{pa -1919  -389}\special{pa -1894  -392}\special{pa -1868  -393}%
\special{pa -1843  -393}\special{pa -1817  -392}\special{pa -1791  -389}\special{pa -1766  -384}%
\special{pa -1740  -377}\special{pa -1715  -369}\special{pa -1689  -359}\special{pa -1663  -348}%
\special{pa -1638  -335}\special{pa -1612  -321}\special{pa -1587  -306}\special{pa -1561  -289}%
\special{pa -1535  -271}\special{pa -1510  -252}\special{pa -1484  -231}\special{pa -1459  -210}%
\special{pa -1433  -188}\special{pa -1407  -165}\special{pa -1382  -142}\special{pa -1356  -118}%
\special{pa -1331   -93}\special{pa -1305   -68}\special{pa -1280   -43}\special{pa -1254   -17}%
\special{pa -1228     9}\special{pa -1203    34}\special{pa -1177    59}\special{pa -1152    85}%
\special{pa -1126   109}\special{pa -1100   134}\special{pa -1075   158}\special{pa -1049   181}%
\special{pa -1024   203}\special{pa  -998   224}\special{pa  -972   245}\special{pa  -947   264}%
\special{pa  -921   283}\special{pa  -896   300}\special{pa  -870   316}\special{pa  -844   331}%
\special{pa  -819   344}\special{pa  -793   355}\special{pa  -768   366}\special{pa  -742   374}%
\special{pa  -717   382}\special{pa  -691   387}\special{pa  -665   391}\special{pa  -640   393}%
\special{pa  -614   394}\special{pa  -589   393}\special{pa  -563   390}\special{pa  -537   385}%
\special{pa  -512   379}\special{pa  -486   372}\special{pa  -461   363}\special{pa  -435   352}%
\special{pa  -409   340}\special{pa  -384   326}\special{pa  -358   311}\special{pa  -333   294}%
\special{pa  -307   277}\special{pa  -281   258}\special{pa  -256   238}\special{pa  -230   217}%
\special{pa  -205   196}\special{pa  -179   173}\special{pa  -154   150}\special{pa  -128   126}%
\special{pa  -102   101}\special{pa   -77    76}\special{pa   -51    51}\special{pa   -26    26}%
\special{pa     0    -0}\special{pa    26   -26}\special{pa    51   -51}\special{pa    77   -76}%
\special{pa   102  -101}\special{pa   128  -126}\special{pa   154  -150}\special{pa   179  -173}%
\special{pa   205  -196}\special{pa   230  -217}\special{pa   256  -238}\special{pa   281  -258}%
\special{pa   307  -277}\special{pa   333  -294}\special{pa   358  -311}\special{pa   384  -326}%
\special{pa   409  -340}\special{pa   435  -352}\special{pa   461  -363}\special{pa   486  -372}%
\special{pa   512  -379}\special{pa   537  -385}\special{pa   563  -390}\special{pa   589  -393}%
\special{pa   614  -394}\special{pa   640  -393}\special{pa   665  -391}\special{pa   691  -387}%
\special{pa   717  -382}\special{pa   742  -374}\special{pa   768  -366}\special{pa   793  -355}%
\special{pa   819  -344}\special{pa   844  -331}\special{pa   870  -316}\special{pa   896  -300}%
\special{pa   921  -283}\special{pa   947  -264}\special{pa   972  -245}\special{pa   998  -224}%
\special{pa  1024  -203}\special{pa  1049  -181}\special{pa  1075  -158}\special{pa  1100  -134}%
\special{pa  1126  -109}\special{pa  1152   -85}\special{pa  1177   -59}\special{pa  1203   -34}%
\special{pa  1228    -9}\special{pa  1254    17}\special{pa  1280    43}\special{pa  1305    68}%
\special{pa  1331    93}\special{pa  1356   118}\special{pa  1382   142}\special{pa  1407   165}%
\special{pa  1433   188}\special{pa  1459   210}\special{pa  1484   231}\special{pa  1510   252}%
\special{pa  1535   271}\special{pa  1561   289}\special{pa  1587   306}\special{pa  1612   321}%
\special{pa  1638   335}\special{pa  1663   348}\special{pa  1689   359}\special{pa  1715   369}%
\special{pa  1740   377}\special{pa  1766   384}\special{pa  1791   389}\special{pa  1817   392}%
\special{pa  1843   393}\special{pa  1868   393}\special{pa  1894   392}\special{pa  1919   389}%
\special{pa  1945   384}\special{pa  1970   377}\special{pa  1996   369}\special{pa  2022   359}%
\special{pa  2047   348}\special{pa  2073   335}\special{pa  2098   321}\special{pa  2124   305}%
\special{pa  2150   289}\special{pa  2175   271}\special{pa  2201   252}\special{pa  2226   231}%
\special{pa  2252   210}\special{pa  2278   188}\special{pa  2303   165}\special{pa  2329   142}%
\special{pa  2354   118}\special{pa  2380    93}\special{pa  2406    68}\special{pa  2431    43}%
\special{pa  2457    17}\special{pa  2482    -9}\special{pa  2508   -34}\special{pa  2533   -60}%
\special{pa  2559   -85}%
\special{fp}%
}%
\end{picture}}%}
\putnoten{80}{6}{\color{red}\small $y=\sin x$}
\end{layer}

\vspace*{28mm}
\begin{itemize}
\item
{\color{red}振幅}は$1$(値の範囲は$-1$から$1$)\vspace{-1mm}
\item
{\color{red}周期}は$2\pi$($2\pi$で元に戻る)\vspace{-1mm}
\item
$\cos x$は$y$軸対称\vspace{-1mm}
\item
$\cos x$は$\sin x$を左に$\frac{\pi}{2}$平行移動({\color{red}位相}が$\frac{\pi}{2}$進む)
\end{itemize}

\newslide{位相のずれ}

\vspace*{18mm}

\slidepage
\begin{itemize}
\item
[](1) $y=\sin x$ (2) $y=\cos x$をもとにする
\item
[課題]\monban 次の関数のグラフは(1)からどのくらい位相がずれているか
(左にずれるときプラスとせよ)\seteda{50}\\
\eda{$y=\sin(x-1)$}\eda{$y=\sin(x+2)$}\\
\eda{$y=\sin(x-\pi)$}\eda{$y=\sin(x+\bunsuu{\pi}{2})$}
\item
[課題]\monban 以下では,$a$を定数とする\seteda{110}\\
\eda{$y=\sin(x-a)$の(1)からの位相のずれを求めよ}\\
\eda{$y=\cos(x+a)$の(2)からの位相のずれを求めよ}
\end{itemize}
%%%%%%%%%%%%

%%%%%%%%%%%%%%%%%%%%


\newslide{角度の和の三角関数}

\vspace*{18mm}

\slidepage
\begin{itemize}
\item
2つの角を$A,\ B$とする(通常はギリシャ文字 $\alpha,\  \beta$)
\item
$\sin(A+B)=\sin A+\sin B$が成り立つかを考えよう
\item
$sin(\deg{30}+\deg{60})=\sin\deg{90}$になるかを調べる
\item
{\large $\sin \deg{90}=1,\ \sin \deg{30}=\bunsuu{1}{2},\ \sin \deg{60}=\bunsuu{\sqrt{3}}{2}$}
\item
[課題]\monban $\sqrt{3}=1.732$を用いて答えよ.\seteda{90}\\
\eda{$\sin\deg{30}+\sin\deg{60}$を計算せよ}\\
\eda{$\sin(A+B)=\sin A+\sin B$は成り立つと言えるか}
\end{itemize}
%%%%%%%%%%%%

%%%%%%%%%%%%%%%%%%%%


\newslide{加法定理}

\vspace*{18mm}

\slidepage
\begin{itemize}
\item
[]$\sin(A+B)=\sin A \cos B+\cos A\sin B$
\item
[]$\sin( A- B)=\sin A\cos B-\cos A\sin B$
\item
[]$\cos( A+ B)=\cos A\cos B-\sin A\sin B$
\item
[]$\cos( A- B)=\cos A\cos B+\sin A\sin B$
\end{itemize}

\newslide{具体例(テキストP181)}

\vspace*{18mm}


\begin{layer}{120}{0}
\putnotew{96}{73}{\hyperlink{para0pg2}{\fbox{\Ctab{2.5mm}{\scalebox{1}{\scriptsize $\mathstrut||\!\lhd$}}}}}
\putnotew{101}{73}{\hyperlink{para1pg1}{\fbox{\Ctab{2.5mm}{\scalebox{1}{\scriptsize $\mathstrut|\!\lhd$}}}}}
\putnotew{108}{73}{\hyperlink{para1pg10}{\fbox{\Ctab{4.5mm}{\scalebox{1}{\scriptsize $\mathstrut\!\!\lhd\!\!$}}}}}
\putnotew{115}{73}{\hyperlink{para1pg11}{\fbox{\Ctab{4.5mm}{\scalebox{1}{\scriptsize $\mathstrut\!\rhd\!$}}}}}
\putnotew{120}{73}{\hyperlink{para1pg11}{\fbox{\Ctab{2.5mm}{\scalebox{1}{\scriptsize $\mathstrut \!\rhd\!\!|$}}}}}
\putnotew{125}{73}{\hyperlink{para2pg1}{\fbox{\Ctab{2.5mm}{\scalebox{1}{\scriptsize $\mathstrut \!\rhd\!\!||$}}}}}
\putnotee{126}{73}{\scriptsize\color{blue} 11/11}
\end{layer}

\slidepage
\begin{itemize}
\item
{\color{blue}\normalsize $\sin 30\degree=\hakoa{$\bunsuu{1}{2}$},\ \sin 45\degree=\hakoa{$\bunsuu{1}{\sqrt{2}}$},\ \sin 60\degree=\hakoa{$\bunsuu{\sqrt{3}}{2}$}$}\\
{\color{blue}\normalsize $\cos 30\degree=\hakoa{$\bunsuu{\sqrt{3}}{2}$},\ \cos 45\degree=\hakoa{$\bunsuu{1}{\sqrt{2}}$},\ \cos 60\degree=\hakoa{$\bunsuu{1}{2}$}$}
\item
$\sin 75\degree$\\
$=\sin(45\degree+30\degree)$
$=\sin 45\degree \cos 30\degree+\cos 45\degree \sin 30\degree$
$=\bunsuu{1}{\sqrt{2}}\bunsuu{\sqrt{3}}{2}+\bunsuu{1}{\sqrt{2}}\bunsuu{1}{2}=$
$\bunsuu{\sqrt{3}+1}{2\sqrt{2}}=$
$\bunsuu{\sqrt{6}+\sqrt{2}}{4}$
\item
[課題]\monban 次を求めよ\seteda{50}\\
\eda{$\sin 15\degree$}\eda{$\cos 75\degree$}
\end{itemize}

\newslide{加法定理による等式証明($-x$)}

\vspace*{18mm}

\slidepage
\begin{itemize}
\item
{\normalsize\color{blue} $\sin 0=0,\ \cos 0=1,\ \sin\pi=0,\ \cos\pi=-1$}
\item
$\sin(-x)$
\\
$=\sin(0-x)=\sin 0 \cos x-\cos 0\sin x=-\sin x$
\item
$\cos(-x)$
\\
$=\cos(0-x)=\cos 0 \cos x+\sin 0\sin x=\cos x$
\end{itemize}

\newslide{グラフでの意味}

\vspace*{18mm}

%%repeat=2
\slidepage

\begin{layer}{120}{0}
\putnotes{60}{5}{\scalebox{0.9}{%%% /polytech22.git/104-0509/presen/fig/graphsin.tex 
%%% Generator=graphsincos.cdy 
{\unitlength=1cm%
\begin{picture}%
(13,2.4)(-6.5,-1.2)%
\linethickness{0.008in}%%
\polyline(-6.50000,1.00000)(-6.40076,1.00000)\polyline(-6.30153,1.00000)(-6.20229,1.00000)%
\polyline(-6.10305,1.00000)(-6.00382,1.00000)\polyline(-5.90458,1.00000)(-5.80534,1.00000)%
\polyline(-5.70611,1.00000)(-5.60687,1.00000)\polyline(-5.50763,1.00000)(-5.40840,1.00000)%
\polyline(-5.30916,1.00000)(-5.20992,1.00000)\polyline(-5.11069,1.00000)(-5.01145,1.00000)%
\polyline(-4.91221,1.00000)(-4.81298,1.00000)\polyline(-4.71374,1.00000)(-4.61450,1.00000)%
\polyline(-4.51527,1.00000)(-4.41603,1.00000)\polyline(-4.31679,1.00000)(-4.21756,1.00000)%
\polyline(-4.11832,1.00000)(-4.01908,1.00000)\polyline(-3.91985,1.00000)(-3.82061,1.00000)%
\polyline(-3.72137,1.00000)(-3.62214,1.00000)\polyline(-3.52290,1.00000)(-3.42366,1.00000)%
\polyline(-3.32443,1.00000)(-3.22519,1.00000)\polyline(-3.12595,1.00000)(-3.02672,1.00000)%
\polyline(-2.92748,1.00000)(-2.82824,1.00000)\polyline(-2.72901,1.00000)(-2.62977,1.00000)%
\polyline(-2.53053,1.00000)(-2.43130,1.00000)\polyline(-2.33206,1.00000)(-2.23282,1.00000)%
\polyline(-2.13359,1.00000)(-2.03435,1.00000)\polyline(-1.93511,1.00000)(-1.83588,1.00000)%
\polyline(-1.73664,1.00000)(-1.63740,1.00000)\polyline(-1.53817,1.00000)(-1.43893,1.00000)%
\polyline(-1.33969,1.00000)(-1.24046,1.00000)\polyline(-1.14122,1.00000)(-1.04198,1.00000)%
\polyline(-0.94275,1.00000)(-0.84351,1.00000)\polyline(-0.74427,1.00000)(-0.64504,1.00000)%
\polyline(-0.54580,1.00000)(-0.44656,1.00000)\polyline(-0.34733,1.00000)(-0.24809,1.00000)%
\polyline(-0.14885,1.00000)(-0.04962,1.00000)\polyline(0.04962,1.00000)(0.14885,1.00000)%
\polyline(0.24809,1.00000)(0.34733,1.00000)\polyline(0.44656,1.00000)(0.54580,1.00000)%
\polyline(0.64504,1.00000)(0.74427,1.00000)\polyline(0.84351,1.00000)(0.94275,1.00000)%
\polyline(1.04198,1.00000)(1.14122,1.00000)\polyline(1.24046,1.00000)(1.33969,1.00000)%
\polyline(1.43893,1.00000)(1.53817,1.00000)\polyline(1.63740,1.00000)(1.73664,1.00000)%
\polyline(1.83588,1.00000)(1.93511,1.00000)\polyline(2.03435,1.00000)(2.13359,1.00000)%
\polyline(2.23282,1.00000)(2.33206,1.00000)\polyline(2.43130,1.00000)(2.53053,1.00000)%
\polyline(2.62977,1.00000)(2.72901,1.00000)\polyline(2.82824,1.00000)(2.92748,1.00000)%
\polyline(3.02672,1.00000)(3.12595,1.00000)\polyline(3.22519,1.00000)(3.32443,1.00000)%
\polyline(3.42366,1.00000)(3.52290,1.00000)\polyline(3.62214,1.00000)(3.72137,1.00000)%
\polyline(3.82061,1.00000)(3.91985,1.00000)\polyline(4.01908,1.00000)(4.11832,1.00000)%
\polyline(4.21756,1.00000)(4.31679,1.00000)\polyline(4.41603,1.00000)(4.51527,1.00000)%
\polyline(4.61450,1.00000)(4.71374,1.00000)\polyline(4.81298,1.00000)(4.91221,1.00000)%
\polyline(5.01145,1.00000)(5.11069,1.00000)\polyline(5.20992,1.00000)(5.30916,1.00000)%
\polyline(5.40840,1.00000)(5.50763,1.00000)\polyline(5.60687,1.00000)(5.70611,1.00000)%
\polyline(5.80534,1.00000)(5.90458,1.00000)\polyline(6.00382,1.00000)(6.10305,1.00000)%
\polyline(6.20229,1.00000)(6.30153,1.00000)\polyline(6.40076,1.00000)(6.50000,1.00000)%
%
%
\polyline(-6.50000,-1.00000)(-6.40076,-1.00000)\polyline(-6.30153,-1.00000)(-6.20229,-1.00000)%
\polyline(-6.10305,-1.00000)(-6.00382,-1.00000)\polyline(-5.90458,-1.00000)(-5.80534,-1.00000)%
\polyline(-5.70611,-1.00000)(-5.60687,-1.00000)\polyline(-5.50763,-1.00000)(-5.40840,-1.00000)%
\polyline(-5.30916,-1.00000)(-5.20992,-1.00000)\polyline(-5.11069,-1.00000)(-5.01145,-1.00000)%
\polyline(-4.91221,-1.00000)(-4.81298,-1.00000)\polyline(-4.71374,-1.00000)(-4.61450,-1.00000)%
\polyline(-4.51527,-1.00000)(-4.41603,-1.00000)\polyline(-4.31679,-1.00000)(-4.21756,-1.00000)%
\polyline(-4.11832,-1.00000)(-4.01908,-1.00000)\polyline(-3.91985,-1.00000)(-3.82061,-1.00000)%
\polyline(-3.72137,-1.00000)(-3.62214,-1.00000)\polyline(-3.52290,-1.00000)(-3.42366,-1.00000)%
\polyline(-3.32443,-1.00000)(-3.22519,-1.00000)\polyline(-3.12595,-1.00000)(-3.02672,-1.00000)%
\polyline(-2.92748,-1.00000)(-2.82824,-1.00000)\polyline(-2.72901,-1.00000)(-2.62977,-1.00000)%
\polyline(-2.53053,-1.00000)(-2.43130,-1.00000)\polyline(-2.33206,-1.00000)(-2.23282,-1.00000)%
\polyline(-2.13359,-1.00000)(-2.03435,-1.00000)\polyline(-1.93511,-1.00000)(-1.83588,-1.00000)%
\polyline(-1.73664,-1.00000)(-1.63740,-1.00000)\polyline(-1.53817,-1.00000)(-1.43893,-1.00000)%
\polyline(-1.33969,-1.00000)(-1.24046,-1.00000)\polyline(-1.14122,-1.00000)(-1.04198,-1.00000)%
\polyline(-0.94275,-1.00000)(-0.84351,-1.00000)\polyline(-0.74427,-1.00000)(-0.64504,-1.00000)%
\polyline(-0.54580,-1.00000)(-0.44656,-1.00000)\polyline(-0.34733,-1.00000)(-0.24809,-1.00000)%
\polyline(-0.14885,-1.00000)(-0.04962,-1.00000)\polyline(0.04962,-1.00000)(0.14885,-1.00000)%
\polyline(0.24809,-1.00000)(0.34733,-1.00000)\polyline(0.44656,-1.00000)(0.54580,-1.00000)%
\polyline(0.64504,-1.00000)(0.74427,-1.00000)\polyline(0.84351,-1.00000)(0.94275,-1.00000)%
\polyline(1.04198,-1.00000)(1.14122,-1.00000)\polyline(1.24046,-1.00000)(1.33969,-1.00000)%
\polyline(1.43893,-1.00000)(1.53817,-1.00000)\polyline(1.63740,-1.00000)(1.73664,-1.00000)%
\polyline(1.83588,-1.00000)(1.93511,-1.00000)\polyline(2.03435,-1.00000)(2.13359,-1.00000)%
\polyline(2.23282,-1.00000)(2.33206,-1.00000)\polyline(2.43130,-1.00000)(2.53053,-1.00000)%
\polyline(2.62977,-1.00000)(2.72901,-1.00000)\polyline(2.82824,-1.00000)(2.92748,-1.00000)%
\polyline(3.02672,-1.00000)(3.12595,-1.00000)\polyline(3.22519,-1.00000)(3.32443,-1.00000)%
\polyline(3.42366,-1.00000)(3.52290,-1.00000)\polyline(3.62214,-1.00000)(3.72137,-1.00000)%
\polyline(3.82061,-1.00000)(3.91985,-1.00000)\polyline(4.01908,-1.00000)(4.11832,-1.00000)%
\polyline(4.21756,-1.00000)(4.31679,-1.00000)\polyline(4.41603,-1.00000)(4.51527,-1.00000)%
\polyline(4.61450,-1.00000)(4.71374,-1.00000)\polyline(4.81298,-1.00000)(4.91221,-1.00000)%
\polyline(5.01145,-1.00000)(5.11069,-1.00000)\polyline(5.20992,-1.00000)(5.30916,-1.00000)%
\polyline(5.40840,-1.00000)(5.50763,-1.00000)\polyline(5.60687,-1.00000)(5.70611,-1.00000)%
\polyline(5.80534,-1.00000)(5.90458,-1.00000)\polyline(6.00382,-1.00000)(6.10305,-1.00000)%
\polyline(6.20229,-1.00000)(6.30153,-1.00000)\polyline(6.40076,-1.00000)(6.50000,-1.00000)%
%
%
\polyline(1.57080,0.05000)(1.57080,-0.05000)%
%
\settowidth{\Width}{$\tfrac{\pi}{2}$}\setlength{\Width}{-0.5\Width}%
\settoheight{\Height}{$\tfrac{\pi}{2}$}\settodepth{\Depth}{$\tfrac{\pi}{2}$}\setlength{\Height}{-\Height}%
\put(1.5700000,-0.1000000){\hspace*{\Width}\raisebox{\Height}{$\tfrac{\pi}{2}$}}%
%
\polyline(3.14159,0.05000)(3.14159,-0.05000)%
%
\settowidth{\Width}{$\pi$}\setlength{\Width}{-0.5\Width}%
\settoheight{\Height}{$\pi$}\settodepth{\Depth}{$\pi$}\setlength{\Height}{-\Height}%
\put(3.1400000,-0.1000000){\hspace*{\Width}\raisebox{\Height}{$\pi$}}%
%
\polyline(6.28319,0.05000)(6.28319,-0.05000)%
%
\settowidth{\Width}{$2\pi$}\setlength{\Width}{-0.5\Width}%
\settoheight{\Height}{$2\pi$}\settodepth{\Depth}{$2\pi$}\setlength{\Height}{-\Height}%
\put(6.2800000,-0.1000000){\hspace*{\Width}\raisebox{\Height}{$2\pi$}}%
%
\polyline(-1.57080,0.05000)(-1.57080,-0.05000)%
%
\settowidth{\Width}{$-\tfrac{\pi}{2}$}\setlength{\Width}{-0.5\Width}%
\settoheight{\Height}{$-\tfrac{\pi}{2}$}\settodepth{\Depth}{$-\tfrac{\pi}{2}$}\setlength{\Height}{-\Height}%
\put(-1.5700000,-0.1000000){\hspace*{\Width}\raisebox{\Height}{$-\tfrac{\pi}{2}$}}%
%
\polyline(-3.14159,0.05000)(-3.14159,-0.05000)%
%
\settowidth{\Width}{$-\pi$}\setlength{\Width}{-0.5\Width}%
\settoheight{\Height}{$-\pi$}\settodepth{\Depth}{$-\pi$}\setlength{\Height}{-\Height}%
\put(-3.1400000,-0.1000000){\hspace*{\Width}\raisebox{\Height}{$-\pi$}}%
%
\polyline(-6.28319,0.05000)(-6.28319,-0.05000)%
%
\settowidth{\Width}{$-2\pi$}\setlength{\Width}{-0.5\Width}%
\settoheight{\Height}{$-2\pi$}\settodepth{\Depth}{$-2\pi$}\setlength{\Height}{-\Height}%
\put(-6.2800000,-0.1000000){\hspace*{\Width}\raisebox{\Height}{$-2\pi$}}%
%
\polyline(0.05000,-1.00000)(-0.05000,-1.00000)%
%
\settowidth{\Width}{$-1$}\setlength{\Width}{-1\Width}%
\settoheight{\Height}{$-1$}\settodepth{\Depth}{$-1$}\setlength{\Height}{-0.5\Height}\setlength{\Depth}{0.5\Depth}\addtolength{\Height}{\Depth}%
\put(-0.1000000,-1.0000000){\hspace*{\Width}\raisebox{\Height}{$-1$}}%
%
\polyline(0.05000,1.00000)(-0.05000,1.00000)%
%
\settowidth{\Width}{$1$}\setlength{\Width}{-1\Width}%
\settoheight{\Height}{$1$}\settodepth{\Depth}{$1$}\setlength{\Height}{-0.5\Height}\setlength{\Depth}{0.5\Depth}\addtolength{\Height}{\Depth}%
\put(-0.1000000,1.0000000){\hspace*{\Width}\raisebox{\Height}{$1$}}%
%
\polyline(-6.50000,-0.21512)(-6.43500,-0.15123)(-6.37000,-0.08671)(-6.30500,-0.02181)%
(-6.24000,0.04317)(-6.17500,0.10797)(-6.11000,0.17232)(-6.04500,0.23594)(-5.98000,0.29856)%
(-5.91500,0.35992)(-5.85000,0.41976)(-5.78500,0.47783)(-5.72000,0.53388)(-5.65500,0.58768)%
(-5.59000,0.63899)(-5.52500,0.68760)(-5.46000,0.73332)(-5.39500,0.77593)(-5.33000,0.81526)%
(-5.26500,0.85116)(-5.20000,0.88345)(-5.13500,0.91202)(-5.07000,0.93674)(-5.00500,0.95749)%
(-4.94000,0.97421)(-4.87500,0.98681)(-4.81000,0.99524)(-4.74500,0.99947)(-4.68000,0.99948)%
(-4.61500,0.99526)(-4.55000,0.98684)(-4.48500,0.97426)(-4.42000,0.95756)(-4.35500,0.93681)%
(-4.29000,0.91211)(-4.22500,0.88356)(-4.16000,0.85127)(-4.09500,0.81539)(-4.03000,0.77607)%
(-3.96500,0.73347)(-3.90000,0.68777)(-3.83500,0.63916)(-3.77000,0.58786)(-3.70500,0.53407)%
(-3.64000,0.47803)(-3.57500,0.41997)(-3.51000,0.36013)(-3.44500,0.29877)(-3.38000,0.23616)%
(-3.31500,0.17254)(-3.25000,0.10820)(-3.18500,0.04339)(-3.12000,-0.02159)(-3.05500,-0.08648)%
(-2.99000,-0.15101)(-2.92500,-0.21490)(-2.86000,-0.27789)(-2.79500,-0.33970)(-2.73000,-0.40007)%
(-2.66500,-0.45875)(-2.60000,-0.51550)(-2.53500,-0.57007)(-2.47000,-0.62223)(-2.40500,-0.67177)%
(-2.34000,-0.71846)(-2.27500,-0.76213)(-2.21000,-0.80257)(-2.14500,-0.83963)(-2.08000,-0.87313)%
(-2.01500,-0.90295)(-1.95000,-0.92896)(-1.88500,-0.95104)(-1.82000,-0.96911)(-1.75500,-0.98308)%
(-1.69000,-0.99290)(-1.62500,-0.99853)(-1.56000,-0.99994)(-1.49500,-0.99713)(-1.43000,-0.99010)%
(-1.36500,-0.97890)(-1.30000,-0.96356)(-1.23500,-0.94415)(-1.17000,-0.92075)(-1.10500,-0.89346)%
(-1.04000,-0.86240)(-0.97500,-0.82770)(-0.91000,-0.78950)(-0.84500,-0.74797)(-0.78000,-0.70328)%
(-0.71500,-0.65562)(-0.65000,-0.60519)(-0.58500,-0.55220)(-0.52000,-0.49688)(-0.45500,-0.43946)%
(-0.39000,-0.38019)(-0.32500,-0.31931)(-0.26000,-0.25708)(-0.19500,-0.19377)(-0.13000,-0.12963)%
(-0.06500,-0.06495)(0.00000,0.00000)(0.06500,0.06495)(0.13000,0.12963)(0.19500,0.19377)%
(0.26000,0.25708)(0.32500,0.31931)(0.39000,0.38019)(0.45500,0.43946)(0.52000,0.49688)%
(0.58500,0.55220)(0.65000,0.60519)(0.71500,0.65562)(0.78000,0.70328)(0.84500,0.74797)%
(0.91000,0.78950)(0.97500,0.82770)(1.04000,0.86240)(1.10500,0.89346)(1.17000,0.92075)%
(1.23500,0.94415)(1.30000,0.96356)(1.36500,0.97890)(1.43000,0.99010)(1.49500,0.99713)%
(1.56000,0.99994)(1.62500,0.99853)(1.69000,0.99290)(1.75500,0.98308)(1.82000,0.96911)%
(1.88500,0.95104)(1.95000,0.92896)(2.01500,0.90295)(2.08000,0.87313)(2.14500,0.83963)%
(2.21000,0.80257)(2.27500,0.76213)(2.34000,0.71846)(2.40500,0.67177)(2.47000,0.62223)%
(2.53500,0.57007)(2.60000,0.51550)(2.66500,0.45875)(2.73000,0.40007)(2.79500,0.33970)%
(2.86000,0.27789)(2.92500,0.21490)(2.99000,0.15101)(3.05500,0.08648)(3.12000,0.02159)%
(3.18500,-0.04339)(3.25000,-0.10820)(3.31500,-0.17254)(3.38000,-0.23616)(3.44500,-0.29877)%
(3.51000,-0.36013)(3.57500,-0.41997)(3.64000,-0.47803)(3.70500,-0.53407)(3.77000,-0.58786)%
(3.83500,-0.63916)(3.90000,-0.68777)(3.96500,-0.73347)(4.03000,-0.77607)(4.09500,-0.81539)%
(4.16000,-0.85127)(4.22500,-0.88356)(4.29000,-0.91211)(4.35500,-0.93681)(4.42000,-0.95756)%
(4.48500,-0.97426)(4.55000,-0.98684)(4.61500,-0.99526)(4.68000,-0.99948)(4.74500,-0.99947)%
(4.81000,-0.99524)(4.87500,-0.98681)(4.94000,-0.97421)(5.00500,-0.95749)(5.07000,-0.93674)%
(5.13500,-0.91202)(5.20000,-0.88345)(5.26500,-0.85116)(5.33000,-0.81526)(5.39500,-0.77593)%
(5.46000,-0.73332)(5.52500,-0.68760)(5.59000,-0.63899)(5.65500,-0.58768)(5.72000,-0.53388)%
(5.78500,-0.47783)(5.85000,-0.41976)(5.91500,-0.35992)(5.98000,-0.29856)(6.04500,-0.23594)%
(6.11000,-0.17232)(6.17500,-0.10797)(6.24000,-0.04317)(6.30500,0.02181)(6.37000,0.08671)%
(6.43500,0.15123)(6.50000,0.21512)%
%
\polyline(-6.50000,0.00000)(6.50000,0.00000)%
%
\polyline(0.00000,-1.20000)(0.00000,1.20000)%
%
\settowidth{\Width}{$x$}\setlength{\Width}{0\Width}%
\settoheight{\Height}{$x$}\settodepth{\Depth}{$x$}\setlength{\Height}{-0.5\Height}\setlength{\Depth}{0.5\Depth}\addtolength{\Height}{\Depth}%
\put(6.5500000,0.0000000){\hspace*{\Width}\raisebox{\Height}{$x$}}%
%
\settowidth{\Width}{$y$}\setlength{\Width}{-0.5\Width}%
\settoheight{\Height}{$y$}\settodepth{\Depth}{$y$}\setlength{\Height}{\Depth}%
\put(0.0000000,1.2500000){\hspace*{\Width}\raisebox{\Height}{$y$}}%
%
\settowidth{\Width}{O}\setlength{\Width}{0\Width}%
\settoheight{\Height}{O}\settodepth{\Depth}{O}\setlength{\Height}{-\Height}%
\put(0.0500000,-0.0500000){\hspace*{\Width}\raisebox{\Height}{O}}%
%
\end{picture}}%}}
\putnotew{55}{5}{\color{red}$\sin(-x)=-\sin x$}
\putnotee{65}{5}{(奇関数)}
\putnotes{60}{45}{\scalebox{0.9}{%%% /polytech.git/n103/fig/graphcos.tex 
%%% Generator=graphsincos.cdy 
{\unitlength=1cm%
\begin{picture}%
(13,2.4)(-6.5,-1.2)%
\special{pn 8}%
%
\small%
{%
\color[rgb]{0,0,0}%
\special{pa -2559  -384}\special{pa -2533  -389}\special{pa -2508  -392}\special{pa -2482  -394}%
\special{pa -2457  -393}\special{pa -2431  -391}\special{pa -2406  -388}\special{pa -2380  -383}%
\special{pa -2354  -376}\special{pa -2329  -367}\special{pa -2303  -357}\special{pa -2278  -346}%
\special{pa -2252  -333}\special{pa -2226  -319}\special{pa -2201  -303}\special{pa -2175  -286}%
\special{pa -2150  -268}\special{pa -2124  -248}\special{pa -2098  -228}\special{pa -2073  -207}%
\special{pa -2047  -184}\special{pa -2022  -161}\special{pa -1996  -138}\special{pa -1970  -114}%
\special{pa -1945   -89}\special{pa -1919   -64}\special{pa -1894   -38}\special{pa -1868   -13}%
\special{pa -1843    13}\special{pa -1817    38}\special{pa -1791    64}\special{pa -1766    89}%
\special{pa -1740   113}\special{pa -1715   138}\special{pa -1689   161}\special{pa -1663   184}%
\special{pa -1638   207}\special{pa -1612   228}\special{pa -1587   248}\special{pa -1561   268}%
\special{pa -1535   286}\special{pa -1510   303}\special{pa -1484   318}\special{pa -1459   333}%
\special{pa -1433   346}\special{pa -1407   357}\special{pa -1382   367}\special{pa -1356   376}%
\special{pa -1331   383}\special{pa -1305   388}\special{pa -1280   391}\special{pa -1254   393}%
\special{pa -1228   394}\special{pa -1203   392}\special{pa -1177   389}\special{pa -1152   385}%
\special{pa -1126   378}\special{pa -1100   370}\special{pa -1075   361}\special{pa -1049   350}%
\special{pa -1024   337}\special{pa  -998   323}\special{pa  -972   308}\special{pa  -947   292}%
\special{pa  -921   274}\special{pa  -896   255}\special{pa  -870   235}\special{pa  -844   214}%
\special{pa  -819   192}\special{pa  -793   169}\special{pa  -768   146}\special{pa  -742   122}%
\special{pa  -717    97}\special{pa  -691    72}\special{pa  -665    47}\special{pa  -640    21}%
\special{pa  -614    -4}\special{pa  -589   -30}\special{pa  -563   -55}\special{pa  -537   -80}%
\special{pa  -512  -105}\special{pa  -486  -130}\special{pa  -461  -154}\special{pa  -435  -177}%
\special{pa  -409  -199}\special{pa  -384  -221}\special{pa  -358  -242}\special{pa  -333  -261}%
\special{pa  -307  -280}\special{pa  -281  -297}\special{pa  -256  -313}\special{pa  -230  -328}%
\special{pa  -205  -342}\special{pa  -179  -354}\special{pa  -154  -364}\special{pa  -128  -373}%
\special{pa  -102  -380}\special{pa   -77  -386}\special{pa   -51  -390}\special{pa   -26  -393}%
\special{pa     0  -394}\special{pa    26  -393}\special{pa    51  -390}\special{pa    77  -386}%
\special{pa   102  -380}\special{pa   128  -373}\special{pa   154  -364}\special{pa   179  -354}%
\special{pa   205  -342}\special{pa   230  -328}\special{pa   256  -313}\special{pa   281  -297}%
\special{pa   307  -280}\special{pa   333  -261}\special{pa   358  -242}\special{pa   384  -221}%
\special{pa   409  -199}\special{pa   435  -177}\special{pa   461  -154}\special{pa   486  -130}%
\special{pa   512  -105}\special{pa   537   -80}\special{pa   563   -55}\special{pa   589   -30}%
\special{pa   614    -4}\special{pa   640    21}\special{pa   665    47}\special{pa   691    72}%
\special{pa   717    97}\special{pa   742   122}\special{pa   768   146}\special{pa   793   169}%
\special{pa   819   192}\special{pa   844   214}\special{pa   870   235}\special{pa   896   255}%
\special{pa   921   274}\special{pa   947   292}\special{pa   972   308}\special{pa   998   323}%
\special{pa  1024   337}\special{pa  1049   350}\special{pa  1075   361}\special{pa  1100   370}%
\special{pa  1126   378}\special{pa  1152   385}\special{pa  1177   389}\special{pa  1203   392}%
\special{pa  1228   394}\special{pa  1254   393}\special{pa  1280   391}\special{pa  1305   388}%
\special{pa  1331   383}\special{pa  1356   376}\special{pa  1382   367}\special{pa  1407   357}%
\special{pa  1433   346}\special{pa  1459   333}\special{pa  1484   318}\special{pa  1510   303}%
\special{pa  1535   286}\special{pa  1561   268}\special{pa  1587   248}\special{pa  1612   228}%
\special{pa  1638   207}\special{pa  1663   184}\special{pa  1689   161}\special{pa  1715   138}%
\special{pa  1740   113}\special{pa  1766    89}\special{pa  1791    64}\special{pa  1817    38}%
\special{pa  1843    13}\special{pa  1868   -13}\special{pa  1894   -38}\special{pa  1919   -64}%
\special{pa  1945   -89}\special{pa  1970  -114}\special{pa  1996  -138}\special{pa  2022  -161}%
\special{pa  2047  -184}\special{pa  2073  -207}\special{pa  2098  -228}\special{pa  2124  -248}%
\special{pa  2150  -268}\special{pa  2175  -286}\special{pa  2201  -303}\special{pa  2226  -319}%
\special{pa  2252  -333}\special{pa  2278  -346}\special{pa  2303  -357}\special{pa  2329  -367}%
\special{pa  2354  -376}\special{pa  2380  -383}\special{pa  2406  -388}\special{pa  2431  -391}%
\special{pa  2457  -393}\special{pa  2482  -394}\special{pa  2508  -392}\special{pa  2533  -389}%
\special{pa  2559  -384}%
\special{fp}%
}%
{%
\color[rgb]{0,0,0}%
\special{pa -2559 -394}\special{pa -2520 -394}\special{fp}\special{pa -2481 -394}\special{pa -2442 -394}\special{fp}%
\special{pa -2403 -394}\special{pa -2364 -394}\special{fp}\special{pa -2325 -394}\special{pa -2286 -394}\special{fp}%
\special{pa -2246 -394}\special{pa -2207 -394}\special{fp}\special{pa -2168 -394}\special{pa -2129 -394}\special{fp}%
\special{pa -2090 -394}\special{pa -2051 -394}\special{fp}\special{pa -2012 -394}\special{pa -1973 -394}\special{fp}%
\special{pa -1934 -394}\special{pa -1895 -394}\special{fp}\special{pa -1856 -394}\special{pa -1817 -394}\special{fp}%
\special{pa -1778 -394}\special{pa -1739 -394}\special{fp}\special{pa -1700 -394}\special{pa -1660 -394}\special{fp}%
\special{pa -1621 -394}\special{pa -1582 -394}\special{fp}\special{pa -1543 -394}\special{pa -1504 -394}\special{fp}%
\special{pa -1465 -394}\special{pa -1426 -394}\special{fp}\special{pa -1387 -394}\special{pa -1348 -394}\special{fp}%
\special{pa -1309 -394}\special{pa -1270 -394}\special{fp}\special{pa -1231 -394}\special{pa -1192 -394}\special{fp}%
\special{pa -1153 -394}\special{pa -1113 -394}\special{fp}\special{pa -1074 -394}\special{pa -1035 -394}\special{fp}%
\special{pa -996 -394}\special{pa -957 -394}\special{fp}\special{pa -918 -394}\special{pa -879 -394}\special{fp}%
\special{pa -840 -394}\special{pa -801 -394}\special{fp}\special{pa -762 -394}\special{pa -723 -394}\special{fp}%
\special{pa -684 -394}\special{pa -645 -394}\special{fp}\special{pa -606 -394}\special{pa -567 -394}\special{fp}%
\special{pa -527 -394}\special{pa -488 -394}\special{fp}\special{pa -449 -394}\special{pa -410 -394}\special{fp}%
\special{pa -371 -394}\special{pa -332 -394}\special{fp}\special{pa -293 -394}\special{pa -254 -394}\special{fp}%
\special{pa -215 -394}\special{pa -176 -394}\special{fp}\special{pa -137 -394}\special{pa -98 -394}\special{fp}%
\special{pa -59 -394}\special{pa -20 -394}\special{fp}\special{pa 20 -394}\special{pa 59 -394}\special{fp}%
\special{pa 98 -394}\special{pa 137 -394}\special{fp}\special{pa 176 -394}\special{pa 215 -394}\special{fp}%
\special{pa 254 -394}\special{pa 293 -394}\special{fp}\special{pa 332 -394}\special{pa 371 -394}\special{fp}%
\special{pa 410 -394}\special{pa 449 -394}\special{fp}\special{pa 488 -394}\special{pa 527 -394}\special{fp}%
\special{pa 567 -394}\special{pa 606 -394}\special{fp}\special{pa 645 -394}\special{pa 684 -394}\special{fp}%
\special{pa 723 -394}\special{pa 762 -394}\special{fp}\special{pa 801 -394}\special{pa 840 -394}\special{fp}%
\special{pa 879 -394}\special{pa 918 -394}\special{fp}\special{pa 957 -394}\special{pa 996 -394}\special{fp}%
\special{pa 1035 -394}\special{pa 1074 -394}\special{fp}\special{pa 1113 -394}\special{pa 1153 -394}\special{fp}%
\special{pa 1192 -394}\special{pa 1231 -394}\special{fp}\special{pa 1270 -394}\special{pa 1309 -394}\special{fp}%
\special{pa 1348 -394}\special{pa 1387 -394}\special{fp}\special{pa 1426 -394}\special{pa 1465 -394}\special{fp}%
\special{pa 1504 -394}\special{pa 1543 -394}\special{fp}\special{pa 1582 -394}\special{pa 1621 -394}\special{fp}%
\special{pa 1660 -394}\special{pa 1700 -394}\special{fp}\special{pa 1739 -394}\special{pa 1778 -394}\special{fp}%
\special{pa 1817 -394}\special{pa 1856 -394}\special{fp}\special{pa 1895 -394}\special{pa 1934 -394}\special{fp}%
\special{pa 1973 -394}\special{pa 2012 -394}\special{fp}\special{pa 2051 -394}\special{pa 2090 -394}\special{fp}%
\special{pa 2129 -394}\special{pa 2168 -394}\special{fp}\special{pa 2207 -394}\special{pa 2246 -394}\special{fp}%
\special{pa 2286 -394}\special{pa 2325 -394}\special{fp}\special{pa 2364 -394}\special{pa 2403 -394}\special{fp}%
\special{pa 2442 -394}\special{pa 2481 -394}\special{fp}\special{pa 2520 -394}\special{pa 2559 -394}\special{fp}%
%
%
}%
{%
\color[rgb]{0,0,0}%
\special{pa -2559 394}\special{pa -2520 394}\special{fp}\special{pa -2481 394}\special{pa -2442 394}\special{fp}%
\special{pa -2403 394}\special{pa -2364 394}\special{fp}\special{pa -2325 394}\special{pa -2286 394}\special{fp}%
\special{pa -2246 394}\special{pa -2207 394}\special{fp}\special{pa -2168 394}\special{pa -2129 394}\special{fp}%
\special{pa -2090 394}\special{pa -2051 394}\special{fp}\special{pa -2012 394}\special{pa -1973 394}\special{fp}%
\special{pa -1934 394}\special{pa -1895 394}\special{fp}\special{pa -1856 394}\special{pa -1817 394}\special{fp}%
\special{pa -1778 394}\special{pa -1739 394}\special{fp}\special{pa -1700 394}\special{pa -1660 394}\special{fp}%
\special{pa -1621 394}\special{pa -1582 394}\special{fp}\special{pa -1543 394}\special{pa -1504 394}\special{fp}%
\special{pa -1465 394}\special{pa -1426 394}\special{fp}\special{pa -1387 394}\special{pa -1348 394}\special{fp}%
\special{pa -1309 394}\special{pa -1270 394}\special{fp}\special{pa -1231 394}\special{pa -1192 394}\special{fp}%
\special{pa -1153 394}\special{pa -1113 394}\special{fp}\special{pa -1074 394}\special{pa -1035 394}\special{fp}%
\special{pa -996 394}\special{pa -957 394}\special{fp}\special{pa -918 394}\special{pa -879 394}\special{fp}%
\special{pa -840 394}\special{pa -801 394}\special{fp}\special{pa -762 394}\special{pa -723 394}\special{fp}%
\special{pa -684 394}\special{pa -645 394}\special{fp}\special{pa -606 394}\special{pa -567 394}\special{fp}%
\special{pa -527 394}\special{pa -488 394}\special{fp}\special{pa -449 394}\special{pa -410 394}\special{fp}%
\special{pa -371 394}\special{pa -332 394}\special{fp}\special{pa -293 394}\special{pa -254 394}\special{fp}%
\special{pa -215 394}\special{pa -176 394}\special{fp}\special{pa -137 394}\special{pa -98 394}\special{fp}%
\special{pa -59 394}\special{pa -20 394}\special{fp}\special{pa 20 394}\special{pa 59 394}\special{fp}%
\special{pa 98 394}\special{pa 137 394}\special{fp}\special{pa 176 394}\special{pa 215 394}\special{fp}%
\special{pa 254 394}\special{pa 293 394}\special{fp}\special{pa 332 394}\special{pa 371 394}\special{fp}%
\special{pa 410 394}\special{pa 449 394}\special{fp}\special{pa 488 394}\special{pa 527 394}\special{fp}%
\special{pa 567 394}\special{pa 606 394}\special{fp}\special{pa 645 394}\special{pa 684 394}\special{fp}%
\special{pa 723 394}\special{pa 762 394}\special{fp}\special{pa 801 394}\special{pa 840 394}\special{fp}%
\special{pa 879 394}\special{pa 918 394}\special{fp}\special{pa 957 394}\special{pa 996 394}\special{fp}%
\special{pa 1035 394}\special{pa 1074 394}\special{fp}\special{pa 1113 394}\special{pa 1153 394}\special{fp}%
\special{pa 1192 394}\special{pa 1231 394}\special{fp}\special{pa 1270 394}\special{pa 1309 394}\special{fp}%
\special{pa 1348 394}\special{pa 1387 394}\special{fp}\special{pa 1426 394}\special{pa 1465 394}\special{fp}%
\special{pa 1504 394}\special{pa 1543 394}\special{fp}\special{pa 1582 394}\special{pa 1621 394}\special{fp}%
\special{pa 1660 394}\special{pa 1700 394}\special{fp}\special{pa 1739 394}\special{pa 1778 394}\special{fp}%
\special{pa 1817 394}\special{pa 1856 394}\special{fp}\special{pa 1895 394}\special{pa 1934 394}\special{fp}%
\special{pa 1973 394}\special{pa 2012 394}\special{fp}\special{pa 2051 394}\special{pa 2090 394}\special{fp}%
\special{pa 2129 394}\special{pa 2168 394}\special{fp}\special{pa 2207 394}\special{pa 2246 394}\special{fp}%
\special{pa 2286 394}\special{pa 2325 394}\special{fp}\special{pa 2364 394}\special{pa 2403 394}\special{fp}%
\special{pa 2442 394}\special{pa 2481 394}\special{fp}\special{pa 2520 394}\special{pa 2559 394}\special{fp}%
%
%
}%
{%
\color[rgb]{0,0,0}%
\special{pa   618   -20}\special{pa   618    20}%
\special{fp}%
}%
{%
\color[rgb]{0,0,0}%
\settowidth{\Width}{$\tfrac{\pi}{2}$}\setlength{\Width}{-0.5\Width}%
\settoheight{\Height}{$\tfrac{\pi}{2}$}\settodepth{\Depth}{$\tfrac{\pi}{2}$}\setlength{\Height}{-\Height}%
\put(1.5700000,-0.1000000){\hspace*{\Width}\raisebox{\Height}{$\tfrac{\pi}{2}$}}%
%
}%
{%
\color[rgb]{0,0,0}%
\special{pa  1237   -20}\special{pa  1237    20}%
\special{fp}%
}%
{%
\color[rgb]{0,0,0}%
\settowidth{\Width}{$\pi$}\setlength{\Width}{-0.5\Width}%
\settoheight{\Height}{$\pi$}\settodepth{\Depth}{$\pi$}\setlength{\Height}{-\Height}%
\put(3.1400000,-0.1000000){\hspace*{\Width}\raisebox{\Height}{$\pi$}}%
%
}%
{%
\color[rgb]{0,0,0}%
\special{pa  2474   -20}\special{pa  2474    20}%
\special{fp}%
}%
{%
\color[rgb]{0,0,0}%
\settowidth{\Width}{$2\pi$}\setlength{\Width}{-0.5\Width}%
\settoheight{\Height}{$2\pi$}\settodepth{\Depth}{$2\pi$}\setlength{\Height}{-\Height}%
\put(6.2800000,-0.1000000){\hspace*{\Width}\raisebox{\Height}{$2\pi$}}%
%
}%
{%
\color[rgb]{0,0,0}%
\special{pa  -618   -20}\special{pa  -618    20}%
\special{fp}%
}%
{%
\color[rgb]{0,0,0}%
\settowidth{\Width}{$-\tfrac{\pi}{2}$}\setlength{\Width}{-0.5\Width}%
\settoheight{\Height}{$-\tfrac{\pi}{2}$}\settodepth{\Depth}{$-\tfrac{\pi}{2}$}\setlength{\Height}{-\Height}%
\put(-1.5700000,-0.1000000){\hspace*{\Width}\raisebox{\Height}{$-\tfrac{\pi}{2}$}}%
%
}%
{%
\color[rgb]{0,0,0}%
\special{pa -1237   -20}\special{pa -1237    20}%
\special{fp}%
}%
{%
\color[rgb]{0,0,0}%
\settowidth{\Width}{$-\pi$}\setlength{\Width}{-0.5\Width}%
\settoheight{\Height}{$-\pi$}\settodepth{\Depth}{$-\pi$}\setlength{\Height}{-\Height}%
\put(-3.1400000,-0.1000000){\hspace*{\Width}\raisebox{\Height}{$-\pi$}}%
%
}%
{%
\color[rgb]{0,0,0}%
\special{pa -2474   -20}\special{pa -2474    20}%
\special{fp}%
}%
{%
\color[rgb]{0,0,0}%
\settowidth{\Width}{$-2\pi$}\setlength{\Width}{-0.5\Width}%
\settoheight{\Height}{$-2\pi$}\settodepth{\Depth}{$-2\pi$}\setlength{\Height}{-\Height}%
\put(-6.2800000,-0.1000000){\hspace*{\Width}\raisebox{\Height}{$-2\pi$}}%
%
}%
{%
\color[rgb]{0,0,0}%
\special{pa    20   394}\special{pa   -20   394}%
\special{fp}%
}%
{%
\color[rgb]{0,0,0}%
\settowidth{\Width}{$-1$}\setlength{\Width}{-1\Width}%
\settoheight{\Height}{$-1$}\settodepth{\Depth}{$-1$}\setlength{\Height}{-0.5\Height}\setlength{\Depth}{0.5\Depth}\addtolength{\Height}{\Depth}%
\put(-0.1000000,-1.0000000){\hspace*{\Width}\raisebox{\Height}{$-1$}}%
%
}%
{%
\color[rgb]{0,0,0}%
\special{pa    20  -394}\special{pa   -20  -394}%
\special{fp}%
}%
{%
\color[rgb]{0,0,0}%
\settowidth{\Width}{$1$}\setlength{\Width}{-1\Width}%
\settoheight{\Height}{$1$}\settodepth{\Depth}{$1$}\setlength{\Height}{-0.5\Height}\setlength{\Depth}{0.5\Depth}\addtolength{\Height}{\Depth}%
\put(-0.1000000,1.0000000){\hspace*{\Width}\raisebox{\Height}{$1$}}%
%
}%
\special{pa -2559    -0}\special{pa  2559    -0}%
\special{fp}%
\special{pa     0   472}\special{pa     0  -472}%
\special{fp}%
\settowidth{\Width}{$x$}\setlength{\Width}{0\Width}%
\settoheight{\Height}{$x$}\settodepth{\Depth}{$x$}\setlength{\Height}{-0.5\Height}\setlength{\Depth}{0.5\Depth}\addtolength{\Height}{\Depth}%
\put(6.5500000,0.0000000){\hspace*{\Width}\raisebox{\Height}{$x$}}%
%
\settowidth{\Width}{$y$}\setlength{\Width}{-0.5\Width}%
\settoheight{\Height}{$y$}\settodepth{\Depth}{$y$}\setlength{\Height}{\Depth}%
\put(0.0000000,1.2500000){\hspace*{\Width}\raisebox{\Height}{$y$}}%
%
\settowidth{\Width}{O}\setlength{\Width}{0\Width}%
\settoheight{\Height}{O}\settodepth{\Depth}{O}\setlength{\Height}{-\Height}%
\put(0.0500000,-0.0500000){\hspace*{\Width}\raisebox{\Height}{O}}%
%
\end{picture}}%}}
\putnotew{55}{40}{\color{red}$\cos(-x)=\cos x$}
\putnotee{65}{40}{(偶関数)}
\end{layer}

%%%%%%%%%%%%

%%%%%%%%%%%%%%%%%%%%

\newslide{加法定理による等式証明($x+\pi$)}

\vspace*{18mm}

\slidepage
\begin{itemize}
\item
以下,角をラジアンで表す
\item
{\normalsize\color{blue} $\sin 0=0,\ \cos 0=1,\ \sin\pi=0,\ \cos\pi=-1$}
\item
$\sin(x+\pi)=$
$\sin x\cos\pi+\cos x\sin\pi$
$=-\sin x$
\item
$\cos(x+\pi)=$
$\cos x\cos\pi-\sin x\sin\pi$
$=-\cos x$
\item
[課題]\monban 次はどうなるか\seteda{50}\\
\eda{$\sin(\pi-x)$}\eda{$\cos(\pi-x)$}
\end{itemize}

\newslide{グラフでの意味 $x+\pi$}

\vspace*{18mm}

%%repeat=2
\slidepage

\begin{layer}{120}{0}
\putnotes{60}{5}{\scalebox{0.9}{%%% /polytech.git/n104/fig/graphsinpi.tex 
%%% Generator=n103sincos.cdy 
{\unitlength=1cm%
\begin{picture}%
(14,3)(-7,-1.5)%
\special{pn 8}%
%
{%
\color[rgb]{0,0,0}%
\special{pa -2756   259}\special{pa -2728   237}\special{pa -2701   215}\special{pa -2673   191}%
\special{pa -2646   167}\special{pa -2618   141}\special{pa -2591   115}\special{pa -2563    89}%
\special{pa -2535    61}\special{pa -2508    34}\special{pa -2480     7}\special{pa -2453   -21}%
\special{pa -2425   -48}\special{pa -2398   -76}\special{pa -2370  -102}\special{pa -2343  -129}%
\special{pa -2315  -154}\special{pa -2287  -179}\special{pa -2260  -203}\special{pa -2232  -227}%
\special{pa -2205  -249}\special{pa -2177  -269}\special{pa -2150  -289}\special{pa -2122  -307}%
\special{pa -2094  -323}\special{pa -2067  -338}\special{pa -2039  -351}\special{pa -2012  -363}%
\special{pa -1984  -373}\special{pa -1957  -381}\special{pa -1929  -387}\special{pa -1902  -391}%
\special{pa -1874  -393}\special{pa -1846  -394}\special{pa -1819  -392}\special{pa -1791  -389}%
\special{pa -1764  -383}\special{pa -1736  -376}\special{pa -1709  -367}\special{pa -1681  -356}%
\special{pa -1654  -343}\special{pa -1626  -329}\special{pa -1598  -313}\special{pa -1571  -295}%
\special{pa -1543  -276}\special{pa -1516  -256}\special{pa -1488  -235}\special{pa -1461  -212}%
\special{pa -1433  -188}\special{pa -1406  -164}\special{pa -1378  -138}\special{pa -1350  -112}%
\special{pa -1323   -85}\special{pa -1295   -58}\special{pa -1268   -31}\special{pa -1240    -3}%
\special{pa -1213    24}\special{pa -1185    52}\special{pa -1157    79}\special{pa -1130   106}%
\special{pa -1102   132}\special{pa -1075   158}\special{pa -1047   182}\special{pa -1020   206}%
\special{pa  -992   229}\special{pa  -965   251}\special{pa  -937   272}\special{pa  -909   291}%
\special{pa  -882   309}\special{pa  -854   325}\special{pa  -827   340}\special{pa  -799   353}%
\special{pa  -772   364}\special{pa  -744   374}\special{pa  -717   382}\special{pa  -689   387}%
\special{pa  -661   391}\special{pa  -634   393}\special{pa  -606   394}\special{pa  -579   392}%
\special{pa  -551   388}\special{pa  -524   382}\special{pa  -496   375}\special{pa  -469   365}%
\special{pa  -441   354}\special{pa  -413   342}\special{pa  -386   327}\special{pa  -358   311}%
\special{pa  -331   293}\special{pa  -303   274}\special{pa  -276   254}\special{pa  -248   232}%
\special{pa  -220   209}\special{pa  -193   185}\special{pa  -165   161}\special{pa  -138   135}%
\special{pa  -110   109}\special{pa   -83    82}\special{pa   -55    55}\special{pa   -28    28}%
\special{pa     0    -0}\special{pa    28   -28}\special{pa    55   -55}\special{pa    83   -82}%
\special{pa   110  -109}\special{pa   138  -135}\special{pa   165  -161}\special{pa   193  -185}%
\special{pa   220  -209}\special{pa   248  -232}\special{pa   276  -254}\special{pa   303  -274}%
\special{pa   331  -293}\special{pa   358  -311}\special{pa   386  -327}\special{pa   413  -342}%
\special{pa   441  -354}\special{pa   469  -365}\special{pa   496  -375}\special{pa   524  -382}%
\special{pa   551  -388}\special{pa   579  -392}\special{pa   606  -394}\special{pa   634  -393}%
\special{pa   661  -391}\special{pa   689  -387}\special{pa   717  -382}\special{pa   744  -374}%
\special{pa   772  -364}\special{pa   799  -353}\special{pa   827  -340}\special{pa   854  -325}%
\special{pa   882  -309}\special{pa   909  -291}\special{pa   937  -272}\special{pa   965  -251}%
\special{pa   992  -229}\special{pa  1020  -206}\special{pa  1047  -182}\special{pa  1075  -158}%
\special{pa  1102  -132}\special{pa  1130  -106}\special{pa  1157   -79}\special{pa  1185   -52}%
\special{pa  1213   -24}\special{pa  1240     3}\special{pa  1268    31}\special{pa  1295    58}%
\special{pa  1323    85}\special{pa  1350   112}\special{pa  1378   138}\special{pa  1406   164}%
\special{pa  1433   188}\special{pa  1461   212}\special{pa  1488   235}\special{pa  1516   256}%
\special{pa  1543   276}\special{pa  1571   295}\special{pa  1598   313}\special{pa  1626   329}%
\special{pa  1654   343}\special{pa  1681   356}\special{pa  1709   367}\special{pa  1736   376}%
\special{pa  1764   383}\special{pa  1791   389}\special{pa  1819   392}\special{pa  1846   394}%
\special{pa  1874   393}\special{pa  1902   391}\special{pa  1929   387}\special{pa  1957   381}%
\special{pa  1984   373}\special{pa  2012   363}\special{pa  2039   351}\special{pa  2067   338}%
\special{pa  2094   323}\special{pa  2122   307}\special{pa  2150   289}\special{pa  2177   269}%
\special{pa  2205   249}\special{pa  2232   227}\special{pa  2260   203}\special{pa  2287   179}%
\special{pa  2315   154}\special{pa  2343   129}\special{pa  2370   102}\special{pa  2398    76}%
\special{pa  2425    48}\special{pa  2453    21}\special{pa  2480    -7}\special{pa  2508   -34}%
\special{pa  2535   -61}\special{pa  2563   -89}\special{pa  2591  -115}\special{pa  2618  -141}%
\special{pa  2646  -167}\special{pa  2673  -191}\special{pa  2701  -215}\special{pa  2728  -237}%
\special{pa  2756  -259}%
\special{fp}%
}%
\special{pa   394   -20}\special{pa   394    20}%
\special{fp}%
\settowidth{\Width}{$x$}\setlength{\Width}{-0.5\Width}%
\settoheight{\Height}{$x$}\settodepth{\Depth}{$x$}\setlength{\Height}{-\Height}%
\put(1.0000000,-0.1000000){\hspace*{\Width}\raisebox{\Height}{$x$}}%
%
\special{pa  1631   -20}\special{pa  1631    20}%
\special{fp}%
\settowidth{\Width}{$x+\pi$}\setlength{\Width}{-0.5\Width}%
\settoheight{\Height}{$x+\pi$}\settodepth{\Depth}{$x+\pi$}\setlength{\Height}{\Depth}%
\put(4.1400000,0.1000000){\hspace*{\Width}\raisebox{\Height}{$x+\pi$}}%
%
{%
\color[cmyk]{0,1,1,0}%
\special{pa   394    -0}\special{pa   394  -331}%
\special{fp}%
}%
{%
\color[cmyk]{0,1,1,0}%
\special{pa  1631    -0}\special{pa  1631   331}%
\special{fp}%
}%
{%
\color[rgb]{0,0,0}%
\special{pn 8}%
\special{pa 390 -1}\special{pa 397 1}\special{fp}\special{pa 427 13}\special{pa 434 16}\special{fp}%
\special{pa 464 27}\special{pa 471 30}\special{fp}\special{pa 501 40}\special{pa 509 43}\special{fp}%
\special{pa 539 53}\special{pa 547 55}\special{fp}\special{pa 577 64}\special{pa 585 66}\special{fp}%
\special{pa 616 74}\special{pa 623 76}\special{fp}\special{pa 654 83}\special{pa 662 85}\special{fp}%
\special{pa 693 92}\special{pa 701 93}\special{fp}\special{pa 732 99}\special{pa 740 100}\special{fp}%
\special{pa 771 105}\special{pa 779 107}\special{fp}\special{pa 810 111}\special{pa 818 112}\special{fp}%
\special{pa 850 115}\special{pa 858 116}\special{fp}\special{pa 889 119}\special{pa 897 120}\special{fp}%
\special{pa 929 122}\special{pa 937 122}\special{fp}\special{pa 968 123}\special{pa 976 123}\special{fp}%
\special{pa 1008 124}\special{pa 1016 124}\special{fp}\special{pa 1048 123}\special{pa 1056 123}\special{fp}%
\special{pa 1087 122}\special{pa 1095 122}\special{fp}\special{pa 1127 120}\special{pa 1135 119}\special{fp}%
\special{pa 1167 116}\special{pa 1175 115}\special{fp}\special{pa 1206 112}\special{pa 1214 111}\special{fp}%
\special{pa 1245 107}\special{pa 1253 105}\special{fp}\special{pa 1285 100}\special{pa 1292 99}\special{fp}%
\special{pa 1324 93}\special{pa 1331 92}\special{fp}\special{pa 1362 85}\special{pa 1370 83}\special{fp}%
\special{pa 1401 76}\special{pa 1409 74}\special{fp}\special{pa 1439 66}\special{pa 1447 64}\special{fp}%
\special{pa 1477 55}\special{pa 1485 53}\special{fp}\special{pa 1515 43}\special{pa 1523 40}\special{fp}%
\special{pa 1553 30}\special{pa 1560 27}\special{fp}\special{pa 1590 16}\special{pa 1597 13}\special{fp}%
\special{pa 1627 1}\special{pa 1634 -1}\special{fp}\special{pn 8}%
}%
\settowidth{\Width}{$\pi$}\setlength{\Width}{-0.5\Width}%
\settoheight{\Height}{$\pi$}\settodepth{\Depth}{$\pi$}\setlength{\Height}{-0.5\Height}\setlength{\Depth}{0.5\Depth}\addtolength{\Height}{\Depth}%
\put(2.5700000,-0.3100000){\hspace*{\Width}\raisebox{\Height}{$\pi$}}%
%
\special{pa -2756    -0}\special{pa  2756    -0}%
\special{fp}%
\special{pa     0   591}\special{pa     0  -591}%
\special{fp}%
\settowidth{\Width}{$x$}\setlength{\Width}{0\Width}%
\settoheight{\Height}{$x$}\settodepth{\Depth}{$x$}\setlength{\Height}{-0.5\Height}\setlength{\Depth}{0.5\Depth}\addtolength{\Height}{\Depth}%
\put(7.0500000,0.0000000){\hspace*{\Width}\raisebox{\Height}{$x$}}%
%
\settowidth{\Width}{$y$}\setlength{\Width}{-0.5\Width}%
\settoheight{\Height}{$y$}\settodepth{\Depth}{$y$}\setlength{\Height}{\Depth}%
\put(0.0000000,1.5500000){\hspace*{\Width}\raisebox{\Height}{$y$}}%
%
\settowidth{\Width}{O}\setlength{\Width}{0\Width}%
\settoheight{\Height}{O}\settodepth{\Depth}{O}\setlength{\Height}{-\Height}%
\put(0.0500000,-0.0500000){\hspace*{\Width}\raisebox{\Height}{O}}%
%
\end{picture}}%}}
\putnotes{60}{45}{\scalebox{0.9}{%%% /polytech.git/n104/fig/graphcospi.tex 
%%% Generator=n103sincos.cdy 
{\unitlength=1cm%
\begin{picture}%
(14,3)(-7,-1.5)%
\special{pn 8}%
%
{%
\color[rgb]{0,0,0}%
\special{pa -2756  -297}\special{pa -2728  -314}\special{pa -2701  -330}\special{pa -2673  -344}%
\special{pa -2646  -357}\special{pa -2618  -368}\special{pa -2591  -376}\special{pa -2563  -384}%
\special{pa -2535  -389}\special{pa -2508  -392}\special{pa -2480  -394}\special{pa -2453  -393}%
\special{pa -2425  -391}\special{pa -2398  -386}\special{pa -2370  -380}\special{pa -2343  -372}%
\special{pa -2315  -362}\special{pa -2287  -350}\special{pa -2260  -337}\special{pa -2232  -322}%
\special{pa -2205  -305}\special{pa -2177  -287}\special{pa -2150  -268}\special{pa -2122  -247}%
\special{pa -2094  -225}\special{pa -2067  -202}\special{pa -2039  -177}\special{pa -2012  -152}%
\special{pa -1984  -127}\special{pa -1957  -100}\special{pa -1929   -73}\special{pa -1902   -46}%
\special{pa -1874   -19}\special{pa -1846     9}\special{pa -1819    36}\special{pa -1791    64}%
\special{pa -1764    91}\special{pa -1736   117}\special{pa -1709   143}\special{pa -1681   169}%
\special{pa -1654   193}\special{pa -1626   217}\special{pa -1598   239}\special{pa -1571   260}%
\special{pa -1543   280}\special{pa -1516   299}\special{pa -1488   316}\special{pa -1461   332}%
\special{pa -1433   346}\special{pa -1406   358}\special{pa -1378   369}\special{pa -1350   377}%
\special{pa -1323   384}\special{pa -1295   389}\special{pa -1268   392}\special{pa -1240   394}%
\special{pa -1213   393}\special{pa -1185   390}\special{pa -1157   386}\special{pa -1130   379}%
\special{pa -1102   371}\special{pa -1075   361}\special{pa -1047   349}\special{pa -1020   335}%
\special{pa  -992   320}\special{pa  -965   303}\special{pa  -937   285}\special{pa  -909   265}%
\special{pa  -882   244}\special{pa  -854   222}\special{pa  -827   199}\special{pa  -799   175}%
\special{pa  -772   149}\special{pa  -744   124}\special{pa  -717    97}\special{pa  -689    70}%
\special{pa  -661    43}\special{pa  -634    15}\special{pa  -606   -12}\special{pa  -579   -40}%
\special{pa  -551   -67}\special{pa  -524   -94}\special{pa  -496  -120}\special{pa  -469  -146}%
\special{pa  -441  -172}\special{pa  -413  -196}\special{pa  -386  -219}\special{pa  -358  -242}%
\special{pa  -331  -263}\special{pa  -303  -283}\special{pa  -276  -301}\special{pa  -248  -318}%
\special{pa  -220  -334}\special{pa  -193  -347}\special{pa  -165  -359}\special{pa  -138  -370}%
\special{pa  -110  -378}\special{pa   -83  -385}\special{pa   -55  -390}\special{pa   -28  -393}%
\special{pa     0  -394}\special{pa    28  -393}\special{pa    55  -390}\special{pa    83  -385}%
\special{pa   110  -378}\special{pa   138  -370}\special{pa   165  -359}\special{pa   193  -347}%
\special{pa   220  -334}\special{pa   248  -318}\special{pa   276  -301}\special{pa   303  -283}%
\special{pa   331  -263}\special{pa   358  -242}\special{pa   386  -219}\special{pa   413  -196}%
\special{pa   441  -172}\special{pa   469  -146}\special{pa   496  -120}\special{pa   524   -94}%
\special{pa   551   -67}\special{pa   579   -40}\special{pa   606   -12}\special{pa   634    15}%
\special{pa   661    43}\special{pa   689    70}\special{pa   717    97}\special{pa   744   124}%
\special{pa   772   149}\special{pa   799   175}\special{pa   827   199}\special{pa   854   222}%
\special{pa   882   244}\special{pa   909   265}\special{pa   937   285}\special{pa   965   303}%
\special{pa   992   320}\special{pa  1020   335}\special{pa  1047   349}\special{pa  1075   361}%
\special{pa  1102   371}\special{pa  1130   379}\special{pa  1157   386}\special{pa  1185   390}%
\special{pa  1213   393}\special{pa  1240   394}\special{pa  1268   392}\special{pa  1295   389}%
\special{pa  1323   384}\special{pa  1350   377}\special{pa  1378   369}\special{pa  1406   358}%
\special{pa  1433   346}\special{pa  1461   332}\special{pa  1488   316}\special{pa  1516   299}%
\special{pa  1543   280}\special{pa  1571   260}\special{pa  1598   239}\special{pa  1626   217}%
\special{pa  1654   193}\special{pa  1681   169}\special{pa  1709   143}\special{pa  1736   117}%
\special{pa  1764    91}\special{pa  1791    64}\special{pa  1819    36}\special{pa  1846     9}%
\special{pa  1874   -19}\special{pa  1902   -46}\special{pa  1929   -73}\special{pa  1957  -100}%
\special{pa  1984  -127}\special{pa  2012  -152}\special{pa  2039  -177}\special{pa  2067  -202}%
\special{pa  2094  -225}\special{pa  2122  -247}\special{pa  2150  -268}\special{pa  2177  -287}%
\special{pa  2205  -305}\special{pa  2232  -322}\special{pa  2260  -337}\special{pa  2287  -350}%
\special{pa  2315  -362}\special{pa  2343  -372}\special{pa  2370  -380}\special{pa  2398  -386}%
\special{pa  2425  -391}\special{pa  2453  -393}\special{pa  2480  -394}\special{pa  2508  -392}%
\special{pa  2535  -389}\special{pa  2563  -384}\special{pa  2591  -376}\special{pa  2618  -368}%
\special{pa  2646  -357}\special{pa  2673  -344}\special{pa  2701  -330}\special{pa  2728  -314}%
\special{pa  2756  -297}%
\special{fp}%
}%
\special{pa   394   -20}\special{pa   394    20}%
\special{fp}%
\settowidth{\Width}{$x$}\setlength{\Width}{-0.5\Width}%
\settoheight{\Height}{$x$}\settodepth{\Depth}{$x$}\setlength{\Height}{-\Height}%
\put(1.0000000,-0.1000000){\hspace*{\Width}\raisebox{\Height}{$x$}}%
%
\special{pa  1631   -20}\special{pa  1631    20}%
\special{fp}%
\settowidth{\Width}{$x+\pi$}\setlength{\Width}{-0.5\Width}%
\settoheight{\Height}{$x+\pi$}\settodepth{\Depth}{$x+\pi$}\setlength{\Height}{\Depth}%
\put(4.1400000,0.1000000){\hspace*{\Width}\raisebox{\Height}{$x+\pi$}}%
%
{%
\color[cmyk]{0,1,1,0}%
\special{pa   394    -0}\special{pa   394  -213}%
\special{fp}%
}%
{%
\color[cmyk]{0,1,1,0}%
\special{pa  1631    -0}\special{pa  1631   213}%
\special{fp}%
}%
{%
\color[rgb]{0,0,0}%
\special{pn 8}%
\special{pa 390 -1}\special{pa 397 1}\special{fp}\special{pa 427 13}\special{pa 434 16}\special{fp}%
\special{pa 464 27}\special{pa 471 30}\special{fp}\special{pa 501 40}\special{pa 509 43}\special{fp}%
\special{pa 539 53}\special{pa 547 55}\special{fp}\special{pa 577 64}\special{pa 585 66}\special{fp}%
\special{pa 616 74}\special{pa 623 76}\special{fp}\special{pa 654 83}\special{pa 662 85}\special{fp}%
\special{pa 693 92}\special{pa 701 93}\special{fp}\special{pa 732 99}\special{pa 740 100}\special{fp}%
\special{pa 771 105}\special{pa 779 107}\special{fp}\special{pa 810 111}\special{pa 818 112}\special{fp}%
\special{pa 850 115}\special{pa 858 116}\special{fp}\special{pa 889 119}\special{pa 897 120}\special{fp}%
\special{pa 929 122}\special{pa 937 122}\special{fp}\special{pa 968 123}\special{pa 976 123}\special{fp}%
\special{pa 1008 124}\special{pa 1016 124}\special{fp}\special{pa 1048 123}\special{pa 1056 123}\special{fp}%
\special{pa 1087 122}\special{pa 1095 122}\special{fp}\special{pa 1127 120}\special{pa 1135 119}\special{fp}%
\special{pa 1167 116}\special{pa 1175 115}\special{fp}\special{pa 1206 112}\special{pa 1214 111}\special{fp}%
\special{pa 1245 107}\special{pa 1253 105}\special{fp}\special{pa 1285 100}\special{pa 1292 99}\special{fp}%
\special{pa 1324 93}\special{pa 1331 92}\special{fp}\special{pa 1362 85}\special{pa 1370 83}\special{fp}%
\special{pa 1401 76}\special{pa 1409 74}\special{fp}\special{pa 1439 66}\special{pa 1447 64}\special{fp}%
\special{pa 1477 55}\special{pa 1485 53}\special{fp}\special{pa 1515 43}\special{pa 1523 40}\special{fp}%
\special{pa 1553 30}\special{pa 1560 27}\special{fp}\special{pa 1590 16}\special{pa 1597 13}\special{fp}%
\special{pa 1627 1}\special{pa 1634 -1}\special{fp}\special{pn 8}%
}%
\settowidth{\Width}{$\pi$}\setlength{\Width}{-0.5\Width}%
\settoheight{\Height}{$\pi$}\settodepth{\Depth}{$\pi$}\setlength{\Height}{-0.5\Height}\setlength{\Depth}{0.5\Depth}\addtolength{\Height}{\Depth}%
\put(2.5700000,-0.3100000){\hspace*{\Width}\raisebox{\Height}{$\pi$}}%
%
\special{pa -2756    -0}\special{pa  2756    -0}%
\special{fp}%
\special{pa     0   591}\special{pa     0  -591}%
\special{fp}%
\settowidth{\Width}{$x$}\setlength{\Width}{0\Width}%
\settoheight{\Height}{$x$}\settodepth{\Depth}{$x$}\setlength{\Height}{-0.5\Height}\setlength{\Depth}{0.5\Depth}\addtolength{\Height}{\Depth}%
\put(7.0500000,0.0000000){\hspace*{\Width}\raisebox{\Height}{$x$}}%
%
\settowidth{\Width}{$y$}\setlength{\Width}{-0.5\Width}%
\settoheight{\Height}{$y$}\settodepth{\Depth}{$y$}\setlength{\Height}{\Depth}%
\put(0.0000000,1.5500000){\hspace*{\Width}\raisebox{\Height}{$y$}}%
%
\settowidth{\Width}{O}\setlength{\Width}{0\Width}%
\settoheight{\Height}{O}\settodepth{\Depth}{O}\setlength{\Height}{-\Height}%
\put(0.0500000,-0.0500000){\hspace*{\Width}\raisebox{\Height}{O}}%
%
\end{picture}}%}}
\end{layer}

%%%%%%%%%%%%

%%%%%%%%%%%%%%%%%%%%

\newslide{単振動の合成}

\vspace*{18mm}

\slidepage
\begin{itemize}
\item
{\normalsize\color{blue} $\sin\bunsuu{\pi}{4}=\bunsuu{1}{\sqrt{2}},\ \cos\bunsuu{\pi}{4}=\bunsuu{1}{\sqrt{2}}
,\ \sin\bunsuu{\pi}{3}=\bunsuu{\sqrt{3}}{2},\ \cos\bunsuu{\pi}{3}=\bunsuu{1}{2}$}
\item
$\sin(x+\bunsuu{\pi}{4})=$
$\sin x\cos\bunsuu{\pi}{4}+\cos x\sin\bunsuu{\pi}{4}$\\
$\phantom{\sin(x+\bunsuu{\pi}{4})}=\bunsuu{1}{\sqrt{2}}\sin x+\bunsuu{1}{\sqrt{2}}\cos x$\\
$\phantom{\sin(x+\bunsuu{\pi}{4})}=\bunsuu{1}{\sqrt{2}}(\sin x+\cos x)$\\
両辺に$\sqrt{2}$を掛けて左辺と右辺を入れかえると\\
\hspace*{2zw}$\sin x+\cos x=\sqrt{2}\sin(x+\bunsuu{\pi}{4})$\\
{\color{red}周期の等しい単振動を足すと1つの単振動になる}
\end{itemize}

\newslide{単振動の合成(課題)}

\vspace*{18mm}

\slidepage
\begin{itemize}
\item
[課題]\monban 次の単振動の和(差)を1つの単振動で表せ.\seteda{80}\\
\eda{$\sqrt{3}\sin x+\cos x$
\hspace{2zw}{\large $\sin(x+\bunsuu{\pi}{6})$を展開せよ}}\\
\eda{$\sin x+\sqrt{3}\cos x$
\hspace*{2zw}{\large $\sin(x+\bunsuu{\pi}{3})$を展開せよ}}\\
\eda{$\sin x-\cos x$
\hspace{3.5zw}{\large $\sin(x-\bunsuu{\pi}{4})$を展開せよ}}%
\item
一般に\\
\hspace*{1zw}{\color{blue}$a\sin x+b\cos x=\sqrt{a^2+b^2} \sin (x+ A)$\\
\hspace*{4zw}角$ A$:$\cos A=\bunsuu{a}{a^2+b^2},\ \sin A=\bunsuu{b}{a^2+b^2}$}
\end{itemize}
%%%%%%%%%%%%

%%%%%%%%%%%%%%%%%%%%


\newslide{三角関数のグラフ(課題)}

\vspace*{18mm}

%%repeat=1
\slidepage
\begin{itemize}
\item
[課題]\monban アプリで次の関数のグラフをかけ.また,特徴を1つあげよ.\seteda{57}\\
\eda{$y=\cos x+\sin x$}\eda{$y=\sin x+\sqrt{3}\cos x$}\\
\eda{$y=\sin x+\sin 2x$}\eda{$y=\cos^2 x=(\cos x)^2$}\\
\hfill {\color{blue}\large KeTMathでは\verb|cos(2,x)|と書けばよい}
\end{itemize}
\label{pageend}\mbox{}

\end{document}

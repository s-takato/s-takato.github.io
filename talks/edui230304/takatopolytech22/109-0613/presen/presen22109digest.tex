%%% Title presen22109
\documentclass[landscape,10pt]{ujarticle}
\special{papersize=\the\paperwidth,\the\paperheight}
\usepackage{ketpic,ketlayer}
\usepackage{ketslide}
\usepackage{amsmath,amssymb}
\usepackage{bm,enumerate}
\usepackage[dvipdfmx]{graphicx}
\usepackage{color}
\definecolor{slidecolora}{cmyk}{0.98,0.13,0,0.43}
\definecolor{slidecolorb}{cmyk}{0.2,0,0,0}
\definecolor{slidecolorc}{cmyk}{0.2,0,0,0}
\definecolor{slidecolord}{cmyk}{0.2,0,0,0}
\definecolor{slidecolore}{cmyk}{0,0,0,0.5}
\definecolor{slidecolorf}{cmyk}{0,0,0,0.5}
\definecolor{slidecolori}{cmyk}{0.98,0.13,0,0.43}
\def\setthin#1{\def\thin{#1}}
\setthin{0}
\newcommand{\slidepage}[1][s]{%
\setcounter{ketpicctra}{18}%
\if#1m \setcounter{ketpicctra}{1}\fi
\hypersetup{linkcolor=black}%

\begin{layer}{118}{0}
\putnotee{122}{-\theketpicctra.05}{\small\thepage/\pageref{pageend}}
\end{layer}\hypersetup{linkcolor=blue}

}
\usepackage{emath}
\usepackage{emathEy}
\usepackage{emathMw}
\usepackage{pict2e}
\usepackage{ketlayermorewith2e}
\usepackage[dvipdfmx,colorlinks=true,linkcolor=blue,filecolor=blue]{hyperref}
\newcommand{\hiduke}{0613}
\newcommand{\hako}[2][1]{\fbox{\raisebox{#1mm}{\mbox{}}\raisebox{-#1mm}{\mbox{}}\,\phantom{#2}\,}}
\newcommand{\hakoa}[2][1]{\fbox{\raisebox{#1mm}{\mbox{}}\raisebox{-#1mm}{\mbox{}}\,#2\,}}
\newcommand{\hakom}[2][1]{\hako[#1]{$#2$}}
\newcommand{\hakoma}[2][1]{\hakoa[#1]{$#2$}}
\def\rad{\;\mathrm{rad}}
\def\deg#1{#1^{\circ}}
\newcommand{\sbunsuu}[2]{\scalebox{0.6}{$\bunsuu{#1}{#2}$}}
\def\pow{$\hspace{-1.5mm}^\hspace{-1mm}$}
\def\dlim{\displaystyle\lim}
\newcommand{\brd}[2][1]{\scalebox{#1}{\color{red}\fbox{\color{black}$#2$}}}
\newcommand\down[1][0.5zw]{\vspace{#1}\\}
\newcommand{\sfrac}[3][0.65]{\scalebox{#1}{$\frac{#2}{#3}$}}
\newcommand{\phn}[1]{\phantom{#1}}
\newcommand{\scb}[2][0.6]{\scalebox{#1}{#2}}
\newcommand{\dsum}{\displaystyle\sum}

\setmargin{25}{145}{15}{100}

\ketslideinit

\pagestyle{empty}

\begin{document}

\begin{layer}{120}{0}
\putnotese{0}{0}{{\Large\bf
\color[cmyk]{1,1,0,0}

\begin{layer}{120}{0}
{\Huge \putnotes{60}{20}{三角関数の性質}}
\putnotes{60}{70}{2022.05.16}
\end{layer}

}
}
\end{layer}

\def\mainslidetitley{22}
\def\ketcletter{slidecolora}
\def\ketcbox{slidecolorb}
\def\ketdbox{slidecolorc}
\def\ketcframe{slidecolord}
\def\ketcshadow{slidecolore}
\def\ketdshadow{slidecolorf}
\def\slidetitlex{6}
\def\slidetitlesize{1.3}
\def\mketcletter{slidecolori}
\def\mketcbox{yellow}
\def\mketdbox{yellow}
\def\mketcframe{yellow}
\def\mslidetitlex{62}
\def\mslidetitlesize{2}

\color{black}
\Large\bf\boldmath
\addtocounter{page}{-1}

\def\MARU{}
\renewcommand{\MARU}[1]{{\ooalign{\hfil$#1$\/\hfil\crcr\raise.167ex\hbox{\mathhexbox20D}}}}
\renewcommand{\slidepage}[1][s]{%
\setcounter{ketpicctra}{18}%
\if#1m \setcounter{ketpicctra}{1}\fi
\hypersetup{linkcolor=black}%
\begin{layer}{118}{0}
\putnotee{115}{-\theketpicctra.05}{\small\hiduke-\thepage/\pageref{pageend}}
\end{layer}\hypersetup{linkcolor=blue}
}
\newcounter{ban}
\setcounter{ban}{1}
\newcommand{\monban}[1][\hiduke]{%
#1-\theban\ %
\addtocounter{ban}{1}%
}
\newcommand{\monbannoadd}[1][\hiduke]{%
#1-\theban\ %
}
\newcommand{\addban}{%
\addtocounter{ban}{1}%%210614
}
\newcounter{edawidth}
\newcounter{edactr}
\newcommand{\seteda}[1]{
\setcounter{edawidth}{#1}
\setcounter{edactr}{1}
}
\newcommand{\eda}[2][\theedawidth ]{%
\noindent\Ltab{#1 mm}{[\theedactr]\ #2}%
\addtocounter{edactr}{1}%
}
%%%%%%%%%%%%

%%%%%%%%%%%%%%%%%%%%

\mainslide{ 平均変化率}


\slidepage[m]
%%%%%%%%%%%%%

%%%%%%%%%%%%%%%%%%%%

\newslide{平均変化率の意味}

\vspace*{18mm}


\begin{layer}{120}{0}
\putnotew{96}{73}{\hyperlink{para0pg0}{\fbox{\Ctab{2.5mm}{\scalebox{1}{\scriptsize $\mathstrut||\!\lhd$}}}}}
\putnotew{101}{73}{\hyperlink{para1pg1}{\fbox{\Ctab{2.5mm}{\scalebox{1}{\scriptsize $\mathstrut|\!\lhd$}}}}}
\putnotew{108}{73}{\hyperlink{para1pg6}{\fbox{\Ctab{4.5mm}{\scalebox{1}{\scriptsize $\mathstrut\!\!\lhd\!\!$}}}}}
\putnotew{115}{73}{\hyperlink{para1pg7}{\fbox{\Ctab{4.5mm}{\scalebox{1}{\scriptsize $\mathstrut\!\rhd\!$}}}}}
\putnotew{120}{73}{\hyperlink{para1pg7}{\fbox{\Ctab{2.5mm}{\scalebox{1}{\scriptsize $\mathstrut \!\rhd\!\!|$}}}}}
\putnotew{125}{73}{\hyperlink{para2pg1}{\fbox{\Ctab{2.5mm}{\scalebox{1}{\scriptsize $\mathstrut \!\rhd\!\!||$}}}}}
\putnotee{126}{73}{\scriptsize\color{blue} 7/7}
\end{layer}

\slidepage

\begin{layer}{120}{0}
\putnotese{80}{10}{\scalebox{0.7}{%%% /polytech.git/n106/fig/henka1.tex 
%%% Generator=henkaritu.cdy 
{\unitlength=1cm%
\begin{picture}%
(6,6)(-1,-1)%
\special{pn 8}%
%
\special{pn 12}%
\special{pa  -195  -265}\special{pa  -136  -264}\special{pa   -76  -266}\special{pa   -17  -271}%
\special{pa    42  -279}\special{pa   101  -290}\special{pa   159  -305}\special{pa   218  -322}%
\special{pa   277  -343}\special{pa   335  -367}\special{pa   394  -394}\special{pa   482  -437}%
\special{pa   572  -483}\special{pa   662  -531}\special{pa   753  -582}\special{pa   844  -637}%
\special{pa   934  -696}\special{pa  1024  -760}\special{pa  1111  -829}\special{pa  1197  -903}%
\special{pa  1280  -984}\special{pa  1333 -1043}\special{pa  1386 -1107}\special{pa  1437 -1176}%
\special{pa  1488 -1251}\special{pa  1537 -1330}\special{pa  1585 -1415}\special{pa  1632 -1506}%
\special{pa  1677 -1601}\special{pa  1722 -1702}\special{pa  1765 -1808}%
\special{fp}%
\special{pn 8}%
\special{pa 0 -394}\special{pa 37 -394}\special{fp}\special{pa 75 -394}\special{pa 112 -394}\special{fp}%
\special{pa 150 -394}\special{pa 187 -394}\special{fp}\special{pa 225 -394}\special{pa 262 -394}\special{fp}%
\special{pa 300 -394}\special{pa 337 -394}\special{fp}\special{pa 375 -394}\special{pa 394 -394}\special{pa 394 -375}\special{fp}%
\special{pa 394 -337}\special{pa 394 -300}\special{fp}\special{pa 394 -262}\special{pa 394 -225}\special{fp}%
\special{pa 394 -187}\special{pa 394 -150}\special{fp}\special{pa 394 -112}\special{pa 394 -75}\special{fp}%
\special{pa 394 -37}\special{pa 394 0}\special{fp}%
%
\special{pa 0 -984}\special{pa 38 -984}\special{fp}\special{pa 77 -984}\special{pa 115 -984}\special{fp}%
\special{pa 153 -984}\special{pa 192 -984}\special{fp}\special{pa 230 -984}\special{pa 269 -984}\special{fp}%
\special{pa 307 -984}\special{pa 345 -984}\special{fp}\special{pa 384 -984}\special{pa 422 -984}\special{fp}%
\special{pa 460 -984}\special{pa 499 -984}\special{fp}\special{pa 537 -984}\special{pa 576 -984}\special{fp}%
\special{pa 614 -984}\special{pa 652 -984}\special{fp}\special{pa 691 -984}\special{pa 729 -984}\special{fp}%
\special{pa 767 -984}\special{pa 806 -984}\special{fp}\special{pa 844 -984}\special{pa 882 -984}\special{fp}%
\special{pa 921 -984}\special{pa 959 -984}\special{fp}\special{pa 998 -984}\special{pa 1036 -984}\special{fp}%
\special{pa 1074 -984}\special{pa 1113 -984}\special{fp}\special{pa 1151 -984}\special{pa 1189 -984}\special{fp}%
\special{pa 1228 -984}\special{pa 1266 -984}\special{fp}\special{pa 1280 -959}\special{pa 1280 -921}\special{fp}%
\special{pa 1280 -882}\special{pa 1280 -844}\special{fp}\special{pa 1280 -806}\special{pa 1280 -767}\special{fp}%
\special{pa 1280 -729}\special{pa 1280 -691}\special{fp}\special{pa 1280 -652}\special{pa 1280 -614}\special{fp}%
\special{pa 1280 -576}\special{pa 1280 -537}\special{fp}\special{pa 1280 -499}\special{pa 1280 -460}\special{fp}%
\special{pa 1280 -422}\special{pa 1280 -384}\special{fp}\special{pa 1280 -345}\special{pa 1280 -307}\special{fp}%
\special{pa 1280 -269}\special{pa 1280 -230}\special{fp}\special{pa 1280 -192}\special{pa 1280 -153}\special{fp}%
\special{pa 1280 -115}\special{pa 1280 -77}\special{fp}\special{pa 1280 -38}\special{pa 1280 0}\special{fp}%
%
%
\special{pa   394   -20}\special{pa   394    20}%
\special{fp}%
\settowidth{\Width}{$a$}\setlength{\Width}{-0.5\Width}%
\settoheight{\Height}{$a$}\settodepth{\Depth}{$a$}\setlength{\Height}{-\Height}%
\put(1.0000000,-0.1000000){\hspace*{\Width}\raisebox{\Height}{$a$}}%
%
\special{pa  1280   -20}\special{pa  1280    20}%
\special{fp}%
\settowidth{\Width}{$b$}\setlength{\Width}{-0.5\Width}%
\settoheight{\Height}{$b$}\settodepth{\Depth}{$b$}\setlength{\Height}{-\Height}%
\put(3.2500000,-0.1000000){\hspace*{\Width}\raisebox{\Height}{$b$}}%
%
\special{pa    20  -394}\special{pa   -20  -394}%
\special{fp}%
\settowidth{\Width}{$f(a)$}\setlength{\Width}{-1\Width}%
\settoheight{\Height}{$f(a)$}\settodepth{\Depth}{$f(a)$}\setlength{\Height}{-0.5\Height}\setlength{\Depth}{0.5\Depth}\addtolength{\Height}{\Depth}%
\put(-0.1000000,1.0000000){\hspace*{\Width}\raisebox{\Height}{$f(a)$}}%
%
\special{pa    20  -984}\special{pa   -20  -984}%
\special{fp}%
\settowidth{\Width}{$f(b)$}\setlength{\Width}{-1\Width}%
\settoheight{\Height}{$f(b)$}\settodepth{\Depth}{$f(b)$}\setlength{\Height}{-0.5\Height}\setlength{\Depth}{0.5\Depth}\addtolength{\Height}{\Depth}%
\put(-0.1000000,2.5000000){\hspace*{\Width}\raisebox{\Height}{$f(b)$}}%
%
\settowidth{\Width}{$y=f(x)$}\setlength{\Width}{-0.5\Width}%
\settoheight{\Height}{$y=f(x)$}\settodepth{\Depth}{$y=f(x)$}\setlength{\Height}{\Depth}%
\put(4.4800000,4.6400000){\hspace*{\Width}\raisebox{\Height}{$y=f(x)$}}%
%
\special{pa  -394    -0}\special{pa  1969    -0}%
\special{fp}%
\special{pa     0   394}\special{pa     0 -1969}%
\special{fp}%
\settowidth{\Width}{$x$}\setlength{\Width}{0\Width}%
\settoheight{\Height}{$x$}\settodepth{\Depth}{$x$}\setlength{\Height}{-0.5\Height}\setlength{\Depth}{0.5\Depth}\addtolength{\Height}{\Depth}%
\put(5.0500000,0.0000000){\hspace*{\Width}\raisebox{\Height}{$x$}}%
%
\settowidth{\Width}{$y$}\setlength{\Width}{-0.5\Width}%
\settoheight{\Height}{$y$}\settodepth{\Depth}{$y$}\setlength{\Height}{\Depth}%
\put(0.0000000,5.0500000){\hspace*{\Width}\raisebox{\Height}{$y$}}%
%
\settowidth{\Width}{O}\setlength{\Width}{-1\Width}%
\settoheight{\Height}{O}\settodepth{\Depth}{O}\setlength{\Height}{-\Height}%
\put(-0.0500000,-0.0500000){\hspace*{\Width}\raisebox{\Height}{O}}%
%
\end{picture}}%}}
\putnotese{80}{10}{\scalebox{0.7}{%%% /polytech.git/n106/fig/henka2.tex 
%%% Generator=henkaritu.cdy 
{\unitlength=1cm%
\begin{picture}%
(6,6)(-1,-1)%
\special{pn 8}%
%
\special{pn 12}%
\special{pa  -195  -265}\special{pa  -136  -264}\special{pa   -76  -266}\special{pa   -17  -271}%
\special{pa    42  -279}\special{pa   101  -290}\special{pa   159  -305}\special{pa   218  -322}%
\special{pa   277  -343}\special{pa   335  -367}\special{pa   394  -394}\special{pa   482  -437}%
\special{pa   572  -483}\special{pa   662  -531}\special{pa   753  -582}\special{pa   844  -637}%
\special{pa   934  -696}\special{pa  1024  -760}\special{pa  1111  -829}\special{pa  1197  -903}%
\special{pa  1280  -984}\special{pa  1333 -1043}\special{pa  1386 -1107}\special{pa  1437 -1176}%
\special{pa  1488 -1251}\special{pa  1537 -1330}\special{pa  1585 -1415}\special{pa  1632 -1506}%
\special{pa  1677 -1601}\special{pa  1722 -1702}\special{pa  1765 -1808}%
\special{fp}%
\special{pn 8}%
\special{pa 0 -394}\special{pa 37 -394}\special{fp}\special{pa 75 -394}\special{pa 112 -394}\special{fp}%
\special{pa 150 -394}\special{pa 187 -394}\special{fp}\special{pa 225 -394}\special{pa 262 -394}\special{fp}%
\special{pa 300 -394}\special{pa 337 -394}\special{fp}\special{pa 375 -394}\special{pa 394 -394}\special{pa 394 -375}\special{fp}%
\special{pa 394 -337}\special{pa 394 -300}\special{fp}\special{pa 394 -262}\special{pa 394 -225}\special{fp}%
\special{pa 394 -187}\special{pa 394 -150}\special{fp}\special{pa 394 -112}\special{pa 394 -75}\special{fp}%
\special{pa 394 -37}\special{pa 394 0}\special{fp}%
%
\special{pa 0 -984}\special{pa 38 -984}\special{fp}\special{pa 77 -984}\special{pa 115 -984}\special{fp}%
\special{pa 153 -984}\special{pa 192 -984}\special{fp}\special{pa 230 -984}\special{pa 269 -984}\special{fp}%
\special{pa 307 -984}\special{pa 345 -984}\special{fp}\special{pa 384 -984}\special{pa 422 -984}\special{fp}%
\special{pa 460 -984}\special{pa 499 -984}\special{fp}\special{pa 537 -984}\special{pa 576 -984}\special{fp}%
\special{pa 614 -984}\special{pa 652 -984}\special{fp}\special{pa 691 -984}\special{pa 729 -984}\special{fp}%
\special{pa 767 -984}\special{pa 806 -984}\special{fp}\special{pa 844 -984}\special{pa 882 -984}\special{fp}%
\special{pa 921 -984}\special{pa 959 -984}\special{fp}\special{pa 998 -984}\special{pa 1036 -984}\special{fp}%
\special{pa 1074 -984}\special{pa 1113 -984}\special{fp}\special{pa 1151 -984}\special{pa 1189 -984}\special{fp}%
\special{pa 1228 -984}\special{pa 1266 -984}\special{fp}\special{pa 1280 -959}\special{pa 1280 -921}\special{fp}%
\special{pa 1280 -882}\special{pa 1280 -844}\special{fp}\special{pa 1280 -806}\special{pa 1280 -767}\special{fp}%
\special{pa 1280 -729}\special{pa 1280 -691}\special{fp}\special{pa 1280 -652}\special{pa 1280 -614}\special{fp}%
\special{pa 1280 -576}\special{pa 1280 -537}\special{fp}\special{pa 1280 -499}\special{pa 1280 -460}\special{fp}%
\special{pa 1280 -422}\special{pa 1280 -384}\special{fp}\special{pa 1280 -345}\special{pa 1280 -307}\special{fp}%
\special{pa 1280 -269}\special{pa 1280 -230}\special{fp}\special{pa 1280 -192}\special{pa 1280 -153}\special{fp}%
\special{pa 1280 -115}\special{pa 1280 -77}\special{fp}\special{pa 1280 -38}\special{pa 1280 0}\special{fp}%
%
%
\special{pa   394   -20}\special{pa   394    20}%
\special{fp}%
\settowidth{\Width}{$a$}\setlength{\Width}{-0.5\Width}%
\settoheight{\Height}{$a$}\settodepth{\Depth}{$a$}\setlength{\Height}{-\Height}%
\put(1.0000000,-0.1000000){\hspace*{\Width}\raisebox{\Height}{$a$}}%
%
\special{pa  1280   -20}\special{pa  1280    20}%
\special{fp}%
\settowidth{\Width}{$b$}\setlength{\Width}{-0.5\Width}%
\settoheight{\Height}{$b$}\settodepth{\Depth}{$b$}\setlength{\Height}{-\Height}%
\put(3.2500000,-0.1000000){\hspace*{\Width}\raisebox{\Height}{$b$}}%
%
\special{pa    20  -394}\special{pa   -20  -394}%
\special{fp}%
\settowidth{\Width}{$f(a)$}\setlength{\Width}{-1\Width}%
\settoheight{\Height}{$f(a)$}\settodepth{\Depth}{$f(a)$}\setlength{\Height}{-0.5\Height}\setlength{\Depth}{0.5\Depth}\addtolength{\Height}{\Depth}%
\put(-0.1000000,1.0000000){\hspace*{\Width}\raisebox{\Height}{$f(a)$}}%
%
\special{pa    20  -984}\special{pa   -20  -984}%
\special{fp}%
\settowidth{\Width}{$f(b)$}\setlength{\Width}{-1\Width}%
\settoheight{\Height}{$f(b)$}\settodepth{\Depth}{$f(b)$}\setlength{\Height}{-0.5\Height}\setlength{\Depth}{0.5\Depth}\addtolength{\Height}{\Depth}%
\put(-0.1000000,2.5000000){\hspace*{\Width}\raisebox{\Height}{$f(b)$}}%
%
\settowidth{\Width}{$y=f(x)$}\setlength{\Width}{-0.5\Width}%
\settoheight{\Height}{$y=f(x)$}\settodepth{\Depth}{$y=f(x)$}\setlength{\Height}{\Depth}%
\put(4.4800000,4.6400000){\hspace*{\Width}\raisebox{\Height}{$y=f(x)$}}%
%
{%
\color[cmyk]{0,1,1,0}%
\special{pa   394  -394}\special{pa  1280  -984}%
\special{fp}%
}%
\special{pa 394 -394}\special{pa 432 -394}\special{fp}\special{pa 471 -394}\special{pa 509 -394}\special{fp}%
\special{pa 548 -394}\special{pa 586 -394}\special{fp}\special{pa 625 -394}\special{pa 663 -394}\special{fp}%
\special{pa 702 -394}\special{pa 740 -394}\special{fp}\special{pa 779 -394}\special{pa 817 -394}\special{fp}%
\special{pa 856 -394}\special{pa 894 -394}\special{fp}\special{pa 933 -394}\special{pa 971 -394}\special{fp}%
\special{pa 1010 -394}\special{pa 1048 -394}\special{fp}\special{pa 1087 -394}\special{pa 1125 -394}\special{fp}%
\special{pa 1164 -394}\special{pa 1202 -394}\special{fp}\special{pa 1241 -394}\special{pa 1280 -394}\special{fp}%
%
%
\special{pa   394  -394}\special{pa   411  -387}\special{pa   427  -380}\special{pa   445  -374}%
\special{pa   462  -368}\special{pa   479  -362}\special{pa   496  -357}\special{pa   514  -351}%
\special{pa   531  -346}\special{pa   549  -342}\special{pa   566  -337}\special{pa   584  -333}%
\special{pa   602  -329}\special{pa   620  -326}\special{pa   638  -322}\special{pa   656  -319}%
\special{pa   673  -317}\special{pa   692  -314}\special{pa   710  -312}\special{pa   728  -310}%
\special{pa   746  -309}\special{pa   764  -307}\special{pa   782  -306}\special{pa   800  -306}%
\special{pa   818  -305}\special{pa   837  -305}\special{pa   855  -305}\special{pa   873  -306}%
\special{pa   891  -306}\special{pa   909  -307}\special{pa   927  -309}\special{pa   946  -310}%
\special{pa   964  -312}\special{pa   982  -314}\special{pa  1000  -317}\special{pa  1018  -319}%
\special{pa  1036  -322}\special{pa  1054  -326}\special{pa  1071  -329}\special{pa  1089  -333}%
\special{pa  1107  -337}\special{pa  1124  -342}\special{pa  1142  -346}\special{pa  1160  -351}%
\special{pa  1177  -357}\special{pa  1194  -362}\special{pa  1212  -368}\special{pa  1229  -374}%
\special{pa  1246  -380}\special{pa  1263  -387}\special{pa  1280  -394}%
\special{fp}%
\settowidth{\Width}{$b-a$}\setlength{\Width}{-0.5\Width}%
\settoheight{\Height}{$b-a$}\settodepth{\Depth}{$b-a$}\setlength{\Height}{-0.5\Height}\setlength{\Depth}{0.5\Depth}\addtolength{\Height}{\Depth}%
\put(2.1200000,0.6700000){\hspace*{\Width}\raisebox{\Height}{$b-a$}}%
%
\special{pa  1280  -394}\special{pa  1284  -405}\special{pa  1288  -416}\special{pa  1293  -428}%
\special{pa  1297  -439}\special{pa  1301  -451}\special{pa  1304  -462}\special{pa  1308  -474}%
\special{pa  1311  -485}\special{pa  1314  -497}\special{pa  1317  -509}\special{pa  1320  -521}%
\special{pa  1322  -532}\special{pa  1325  -544}\special{pa  1327  -556}\special{pa  1329  -568}%
\special{pa  1331  -580}\special{pa  1332  -592}\special{pa  1334  -604}\special{pa  1335  -616}%
\special{pa  1336  -628}\special{pa  1337  -641}\special{pa  1338  -653}\special{pa  1338  -665}%
\special{pa  1338  -677}\special{pa  1339  -689}\special{pa  1338  -701}\special{pa  1338  -713}%
\special{pa  1338  -725}\special{pa  1337  -737}\special{pa  1336  -750}\special{pa  1335  -762}%
\special{pa  1334  -774}\special{pa  1332  -786}\special{pa  1331  -798}\special{pa  1329  -810}%
\special{pa  1327  -822}\special{pa  1325  -834}\special{pa  1322  -845}\special{pa  1320  -857}%
\special{pa  1317  -869}\special{pa  1314  -881}\special{pa  1311  -893}\special{pa  1308  -904}%
\special{pa  1304  -916}\special{pa  1301  -927}\special{pa  1297  -939}\special{pa  1293  -950}%
\special{pa  1288  -962}\special{pa  1284  -973}\special{pa  1280  -984}%
\special{fp}%
\settowidth{\Width}{$f(b)-f(a)$}\setlength{\Width}{0\Width}%
\settoheight{\Height}{$f(b)-f(a)$}\settodepth{\Depth}{$f(b)-f(a)$}\setlength{\Height}{-0.5\Height}\setlength{\Depth}{0.5\Depth}\addtolength{\Height}{\Depth}%
\put(3.5000000,1.7500000){\hspace*{\Width}\raisebox{\Height}{$f(b)-f(a)$}}%
%
\settowidth{\Width}{A}\setlength{\Width}{-1\Width}%
\settoheight{\Height}{A}\settodepth{\Depth}{A}\setlength{\Height}{\Depth}%
\put(0.9500000,1.0500000){\hspace*{\Width}\raisebox{\Height}{A}}%
%
\settowidth{\Width}{B}\setlength{\Width}{-1\Width}%
\settoheight{\Height}{B}\settodepth{\Depth}{B}\setlength{\Height}{\Depth}%
\put(3.2000000,2.5500000){\hspace*{\Width}\raisebox{\Height}{B}}%
%
\special{pa  -394    -0}\special{pa  1969    -0}%
\special{fp}%
\special{pa     0   394}\special{pa     0 -1969}%
\special{fp}%
\settowidth{\Width}{$x$}\setlength{\Width}{0\Width}%
\settoheight{\Height}{$x$}\settodepth{\Depth}{$x$}\setlength{\Height}{-0.5\Height}\setlength{\Depth}{0.5\Depth}\addtolength{\Height}{\Depth}%
\put(5.0500000,0.0000000){\hspace*{\Width}\raisebox{\Height}{$x$}}%
%
\settowidth{\Width}{$y$}\setlength{\Width}{-0.5\Width}%
\settoheight{\Height}{$y$}\settodepth{\Depth}{$y$}\setlength{\Height}{\Depth}%
\put(0.0000000,5.0500000){\hspace*{\Width}\raisebox{\Height}{$y$}}%
%
\settowidth{\Width}{O}\setlength{\Width}{-1\Width}%
\settoheight{\Height}{O}\settodepth{\Depth}{O}\setlength{\Height}{-\Height}%
\put(-0.0500000,-0.0500000){\hspace*{\Width}\raisebox{\Height}{O}}%
%
\end{picture}}%}}
\end{layer}

\begin{itemize}
\item
関数$y=f(x)$,\ 区間$[a,\ b]$
\item
$f(x)$の$[a,\ b]$での変化量は\\
\hspace*{2zw}$f(b)-f(a)$
\item
[]区間幅$b-a$で割る\\
\hspace*{2zw}$\bunsuu{f(b)-f(a)}{b-a}$
\item
[]これを{\color{red}平均変化率}という
\item
平均変化率は直線ABの傾き
\end{itemize}

\newslide{平均変化率の計算例}

\vspace*{18mm}


\begin{layer}{120}{0}
\putnotew{96}{73}{\hyperlink{para1pg7}{\fbox{\Ctab{2.5mm}{\scalebox{1}{\scriptsize $\mathstrut||\!\lhd$}}}}}
\putnotew{101}{73}{\hyperlink{para2pg1}{\fbox{\Ctab{2.5mm}{\scalebox{1}{\scriptsize $\mathstrut|\!\lhd$}}}}}
\putnotew{108}{73}{\hyperlink{para2pg9}{\fbox{\Ctab{4.5mm}{\scalebox{1}{\scriptsize $\mathstrut\!\!\lhd\!\!$}}}}}
\putnotew{115}{73}{\hyperlink{para2pg10}{\fbox{\Ctab{4.5mm}{\scalebox{1}{\scriptsize $\mathstrut\!\rhd\!$}}}}}
\putnotew{120}{73}{\hyperlink{para2pg10}{\fbox{\Ctab{2.5mm}{\scalebox{1}{\scriptsize $\mathstrut \!\rhd\!\!|$}}}}}
\putnotew{125}{73}{\hyperlink{para3pg1}{\fbox{\Ctab{2.5mm}{\scalebox{1}{\scriptsize $\mathstrut \!\rhd\!\!||$}}}}}
\putnotee{126}{73}{\scriptsize\color{blue} 10/10}
\end{layer}

\slidepage
\begin{itemize}
\item
$f(x)=x^2$の$[1,3]$での平均変化率($r$とおく)\\
\hspace*{1zw}$r=\bunsuu{f(3)-f(1)}{3-1}=$
$\bunsuu{3^2-1^2}{3-1}=\bunsuu{9-1}{3-1}=4$
\item
$f(x)=x^2$の$[a,b]$での平均変化率\\
\hspace*{1zw}$r=\bunsuu{b^2-a^2}{b-a}=$
$\bunsuu{(b-a)(b+a)}{b-a}=b+a$
\item
[課題]\monbannoadd 次を求めよ.\seteda{100}\\
\eda{$f(x)=4x^2$の$[2,4]$での平均変化率}\\
\eda{$f(x)=3x$の$[a,b]$での平均変化率}
\end{itemize}
\addban

\newslide{$b$を$a$に近づけたときの変化率}

\vspace*{18mm}


\begin{layer}{120}{0}
\putnotew{96}{73}{\hyperlink{para2pg10}{\fbox{\Ctab{2.5mm}{\scalebox{1}{\scriptsize $\mathstrut||\!\lhd$}}}}}
\putnotew{101}{73}{\hyperlink{para3pg1}{\fbox{\Ctab{2.5mm}{\scalebox{1}{\scriptsize $\mathstrut|\!\lhd$}}}}}
\putnotew{108}{73}{\hyperlink{para3pg7}{\fbox{\Ctab{4.5mm}{\scalebox{1}{\scriptsize $\mathstrut\!\!\lhd\!\!$}}}}}
\putnotew{115}{73}{\hyperlink{para3pg8}{\fbox{\Ctab{4.5mm}{\scalebox{1}{\scriptsize $\mathstrut\!\rhd\!$}}}}}
\putnotew{120}{73}{\hyperlink{para3pg8}{\fbox{\Ctab{2.5mm}{\scalebox{1}{\scriptsize $\mathstrut \!\rhd\!\!|$}}}}}
\putnotew{125}{73}{\hyperlink{para4pg1}{\fbox{\Ctab{2.5mm}{\scalebox{1}{\scriptsize $\mathstrut \!\rhd\!\!||$}}}}}
\putnotee{126}{73}{\scriptsize\color{blue} 8/8}
\end{layer}

\slidepage

\begin{layer}{120}{0}
\putnotese{75}{15}{\scalebox{0.8}{%%% /polytech.git/n106/fig/henka3.tex 
%%% Generator=henkaritu.cdy 
{\unitlength=1cm%
\begin{picture}%
(6,6)(-1,-1)%
\special{pn 8}%
%
\special{pn 12}%
\special{pa  -394  -394}\special{pa  -346  -305}\special{pa  -299  -227}\special{pa  -252  -161}%
\special{pa  -205  -106}\special{pa  -157   -63}\special{pa  -110   -31}\special{pa   -63   -10}%
\special{pa   -16    -1}\special{pa    31    -3}\special{pa    79   -16}\special{pa   126   -40}%
\special{pa   173   -76}\special{pa   220  -123}\special{pa   268  -182}\special{pa   315  -252}%
\special{pa   362  -333}\special{pa   409  -426}\special{pa   457  -530}\special{pa   504  -645}%
\special{pa   551  -772}\special{pa   598  -910}\special{pa   646 -1059}\special{pa   693 -1220}%
\special{pa   740 -1391}\special{pa   787 -1575}\special{pa   835 -1769}\special{pa   880 -1969}%
\special{fp}%
\special{pn 8}%
{%
\color[cmyk]{0,1,1,0}%
\special{pa   394  -394}\special{pa   787 -1575}%
\special{fp}%
}%
\special{pn 8}%
\special{pa 394 -398}\special{pa 394 -390}\special{fp}\special{pa 394 -358}\special{pa 394 -350}\special{fp}%
\special{pa 394 -319}\special{pa 394 -311}\special{fp}\special{pa 394 -280}\special{pa 394 -272}\special{fp}%
\special{pa 394 -240}\special{pa 394 -232}\special{fp}\special{pa 394 -201}\special{pa 394 -193}\special{fp}%
\special{pa 394 -161}\special{pa 394 -153}\special{fp}\special{pa 394 -122}\special{pa 394 -114}\special{fp}%
\special{pa 394 -83}\special{pa 394 -75}\special{fp}\special{pa 394 -43}\special{pa 394 -35}\special{fp}%
\special{pa 394 -4}\special{pa 394 4}\special{fp}\special{pn 8}%
\special{pn 8}%
\special{pa 787 -1579}\special{pa 787 -1571}\special{fp}\special{pa 787 -1539}\special{pa 787 -1531}\special{fp}%
\special{pa 787 -1500}\special{pa 787 -1492}\special{fp}\special{pa 787 -1461}\special{pa 787 -1453}\special{fp}%
\special{pa 787 -1421}\special{pa 787 -1413}\special{fp}\special{pa 787 -1382}\special{pa 787 -1374}\special{fp}%
\special{pa 787 -1343}\special{pa 787 -1335}\special{fp}\special{pa 787 -1303}\special{pa 787 -1295}\special{fp}%
\special{pa 787 -1264}\special{pa 787 -1256}\special{fp}\special{pa 787 -1224}\special{pa 787 -1216}\special{fp}%
\special{pa 787 -1185}\special{pa 787 -1177}\special{fp}\special{pa 787 -1146}\special{pa 787 -1138}\special{fp}%
\special{pa 787 -1106}\special{pa 787 -1098}\special{fp}\special{pa 787 -1067}\special{pa 787 -1059}\special{fp}%
\special{pa 787 -1028}\special{pa 787 -1020}\special{fp}\special{pa 787 -988}\special{pa 787 -980}\special{fp}%
\special{pa 787 -949}\special{pa 787 -941}\special{fp}\special{pa 787 -910}\special{pa 787 -902}\special{fp}%
\special{pa 787 -870}\special{pa 787 -862}\special{fp}\special{pa 787 -831}\special{pa 787 -823}\special{fp}%
\special{pa 787 -791}\special{pa 787 -783}\special{fp}\special{pa 787 -752}\special{pa 787 -744}\special{fp}%
\special{pa 787 -713}\special{pa 787 -705}\special{fp}\special{pa 787 -673}\special{pa 787 -665}\special{fp}%
\special{pa 787 -634}\special{pa 787 -626}\special{fp}\special{pa 787 -595}\special{pa 787 -587}\special{fp}%
\special{pa 787 -555}\special{pa 787 -547}\special{fp}\special{pa 787 -516}\special{pa 787 -508}\special{fp}%
\special{pa 787 -476}\special{pa 787 -468}\special{fp}\special{pa 787 -437}\special{pa 787 -429}\special{fp}%
\special{pa 787 -398}\special{pa 787 -390}\special{fp}\special{pa 787 -358}\special{pa 787 -350}\special{fp}%
\special{pa 787 -319}\special{pa 787 -311}\special{fp}\special{pa 787 -280}\special{pa 787 -272}\special{fp}%
\special{pa 787 -240}\special{pa 787 -232}\special{fp}\special{pa 787 -201}\special{pa 787 -193}\special{fp}%
\special{pa 787 -161}\special{pa 787 -153}\special{fp}\special{pa 787 -122}\special{pa 787 -114}\special{fp}%
\special{pa 787 -83}\special{pa 787 -75}\special{fp}\special{pa 787 -43}\special{pa 787 -35}\special{fp}%
\special{pa 787 -4}\special{pa 787 4}\special{fp}\special{pn 8}%
\special{pn 8}%
\special{pa 390 -394}\special{pa 398 -394}\special{fp}\special{pa 429 -394}\special{pa 437 -394}\special{fp}%
\special{pa 468 -394}\special{pa 476 -394}\special{fp}\special{pa 508 -394}\special{pa 516 -394}\special{fp}%
\special{pa 547 -394}\special{pa 555 -394}\special{fp}\special{pa 587 -394}\special{pa 595 -394}\special{fp}%
\special{pa 626 -394}\special{pa 634 -394}\special{fp}\special{pa 665 -394}\special{pa 673 -394}\special{fp}%
\special{pa 705 -394}\special{pa 713 -394}\special{fp}\special{pa 744 -394}\special{pa 752 -394}\special{fp}%
\special{pa 783 -394}\special{pa 791 -394}\special{fp}\special{pn 8}%
\special{pa   394   -20}\special{pa   394    20}%
\special{fp}%
\settowidth{\Width}{$1$}\setlength{\Width}{-0.5\Width}%
\settoheight{\Height}{$1$}\settodepth{\Depth}{$1$}\setlength{\Height}{-\Height}%
\put(1.0000000,-0.1000000){\hspace*{\Width}\raisebox{\Height}{$1$}}%
%
\special{pa   787   -20}\special{pa   787    20}%
\special{fp}%
\settowidth{\Width}{$2$}\setlength{\Width}{-0.5\Width}%
\settoheight{\Height}{$2$}\settodepth{\Depth}{$2$}\setlength{\Height}{-\Height}%
\put(2.0000000,-0.1000000){\hspace*{\Width}\raisebox{\Height}{$2$}}%
%
%
\settowidth{\Width}{$y=x^2$}\setlength{\Width}{0\Width}%
\settoheight{\Height}{$y=x^2$}\settodepth{\Depth}{$y=x^2$}\setlength{\Height}{-0.5\Height}\setlength{\Depth}{0.5\Depth}\addtolength{\Height}{\Depth}%
\put(2.2900000,5.0000000){\hspace*{\Width}\raisebox{\Height}{$y=x^2$}}%
%
\special{pa  -394    -0}\special{pa  1969    -0}%
\special{fp}%
\special{pa     0   394}\special{pa     0 -1969}%
\special{fp}%
\settowidth{\Width}{$x$}\setlength{\Width}{0\Width}%
\settoheight{\Height}{$x$}\settodepth{\Depth}{$x$}\setlength{\Height}{-0.5\Height}\setlength{\Depth}{0.5\Depth}\addtolength{\Height}{\Depth}%
\put(5.0500000,0.0000000){\hspace*{\Width}\raisebox{\Height}{$x$}}%
%
\settowidth{\Width}{$y$}\setlength{\Width}{-0.5\Width}%
\settoheight{\Height}{$y$}\settodepth{\Depth}{$y$}\setlength{\Height}{\Depth}%
\put(0.0000000,5.0500000){\hspace*{\Width}\raisebox{\Height}{$y$}}%
%
\settowidth{\Width}{O}\setlength{\Width}{-1\Width}%
\settoheight{\Height}{O}\settodepth{\Depth}{O}\setlength{\Height}{-\Height}%
\put(-0.0500000,-0.0500000){\hspace*{\Width}\raisebox{\Height}{O}}%
%
\end{picture}}%}}
\end{layer}

\begin{itemize}
\item
関数$y=x^2$の$[a,b]$での平均変化率\\
\hspace*{1zw}$r=\bunsuu{b^2-a^2}{b-a}$
\item
[]$[1,b]$のとき\\
\hspace*{1zw}$r=\bunsuu{b^2-1}{b-1}$
\item
$b=2$のとき $r=\hakoma{3}$
\item
\href{https://s-takato.github.io/polytech21/netmaterials/offline/henka3mainoff.html}%
{「14.1点における変化率」}
\end{itemize}

\newslide{割り算(分数)の意味}

\vspace*{18mm}


\begin{layer}{120}{0}
\putnotew{96}{73}{\hyperlink{para3pg8}{\fbox{\Ctab{2.5mm}{\scalebox{1}{\scriptsize $\mathstrut||\!\lhd$}}}}}
\putnotew{101}{73}{\hyperlink{para4pg1}{\fbox{\Ctab{2.5mm}{\scalebox{1}{\scriptsize $\mathstrut|\!\lhd$}}}}}
\putnotew{108}{73}{\hyperlink{para4pg5}{\fbox{\Ctab{4.5mm}{\scalebox{1}{\scriptsize $\mathstrut\!\!\lhd\!\!$}}}}}
\putnotew{115}{73}{\hyperlink{para4pg6}{\fbox{\Ctab{4.5mm}{\scalebox{1}{\scriptsize $\mathstrut\!\rhd\!$}}}}}
\putnotew{120}{73}{\hyperlink{para4pg6}{\fbox{\Ctab{2.5mm}{\scalebox{1}{\scriptsize $\mathstrut \!\rhd\!\!|$}}}}}
\putnotew{125}{73}{\hyperlink{para5pg1}{\fbox{\Ctab{2.5mm}{\scalebox{1}{\scriptsize $\mathstrut \!\rhd\!\!||$}}}}}
\putnotee{126}{73}{\scriptsize\color{blue} 6/6}
\end{layer}

\slidepage
\begin{itemize}
\item
$a \div b\ \ \Bigl(\bunsuu{a}{b}\Bigr)$とは
\item
[]例)$x=\bunsuu{6}{2}$
$\Longleftrightarrow\ 2x=6$となる$x$のこと
\item
[]例)$x=\bunsuu{3}{5}$
$\Longleftrightarrow\ 5x=3$となる$x$のこと
\item
{\color{red}$x=\bunsuu{a}{b}\ \Longleftrightarrow\ bx=a$となる$x$のこと}
\end{itemize}

\newslide{分母が$0$になると?}

\vspace*{18mm}


\begin{layer}{120}{0}
\putnotew{96}{73}{\hyperlink{para4pg6}{\fbox{\Ctab{2.5mm}{\scalebox{1}{\scriptsize $\mathstrut||\!\lhd$}}}}}
\putnotew{101}{73}{\hyperlink{para5pg1}{\fbox{\Ctab{2.5mm}{\scalebox{1}{\scriptsize $\mathstrut|\!\lhd$}}}}}
\putnotew{108}{73}{\hyperlink{para5pg7}{\fbox{\Ctab{4.5mm}{\scalebox{1}{\scriptsize $\mathstrut\!\!\lhd\!\!$}}}}}
\putnotew{115}{73}{\hyperlink{para5pg8}{\fbox{\Ctab{4.5mm}{\scalebox{1}{\scriptsize $\mathstrut\!\rhd\!$}}}}}
\putnotew{120}{73}{\hyperlink{para5pg8}{\fbox{\Ctab{2.5mm}{\scalebox{1}{\scriptsize $\mathstrut \!\rhd\!\!|$}}}}}
\putnotew{125}{73}{\hyperlink{para6pg1}{\fbox{\Ctab{2.5mm}{\scalebox{1}{\scriptsize $\mathstrut \!\rhd\!\!||$}}}}}
\putnotee{126}{73}{\scriptsize\color{blue} 8/8}
\end{layer}

\slidepage
\begin{itemize}
\item
[(1)]$\bunsuu{1}{0}$は\;\hakoa{求まらない}
\item
[(2)]$\bunsuu{0}{0}$は\;\hakoa{決まらない}
{\normalsize\color{blue}
\item
[]$x=\bunsuu{1}{0}\ \Longleftrightarrow\  \hakoma{0}\;x= \hakoma{1}$
\item
[]$x=\bunsuu{0}{0}\ \Longleftrightarrow\  \hakoma{0}\;x= \hakoma{0}$
}
\item
{\color{red}分母が0となる分数は考えない}
\end{itemize}

\newslide{1点における変化率}

\vspace*{18mm}


\begin{layer}{120}{0}
\putnotew{96}{73}{\hyperlink{para5pg8}{\fbox{\Ctab{2.5mm}{\scalebox{1}{\scriptsize $\mathstrut||\!\lhd$}}}}}
\putnotew{101}{73}{\hyperlink{para6pg1}{\fbox{\Ctab{2.5mm}{\scalebox{1}{\scriptsize $\mathstrut|\!\lhd$}}}}}
\putnotew{108}{73}{\hyperlink{para6pg2}{\fbox{\Ctab{4.5mm}{\scalebox{1}{\scriptsize $\mathstrut\!\!\lhd\!\!$}}}}}
\putnotew{115}{73}{\hyperlink{para6pg3}{\fbox{\Ctab{4.5mm}{\scalebox{1}{\scriptsize $\mathstrut\!\rhd\!$}}}}}
\putnotew{120}{73}{\hyperlink{para6pg3}{\fbox{\Ctab{2.5mm}{\scalebox{1}{\scriptsize $\mathstrut \!\rhd\!\!|$}}}}}
\putnotew{125}{73}{\hyperlink{para7pg1}{\fbox{\Ctab{2.5mm}{\scalebox{1}{\scriptsize $\mathstrut \!\rhd\!\!||$}}}}}
\putnotee{126}{73}{\scriptsize\color{blue} 3/3}
\end{layer}

\slidepage
\begin{itemize}
\item
区間$[a,b]$の平均変化率$r=\bunsuu{f(b)-f(a)}{b-a}$
\item
1点$a$における変化率$r=\bunsuu{f(a)-f(a)}{a-a}$\\
\hspace*{3zw}{\color{red}分母が0になってしまう}
\item
1点における変化率はどうやって求めればいいか
\end{itemize}

\mainslide{微分係数}


\slidepage[m]
%%%%%%%%%%%%%

%%%%%%%%%%%%%%%%%%%%

\newslide{関数の極限}

\vspace*{18mm}


\begin{layer}{120}{0}
\putnotew{96}{73}{\hyperlink{para6pg3}{\fbox{\Ctab{2.5mm}{\scalebox{1}{\scriptsize $\mathstrut||\!\lhd$}}}}}
\putnotew{101}{73}{\hyperlink{para7pg1}{\fbox{\Ctab{2.5mm}{\scalebox{1}{\scriptsize $\mathstrut|\!\lhd$}}}}}
\putnotew{108}{73}{\hyperlink{para7pg6}{\fbox{\Ctab{4.5mm}{\scalebox{1}{\scriptsize $\mathstrut\!\!\lhd\!\!$}}}}}
\putnotew{115}{73}{\hyperlink{para7pg7}{\fbox{\Ctab{4.5mm}{\scalebox{1}{\scriptsize $\mathstrut\!\rhd\!$}}}}}
\putnotew{120}{73}{\hyperlink{para7pg7}{\fbox{\Ctab{2.5mm}{\scalebox{1}{\scriptsize $\mathstrut \!\rhd\!\!|$}}}}}
\putnotew{125}{73}{\hyperlink{para8pg1}{\fbox{\Ctab{2.5mm}{\scalebox{1}{\scriptsize $\mathstrut \!\rhd\!\!||$}}}}}
\putnotee{126}{73}{\scriptsize\color{blue} 7/7}
\end{layer}

\slidepage
\begin{itemize}
\item
$x$が$a$に\underline{限りなく近づく}とする({\color{red}$x\to a$})\\
\hspace*{2zw}{\color{blue}$a$に等しくはないが,いくらでも近くなること}\vspace{-1mm}
\item
$f(x)$が$\alpha$に近づくとき,$\alpha$を{\color{red}極限値}という\\
\hspace*{2zw}{\color{blue}$\dlim_{x \to a}f(x)=\alpha$}と書く\vspace{-1mm}
\item
[例]$\dlim_{x \to 1}(2x+3)=5$
\item
[課題]\monban 次の極限値を求めよ\hfill TextP3\seteda{60}\\
\eda{$\dlim_{x \to 4}(x^2-2x)$}
\eda{$\dlim_{x \to 2}\bunsuu{5x+2}{x+2}$}
\end{itemize}
\label{pageend}\mbox{}

\end{document}

%%% タイトル presen22202
\documentclass[landscape,10pt]{jarticle}
\special{papersize=\the\paperwidth,\the\paperheight}
\usepackage{ketpic,ketlayer}
\usepackage{ketslide}
\usepackage{amsmath,amssymb}
\usepackage{bm,enumerate}
\usepackage[dvipdfmx]{graphicx}
\usepackage{color}
\definecolor{slidecolora}{cmyk}{0.98,0.13,0,0.43}
\definecolor{slidecolorb}{cmyk}{0.2,0,0,0}
\definecolor{slidecolorc}{cmyk}{0.2,0,0,0}
\definecolor{slidecolord}{cmyk}{0.2,0,0,0}
\definecolor{slidecolore}{cmyk}{0,0,0,0.5}
\definecolor{slidecolorf}{cmyk}{0,0,0,0.5}
\definecolor{slidecolori}{cmyk}{0.98,0.13,0,0.43}
\def\setthin#1{\def\thin{#1}}
\setthin{0}
\newcommand{\slidepage}[1][s]{%
\setcounter{ketpicctra}{18}%
\if#1m \setcounter{ketpicctra}{1}\fi
\hypersetup{linkcolor=black}%

\begin{layer}{118}{0}
\putnotee{122}{-\theketpicctra.05}{\small\thepage/\pageref{pageend}}
\end{layer}\hypersetup{linkcolor=blue}

}
\usepackage{emath}
\usepackage{pict2e}
\usepackage{ketlayermorewith2e}
\usepackage[dvipdfmx,colorlinks=true,linkcolor=blue,filecolor=blue]{hyperref}
\newcommand{\hiduke}{0704}
\newcommand{\hako}[2][1]{\fbox{\raisebox{#1mm}{\mbox{}}\raisebox{-#1mm}{\mbox{}}\,\phantom{#2}\,}}
\newcommand{\hakoa}[2][1]{\fbox{\raisebox{#1mm}{\mbox{}}\raisebox{-#1mm}{\mbox{}}\,#2\,}}
\newcommand{\hakom}[2][1]{\hako[#1]{$#2$}}
\newcommand{\hakoma}[2][1]{\hakoa[#1]{$#2$}}
\def\rad{\;\mathrm{rad}}
\def\deg#1{#1^{\circ}}
\newcommand{\sbunsuu}[2]{\scalebox{0.6}{$\bunsuu{#1}{#2}$}}
\def\pow{$\hspace{-1.5mm}^\hspace{-1mm}$}
\def\dlim{\displaystyle\lim}
\newcommand{\brd}[2][1]{\scalebox{#1}{\color{red}\fbox{\color{black}$#2$}}}
\newcommand\down[1][0.5zw]{\vspace{#1}\\}
\newcommand{\sfrac}[3][0.65]{\scalebox{#1}{$\frac{#2}{#3}$}}
\newcommand{\phn}[1]{\phantom{#1}}
\newcommand{\scb}[2][0.6]{\scalebox{#1}{#2}}
\newcommand{\dsum}{\displaystyle\sum}

\setmargin{25}{145}{15}{100}

\ketslideinit

\pagestyle{empty}

\begin{document}

\begin{layer}{120}{0}
\putnotese{0}{0}{{\Large\bf
\color[cmyk]{1,1,0,0}

\begin{layer}{120}{0}
{\Huge \putnotes{60}{20}{三角関数の性質}}
\putnotes{60}{70}{2022.05.16}
\end{layer}

}
}
\end{layer}

\def\mainslidetitley{22}
\def\ketcletter{slidecolora}
\def\ketcbox{slidecolorb}
\def\ketdbox{slidecolorc}
\def\ketcframe{slidecolord}
\def\ketcshadow{slidecolore}
\def\ketdshadow{slidecolorf}
\def\slidetitlex{6}
\def\slidetitlesize{1.3}
\def\mketcletter{slidecolori}
\def\mketcbox{yellow}
\def\mketdbox{yellow}
\def\mketcframe{yellow}
\def\mslidetitlex{62}
\def\mslidetitlesize{2}

\color{black}
\Large\bf\boldmath
\addtocounter{page}{-1}

\def\MARU{}
\renewcommand{\MARU}[1]{{\ooalign{\hfil$#1$\/\hfil\crcr\raise.167ex\hbox{\mathhexbox20D}}}}
\renewcommand{\slidepage}[1][s]{%
\setcounter{ketpicctra}{18}%
\if#1m \setcounter{ketpicctra}{1}\fi
\hypersetup{linkcolor=black}%
\begin{layer}{118}{0}
\putnotee{115}{-\theketpicctra.05}{\small\hiduke-\thepage/\pageref{pageend}}
\end{layer}\hypersetup{linkcolor=blue}
}
\newcounter{ban}
\setcounter{ban}{1}
\newcommand{\monban}[1][\hiduke]{%
#1-\theban\ %
\addtocounter{ban}{1}%
}
\newcommand{\monbannoadd}[1][\hiduke]{%
#1-\theban\ %
}
\newcommand{\addban}{%
\addtocounter{ban}{1}%
}
\newcounter{edawidth}
\newcounter{edactr}
\newcommand{\seteda}[1]{
\setcounter{edawidth}{#1}
\setcounter{edactr}{1}
}
\newcommand{\eda}[1]{%
\Ltab{\theedawidth mm}{[\theedactr]\ #1}%
\addtocounter{edactr}{1}%
}
%%%%%%%%%%%%

%%%%%%%%%%%%%%%%%%%%

\mainslide{ 復習+}


\slidepage[m]
%%%%%%%%%%%%%

%%%%%%%%%%%%%%%%%%%%

\newslide{微分係数と導関数}

\vspace*{18mm}


\begin{layer}{120}{0}
\putnotew{96}{73}{\hyperlink{para0pg0}{\fbox{\Ctab{2.5mm}{\scalebox{1}{\scriptsize $\mathstrut||\!\lhd$}}}}}
\putnotew{101}{73}{\hyperlink{para1pg1}{\fbox{\Ctab{2.5mm}{\scalebox{1}{\scriptsize $\mathstrut|\!\lhd$}}}}}
\putnotew{108}{73}{\hyperlink{para1pg1}{\fbox{\Ctab{4.5mm}{\scalebox{1}{\scriptsize $\mathstrut\!\!\lhd\!\!$}}}}}
\putnotew{115}{73}{\hyperlink{para1pg2}{\fbox{\Ctab{4.5mm}{\scalebox{1}{\scriptsize $\mathstrut\!\rhd\!$}}}}}
\putnotew{120}{73}{\hyperlink{para1pg2}{\fbox{\Ctab{2.5mm}{\scalebox{1}{\scriptsize $\mathstrut \!\rhd\!\!|$}}}}}
\putnotew{125}{73}{\hyperlink{para2pg1}{\fbox{\Ctab{2.5mm}{\scalebox{1}{\scriptsize $\mathstrut \!\rhd\!\!||$}}}}}
\putnotee{126}{73}{\scriptsize\color{blue} 2/2}
\end{layer}

\slidepage
\begin{itemize}
\item
$a$における微分係数$f'(a)$\\
\hspace*{2zw}$f'(a)=\dlim_{z\to a}\bunsuu{f(z)-f(a)}{z-a}$
\item
導関数\\
・微分係数$f'(a)$は$a$の関数\\
・$a$を$x$と書き,導関数という\\
\hspace*{2zw}$f'(x)=\dlim_{z\to x}\bunsuu{f(z)-f(x)}{z-x}$\\
・導関数を求めることを「微分する」
\end{itemize}

\newslide{微分の公式}

\vspace*{18mm}


\begin{layer}{120}{0}
\putnotew{96}{73}{\hyperlink{para1pg2}{\fbox{\Ctab{2.5mm}{\scalebox{1}{\scriptsize $\mathstrut||\!\lhd$}}}}}
\putnotew{101}{73}{\hyperlink{para2pg1}{\fbox{\Ctab{2.5mm}{\scalebox{1}{\scriptsize $\mathstrut|\!\lhd$}}}}}
\putnotew{108}{73}{\hyperlink{para2pg9}{\fbox{\Ctab{4.5mm}{\scalebox{1}{\scriptsize $\mathstrut\!\!\lhd\!\!$}}}}}
\putnotew{115}{73}{\hyperlink{para2pg10}{\fbox{\Ctab{4.5mm}{\scalebox{1}{\scriptsize $\mathstrut\!\rhd\!$}}}}}
\putnotew{120}{73}{\hyperlink{para2pg10}{\fbox{\Ctab{2.5mm}{\scalebox{1}{\scriptsize $\mathstrut \!\rhd\!\!|$}}}}}
\putnotew{125}{73}{\hyperlink{para3pg1}{\fbox{\Ctab{2.5mm}{\scalebox{1}{\scriptsize $\mathstrut \!\rhd\!\!||$}}}}}
\putnotee{126}{73}{\scriptsize\color{blue} 10/10}
\end{layer}

\slidepage
{\color{red}\large

\begin{layer}{120}{0}
\putnotee{65}{32}{$z^2-x^2=(z-x)(z+x)$}
\putnotee{65}{47}{$z^3-x^3=(z-x)(z^2+zx+x^2)$}
\end{layer}

}
\begin{itemize}
\item
定数$c$について $(c)'=\dlim_{z \to x}\bunsuu{c-c}{z-x}$
$=0$
\item
$(x)'=\dlim_{z \to x}\bunsuu{z-x}{z-x}$
$=1$
\item
$(x^2)'=\dlim_{z \to x}\bunsuu{z^2-x^2}{z-x}=\dlim_{z \to x}(z+x)$
$=2x$
\item
$(x^3)'=\dlim_{z \to x}\bunsuu{z^3-x^3}{z-x}=\dlim_{z \to x}(x^2+zx+x^2)$
$=3x^2$
\item
一般に $(x^n)'=\hakoa{\color{blue}$n x^{n-1}$}$
\end{itemize}

\newslide{微分の性質}

\vspace*{18mm}


\begin{layer}{120}{0}
\putnotew{96}{73}{\hyperlink{para2pg10}{\fbox{\Ctab{2.5mm}{\scalebox{1}{\scriptsize $\mathstrut||\!\lhd$}}}}}
\putnotew{101}{73}{\hyperlink{para3pg1}{\fbox{\Ctab{2.5mm}{\scalebox{1}{\scriptsize $\mathstrut|\!\lhd$}}}}}
\putnotew{108}{73}{\hyperlink{para3pg5}{\fbox{\Ctab{4.5mm}{\scalebox{1}{\scriptsize $\mathstrut\!\!\lhd\!\!$}}}}}
\putnotew{115}{73}{\hyperlink{para3pg6}{\fbox{\Ctab{4.5mm}{\scalebox{1}{\scriptsize $\mathstrut\!\rhd\!$}}}}}
\putnotew{120}{73}{\hyperlink{para3pg6}{\fbox{\Ctab{2.5mm}{\scalebox{1}{\scriptsize $\mathstrut \!\rhd\!\!|$}}}}}
\putnotew{125}{73}{\hyperlink{para4pg1}{\fbox{\Ctab{2.5mm}{\scalebox{1}{\scriptsize $\mathstrut \!\rhd\!\!||$}}}}}
\putnotee{126}{73}{\scriptsize\color{blue} 6/6}
\end{layer}

\slidepage
\begin{itemize}
\item
[]$f(x),\ g(x)$と定数$c$について
\item
$(f+g)'=f'+g'$
\item
$(f-g)'=f'-g'$
\item
$(c f)'=c f'$
\item
[例]\hspace{-1mm})\ $(x^2+3x+4)'=(x^2)'+(3x)'+(4)'$
$=2x+3$
\end{itemize}

\mainslide{積と商の微分・記法}


\slidepage[m]
%%%%%%%%%%%%%

%%%%%%%%%%%%%%%%%%%%

\newslide{積の微分}

\vspace*{18mm}

\slidepage
\begin{itemize}
\item
\fbox{$(f\,g)'=f'\,g+f\,g'$}\hspace{1zw}{\color{red}積の微分公式}
\item
[]$\bigl(f(x)g(x)\bigr)'=\dlim_{z \to x}\bunsuu{f(z)g(z)-f(x)g(x)}{z-x}$\\
\hspace*{1zw}$=\dlim_{z \to x}\bunsuu{\bigl(f(z)-f(x)\bigr)g(z)+f(x)\bigl(g(z)-g(x)\bigr)}{z-x}$\\\hspace*{1zw}$=\dlim_{z \to x}\left(\bunsuu{f(z)-f(x)}{z-x}g(z)+f(x)\bunsuu{g(z)-g(x)}{z-x}\right)$\vspace{1mm}\\
\hspace*{1zw}$=f'(x)g(x)+f(x)g'(x)$
\end{itemize}

\newslide{積の微分の例}

\vspace*{18mm}

\slidepage
\begin{itemize}
\item
[例(1)]$y'=\bigl((x+1)(x^2+2x+3) \bigr)'$\\
$\phn{y'}=(x+1)'(x^2+2x+3)+(x+1)(x^2+2x+3)'$\\
$\phn{y'}=(x^2+2x+3)+(x+1)(2x+2)$\\
$\phn{y'}=3x^2+6x+5$
\item
[例(2)]$\bigl(x\cdot \bunsuu{1}{x}\bigr)'=(1)'=0$\\
積の微分で $x'\cdot \bunsuu{1}{x}+x\bigl(\bunsuu{1}{x}\bigr)'=0$\\
$\bunsuu{1}{x}+x\bigl(\bunsuu{1}{x}\bigr)'=0$
 よって \hspace{1zw}\fbox{$\bigl(\bunsuu{1}{x}\bigr)'=-\bunsuu{1}{x^2}$}
\end{itemize}

\newslide{商の微分}

\vspace*{18mm}

\slidepage
\begin{itemize}
\item
\fbox{$\left(\bunsuu{f}{g}\right)'=\bunsuu{f'\,g-f\,g'}{g^2}$}\hspace{1zw}{\color{red}商の微分公式}\vspace{-2mm}
\large
\item
[例(1)]$\left(\bunsuu{2x+1}{3x+1}\right)'$
$=\bunsuu{(2x+1)'(3x+1)-(2x+1)(3x+1)'}{(3x+1)^2}$\\
$\phn{\left(\bunsuu{2x+1}{3x+1}\right)'}=\bunsuu{2(3x+1)-3(2x+1)}{(3x+1)^2}$
$=\bunsuu{-1}{(3x+1)^2}$\vspace{-2mm}
\item
[例(2)]$\left(\bunsuu{1}{x}\right)'$
$=\bunsuu{(1)'(x)-1(x)'}{x^2}$
$=\bunsuu{0-1}{x^2}$
$=-\bunsuu{1}{x^2}$\vspace{-2mm}
\Large
\item
[課題]\monban 次を微分せよ.\seteda{50}\\
\eda{$y=\bunsuu{x}{x+1}$}\eda{$y=\bunsuu{1}{x^2}$}
\end{itemize}

\newslide{導関数の書き方}

\vspace*{18mm}


\begin{layer}{120}{0}
\putnotew{96}{73}{\hyperlink{para3pg10}{\fbox{\Ctab{2.5mm}{\scalebox{1}{\scriptsize $\mathstrut||\!\lhd$}}}}}
\putnotew{101}{73}{\hyperlink{para4pg1}{\fbox{\Ctab{2.5mm}{\scalebox{1}{\scriptsize $\mathstrut|\!\lhd$}}}}}
\putnotew{108}{73}{\hyperlink{para4pg6}{\fbox{\Ctab{4.5mm}{\scalebox{1}{\scriptsize $\mathstrut\!\!\lhd\!\!$}}}}}
\putnotew{115}{73}{\hyperlink{para4pg7}{\fbox{\Ctab{4.5mm}{\scalebox{1}{\scriptsize $\mathstrut\!\rhd\!$}}}}}
\putnotew{120}{73}{\hyperlink{para4pg7}{\fbox{\Ctab{2.5mm}{\scalebox{1}{\scriptsize $\mathstrut \!\rhd\!\!|$}}}}}
\putnotew{125}{73}{\hyperlink{para5pg1}{\fbox{\Ctab{2.5mm}{\scalebox{1}{\scriptsize $\mathstrut \!\rhd\!\!||$}}}}}
\putnotee{126}{73}{\scriptsize\color{blue} 7/7}
\end{layer}

\slidepage

\begin{layer}{120}{0}
\putnotee{70}{24}{\color{red}$\dlim_{z \to x}\bunsuu{f(z)-f(x)}{z-x}$}
\putnotee{75}{36}{\color{red}$=\dlim_{z \to x}\bunsuu{\varDelta y}{\varDelta x}$}
\end{layer}

\begin{itemize}
\item
関数$y=f(x)$を変数$x$で微分する\\
\hspace*{1zw}$y',\ f'(x)$(ラグランジュ)\vspace{2mm}\\
\hspace*{1zw}$\bunsuu{dy}{dx}$(ライプニッツ)
\item
[例]\hspace{-1mm})\ $y=f(x)=x^3$\\
\hspace*{1zw}$y'=f'(x)=f'=\bigl(x^3\bigr)'=3x^2$\vspace{2mm}\\
\hspace*{1zw}$\bunsuu{dy}{dx}=\bunsuu{df}{dx}=\bunsuu{d}{dx}f(x)=\bunsuu{d}{dx}(x^3)=3x^2$
\end{itemize}

\mainslide{べき関数の微分}


\slidepage[m]
%%%%%%%%%%%%%

%%%%%%%%%%%%%%%%%%%%

\newslide{$x^{p}$の微分}

\vspace*{18mm}


\begin{layer}{120}{0}
\putnotew{96}{73}{\hyperlink{para4pg7}{\fbox{\Ctab{2.5mm}{\scalebox{1}{\scriptsize $\mathstrut||\!\lhd$}}}}}
\putnotew{101}{73}{\hyperlink{para5pg1}{\fbox{\Ctab{2.5mm}{\scalebox{1}{\scriptsize $\mathstrut|\!\lhd$}}}}}
\putnotew{108}{73}{\hyperlink{para5pg9}{\fbox{\Ctab{4.5mm}{\scalebox{1}{\scriptsize $\mathstrut\!\!\lhd\!\!$}}}}}
\putnotew{115}{73}{\hyperlink{para5pg10}{\fbox{\Ctab{4.5mm}{\scalebox{1}{\scriptsize $\mathstrut\!\rhd\!$}}}}}
\putnotew{120}{73}{\hyperlink{para5pg10}{\fbox{\Ctab{2.5mm}{\scalebox{1}{\scriptsize $\mathstrut \!\rhd\!\!|$}}}}}
\putnotew{125}{73}{\hyperlink{para6pg1}{\fbox{\Ctab{2.5mm}{\scalebox{1}{\scriptsize $\mathstrut \!\rhd\!\!||$}}}}}
\putnotee{126}{73}{\scriptsize\color{blue} 10/10}
\end{layer}

\slidepage

\begin{layer}{120}{0}
\putnotee{55}{38}{\color{red}\normalsize $\sqrt{z}=w,\sqrt{x}=u$とおくと $z=w^2,x=u^2$}
\end{layer}

\begin{itemize}
\item
$n$が正の整数のとき \fbox{$(x^n)'=n x^{n-1}$}
\item
分数乗\\
\hspace*{0.5zw}$(x^{\frac{1}{2}})'=(\sqrt{x})'$
$=\dlim_{z \to x}\bunsuu{\sqrt{z}-\sqrt{x}}{z-x}$
$=\dlim_{w \to u}\bunsuu{w-u}{w^2-u^2}$\vspace{6mm}\\
\hspace*{0.5zw}$\phn{(x^{\frac{1}{2}})'}=\dlim_{w \to u}\bunsuu{1}{w+u}$
$=\bunsuu{1}{2u}$
$=\bunsuu{1}{2\sqrt{x}}$
$=\bunsuu{1}{2}x^{-\frac{1}{2}}$
\item
[課題]\monban $w^3-u^3=(w-u)(w^2+wu+u^2)$を用いて$(x^{\frac{1}{3}})'$を求めよ.
\end{itemize}

\newslide{$x^{p}$の微分公式}

\vspace*{18mm}


\begin{layer}{120}{0}
\putnotew{96}{73}{\hyperlink{para5pg10}{\fbox{\Ctab{2.5mm}{\scalebox{1}{\scriptsize $\mathstrut||\!\lhd$}}}}}
\putnotew{101}{73}{\hyperlink{para6pg1}{\fbox{\Ctab{2.5mm}{\scalebox{1}{\scriptsize $\mathstrut|\!\lhd$}}}}}
\putnotew{108}{73}{\hyperlink{para6pg6}{\fbox{\Ctab{4.5mm}{\scalebox{1}{\scriptsize $\mathstrut\!\!\lhd\!\!$}}}}}
\putnotew{115}{73}{\hyperlink{para6pg7}{\fbox{\Ctab{4.5mm}{\scalebox{1}{\scriptsize $\mathstrut\!\rhd\!$}}}}}
\putnotew{120}{73}{\hyperlink{para6pg7}{\fbox{\Ctab{2.5mm}{\scalebox{1}{\scriptsize $\mathstrut \!\rhd\!\!|$}}}}}
\putnotew{125}{73}{\hyperlink{para7pg1}{\fbox{\Ctab{2.5mm}{\scalebox{1}{\scriptsize $\mathstrut \!\rhd\!\!||$}}}}}
\putnotee{126}{73}{\scriptsize\color{blue} 7/7}
\end{layer}

\slidepage
\begin{itemize}
\item
$(x^p)'=\hakoma{p x^{p-1}}$
\item
マイナス乗も同じ\\
\hspace*{1zw}$(\bunsuu{1}{x})'$
$=(x^{-1})'$
$=-x^{-2}$
$=-\bunsuu{1}{x^2}$
\item
[課題]\monban 次の関数を微分せよ.\seteda{40}\\
\eda{$y=x^{\frac{1}{4}}$}\eda{$y=x^{-2}$}\eda{$y=x^{-\frac{1}{2}}$}
\end{itemize}

\mainslide{三角関数の微分}


\slidepage[m]
%%%%%%%%%%%%%

%%%%%%%%%%%%%%%%%%%%

\newslide{三角関数のグラフ}

\vspace*{18mm}

\slidepage

\begin{layer}{120}{0}
\putnotes{60}{10}{\scalebox{0.8}{%%% /Users/takatoosetsuo/Desktop/polytech22.git/202-0704/presen/fig/sincosgraph.tex 
%%% Generator=fig22202.cdy 
{\unitlength=18mm%
\begin{picture}%
(6.28,2.4)(-3.14,-1.2)%
\linethickness{0.008in}%%
\polyline(-3.14000,-0.00159)(-3.01440,-0.12685)(-2.88880,-0.25011)(-2.76320,-0.36943)%
(-2.63760,-0.48293)(-2.51200,-0.58882)(-2.38640,-0.68543)(-2.26080,-0.77124)(-2.13520,-0.84491)%
(-2.00960,-0.90526)(-1.88400,-0.95135)(-1.75840,-0.98245)(-1.63280,-0.99808)(-1.50720,-0.99798)%
(-1.38160,-0.98216)(-1.25600,-0.95086)(-1.13040,-0.90458)(-1.00480,-0.84405)(-0.87920,-0.77023)%
(-0.75360,-0.68427)(-0.62800,-0.58753)(-0.50240,-0.48153)(-0.37680,-0.36795)(-0.25120,-0.24857)%
(-0.12560,-0.12527)(0.00000,0.00000)(0.12560,0.12527)(0.25120,0.24857)(0.37680,0.36795)%
(0.50240,0.48153)(0.62800,0.58753)(0.75360,0.68427)(0.87920,0.77023)(1.00480,0.84405)%
(1.13040,0.90458)(1.25600,0.95086)(1.38160,0.98216)(1.50720,0.99798)(1.63280,0.99808)%
(1.75840,0.98245)(1.88400,0.95135)(2.00960,0.90526)(2.13520,0.84491)(2.26080,0.77124)%
(2.38640,0.68543)(2.51200,0.58882)(2.63760,0.48293)(2.76320,0.36943)(2.88880,0.25011)%
(3.01440,0.12685)(3.14000,0.00159)%
%
\polyline(-3.14000,-1.00000)(-3.01440,-0.99192)(-2.88880,-0.96822)(-2.76320,-0.92926)%
(-2.63760,-0.87566)(-2.51200,-0.80827)(-2.38640,-0.72814)(-2.26080,-0.63654)(-2.13520,-0.53491)%
(-2.00960,-0.42486)(-1.88400,-0.30811)(-1.75840,-0.18651)(-1.63280,-0.06196)(-1.50720,0.06355)%
(-1.38160,0.18807)(-1.25600,0.30962)(-1.13040,0.42630)(-1.00480,0.53626)(-0.87920,0.63777)%
(-0.75360,0.72923)(-0.62800,0.80920)(-0.50240,0.87643)(-0.37680,0.92985)(-0.25120,0.96861)%
(-0.12560,0.99212)(0.00000,1.00000)(0.12560,0.99212)(0.25120,0.96861)(0.37680,0.92985)%
(0.50240,0.87643)(0.62800,0.80920)(0.75360,0.72923)(0.87920,0.63777)(1.00480,0.53626)%
(1.13040,0.42630)(1.25600,0.30962)(1.38160,0.18807)(1.50720,0.06355)(1.63280,-0.06196)%
(1.75840,-0.18651)(1.88400,-0.30811)(2.00960,-0.42486)(2.13520,-0.53491)(2.26080,-0.63654)%
(2.38640,-0.72814)(2.51200,-0.80827)(2.63760,-0.87566)(2.76320,-0.92926)(2.88880,-0.96822)%
(3.01440,-0.99192)(3.14000,-1.00000)%
%
\polyline(-3.14000,0.00159)(-3.01440,0.12685)(-2.88880,0.25011)(-2.76320,0.36943)%
(-2.63760,0.48293)(-2.51200,0.58882)(-2.38640,0.68543)(-2.26080,0.77124)(-2.13520,0.84491)%
(-2.00960,0.90526)(-1.88400,0.95135)(-1.75840,0.98245)(-1.63280,0.99808)(-1.50720,0.99798)%
(-1.38160,0.98216)(-1.25600,0.95086)(-1.13040,0.90458)(-1.00480,0.84405)(-0.87920,0.77023)%
(-0.75360,0.68427)(-0.62800,0.58753)(-0.50240,0.48153)(-0.37680,0.36795)(-0.25120,0.24857)%
(-0.12560,0.12527)(0.00000,0.00000)(0.12560,-0.12527)(0.25120,-0.24857)(0.37680,-0.36795)%
(0.50240,-0.48153)(0.62800,-0.58753)(0.75360,-0.68427)(0.87920,-0.77023)(1.00480,-0.84405)%
(1.13040,-0.90458)(1.25600,-0.95086)(1.38160,-0.98216)(1.50720,-0.99798)(1.63280,-0.99808)%
(1.75840,-0.98245)(1.88400,-0.95135)(2.00960,-0.90526)(2.13520,-0.84491)(2.26080,-0.77124)%
(2.38640,-0.68543)(2.51200,-0.58882)(2.63760,-0.48293)(2.76320,-0.36943)(2.88880,-0.25011)%
(3.01440,-0.12685)(3.14000,-0.00159)%
%
\polyline(-3.14000,1.00000)(-3.01440,0.99192)(-2.88880,0.96822)(-2.76320,0.92926)%
(-2.63760,0.87566)(-2.51200,0.80827)(-2.38640,0.72814)(-2.26080,0.63654)(-2.13520,0.53491)%
(-2.00960,0.42486)(-1.88400,0.30811)(-1.75840,0.18651)(-1.63280,0.06196)(-1.50720,-0.06355)%
(-1.38160,-0.18807)(-1.25600,-0.30962)(-1.13040,-0.42630)(-1.00480,-0.53626)(-0.87920,-0.63777)%
(-0.75360,-0.72923)(-0.62800,-0.80920)(-0.50240,-0.87643)(-0.37680,-0.92985)(-0.25120,-0.96861)%
(-0.12560,-0.99212)(0.00000,-1.00000)(0.12560,-0.99212)(0.25120,-0.96861)(0.37680,-0.92985)%
(0.50240,-0.87643)(0.62800,-0.80920)(0.75360,-0.72923)(0.87920,-0.63777)(1.00480,-0.53626)%
(1.13040,-0.42630)(1.25600,-0.30962)(1.38160,-0.18807)(1.50720,-0.06355)(1.63280,0.06196)%
(1.75840,0.18651)(1.88400,0.30811)(2.00960,0.42486)(2.13520,0.53491)(2.26080,0.63654)%
(2.38640,0.72814)(2.51200,0.80827)(2.63760,0.87566)(2.76320,0.92926)(2.88880,0.96822)%
(3.01440,0.99192)(3.14000,1.00000)%
%
\polyline(0.02778,-1.00000)(-0.02778,-1.00000)%
%
\settowidth{\Width}{$-1$}\setlength{\Width}{-1\Width}%
\settoheight{\Height}{$-1$}\settodepth{\Depth}{$-1$}\setlength{\Height}{-0.5\Height}\setlength{\Depth}{0.5\Depth}\addtolength{\Height}{\Depth}%
\put(-0.0555556,-1.0000000){\hspace*{\Width}\raisebox{\Height}{$-1$}}%
%
\polyline(0.02778,1.00000)(-0.02778,1.00000)%
%
\settowidth{\Width}{$1$}\setlength{\Width}{-1\Width}%
\settoheight{\Height}{$1$}\settodepth{\Depth}{$1$}\setlength{\Height}{-0.5\Height}\setlength{\Depth}{0.5\Depth}\addtolength{\Height}{\Depth}%
\put(-0.0555556,1.0000000){\hspace*{\Width}\raisebox{\Height}{$1$}}%
%
\polyline(-1.57080,0.02778)(-1.57080,-0.02778)%
%
\settowidth{\Width}{$-\frac{\pi}{2}$}\setlength{\Width}{-0.5\Width}%
\settoheight{\Height}{$-\frac{\pi}{2}$}\settodepth{\Depth}{$-\frac{\pi}{2}$}\setlength{\Height}{-\Height}%
\put(-1.5700000,-0.0555556){\hspace*{\Width}\raisebox{\Height}{$-\frac{\pi}{2}$}}%
%
\polyline(1.57080,0.02778)(1.57080,-0.02778)%
%
\settowidth{\Width}{$\frac{\pi}{2}$}\setlength{\Width}{-0.5\Width}%
\settoheight{\Height}{$\frac{\pi}{2}$}\settodepth{\Depth}{$\frac{\pi}{2}$}\setlength{\Height}{-\Height}%
\put(1.5700000,-0.0555556){\hspace*{\Width}\raisebox{\Height}{$\frac{\pi}{2}$}}%
%
\settowidth{\Width}{[1]}\setlength{\Width}{-0.5\Width}%
\settoheight{\Height}{[1]}\settodepth{\Depth}{[1]}\setlength{\Height}{\Depth}%
\put(-2.9600000,1.1577778){\hspace*{\Width}\raisebox{\Height}{[1]}}%
%
\settowidth{\Width}{[2]}\setlength{\Width}{-0.5\Width}%
\settoheight{\Height}{[2]}\settodepth{\Depth}{[2]}\setlength{\Height}{\Depth}%
\put(-1.5200000,1.1577778){\hspace*{\Width}\raisebox{\Height}{[2]}}%
%
\settowidth{\Width}{[3]}\setlength{\Width}{-0.5\Width}%
\settoheight{\Height}{[3]}\settodepth{\Depth}{[3]}\setlength{\Height}{\Depth}%
\put(0.2800000,1.1577778){\hspace*{\Width}\raisebox{\Height}{[3]}}%
%
\settowidth{\Width}{[4]}\setlength{\Width}{-0.5\Width}%
\settoheight{\Height}{[4]}\settodepth{\Depth}{[4]}\setlength{\Height}{\Depth}%
\put(1.5400000,1.1677778){\hspace*{\Width}\raisebox{\Height}{[4]}}%
%
\polyline(-3.14000,0.00000)(3.14000,0.00000)%
%
\polyline(0.00000,-1.20000)(0.00000,1.20000)%
%
\settowidth{\Width}{$x$}\setlength{\Width}{0\Width}%
\settoheight{\Height}{$x$}\settodepth{\Depth}{$x$}\setlength{\Height}{-0.5\Height}\setlength{\Depth}{0.5\Depth}\addtolength{\Height}{\Depth}%
\put(3.1677778,0.0000000){\hspace*{\Width}\raisebox{\Height}{$x$}}%
%
\settowidth{\Width}{$y$}\setlength{\Width}{-0.5\Width}%
\settoheight{\Height}{$y$}\settodepth{\Depth}{$y$}\setlength{\Height}{\Depth}%
\put(0.0000000,1.2277778){\hspace*{\Width}\raisebox{\Height}{$y$}}%
%
\settowidth{\Width}{O}\setlength{\Width}{-1\Width}%
\settoheight{\Height}{O}\settodepth{\Depth}{O}\setlength{\Height}{-\Height}%
\put(-0.0277778,-0.0277778){\hspace*{\Width}\raisebox{\Height}{O}}%
%
\end{picture}}%}}
\end{layer}

\vspace{40mm}
\begin{itemize}
\item
[課題]\monban 上の図は\\
\hspace*{2zw}$y=\sin x,y=\cos x, y=-\sin x, y=-\cos x$\\
のグラフである.[1]--[4]の関数を答えよ.
\end{itemize}
%%%%%%%%%%%%%

%%%%%%%%%%%%%%%%%%%%


\newslide{$\sin x,\cos x$の微分}

\vspace*{18mm}


\begin{layer}{120}{0}
\putnotew{96}{73}{\hyperlink{para6pg1}{\fbox{\Ctab{2.5mm}{\scalebox{1}{\scriptsize $\mathstrut||\!\lhd$}}}}}
\putnotew{101}{73}{\hyperlink{para7pg1}{\fbox{\Ctab{2.5mm}{\scalebox{1}{\scriptsize $\mathstrut|\!\lhd$}}}}}
\putnotew{108}{73}{\hyperlink{para7pg2}{\fbox{\Ctab{4.5mm}{\scalebox{1}{\scriptsize $\mathstrut\!\!\lhd\!\!$}}}}}
\putnotew{115}{73}{\hyperlink{para7pg3}{\fbox{\Ctab{4.5mm}{\scalebox{1}{\scriptsize $\mathstrut\!\rhd\!$}}}}}
\putnotew{120}{73}{\hyperlink{para7pg3}{\fbox{\Ctab{2.5mm}{\scalebox{1}{\scriptsize $\mathstrut \!\rhd\!\!|$}}}}}
\putnotew{125}{73}{\hyperlink{para8pg1}{\fbox{\Ctab{2.5mm}{\scalebox{1}{\scriptsize $\mathstrut \!\rhd\!\!||$}}}}}
\putnotee{126}{73}{\scriptsize\color{blue} 3/3}
\end{layer}

\slidepage
\begin{itemize}
\item
[課題]\monbannoadd 「導関数の意味」を用いて導関数を求めよ.\seteda{50}\\
\eda{$y=\sin x$}\eda{$y=\cos x$}
\item
微分公式\\
\hspace*{2zw}\fbox{$(\sin x)'=\cos x,\ (\cos x)'=-\sin x$}
\addban
\item
[課題]\monban 次の問いに答えよ\seteda{100}\\
\eda{$y=\sin x$の$(0,\ 0)$における接線の傾きを求めよ}\\
\eda{$y=2\sin x-3\cos x$を微分せよ}
\end{itemize}

\newslide{$\tan x$の微分}

\vspace*{18mm}


\begin{layer}{120}{0}
\putnotew{96}{73}{\hyperlink{para7pg3}{\fbox{\Ctab{2.5mm}{\scalebox{1}{\scriptsize $\mathstrut||\!\lhd$}}}}}
\putnotew{101}{73}{\hyperlink{para8pg1}{\fbox{\Ctab{2.5mm}{\scalebox{1}{\scriptsize $\mathstrut|\!\lhd$}}}}}
\putnotew{108}{73}{\hyperlink{para8pg4}{\fbox{\Ctab{4.5mm}{\scalebox{1}{\scriptsize $\mathstrut\!\!\lhd\!\!$}}}}}
\putnotew{115}{73}{\hyperlink{para8pg5}{\fbox{\Ctab{4.5mm}{\scalebox{1}{\scriptsize $\mathstrut\!\rhd\!$}}}}}
\putnotew{120}{73}{\hyperlink{para8pg5}{\fbox{\Ctab{2.5mm}{\scalebox{1}{\scriptsize $\mathstrut \!\rhd\!\!|$}}}}}
\putnotew{125}{73}{\hyperlink{para9pg1}{\fbox{\Ctab{2.5mm}{\scalebox{1}{\scriptsize $\mathstrut \!\rhd\!\!||$}}}}}
\putnotee{126}{73}{\scriptsize\color{blue} 5/5}
\end{layer}

\slidepage
{\color{red}\large

\begin{layer}{120}{0}
\putnotee{60}{10}{$\tan x=\bunsuu{\sin x}{\cos x}$}
\putnotee{60}{20}{$\cos^2 x=(\cos x)^2$}
\end{layer}

}
\begin{itemize}
\item
\fbox{$(\tan x)'=\bunsuu{1}{\cos^2 x}$}
\item
[]$(\tan x)'=\bigl(\bunsuu{\sin x}{\cos x}\bigr)'$\\
$\phn{(\tan x)'}=\bunsuu{(\sin x)'(\cos x)-(\sin x)(\cos x)'}{\cos^2 x}$\\
$\phn{(\tan x)'}=\bunsuu{(\cos x \cos x)-\sin x(-\sin x)'}{\cos^2 x}$\\
$\phn{(\tan x)'}=\bunsuu{\cos^2 x+\sin^2 x}{\cos^2 x}=\bunsuu{1}{\cos^2 x}$\\
\end{itemize}

\newslide{課題}

\vspace*{18mm}

\slidepage
\begin{itemize}
\item
[課題]\monban 次の関数を微分せよ\seteda{100}\\
\eda{$y=\sin x \cos x$}\\
\eda{$y=\sin^2 x(=\sin x \sin x)$}\\
\eda{$y=x \tan x$}\\
\eda{$y=\tan x-x$}
\end{itemize}
%%%%%%%%%%%%%

%%%%%%%%%%%%%%%%%%%%


\newslide{ $\sin(ax+b)$の微分}

\vspace*{18mm}

\slidepage

\begin{layer}{120}{0}
\putnotee{75}{50}{\large\color{blue}$(\sin x)'=\dlim_{z \to x}\bunsuu{\sin z-\sin x}{z-x}$}
\end{layer}

{\color{red}\large

\begin{layer}{120}{0}
\putnotec{63}{66}{\MARU{ }}
\putnotec{77}{66}{\MARU{ }}
\qarrowline{63}{64}{14}{-3}{15}
\qarrowline{70}{68}{43}{180}{10}
\putnotec{50}{74}{微分}
\end{layer}

}
\begin{itemize}
\item
[]$y'=(\sin(ax+b))'=\dlim_{z\to x}\bunsuu{\sin(az+b)-\sin(ax+b)}{z-x}$
\\\hspace*{2zw}{\color{blue}$ax+b=u,\ az+b=w$とおくと\\
\hspace*{4zw}$w-u=a(z-x),\ w\to u$}\vspace{-1mm}
\item
[]\hspace*{-0.5zw}$y'=\dlim_{w\to u}\bunsuu{\sin(w)-\sin(u)}{\tfrac{w-u}{a}}=a\dlim_{w\to u}\bunsuu{\sin(w)-\sin(u)}{w-u}$\vspace{6mm}

\hspace*{-0.5zw}$\phantom{y'}=a\cos u=a\cos(ax+b)$
\item
[]\hspace*{3zw}\fbox{$\sin(ax+b)'=a\cos(ax+b)$}
\end{itemize}

\newslide{ $f(ax+b)$の微分}

\vspace*{18mm}

\slidepage
{\color{red}\large

\begin{layer}{120}{0}
\putnotec{44}{10}{\MARU{ }}
\putnotec{55}{10}{\MARU{ }}
\qarrowline{44}{8}{10}{0}{15}
\qarrowline{48}{13}{37}{180}{15}
\putnotec{37}{18}{そのまま微分}
\end{layer}

}
\begin{itemize}
\item
\fbox{$f(ax+b)'=a f'(ax+b)$}\vspace{7mm}
\item
$\bigl(\cos(3x+1)\bigr)'$
$=3(-\sin(3x+1))$
$=-3\sin(3x+1)$
\item
$\bigl((2x+3)^5\bigr)'$
$=2\cdot 5(2x+3)^4$
$=10(2x+3)^4$
\item
[課題]\monban 微分せよ\seteda{50}\\
\eda{$y=\sin 3x$}\eda{$y=(5x+1)^3$}\\
\eda{$y=\sqrt{2x+3}$}\eda{$y=\tan(-x+1)$}
\end{itemize}
\label{pageend}\mbox{}

\end{document}

%%% タイトル presen22204
\documentclass[landscape,10pt]{jarticle}
\special{papersize=\the\paperwidth,\the\paperheight}
\usepackage{ketpic,ketlayer}
\usepackage{ketslide}
\usepackage{amsmath,amssymb}
\usepackage{bm,enumerate}
\usepackage[dvipdfmx]{graphicx}
\usepackage{color}
\definecolor{slidecolora}{cmyk}{0.98,0.13,0,0.43}
\definecolor{slidecolorb}{cmyk}{0.2,0,0,0}
\definecolor{slidecolorc}{cmyk}{0.2,0,0,0}
\definecolor{slidecolord}{cmyk}{0.2,0,0,0}
\definecolor{slidecolore}{cmyk}{0,0,0,0.5}
\definecolor{slidecolorf}{cmyk}{0,0,0,0.5}
\definecolor{slidecolori}{cmyk}{0.98,0.13,0,0.43}
\def\setthin#1{\def\thin{#1}}
\setthin{0}
\newcommand{\slidepage}[1][s]{%
\setcounter{ketpicctra}{18}%
\if#1m \setcounter{ketpicctra}{1}\fi
\hypersetup{linkcolor=black}%

\begin{layer}{118}{0}
\putnotee{122}{-\theketpicctra.05}{\small\thepage/\pageref{pageend}}
\end{layer}\hypersetup{linkcolor=blue}

}
\usepackage{emath}
\usepackage{pict2e}
\usepackage{ketlayermorewith2e}
\usepackage[dvipdfmx,colorlinks=true,linkcolor=blue,filecolor=blue]{hyperref}
\newcommand{\hiduke}{0829}
\newcommand{\hako}[2][1]{\fbox{\raisebox{#1mm}{\mbox{}}\raisebox{-#1mm}{\mbox{}}\,\phantom{#2}\,}}
\newcommand{\hakoa}[2][1]{\fbox{\raisebox{#1mm}{\mbox{}}\raisebox{-#1mm}{\mbox{}}\,#2\,}}
\newcommand{\hakom}[2][1]{\hako[#1]{$#2$}}
\newcommand{\hakoma}[2][1]{\hakoa[#1]{$#2$}}
\def\rad{\;\mathrm{rad}}
\def\deg#1{#1^{\circ}}
\newcommand{\sbunsuu}[2]{\scalebox{0.6}{$\bunsuu{#1}{#2}$}}
\def\pow{$\hspace{-1.5mm}^\hspace{-1mm}$}
\def\dlim{\displaystyle\lim}
\newcommand{\brd}[2][1]{\scalebox{#1}{\color{red}\fbox{\color{black}$#2$}}}
\newcommand\down[1][0.5zw]{\vspace{#1}\\}
\newcommand{\sfrac}[3][0.65]{\scalebox{#1}{$\frac{#2}{#3}$}}
\newcommand{\phn}[1]{\phantom{#1}}
\newcommand{\scb}[2][0.6]{\scalebox{#1}{#2}}
\newcommand{\dsum}{\displaystyle\sum}
\def\pow{$\hspace{-1.5mm}^\hspace{-1mm}$}
\def\dlim{\displaystyle\lim}
\def\dint{\displaystyle\int}

\setmargin{25}{145}{15}{100}

\ketslideinit

\pagestyle{empty}

\begin{document}

\begin{layer}{120}{0}
\putnotese{0}{0}{{\Large\bf
\color[cmyk]{1,1,0,0}

\begin{layer}{120}{0}
{\Huge \putnotes{60}{20}{積分法1}}
\putnotes{60}{70}{2022.8.29}
\end{layer}

}
}
\end{layer}

\def\mainslidetitley{22}
\def\ketcletter{slidecolora}
\def\ketcbox{slidecolorb}
\def\ketdbox{slidecolorc}
\def\ketcframe{slidecolord}
\def\ketcshadow{slidecolore}
\def\ketdshadow{slidecolorf}
\def\slidetitlex{6}
\def\slidetitlesize{1.3}
\def\mketcletter{slidecolori}
\def\mketcbox{yellow}
\def\mketdbox{yellow}
\def\mketcframe{yellow}
\def\mslidetitlex{62}
\def\mslidetitlesize{2}

\color{black}
\Large\bf\boldmath
\addtocounter{page}{-1}

\def\MARU{}
\renewcommand{\MARU}[1]{{\ooalign{\hfil$#1$\/\hfil\crcr\raise.167ex\hbox{\mathhexbox20D}}}}
\renewcommand{\slidepage}[1][s]{%
\setcounter{ketpicctra}{18}%
\if#1m \setcounter{ketpicctra}{1}\fi
\hypersetup{linkcolor=black}%
\begin{layer}{118}{0}
\putnotee{115}{-\theketpicctra.05}{\small\hiduke-\thepage/\pageref{pageend}}
\end{layer}\hypersetup{linkcolor=blue}
}
\newcounter{ban}
\setcounter{ban}{1}
\newcommand{\monban}[1][\hiduke]{%
#1-\theban\ %
\addtocounter{ban}{1}%
}
\newcommand{\monbannoadd}[1][\hiduke]{%
#1-\theban\ %
}
\newcommand{\addban}{%
\addtocounter{ban}{1}%
}
\newcounter{edawidth}
\newcounter{edactr}
\newcommand{\seteda}[1]{% 20220708 modified
\setcounter{edawidth}{#1}
\setcounter{edactr}{1}
}
\newcommand{\eda}[2][\theedawidth]{%
\Ltab{#1 mm}{[\theedactr]\ #2}%
\addtocounter{edactr}{1}%
}
%%%%%%%%%%%%

%%%%%%%%%%%%%%%%%%%%

\mainslide{ 復習(微分)}


\slidepage[m]
%%%%%%%%%%%%%

%%%%%%%%%%%%%%%%%%%%

\newslide{微分と導関数}

\vspace*{18mm}


\begin{layer}{120}{0}
\putnotew{96}{73}{\hyperlink{para0pg0}{\fbox{\Ctab{2.5mm}{\scalebox{1}{\scriptsize $\mathstrut||\!\lhd$}}}}}
\putnotew{101}{73}{\hyperlink{para1pg1}{\fbox{\Ctab{2.5mm}{\scalebox{1}{\scriptsize $\mathstrut|\!\lhd$}}}}}
\putnotew{108}{73}{\hyperlink{para1pg2}{\fbox{\Ctab{4.5mm}{\scalebox{1}{\scriptsize $\mathstrut\!\!\lhd\!\!$}}}}}
\putnotew{115}{73}{\hyperlink{para1pg3}{\fbox{\Ctab{4.5mm}{\scalebox{1}{\scriptsize $\mathstrut\!\rhd\!$}}}}}
\putnotew{120}{73}{\hyperlink{para1pg3}{\fbox{\Ctab{2.5mm}{\scalebox{1}{\scriptsize $\mathstrut \!\rhd\!\!|$}}}}}
\putnotew{125}{73}{\hyperlink{para2pg1}{\fbox{\Ctab{2.5mm}{\scalebox{1}{\scriptsize $\mathstrut \!\rhd\!\!||$}}}}}
\putnotee{126}{73}{\scriptsize\color{blue} 3/3}
\end{layer}

\slidepage
\begin{itemize}
\item
$a$における微分係数$f'(a)=$接線の傾き\\
\hspace*{2zw}$f'(a)=\dlim_{z\to a}\bunsuu{f(z)-f(a)}{z-a}$\vspace{-2mm}
\item
導関数\\
・微分係数を$x$の関数$f'(x)$としたもの\\
\hspace*{2zw}$y'=f'(x)=\dlim_{z\to x}\bunsuu{f(z)-f(x)}{z-x}$\\
・導関数を求めることを「微分する」
\item
[課題]\monban 次の関数の導関数はどうなるか.\seteda{50}\\
\eda{$y=x^2-2x$}\eda{$y=\sin x$}
\end{itemize}

\newslide{$x^n$の微分}

\vspace*{18mm}


\begin{layer}{120}{0}
\putnotew{96}{73}{\hyperlink{para1pg3}{\fbox{\Ctab{2.5mm}{\scalebox{1}{\scriptsize $\mathstrut||\!\lhd$}}}}}
\putnotew{125}{73}{\hyperlink{para3pg1}{\fbox{\Ctab{2.5mm}{\scalebox{1}{\scriptsize $\mathstrut \!\rhd\!\!||$}}}}}
\putnotee{126}{73}{\scriptsize\color{blue} 1/1}
\end{layer}

\slidepage
\begin{itemize}
\item
$(c)'=0$($c$は定数)
\item
$(x)'=1$
\item
$(x^2)'=2x$
\item
$(x^3)'=3x^2$
\item
一般に $(x^n)'=n x^{n-1}$\\
 ・$n$は負の整数でも分数(実数)でもよい
\end{itemize}
%%%%%%%%%%%%%

%%%%%%%%%%%%%%%%%%%%


\newslide{微分の性質}

\vspace*{18mm}


\begin{layer}{120}{0}
\putnotew{96}{73}{\hyperlink{para2pg1}{\fbox{\Ctab{2.5mm}{\scalebox{1}{\scriptsize $\mathstrut||\!\lhd$}}}}}
\putnotew{101}{73}{\hyperlink{para3pg1}{\fbox{\Ctab{2.5mm}{\scalebox{1}{\scriptsize $\mathstrut|\!\lhd$}}}}}
\putnotew{108}{73}{\hyperlink{para3pg4}{\fbox{\Ctab{4.5mm}{\scalebox{1}{\scriptsize $\mathstrut\!\!\lhd\!\!$}}}}}
\putnotew{115}{73}{\hyperlink{para3pg5}{\fbox{\Ctab{4.5mm}{\scalebox{1}{\scriptsize $\mathstrut\!\rhd\!$}}}}}
\putnotew{120}{73}{\hyperlink{para3pg5}{\fbox{\Ctab{2.5mm}{\scalebox{1}{\scriptsize $\mathstrut \!\rhd\!\!|$}}}}}
\putnotew{125}{73}{\hyperlink{para4pg1}{\fbox{\Ctab{2.5mm}{\scalebox{1}{\scriptsize $\mathstrut \!\rhd\!\!||$}}}}}
\putnotee{126}{73}{\scriptsize\color{blue} 5/5}
\end{layer}

\slidepage
\begin{itemize}
\item
\Ltab{70mm}{$(f+g)'=f'+g'$}和の微分
\item
\Ltab{70mm}{$(cf)'=cf'$ ($c$は定数)}定数倍の微分
\item
\Ltab{70mm}{$(fg)'=f'g+fg'$}積の微分
\item
\Ltab{70mm}{$\left(\bunsuu{f}{g}\right)'=\bunsuu{f'g-fg'}{g^2}$}商の微分
\item
\Ltab{70mm}{$\bigl(f(ax+b)\bigr)'=af'(ax+b)$}$ax+b$型の微分
\end{itemize}

\newslide{課題(微分の性質)}

\vspace*{18mm}


\begin{layer}{120}{0}
\putnotew{96}{73}{\hyperlink{para3pg5}{\fbox{\Ctab{2.5mm}{\scalebox{1}{\scriptsize $\mathstrut||\!\lhd$}}}}}
\putnotew{125}{73}{\hyperlink{para5pg1}{\fbox{\Ctab{2.5mm}{\scalebox{1}{\scriptsize $\mathstrut \!\rhd\!\!||$}}}}}
\putnotee{126}{73}{\scriptsize\color{blue} 1/1}
\end{layer}

\slidepage
\begin{itemize}
\item
[課題]\monban 微分せよ.\seteda{100}\\
\eda{$y=x^4+x^3-x^2$}\\
\eda{$y=3x^5$}\\
\eda{$y=(x+1)\sqrt{x}$\\
\hspace*{2zw}{\color{blue}\large
ヒント:$(\sqrt{x})'=\bigl(x^{\sbunsuu{1}{2}}\bigr)'=\bunsuu{1}{2}x^{-\sbunsuu{1}{2}}$}}\\
\eda{$y=\bunsuu{x^2+1}{x+2}$}\\
\eda{$y=(2x+3)^5$}\\
\hspace*{2zw}{\color{blue}\large
ヒント:$\Bigl(\hakom{2x+3}^5\Bigr)'=5\hakom{2x+3}^4$}
\end{itemize}
%%%%%%%%%%%%%

%%%%%%%%%%%%%%%%%%%%


\newslide{三角関数の微分}

\vspace*{18mm}


\begin{layer}{120}{0}
\putnotew{96}{73}{\hyperlink{para4pg1}{\fbox{\Ctab{2.5mm}{\scalebox{1}{\scriptsize $\mathstrut||\!\lhd$}}}}}
\putnotew{101}{73}{\hyperlink{para5pg1}{\fbox{\Ctab{2.5mm}{\scalebox{1}{\scriptsize $\mathstrut|\!\lhd$}}}}}
\putnotew{108}{73}{\hyperlink{para5pg3}{\fbox{\Ctab{4.5mm}{\scalebox{1}{\scriptsize $\mathstrut\!\!\lhd\!\!$}}}}}
\putnotew{115}{73}{\hyperlink{para5pg4}{\fbox{\Ctab{4.5mm}{\scalebox{1}{\scriptsize $\mathstrut\!\rhd\!$}}}}}
\putnotew{120}{73}{\hyperlink{para5pg4}{\fbox{\Ctab{2.5mm}{\scalebox{1}{\scriptsize $\mathstrut \!\rhd\!\!|$}}}}}
\putnotew{125}{73}{\hyperlink{para6pg1}{\fbox{\Ctab{2.5mm}{\scalebox{1}{\scriptsize $\mathstrut \!\rhd\!\!||$}}}}}
\putnotee{126}{73}{\scriptsize\color{blue} 4/4}
\end{layer}

\slidepage
\begin{itemize}
\item
$y=\sin x, \cos x, \tan x$\\
\hspace*{4zw}角$x$の単位はラジアン
\item
$(\sin x)'=\cos x$
\item
$(\cos x)'=-\sin x$
\item
$(\tan x)'=\bunsuu{1}{\cos^2 x}$
\item
[課題]\monban 次の関数を微分せよ.\seteda{40}\\
\eda[45]{$y=x+\cos x$}\eda{$y=x\sin x$}\eda{$y=\sin 4x$}
\end{itemize}

\newslide{指数関数$y=e^{x}$の微分}

\vspace*{18mm}


\begin{layer}{120}{0}
\putnotew{96}{73}{\hyperlink{para5pg4}{\fbox{\Ctab{2.5mm}{\scalebox{1}{\scriptsize $\mathstrut||\!\lhd$}}}}}
\putnotew{101}{73}{\hyperlink{para6pg1}{\fbox{\Ctab{2.5mm}{\scalebox{1}{\scriptsize $\mathstrut|\!\lhd$}}}}}
\putnotew{108}{73}{\hyperlink{para6pg3}{\fbox{\Ctab{4.5mm}{\scalebox{1}{\scriptsize $\mathstrut\!\!\lhd\!\!$}}}}}
\putnotew{115}{73}{\hyperlink{para6pg4}{\fbox{\Ctab{4.5mm}{\scalebox{1}{\scriptsize $\mathstrut\!\rhd\!$}}}}}
\putnotew{120}{73}{\hyperlink{para6pg4}{\fbox{\Ctab{2.5mm}{\scalebox{1}{\scriptsize $\mathstrut \!\rhd\!\!|$}}}}}
\putnotew{125}{73}{\hyperlink{para7pg1}{\fbox{\Ctab{2.5mm}{\scalebox{1}{\scriptsize $\mathstrut \!\rhd\!\!||$}}}}}
\putnotee{126}{73}{\scriptsize\color{blue} 4/4}
\end{layer}

\slidepage
\begin{itemize}
\item
$e$はネピアの定数
\item
[課題]\monbannoadd $e$の値を小数点以下5位まで書け.
\addban
\item
\fbox{$(e^x)'=e^x$}
\item
[課題]\monban 次の関数を微分せよ.\seteda{40}\\
\eda[46]{$y=e^x+x^2$}\eda{$y=e^{2x}$}\eda{$y=e^{-x}$}
\end{itemize}

\newslide{自然対数$y=\log x(=\ln x)$の微分}

\vspace*{18mm}


\begin{layer}{120}{0}
\putnotew{96}{73}{\hyperlink{para6pg4}{\fbox{\Ctab{2.5mm}{\scalebox{1}{\scriptsize $\mathstrut||\!\lhd$}}}}}
\putnotew{101}{73}{\hyperlink{para7pg1}{\fbox{\Ctab{2.5mm}{\scalebox{1}{\scriptsize $\mathstrut|\!\lhd$}}}}}
\putnotew{108}{73}{\hyperlink{para7pg2}{\fbox{\Ctab{4.5mm}{\scalebox{1}{\scriptsize $\mathstrut\!\!\lhd\!\!$}}}}}
\putnotew{115}{73}{\hyperlink{para7pg3}{\fbox{\Ctab{4.5mm}{\scalebox{1}{\scriptsize $\mathstrut\!\rhd\!$}}}}}
\putnotew{120}{73}{\hyperlink{para7pg3}{\fbox{\Ctab{2.5mm}{\scalebox{1}{\scriptsize $\mathstrut \!\rhd\!\!|$}}}}}
\putnotew{125}{73}{\hyperlink{para8pg1}{\fbox{\Ctab{2.5mm}{\scalebox{1}{\scriptsize $\mathstrut \!\rhd\!\!||$}}}}}
\putnotee{126}{73}{\scriptsize\color{blue} 3/3}
\end{layer}

\slidepage
\begin{itemize}
\item
ネピア数$e$を底とする対数
\item
[]\hspace*{2zw}$y=\log x\ \Longleftrightarrow\ e^y=x$
\item
\fbox{$(\log x)'=\bunsuu{1}{x}$}
\item
[課題]\monban 次の関数を微分せよ.\seteda{35}\\
\eda[45]{$y=\log x+e^{x}$}\eda{$y=\log 2x$}\eda{$y=\log(x+2)$}
\end{itemize}

\mainslide{不定積分}


\slidepage[m]
%%%%%%%%%%%%%

%%%%%%%%%%%%%%%%%%%%

\newslide{微分と積分}

\vspace*{18mm}


\begin{layer}{120}{0}
\putnotew{96}{73}{\hyperlink{para7pg3}{\fbox{\Ctab{2.5mm}{\scalebox{1}{\scriptsize $\mathstrut||\!\lhd$}}}}}
\putnotew{101}{73}{\hyperlink{para8pg1}{\fbox{\Ctab{2.5mm}{\scalebox{1}{\scriptsize $\mathstrut|\!\lhd$}}}}}
\putnotew{108}{73}{\hyperlink{para8pg6}{\fbox{\Ctab{4.5mm}{\scalebox{1}{\scriptsize $\mathstrut\!\!\lhd\!\!$}}}}}
\putnotew{115}{73}{\hyperlink{para8pg7}{\fbox{\Ctab{4.5mm}{\scalebox{1}{\scriptsize $\mathstrut\!\rhd\!$}}}}}
\putnotew{120}{73}{\hyperlink{para8pg7}{\fbox{\Ctab{2.5mm}{\scalebox{1}{\scriptsize $\mathstrut \!\rhd\!\!|$}}}}}
\putnotew{125}{73}{\hyperlink{para9pg1}{\fbox{\Ctab{2.5mm}{\scalebox{1}{\scriptsize $\mathstrut \!\rhd\!\!||$}}}}}
\putnotee{126}{73}{\scriptsize\color{blue} 7/7}
\end{layer}

\slidepage

\begin{layer}{120}{0}
\putnotese{5}{30}{\normalsize (1)}
\putnotese{10}{33}{\scalebox{0.8}{%%% /Users/takatoosetsuo/Dropbox/2021polytech/203/fig/biseki1.tex 
%%% Generator=presen203.cdy 
{\unitlength=7mm%
\begin{picture}%
(5,5)(-0.5,-0.5)%
\special{pn 8}%
%
\Large\bf\boldmath%
\normalsize%
\special{pn 8}%
\special{pa -4 -276}\special{pa 4 -276}\special{fp}\special{pa 35 -276}\special{pa 43 -276}\special{fp}%
\special{pa 75 -276}\special{pa 83 -276}\special{fp}\special{pa 114 -276}\special{pa 122 -276}\special{fp}%
\special{pa 153 -276}\special{pa 161 -276}\special{fp}\special{pa 193 -276}\special{pa 201 -276}\special{fp}%
\special{pa 232 -276}\special{pa 240 -276}\special{fp}\special{pa 272 -276}\special{pa 280 -276}\special{fp}%
\special{pa 311 -276}\special{pa 319 -276}\special{fp}\special{pa 350 -276}\special{pa 358 -276}\special{fp}%
\special{pa 390 -276}\special{pa 398 -276}\special{fp}\special{pa 429 -276}\special{pa 437 -276}\special{fp}%
\special{pa 468 -276}\special{pa 476 -276}\special{fp}\special{pa 508 -276}\special{pa 516 -276}\special{fp}%
\special{pa 547 -276}\special{pa 555 -276}\special{fp}\special{pa 587 -276}\special{pa 595 -276}\special{fp}%
\special{pa 626 -276}\special{pa 634 -276}\special{fp}\special{pa 665 -276}\special{pa 673 -276}\special{fp}%
\special{pa 705 -276}\special{pa 713 -276}\special{fp}\special{pa 744 -276}\special{pa 752 -276}\special{fp}%
\special{pa 783 -276}\special{pa 791 -276}\special{fp}\special{pa 823 -276}\special{pa 831 -276}\special{fp}%
\special{pa 862 -276}\special{pa 870 -276}\special{fp}\special{pa 902 -276}\special{pa 910 -276}\special{fp}%
\special{pa 941 -276}\special{pa 949 -276}\special{fp}\special{pa 980 -276}\special{pa 988 -276}\special{fp}%
\special{pa 1020 -276}\special{pa 1028 -276}\special{fp}\special{pa 1059 -276}\special{pa 1067 -276}\special{fp}%
\special{pa 1098 -276}\special{pa 1106 -276}\special{fp}\special{pn 8}%
\special{pn 8}%
\special{pa 276 4}\special{pa 276 -4}\special{fp}\special{pa 276 -35}\special{pa 276 -43}\special{fp}%
\special{pa 276 -75}\special{pa 276 -83}\special{fp}\special{pa 276 -114}\special{pa 276 -122}\special{fp}%
\special{pa 276 -153}\special{pa 276 -161}\special{fp}\special{pa 276 -193}\special{pa 276 -201}\special{fp}%
\special{pa 276 -232}\special{pa 276 -240}\special{fp}\special{pa 276 -272}\special{pa 276 -280}\special{fp}%
\special{pa 276 -311}\special{pa 276 -319}\special{fp}\special{pa 276 -350}\special{pa 276 -358}\special{fp}%
\special{pa 276 -390}\special{pa 276 -398}\special{fp}\special{pa 276 -429}\special{pa 276 -437}\special{fp}%
\special{pa 276 -468}\special{pa 276 -476}\special{fp}\special{pa 276 -508}\special{pa 276 -516}\special{fp}%
\special{pa 276 -547}\special{pa 276 -555}\special{fp}\special{pa 276 -587}\special{pa 276 -595}\special{fp}%
\special{pa 276 -626}\special{pa 276 -634}\special{fp}\special{pa 276 -665}\special{pa 276 -673}\special{fp}%
\special{pa 276 -705}\special{pa 276 -713}\special{fp}\special{pa 276 -744}\special{pa 276 -752}\special{fp}%
\special{pa 276 -783}\special{pa 276 -791}\special{fp}\special{pa 276 -823}\special{pa 276 -831}\special{fp}%
\special{pa 276 -862}\special{pa 276 -870}\special{fp}\special{pa 276 -902}\special{pa 276 -910}\special{fp}%
\special{pa 276 -941}\special{pa 276 -949}\special{fp}\special{pa 276 -980}\special{pa 276 -988}\special{fp}%
\special{pa 276 -1020}\special{pa 276 -1028}\special{fp}\special{pa 276 -1059}\special{pa 276 -1067}\special{fp}%
\special{pa 276 -1098}\special{pa 276 -1106}\special{fp}\special{pn 8}%
\special{pn 8}%
\special{pa -4 -551}\special{pa 4 -551}\special{fp}\special{pa 35 -551}\special{pa 43 -551}\special{fp}%
\special{pa 75 -551}\special{pa 83 -551}\special{fp}\special{pa 114 -551}\special{pa 122 -551}\special{fp}%
\special{pa 153 -551}\special{pa 161 -551}\special{fp}\special{pa 193 -551}\special{pa 201 -551}\special{fp}%
\special{pa 232 -551}\special{pa 240 -551}\special{fp}\special{pa 272 -551}\special{pa 280 -551}\special{fp}%
\special{pa 311 -551}\special{pa 319 -551}\special{fp}\special{pa 350 -551}\special{pa 358 -551}\special{fp}%
\special{pa 390 -551}\special{pa 398 -551}\special{fp}\special{pa 429 -551}\special{pa 437 -551}\special{fp}%
\special{pa 468 -551}\special{pa 476 -551}\special{fp}\special{pa 508 -551}\special{pa 516 -551}\special{fp}%
\special{pa 547 -551}\special{pa 555 -551}\special{fp}\special{pa 587 -551}\special{pa 595 -551}\special{fp}%
\special{pa 626 -551}\special{pa 634 -551}\special{fp}\special{pa 665 -551}\special{pa 673 -551}\special{fp}%
\special{pa 705 -551}\special{pa 713 -551}\special{fp}\special{pa 744 -551}\special{pa 752 -551}\special{fp}%
\special{pa 783 -551}\special{pa 791 -551}\special{fp}\special{pa 823 -551}\special{pa 831 -551}\special{fp}%
\special{pa 862 -551}\special{pa 870 -551}\special{fp}\special{pa 902 -551}\special{pa 910 -551}\special{fp}%
\special{pa 941 -551}\special{pa 949 -551}\special{fp}\special{pa 980 -551}\special{pa 988 -551}\special{fp}%
\special{pa 1020 -551}\special{pa 1028 -551}\special{fp}\special{pa 1059 -551}\special{pa 1067 -551}\special{fp}%
\special{pa 1098 -551}\special{pa 1106 -551}\special{fp}\special{pn 8}%
\special{pn 8}%
\special{pa 551 4}\special{pa 551 -4}\special{fp}\special{pa 551 -35}\special{pa 551 -43}\special{fp}%
\special{pa 551 -75}\special{pa 551 -83}\special{fp}\special{pa 551 -114}\special{pa 551 -122}\special{fp}%
\special{pa 551 -153}\special{pa 551 -161}\special{fp}\special{pa 551 -193}\special{pa 551 -201}\special{fp}%
\special{pa 551 -232}\special{pa 551 -240}\special{fp}\special{pa 551 -272}\special{pa 551 -280}\special{fp}%
\special{pa 551 -311}\special{pa 551 -319}\special{fp}\special{pa 551 -350}\special{pa 551 -358}\special{fp}%
\special{pa 551 -390}\special{pa 551 -398}\special{fp}\special{pa 551 -429}\special{pa 551 -437}\special{fp}%
\special{pa 551 -468}\special{pa 551 -476}\special{fp}\special{pa 551 -508}\special{pa 551 -516}\special{fp}%
\special{pa 551 -547}\special{pa 551 -555}\special{fp}\special{pa 551 -587}\special{pa 551 -595}\special{fp}%
\special{pa 551 -626}\special{pa 551 -634}\special{fp}\special{pa 551 -665}\special{pa 551 -673}\special{fp}%
\special{pa 551 -705}\special{pa 551 -713}\special{fp}\special{pa 551 -744}\special{pa 551 -752}\special{fp}%
\special{pa 551 -783}\special{pa 551 -791}\special{fp}\special{pa 551 -823}\special{pa 551 -831}\special{fp}%
\special{pa 551 -862}\special{pa 551 -870}\special{fp}\special{pa 551 -902}\special{pa 551 -910}\special{fp}%
\special{pa 551 -941}\special{pa 551 -949}\special{fp}\special{pa 551 -980}\special{pa 551 -988}\special{fp}%
\special{pa 551 -1020}\special{pa 551 -1028}\special{fp}\special{pa 551 -1059}\special{pa 551 -1067}\special{fp}%
\special{pa 551 -1098}\special{pa 551 -1106}\special{fp}\special{pn 8}%
\special{pn 8}%
\special{pa -4 -827}\special{pa 4 -827}\special{fp}\special{pa 35 -827}\special{pa 43 -827}\special{fp}%
\special{pa 75 -827}\special{pa 83 -827}\special{fp}\special{pa 114 -827}\special{pa 122 -827}\special{fp}%
\special{pa 153 -827}\special{pa 161 -827}\special{fp}\special{pa 193 -827}\special{pa 201 -827}\special{fp}%
\special{pa 232 -827}\special{pa 240 -827}\special{fp}\special{pa 272 -827}\special{pa 280 -827}\special{fp}%
\special{pa 311 -827}\special{pa 319 -827}\special{fp}\special{pa 350 -827}\special{pa 358 -827}\special{fp}%
\special{pa 390 -827}\special{pa 398 -827}\special{fp}\special{pa 429 -827}\special{pa 437 -827}\special{fp}%
\special{pa 468 -827}\special{pa 476 -827}\special{fp}\special{pa 508 -827}\special{pa 516 -827}\special{fp}%
\special{pa 547 -827}\special{pa 555 -827}\special{fp}\special{pa 587 -827}\special{pa 595 -827}\special{fp}%
\special{pa 626 -827}\special{pa 634 -827}\special{fp}\special{pa 665 -827}\special{pa 673 -827}\special{fp}%
\special{pa 705 -827}\special{pa 713 -827}\special{fp}\special{pa 744 -827}\special{pa 752 -827}\special{fp}%
\special{pa 783 -827}\special{pa 791 -827}\special{fp}\special{pa 823 -827}\special{pa 831 -827}\special{fp}%
\special{pa 862 -827}\special{pa 870 -827}\special{fp}\special{pa 902 -827}\special{pa 910 -827}\special{fp}%
\special{pa 941 -827}\special{pa 949 -827}\special{fp}\special{pa 980 -827}\special{pa 988 -827}\special{fp}%
\special{pa 1020 -827}\special{pa 1028 -827}\special{fp}\special{pa 1059 -827}\special{pa 1067 -827}\special{fp}%
\special{pa 1098 -827}\special{pa 1106 -827}\special{fp}\special{pn 8}%
\special{pn 8}%
\special{pa 827 4}\special{pa 827 -4}\special{fp}\special{pa 827 -35}\special{pa 827 -43}\special{fp}%
\special{pa 827 -75}\special{pa 827 -83}\special{fp}\special{pa 827 -114}\special{pa 827 -122}\special{fp}%
\special{pa 827 -153}\special{pa 827 -161}\special{fp}\special{pa 827 -193}\special{pa 827 -201}\special{fp}%
\special{pa 827 -232}\special{pa 827 -240}\special{fp}\special{pa 827 -272}\special{pa 827 -280}\special{fp}%
\special{pa 827 -311}\special{pa 827 -319}\special{fp}\special{pa 827 -350}\special{pa 827 -358}\special{fp}%
\special{pa 827 -390}\special{pa 827 -398}\special{fp}\special{pa 827 -429}\special{pa 827 -437}\special{fp}%
\special{pa 827 -468}\special{pa 827 -476}\special{fp}\special{pa 827 -508}\special{pa 827 -516}\special{fp}%
\special{pa 827 -547}\special{pa 827 -555}\special{fp}\special{pa 827 -587}\special{pa 827 -595}\special{fp}%
\special{pa 827 -626}\special{pa 827 -634}\special{fp}\special{pa 827 -665}\special{pa 827 -673}\special{fp}%
\special{pa 827 -705}\special{pa 827 -713}\special{fp}\special{pa 827 -744}\special{pa 827 -752}\special{fp}%
\special{pa 827 -783}\special{pa 827 -791}\special{fp}\special{pa 827 -823}\special{pa 827 -831}\special{fp}%
\special{pa 827 -862}\special{pa 827 -870}\special{fp}\special{pa 827 -902}\special{pa 827 -910}\special{fp}%
\special{pa 827 -941}\special{pa 827 -949}\special{fp}\special{pa 827 -980}\special{pa 827 -988}\special{fp}%
\special{pa 827 -1020}\special{pa 827 -1028}\special{fp}\special{pa 827 -1059}\special{pa 827 -1067}\special{fp}%
\special{pa 827 -1098}\special{pa 827 -1106}\special{fp}\special{pn 8}%
\special{pn 8}%
\special{pa -4 -1102}\special{pa 4 -1102}\special{fp}\special{pa 35 -1102}\special{pa 43 -1102}\special{fp}%
\special{pa 75 -1102}\special{pa 83 -1102}\special{fp}\special{pa 114 -1102}\special{pa 122 -1102}\special{fp}%
\special{pa 153 -1102}\special{pa 161 -1102}\special{fp}\special{pa 193 -1102}\special{pa 201 -1102}\special{fp}%
\special{pa 232 -1102}\special{pa 240 -1102}\special{fp}\special{pa 272 -1102}\special{pa 280 -1102}\special{fp}%
\special{pa 311 -1102}\special{pa 319 -1102}\special{fp}\special{pa 350 -1102}\special{pa 358 -1102}\special{fp}%
\special{pa 390 -1102}\special{pa 398 -1102}\special{fp}\special{pa 429 -1102}\special{pa 437 -1102}\special{fp}%
\special{pa 468 -1102}\special{pa 476 -1102}\special{fp}\special{pa 508 -1102}\special{pa 516 -1102}\special{fp}%
\special{pa 547 -1102}\special{pa 555 -1102}\special{fp}\special{pa 587 -1102}\special{pa 595 -1102}\special{fp}%
\special{pa 626 -1102}\special{pa 634 -1102}\special{fp}\special{pa 665 -1102}\special{pa 673 -1102}\special{fp}%
\special{pa 705 -1102}\special{pa 713 -1102}\special{fp}\special{pa 744 -1102}\special{pa 752 -1102}\special{fp}%
\special{pa 783 -1102}\special{pa 791 -1102}\special{fp}\special{pa 823 -1102}\special{pa 831 -1102}\special{fp}%
\special{pa 862 -1102}\special{pa 870 -1102}\special{fp}\special{pa 902 -1102}\special{pa 910 -1102}\special{fp}%
\special{pa 941 -1102}\special{pa 949 -1102}\special{fp}\special{pa 980 -1102}\special{pa 988 -1102}\special{fp}%
\special{pa 1020 -1102}\special{pa 1028 -1102}\special{fp}\special{pa 1059 -1102}\special{pa 1067 -1102}\special{fp}%
\special{pa 1098 -1102}\special{pa 1106 -1102}\special{fp}\special{pn 8}%
\special{pn 8}%
\special{pa 1102 4}\special{pa 1102 -4}\special{fp}\special{pa 1102 -35}\special{pa 1102 -43}\special{fp}%
\special{pa 1102 -75}\special{pa 1102 -83}\special{fp}\special{pa 1102 -114}\special{pa 1102 -122}\special{fp}%
\special{pa 1102 -153}\special{pa 1102 -161}\special{fp}\special{pa 1102 -193}\special{pa 1102 -201}\special{fp}%
\special{pa 1102 -232}\special{pa 1102 -240}\special{fp}\special{pa 1102 -272}\special{pa 1102 -280}\special{fp}%
\special{pa 1102 -311}\special{pa 1102 -319}\special{fp}\special{pa 1102 -350}\special{pa 1102 -358}\special{fp}%
\special{pa 1102 -390}\special{pa 1102 -398}\special{fp}\special{pa 1102 -429}\special{pa 1102 -437}\special{fp}%
\special{pa 1102 -468}\special{pa 1102 -476}\special{fp}\special{pa 1102 -508}\special{pa 1102 -516}\special{fp}%
\special{pa 1102 -547}\special{pa 1102 -555}\special{fp}\special{pa 1102 -587}\special{pa 1102 -595}\special{fp}%
\special{pa 1102 -626}\special{pa 1102 -634}\special{fp}\special{pa 1102 -665}\special{pa 1102 -673}\special{fp}%
\special{pa 1102 -705}\special{pa 1102 -713}\special{fp}\special{pa 1102 -744}\special{pa 1102 -752}\special{fp}%
\special{pa 1102 -783}\special{pa 1102 -791}\special{fp}\special{pa 1102 -823}\special{pa 1102 -831}\special{fp}%
\special{pa 1102 -862}\special{pa 1102 -870}\special{fp}\special{pa 1102 -902}\special{pa 1102 -910}\special{fp}%
\special{pa 1102 -941}\special{pa 1102 -949}\special{fp}\special{pa 1102 -980}\special{pa 1102 -988}\special{fp}%
\special{pa 1102 -1020}\special{pa 1102 -1028}\special{fp}\special{pa 1102 -1059}\special{pa 1102 -1067}\special{fp}%
\special{pa 1102 -1098}\special{pa 1102 -1106}\special{fp}\special{pn 8}%
{%
\color[cmyk]{0,1,1,0}%
\special{pn 16}%
\special{pa     0  -551}\special{pa   551  -551}\special{pa  1102  -551}%
\special{fp}%
\special{pn 8}%
}%
\special{pa 1102 -551}\special{pa 1102 -514}\special{fp}\special{pa 1102 -478}\special{pa 1102 -441}\special{fp}%
\special{pa 1102 -404}\special{pa 1102 -367}\special{fp}\special{pa 1102 -331}\special{pa 1102 -294}\special{fp}%
\special{pa 1102 -257}\special{pa 1102 -220}\special{fp}\special{pa 1102 -184}\special{pa 1102 -147}\special{fp}%
\special{pa 1102 -110}\special{pa 1102 -73}\special{fp}\special{pa 1102 -37}\special{pa 1102 0}\special{fp}%
%
%
\special{pa 551 -551}\special{pa 551 -514}\special{fp}\special{pa 551 -478}\special{pa 551 -441}\special{fp}%
\special{pa 551 -404}\special{pa 551 -367}\special{fp}\special{pa 551 -331}\special{pa 551 -294}\special{fp}%
\special{pa 551 -257}\special{pa 551 -220}\special{fp}\special{pa 551 -184}\special{pa 551 -147}\special{fp}%
\special{pa 551 -110}\special{pa 551 -73}\special{fp}\special{pa 551 -37}\special{pa 551 0}\special{fp}%
%
%
\special{pa   276   -20}\special{pa   276    20}%
\special{fp}%
\settowidth{\Width}{$1$}\setlength{\Width}{-0.5\Width}%
\settoheight{\Height}{$1$}\settodepth{\Depth}{$1$}\setlength{\Height}{-\Height}%
\put(1.0000000,-0.1428571){\hspace*{\Width}\raisebox{\Height}{$1$}}%
%
\special{pa   551   -20}\special{pa   551    20}%
\special{fp}%
\settowidth{\Width}{$2$}\setlength{\Width}{-0.5\Width}%
\settoheight{\Height}{$2$}\settodepth{\Depth}{$2$}\setlength{\Height}{-\Height}%
\put(2.0000000,-0.1428571){\hspace*{\Width}\raisebox{\Height}{$2$}}%
%
\special{pa   827   -20}\special{pa   827    20}%
\special{fp}%
\settowidth{\Width}{$3$}\setlength{\Width}{-0.5\Width}%
\settoheight{\Height}{$3$}\settodepth{\Depth}{$3$}\setlength{\Height}{-\Height}%
\put(3.0000000,-0.1428571){\hspace*{\Width}\raisebox{\Height}{$3$}}%
%
\special{pa  1102   -20}\special{pa  1102    20}%
\special{fp}%
\settowidth{\Width}{$4$}\setlength{\Width}{-0.5\Width}%
\settoheight{\Height}{$4$}\settodepth{\Depth}{$4$}\setlength{\Height}{-\Height}%
\put(4.0000000,-0.1428571){\hspace*{\Width}\raisebox{\Height}{$4$}}%
%
\special{pa    20  -276}\special{pa   -20  -276}%
\special{fp}%
\settowidth{\Width}{$1$}\setlength{\Width}{-1\Width}%
\settoheight{\Height}{$1$}\settodepth{\Depth}{$1$}\setlength{\Height}{-0.5\Height}\setlength{\Depth}{0.5\Depth}\addtolength{\Height}{\Depth}%
\put(-0.1428571,1.0000000){\hspace*{\Width}\raisebox{\Height}{$1$}}%
%
\special{pa    20  -551}\special{pa   -20  -551}%
\special{fp}%
\settowidth{\Width}{$2$}\setlength{\Width}{-1\Width}%
\settoheight{\Height}{$2$}\settodepth{\Depth}{$2$}\setlength{\Height}{-0.5\Height}\setlength{\Depth}{0.5\Depth}\addtolength{\Height}{\Depth}%
\put(-0.1428571,2.0000000){\hspace*{\Width}\raisebox{\Height}{$2$}}%
%
\special{pa    20  -827}\special{pa   -20  -827}%
\special{fp}%
\settowidth{\Width}{$3$}\setlength{\Width}{-1\Width}%
\settoheight{\Height}{$3$}\settodepth{\Depth}{$3$}\setlength{\Height}{-0.5\Height}\setlength{\Depth}{0.5\Depth}\addtolength{\Height}{\Depth}%
\put(-0.1428571,3.0000000){\hspace*{\Width}\raisebox{\Height}{$3$}}%
%
\special{pa    20 -1102}\special{pa   -20 -1102}%
\special{fp}%
\settowidth{\Width}{$4$}\setlength{\Width}{-1\Width}%
\settoheight{\Height}{$4$}\settodepth{\Depth}{$4$}\setlength{\Height}{-0.5\Height}\setlength{\Depth}{0.5\Depth}\addtolength{\Height}{\Depth}%
\put(-0.1428571,4.0000000){\hspace*{\Width}\raisebox{\Height}{$4$}}%
%
\special{pa  -138    -0}\special{pa  1240    -0}%
\special{fp}%
\special{pa     0   138}\special{pa     0 -1240}%
\special{fp}%
\settowidth{\Width}{$x$}\setlength{\Width}{0\Width}%
\settoheight{\Height}{$x$}\settodepth{\Depth}{$x$}\setlength{\Height}{-0.5\Height}\setlength{\Depth}{0.5\Depth}\addtolength{\Height}{\Depth}%
\put(4.5714286,0.0000000){\hspace*{\Width}\raisebox{\Height}{$x$}}%
%
\settowidth{\Width}{$y$}\setlength{\Width}{-0.5\Width}%
\settoheight{\Height}{$y$}\settodepth{\Depth}{$y$}\setlength{\Height}{\Depth}%
\put(0.0000000,4.5714286){\hspace*{\Width}\raisebox{\Height}{$y$}}%
%
\settowidth{\Width}{O}\setlength{\Width}{-1\Width}%
\settoheight{\Height}{O}\settodepth{\Depth}{O}\setlength{\Height}{-\Height}%
\put(-0.0714286,-0.0714286){\hspace*{\Width}\raisebox{\Height}{O}}%
%
\end{picture}}%}}
\putnotese{45}{30}{\normalsize (2)}
\putnotese{50}{33}{\scalebox{0.8}{%%% /Users/takatoosetsuo/Dropbox/2021polytech/203/fig/biseki2.tex 
%%% Generator=presen203.cdy 
{\unitlength=7mm%
\begin{picture}%
(5,5)(-0.5,-0.5)%
\special{pn 8}%
%
\Large\bf\boldmath%
\normalsize%
\special{pn 8}%
\special{pa -4 -276}\special{pa 4 -276}\special{fp}\special{pa 35 -276}\special{pa 43 -276}\special{fp}%
\special{pa 75 -276}\special{pa 83 -276}\special{fp}\special{pa 114 -276}\special{pa 122 -276}\special{fp}%
\special{pa 153 -276}\special{pa 161 -276}\special{fp}\special{pa 193 -276}\special{pa 201 -276}\special{fp}%
\special{pa 232 -276}\special{pa 240 -276}\special{fp}\special{pa 272 -276}\special{pa 280 -276}\special{fp}%
\special{pa 311 -276}\special{pa 319 -276}\special{fp}\special{pa 350 -276}\special{pa 358 -276}\special{fp}%
\special{pa 390 -276}\special{pa 398 -276}\special{fp}\special{pa 429 -276}\special{pa 437 -276}\special{fp}%
\special{pa 468 -276}\special{pa 476 -276}\special{fp}\special{pa 508 -276}\special{pa 516 -276}\special{fp}%
\special{pa 547 -276}\special{pa 555 -276}\special{fp}\special{pa 587 -276}\special{pa 595 -276}\special{fp}%
\special{pa 626 -276}\special{pa 634 -276}\special{fp}\special{pa 665 -276}\special{pa 673 -276}\special{fp}%
\special{pa 705 -276}\special{pa 713 -276}\special{fp}\special{pa 744 -276}\special{pa 752 -276}\special{fp}%
\special{pa 783 -276}\special{pa 791 -276}\special{fp}\special{pa 823 -276}\special{pa 831 -276}\special{fp}%
\special{pa 862 -276}\special{pa 870 -276}\special{fp}\special{pa 902 -276}\special{pa 910 -276}\special{fp}%
\special{pa 941 -276}\special{pa 949 -276}\special{fp}\special{pa 980 -276}\special{pa 988 -276}\special{fp}%
\special{pa 1020 -276}\special{pa 1028 -276}\special{fp}\special{pa 1059 -276}\special{pa 1067 -276}\special{fp}%
\special{pa 1098 -276}\special{pa 1106 -276}\special{fp}\special{pn 8}%
\special{pn 8}%
\special{pa 276 4}\special{pa 276 -4}\special{fp}\special{pa 276 -35}\special{pa 276 -43}\special{fp}%
\special{pa 276 -75}\special{pa 276 -83}\special{fp}\special{pa 276 -114}\special{pa 276 -122}\special{fp}%
\special{pa 276 -153}\special{pa 276 -161}\special{fp}\special{pa 276 -193}\special{pa 276 -201}\special{fp}%
\special{pa 276 -232}\special{pa 276 -240}\special{fp}\special{pa 276 -272}\special{pa 276 -280}\special{fp}%
\special{pa 276 -311}\special{pa 276 -319}\special{fp}\special{pa 276 -350}\special{pa 276 -358}\special{fp}%
\special{pa 276 -390}\special{pa 276 -398}\special{fp}\special{pa 276 -429}\special{pa 276 -437}\special{fp}%
\special{pa 276 -468}\special{pa 276 -476}\special{fp}\special{pa 276 -508}\special{pa 276 -516}\special{fp}%
\special{pa 276 -547}\special{pa 276 -555}\special{fp}\special{pa 276 -587}\special{pa 276 -595}\special{fp}%
\special{pa 276 -626}\special{pa 276 -634}\special{fp}\special{pa 276 -665}\special{pa 276 -673}\special{fp}%
\special{pa 276 -705}\special{pa 276 -713}\special{fp}\special{pa 276 -744}\special{pa 276 -752}\special{fp}%
\special{pa 276 -783}\special{pa 276 -791}\special{fp}\special{pa 276 -823}\special{pa 276 -831}\special{fp}%
\special{pa 276 -862}\special{pa 276 -870}\special{fp}\special{pa 276 -902}\special{pa 276 -910}\special{fp}%
\special{pa 276 -941}\special{pa 276 -949}\special{fp}\special{pa 276 -980}\special{pa 276 -988}\special{fp}%
\special{pa 276 -1020}\special{pa 276 -1028}\special{fp}\special{pa 276 -1059}\special{pa 276 -1067}\special{fp}%
\special{pa 276 -1098}\special{pa 276 -1106}\special{fp}\special{pn 8}%
\special{pn 8}%
\special{pa -4 -551}\special{pa 4 -551}\special{fp}\special{pa 35 -551}\special{pa 43 -551}\special{fp}%
\special{pa 75 -551}\special{pa 83 -551}\special{fp}\special{pa 114 -551}\special{pa 122 -551}\special{fp}%
\special{pa 153 -551}\special{pa 161 -551}\special{fp}\special{pa 193 -551}\special{pa 201 -551}\special{fp}%
\special{pa 232 -551}\special{pa 240 -551}\special{fp}\special{pa 272 -551}\special{pa 280 -551}\special{fp}%
\special{pa 311 -551}\special{pa 319 -551}\special{fp}\special{pa 350 -551}\special{pa 358 -551}\special{fp}%
\special{pa 390 -551}\special{pa 398 -551}\special{fp}\special{pa 429 -551}\special{pa 437 -551}\special{fp}%
\special{pa 468 -551}\special{pa 476 -551}\special{fp}\special{pa 508 -551}\special{pa 516 -551}\special{fp}%
\special{pa 547 -551}\special{pa 555 -551}\special{fp}\special{pa 587 -551}\special{pa 595 -551}\special{fp}%
\special{pa 626 -551}\special{pa 634 -551}\special{fp}\special{pa 665 -551}\special{pa 673 -551}\special{fp}%
\special{pa 705 -551}\special{pa 713 -551}\special{fp}\special{pa 744 -551}\special{pa 752 -551}\special{fp}%
\special{pa 783 -551}\special{pa 791 -551}\special{fp}\special{pa 823 -551}\special{pa 831 -551}\special{fp}%
\special{pa 862 -551}\special{pa 870 -551}\special{fp}\special{pa 902 -551}\special{pa 910 -551}\special{fp}%
\special{pa 941 -551}\special{pa 949 -551}\special{fp}\special{pa 980 -551}\special{pa 988 -551}\special{fp}%
\special{pa 1020 -551}\special{pa 1028 -551}\special{fp}\special{pa 1059 -551}\special{pa 1067 -551}\special{fp}%
\special{pa 1098 -551}\special{pa 1106 -551}\special{fp}\special{pn 8}%
\special{pn 8}%
\special{pa 551 4}\special{pa 551 -4}\special{fp}\special{pa 551 -35}\special{pa 551 -43}\special{fp}%
\special{pa 551 -75}\special{pa 551 -83}\special{fp}\special{pa 551 -114}\special{pa 551 -122}\special{fp}%
\special{pa 551 -153}\special{pa 551 -161}\special{fp}\special{pa 551 -193}\special{pa 551 -201}\special{fp}%
\special{pa 551 -232}\special{pa 551 -240}\special{fp}\special{pa 551 -272}\special{pa 551 -280}\special{fp}%
\special{pa 551 -311}\special{pa 551 -319}\special{fp}\special{pa 551 -350}\special{pa 551 -358}\special{fp}%
\special{pa 551 -390}\special{pa 551 -398}\special{fp}\special{pa 551 -429}\special{pa 551 -437}\special{fp}%
\special{pa 551 -468}\special{pa 551 -476}\special{fp}\special{pa 551 -508}\special{pa 551 -516}\special{fp}%
\special{pa 551 -547}\special{pa 551 -555}\special{fp}\special{pa 551 -587}\special{pa 551 -595}\special{fp}%
\special{pa 551 -626}\special{pa 551 -634}\special{fp}\special{pa 551 -665}\special{pa 551 -673}\special{fp}%
\special{pa 551 -705}\special{pa 551 -713}\special{fp}\special{pa 551 -744}\special{pa 551 -752}\special{fp}%
\special{pa 551 -783}\special{pa 551 -791}\special{fp}\special{pa 551 -823}\special{pa 551 -831}\special{fp}%
\special{pa 551 -862}\special{pa 551 -870}\special{fp}\special{pa 551 -902}\special{pa 551 -910}\special{fp}%
\special{pa 551 -941}\special{pa 551 -949}\special{fp}\special{pa 551 -980}\special{pa 551 -988}\special{fp}%
\special{pa 551 -1020}\special{pa 551 -1028}\special{fp}\special{pa 551 -1059}\special{pa 551 -1067}\special{fp}%
\special{pa 551 -1098}\special{pa 551 -1106}\special{fp}\special{pn 8}%
\special{pn 8}%
\special{pa -4 -827}\special{pa 4 -827}\special{fp}\special{pa 35 -827}\special{pa 43 -827}\special{fp}%
\special{pa 75 -827}\special{pa 83 -827}\special{fp}\special{pa 114 -827}\special{pa 122 -827}\special{fp}%
\special{pa 153 -827}\special{pa 161 -827}\special{fp}\special{pa 193 -827}\special{pa 201 -827}\special{fp}%
\special{pa 232 -827}\special{pa 240 -827}\special{fp}\special{pa 272 -827}\special{pa 280 -827}\special{fp}%
\special{pa 311 -827}\special{pa 319 -827}\special{fp}\special{pa 350 -827}\special{pa 358 -827}\special{fp}%
\special{pa 390 -827}\special{pa 398 -827}\special{fp}\special{pa 429 -827}\special{pa 437 -827}\special{fp}%
\special{pa 468 -827}\special{pa 476 -827}\special{fp}\special{pa 508 -827}\special{pa 516 -827}\special{fp}%
\special{pa 547 -827}\special{pa 555 -827}\special{fp}\special{pa 587 -827}\special{pa 595 -827}\special{fp}%
\special{pa 626 -827}\special{pa 634 -827}\special{fp}\special{pa 665 -827}\special{pa 673 -827}\special{fp}%
\special{pa 705 -827}\special{pa 713 -827}\special{fp}\special{pa 744 -827}\special{pa 752 -827}\special{fp}%
\special{pa 783 -827}\special{pa 791 -827}\special{fp}\special{pa 823 -827}\special{pa 831 -827}\special{fp}%
\special{pa 862 -827}\special{pa 870 -827}\special{fp}\special{pa 902 -827}\special{pa 910 -827}\special{fp}%
\special{pa 941 -827}\special{pa 949 -827}\special{fp}\special{pa 980 -827}\special{pa 988 -827}\special{fp}%
\special{pa 1020 -827}\special{pa 1028 -827}\special{fp}\special{pa 1059 -827}\special{pa 1067 -827}\special{fp}%
\special{pa 1098 -827}\special{pa 1106 -827}\special{fp}\special{pn 8}%
\special{pn 8}%
\special{pa 827 4}\special{pa 827 -4}\special{fp}\special{pa 827 -35}\special{pa 827 -43}\special{fp}%
\special{pa 827 -75}\special{pa 827 -83}\special{fp}\special{pa 827 -114}\special{pa 827 -122}\special{fp}%
\special{pa 827 -153}\special{pa 827 -161}\special{fp}\special{pa 827 -193}\special{pa 827 -201}\special{fp}%
\special{pa 827 -232}\special{pa 827 -240}\special{fp}\special{pa 827 -272}\special{pa 827 -280}\special{fp}%
\special{pa 827 -311}\special{pa 827 -319}\special{fp}\special{pa 827 -350}\special{pa 827 -358}\special{fp}%
\special{pa 827 -390}\special{pa 827 -398}\special{fp}\special{pa 827 -429}\special{pa 827 -437}\special{fp}%
\special{pa 827 -468}\special{pa 827 -476}\special{fp}\special{pa 827 -508}\special{pa 827 -516}\special{fp}%
\special{pa 827 -547}\special{pa 827 -555}\special{fp}\special{pa 827 -587}\special{pa 827 -595}\special{fp}%
\special{pa 827 -626}\special{pa 827 -634}\special{fp}\special{pa 827 -665}\special{pa 827 -673}\special{fp}%
\special{pa 827 -705}\special{pa 827 -713}\special{fp}\special{pa 827 -744}\special{pa 827 -752}\special{fp}%
\special{pa 827 -783}\special{pa 827 -791}\special{fp}\special{pa 827 -823}\special{pa 827 -831}\special{fp}%
\special{pa 827 -862}\special{pa 827 -870}\special{fp}\special{pa 827 -902}\special{pa 827 -910}\special{fp}%
\special{pa 827 -941}\special{pa 827 -949}\special{fp}\special{pa 827 -980}\special{pa 827 -988}\special{fp}%
\special{pa 827 -1020}\special{pa 827 -1028}\special{fp}\special{pa 827 -1059}\special{pa 827 -1067}\special{fp}%
\special{pa 827 -1098}\special{pa 827 -1106}\special{fp}\special{pn 8}%
\special{pn 8}%
\special{pa -4 -1102}\special{pa 4 -1102}\special{fp}\special{pa 35 -1102}\special{pa 43 -1102}\special{fp}%
\special{pa 75 -1102}\special{pa 83 -1102}\special{fp}\special{pa 114 -1102}\special{pa 122 -1102}\special{fp}%
\special{pa 153 -1102}\special{pa 161 -1102}\special{fp}\special{pa 193 -1102}\special{pa 201 -1102}\special{fp}%
\special{pa 232 -1102}\special{pa 240 -1102}\special{fp}\special{pa 272 -1102}\special{pa 280 -1102}\special{fp}%
\special{pa 311 -1102}\special{pa 319 -1102}\special{fp}\special{pa 350 -1102}\special{pa 358 -1102}\special{fp}%
\special{pa 390 -1102}\special{pa 398 -1102}\special{fp}\special{pa 429 -1102}\special{pa 437 -1102}\special{fp}%
\special{pa 468 -1102}\special{pa 476 -1102}\special{fp}\special{pa 508 -1102}\special{pa 516 -1102}\special{fp}%
\special{pa 547 -1102}\special{pa 555 -1102}\special{fp}\special{pa 587 -1102}\special{pa 595 -1102}\special{fp}%
\special{pa 626 -1102}\special{pa 634 -1102}\special{fp}\special{pa 665 -1102}\special{pa 673 -1102}\special{fp}%
\special{pa 705 -1102}\special{pa 713 -1102}\special{fp}\special{pa 744 -1102}\special{pa 752 -1102}\special{fp}%
\special{pa 783 -1102}\special{pa 791 -1102}\special{fp}\special{pa 823 -1102}\special{pa 831 -1102}\special{fp}%
\special{pa 862 -1102}\special{pa 870 -1102}\special{fp}\special{pa 902 -1102}\special{pa 910 -1102}\special{fp}%
\special{pa 941 -1102}\special{pa 949 -1102}\special{fp}\special{pa 980 -1102}\special{pa 988 -1102}\special{fp}%
\special{pa 1020 -1102}\special{pa 1028 -1102}\special{fp}\special{pa 1059 -1102}\special{pa 1067 -1102}\special{fp}%
\special{pa 1098 -1102}\special{pa 1106 -1102}\special{fp}\special{pn 8}%
\special{pn 8}%
\special{pa 1102 4}\special{pa 1102 -4}\special{fp}\special{pa 1102 -35}\special{pa 1102 -43}\special{fp}%
\special{pa 1102 -75}\special{pa 1102 -83}\special{fp}\special{pa 1102 -114}\special{pa 1102 -122}\special{fp}%
\special{pa 1102 -153}\special{pa 1102 -161}\special{fp}\special{pa 1102 -193}\special{pa 1102 -201}\special{fp}%
\special{pa 1102 -232}\special{pa 1102 -240}\special{fp}\special{pa 1102 -272}\special{pa 1102 -280}\special{fp}%
\special{pa 1102 -311}\special{pa 1102 -319}\special{fp}\special{pa 1102 -350}\special{pa 1102 -358}\special{fp}%
\special{pa 1102 -390}\special{pa 1102 -398}\special{fp}\special{pa 1102 -429}\special{pa 1102 -437}\special{fp}%
\special{pa 1102 -468}\special{pa 1102 -476}\special{fp}\special{pa 1102 -508}\special{pa 1102 -516}\special{fp}%
\special{pa 1102 -547}\special{pa 1102 -555}\special{fp}\special{pa 1102 -587}\special{pa 1102 -595}\special{fp}%
\special{pa 1102 -626}\special{pa 1102 -634}\special{fp}\special{pa 1102 -665}\special{pa 1102 -673}\special{fp}%
\special{pa 1102 -705}\special{pa 1102 -713}\special{fp}\special{pa 1102 -744}\special{pa 1102 -752}\special{fp}%
\special{pa 1102 -783}\special{pa 1102 -791}\special{fp}\special{pa 1102 -823}\special{pa 1102 -831}\special{fp}%
\special{pa 1102 -862}\special{pa 1102 -870}\special{fp}\special{pa 1102 -902}\special{pa 1102 -910}\special{fp}%
\special{pa 1102 -941}\special{pa 1102 -949}\special{fp}\special{pa 1102 -980}\special{pa 1102 -988}\special{fp}%
\special{pa 1102 -1020}\special{pa 1102 -1028}\special{fp}\special{pa 1102 -1059}\special{pa 1102 -1067}\special{fp}%
\special{pa 1102 -1098}\special{pa 1102 -1106}\special{fp}\special{pn 8}%
{%
\color[cmyk]{0,1,1,0}%
\special{pn 16}%
\special{pa     0  -276}\special{pa   551  -551}\special{pa  1102  -827}%
\special{fp}%
\special{pn 8}%
}%
\special{pa 1102 -827}\special{pa 1102 -787}\special{fp}\special{pa 1102 -748}\special{pa 1102 -709}\special{fp}%
\special{pa 1102 -669}\special{pa 1102 -630}\special{fp}\special{pa 1102 -591}\special{pa 1102 -551}\special{fp}%
\special{pa 1102 -512}\special{pa 1102 -472}\special{fp}\special{pa 1102 -433}\special{pa 1102 -394}\special{fp}%
\special{pa 1102 -354}\special{pa 1102 -315}\special{fp}\special{pa 1102 -276}\special{pa 1102 -236}\special{fp}%
\special{pa 1102 -197}\special{pa 1102 -157}\special{fp}\special{pa 1102 -118}\special{pa 1102 -79}\special{fp}%
\special{pa 1102 -39}\special{pa 1102 0}\special{fp}%
%
\special{pa 551 -551}\special{pa 551 -514}\special{fp}\special{pa 551 -478}\special{pa 551 -441}\special{fp}%
\special{pa 551 -404}\special{pa 551 -367}\special{fp}\special{pa 551 -331}\special{pa 551 -294}\special{fp}%
\special{pa 551 -257}\special{pa 551 -220}\special{fp}\special{pa 551 -184}\special{pa 551 -147}\special{fp}%
\special{pa 551 -110}\special{pa 551 -73}\special{fp}\special{pa 551 -37}\special{pa 551 0}\special{fp}%
%
%
\special{pa   276   -20}\special{pa   276    20}%
\special{fp}%
\settowidth{\Width}{$1$}\setlength{\Width}{-0.5\Width}%
\settoheight{\Height}{$1$}\settodepth{\Depth}{$1$}\setlength{\Height}{-\Height}%
\put(1.0000000,-0.1428571){\hspace*{\Width}\raisebox{\Height}{$1$}}%
%
\special{pa   551   -20}\special{pa   551    20}%
\special{fp}%
\settowidth{\Width}{$2$}\setlength{\Width}{-0.5\Width}%
\settoheight{\Height}{$2$}\settodepth{\Depth}{$2$}\setlength{\Height}{-\Height}%
\put(2.0000000,-0.1428571){\hspace*{\Width}\raisebox{\Height}{$2$}}%
%
\special{pa   827   -20}\special{pa   827    20}%
\special{fp}%
\settowidth{\Width}{$3$}\setlength{\Width}{-0.5\Width}%
\settoheight{\Height}{$3$}\settodepth{\Depth}{$3$}\setlength{\Height}{-\Height}%
\put(3.0000000,-0.1428571){\hspace*{\Width}\raisebox{\Height}{$3$}}%
%
\special{pa  1102   -20}\special{pa  1102    20}%
\special{fp}%
\settowidth{\Width}{$4$}\setlength{\Width}{-0.5\Width}%
\settoheight{\Height}{$4$}\settodepth{\Depth}{$4$}\setlength{\Height}{-\Height}%
\put(4.0000000,-0.1428571){\hspace*{\Width}\raisebox{\Height}{$4$}}%
%
\special{pa    20  -276}\special{pa   -20  -276}%
\special{fp}%
\settowidth{\Width}{$1$}\setlength{\Width}{-1\Width}%
\settoheight{\Height}{$1$}\settodepth{\Depth}{$1$}\setlength{\Height}{-0.5\Height}\setlength{\Depth}{0.5\Depth}\addtolength{\Height}{\Depth}%
\put(-0.1428571,1.0000000){\hspace*{\Width}\raisebox{\Height}{$1$}}%
%
\special{pa    20  -551}\special{pa   -20  -551}%
\special{fp}%
\settowidth{\Width}{$2$}\setlength{\Width}{-1\Width}%
\settoheight{\Height}{$2$}\settodepth{\Depth}{$2$}\setlength{\Height}{-0.5\Height}\setlength{\Depth}{0.5\Depth}\addtolength{\Height}{\Depth}%
\put(-0.1428571,2.0000000){\hspace*{\Width}\raisebox{\Height}{$2$}}%
%
\special{pa    20  -827}\special{pa   -20  -827}%
\special{fp}%
\settowidth{\Width}{$3$}\setlength{\Width}{-1\Width}%
\settoheight{\Height}{$3$}\settodepth{\Depth}{$3$}\setlength{\Height}{-0.5\Height}\setlength{\Depth}{0.5\Depth}\addtolength{\Height}{\Depth}%
\put(-0.1428571,3.0000000){\hspace*{\Width}\raisebox{\Height}{$3$}}%
%
\special{pa    20 -1102}\special{pa   -20 -1102}%
\special{fp}%
\settowidth{\Width}{$4$}\setlength{\Width}{-1\Width}%
\settoheight{\Height}{$4$}\settodepth{\Depth}{$4$}\setlength{\Height}{-0.5\Height}\setlength{\Depth}{0.5\Depth}\addtolength{\Height}{\Depth}%
\put(-0.1428571,4.0000000){\hspace*{\Width}\raisebox{\Height}{$4$}}%
%
\special{pa  -138    -0}\special{pa  1240    -0}%
\special{fp}%
\special{pa     0   138}\special{pa     0 -1240}%
\special{fp}%
\settowidth{\Width}{$x$}\setlength{\Width}{0\Width}%
\settoheight{\Height}{$x$}\settodepth{\Depth}{$x$}\setlength{\Height}{-0.5\Height}\setlength{\Depth}{0.5\Depth}\addtolength{\Height}{\Depth}%
\put(4.5714286,0.0000000){\hspace*{\Width}\raisebox{\Height}{$x$}}%
%
\settowidth{\Width}{$y$}\setlength{\Width}{-0.5\Width}%
\settoheight{\Height}{$y$}\settodepth{\Depth}{$y$}\setlength{\Height}{\Depth}%
\put(0.0000000,4.5714286){\hspace*{\Width}\raisebox{\Height}{$y$}}%
%
\settowidth{\Width}{O}\setlength{\Width}{-1\Width}%
\settoheight{\Height}{O}\settodepth{\Depth}{O}\setlength{\Height}{-\Height}%
\put(-0.0714286,-0.0714286){\hspace*{\Width}\raisebox{\Height}{O}}%
%
\end{picture}}%}}
\putnotese{85}{30}{\normalsize (3)}
\putnotese{90}{33}{\scalebox{0.8}{%%% /Users/takatoosetsuo/Dropbox/2021polytech/203/fig/biseki3.tex 
%%% Generator=presen203.cdy 
{\unitlength=7mm%
\begin{picture}%
(5,5)(-0.5,-0.5)%
\special{pn 8}%
%
\Large\bf\boldmath%
\normalsize%
\special{pn 8}%
\special{pa -4 -276}\special{pa 4 -276}\special{fp}\special{pa 35 -276}\special{pa 43 -276}\special{fp}%
\special{pa 75 -276}\special{pa 83 -276}\special{fp}\special{pa 114 -276}\special{pa 122 -276}\special{fp}%
\special{pa 153 -276}\special{pa 161 -276}\special{fp}\special{pa 193 -276}\special{pa 201 -276}\special{fp}%
\special{pa 232 -276}\special{pa 240 -276}\special{fp}\special{pa 272 -276}\special{pa 280 -276}\special{fp}%
\special{pa 311 -276}\special{pa 319 -276}\special{fp}\special{pa 350 -276}\special{pa 358 -276}\special{fp}%
\special{pa 390 -276}\special{pa 398 -276}\special{fp}\special{pa 429 -276}\special{pa 437 -276}\special{fp}%
\special{pa 468 -276}\special{pa 476 -276}\special{fp}\special{pa 508 -276}\special{pa 516 -276}\special{fp}%
\special{pa 547 -276}\special{pa 555 -276}\special{fp}\special{pa 587 -276}\special{pa 595 -276}\special{fp}%
\special{pa 626 -276}\special{pa 634 -276}\special{fp}\special{pa 665 -276}\special{pa 673 -276}\special{fp}%
\special{pa 705 -276}\special{pa 713 -276}\special{fp}\special{pa 744 -276}\special{pa 752 -276}\special{fp}%
\special{pa 783 -276}\special{pa 791 -276}\special{fp}\special{pa 823 -276}\special{pa 831 -276}\special{fp}%
\special{pa 862 -276}\special{pa 870 -276}\special{fp}\special{pa 902 -276}\special{pa 910 -276}\special{fp}%
\special{pa 941 -276}\special{pa 949 -276}\special{fp}\special{pa 980 -276}\special{pa 988 -276}\special{fp}%
\special{pa 1020 -276}\special{pa 1028 -276}\special{fp}\special{pa 1059 -276}\special{pa 1067 -276}\special{fp}%
\special{pa 1098 -276}\special{pa 1106 -276}\special{fp}\special{pn 8}%
\special{pn 8}%
\special{pa 276 4}\special{pa 276 -4}\special{fp}\special{pa 276 -35}\special{pa 276 -43}\special{fp}%
\special{pa 276 -75}\special{pa 276 -83}\special{fp}\special{pa 276 -114}\special{pa 276 -122}\special{fp}%
\special{pa 276 -153}\special{pa 276 -161}\special{fp}\special{pa 276 -193}\special{pa 276 -201}\special{fp}%
\special{pa 276 -232}\special{pa 276 -240}\special{fp}\special{pa 276 -272}\special{pa 276 -280}\special{fp}%
\special{pa 276 -311}\special{pa 276 -319}\special{fp}\special{pa 276 -350}\special{pa 276 -358}\special{fp}%
\special{pa 276 -390}\special{pa 276 -398}\special{fp}\special{pa 276 -429}\special{pa 276 -437}\special{fp}%
\special{pa 276 -468}\special{pa 276 -476}\special{fp}\special{pa 276 -508}\special{pa 276 -516}\special{fp}%
\special{pa 276 -547}\special{pa 276 -555}\special{fp}\special{pa 276 -587}\special{pa 276 -595}\special{fp}%
\special{pa 276 -626}\special{pa 276 -634}\special{fp}\special{pa 276 -665}\special{pa 276 -673}\special{fp}%
\special{pa 276 -705}\special{pa 276 -713}\special{fp}\special{pa 276 -744}\special{pa 276 -752}\special{fp}%
\special{pa 276 -783}\special{pa 276 -791}\special{fp}\special{pa 276 -823}\special{pa 276 -831}\special{fp}%
\special{pa 276 -862}\special{pa 276 -870}\special{fp}\special{pa 276 -902}\special{pa 276 -910}\special{fp}%
\special{pa 276 -941}\special{pa 276 -949}\special{fp}\special{pa 276 -980}\special{pa 276 -988}\special{fp}%
\special{pa 276 -1020}\special{pa 276 -1028}\special{fp}\special{pa 276 -1059}\special{pa 276 -1067}\special{fp}%
\special{pa 276 -1098}\special{pa 276 -1106}\special{fp}\special{pn 8}%
\special{pn 8}%
\special{pa -4 -551}\special{pa 4 -551}\special{fp}\special{pa 35 -551}\special{pa 43 -551}\special{fp}%
\special{pa 75 -551}\special{pa 83 -551}\special{fp}\special{pa 114 -551}\special{pa 122 -551}\special{fp}%
\special{pa 153 -551}\special{pa 161 -551}\special{fp}\special{pa 193 -551}\special{pa 201 -551}\special{fp}%
\special{pa 232 -551}\special{pa 240 -551}\special{fp}\special{pa 272 -551}\special{pa 280 -551}\special{fp}%
\special{pa 311 -551}\special{pa 319 -551}\special{fp}\special{pa 350 -551}\special{pa 358 -551}\special{fp}%
\special{pa 390 -551}\special{pa 398 -551}\special{fp}\special{pa 429 -551}\special{pa 437 -551}\special{fp}%
\special{pa 468 -551}\special{pa 476 -551}\special{fp}\special{pa 508 -551}\special{pa 516 -551}\special{fp}%
\special{pa 547 -551}\special{pa 555 -551}\special{fp}\special{pa 587 -551}\special{pa 595 -551}\special{fp}%
\special{pa 626 -551}\special{pa 634 -551}\special{fp}\special{pa 665 -551}\special{pa 673 -551}\special{fp}%
\special{pa 705 -551}\special{pa 713 -551}\special{fp}\special{pa 744 -551}\special{pa 752 -551}\special{fp}%
\special{pa 783 -551}\special{pa 791 -551}\special{fp}\special{pa 823 -551}\special{pa 831 -551}\special{fp}%
\special{pa 862 -551}\special{pa 870 -551}\special{fp}\special{pa 902 -551}\special{pa 910 -551}\special{fp}%
\special{pa 941 -551}\special{pa 949 -551}\special{fp}\special{pa 980 -551}\special{pa 988 -551}\special{fp}%
\special{pa 1020 -551}\special{pa 1028 -551}\special{fp}\special{pa 1059 -551}\special{pa 1067 -551}\special{fp}%
\special{pa 1098 -551}\special{pa 1106 -551}\special{fp}\special{pn 8}%
\special{pn 8}%
\special{pa 551 4}\special{pa 551 -4}\special{fp}\special{pa 551 -35}\special{pa 551 -43}\special{fp}%
\special{pa 551 -75}\special{pa 551 -83}\special{fp}\special{pa 551 -114}\special{pa 551 -122}\special{fp}%
\special{pa 551 -153}\special{pa 551 -161}\special{fp}\special{pa 551 -193}\special{pa 551 -201}\special{fp}%
\special{pa 551 -232}\special{pa 551 -240}\special{fp}\special{pa 551 -272}\special{pa 551 -280}\special{fp}%
\special{pa 551 -311}\special{pa 551 -319}\special{fp}\special{pa 551 -350}\special{pa 551 -358}\special{fp}%
\special{pa 551 -390}\special{pa 551 -398}\special{fp}\special{pa 551 -429}\special{pa 551 -437}\special{fp}%
\special{pa 551 -468}\special{pa 551 -476}\special{fp}\special{pa 551 -508}\special{pa 551 -516}\special{fp}%
\special{pa 551 -547}\special{pa 551 -555}\special{fp}\special{pa 551 -587}\special{pa 551 -595}\special{fp}%
\special{pa 551 -626}\special{pa 551 -634}\special{fp}\special{pa 551 -665}\special{pa 551 -673}\special{fp}%
\special{pa 551 -705}\special{pa 551 -713}\special{fp}\special{pa 551 -744}\special{pa 551 -752}\special{fp}%
\special{pa 551 -783}\special{pa 551 -791}\special{fp}\special{pa 551 -823}\special{pa 551 -831}\special{fp}%
\special{pa 551 -862}\special{pa 551 -870}\special{fp}\special{pa 551 -902}\special{pa 551 -910}\special{fp}%
\special{pa 551 -941}\special{pa 551 -949}\special{fp}\special{pa 551 -980}\special{pa 551 -988}\special{fp}%
\special{pa 551 -1020}\special{pa 551 -1028}\special{fp}\special{pa 551 -1059}\special{pa 551 -1067}\special{fp}%
\special{pa 551 -1098}\special{pa 551 -1106}\special{fp}\special{pn 8}%
\special{pn 8}%
\special{pa -4 -827}\special{pa 4 -827}\special{fp}\special{pa 35 -827}\special{pa 43 -827}\special{fp}%
\special{pa 75 -827}\special{pa 83 -827}\special{fp}\special{pa 114 -827}\special{pa 122 -827}\special{fp}%
\special{pa 153 -827}\special{pa 161 -827}\special{fp}\special{pa 193 -827}\special{pa 201 -827}\special{fp}%
\special{pa 232 -827}\special{pa 240 -827}\special{fp}\special{pa 272 -827}\special{pa 280 -827}\special{fp}%
\special{pa 311 -827}\special{pa 319 -827}\special{fp}\special{pa 350 -827}\special{pa 358 -827}\special{fp}%
\special{pa 390 -827}\special{pa 398 -827}\special{fp}\special{pa 429 -827}\special{pa 437 -827}\special{fp}%
\special{pa 468 -827}\special{pa 476 -827}\special{fp}\special{pa 508 -827}\special{pa 516 -827}\special{fp}%
\special{pa 547 -827}\special{pa 555 -827}\special{fp}\special{pa 587 -827}\special{pa 595 -827}\special{fp}%
\special{pa 626 -827}\special{pa 634 -827}\special{fp}\special{pa 665 -827}\special{pa 673 -827}\special{fp}%
\special{pa 705 -827}\special{pa 713 -827}\special{fp}\special{pa 744 -827}\special{pa 752 -827}\special{fp}%
\special{pa 783 -827}\special{pa 791 -827}\special{fp}\special{pa 823 -827}\special{pa 831 -827}\special{fp}%
\special{pa 862 -827}\special{pa 870 -827}\special{fp}\special{pa 902 -827}\special{pa 910 -827}\special{fp}%
\special{pa 941 -827}\special{pa 949 -827}\special{fp}\special{pa 980 -827}\special{pa 988 -827}\special{fp}%
\special{pa 1020 -827}\special{pa 1028 -827}\special{fp}\special{pa 1059 -827}\special{pa 1067 -827}\special{fp}%
\special{pa 1098 -827}\special{pa 1106 -827}\special{fp}\special{pn 8}%
\special{pn 8}%
\special{pa 827 4}\special{pa 827 -4}\special{fp}\special{pa 827 -35}\special{pa 827 -43}\special{fp}%
\special{pa 827 -75}\special{pa 827 -83}\special{fp}\special{pa 827 -114}\special{pa 827 -122}\special{fp}%
\special{pa 827 -153}\special{pa 827 -161}\special{fp}\special{pa 827 -193}\special{pa 827 -201}\special{fp}%
\special{pa 827 -232}\special{pa 827 -240}\special{fp}\special{pa 827 -272}\special{pa 827 -280}\special{fp}%
\special{pa 827 -311}\special{pa 827 -319}\special{fp}\special{pa 827 -350}\special{pa 827 -358}\special{fp}%
\special{pa 827 -390}\special{pa 827 -398}\special{fp}\special{pa 827 -429}\special{pa 827 -437}\special{fp}%
\special{pa 827 -468}\special{pa 827 -476}\special{fp}\special{pa 827 -508}\special{pa 827 -516}\special{fp}%
\special{pa 827 -547}\special{pa 827 -555}\special{fp}\special{pa 827 -587}\special{pa 827 -595}\special{fp}%
\special{pa 827 -626}\special{pa 827 -634}\special{fp}\special{pa 827 -665}\special{pa 827 -673}\special{fp}%
\special{pa 827 -705}\special{pa 827 -713}\special{fp}\special{pa 827 -744}\special{pa 827 -752}\special{fp}%
\special{pa 827 -783}\special{pa 827 -791}\special{fp}\special{pa 827 -823}\special{pa 827 -831}\special{fp}%
\special{pa 827 -862}\special{pa 827 -870}\special{fp}\special{pa 827 -902}\special{pa 827 -910}\special{fp}%
\special{pa 827 -941}\special{pa 827 -949}\special{fp}\special{pa 827 -980}\special{pa 827 -988}\special{fp}%
\special{pa 827 -1020}\special{pa 827 -1028}\special{fp}\special{pa 827 -1059}\special{pa 827 -1067}\special{fp}%
\special{pa 827 -1098}\special{pa 827 -1106}\special{fp}\special{pn 8}%
\special{pn 8}%
\special{pa -4 -1102}\special{pa 4 -1102}\special{fp}\special{pa 35 -1102}\special{pa 43 -1102}\special{fp}%
\special{pa 75 -1102}\special{pa 83 -1102}\special{fp}\special{pa 114 -1102}\special{pa 122 -1102}\special{fp}%
\special{pa 153 -1102}\special{pa 161 -1102}\special{fp}\special{pa 193 -1102}\special{pa 201 -1102}\special{fp}%
\special{pa 232 -1102}\special{pa 240 -1102}\special{fp}\special{pa 272 -1102}\special{pa 280 -1102}\special{fp}%
\special{pa 311 -1102}\special{pa 319 -1102}\special{fp}\special{pa 350 -1102}\special{pa 358 -1102}\special{fp}%
\special{pa 390 -1102}\special{pa 398 -1102}\special{fp}\special{pa 429 -1102}\special{pa 437 -1102}\special{fp}%
\special{pa 468 -1102}\special{pa 476 -1102}\special{fp}\special{pa 508 -1102}\special{pa 516 -1102}\special{fp}%
\special{pa 547 -1102}\special{pa 555 -1102}\special{fp}\special{pa 587 -1102}\special{pa 595 -1102}\special{fp}%
\special{pa 626 -1102}\special{pa 634 -1102}\special{fp}\special{pa 665 -1102}\special{pa 673 -1102}\special{fp}%
\special{pa 705 -1102}\special{pa 713 -1102}\special{fp}\special{pa 744 -1102}\special{pa 752 -1102}\special{fp}%
\special{pa 783 -1102}\special{pa 791 -1102}\special{fp}\special{pa 823 -1102}\special{pa 831 -1102}\special{fp}%
\special{pa 862 -1102}\special{pa 870 -1102}\special{fp}\special{pa 902 -1102}\special{pa 910 -1102}\special{fp}%
\special{pa 941 -1102}\special{pa 949 -1102}\special{fp}\special{pa 980 -1102}\special{pa 988 -1102}\special{fp}%
\special{pa 1020 -1102}\special{pa 1028 -1102}\special{fp}\special{pa 1059 -1102}\special{pa 1067 -1102}\special{fp}%
\special{pa 1098 -1102}\special{pa 1106 -1102}\special{fp}\special{pn 8}%
\special{pn 8}%
\special{pa 1102 4}\special{pa 1102 -4}\special{fp}\special{pa 1102 -35}\special{pa 1102 -43}\special{fp}%
\special{pa 1102 -75}\special{pa 1102 -83}\special{fp}\special{pa 1102 -114}\special{pa 1102 -122}\special{fp}%
\special{pa 1102 -153}\special{pa 1102 -161}\special{fp}\special{pa 1102 -193}\special{pa 1102 -201}\special{fp}%
\special{pa 1102 -232}\special{pa 1102 -240}\special{fp}\special{pa 1102 -272}\special{pa 1102 -280}\special{fp}%
\special{pa 1102 -311}\special{pa 1102 -319}\special{fp}\special{pa 1102 -350}\special{pa 1102 -358}\special{fp}%
\special{pa 1102 -390}\special{pa 1102 -398}\special{fp}\special{pa 1102 -429}\special{pa 1102 -437}\special{fp}%
\special{pa 1102 -468}\special{pa 1102 -476}\special{fp}\special{pa 1102 -508}\special{pa 1102 -516}\special{fp}%
\special{pa 1102 -547}\special{pa 1102 -555}\special{fp}\special{pa 1102 -587}\special{pa 1102 -595}\special{fp}%
\special{pa 1102 -626}\special{pa 1102 -634}\special{fp}\special{pa 1102 -665}\special{pa 1102 -673}\special{fp}%
\special{pa 1102 -705}\special{pa 1102 -713}\special{fp}\special{pa 1102 -744}\special{pa 1102 -752}\special{fp}%
\special{pa 1102 -783}\special{pa 1102 -791}\special{fp}\special{pa 1102 -823}\special{pa 1102 -831}\special{fp}%
\special{pa 1102 -862}\special{pa 1102 -870}\special{fp}\special{pa 1102 -902}\special{pa 1102 -910}\special{fp}%
\special{pa 1102 -941}\special{pa 1102 -949}\special{fp}\special{pa 1102 -980}\special{pa 1102 -988}\special{fp}%
\special{pa 1102 -1020}\special{pa 1102 -1028}\special{fp}\special{pa 1102 -1059}\special{pa 1102 -1067}\special{fp}%
\special{pa 1102 -1098}\special{pa 1102 -1106}\special{fp}\special{pn 8}%
{%
\color[cmyk]{0,1,1,0}%
\special{pn 16}%
\special{pa     0 -1102}\special{pa   551  -551}\special{pa   827  -276}\special{pa  1102  -551}%
\special{fp}%
\special{pn 8}%
}%
\special{pa 827 -276}\special{pa 827 -236}\special{fp}\special{pa 827 -197}\special{pa 827 -157}\special{fp}%
\special{pa 827 -118}\special{pa 827 -79}\special{fp}\special{pa 827 -39}\special{pa 827 0}\special{fp}%
%
%
\special{pa 551 -551}\special{pa 551 -514}\special{fp}\special{pa 551 -478}\special{pa 551 -441}\special{fp}%
\special{pa 551 -404}\special{pa 551 -367}\special{fp}\special{pa 551 -331}\special{pa 551 -294}\special{fp}%
\special{pa 551 -257}\special{pa 551 -220}\special{fp}\special{pa 551 -184}\special{pa 551 -147}\special{fp}%
\special{pa 551 -110}\special{pa 551 -73}\special{fp}\special{pa 551 -37}\special{pa 551 0}\special{fp}%
%
%
\special{pa   276   -20}\special{pa   276    20}%
\special{fp}%
\settowidth{\Width}{$1$}\setlength{\Width}{-0.5\Width}%
\settoheight{\Height}{$1$}\settodepth{\Depth}{$1$}\setlength{\Height}{-\Height}%
\put(1.0000000,-0.1428571){\hspace*{\Width}\raisebox{\Height}{$1$}}%
%
\special{pa   551   -20}\special{pa   551    20}%
\special{fp}%
\settowidth{\Width}{$2$}\setlength{\Width}{-0.5\Width}%
\settoheight{\Height}{$2$}\settodepth{\Depth}{$2$}\setlength{\Height}{-\Height}%
\put(2.0000000,-0.1428571){\hspace*{\Width}\raisebox{\Height}{$2$}}%
%
\special{pa   827   -20}\special{pa   827    20}%
\special{fp}%
\settowidth{\Width}{$3$}\setlength{\Width}{-0.5\Width}%
\settoheight{\Height}{$3$}\settodepth{\Depth}{$3$}\setlength{\Height}{-\Height}%
\put(3.0000000,-0.1428571){\hspace*{\Width}\raisebox{\Height}{$3$}}%
%
\special{pa  1102   -20}\special{pa  1102    20}%
\special{fp}%
\settowidth{\Width}{$4$}\setlength{\Width}{-0.5\Width}%
\settoheight{\Height}{$4$}\settodepth{\Depth}{$4$}\setlength{\Height}{-\Height}%
\put(4.0000000,-0.1428571){\hspace*{\Width}\raisebox{\Height}{$4$}}%
%
\special{pa    20  -276}\special{pa   -20  -276}%
\special{fp}%
\settowidth{\Width}{$1$}\setlength{\Width}{-1\Width}%
\settoheight{\Height}{$1$}\settodepth{\Depth}{$1$}\setlength{\Height}{-0.5\Height}\setlength{\Depth}{0.5\Depth}\addtolength{\Height}{\Depth}%
\put(-0.1428571,1.0000000){\hspace*{\Width}\raisebox{\Height}{$1$}}%
%
\special{pa    20  -551}\special{pa   -20  -551}%
\special{fp}%
\settowidth{\Width}{$2$}\setlength{\Width}{-1\Width}%
\settoheight{\Height}{$2$}\settodepth{\Depth}{$2$}\setlength{\Height}{-0.5\Height}\setlength{\Depth}{0.5\Depth}\addtolength{\Height}{\Depth}%
\put(-0.1428571,2.0000000){\hspace*{\Width}\raisebox{\Height}{$2$}}%
%
\special{pa    20  -827}\special{pa   -20  -827}%
\special{fp}%
\settowidth{\Width}{$3$}\setlength{\Width}{-1\Width}%
\settoheight{\Height}{$3$}\settodepth{\Depth}{$3$}\setlength{\Height}{-0.5\Height}\setlength{\Depth}{0.5\Depth}\addtolength{\Height}{\Depth}%
\put(-0.1428571,3.0000000){\hspace*{\Width}\raisebox{\Height}{$3$}}%
%
\special{pa    20 -1102}\special{pa   -20 -1102}%
\special{fp}%
\settowidth{\Width}{$4$}\setlength{\Width}{-1\Width}%
\settoheight{\Height}{$4$}\settodepth{\Depth}{$4$}\setlength{\Height}{-0.5\Height}\setlength{\Depth}{0.5\Depth}\addtolength{\Height}{\Depth}%
\put(-0.1428571,4.0000000){\hspace*{\Width}\raisebox{\Height}{$4$}}%
%
\special{pa  -138    -0}\special{pa  1240    -0}%
\special{fp}%
\special{pa     0   138}\special{pa     0 -1240}%
\special{fp}%
\settowidth{\Width}{$x$}\setlength{\Width}{0\Width}%
\settoheight{\Height}{$x$}\settodepth{\Depth}{$x$}\setlength{\Height}{-0.5\Height}\setlength{\Depth}{0.5\Depth}\addtolength{\Height}{\Depth}%
\put(4.5714286,0.0000000){\hspace*{\Width}\raisebox{\Height}{$x$}}%
%
\settowidth{\Width}{$y$}\setlength{\Width}{-0.5\Width}%
\settoheight{\Height}{$y$}\settodepth{\Depth}{$y$}\setlength{\Height}{\Depth}%
\put(0.0000000,4.5714286){\hspace*{\Width}\raisebox{\Height}{$y$}}%
%
\settowidth{\Width}{O}\setlength{\Width}{-1\Width}%
\settoheight{\Height}{O}\settodepth{\Depth}{O}\setlength{\Height}{-\Height}%
\put(-0.0714286,-0.0714286){\hspace*{\Width}\raisebox{\Height}{O}}%
%
\end{picture}}%}}
\end{layer}

\begin{itemize}
\item
関数の特徴を調べる\vspace{-2mm}
\item
微分:各点での値の変化を見る\vspace{-2mm}
\item
積分:個別の値より全体の値(合計)を見る\vspace{-2mm}
\vspace{33mm}
\item
$x=2$での傾きと全体の面積は?
\end{itemize}

\newslide{定積分と不定積分}

\vspace*{18mm}


\begin{layer}{120}{0}
\putnotew{96}{73}{\hyperlink{para8pg7}{\fbox{\Ctab{2.5mm}{\scalebox{1}{\scriptsize $\mathstrut||\!\lhd$}}}}}
\putnotew{101}{73}{\hyperlink{para9pg1}{\fbox{\Ctab{2.5mm}{\scalebox{1}{\scriptsize $\mathstrut|\!\lhd$}}}}}
\putnotew{108}{73}{\hyperlink{para9pg5}{\fbox{\Ctab{4.5mm}{\scalebox{1}{\scriptsize $\mathstrut\!\!\lhd\!\!$}}}}}
\putnotew{115}{73}{\hyperlink{para9pg6}{\fbox{\Ctab{4.5mm}{\scalebox{1}{\scriptsize $\mathstrut\!\rhd\!$}}}}}
\putnotew{120}{73}{\hyperlink{para9pg6}{\fbox{\Ctab{2.5mm}{\scalebox{1}{\scriptsize $\mathstrut \!\rhd\!\!|$}}}}}
\putnotew{125}{73}{\hyperlink{para10pg1}{\fbox{\Ctab{2.5mm}{\scalebox{1}{\scriptsize $\mathstrut \!\rhd\!\!||$}}}}}
\putnotee{126}{73}{\scriptsize\color{blue} 6/6}
\end{layer}

\slidepage
\begin{itemize}
\item
定積分:全体の面積\\
 それ自体は微分と無関係
\item
不定積分:微分の逆計算\\
 微分と密接な関係
\item
微分と定積分は関係なさそうだが\\
 実は密接に関係する(17世紀に発見)
\end{itemize}

\newslide{$f(x)$の不定積分}

\vspace*{18mm}


\begin{layer}{120}{0}
\putnotew{96}{73}{\hyperlink{para9pg6}{\fbox{\Ctab{2.5mm}{\scalebox{1}{\scriptsize $\mathstrut||\!\lhd$}}}}}
\putnotew{101}{73}{\hyperlink{para10pg1}{\fbox{\Ctab{2.5mm}{\scalebox{1}{\scriptsize $\mathstrut|\!\lhd$}}}}}
\putnotew{108}{73}{\hyperlink{para10pg6}{\fbox{\Ctab{4.5mm}{\scalebox{1}{\scriptsize $\mathstrut\!\!\lhd\!\!$}}}}}
\putnotew{115}{73}{\hyperlink{para10pg7}{\fbox{\Ctab{4.5mm}{\scalebox{1}{\scriptsize $\mathstrut\!\rhd\!$}}}}}
\putnotew{120}{73}{\hyperlink{para10pg7}{\fbox{\Ctab{2.5mm}{\scalebox{1}{\scriptsize $\mathstrut \!\rhd\!\!|$}}}}}
\putnotew{125}{73}{\hyperlink{para11pg1}{\fbox{\Ctab{2.5mm}{\scalebox{1}{\scriptsize $\mathstrut \!\rhd\!\!||$}}}}}
\putnotee{126}{73}{\scriptsize\color{blue} 7/7}
\end{layer}

\slidepage
\begin{itemize}
\item
微分したら$f(x)$になる関数\vspace{-2mm}
\item
$F'(x)=f(x)$となる関数$F(x)$のこと\vspace{-2mm}
\item
$f(x)$の不定積分$F(x)$を$\dint f(x)\,dx$と書く\vspace{-2mm}
\item
[]例) $(\bunsuu{1}{2}x^2)'=x$より %%
 $\dint x\,dx=\bunsuu{1}{2}x^2$\vspace{-2mm}
\item
$C$が定数のとき $(\bunsuu{1}{2}x^2+C)'=x$\\
したがって,$\dint x\,dx=\bunsuu{1}{2}x^2+C$でもある.
\end{itemize}

\newslide{不定積分の書き方}

\vspace*{18mm}


\begin{layer}{120}{0}
\putnotew{96}{73}{\hyperlink{para10pg7}{\fbox{\Ctab{2.5mm}{\scalebox{1}{\scriptsize $\mathstrut||\!\lhd$}}}}}
\putnotew{101}{73}{\hyperlink{para11pg1}{\fbox{\Ctab{2.5mm}{\scalebox{1}{\scriptsize $\mathstrut|\!\lhd$}}}}}
\putnotew{108}{73}{\hyperlink{para11pg4}{\fbox{\Ctab{4.5mm}{\scalebox{1}{\scriptsize $\mathstrut\!\!\lhd\!\!$}}}}}
\putnotew{115}{73}{\hyperlink{para11pg5}{\fbox{\Ctab{4.5mm}{\scalebox{1}{\scriptsize $\mathstrut\!\rhd\!$}}}}}
\putnotew{120}{73}{\hyperlink{para11pg5}{\fbox{\Ctab{2.5mm}{\scalebox{1}{\scriptsize $\mathstrut \!\rhd\!\!|$}}}}}
\putnotew{125}{73}{\hyperlink{para12pg1}{\fbox{\Ctab{2.5mm}{\scalebox{1}{\scriptsize $\mathstrut \!\rhd\!\!||$}}}}}
\putnotee{126}{73}{\scriptsize\color{blue} 5/5}
\end{layer}

\slidepage
\begin{itemize}
\item
不定積分には$+C$の{\color{blue}任意性}がある.
\item
この$C$を{\color{red}積分定数}という.
\item
$f(x)$の不定積分を求めるには
\begin{enumerate}[(1)]
\item 微分して$f(x)$になる関数を求める.
\item それに$+C$をつけて$\dint$で表せばよい.
\end{enumerate}
\end{itemize}

\newslide{不定積分の例と課題}

\vspace*{18mm}


\begin{layer}{120}{0}
\putnotew{96}{73}{\hyperlink{para11pg5}{\fbox{\Ctab{2.5mm}{\scalebox{1}{\scriptsize $\mathstrut||\!\lhd$}}}}}
\putnotew{101}{73}{\hyperlink{para12pg1}{\fbox{\Ctab{2.5mm}{\scalebox{1}{\scriptsize $\mathstrut|\!\lhd$}}}}}
\putnotew{108}{73}{\hyperlink{para12pg7}{\fbox{\Ctab{4.5mm}{\scalebox{1}{\scriptsize $\mathstrut\!\!\lhd\!\!$}}}}}
\putnotew{115}{73}{\hyperlink{para12pg8}{\fbox{\Ctab{4.5mm}{\scalebox{1}{\scriptsize $\mathstrut\!\rhd\!$}}}}}
\putnotew{120}{73}{\hyperlink{para12pg8}{\fbox{\Ctab{2.5mm}{\scalebox{1}{\scriptsize $\mathstrut \!\rhd\!\!|$}}}}}
\putnotew{125}{73}{\hyperlink{para13pg1}{\fbox{\Ctab{2.5mm}{\scalebox{1}{\scriptsize $\mathstrut \!\rhd\!\!||$}}}}}
\putnotee{126}{73}{\scriptsize\color{blue} 8/8}
\end{layer}

\slidepage
\begin{itemize}
\item
[例] $\dint 1\,dx$
\item
[]微分して$1$になる関数
\hspace{1zw}$(\hakoma{x})'=1$\\
したがって $\dint 1\,dx=x+C$
\item
[例] $\dint x\,dx$
$=\bunsuu{1}{2}x^2+C$
\item
[課題]\monban $\dint x^2\,dx$はどうなるか.
\end{itemize}

\newslide{不定積分の公式(べき関数)}

\vspace*{18mm}


\begin{layer}{120}{0}
\putnotew{96}{73}{\hyperlink{para12pg8}{\fbox{\Ctab{2.5mm}{\scalebox{1}{\scriptsize $\mathstrut||\!\lhd$}}}}}
\putnotew{101}{73}{\hyperlink{para13pg1}{\fbox{\Ctab{2.5mm}{\scalebox{1}{\scriptsize $\mathstrut|\!\lhd$}}}}}
\putnotew{108}{73}{\hyperlink{para13pg4}{\fbox{\Ctab{4.5mm}{\scalebox{1}{\scriptsize $\mathstrut\!\!\lhd\!\!$}}}}}
\putnotew{115}{73}{\hyperlink{para13pg5}{\fbox{\Ctab{4.5mm}{\scalebox{1}{\scriptsize $\mathstrut\!\rhd\!$}}}}}
\putnotew{120}{73}{\hyperlink{para13pg5}{\fbox{\Ctab{2.5mm}{\scalebox{1}{\scriptsize $\mathstrut \!\rhd\!\!|$}}}}}
\putnotew{125}{73}{\hyperlink{para14pg1}{\fbox{\Ctab{2.5mm}{\scalebox{1}{\scriptsize $\mathstrut \!\rhd\!\!||$}}}}}
\putnotee{126}{73}{\scriptsize\color{blue} 5/5}
\end{layer}

\slidepage
\begin{itemize}
\item
$\dint 1\,dx=x+C$\vspace{-2mm}
\item
$\dint x\,dx=\bunsuu{1}{2}x^2+C$\vspace{-2mm}
\item
$\dint x^2\,dx=\bunsuu{1}{3}x^3+C$\vspace{-2mm}
\item
$\dint x^3\,dx=\hakoma{\bunsuu{1}{4}x^4+C}$\vspace{-2mm}
\item
$\dint x^n\,dx=\hakoma{\bunsuu{1}{n+1}x^{n+1}+C}$\vspace{-2mm}
\end{itemize}

\newslide{不定積分の性質}

\vspace*{18mm}


\begin{layer}{120}{0}
\putnotew{96}{73}{\hyperlink{para13pg5}{\fbox{\Ctab{2.5mm}{\scalebox{1}{\scriptsize $\mathstrut||\!\lhd$}}}}}
\putnotew{101}{73}{\hyperlink{para14pg1}{\fbox{\Ctab{2.5mm}{\scalebox{1}{\scriptsize $\mathstrut|\!\lhd$}}}}}
\putnotew{108}{73}{\hyperlink{para14pg3}{\fbox{\Ctab{4.5mm}{\scalebox{1}{\scriptsize $\mathstrut\!\!\lhd\!\!$}}}}}
\putnotew{115}{73}{\hyperlink{para14pg4}{\fbox{\Ctab{4.5mm}{\scalebox{1}{\scriptsize $\mathstrut\!\rhd\!$}}}}}
\putnotew{120}{73}{\hyperlink{para14pg4}{\fbox{\Ctab{2.5mm}{\scalebox{1}{\scriptsize $\mathstrut \!\rhd\!\!|$}}}}}
\putnotew{125}{73}{\hyperlink{para15pg1}{\fbox{\Ctab{2.5mm}{\scalebox{1}{\scriptsize $\mathstrut \!\rhd\!\!||$}}}}}
\putnotee{126}{73}{\scriptsize\color{blue} 4/4}
\end{layer}

\slidepage
\begin{itemize}
\item
$\dint\bigl(f(x)+g(x)\bigr)\,dx=\int f(x)\,dx+\int g(x)\,dx$
\item
$\dint\bigl(f(x)-g(x)\bigr)\,dx=\int f(x)\,dx-\int g(x)\,dx$
\item
$\dint c f(x)\,dx=c\int f(x)\,dx$($c$は定数)
\item
[注)]定数の違いがあっても,$=$と考える
\end{itemize}

\newslide{不定積分の計算例}

\vspace*{18mm}


\begin{layer}{120}{0}
\putnotew{96}{73}{\hyperlink{para14pg4}{\fbox{\Ctab{2.5mm}{\scalebox{1}{\scriptsize $\mathstrut||\!\lhd$}}}}}
\putnotew{101}{73}{\hyperlink{para15pg1}{\fbox{\Ctab{2.5mm}{\scalebox{1}{\scriptsize $\mathstrut|\!\lhd$}}}}}
\putnotew{108}{73}{\hyperlink{para15pg5}{\fbox{\Ctab{4.5mm}{\scalebox{1}{\scriptsize $\mathstrut\!\!\lhd\!\!$}}}}}
\putnotew{115}{73}{\hyperlink{para15pg6}{\fbox{\Ctab{4.5mm}{\scalebox{1}{\scriptsize $\mathstrut\!\rhd\!$}}}}}
\putnotew{120}{73}{\hyperlink{para15pg6}{\fbox{\Ctab{2.5mm}{\scalebox{1}{\scriptsize $\mathstrut \!\rhd\!\!|$}}}}}
\putnotew{125}{73}{\hyperlink{para16pg1}{\fbox{\Ctab{2.5mm}{\scalebox{1}{\scriptsize $\mathstrut \!\rhd\!\!||$}}}}}
\putnotee{126}{73}{\scriptsize\color{blue} 6/6}
\end{layer}

\slidepage
\begin{itemize}
\item
[例1]$\dint (3x^2+2x+1)\,dx$\\
$=\dint 3x^2\,dx+\dint 2x\,dx+\dint 1\,dx$\\
$=x^3+x^2+x+C$($C$は積分定数)
\item
[例2]$\dint x(x-2)\,dx$\\
$=\dint (x^2-2x)\,dx$\\
$=\bunsuu{1}{3}x^3-x^2+C$($C$は積分定数)
\end{itemize}

\newslide{不定積分の計算(課題)}

\vspace*{18mm}

\slidepage
\seteda{57}
\begin{itemize}
\item
[課題]\monban 次の不定積分を求めよ.\hfill TextP19問1\vspace{2mm}\\
\hspace*{-2mm}\eda{$\dint (x^3-5x^2+1)\,dx$}
\eda{$\dint (1-x-x^2)\,dx$}\\
\hspace*{-2mm}\eda{$\dint (3x^2)\,dx$}
\eda{$\dint (-3x^2+2x+3)\,dx$}\\
\hspace*{-2mm}\eda{$\dint (4x^3-8x+3)\,dx$}
\eda{$\dint (2x^3+4x-3)\,dx$}
\end{itemize}
%%%%%%%%%%%%%

%%%%%%%%%%%%%%%%%%%%


\newslide{不定積分の公式(三角関数)}

\vspace*{18mm}


\begin{layer}{120}{0}
\putnotew{96}{73}{\hyperlink{para15pg1}{\fbox{\Ctab{2.5mm}{\scalebox{1}{\scriptsize $\mathstrut||\!\lhd$}}}}}
\putnotew{101}{73}{\hyperlink{para16pg1}{\fbox{\Ctab{2.5mm}{\scalebox{1}{\scriptsize $\mathstrut|\!\lhd$}}}}}
\putnotew{108}{73}{\hyperlink{para16pg10}{\fbox{\Ctab{4.5mm}{\scalebox{1}{\scriptsize $\mathstrut\!\!\lhd\!\!$}}}}}
\putnotew{115}{73}{\hyperlink{para16pg11}{\fbox{\Ctab{4.5mm}{\scalebox{1}{\scriptsize $\mathstrut\!\rhd\!$}}}}}
\putnotew{120}{73}{\hyperlink{para16pg11}{\fbox{\Ctab{2.5mm}{\scalebox{1}{\scriptsize $\mathstrut \!\rhd\!\!|$}}}}}
\putnotew{125}{73}{\hyperlink{para17pg1}{\fbox{\Ctab{2.5mm}{\scalebox{1}{\scriptsize $\mathstrut \!\rhd\!\!||$}}}}}
\putnotee{126}{73}{\scriptsize\color{blue} 11/11}
\end{layer}

\slidepage
\begin{itemize}
\item
\Ltab{70mm}{$\dint \cos x\,dx=\hakoma{\sin x+C}$}$(\hakoma{\sin x})'=\cos x$\vspace{-3mm}
\item
\Ltab{70mm}{$\dint \sin x\,dx=\hakoma{-\cos x+C}$}$(\hakoma{-\cos x})'=\sin x$\vspace{-3mm}
\item
\Ltab{70mm}{$\dint \bunsuu{1}{\cos^2 x}\,dx=\hakoma{\tan x+C}$}$(\hakoma{\tan x})'=\bunsuu{1}{\cos^2 x}$\vspace{-3mm}
\item
$\dint \cos ax\,dx=\bunsuu{1}{a}\sin ax+C$\vspace{-3mm}
\item
$\dint \sin ax\,dx=-\bunsuu{1}{a}\cos ax+C$
\end{itemize}

\newslide{不定積分の計算(三角関数)}

\vspace*{18mm}


\begin{layer}{120}{0}
\putnotew{96}{73}{\hyperlink{para16pg11}{\fbox{\Ctab{2.5mm}{\scalebox{1}{\scriptsize $\mathstrut||\!\lhd$}}}}}
\putnotew{101}{73}{\hyperlink{para17pg1}{\fbox{\Ctab{2.5mm}{\scalebox{1}{\scriptsize $\mathstrut|\!\lhd$}}}}}
\putnotew{108}{73}{\hyperlink{para17pg7}{\fbox{\Ctab{4.5mm}{\scalebox{1}{\scriptsize $\mathstrut\!\!\lhd\!\!$}}}}}
\putnotew{115}{73}{\hyperlink{para17pg8}{\fbox{\Ctab{4.5mm}{\scalebox{1}{\scriptsize $\mathstrut\!\rhd\!$}}}}}
\putnotew{120}{73}{\hyperlink{para17pg8}{\fbox{\Ctab{2.5mm}{\scalebox{1}{\scriptsize $\mathstrut \!\rhd\!\!|$}}}}}
\putnotew{125}{73}{\hyperlink{para18pg1}{\fbox{\Ctab{2.5mm}{\scalebox{1}{\scriptsize $\mathstrut \!\rhd\!\!||$}}}}}
\putnotee{126}{73}{\scriptsize\color{blue} 8/8}
\end{layer}

\slidepage
{\large\color{red}

\begin{layer}{120}{0}
\end{layer}

}
\begin{itemize}
\item
[例1)] $\dint(\sin 3x+\cos 4x)\,dx=-\bunsuu{1}{3}\cos 3x+\bunsuu{1}{4}\sin 4x+C$\vspace{-2mm}
\item
[例2)] $\dint \tan^2 x\,dx$
$=\dint\bunsuu{\sin^2 x}{\cos^2 x}\,dx$
$=\dint\bunsuu{1-\cos^2 x}{\cos^2 x}\,dx$\vspace{-2mm}\\
$=\dint\Bigl(\bunsuu{1}{\cos^2 x}-1\Bigr)\,dx$
$=\tan x-x+C$
\item
[課題]\monban 次の不定積分を求めよ.\seteda{55}\\
\eda{$\dint(3\sin x+\cos 3x)\,dx$}\\%
\eda{$\dint(1+\bunsuu{1}{\cos x})(1-\bunsuu{1}{\cos x})\,dx$}
\end{itemize}

\newslide{不定積分の公式(指数対数)}

\vspace*{18mm}


\begin{layer}{120}{0}
\putnotew{96}{73}{\hyperlink{para17pg8}{\fbox{\Ctab{2.5mm}{\scalebox{1}{\scriptsize $\mathstrut||\!\lhd$}}}}}
\putnotew{101}{73}{\hyperlink{para18pg1}{\fbox{\Ctab{2.5mm}{\scalebox{1}{\scriptsize $\mathstrut|\!\lhd$}}}}}
\putnotew{108}{73}{\hyperlink{para18pg11}{\fbox{\Ctab{4.5mm}{\scalebox{1}{\scriptsize $\mathstrut\!\!\lhd\!\!$}}}}}
\putnotew{115}{73}{\hyperlink{para18pg12}{\fbox{\Ctab{4.5mm}{\scalebox{1}{\scriptsize $\mathstrut\!\rhd\!$}}}}}
\putnotew{120}{73}{\hyperlink{para18pg12}{\fbox{\Ctab{2.5mm}{\scalebox{1}{\scriptsize $\mathstrut \!\rhd\!\!|$}}}}}
\putnotew{125}{73}{\hyperlink{para19pg1}{\fbox{\Ctab{2.5mm}{\scalebox{1}{\scriptsize $\mathstrut \!\rhd\!\!||$}}}}}
\putnotee{126}{73}{\scriptsize\color{blue} 12/12}
\end{layer}

\slidepage

\begin{layer}{120}{0}
\putnotee{100}{55}{\color{red}\large$(\log(ax))'=\bunsuu{1}{x}$}
\end{layer}

\begin{itemize}
\item
\Ltab{55mm}{$\dint e^x\,dx=\hakoma{e^x+C}$}$(\hakoma{e^x})'=e^x$\vspace{-2mm}
\item
\Ltab{60mm}{$\dint e^{ax}\,dx=\hakoma{\bunsuu{1}{a}e^{ax}+C}$}$(e^{ax})'=\hakoma{ae^{ax}}$\vspace{-2mm}
\item
\Ltab{60mm}{$\dint \bunsuu{1}{x}\,dx=\hakoma{\log x+C}$}$(\hakoma{\log x})'=\bunsuu{1}{x}$\\
注) $x<0$のとき $\dint\bunsuu{1}{x}\,dx=\log(-x)+C$\\
  合わせて $\dint\bunsuu{1}{x}\,dx=\log|x|+C$
\end{itemize}

\newslide{不定積分の計算(指数対数)}

\vspace*{18mm}


\begin{layer}{120}{0}
\putnotew{96}{73}{\hyperlink{para18pg12}{\fbox{\Ctab{2.5mm}{\scalebox{1}{\scriptsize $\mathstrut||\!\lhd$}}}}}
\putnotew{101}{73}{\hyperlink{para19pg1}{\fbox{\Ctab{2.5mm}{\scalebox{1}{\scriptsize $\mathstrut|\!\lhd$}}}}}
\putnotew{108}{73}{\hyperlink{para19pg9}{\fbox{\Ctab{4.5mm}{\scalebox{1}{\scriptsize $\mathstrut\!\!\lhd\!\!$}}}}}
\putnotew{115}{73}{\hyperlink{para19pg10}{\fbox{\Ctab{4.5mm}{\scalebox{1}{\scriptsize $\mathstrut\!\rhd\!$}}}}}
\putnotew{120}{73}{\hyperlink{para19pg10}{\fbox{\Ctab{2.5mm}{\scalebox{1}{\scriptsize $\mathstrut \!\rhd\!\!|$}}}}}
\putnotew{125}{73}{\hyperlink{para20pg1}{\fbox{\Ctab{2.5mm}{\scalebox{1}{\scriptsize $\mathstrut \!\rhd\!\!||$}}}}}
\putnotee{126}{73}{\scriptsize\color{blue} 10/10}
\end{layer}

\slidepage
{\large\color{red}

\begin{layer}{120}{0}
\end{layer}

}
\begin{itemize}
\item
[例1)] $\dint(e^{2x}+e^{-x})\,dx$
$=\bunsuu{1}{2}e^{2x}-e^{-x}+C$\vspace{-2mm}
\item
[例2)] $\dint(x+\bunsuu{1}{x})\,dx$
$=\bunsuu{1}{2}x^2+\log|x|+C$\ ($x<0$も含む)\vspace{-2mm}
\item
[例3)] $\dint(e^{x}+e^{-x})^2\,dx$
$=\dint((e^x)^2+2e^x e^{-x}+(e^{-x})^2)\,dx$\\
$=\dint(e^{2x}+2+e^{-2x})\,dx$
$=\bunsuu{1}{2}e^{2x}-\bunsuu{1}{2}e^{-2x}+C$\vspace{-2mm}
\item
[課題]\monbannoadd 次の不定積分を求めよ.\seteda{55}\\
\eda{$\dint(2e^x+\bunsuu{3}{x})\,dx$}%
\eda{$\dint(e^x+1)^2\,dx$}
\end{itemize}
\addban
\label{pageend}\mbox{}

\end{document}

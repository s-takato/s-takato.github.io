\documentclass[10pt,dvipdfmx]{jarticle}
\usepackage{amsmath,amssymb}
\usepackage{graphicx}
\usepackage{color}
\usepackage[dvipdfmx]{pict2e}
\usepackage{ketpic2e}
\usepackage{ketlayer2e}
\usepackage{ketlayermorewith2e}
\setmargin{15}{15}{15}{15}
\pagestyle{headings}
\begin{document}
\begin{center}
\verb|question0829-01.txt|\\
\end{center}
Q01 
$\text{次の関数の導関数はどうなるか.}//$\\
$\text{アプリ「導関数の意味」を用いよ}$\\
$[1]y=x^2-3x$\\
$[2]y=\sin x $\\
\text{Sheet} 
$[1]y'=\;\;::2$ 
$[2]y'=\;\;::2$ 
\\
\text{Ans}\\
$[1]\;y'=2x-3$\\
$[2]\;y'=\cos x $\\
\newpage
\begin{center}
\verb|question0829-02.txt|\\
\end{center}
Q02 
$\text{次の関数を微分せよ.}$\\
$[1]\;y=x^4+x^3-x^2$\\
$[2]\;y=3x^5$\\
$[3]\;y=(x+1)\sqrt{x}$\\
$[4]\;y=\dfrac{x^2+1}{x+2}$\\
$[5]\;y=(2x+3)^5$\\
\text{Sheet} 
$[1]y'=\;\;::2$ 
$[2]y'=\;\;::2$ 
$[3]y'=\;\;::2$ 
$[4]y'=\;\;::2$ 
$[5]y'=\;\;::2$ 
\\
\text{Ans}\\
$[1]\;y'=4x^3+3x^2-2x$\\
$[2]\;y'=15x^4$\\
$[3]\;y'=\sqrt{x}+\dfrac{x+1}{2\sqrt{x}}$\\
$[4]\;y'=\dfrac{x^2+4x-1}{(x+2)^2}$\\
$[5]\;y'=10(2x+3)^4$\\
\newpage
\begin{center}
\verb|question0829-03.txt|\\
\end{center}
Q03 
$\text{次の関数を微分せよ.}$\\
$[1]y=x+\cos x $\\
$[2]y=x\sin x $\\
$[3]y=\sin 4x $\\
\text{Sheet} 
$[1]y'=\;\;::2$ 
$[2]y'=\;\;::2$ 
$[3]y'=\;\;::2$ 
\\
\text{Ans}\\
$[1]\;y'=1-\sin x $\\
$[2]\;y'=\sin x +x\cos x $\\
$[3]\;y'=4\cos 4x $\\
\newpage
\begin{center}
\verb|question0829-04.txt|\\
\end{center}
Q04 
$\text{次の問いに答えよ}$\\
$[]e\text{の値を小数第5位まで書け.}$\\
\text{Sheet} 
$[]e=\;\;::2$ 
\\
\text{Ans}\\
$[]\;2.71828$\\
Q04 
$\text{微分せよ.}$\\
$[1]\;y=e^{x}+x^{2}$\\
$[2]\;y=e^{2x}$\\
$[3]\;y=e^{-x}$\\
\text{Sheet} 
$[1]\;y'=\;\;::2$ 
$[2]\;y'=\;\;::2$ 
$[3]\;y'=\;\;::2$ 
\\
\text{Ans}\\
$[1]\;y'=e^{x}+2x$\\
$[2]\;y'=2e^{x}$\\
$[3]\;y'=-e^{-x}$\\
\newpage
\begin{center}
\verb|question0829-06.txt|\\
\end{center}
Q05 
$\text{微分せよ.}$\\
$[1]\;y=\log x +e^{x}$\\
$[2]\;y=\log 2x $\\
$[3]\;y=\log (x+2) $\\
\text{Sheet} 
$[1]\;y'=\;\;::2$ 
$[2]\;y'=\;\;::2$ 
$[3]\;y'=\;\;::2$ 
\\
\text{Ans}\\
$[1]\;y'=\dfrac{1}{x}+e^{x}$\\
$[2]\;y'=\dfrac{1}{x}$\\
$[3]\;y'=\dfrac{1}{x+2}$\\
\newpage
\begin{center}
\verb|question0829-07.txt|\\
\end{center}
Q06 
$\text{問いに答えよ.}$\\
$[]\;\displaystyle\int x^2 dx \text{はどうなるか.}$\\
\text{Sheet} 
$[]\;\displaystyle\int x^2 dx =\;\;::2$ 
\\
\text{Ans}\\
$[]\;y'=\dfrac{1}{3}x^3+C$\\
\newpage
\begin{center}
\verb|question0829-08.txt|\\
\end{center}
Q07 
$\text{次の不定積分を求めよ.}$\\
$[1]\;\displaystyle\int (x^3-5x^2+1) dx $\\
$[2]\;\displaystyle\int (1-x-x^2) dx $\\
$[3]\;\displaystyle\int 3x^2 dx $\\
$[4]\;\displaystyle\int (-3x^2+2x+3) dx $\\
$[5]\;\displaystyle\int (4x^3-8x+3) dx $\\
$[6]\;\displaystyle\int (2x^3+4x-3) dx $\\
\text{Sheet} 
$[1]\;=\;\;::2$ 
$[2]\;=\;\;::2$ 
$[3]\;=\;\;::2$ 
$[4]\;=\;\;::2$ 
$[5]\;=\;\;::2$ 
$[6]\;=\;\;::2$ 
\\
\text{Ans}\\
$[1]\;=\dfrac{1}{4}x^4-\dfrac{5}{3}x^3+x+C$\\
$[2]\;=x-\dfrac{1}{2}x^2-\dfrac{1}{3}x^3+C$\\
$[3]\;=x^3+C$\\
$[4]\;=-x^3+x^2+3x+C$\\
$[5]\;=x^4-4x^2+3x+C$\\
$[6]\;=\dfrac{1}{2}x^4+2x^2-3x+C$\\
\newpage
\begin{center}
\verb|question0829-09.txt|\\
\end{center}
Q08 
$\text{次の不定積分を求めよ.}$\\
$[1]\;\displaystyle\int (3\sin x +\cos 3x ) dx $\\
$[2]\;\displaystyle\int (1+\dfrac{1}{\cos x })(1-\dfrac{1}{\cos x }) dx $\\
\text{Sheet} 
$[1]\;=\;::2$ 
$[2]\;=\;::2$ 
\\
\text{Ans}\\
$[1]\;=-3\cos x +\dfrac{1}{3}\sin 3x +C$\\
$[2]\;=x-\tan x +C$\\
\newpage
\begin{center}
\verb|question0829-10.txt|\\
\end{center}
Q09 
$\text{次の不定積分を求めよ.}$\\
$[1]\;\displaystyle\int (2e^x+\dfrac{3}{x}) dx $\\
$[2]\;\displaystyle\int (e^x+1)^2 dx $\\
\text{Sheet} 
$[1]\;=\;::2$ 
$[2]\;=\;::2$ 
\\
\text{Ans}\\
$[1]\;=2e^x+3\log x +C$\\
$[2]\;=\dfrac{1}{2}e^x+2e^x+x+C$\\
\newpage
\end{document}

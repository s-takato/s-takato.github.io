%%% タイトル presen22206
\documentclass[landscape,10pt]{jarticle}
\special{papersize=\the\paperwidth,\the\paperheight}
\usepackage{ketpic,ketlayer}
\usepackage{ketslide}
\usepackage{amsmath,amssymb}
\usepackage{bm,enumerate}
\usepackage[dvipdfmx]{graphicx}
\usepackage{color}
\definecolor{slidecolora}{cmyk}{0.98,0.13,0,0.43}
\definecolor{slidecolorb}{cmyk}{0.2,0,0,0}
\definecolor{slidecolorc}{cmyk}{0.2,0,0,0}
\definecolor{slidecolord}{cmyk}{0.2,0,0,0}
\definecolor{slidecolore}{cmyk}{0,0,0,0.5}
\definecolor{slidecolorf}{cmyk}{0,0,0,0.5}
\definecolor{slidecolori}{cmyk}{0.98,0.13,0,0.43}
\def\setthin#1{\def\thin{#1}}
\setthin{0}
\newcommand{\slidepage}[1][s]{%
\setcounter{ketpicctra}{18}%
\if#1m \setcounter{ketpicctra}{1}\fi
\hypersetup{linkcolor=black}%

\begin{layer}{118}{0}
\putnotee{122}{-\theketpicctra.05}{\small\thepage/\pageref{pageend}}
\end{layer}\hypersetup{linkcolor=blue}

}
\usepackage{emath}
\usepackage{pict2e}
\usepackage{ketlayermorewith2e}
\usepackage[dvipdfmx,colorlinks=true,linkcolor=blue,filecolor=blue]{hyperref}
\newcommand{\hiduke}{0912}
\newcommand{\hako}[2][1]{\fbox{\raisebox{#1mm}{\mbox{}}\raisebox{-#1mm}{\mbox{}}\,\phantom{#2}\,}}
\newcommand{\hakoa}[2][1]{\fbox{\raisebox{#1mm}{\mbox{}}\raisebox{-#1mm}{\mbox{}}\,#2\,}}
\newcommand{\hakom}[2][1]{\hako[#1]{$#2$}}
\newcommand{\hakoma}[2][1]{\hakoa[#1]{$#2$}}
\def\rad{\;\mathrm{rad}}
\def\deg#1{#1^{\circ}}
\newcommand{\sbunsuu}[2]{\scalebox{0.6}{$\bunsuu{#1}{#2}$}}
\def\pow{$\hspace{-1.5mm}^\hspace{-1mm}$}
\def\dlim{\displaystyle\lim}
\newcommand{\brd}[2][1]{\scalebox{#1}{\color{red}\fbox{\color{black}$#2$}}}
\newcommand\down[1][0.5zw]{\vspace{#1}\\}
\newcommand{\sfrac}[3][0.65]{\scalebox{#1}{$\frac{#2}{#3}$}}
\newcommand{\phn}[1]{\phantom{#1}}
\newcommand{\scb}[2][0.6]{\scalebox{#1}{#2}}
\newcommand{\dsum}{\displaystyle\sum}
\def\pow{$\hspace{-1.5mm}^\hspace{-1mm}$}
\def\dlim{\displaystyle\lim}
\def\dint{\displaystyle\int}

\setmargin{25}{145}{15}{100}

\ketslideinit

\pagestyle{empty}

\begin{document}

\begin{layer}{120}{0}
\putnotese{0}{0}{{\Large\bf
\color[cmyk]{1,1,0,0}

\begin{layer}{120}{0}
{\Huge \putnotes{60}{20}{積分法2}}
\putnotes{60}{70}{2022.9.12}
\end{layer}

}
}
\end{layer}

\def\mainslidetitley{22}
\def\ketcletter{slidecolora}
\def\ketcbox{slidecolorb}
\def\ketdbox{slidecolorc}
\def\ketcframe{slidecolord}
\def\ketcshadow{slidecolore}
\def\ketdshadow{slidecolorf}
\def\slidetitlex{6}
\def\slidetitlesize{1.3}
\def\mketcletter{slidecolori}
\def\mketcbox{yellow}
\def\mketdbox{yellow}
\def\mketcframe{yellow}
\def\mslidetitlex{62}
\def\mslidetitlesize{2}

\color{black}
\Large\bf\boldmath
\addtocounter{page}{-1}

\def\MARU{}
\renewcommand{\MARU}[1]{{\ooalign{\hfil$#1$\/\hfil\crcr\raise.167ex\hbox{\mathhexbox20D}}}}
\renewcommand{\slidepage}[1][s]{%
\setcounter{ketpicctra}{18}%
\if#1m \setcounter{ketpicctra}{1}\fi
\hypersetup{linkcolor=black}%
\begin{layer}{118}{0}
\putnotee{115}{-\theketpicctra.05}{\small\hiduke-\thepage/\pageref{pageend}}
\end{layer}\hypersetup{linkcolor=blue}
}
\newcounter{ban}
\setcounter{ban}{1}
\newcommand{\monban}[1][\hiduke]{%
#1-\theban\ %
\addtocounter{ban}{1}%
}
\newcommand{\monbannoadd}[1][\hiduke]{%
#1-\theban\ %
}
\newcommand{\addban}{%
\addtocounter{ban}{1}%
}
\newcounter{edawidth}
\newcounter{edactr}
\newcommand{\seteda}[1]{% 20220708 modified
\setcounter{edawidth}{#1}
\setcounter{edactr}{1}
}
\newcommand{\eda}[2][\theedawidth]{%
\Ltab{#1 mm}{[\theedactr]\ #2}%
\addtocounter{edactr}{1}%
}
%%%%%%%%%%%%

%%%%%%%%%%%%%%%%%%%%

\mainslide{ 定積分}


\slidepage[m]
%%%%%%%%%%%%%

%%%%%%%%%%%%%%%%%%%%

\newslide{区分求積法による定義}

\vspace*{18mm}


\begin{layer}{120}{0}
\putnotew{96}{73}{\hyperlink{para0pg0}{\fbox{\Ctab{2.5mm}{\scalebox{1}{\scriptsize $\mathstrut||\!\lhd$}}}}}
\putnotew{101}{73}{\hyperlink{para1pg1}{\fbox{\Ctab{2.5mm}{\scalebox{1}{\scriptsize $\mathstrut|\!\lhd$}}}}}
\putnotew{108}{73}{\hyperlink{para1pg3}{\fbox{\Ctab{4.5mm}{\scalebox{1}{\scriptsize $\mathstrut\!\!\lhd\!\!$}}}}}
\putnotew{115}{73}{\hyperlink{para1pg4}{\fbox{\Ctab{4.5mm}{\scalebox{1}{\scriptsize $\mathstrut\!\rhd\!$}}}}}
\putnotew{120}{73}{\hyperlink{para1pg4}{\fbox{\Ctab{2.5mm}{\scalebox{1}{\scriptsize $\mathstrut \!\rhd\!\!|$}}}}}
\putnotew{125}{73}{\hyperlink{para2pg1}{\fbox{\Ctab{2.5mm}{\scalebox{1}{\scriptsize $\mathstrut \!\rhd\!\!||$}}}}}
\putnotee{126}{73}{\scriptsize\color{blue} 4/4}
\end{layer}

\slidepage

\begin{layer}{120}{0}
\putnotese{83}{6}{%%% /Users/takatoosetsuo/polytech22.git/205-0912/presen/fig/kubunkyuuseki4.tex 
%%% Generator=presen22206.cdy 
{\unitlength=1cm%
\begin{picture}%
(4.5,3)(-0.5,-0.5)%
\linethickness{0.008in}%%
\Large\bf\boldmath%
\small%
\linethickness{0.012in}%%
\polyline(0.50000,1.00000)(0.62137,0.99804)(0.73932,0.99942)(0.85385,1.00414)(0.96494,1.01218)%
(1.07262,1.02357)(1.17687,1.03829)(1.27769,1.05634)(1.37509,1.07773)(1.46907,1.10245)%
(1.55962,1.13051)(1.68911,1.18163)(1.81192,1.24364)(1.92933,1.31325)(2.04259,1.38719)%
(2.15297,1.46219)(2.26175,1.53495)(2.37020,1.60221)(2.47957,1.66069)(2.59115,1.70711)%
(2.70619,1.73819)(2.78445,1.74920)(2.86296,1.75247)(2.94171,1.74800)(3.02072,1.73579)%
(3.09998,1.71584)(3.17948,1.68815)(3.25924,1.65272)(3.33924,1.60955)(3.41950,1.55865)%
(3.50000,1.50000)%
%
\linethickness{0.008in}%%
\linethickness{0.004in}%%
\polyline(0.50000,0.00000)(0.50000,1.00000)%
%
\linethickness{0.008in}%%
\linethickness{0.004in}%%
\polyline(3.50000,0.00000)(3.50000,1.50000)%
%
\linethickness{0.008in}%%
\settowidth{\Width}{$y=f(x)$}\setlength{\Width}{-1\Width}%
\settoheight{\Height}{$y=f(x)$}\settodepth{\Depth}{$y=f(x)$}\setlength{\Height}{-0.5\Height}\setlength{\Depth}{0.5\Depth}\addtolength{\Height}{\Depth}%
\put(1.9400000,1.7700000){\hspace*{\Width}\raisebox{\Height}{$y=f(x)$}}%
%
\polyline(0.50000,0.05000)(0.50000,-0.05000)%
%
\settowidth{\Width}{$a$}\setlength{\Width}{-0.5\Width}%
\settoheight{\Height}{$a$}\settodepth{\Depth}{$a$}\setlength{\Height}{-\Height}%
\put(0.5000000,-0.1000000){\hspace*{\Width}\raisebox{\Height}{$a$}}%
%
\polyline(3.50000,0.05000)(3.50000,-0.05000)%
%
\settowidth{\Width}{$b$}\setlength{\Width}{-0.5\Width}%
\settoheight{\Height}{$b$}\settodepth{\Depth}{$b$}\setlength{\Height}{-\Height}%
\put(3.5000000,-0.1000000){\hspace*{\Width}\raisebox{\Height}{$b$}}%
%
\linethickness{0.004in}%%
\polyline(0.50000,0.00000)(0.50000,1.00000)(1.25000,1.00000)(1.25000,0.00000)%
%
\linethickness{0.008in}%%
\linethickness{0.004in}%%
\polyline(1.25000,0.00000)(1.25000,1.05138)(2.00000,1.05138)(2.00000,0.00000)%
%
\linethickness{0.008in}%%
\linethickness{0.004in}%%
\polyline(2.00000,0.00000)(2.00000,1.35939)(2.75000,1.35939)(2.75000,0.00000)%
%
\linethickness{0.008in}%%
\linethickness{0.004in}%%
\polyline(2.75000,0.00000)(2.75000,1.74435)(3.50000,1.74435)(3.50000,0.00000)%
%
\linethickness{0.008in}%%
\polyline(-0.50000,0.00000)(4.00000,0.00000)%
%
\polyline(0.00000,-0.50000)(0.00000,2.50000)%
%
\settowidth{\Width}{$x$}\setlength{\Width}{0\Width}%
\settoheight{\Height}{$x$}\settodepth{\Depth}{$x$}\setlength{\Height}{-0.5\Height}\setlength{\Depth}{0.5\Depth}\addtolength{\Height}{\Depth}%
\put(4.0500000,0.0000000){\hspace*{\Width}\raisebox{\Height}{$x$}}%
%
\settowidth{\Width}{$y$}\setlength{\Width}{-0.5\Width}%
\settoheight{\Height}{$y$}\settodepth{\Depth}{$y$}\setlength{\Height}{\Depth}%
\put(0.0000000,2.5500000){\hspace*{\Width}\raisebox{\Height}{$y$}}%
%
\settowidth{\Width}{O}\setlength{\Width}{-1\Width}%
\settoheight{\Height}{O}\settodepth{\Depth}{O}\setlength{\Height}{-\Height}%
\put(-0.0500000,-0.0500000){\hspace*{\Width}\raisebox{\Height}{O}}%
%
\end{picture}}%}
\end{layer}

\begin{itemize}
\item
$\displaystyle\int_a^b f(x)\,dx$
\item
区間を$n$個に分けて長方形で近似\\
\hspace*{2zw}$a=x_0,\ x_1,\ \cdots,\ x_n=b$
\item
区間の幅を$dx_j$,区間内の1点を$x_j$とすると\\
\hspace*{2zw}長方形の面積$\fallingdotseq f(x_j)dx_j$
\item
長方形の面積の合計(近似値)は\\
\hspace*{2zw}$\displaystyle\sum_j f(x_j)dx_j$
\end{itemize}

\newslide{区分求積法による定義(続)}

\vspace*{18mm}


\begin{layer}{120}{0}
\putnotew{96}{73}{\hyperlink{para1pg4}{\fbox{\Ctab{2.5mm}{\scalebox{1}{\scriptsize $\mathstrut||\!\lhd$}}}}}
\putnotew{101}{73}{\hyperlink{para2pg1}{\fbox{\Ctab{2.5mm}{\scalebox{1}{\scriptsize $\mathstrut|\!\lhd$}}}}}
\putnotew{108}{73}{\hyperlink{para2pg7}{\fbox{\Ctab{4.5mm}{\scalebox{1}{\scriptsize $\mathstrut\!\!\lhd\!\!$}}}}}
\putnotew{115}{73}{\hyperlink{para2pg8}{\fbox{\Ctab{4.5mm}{\scalebox{1}{\scriptsize $\mathstrut\!\rhd\!$}}}}}
\putnotew{120}{73}{\hyperlink{para2pg8}{\fbox{\Ctab{2.5mm}{\scalebox{1}{\scriptsize $\mathstrut \!\rhd\!\!|$}}}}}
\putnotew{125}{73}{\hyperlink{para3pg1}{\fbox{\Ctab{2.5mm}{\scalebox{1}{\scriptsize $\mathstrut \!\rhd\!\!||$}}}}}
\putnotee{126}{73}{\scriptsize\color{blue} 8/8}
\end{layer}

\slidepage

\begin{layer}{120}{0}
\putnotese{83}{6}{%%% /Users/takatoosetsuo/polytech22.git/205-0912/presen/fig/kubunkyuuseki16.tex 
%%% Generator=presen22206.cdy 
{\unitlength=1cm%
\begin{picture}%
(4.5,3)(-0.5,-0.5)%
\linethickness{0.008in}%%
\Large\bf\boldmath%
\small%
\linethickness{0.012in}%%
\polyline(0.50000,1.00000)(0.62137,0.99804)(0.73932,0.99942)(0.85385,1.00414)(0.96494,1.01218)%
(1.07262,1.02357)(1.17687,1.03829)(1.27769,1.05634)(1.37509,1.07773)(1.46907,1.10245)%
(1.55962,1.13051)(1.68911,1.18163)(1.81192,1.24364)(1.92933,1.31325)(2.04259,1.38719)%
(2.15297,1.46219)(2.26175,1.53495)(2.37020,1.60221)(2.47957,1.66069)(2.59115,1.70711)%
(2.70619,1.73819)(2.78445,1.74920)(2.86296,1.75247)(2.94171,1.74800)(3.02072,1.73579)%
(3.09998,1.71584)(3.17948,1.68815)(3.25924,1.65272)(3.33924,1.60955)(3.41950,1.55865)%
(3.50000,1.50000)%
%
\linethickness{0.008in}%%
\linethickness{0.004in}%%
\polyline(0.50000,0.00000)(0.50000,1.00000)%
%
\linethickness{0.008in}%%
\linethickness{0.004in}%%
\polyline(3.50000,0.00000)(3.50000,1.50000)%
%
\linethickness{0.008in}%%
\settowidth{\Width}{$y=f(x)$}\setlength{\Width}{-1\Width}%
\settoheight{\Height}{$y=f(x)$}\settodepth{\Depth}{$y=f(x)$}\setlength{\Height}{-0.5\Height}\setlength{\Depth}{0.5\Depth}\addtolength{\Height}{\Depth}%
\put(1.9400000,1.7700000){\hspace*{\Width}\raisebox{\Height}{$y=f(x)$}}%
%
\polyline(0.50000,0.05000)(0.50000,-0.05000)%
%
\settowidth{\Width}{$a$}\setlength{\Width}{-0.5\Width}%
\settoheight{\Height}{$a$}\settodepth{\Depth}{$a$}\setlength{\Height}{-\Height}%
\put(0.5000000,-0.1000000){\hspace*{\Width}\raisebox{\Height}{$a$}}%
%
\polyline(3.50000,0.05000)(3.50000,-0.05000)%
%
\settowidth{\Width}{$b$}\setlength{\Width}{-0.5\Width}%
\settoheight{\Height}{$b$}\settodepth{\Depth}{$b$}\setlength{\Height}{-\Height}%
\put(3.5000000,-0.1000000){\hspace*{\Width}\raisebox{\Height}{$b$}}%
%
\linethickness{0.004in}%%
\polyline(0.50000,0.00000)(0.50000,1.00000)(0.68750,1.00000)(0.68750,0.00000)%
%
\linethickness{0.008in}%%
\linethickness{0.004in}%%
\polyline(0.68750,0.00000)(0.68750,0.99882)(0.87500,0.99882)(0.87500,0.00000)%
%
\linethickness{0.008in}%%
\linethickness{0.004in}%%
\polyline(0.87500,0.00000)(0.87500,1.00567)(1.06250,1.00567)(1.06250,0.00000)%
%
\linethickness{0.008in}%%
\linethickness{0.004in}%%
\polyline(1.06250,0.00000)(1.06250,1.02250)(1.25000,1.02250)(1.25000,0.00000)%
%
\linethickness{0.008in}%%
\linethickness{0.004in}%%
\polyline(1.25000,0.00000)(1.25000,1.05138)(1.43750,1.05138)(1.43750,0.00000)%
%
\linethickness{0.008in}%%
\linethickness{0.004in}%%
\polyline(1.43750,0.00000)(1.43750,1.09414)(1.62500,1.09414)(1.62500,0.00000)%
%
\linethickness{0.008in}%%
\linethickness{0.004in}%%
\polyline(1.62500,0.00000)(1.62500,1.15632)(1.81250,1.15632)(1.81250,0.00000)%
%
\linethickness{0.008in}%%
\linethickness{0.004in}%%
\polyline(1.81250,0.00000)(1.81250,1.24398)(2.00000,1.24398)(2.00000,0.00000)%
%
\linethickness{0.008in}%%
\linethickness{0.004in}%%
\polyline(2.00000,0.00000)(2.00000,1.35939)(2.18750,1.35939)(2.18750,0.00000)%
%
\linethickness{0.008in}%%
\linethickness{0.004in}%%
\polyline(2.18750,0.00000)(2.18750,1.48528)(2.37500,1.48528)(2.37500,0.00000)%
%
\linethickness{0.008in}%%
\linethickness{0.004in}%%
\polyline(2.37500,0.00000)(2.37500,1.60478)(2.56250,1.60478)(2.56250,0.00000)%
%
\linethickness{0.008in}%%
\linethickness{0.004in}%%
\polyline(2.56250,0.00000)(2.56250,1.69519)(2.75000,1.69519)(2.75000,0.00000)%
%
\linethickness{0.008in}%%
\linethickness{0.004in}%%
\polyline(2.75000,0.00000)(2.75000,1.74435)(2.93750,1.74435)(2.93750,0.00000)%
%
\linethickness{0.008in}%%
\linethickness{0.004in}%%
\polyline(2.93750,0.00000)(2.93750,1.74824)(3.12500,1.74824)(3.12500,0.00000)%
%
\linethickness{0.008in}%%
\linethickness{0.004in}%%
\polyline(3.12500,0.00000)(3.12500,1.70712)(3.31250,1.70712)(3.31250,0.00000)%
%
\linethickness{0.008in}%%
\linethickness{0.004in}%%
\polyline(3.31250,0.00000)(3.31250,1.62398)(3.50000,1.62398)(3.50000,0.00000)%
%
\linethickness{0.008in}%%
\polyline(-0.50000,0.00000)(4.00000,0.00000)%
%
\polyline(0.00000,-0.50000)(0.00000,2.50000)%
%
\settowidth{\Width}{$x$}\setlength{\Width}{0\Width}%
\settoheight{\Height}{$x$}\settodepth{\Depth}{$x$}\setlength{\Height}{-0.5\Height}\setlength{\Depth}{0.5\Depth}\addtolength{\Height}{\Depth}%
\put(4.0500000,0.0000000){\hspace*{\Width}\raisebox{\Height}{$x$}}%
%
\settowidth{\Width}{$y$}\setlength{\Width}{-0.5\Width}%
\settoheight{\Height}{$y$}\settodepth{\Depth}{$y$}\setlength{\Height}{\Depth}%
\put(0.0000000,2.5500000){\hspace*{\Width}\raisebox{\Height}{$y$}}%
%
\settowidth{\Width}{O}\setlength{\Width}{-1\Width}%
\settoheight{\Height}{O}\settodepth{\Depth}{O}\setlength{\Height}{-\Height}%
\put(-0.0500000,-0.0500000){\hspace*{\Width}\raisebox{\Height}{O}}%
%
\end{picture}}%}
\end{layer}

\begin{itemize}
\item
$n$を限りなく大きくする\vspace{2mm}\\
\hspace*{2zw}$\displaystyle\lim_{n \to \infty}\sum f(x_j)dx_j$
\item
その極限値が定積分である.\vspace{2mm}\\
\hspace*{2zw}$\displaystyle\int_a^b f(x)\,dx=\lim_{n \to \infty}\sum_j f(x_j)dx_j$
\item
[注)]$f(x)$が負の場合もこの式は有効である.
\end{itemize}

\newslide{基本定理と定積分の計算公式}

\vspace*{18mm}


\begin{layer}{120}{0}
\putnotew{96}{73}{\hyperlink{para2pg8}{\fbox{\Ctab{2.5mm}{\scalebox{1}{\scriptsize $\mathstrut||\!\lhd$}}}}}
\putnotew{101}{73}{\hyperlink{para3pg1}{\fbox{\Ctab{2.5mm}{\scalebox{1}{\scriptsize $\mathstrut|\!\lhd$}}}}}
\putnotew{108}{73}{\hyperlink{para3pg3}{\fbox{\Ctab{4.5mm}{\scalebox{1}{\scriptsize $\mathstrut\!\!\lhd\!\!$}}}}}
\putnotew{115}{73}{\hyperlink{para3pg4}{\fbox{\Ctab{4.5mm}{\scalebox{1}{\scriptsize $\mathstrut\!\rhd\!$}}}}}
\putnotew{120}{73}{\hyperlink{para3pg4}{\fbox{\Ctab{2.5mm}{\scalebox{1}{\scriptsize $\mathstrut \!\rhd\!\!|$}}}}}
\putnotew{125}{73}{\hyperlink{para4pg1}{\fbox{\Ctab{2.5mm}{\scalebox{1}{\scriptsize $\mathstrut \!\rhd\!\!||$}}}}}
\putnotee{126}{73}{\scriptsize\color{blue} 4/4}
\end{layer}

\slidepage
\begin{itemize}
\item
基本定理\\
\hspace*{1zw}$\dint_a^x f(x)\,dx$は$f(x)$の不定積分の1つである.\\
\hspace*{3zw}$\left(\dint_a^x f(x)\,dx\right)'=f(x)$
\item
計算公式\\
\hspace*{1zw}$f(x)$の不定積分の1つを$F(x)$とすると\\
\hspace*{3zw}$\dint_a^b f(x)\,dx=F(b)-F(a)=\Bigl[F(x)\Bigr]_a^b$
\end{itemize}

\newslide{定積分の計算例1}

\vspace*{18mm}


\begin{layer}{120}{0}
\putnotew{96}{73}{\hyperlink{para3pg4}{\fbox{\Ctab{2.5mm}{\scalebox{1}{\scriptsize $\mathstrut||\!\lhd$}}}}}
\putnotew{101}{73}{\hyperlink{para4pg1}{\fbox{\Ctab{2.5mm}{\scalebox{1}{\scriptsize $\mathstrut|\!\lhd$}}}}}
\putnotew{108}{73}{\hyperlink{para4pg3}{\fbox{\Ctab{4.5mm}{\scalebox{1}{\scriptsize $\mathstrut\!\!\lhd\!\!$}}}}}
\putnotew{115}{73}{\hyperlink{para4pg4}{\fbox{\Ctab{4.5mm}{\scalebox{1}{\scriptsize $\mathstrut\!\rhd\!$}}}}}
\putnotew{120}{73}{\hyperlink{para4pg4}{\fbox{\Ctab{2.5mm}{\scalebox{1}{\scriptsize $\mathstrut \!\rhd\!\!|$}}}}}
\putnotew{125}{73}{\hyperlink{para5pg1}{\fbox{\Ctab{2.5mm}{\scalebox{1}{\scriptsize $\mathstrut \!\rhd\!\!||$}}}}}
\putnotee{126}{73}{\scriptsize\color{blue} 4/4}
\end{layer}

\slidepage

\begin{layer}{120}{0}
\putnotese{85}{10}{%%% /Users/takatoosetsuo/polytech22.git/205-0912/presen/fig/sekibun1a.tex 
%%% Generator=presen22206.cdy 
{\unitlength=1cm%
\begin{picture}%
(3.5,4)(-0.5,-1.5)%
\linethickness{0.008in}%%
\Large\bf\boldmath%
\small%
\polyline(-0.50000,1.25000)(-0.43000,1.04490)(-0.36000,0.84960)(-0.29000,0.66410)%
(-0.22000,0.48840)(-0.15000,0.32250)(-0.08000,0.16640)(-0.01000,0.02010)(0.06000,-0.11640)%
(0.13000,-0.24310)(0.20000,-0.36000)(0.27000,-0.46710)(0.34000,-0.56440)(0.41000,-0.65190)%
(0.48000,-0.72960)(0.55000,-0.79750)(0.62000,-0.85560)(0.69000,-0.90390)(0.76000,-0.94240)%
(0.83000,-0.97110)(0.90000,-0.99000)(0.97000,-0.99910)(1.04000,-0.99840)(1.11000,-0.98790)%
(1.18000,-0.96760)(1.25000,-0.93750)(1.32000,-0.89760)(1.39000,-0.84790)(1.46000,-0.78840)%
(1.53000,-0.71910)(1.60000,-0.64000)(1.67000,-0.55110)(1.74000,-0.45240)(1.81000,-0.34390)%
(1.88000,-0.22560)(1.95000,-0.09750)(2.02000,0.04040)(2.09000,0.18810)(2.16000,0.34560)%
(2.23000,0.51290)(2.30000,0.69000)(2.37000,0.87690)(2.44000,1.07360)(2.51000,1.28010)%
(2.58000,1.49640)(2.65000,1.72250)(2.72000,1.95840)(2.79000,2.20410)(2.86000,2.45960)%
(2.87066,2.50000)%
%
\polyline(2.00000,0.05000)(2.00000,-0.05000)%
%
\settowidth{\Width}{$2$}\setlength{\Width}{0\Width}%
\settoheight{\Height}{$2$}\settodepth{\Depth}{$2$}\setlength{\Height}{-\Height}%
\put(2.0500000,-0.1000000){\hspace*{\Width}\raisebox{\Height}{$2$}}%
%
\polyline(0.00031,0.00000)(0.06000,-0.11640)(0.13000,-0.24310)(0.20000,-0.36000)(0.27000,-0.46710)%
(0.34000,-0.56440)(0.41000,-0.65190)(0.48000,-0.72960)(0.55000,-0.79750)(0.62000,-0.85560)%
(0.69000,-0.90390)(0.76000,-0.94240)(0.83000,-0.97110)(0.90000,-0.99000)(0.97000,-0.99910)%
(1.04000,-0.99840)(1.11000,-0.98790)(1.18000,-0.96760)(1.25000,-0.93750)(1.32000,-0.89760)%
(1.39000,-0.84790)(1.46000,-0.78840)(1.53000,-0.71910)(1.60000,-0.64000)(1.67000,-0.55110)%
(1.74000,-0.45240)(1.81000,-0.34390)(1.88000,-0.22560)(1.95000,-0.09750)(1.99949,0.00000)%
(0.00031,0.00000)%
%
\polyline(1.20915,-0.95507)(1.79125,-0.37297)%
%
\polyline(0.83753,-0.97313)(1.81066,0.00000)%
%
\polyline(0.60989,-0.84721)(1.45711,0.00000)%
%
\polyline(0.42974,-0.67381)(1.10355,0.00000)%
%
\polyline(0.27540,-0.47460)(0.75000,0.00000)%
%
\polyline(0.13874,-0.25770)(0.39645,0.00000)%
%
\polyline(0.01474,-0.02815)(0.04289,0.00000)%
%
\polyline(-0.50000,0.00000)(3.00000,0.00000)%
%
\polyline(0.00000,-1.50000)(0.00000,2.50000)%
%
\settowidth{\Width}{$x$}\setlength{\Width}{0\Width}%
\settoheight{\Height}{$x$}\settodepth{\Depth}{$x$}\setlength{\Height}{-0.5\Height}\setlength{\Depth}{0.5\Depth}\addtolength{\Height}{\Depth}%
\put(3.0500000,0.0000000){\hspace*{\Width}\raisebox{\Height}{$x$}}%
%
\settowidth{\Width}{$y$}\setlength{\Width}{-0.5\Width}%
\settoheight{\Height}{$y$}\settodepth{\Depth}{$y$}\setlength{\Height}{\Depth}%
\put(0.0000000,2.5500000){\hspace*{\Width}\raisebox{\Height}{$y$}}%
%
\settowidth{\Width}{O}\setlength{\Width}{-1\Width}%
\settoheight{\Height}{O}\settodepth{\Depth}{O}\setlength{\Height}{-\Height}%
\put(-0.0500000,-0.0500000){\hspace*{\Width}\raisebox{\Height}{O}}%
%
\end{picture}}%}
\end{layer}

\begin{itemize}
\item
$\dint_0^2 (3x^2-2x)\,dx$\\
\hspace*{2zw}$=\Bigl[\bunsuu{1}{3}x^3-x^2\Bigr]_0^2$\vspace{1mm}\\
\hspace*{2zw}$=\bunsuu{8}{3}-4=-\bunsuu{4}{3}$
\item
[課題]\monbannoadd 問いに答えよ.\seteda{65}\\
\eda{なぜマイナスか.理由を書け}\\
\eda{図の斜線部分の面積を答えよ}
\end{itemize}
\addban

\newslide{定積分の計算例2}

\vspace*{18mm}


\begin{layer}{120}{0}
\putnotew{96}{73}{\hyperlink{para4pg4}{\fbox{\Ctab{2.5mm}{\scalebox{1}{\scriptsize $\mathstrut||\!\lhd$}}}}}
\putnotew{101}{73}{\hyperlink{para5pg1}{\fbox{\Ctab{2.5mm}{\scalebox{1}{\scriptsize $\mathstrut|\!\lhd$}}}}}
\putnotew{108}{73}{\hyperlink{para5pg3}{\fbox{\Ctab{4.5mm}{\scalebox{1}{\scriptsize $\mathstrut\!\!\lhd\!\!$}}}}}
\putnotew{115}{73}{\hyperlink{para5pg4}{\fbox{\Ctab{4.5mm}{\scalebox{1}{\scriptsize $\mathstrut\!\rhd\!$}}}}}
\putnotew{120}{73}{\hyperlink{para5pg4}{\fbox{\Ctab{2.5mm}{\scalebox{1}{\scriptsize $\mathstrut \!\rhd\!\!|$}}}}}
\putnotew{125}{73}{\hyperlink{para6pg1}{\fbox{\Ctab{2.5mm}{\scalebox{1}{\scriptsize $\mathstrut \!\rhd\!\!||$}}}}}
\putnotee{126}{73}{\scriptsize\color{blue} 4/4}
\end{layer}

\slidepage

\begin{layer}{120}{0}
\putnotese{70}{10}{%%% /Users/takatoosetsuo/polytech22.git/205-0912/presen/fig/sekibun2.tex 
%%% Generator=presen22206.cdy 
{\unitlength=15mm%
\begin{picture}%
(3,3)(-1.5,-1.5)%
\linethickness{0.008in}%%
\Large\bf\boldmath%
\small%
\polyline(-1.14449,-1.50000)(-1.14000,-1.48154)(-1.08000,-1.25971)(-1.02000,-1.06121)%
(-0.96000,-0.88474)(-0.90000,-0.72900)(-0.84000,-0.59270)(-0.78000,-0.47455)(-0.72000,-0.37325)%
(-0.66000,-0.28750)(-0.60000,-0.21600)(-0.54000,-0.15746)(-0.48000,-0.11059)(-0.42000,-0.07409)%
(-0.36000,-0.04666)(-0.30000,-0.02700)(-0.24000,-0.01382)(-0.18000,-0.00583)(-0.12000,-0.00173)%
(-0.06000,-0.00022)(0.00000,0.00000)(0.06000,0.00022)(0.12000,0.00173)(0.18000,0.00583)%
(0.24000,0.01382)(0.30000,0.02700)(0.36000,0.04666)(0.42000,0.07409)(0.48000,0.11059)%
(0.54000,0.15746)(0.60000,0.21600)(0.66000,0.28750)(0.72000,0.37325)(0.78000,0.47455)%
(0.84000,0.59270)(0.90000,0.72900)(0.96000,0.88474)(1.02000,1.06121)(1.08000,1.25971)%
(1.14000,1.48154)(1.14449,1.50000)%
%
\polyline(-1.00000,0.03333)(-1.00000,-0.03333)%
%
\settowidth{\Width}{$-1$}\setlength{\Width}{-0.5\Width}%
\settoheight{\Height}{$-1$}\settodepth{\Depth}{$-1$}\setlength{\Height}{\Depth}%
\put(-1.0000000,0.0666667){\hspace*{\Width}\raisebox{\Height}{$-1$}}%
%
\polyline(1.00000,0.03333)(1.00000,-0.03333)%
%
\settowidth{\Width}{$1$}\setlength{\Width}{-0.5\Width}%
\settoheight{\Height}{$1$}\settodepth{\Depth}{$1$}\setlength{\Height}{-\Height}%
\put(1.0000000,-0.0666667){\hspace*{\Width}\raisebox{\Height}{$1$}}%
%
\polyline(-1.00000,-1.00000)(-1.00000,0.00000)%
%
\polyline(-1.00000,-1.00000)(-0.99877,-0.99877)%
%
\polyline(-1.00000,-0.88215)(-0.93331,-0.81546)%
%
\polyline(-0.11957,-0.00172)(-0.11785,0.00000)%
%
\polyline(-1.00000,-0.76430)(-0.84912,-0.61342)%
%
\polyline(-0.25221,-0.01650)(-0.23570,0.00000)%
%
\polyline(-1.00000,-0.64645)(-0.70415,-0.35060)%
%
\polyline(-0.43951,-0.08596)(-0.35355,0.00000)%
%
\polyline(-1.00000,-0.52860)(-0.47140,0.00000)%
%
\polyline(-1.00000,-0.41074)(-0.58926,0.00000)%
%
\polyline(-1.00000,-0.29289)(-0.70711,0.00000)%
%
\polyline(-1.00000,-0.17504)(-0.82496,0.00000)%
%
\polyline(-1.00000,-0.05719)(-0.94281,0.00000)%
%
\polyline(1.00000,0.00000)(1.00000,1.00000)%
%
\polyline(0.94281,0.00000)(1.00000,0.05719)%
%
\polyline(0.82496,0.00000)(1.00000,0.17504)%
%
\polyline(0.70711,0.00000)(1.00000,0.29289)%
%
\polyline(0.58926,0.00000)(1.00000,0.41074)%
%
\polyline(0.47140,0.00000)(1.00000,0.52860)%
%
\polyline(0.35355,0.00000)(0.43951,0.08596)%
%
\polyline(0.70415,0.35060)(1.00000,0.64645)%
%
\polyline(0.23570,0.00000)(0.25221,0.01650)%
%
\polyline(0.84912,0.61342)(1.00000,0.76430)%
%
\polyline(0.11785,0.00000)(0.11957,0.00172)%
%
\polyline(0.93331,0.81546)(1.00000,0.88215)%
%
\polyline(0.99877,0.99877)(1.00000,1.00000)%
%
\settowidth{\Width}{$y=x^3$}\setlength{\Width}{0\Width}%
\settoheight{\Height}{$y=x^3$}\settodepth{\Depth}{$y=x^3$}\setlength{\Height}{-0.5\Height}\setlength{\Depth}{0.5\Depth}\addtolength{\Height}{\Depth}%
\put(1.2633333,1.6200000){\hspace*{\Width}\raisebox{\Height}{$y=x^3$}}%
%
\polyline(-1.50000,0.00000)(1.50000,0.00000)%
%
\polyline(0.00000,-1.50000)(0.00000,1.50000)%
%
\settowidth{\Width}{$x$}\setlength{\Width}{0\Width}%
\settoheight{\Height}{$x$}\settodepth{\Depth}{$x$}\setlength{\Height}{-0.5\Height}\setlength{\Depth}{0.5\Depth}\addtolength{\Height}{\Depth}%
\put(1.5333333,0.0000000){\hspace*{\Width}\raisebox{\Height}{$x$}}%
%
\settowidth{\Width}{$y$}\setlength{\Width}{-0.5\Width}%
\settoheight{\Height}{$y$}\settodepth{\Depth}{$y$}\setlength{\Height}{\Depth}%
\put(0.0000000,1.5333333){\hspace*{\Width}\raisebox{\Height}{$y$}}%
%
\settowidth{\Width}{O}\setlength{\Width}{-1\Width}%
\settoheight{\Height}{O}\settodepth{\Depth}{O}\setlength{\Height}{-\Height}%
\put(-0.0333333,-0.0333333){\hspace*{\Width}\raisebox{\Height}{O}}%
%
\end{picture}}%}
\end{layer}

\begin{itemize}
\item
$\dint_{-1}^1 x^3\,dx$\\
\hspace*{2zw}$=\Bigl[\bunsuu{1}{4}x^4\Bigr]_{-1}^1$\vspace{1mm}\\
\hspace*{2zw}$=\bunsuu{1}{4}-\bunsuu{1}{4}=0$
\item
[課題]\monbannoadd 問いに答えよ.\seteda{65}\\
\eda{なぜ0か.理由を書け}\\
\eda{図の斜線部分の面積を答えよ}
\end{itemize}
\addban

\mainslide{三角関数の定積分}


\slidepage[m]
%%%%%%%%%%%%

%%%%%%%%%%%%%%%%%%%%

\newslide{三角関数(復習)}

\vspace*{18mm}

\slidepage

\begin{layer}{140}{0}
\putnotese{88}{-10}{%%% /Users/takatoosetsuo/polytech22.git/205-0912/presen/fig/sankakuatai.tex 
%%% Generator=presen22206.cdy 
{\unitlength=15mm%
\begin{picture}%
(2.4,2.4)(-1.2,-1.2)%
\linethickness{0.008in}%%
\Large\bf\boldmath%
\small%
\newcommand{\ssbunsuu}[2]{\scalebox{0.6}{$\bunsuu{#1}{#2}$}}%
\polyline(1.00000,0.00000)(0.99211,0.12533)(0.96858,0.24869)(0.92978,0.36812)(0.87631,0.48175)%
(0.80902,0.58779)(0.72897,0.68455)(0.63742,0.77051)(0.53583,0.84433)(0.42578,0.90483)%
(0.30902,0.95106)(0.18738,0.98229)(0.06279,0.99803)(-0.06279,0.99803)(-0.18738,0.98229)%
(-0.30902,0.95106)(-0.42578,0.90483)(-0.53583,0.84433)(-0.63742,0.77051)(-0.72897,0.68455)%
(-0.80902,0.58779)(-0.87631,0.48175)(-0.92978,0.36812)(-0.96858,0.24869)(-0.99211,0.12533)%
(-1.00000,-0.00000)(-0.99211,-0.12533)(-0.96858,-0.24869)(-0.92978,-0.36812)(-0.87631,-0.48175)%
(-0.80902,-0.58779)(-0.72897,-0.68455)(-0.63742,-0.77051)(-0.53583,-0.84433)(-0.42578,-0.90483)%
(-0.30902,-0.95106)(-0.18738,-0.98229)(-0.06279,-0.99803)(0.06279,-0.99803)(0.18738,-0.98229)%
(0.30902,-0.95106)(0.42578,-0.90483)(0.53583,-0.84433)(0.63742,-0.77051)(0.72897,-0.68455)%
(0.80902,-0.58779)(0.87631,-0.48175)(0.92978,-0.36812)(0.96858,-0.24869)(0.99211,-0.12533)%
(1.00000,-0.00000)%
%
{%
\color[cmyk]{0,1,1,0}%
\linethickness{0.024in}%%
\polyline(0.00000,0.00000)(1.00000,0.00000)%
%
\linethickness{0.008in}%%
}%
{%
\color[cmyk]{0,1,1,0}%
\linethickness{0.024in}%%
\polyline(0.00000,0.00000)(0.86603,0.50000)%
%
\linethickness{0.008in}%%
}%
{%
\color[cmyk]{0,1,1,0}%
\linethickness{0.024in}%%
\polyline(0.00000,0.00000)(0.70711,0.70711)%
%
\linethickness{0.008in}%%
}%
{%
\color[cmyk]{0,1,1,0}%
\linethickness{0.024in}%%
\polyline(0.00000,0.00000)(0.50000,0.86603)%
%
\linethickness{0.008in}%%
}%
{%
\color[cmyk]{0,1,1,0}%
\linethickness{0.024in}%%
\polyline(0.00000,0.00000)(0.00000,1.00000)%
%
\linethickness{0.008in}%%
}%
{%
\color[cmyk]{0,1,1,0}%
\linethickness{0.024in}%%
\polyline(0.00000,0.00000)(-1.00000,0.00000)%
%
\linethickness{0.008in}%%
}%
\put(0.86603,0.50000){\circle*{0.013547}}\put(0.86603,0.43750){\circle*{0.013547}}
\put(0.86603,0.37500){\circle*{0.013547}}\put(0.86603,0.31250){\circle*{0.013547}}
\put(0.86603,0.25000){\circle*{0.013547}}\put(0.86603,0.18750){\circle*{0.013547}}
\put(0.86603,0.12500){\circle*{0.013547}}\put(0.86603,0.06250){\circle*{0.013547}}
\put(0.86603,0.00000){\circle*{0.013547}}
\linethickness{0.008in}%%
\put(0.70711,0.70711){\circle*{0.013547}}\put(0.70711,0.64283){\circle*{0.013547}}
\put(0.70711,0.57854){\circle*{0.013547}}\put(0.70711,0.51426){\circle*{0.013547}}
\put(0.70711,0.44998){\circle*{0.013547}}\put(0.70711,0.38570){\circle*{0.013547}}
\put(0.70711,0.32141){\circle*{0.013547}}\put(0.70711,0.25713){\circle*{0.013547}}
\put(0.70711,0.19285){\circle*{0.013547}}\put(0.70711,0.12857){\circle*{0.013547}}
\put(0.70711,0.06428){\circle*{0.013547}}\put(0.70711,0.00000){\circle*{0.013547}}
\linethickness{0.008in}%%
\put(0.50000,0.86603){\circle*{0.013547}}\put(0.50000,0.79941){\circle*{0.013547}}
\put(0.50000,0.73279){\circle*{0.013547}}\put(0.50000,0.66618){\circle*{0.013547}}
\put(0.50000,0.59956){\circle*{0.013547}}\put(0.50000,0.53294){\circle*{0.013547}}
\put(0.50000,0.46632){\circle*{0.013547}}\put(0.50000,0.39971){\circle*{0.013547}}
\put(0.50000,0.33309){\circle*{0.013547}}\put(0.50000,0.26647){\circle*{0.013547}}
\put(0.50000,0.19985){\circle*{0.013547}}\put(0.50000,0.13324){\circle*{0.013547}}
\put(0.50000,0.06662){\circle*{0.013547}}\put(0.50000,0.00000){\circle*{0.013547}}
\linethickness{0.008in}%%
\polyline(-1.00000,0.03333)(-1.00000,-0.03333)%
%
\settowidth{\Width}{$-1$}\setlength{\Width}{-1\Width}%
\settoheight{\Height}{$-1$}\settodepth{\Depth}{$-1$}\setlength{\Height}{-\Height}%
\put(-1.0333333,-0.0666667){\hspace*{\Width}\raisebox{\Height}{$-1$}}%
%
\polyline(1.00000,0.03333)(1.00000,-0.03333)%
%
\settowidth{\Width}{$1$}\setlength{\Width}{0\Width}%
\settoheight{\Height}{$1$}\settodepth{\Depth}{$1$}\setlength{\Height}{-\Height}%
\put(1.0333333,-0.0666667){\hspace*{\Width}\raisebox{\Height}{$1$}}%
%
\polyline(0.03333,-1.00000)(-0.03333,-1.00000)%
%
\settowidth{\Width}{$-1$}\setlength{\Width}{-1\Width}%
\settoheight{\Height}{$-1$}\settodepth{\Depth}{$-1$}\setlength{\Height}{-\Height}%
\put(-0.0666667,-1.0333333){\hspace*{\Width}\raisebox{\Height}{$-1$}}%
%
\polyline(0.03333,1.00000)(-0.03333,1.00000)%
%
\settowidth{\Width}{$1$}\setlength{\Width}{-1\Width}%
\settoheight{\Height}{$1$}\settodepth{\Depth}{$1$}\setlength{\Height}{\Depth}%
\put(-0.0666667,1.0333333){\hspace*{\Width}\raisebox{\Height}{$1$}}%
%
\polygon*(0.53054,-0.12979)(0.50000,0.00000)(0.48062,-0.13192)(0.50447,-0.10468)(0.53054,-0.12979)%
(0.53054,-0.12979)\linethickness{0.001in}%%
\polyline(0.53054,-0.12979)(0.50000,0.00000)(0.48062,-0.13192)(0.50447,-0.10468)(0.53054,-0.12979)%
%
\linethickness{0.008in}%%
\linethickness{0.004in}%%
\polyline(0.54571,-1.07120)(0.50447,-0.10468)%
%
\linethickness{0.008in}%%
\polygon*(0.74479,-0.12790)(0.70711,0.00000)(0.69506,-0.13279)(0.71736,-0.10427)(0.74479,-0.12790)%
(0.74479,-0.12790)\linethickness{0.001in}%%
\polyline(0.74479,-0.12790)(0.70711,0.00000)(0.69506,-0.13279)(0.71736,-0.10427)(0.74479,-0.12790)%
%
\linethickness{0.008in}%%
\linethickness{0.004in}%%
\polyline(0.80845,-1.03078)(0.71736,-0.10427)%
%
\linethickness{0.008in}%%
\polygon*(0.91702,-0.12320)(0.86603,0.00000)(0.86809,-0.13332)(0.88725,-0.10261)(0.91702,-0.12320)%
(0.91702,-0.12320)\linethickness{0.001in}%%
\polyline(0.91702,-0.12320)(0.86603,0.00000)(0.86809,-0.13332)(0.88725,-0.10261)(0.91702,-0.12320)%
%
\linethickness{0.008in}%%
\linethickness{0.004in}%%
\polyline(1.08130,-1.04088)(0.88725,-0.10261)%
%
\linethickness{0.008in}%%
\small%
\settowidth{\Width}{$\theta=\bunsuu{\pi}{6}$}\setlength{\Width}{0\Width}%
\settoheight{\Height}{$\theta=\bunsuu{\pi}{6}$}\settodepth{\Depth}{$\theta=\bunsuu{\pi}{6}$}\setlength{\Height}{\Depth}%
\put(0.9033333,0.3333333){\hspace*{\Width}\raisebox{\Height}{$\theta=\bunsuu{\pi}{6}$}}%
%
\settowidth{\Width}{$\theta=\bunsuu{\pi}{4}$}\setlength{\Width}{0\Width}%
\settoheight{\Height}{$\theta=\bunsuu{\pi}{4}$}\settodepth{\Depth}{$\theta=\bunsuu{\pi}{4}$}\setlength{\Height}{\Depth}%
\put(0.7433333,0.6100000){\hspace*{\Width}\raisebox{\Height}{$\theta=\bunsuu{\pi}{4}$}}%
%
\settowidth{\Width}{$\theta=\bunsuu{\pi}{3}$}\setlength{\Width}{0\Width}%
\settoheight{\Height}{$\theta=\bunsuu{\pi}{3}$}\settodepth{\Depth}{$\theta=\bunsuu{\pi}{3}$}\setlength{\Height}{\Depth}%
\put(0.5333333,0.9033333){\hspace*{\Width}\raisebox{\Height}{$\theta=\bunsuu{\pi}{3}$}}%
%
\settowidth{\Width}{$\ssbunsuu{1}{2}$}\setlength{\Width}{-0.5\Width}%
\settoheight{\Height}{$\ssbunsuu{1}{2}$}\settodepth{\Depth}{$\ssbunsuu{1}{2}$}\setlength{\Height}{-\Height}%
\put(0.5500000,-1.1033333){\hspace*{\Width}\raisebox{\Height}{$\ssbunsuu{1}{2}$}}%
%
\settowidth{\Width}{$\ssbunsuu{1}{\sqrt{2}}$}\setlength{\Width}{-0.5\Width}%
\settoheight{\Height}{$\ssbunsuu{1}{\sqrt{2}}$}\settodepth{\Depth}{$\ssbunsuu{1}{\sqrt{2}}$}\setlength{\Height}{-\Height}%
\put(0.8100000,-1.0633333){\hspace*{\Width}\raisebox{\Height}{$\ssbunsuu{1}{\sqrt{2}}$}}%
%
\settowidth{\Width}{$\ssbunsuu{\sqrt{3}}{2}$}\setlength{\Width}{-0.5\Width}%
\settoheight{\Height}{$\ssbunsuu{\sqrt{3}}{2}$}\settodepth{\Depth}{$\ssbunsuu{\sqrt{3}}{2}$}\setlength{\Height}{-\Height}%
\put(1.0800000,-1.0733333){\hspace*{\Width}\raisebox{\Height}{$\ssbunsuu{\sqrt{3}}{2}$}}%
%
\normalsize%
\polyline(-1.20000,0.00000)(1.20000,0.00000)%
%
\polyline(0.00000,-1.20000)(0.00000,1.20000)%
%
\settowidth{\Width}{$x$}\setlength{\Width}{0\Width}%
\settoheight{\Height}{$x$}\settodepth{\Depth}{$x$}\setlength{\Height}{-0.5\Height}\setlength{\Depth}{0.5\Depth}\addtolength{\Height}{\Depth}%
\put(1.2333333,0.0000000){\hspace*{\Width}\raisebox{\Height}{$x$}}%
%
\settowidth{\Width}{$y$}\setlength{\Width}{-0.5\Width}%
\settoheight{\Height}{$y$}\settodepth{\Depth}{$y$}\setlength{\Height}{\Depth}%
\put(0.0000000,1.2333333){\hspace*{\Width}\raisebox{\Height}{$y$}}%
%
\settowidth{\Width}{O}\setlength{\Width}{-1\Width}%
\settoheight{\Height}{O}\settodepth{\Depth}{O}\setlength{\Height}{-\Height}%
\put(-0.0333333,-0.0333333){\hspace*{\Width}\raisebox{\Height}{O}}%
%
\end{picture}}%}
\putnotese{85}{58}{\normalsize\color{blue}$\dint \cos ax\,dx=\bunsuu{1}{a}\sin ax+C$}
\end{layer}

\begin{itemize}
\item
[課題]\monban 次の値を求めよ.\seteda{28}\\
\eda{$\sin 0$}\eda{$\cos 0$}\eda{$\sin\pi$}\vspace{1mm}\\
\eda{$\cos\pi$}\eda{$\sin\bunsuu{\pi}{2}$}\eda{$\cos\bunsuu{\pi}{2}$}\vspace{1mm}\\
\eda{$\sin\bunsuu{\pi}{6}$}\eda{$\cos\bunsuu{\pi}{4}$}\eda{$\cos\bunsuu{\pi}{3}$}
\item
[課題]\monban 次の不定積分を求めよ(積分定数$C$)\seteda{45}\\
\eda{$\dint \sin x\,dx$}\eda{$\dint\cos x\,dx$}\vspace{2mm}\\
\eda{$\dint \cos 2x\,dx$}\eda{$\dint \sin 3x\,dx$}
\end{itemize}

\newslide{課題(三角関数の定積分)}

\vspace*{18mm}

%%repeat=1
\slidepage

\begin{layer}{120}{0}
\putnotese{25}{45}{\scalebox{0.75}{%%% /Users/takatoosetsuo/Dropbox/2018polytec/lecture/0901/presen/fig/intsine.tex 
%%% Generator=presen0901.cdy 
{\unitlength=1cm%
\begin{picture}%
(13,4)(-6.5,-2)%
\special{pn 8}%
%
\Large\bf\boldmath%
\normalsize%
{%
\color[rgb]{0,0,0}%
\special{pn 12}%
\special{pa -2559    85}\special{pa -2508    34}\special{pa -2457   -17}\special{pa -2406   -68}%
\special{pa -2354  -118}\special{pa -2303  -165}\special{pa -2252  -210}\special{pa -2201  -252}%
\special{pa -2150  -289}\special{pa -2098  -321}\special{pa -2047  -348}\special{pa -1996  -369}%
\special{pa -1945  -384}\special{pa -1894  -392}\special{pa -1843  -393}\special{pa -1791  -389}%
\special{pa -1740  -377}\special{pa -1689  -359}\special{pa -1638  -335}\special{pa -1587  -306}%
\special{pa -1535  -271}\special{pa -1484  -231}\special{pa -1433  -188}\special{pa -1382  -142}%
\special{pa -1331   -93}\special{pa -1280   -43}\special{pa -1228     9}\special{pa -1177    59}%
\special{pa -1126   109}\special{pa -1075   158}\special{pa -1024   203}\special{pa  -972   245}%
\special{pa  -921   283}\special{pa  -870   316}\special{pa  -819   344}\special{pa  -768   366}%
\special{pa  -717   382}\special{pa  -665   391}\special{pa  -614   394}\special{pa  -563   390}%
\special{pa  -512   379}\special{pa  -461   363}\special{pa  -409   340}\special{pa  -358   311}%
\special{pa  -307   277}\special{pa  -256   238}\special{pa  -205   196}\special{pa  -154   150}%
\special{pa  -102   101}\special{pa   -51    51}\special{pa     0    -0}\special{pa    51   -51}%
\special{pa   102  -101}\special{pa   154  -150}\special{pa   205  -196}\special{pa   256  -238}%
\special{pa   307  -277}\special{pa   358  -311}\special{pa   409  -340}\special{pa   461  -363}%
\special{pa   512  -379}\special{pa   563  -390}\special{pa   614  -394}\special{pa   665  -391}%
\special{pa   717  -382}\special{pa   768  -366}\special{pa   819  -344}\special{pa   870  -316}%
\special{pa   921  -283}\special{pa   972  -245}\special{pa  1024  -203}\special{pa  1075  -158}%
\special{pa  1126  -109}\special{pa  1177   -59}\special{pa  1228    -9}\special{pa  1280    43}%
\special{pa  1331    93}\special{pa  1382   142}\special{pa  1433   188}\special{pa  1484   231}%
\special{pa  1535   271}\special{pa  1587   306}\special{pa  1638   335}\special{pa  1689   359}%
\special{pa  1740   377}\special{pa  1791   389}\special{pa  1843   393}\special{pa  1894   392}%
\special{pa  1945   384}\special{pa  1996   369}\special{pa  2047   348}\special{pa  2098   321}%
\special{pa  2150   289}\special{pa  2201   252}\special{pa  2252   210}\special{pa  2303   165}%
\special{pa  2354   118}\special{pa  2406    68}\special{pa  2457    17}\special{pa  2508   -34}%
\special{pa  2559   -85}%
\special{fp}%
\special{pn 8}%
}%
{%
\color[rgb]{0,0,0}%
}%
{%
\color[rgb]{0,0,0}%
}%
{%
\color[rgb]{0,0,0}%
\special{pa  1114     0}\special{pa  1175   -62}%
\special{fp}%
\special{pa   974     0}\special{pa  1104  -130}%
\special{fp}%
\special{pa   835     0}\special{pa  1031  -196}%
\special{fp}%
\special{pa   696     0}\special{pa   954  -258}%
\special{fp}%
\special{pa   557     0}\special{pa   872  -315}%
\special{fp}%
\special{pa   418     0}\special{pa   779  -361}%
\special{fp}%
\special{pa   278     0}\special{pa   669  -390}%
\special{fp}%
\special{pa   139     0}\special{pa   520  -381}%
\special{fp}%
}%
{%
\color[rgb]{0,0,0}%
\settowidth{\Width}{$y=\sin x$}\setlength{\Width}{0\Width}%
\settoheight{\Height}{$y=\sin x$}\settodepth{\Depth}{$y=\sin x$}\setlength{\Height}{-0.5\Height}\setlength{\Depth}{0.5\Depth}\addtolength{\Height}{\Depth}%
\put(2.2900000,1.0900000){\hspace*{\Width}\raisebox{\Height}{$y=\sin x$}}%
%
}%
{%
\color[rgb]{0,0,0}%
\special{pa  1237   -20}\special{pa  1237    20}%
\special{fp}%
}%
{%
\color[rgb]{0,0,0}%
\settowidth{\Width}{$\pi$}\setlength{\Width}{-0.5\Width}%
\settoheight{\Height}{$\pi$}\settodepth{\Depth}{$\pi$}\setlength{\Height}{-\Height}%
\put(3.1400000,-0.1000000){\hspace*{\Width}\raisebox{\Height}{$\pi$}}%
%
}%
{%
\color[rgb]{0,0,0}%
\special{pa    20   394}\special{pa   -20   394}%
\special{fp}%
}%
{%
\color[rgb]{0,0,0}%
\settowidth{\Width}{$-1$}\setlength{\Width}{-1\Width}%
\settoheight{\Height}{$-1$}\settodepth{\Depth}{$-1$}\setlength{\Height}{-0.5\Height}\setlength{\Depth}{0.5\Depth}\addtolength{\Height}{\Depth}%
\put(-0.1000000,-1.0000000){\hspace*{\Width}\raisebox{\Height}{$-1$}}%
%
}%
{%
\color[rgb]{0,0,0}%
\special{pa    20  -394}\special{pa   -20  -394}%
\special{fp}%
}%
{%
\color[rgb]{0,0,0}%
\settowidth{\Width}{$1$}\setlength{\Width}{-1\Width}%
\settoheight{\Height}{$1$}\settodepth{\Depth}{$1$}\setlength{\Height}{-0.5\Height}\setlength{\Depth}{0.5\Depth}\addtolength{\Height}{\Depth}%
\put(-0.1000000,1.0000000){\hspace*{\Width}\raisebox{\Height}{$1$}}%
%
}%
\special{pa -2559    -0}\special{pa  2559    -0}%
\special{fp}%
\special{pa     0   787}\special{pa     0  -787}%
\special{fp}%
\settowidth{\Width}{$x$}\setlength{\Width}{0\Width}%
\settoheight{\Height}{$x$}\settodepth{\Depth}{$x$}\setlength{\Height}{-0.5\Height}\setlength{\Depth}{0.5\Depth}\addtolength{\Height}{\Depth}%
\put(6.5500000,0.0000000){\hspace*{\Width}\raisebox{\Height}{$x$}}%
%
\settowidth{\Width}{$y$}\setlength{\Width}{-0.5\Width}%
\settoheight{\Height}{$y$}\settodepth{\Depth}{$y$}\setlength{\Height}{\Depth}%
\put(0.0000000,2.0500000){\hspace*{\Width}\raisebox{\Height}{$y$}}%
%
\settowidth{\Width}{O}\setlength{\Width}{0\Width}%
\settoheight{\Height}{O}\settodepth{\Depth}{O}\setlength{\Height}{-\Height}%
\put(0.0500000,-0.0500000){\hspace*{\Width}\raisebox{\Height}{O}}%
%
\end{picture}}%}}
\end{layer}

\seteda{60}
\begin{itemize}
\item
[課題]\monban 次を求めよ.\vspace{1mm}\\
\eda{$\dint_0^{\pi}(\bunsuu{1}{2}\cos x)\,dx$}\vspace{2mm}\\
\eda{$\dint_0^{\sbunsuu{\pi}{2}}(2\cos x-3\sin x)\,dx$}\vspace{2mm}\\
\eda{図の斜線部の面積}
\end{itemize}

%%%%%%%%%%%%%%%%%%%%

\mainslide{指数対数関数の定積分}


\slidepage[m]
%%%%%%%%%%%%

%%%%%%%%%%%%%%%%%%%%

\newslide{指数対数(復習)}

\vspace*{18mm}

\slidepage
\begin{itemize}
\item
$e$はネピアの数,$\log x$は自然対数($=\log_e x$)
\item
[課題]\monban 次の値を求めよ.\seteda{38}\\
\eda{$e^0$}\eda{$\log 1$}\eda{$\log e$}\vspace{1mm}
\item
[課題]\monban 次の関数を微分せよ.\seteda{38}\\
\eda{$y=e^x$}\eda{$y=e^{2x}$}\eda{$y=\log x$}\vspace{2mm}
\item
[課題]\monban 次の不定積分を求めよ.\seteda{38}\\
\eda{$\dint e^x\,dx$}\eda{$\dint e^{2x}\,dx$}\eda{$\dint\bunsuu{1}{x}\,dx$}\\
\hfill $(x>0)$
\end{itemize}
%%%%%%%%%%%%

%%%%%%%%%%%%%%%%%%%%


\newslide{対数関数の積分についての注}

\vspace*{18mm}


\begin{layer}{120}{0}
\putnotew{96}{73}{\hyperlink{para5pg1}{\fbox{\Ctab{2.5mm}{\scalebox{1}{\scriptsize $\mathstrut||\!\lhd$}}}}}
\putnotew{101}{73}{\hyperlink{para6pg1}{\fbox{\Ctab{2.5mm}{\scalebox{1}{\scriptsize $\mathstrut|\!\lhd$}}}}}
\putnotew{108}{73}{\hyperlink{para6pg7}{\fbox{\Ctab{4.5mm}{\scalebox{1}{\scriptsize $\mathstrut\!\!\lhd\!\!$}}}}}
\putnotew{115}{73}{\hyperlink{para6pg8}{\fbox{\Ctab{4.5mm}{\scalebox{1}{\scriptsize $\mathstrut\!\rhd\!$}}}}}
\putnotew{120}{73}{\hyperlink{para6pg8}{\fbox{\Ctab{2.5mm}{\scalebox{1}{\scriptsize $\mathstrut \!\rhd\!\!|$}}}}}
\putnotew{125}{73}{\hyperlink{para7pg1}{\fbox{\Ctab{2.5mm}{\scalebox{1}{\scriptsize $\mathstrut \!\rhd\!\!||$}}}}}
\putnotee{126}{73}{\scriptsize\color{blue} 8/8}
\end{layer}

\slidepage
{\color{red}

\begin{layer}{120}{0}
\putnotee{80}{28}{$(\log ax)'=a\cdot\bunsuu{1}{ax}$}
\end{layer}

}
\begin{itemize}
\item
$x<0$のとき,$\bunsuu{1}{x}$の不定積分を求める.
\item
[] $x<0$のとき $y=\log|x|=\log(-x)$\\
\hspace*{3zw}微分する\\
\hspace*{5zw}$y'=$
$(-1)\bunsuu{1}{-x}$
$=\bunsuu{1}{x}$
\item
$x$が負のとき $\bigr(\log(-x)\bigr)'=\bunsuu{1}{x}$
\item
\fbox{$x\neqq 0$のとき $\dint \bunsuu{1}{x}\,dx=\log|x|+C$}
\end{itemize}

\newslide{指数対数の定積分}

\vspace*{18mm}


\begin{layer}{120}{0}
\putnotew{96}{73}{\hyperlink{para6pg8}{\fbox{\Ctab{2.5mm}{\scalebox{1}{\scriptsize $\mathstrut||\!\lhd$}}}}}
\putnotew{101}{73}{\hyperlink{para7pg1}{\fbox{\Ctab{2.5mm}{\scalebox{1}{\scriptsize $\mathstrut|\!\lhd$}}}}}
\putnotew{108}{73}{\hyperlink{para7pg6}{\fbox{\Ctab{4.5mm}{\scalebox{1}{\scriptsize $\mathstrut\!\!\lhd\!\!$}}}}}
\putnotew{115}{73}{\hyperlink{para7pg7}{\fbox{\Ctab{4.5mm}{\scalebox{1}{\scriptsize $\mathstrut\!\rhd\!$}}}}}
\putnotew{120}{73}{\hyperlink{para7pg7}{\fbox{\Ctab{2.5mm}{\scalebox{1}{\scriptsize $\mathstrut \!\rhd\!\!|$}}}}}
\putnotew{125}{73}{\hyperlink{para8pg1}{\fbox{\Ctab{2.5mm}{\scalebox{1}{\scriptsize $\mathstrut \!\rhd\!\!||$}}}}}
\putnotee{126}{73}{\scriptsize\color{blue} 7/7}
\end{layer}

\slidepage
\begin{itemize}
\item
$\dint_0^1e^x\,dx=$
$\Bigl[e^x\Bigr]_0^1=$
$e^1-e^0=e-1$
\item
$\dint_1^e \bunsuu{1}{x}\,dx=$
$\Bigl[\log x\Bigr]_1^e=$
$\log e-\log 1=1-0=1$\vspace{4mm}
[課題]\monban 次の値を求めよ.\seteda{45}\\
\eda{$\dint_{-1}^1 e^x\,dx$}\eda{$\dint_e^{e^2}\bunsuu{1}{x}\,dx$}\\
\eda{$\dint_1^2 -2e^x\,dx$}\eda{$\dint_1^2 \bunsuu{x+1}{x}\,dx$}
\end{itemize}

\newslide{$e^{ax+b}$型の積分}

\vspace*{18mm}


\begin{layer}{120}{0}
\putnotew{96}{73}{\hyperlink{para7pg7}{\fbox{\Ctab{2.5mm}{\scalebox{1}{\scriptsize $\mathstrut||\!\lhd$}}}}}
\putnotew{101}{73}{\hyperlink{para8pg1}{\fbox{\Ctab{2.5mm}{\scalebox{1}{\scriptsize $\mathstrut|\!\lhd$}}}}}
\putnotew{108}{73}{\hyperlink{para8pg4}{\fbox{\Ctab{4.5mm}{\scalebox{1}{\scriptsize $\mathstrut\!\!\lhd\!\!$}}}}}
\putnotew{115}{73}{\hyperlink{para8pg5}{\fbox{\Ctab{4.5mm}{\scalebox{1}{\scriptsize $\mathstrut\!\rhd\!$}}}}}
\putnotew{120}{73}{\hyperlink{para8pg5}{\fbox{\Ctab{2.5mm}{\scalebox{1}{\scriptsize $\mathstrut \!\rhd\!\!|$}}}}}
\putnotew{125}{73}{\hyperlink{para9pg1}{\fbox{\Ctab{2.5mm}{\scalebox{1}{\scriptsize $\mathstrut \!\rhd\!\!||$}}}}}
\putnotee{126}{73}{\scriptsize\color{blue} 5/5}
\end{layer}

\slidepage
\begin{itemize}
\item
微分 $(e^{ax})'=ae^{ax},\ (e^{ax+b})'=ae^{ax+b}$
\item
積分 $\dint e^{ax}\,dx=\bunsuu{1}{a}e^{ax}+C$
\item
[例題]$\dint_0^1 e^{2x}\,dx=$
$\Bigl[\bunsuu{1}{2}e^{2x}\Bigr]_0^1=\bunsuu{1}{2}(e^2-1)$
\item
[課題]\monban 次の値を求めよ.\seteda{50}\\
\eda{$\dint_0^1 (e^x+e^{-x})\,dx$}
\eda{$\dint_0^1 (e^x+1)(e^x-1)\,dx$}
\end{itemize}
\label{pageend}\mbox{}

\end{document}

\documentclass{b5-kaku}
%\usepackage{bxpapersize}
\setlength{\oddsidemargin}{-8mm}
\setlength{\evensidemargin}{23mm}
\setlength{\topmargin}{-10mm}


\usepackage{amsmath,amssymb}
\usepackage{float,multicol,eclbkbox}
\usepackage{emath,emathE,emathMw,emathEy,eqlist}
  \enumSep{\topsep=0pt\parskip=0pt\parsep0pt\itemsep0pt}
\renewcommand{\tagform}[1]{(\theequation)}%
\preEqlabel{}%

\usepackage[hang,small]{caption}
    \setlength{\captionmargin}{2zw}
    \renewcommand{\captionlabelfont}{\bf}
\usepackage[dvips]{graphicx,color}
\usepackage{ketpic,ketlayer}
\usepackage{mymacros}
%


\begin{document}

\setcounter{chapter}{1}

\section*{片対数グラフ}

底が$10$の対数を{\bf 常用対数}といい,
数値計算の分野ではよく用いられる.

\begin{example}
$\log_{10} 1=0,\ \log_{15} 10=1,\ \log_{10}100 =2,\ %
\log_{10}0.1 =\log_{10}10^{-1}=-1$
\end{example}

\begin{layer}{130}{0}
\putnotese{65}{20}{\input{fig/fig1_3_katataisu_1pr.tex}}
\end{layer}

\begin{mawarikomi}<2>[18]{6.5cm}{}
$f(x)>0$であるとき,関数$y=f(x)$のグラフを,縦軸方向には値$y$の常用対数
$\log_{10}y$の値をとってかくことがある.これを{\bf 片対数グラフ}という.

片対数グラフでは,真数の値を縦軸の目盛りにするのが普通である.また,
$\log_{10} 1=0$より,目盛り1の横線を横軸とする.
目盛り1, 10, 100などに対応する横線の間隔はすべて等しいから,
横軸をどこにとってもよい.
\begin{example}
$y=2\cdot 5^{2-x}$において\\
\hspace*{1zw}$x=0,\ 1,\ 2,\ 3$のとき\\
\hspace*{3zw}$y=50,\ 10,\ 2,\ 0.4$\\
このときの点は図のようになる.
\end{example}

\end{mawarikomi}

指数関数$y=c\,a^x$\ \ ($c,\ a$は正の定数,$a\neqq 1$)について
\begin{equation}\label{logの等式}
\log_{10}y=\log_{10}\bigl(c a^x\bigr)=\bigl(\log_{10}a\bigr)x+\log_{10}c
\end{equation}
したがって,$\log_{10} y$は$x$の1次式で表されるから,
片対数グラフでは直線となる.また,逆も成り立つ.

(\ref{logの等式})より,直線の傾きおよび切片から$a,\ c$が求められる.

\begin{exercise}
2点$(0,\ 1.6)$,$(2,\ 10)$を通る関数$y=f(x)$のグラフは,片対数グラフ
では直線となる.この関数を定めよ.
\end{exercise}

\begin{layer}{150}{0}
\putnotese{70}{-3}{\input{fig/fig1_3_katataisu2_1pr}}

\end{layer}

\begin{solution}
\begin{mawarikomi}[10]{60mm}{}
片対数グラフをかいたとき,
2点の実際の座標は$(0,\ \log_{10}1.6)$\vspace{1mm}および
$(2,\log_{10}10)$となるから,直線の傾き$m$
および切片$b$は
\begin{align*}
m&=\frac{\log_{10}10-\log_{10}1.6}{2-0}\\
&=\frac{1}{2}\log_{10}\dfrac{10}{1.6}=\log_{10}(2.5)\\
b&=\log_{10}1.6
\end{align*}

(\ref{logの等式})より
\begin{align*}
\log_{10}a&=\log_{10}(2.5)\\ 
\log_{10}c&=\log_{10}{1.6}
\end{align*}
これから$a=2.5,\ c=1.6$ 

\noindent
したがって,求める関数は
\[
y=1.6\,(2.5)^x\tag*{\qed}
\]
\end{mawarikomi}
\end{solution}

\begin{question}
$(0,\ 1),\ (5,\ 0.1)$を通る関数のグラフは片対数グラフでは直線となる.
この関数を定めよ.ただし,$10^{-0.2}=0.63$とする.
\end{question}

\newpage

\begin{center}
{\large\bf 解 答}
\end{center}

\end{document}

\documentclass{b5-kaku}
%\usepackage{bxpapersize}
\setlength{\oddsidemargin}{-8mm}
\setlength{\evensidemargin}{23mm}
\setlength{\topmargin}{-10mm}


\usepackage{amsmath,amssymb}
\usepackage{float,multicol,eclbkbox}
\usepackage{emath,emathE,emathMw,emathEy,eqlist}
  \enumSep{\topsep=0pt\parskip=0pt\parsep0pt\itemsep0pt}
\renewcommand{\tagform}[1]{(\theequation)}%
\preEqlabel{}%

\usepackage[hang,small]{caption}
    \setlength{\captionmargin}{2zw}
    \renewcommand{\captionlabelfont}{\bf}
\usepackage[dvips]{graphicx,color}
\usepackage{ketpic,ketlayer}
\usepackage{mymacros}
%


\begin{document}

\setcounter{chapter}{1}

\section*{片対数グラフ}

底が$10$の対数を{\bf 常用対数}といい,
数値計算の分野ではよく用いられる.

\begin{example}
$\log_{10} 1=0,\ \log_{10} 10=1,\ \log_{10}100 =2,\ %
\log_{10}0.1 =\log_{10}10^{-1}=-1$
\end{example}

\begin{mawarikomi}[+1](1.2cm,-.2cm){6.5cm}{%%% fig1_3_katataisu_1pr.tex
%  chap0103_b_1pr.sce%
{\unitlength=1cm%
\begin{picture}%
(   5.00000,   5.00000)(  -0.00000,  -0.00000)%
\special{pn 8}%
%
\special{pa 1968 -1968}\special{pa 0 -1968}\special{pa 0 -0}\special{pa 1968 -0}\special{pa 1968 -1968}%
\special{fp}%
\special{pn 5}%
\special{pa 394 -0}\special{pa 394 -1969}%
\special{fp}%
\special{pa 787 -0}\special{pa 787 -1969}%
\special{fp}%
\special{pa 1181 -0}\special{pa 1181 -1969}%
\special{fp}%
\special{pa 1575 -0}\special{pa 1575 -1969}%
\special{fp}%
\special{pa 0 -1575}\special{pa 1969 -1575}%
\special{fp}%
\special{pa 0 -1181}\special{pa 1969 -1181}%
\special{fp}%
\special{pa 0 -787}\special{pa 1969 -787}%
\special{fp}%
\special{pa 0 -394}\special{pa 1969 -394}%
\special{fp}%
\special{pn 8}%
\special{pn 10}%
\special{pa 0 -787}\special{pa 1969 -787}%
\special{fp}%
\special{pn 8}%
\color{blue}%
\special{pn 2}%
\special{pa -0 -119}\special{pa 1969 -119}%
\special{fp}%
\special{pa -0 -188}\special{pa 1969 -188}%
\special{fp}%
\special{pa -0 -237}\special{pa 1969 -237}%
\special{fp}%
\special{pa -0 -276}\special{pa 1969 -276}%
\special{fp}%
\special{pa -0 -307}\special{pa 1969 -307}%
\special{fp}%
\special{pa -0 -333}\special{pa 1969 -333}%
\special{fp}%
\special{pa -0 -356}\special{pa 1969 -356}%
\special{fp}%
\special{pa -0 -376}\special{pa 1969 -376}%
\special{fp}%
\special{pa -0 -512}\special{pa 1969 -512}%
\special{fp}%
\special{pa -0 -582}\special{pa 1969 -582}%
\special{fp}%
\special{pa -0 -631}\special{pa 1969 -631}%
\special{fp}%
\special{pa -0 -669}\special{pa 1969 -669}%
\special{fp}%
\special{pa -0 -700}\special{pa 1969 -700}%
\special{fp}%
\special{pa -0 -726}\special{pa 1969 -726}%
\special{fp}%
\special{pa -0 -749}\special{pa 1969 -749}%
\special{fp}%
\special{pa -0 -769}\special{pa 1969 -769}%
\special{fp}%
\special{pa -0 -906}\special{pa 1969 -906}%
\special{fp}%
\special{pa -0 -975}\special{pa 1969 -975}%
\special{fp}%
\special{pa -0 -1024}\special{pa 1969 -1024}%
\special{fp}%
\special{pa -0 -1063}\special{pa 1969 -1063}%
\special{fp}%
\special{pa -0 -1094}\special{pa 1969 -1094}%
\special{fp}%
\special{pa -0 -1120}\special{pa 1969 -1120}%
\special{fp}%
\special{pa -0 -1143}\special{pa 1969 -1143}%
\special{fp}%
\special{pa -0 -1163}\special{pa 1969 -1163}%
\special{fp}%
\special{pa -0 -1300}\special{pa 1969 -1300}%
\special{fp}%
\special{pa -0 -1369}\special{pa 1969 -1369}%
\special{fp}%
\special{pa -0 -1418}\special{pa 1969 -1418}%
\special{fp}%
\special{pa -0 -1456}\special{pa 1969 -1456}%
\special{fp}%
\special{pa -0 -1487}\special{pa 1969 -1487}%
\special{fp}%
\special{pa -0 -1514}\special{pa 1969 -1514}%
\special{fp}%
\special{pa -0 -1537}\special{pa 1969 -1537}%
\special{fp}%
\special{pa -0 -1557}\special{pa 1969 -1557}%
\special{fp}%
\special{pa -0 -1693}\special{pa 1969 -1693}%
\special{fp}%
\special{pa -0 -1763}\special{pa 1969 -1763}%
\special{fp}%
\special{pa -0 -1812}\special{pa 1969 -1812}%
\special{fp}%
\special{pa -0 -1850}\special{pa 1969 -1850}%
\special{fp}%
\special{pa -0 -1881}\special{pa 1969 -1881}%
\special{fp}%
\special{pa -0 -1908}\special{pa 1969 -1908}%
\special{fp}%
\special{pa -0 -1930}\special{pa 1969 -1930}%
\special{fp}%
\special{pa -0 -1950}\special{pa 1969 -1950}%
\special{fp}%
\special{pn 8}%
\color{black}%
\small%
\settowidth{\Width}{$0.01$}\setlength{\Width}{-1\Width}%
\settoheight{\Height}{$0.01$}\settodepth{\Depth}{$0.01$}\setlength{\Height}{-0.5\Height}\setlength{\Depth}{0.5\Depth}\addtolength{\Height}{\Depth}%
\put(-0.0500,0.0000){\hspace*{\Width}\raisebox{\Height}{$0.01$}}%
%
%
\settowidth{\Width}{$0.1$}\setlength{\Width}{-1\Width}%
\settoheight{\Height}{$0.1$}\settodepth{\Depth}{$0.1$}\setlength{\Height}{-0.5\Height}\setlength{\Depth}{0.5\Depth}\addtolength{\Height}{\Depth}%
\put(-0.0500,1.0000){\hspace*{\Width}\raisebox{\Height}{$0.1$}}%
%
%
\settowidth{\Width}{$1$}\setlength{\Width}{-1\Width}%
\settoheight{\Height}{$1$}\settodepth{\Depth}{$1$}\setlength{\Height}{-0.5\Height}\setlength{\Depth}{0.5\Depth}\addtolength{\Height}{\Depth}%
\put(-0.0500,2.0000){\hspace*{\Width}\raisebox{\Height}{$1$}}%
%
%
\settowidth{\Width}{$10$}\setlength{\Width}{-1\Width}%
\settoheight{\Height}{$10$}\settodepth{\Depth}{$10$}\setlength{\Height}{-0.5\Height}\setlength{\Depth}{0.5\Depth}\addtolength{\Height}{\Depth}%
\put(-0.0500,3.0000){\hspace*{\Width}\raisebox{\Height}{$10$}}%
%
%
\settowidth{\Width}{$100$}\setlength{\Width}{-1\Width}%
\settoheight{\Height}{$100$}\settodepth{\Depth}{$100$}\setlength{\Height}{-0.5\Height}\setlength{\Depth}{0.5\Depth}\addtolength{\Height}{\Depth}%
\put(-0.0500,4.0000){\hspace*{\Width}\raisebox{\Height}{$100$}}%
%
%
\settowidth{\Width}{$1000$}\setlength{\Width}{-1\Width}%
\settoheight{\Height}{$1000$}\settodepth{\Depth}{$1000$}\setlength{\Height}{-0.5\Height}\setlength{\Depth}{0.5\Depth}\addtolength{\Height}{\Depth}%
\put(-0.0500,5.0000){\hspace*{\Width}\raisebox{\Height}{$1000$}}%
%
%
\settowidth{\Width}{$0$}\setlength{\Width}{-0.5\Width}%
\settoheight{\Height}{$0$}\settodepth{\Depth}{$0$}\setlength{\Height}{-\Height}%
\put(1.0000,-0.1500){\hspace*{\Width}\raisebox{\Height}{$0$}}%
%
%
\settowidth{\Width}{$1$}\setlength{\Width}{-0.5\Width}%
\settoheight{\Height}{$1$}\settodepth{\Depth}{$1$}\setlength{\Height}{-\Height}%
\put(2.0000,-0.1500){\hspace*{\Width}\raisebox{\Height}{$1$}}%
%
%
\settowidth{\Width}{$2$}\setlength{\Width}{-0.5\Width}%
\settoheight{\Height}{$2$}\settodepth{\Depth}{$2$}\setlength{\Height}{-\Height}%
\put(3.0000,-0.1500){\hspace*{\Width}\raisebox{\Height}{$2$}}%
%
%
\settowidth{\Width}{$3$}\setlength{\Width}{-0.5\Width}%
\settoheight{\Height}{$3$}\settodepth{\Depth}{$3$}\setlength{\Height}{-\Height}%
\put(4.0000,-0.1500){\hspace*{\Width}\raisebox{\Height}{$3$}}%
%
%
\special{pa 408 -1470}\special{pa 400 -1475}\special{pa 390 -1476}\special{pa 382 -1472}%
\special{pa 376 -1464}\special{pa 374 -1455}\special{pa 377 -1446}\special{pa 384 -1439}%
\special{pa 393 -1437}\special{pa 402 -1438}\special{pa 409 -1444}\special{pa 413 -1453}%
\special{pa 412 -1462}\special{pa 408 -1470}\special{sh 1}\special{fp}%
\special{pa 801 -1195}\special{pa 793 -1200}\special{pa 784 -1200}\special{pa 775 -1197}%
\special{pa 769 -1189}\special{pa 768 -1180}\special{pa 771 -1171}\special{pa 777 -1164}%
\special{pa 786 -1161}\special{pa 795 -1163}\special{pa 803 -1169}\special{pa 807 -1178}%
\special{pa 806 -1187}\special{pa 801 -1195}\special{sh 1}\special{fp}%
\special{pa 1195 -920}\special{pa 1187 -925}\special{pa 1178 -925}\special{pa 1169 -921}%
\special{pa 1163 -914}\special{pa 1161 -905}\special{pa 1164 -896}\special{pa 1171 -889}%
\special{pa 1180 -886}\special{pa 1189 -888}\special{pa 1197 -894}\special{pa 1200 -902}%
\special{pa 1200 -912}\special{pa 1195 -920}\special{sh 1}\special{fp}%
\special{pa 1589 -645}\special{pa 1581 -650}\special{pa 1571 -650}\special{pa 1563 -646}%
\special{pa 1557 -639}\special{pa 1555 -630}\special{pa 1558 -621}\special{pa 1565 -614}%
\special{pa 1574 -611}\special{pa 1583 -613}\special{pa 1590 -619}\special{pa 1594 -627}%
\special{pa 1594 -637}\special{pa 1589 -645}\special{sh 1}\special{fp}%
\end{picture}}%}
$f(x)>0$であるとき,関数$y=f(x)$のグラフを,縦軸方向には値$y$の常用対数
$\log_{10}y$の値をとって描くことがある.これを{\bf 片対数グラフ}という.
片対数グラフでは,真数の値を縦軸の目盛りにするのが普通である.また,
$\log_{10} 1=0$より,目盛り1の横線を横軸とする.
目盛り1, 10, 100などに対応する横線の間隔はすべて等しいから,
横軸をどこにとってもよい.
\end{mawarikomi}

\begin{example}
$y=2\cdot 5^{2-x}$において
\[
x=0,\ 1,\ 2,\ 3\quad\text{のとき}\quad y=50,\ 10,\ 2,\ 0.4
\]
このときの点は図のようになる.
\end{example}

指数関数$y=c\,a^x$\ \ ($c,\ a$は正の定数,$a\neqq 1$)について
\begin{equation}\label{logの等式}
\log_{10}y=\log_{10}\bigl(c a^x\bigr)=\bigl(\log_{10}a\bigr)x+\log_{10}c
\end{equation}
したがって,$\log_{10} y$は$x$の1次式で表されるから,
片対数グラフでは直線となる.また,逆も成り立つ.

(\ref{logの等式})より,直線の傾きおよび切片から$a,\ c$が求められる.

\begin{exercise}
2点$(0,\ 1.6)$,$(2,\ 10)$を通る関数$y=f(x)$のグラフは,片対数グラフ
では直線となる.この関数を定めよ.
\end{exercise}

\begin{layer}{150}{0}
\putnotese{70}{-3}{%%% fig1_3_katataisu2_1pr.tex
%  chap0103_b_1pr.sce%
{\unitlength=1cm%
\begin{picture}%
(   5.00000,   5.00000)(  -0.00000,  -0.00000)%
\special{pn 8}%
%
\special{pa 1968 -1968}\special{pa 0 -1968}\special{pa 0 -0}\special{pa 1968 -0}\special{pa 1968 -1968}%
\special{fp}%
\special{pn 4}%
\special{pa 394 -0}\special{pa 394 -1969}%
\special{fp}%
\special{pa 787 -0}\special{pa 787 -1969}%
\special{fp}%
\special{pa 1181 -0}\special{pa 1181 -1969}%
\special{fp}%
\special{pa 1575 -0}\special{pa 1575 -1969}%
\special{fp}%
\special{pa 0 -1575}\special{pa 1969 -1575}%
\special{fp}%
\special{pa 0 -1181}\special{pa 1969 -1181}%
\special{fp}%
\special{pa 0 -787}\special{pa 1969 -787}%
\special{fp}%
\special{pa 0 -394}\special{pa 1969 -394}%
\special{fp}%
\special{pn 8}%
\special{pn 10}%
\special{pa 0 -787}\special{pa 1969 -787}%
\special{fp}%
\special{pa 394 -0}\special{pa 394 -1969}%
\special{fp}%
\special{pn 8}%
\special{pn 12}%
\special{pa 0 -711}\special{pa 1969 -1494}%
\special{fp}%
\special{pn 8}%
\color{blue}%
\special{pn 2}%
\special{pa -0 -119}\special{pa 1969 -119}%
\special{fp}%
\special{pa -0 -188}\special{pa 1969 -188}%
\special{fp}%
\special{pa -0 -237}\special{pa 1969 -237}%
\special{fp}%
\special{pa -0 -276}\special{pa 1969 -276}%
\special{fp}%
\special{pa -0 -307}\special{pa 1969 -307}%
\special{fp}%
\special{pa -0 -333}\special{pa 1969 -333}%
\special{fp}%
\special{pa -0 -356}\special{pa 1969 -356}%
\special{fp}%
\special{pa -0 -376}\special{pa 1969 -376}%
\special{fp}%
\special{pa -0 -512}\special{pa 1969 -512}%
\special{fp}%
\special{pa -0 -582}\special{pa 1969 -582}%
\special{fp}%
\special{pa -0 -631}\special{pa 1969 -631}%
\special{fp}%
\special{pa -0 -669}\special{pa 1969 -669}%
\special{fp}%
\special{pa -0 -700}\special{pa 1969 -700}%
\special{fp}%
\special{pa -0 -726}\special{pa 1969 -726}%
\special{fp}%
\special{pa -0 -749}\special{pa 1969 -749}%
\special{fp}%
\special{pa -0 -769}\special{pa 1969 -769}%
\special{fp}%
\special{pa -0 -906}\special{pa 1969 -906}%
\special{fp}%
\special{pa -0 -975}\special{pa 1969 -975}%
\special{fp}%
\special{pa -0 -1024}\special{pa 1969 -1024}%
\special{fp}%
\special{pa -0 -1063}\special{pa 1969 -1063}%
\special{fp}%
\special{pa -0 -1094}\special{pa 1969 -1094}%
\special{fp}%
\special{pa -0 -1120}\special{pa 1969 -1120}%
\special{fp}%
\special{pa -0 -1143}\special{pa 1969 -1143}%
\special{fp}%
\special{pa -0 -1163}\special{pa 1969 -1163}%
\special{fp}%
\special{pa -0 -1300}\special{pa 1969 -1300}%
\special{fp}%
\special{pa -0 -1369}\special{pa 1969 -1369}%
\special{fp}%
\special{pa -0 -1418}\special{pa 1969 -1418}%
\special{fp}%
\special{pa -0 -1456}\special{pa 1969 -1456}%
\special{fp}%
\special{pa -0 -1487}\special{pa 1969 -1487}%
\special{fp}%
\special{pa -0 -1514}\special{pa 1969 -1514}%
\special{fp}%
\special{pa -0 -1537}\special{pa 1969 -1537}%
\special{fp}%
\special{pa -0 -1557}\special{pa 1969 -1557}%
\special{fp}%
\special{pa -0 -1693}\special{pa 1969 -1693}%
\special{fp}%
\special{pa -0 -1763}\special{pa 1969 -1763}%
\special{fp}%
\special{pa -0 -1812}\special{pa 1969 -1812}%
\special{fp}%
\special{pa -0 -1850}\special{pa 1969 -1850}%
\special{fp}%
\special{pa -0 -1881}\special{pa 1969 -1881}%
\special{fp}%
\special{pa -0 -1908}\special{pa 1969 -1908}%
\special{fp}%
\special{pa -0 -1930}\special{pa 1969 -1930}%
\special{fp}%
\special{pa -0 -1950}\special{pa 1969 -1950}%
\special{fp}%
\special{pn 8}%
\color{black}%
\small%
\settowidth{\Width}{$0.01$}\setlength{\Width}{-1\Width}%
\settoheight{\Height}{$0.01$}\settodepth{\Depth}{$0.01$}\setlength{\Height}{-0.5\Height}\setlength{\Depth}{0.5\Depth}\addtolength{\Height}{\Depth}%
\put(-0.0500,0.0000){\hspace*{\Width}\raisebox{\Height}{$0.01$}}%
%
%
\settowidth{\Width}{$0.1$}\setlength{\Width}{-1\Width}%
\settoheight{\Height}{$0.1$}\settodepth{\Depth}{$0.1$}\setlength{\Height}{-0.5\Height}\setlength{\Depth}{0.5\Depth}\addtolength{\Height}{\Depth}%
\put(-0.0500,1.0000){\hspace*{\Width}\raisebox{\Height}{$0.1$}}%
%
%
\settowidth{\Width}{$1$}\setlength{\Width}{-1\Width}%
\settoheight{\Height}{$1$}\settodepth{\Depth}{$1$}\setlength{\Height}{-0.5\Height}\setlength{\Depth}{0.5\Depth}\addtolength{\Height}{\Depth}%
\put(-0.0500,2.0000){\hspace*{\Width}\raisebox{\Height}{$1$}}%
%
%
\settowidth{\Width}{$10$}\setlength{\Width}{-1\Width}%
\settoheight{\Height}{$10$}\settodepth{\Depth}{$10$}\setlength{\Height}{-0.5\Height}\setlength{\Depth}{0.5\Depth}\addtolength{\Height}{\Depth}%
\put(-0.0500,3.0000){\hspace*{\Width}\raisebox{\Height}{$10$}}%
%
%
\settowidth{\Width}{$100$}\setlength{\Width}{-1\Width}%
\settoheight{\Height}{$100$}\settodepth{\Depth}{$100$}\setlength{\Height}{-0.5\Height}\setlength{\Depth}{0.5\Depth}\addtolength{\Height}{\Depth}%
\put(-0.0500,4.0000){\hspace*{\Width}\raisebox{\Height}{$100$}}%
%
%
\settowidth{\Width}{$1000$}\setlength{\Width}{-1\Width}%
\settoheight{\Height}{$1000$}\settodepth{\Depth}{$1000$}\setlength{\Height}{-0.5\Height}\setlength{\Depth}{0.5\Depth}\addtolength{\Height}{\Depth}%
\put(-0.0500,5.0000){\hspace*{\Width}\raisebox{\Height}{$1000$}}%
%
%
\settowidth{\Width}{$0$}\setlength{\Width}{-0.5\Width}%
\settoheight{\Height}{$0$}\settodepth{\Depth}{$0$}\setlength{\Height}{-\Height}%
\put(1.0000,-0.1500){\hspace*{\Width}\raisebox{\Height}{$0$}}%
%
%
\settowidth{\Width}{$1$}\setlength{\Width}{-0.5\Width}%
\settoheight{\Height}{$1$}\settodepth{\Depth}{$1$}\setlength{\Height}{-\Height}%
\put(2.0000,-0.1500){\hspace*{\Width}\raisebox{\Height}{$1$}}%
%
%
\settowidth{\Width}{$2$}\setlength{\Width}{-0.5\Width}%
\settoheight{\Height}{$2$}\settodepth{\Depth}{$2$}\setlength{\Height}{-\Height}%
\put(3.0000,-0.1500){\hspace*{\Width}\raisebox{\Height}{$2$}}%
%
%
\settowidth{\Width}{$3$}\setlength{\Width}{-0.5\Width}%
\settoheight{\Height}{$3$}\settodepth{\Depth}{$3$}\setlength{\Height}{-\Height}%
\put(4.0000,-0.1500){\hspace*{\Width}\raisebox{\Height}{$3$}}%
%
%
\end{picture}}%}

\end{layer}

\begin{solution}
\begin{mawarikomi}[10]{60mm}{}
片対数グラフを描いたとき,
2点の実際の座標は$(0,\ \log_{10}1.6)$\vspace{1mm}および
$(2,\log_{10}10)$となるから,直線の傾き$m$
および切片$b$は

\begin{align*}
m&=\frac{\log_{10}10-\log_{10}1.6}{2-0}\\
&=\frac{1}{2}\log_{10}\dfrac{10}{1.6}=\log_{10}(2.5)\\
b&=\log_{10}1.6
\end{align*}

(\ref{logの等式})より
\begin{align*}
\log_{10}a&=\log_{10}(2.5)\\ 
\log_{10}c&=\log_{10}{1.6}
\end{align*}
これから$a=2.5,\ c=1.6$ 

\noindent
したがって,求める関数は
\[
y=1.6\,(2.5)^x\tag*{\qed}
\]
\end{mawarikomi}
\end{solution}

\begin{question}
$(0,\ 1),\ (5,\ 0.1)$を通る関数のグラフは片対数グラフでは直線となる.
この関数を定めよ.ただし,$10^{-0.2}=0.63$とする.
\end{question}

\end{document}
